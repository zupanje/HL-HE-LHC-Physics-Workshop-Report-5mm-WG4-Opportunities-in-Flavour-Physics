%
\documentclass[../report.tex]{subfiles}
\providecommand{\main}{..}
\IfEq{\jobname}{\currfilebase}{\AtEndDocument{\biblio}}{}
%

\begin{document}

\section{The high \pt flavour physics programme} 
\label{sec:highpT}

Flavour and high \pt searches are intertwined in several ways. On the one hand the stringent bounds from low-energy constraints put severe bounds on the NP models that contain states with TeV masses with couplings to quarks.  On the other hand, TeV scale New Physics is suggested by several solutions of long standing problems of the SM, e.g., the hierarchy problem. Quite often the low energy constraints are avoided by assuming Minimal Flavour Violation (MFV), where the only flavour breaking, even in the NP sector, is due to the SM Yukawa matrices. However, more general flavour structures for the NP states are still allowed. In fact such non-MFV couplings can have interesting consequences. In general we can group the NP models into two broad classes: \textit{(i)} models that address outstanding problems of the SM, such at the SM flavour puzzle, the origin of dark matter, or the hierarchy problem, which may have non trivial flavour structure,  and \textit{(ii)} models designed to explain the $b\to s\ell\ell$ and $b\to c\tau \nu$ flavour anomalies, that almost inevitably have quite a distinct flavour structure.  The NP mediators potentially responsible for the anomalies, $Z'$, $W'$ or leptoquarks, could be found at high \pt searches in the HL- or HE-LHC.

In the rest of this section we first briefly review the models that address the SM flavour puzzle and have states that could be probed at the HL-/HE-LHC, and their implications for high \pt searches. The second part of this section is devoted to  high \pt implications of $B$ physics anomalies. 

\subsection{Models of flavour and TeV Physics} 


\subsubsection{\bf Randall-Sundrum models of flavour}
Models of flavour based on warped extra dimension~\cite{Randall:1999ee} attempt to simultaneously solve
 the hierarchy problem as well as the SM flavour 
problem~\cite{Grossman:1999ra,Gherghetta:2000qt}. 
In the Randall-Sundrum (RS) models the 5-dimensional space-time has 
anti-de Sitter geometry (AdS$_5$), truncated by flat 4D boundaries, 
the Planck (UV) and the TeV (IR) branes.  
This setup gives a warped metric in the bulk~\cite{Randall:1999ee}  
$ds^2 = \exp(-2 k r_c |\phi| ) \eta_{\mu\nu} dx^\mu dx^\nu - r_c^2 d\phi^2$,
where $k$ is the 5D curvature scale, $r_c$ the radius of compactification and  
$\phi\in [-\pi,\pi]$ the coordinate along the $5$-th dimension. 
The warp factor, $\exp({-2 k r_c |\phi|})$, leads to different length scales 
in different 4D slices along the $\phi$ direction, 
which provides a solution to the hierarchy problem.  
In particular, the Higgs field is assumed to be localized
near the TeV-brane so that the metric ``warps'' 
$\langle{H}\rangle_5 \sim M_5\sim M_P \sim 10^{19}$~GeV down to the weak scale, 
$\langle{H}\rangle_4 = \exp({-kr_c \pi}) \langle{H}\rangle_5$.  
For $kr_c \approx 12$ then $\langle{H}\rangle_{\rm SM} \equiv \langle{H}\rangle_4 \sim 1$~TeV.  



The hierarchies among the quark masses can be realized by localizing the Higgs on the IR brane, while the fermions have different profiles in the $5$-th dimension. 
The first and second generation zero mode fermions are localized
close to the UV-brane and have small overlaps with the Higgs, giving small effective 4D SM Yukawa interactions, and thus small quark masses after electroweak symmetry breaking.
The top quark, on the other hand, is localized near the TeV brane 
resulting in a large top Yukawa coupling. 


This configuration has a built in automatic suppression of FCNCs, which are suppressed by the 
same zero mode overlaps that gives the hierarchy of masses~\cite{Grossman:1999ra,Gherghetta:2000qt}. This feature of the RS framework plays a similar role as
the SM Glashow-Iliopoulos-Maiani (GIM) mechanism, and was dubbed RS-GIM in  \cite{Agashe:2004cp,Agashe:2004ay}.
Similarly to the SM GIM, the RS GIM is violated by the large top quark mass. 
In particular, $(t,b)_L$ needs to be localized near the TeV brane, 
otherwise the 5D Yukawa coupling becomes too large and makes the theory
strongly coupled at the scale of the first Kaluza-Klein (KK) excitation. In general this 
leads to sizeable corrections to electroweak precision observables, such as the $Z b_L b_L$ couplings. 
Such problems can be largely ameliorated by enlarging the bulk symmetry such that it contains a custodial $SU(2)_L\times SU(2)_R$ symmetry \cite{Agashe:2003zs}, which for instance lowers the KK scale bounds from EW precision tests from 5 TeV to about 2TeV \cite{Agashe:2013kxa}. The consequences for flavour phenomenology have been worked out in a series of papers, see, e.g., \cite{Blanke:2008yr,Albrecht:2009xr,Casagrande:2010si,Casagrande:2008hr}, with $K-\bar K$ mixing for instance requiring the KK scale to be above 8 TeV \cite{Agashe:2013kxa}. With flavour alignment the scale of KK modes could be substantially lowered \cite{Csaki:2009wc} and could be reachable by HL/HE-LHC.

The KK gluon resonances cannot be produced from gluons \cite{Allanach:2009vz}, so that the LHC production is restricted to the quark-antiquark fusion, even though this is suppressed by the flavor dependent zero mode overlaps. This means that the LHC cross section for the first KK gluon resonances are small, suppressed also by the quark-anti-quark parton density functions (PDFs). The dominant decay mode is into $t\bar t$ final state, due to the large zero mode overlaps \cite{Agashe:2006hk}. Using the benchmark RS model from \cite{Lillie:2007yh}, the most recent CMS analysis for $t\bar t$ resonance searches, using both hadronic and leptonic tops, sets a lower bound of 4.55 TeV on the mass of the KK gluon  \cite{Sirunyan:2018ryr}. The projected reach for 33TeV and 100TeV $pp$ colliders can be found in \cite{Agashe:2013kxa}.



\subsubsection{Partial compositeness}
Partial compositeness as the origin of the flavour hierarchies in composite Higgs models \cite{Kaplan:1991dc} is the holographic dual to the RS models of flavour. While the Higgs is the lightest state of the composite sector, usually a pseudo-Nambu Goldstone boson from global symmetry breaking, the SM fermions and gauge bosons are elementary (for a review, see, e.g, \cite{Panico:2015jxa}). The elementary fermions, $Q,U,D$, are coupled to the composite sector through linear mixing with the composite operators, ${\mathcal O}_Q$, ${\mathcal O}_U$, ${\mathcal O}_D$,
\begin{equation}
{\cal L}\supset \epsilon_Q \bar Q_L {\mathcal O}_Q +\epsilon_U \bar U_R{\mathcal O}_U+\epsilon_D \bar D_R {\mathcal O}_D.
\end{equation}
The mixing parameters $\epsilon_a$ exhibit exponentially large hierarchies because of large, yet still ${\mathcal O}(1)$, differences in anomalous dimensions of the corresponding composite operators. This is the analog of the zero mode overlaps in the RS models. The SM Yukawa are given by $(Y_{U(D)})_{ij}\sim \epsilon_{Q}^i \epsilon_{U(D)}^j$. For 
$\epsilon_Q^1\ll \epsilon_Q^2\ll \epsilon_Q^3\sim 1$, $\epsilon_U^1\ll \epsilon_U^2\ll \epsilon_U^3\sim 1$, $\epsilon_D^1\ll \epsilon_D^2\ll \epsilon_D^3\ll 1$ one can obtain the SM structure of quark masses and CKM mixings. 

The composite Higgs models are described by the compositeness scale $f$ and the mass of the first composite resonances, $M_*\sim g_* f$, with $g_*$ the typical strength of the resonances in the composite sector. The searches at the HL-/HE-LHC consist of Higgs coupling measurements, including deviations in Higgs Yukawa couplings,
and searches for composite resonances preferably coupled to third generation fermions, electroweak gauge bosons, or the Higgs.
Flavour observables put strong bounds on $M_*$, if the flavour structure is assumed to be generic. Such bounds can be relaxed in the case of approximate flavour symmetries, see, e.g., Ref. \cite{Panico:2015jxa} for a review.



\subsubsection{Low scale gauge flavour  symmetries}
The SM has in the limit of vanishing Yukawa couplings a large global symmetry. In the quark sector this is $G_F=SU(3)_Q\times SU(3)_U\times SU(3)_D$. Ref. \cite{Grinstein:2010ve} showed that the SM flavor symmetry group $G_F$ is anomaly free, if one adds a set of fermions that are vector-like under the SM gauged group, but chiral under the $G_F$. This means that $G_F$ can be gauged. It is broken by a set of scalar fields that have hiearchical vevs and lead to hierarchy of SM quark masses. This also implies a hierarchy  for the masses of the flavoured gauge bosons, with gauge bosons that more strongly couple to third generation being lighter, while the flavoured gauge bosons that couple more strongly to the first two generations are significantly heavier. This pattern in the spectrum of flavoured gauge bosons then avoids too large contributions to FCNCs \cite{Buras:2011wi}.  

At the LHC one searches for the lightest flavoured gauge bosons, with ${\mathcal O}({\rm TeV})$ masses, which couple mostly to $b$ quarks and $t$ quarks, but could also have non-negligible couplings to the first two generations. The di-jet and $t\bar t$ resonance searches are thus sensitive probes. A signal could also come from production of the lightest vectorlike fermions, $t'$, or $b'$ \cite{Grinstein:2010ve}. For several further benchmarks see, e.g., Ref. \cite{Bishara:2015mha}, where also a connection with dark matter was explored.

\subsubsection{2HDM and low scale flavour models}
\label{sec:2HDM:flavour}
{\it\small Authors (TH): Martin Bauer, Marcela Carena and Adri\'an Carmona.}\\
In 2 Higgs Doublet Models (2HDMs), the two Higgs doublets, $H_1$ and $H_2$ are usually assumed not to carry flavour quantum numbers. The collider phenomenology, on the other hand, changes substantially, if they do. 
 This is an interesting possibility that could solve the SM flavour puzzle via the Froggatt-Nielsen (FN) mechanism where the flavon is replaced by the $H_1 H_2\equiv H_1^T (i\sigma_2) H_2$ operator. In this way, the NP scale $\Lambda$ where the higher dimensional FN operators are generated is tied to the electroweak scale, leading to much stronger phenomenological consequences. Let us assume for concreteness a type-I like 2DHM with the following Yukawa Lagrangian in the quark sector \cite{Bauer:2015kzy,Bauer:2015fxa}
\begin{equation}
\label{eq:yuk1}
\mathcal{L}_Y\supset  y^u_{ij} \left(\frac{H_1 H_2}{\Lambda^2}\right)^{n_{u_{ij}}}\bar{q}_L^{i}H_1 u_{R}^j+y^d_{ij} \left(\frac{H_1^{\dagger} H_2^{\dagger}}{\Lambda^2}\right)^{n_{d_{ij}}}\bar{q}_L^{i}\tilde{H}_1 d_{R}^j
+\mathrm{h.c.}\,,
\end{equation}
where $\tilde{H}_1\equiv i\sigma_2 H^{\ast}_1$ as usual, and the charges $n_{u,d,e}$ are a combination of the $U(1)$ charges of $H_1$, $(H_1H_2)$ and the different SM fermion fields (for an alternative discussion, where $H_1, H_2$ carry flavour charges, but the Yukawa interactions are taken to be renormalizable, see \cite{Dery:2016fyj}). For simplicity, we set the flavour charges of $H_1$ and $H_2$ to $0$ and $1$, respectively, such that  
$n_{u_{ij}}=a_{q_i}-a_{u_j}, n_{d_{ij}}=-a_{q_i}+a_{d_j}$,
if we denote by $a_{q_i},a_{u_i}, \ldots,$ the $U(1)$ charges of the SM fermions. In general, the fermion masses are given by
\begin{equation}\label{eq:epsilon}
m_\psi=y_{\psi} \varepsilon^{n_\psi} \frac{v}{\sqrt{2}},  \qquad \varepsilon = \frac{v_1 v_2}{2\Lambda^2}=\frac{t_\beta}{1+t_\beta^2}\frac{v^2}{2\Lambda^2},
\end{equation}
with the vacuum expectation values $\langle H_{1, 2}\rangle=v_{1, 2}$ and $t_\beta \equiv v_1/v_2$. For the right assignment of flavour charges one is able to accommodate the observed hierarchy of SM fermion masses and mixing angles.  This framework also leads to enhanced diagonal Yukawa couplings between the Higgs and the SM fermions, while FCNCs are suppressed. If we denote by $h$ and $H$ the two neutral scalar mass eigenstates, with $h$ the observed $125$ GeV Higgs, the couplings between the scalars $\varphi=h,H$ and SM fermions $\psi_{L_i, R_i}= P_{L,R} \psi_i$ in the mass eigenbasis read 
\begin{equation}\label{eq:newlagrangian}
\mathcal{L}= g_{\varphi \psi_{L_i} \psi_{R_j} }\, \varphi \,\bar \psi_{L_i} \psi_{R_j}+\mathrm{h.c.}
\end{equation}
with $i$, such that $u_i=u, c,t$,\,  $d_i=d,s,b$ and $e_i=e,\mu,\tau$. This induces 
flavour-diagonal couplings 
\begin{align}\label{eq:diagocoup}
g_{\varphi \psi_{L_i}\psi_{R_i}}= \kappa^\varphi_{\psi_i}\, \frac{m_{\psi_i}}{v} =\left(g^{\varphi}_{\psi_i}(\alpha,\beta)+n_{\psi_i}\, f^\varphi(\alpha, \beta)\right)\frac{m_{\psi_i}}{v},
\end{align}
as well as flavour off-diagonal couplings
 \begin{align}\label{eq:foffc}
g_{\varphi \psi_{L_i}\psi_{R_j}}&=  f^\varphi(\alpha, \beta)\left(\mathcal{A}_{ij}\frac{m_{\psi_j}}{v}-\frac{m_{\psi_i}}{v}\mathcal{B}_{ij} \right)\,.
\end{align}
The flavour universal functions in \eqref{eq:diagocoup} and \eqref{eq:foffc} are
$g^h_{\psi_i}={c_{\beta-\alpha}}/{t_\beta}+s_{\beta-\alpha}$, $ g^H_{\psi_i}=c_{\beta-\alpha}-{s_{\beta-\alpha}}/{t_\beta}$,
and
$ f^h(\alpha,\beta)=c_{\beta-\alpha}\big({1}/{t_\beta}-t_\beta\big)+2s_{\beta-\alpha}$, $f^H(\alpha,\beta)=-s_{\beta-\alpha}\big({1}/{t_\beta}-t_\beta\big)+2c_{\beta-\alpha}$,
where $c_{x}\equiv \cos x$, $s_x\equiv \sin x$. The entries in matrices $\mathcal{A}$ and $\mathcal{B}$ are proportional to the flavour charges of the corresponding fermions that define the coefficients in \eqref{eq:yuk1}. Unless all flavour charges for a given type of fermions are equal, the off-diagonal elements in matrices $\mathcal{A}$ and $\mathcal{B}$ lead to FCNCs which are chirally suppressed by powers of the ratio~$\varepsilon$, see \cite{Bauer:2017cov} for more details and explicit examples for scalings of matrix elements in $\mathcal{A}$ and $\mathcal{B}$.


\subsubsection{\bf A Clockwork solution to the flavour puzzle}
{\it\small Author (TH): Adri\'an Carmona.}\\
The clockwork mechanism, introduced in \cite{Choi:2015fiu, Kaplan:2015fuy} and later generalized to a broader context in Ref.~\cite{Giudice:2016yja}, allows one to obtain large hierarchies in couplings or mass scales. Ref. \cite{Alonso:2018bcg} showed that it can also be used to generate the observed hierarchy of quark masses and mixing angles with anarchic Yukawa couplings, providing a solution to the flavour puzzle. 

In the clockwork solution to the flavour puzzle, each SM chiral fermion $\psi$ is accompanied with a $N_\psi$-node chain of vector-like fermions, $\psi_{L,j},\psi_{R,j}$, with masses $m$, and
a series of nearest neighbour mass terms, $qm\bar{\psi}_{L,j}\psi_{R,j-1}$, between the nodes, where  $j=1,\ldots,N_\psi$. The mass spectrum of each chain has one chiral zero mode, the would-be SM fermion, 
and $N_\psi$  heavy Dirac fermion mass-eigenstates -- the gears.
For $q\gg 1$,  the spectrum of the gears is compressed in a $2m$ band around $qm$, with $(M_{N_\psi}-M_1)\ll M_1$. The massless zero-mode interacts with the SM Higgs, which is on the $0$-th node, through a set of Yukawa interactions described by ${\mathcal O}(1)$ Yukawa matrices, $Y_{U,D}$. 
The component of the massless mode on the $0$-th node is, 
in the large $q$ limit, given by $1/q^{N_\psi}$. This suppression is the origin of the SM Yukawa hierarchies, 
\begin{align}
\label{eqa:YuSM}
\left(Y_u^{\rm SM}\right)_{ij}&\sim q^{-N_{Q(i)}}_{Q(i)}\left(Y_U\right)_{ij}q^{-N_{u(j)}}_{u(j)},  \quad \left(Y_d^{\rm SM}\right)_{ij} \sim q^{-N_{Q(i)}}_{Q(i)}\left(Y_D\right)_{ij}q^{-N_{d(j)}}_{d(j)}.
\end{align}
The hierarchy of quark masses can be then naturally obtained for anarchical $Y_U$ and $Y_D$ Yukawa matrices if $q^{-N_{Q(i)}}\ll q^{-N_{Q(j)}}$, $q^{-N_{u(i)}}_{u(i)}\ll q^{-N_{u(j)}}_{u(j)}$, $q^{-N_{d(i)}}_{d(i)}\ll q^{-N_{d(j)}}_{d(j)}$, for $i<j$ (in the benchmark below we take $q_i$ to be universal and equal to $q$).

\begin{figure}
\begin{center}
\includegraphics[width=0.4\textwidth]{\main/section10/img/GearXsec_all_fn.pdf}
	\includegraphics[width=0.59\textwidth]{\main/section10/img/GearSpheres_all_fn.pdf}
\caption{\textit{Left:} The total gear pair production cross-sections in the final states $tW+X$, $tH+X$ and $tZ+X$ for the benchmark model from Ref. \cite{Alonso:2018bcg}, with contributions from individual gears shown stacked.
The currently most stringent upper bounds~\cite{Aaboud:2018uek,Aaboud:2018xuw} are denoted with dashed lines (for the $tZ+X$~\cite{Aaboud:2018xuw} final state the bound is too weak to be shown). 	
\textit{Right:} Invariant mass spectrum of individual pseudojets clustered using the hemisphere algorithm applied to partonic gear pair production and decay at the 13 TeV LHC. The original hemisphere clustering results are shown in light gray, and the modified hemisphere clustering results in mid-gray (dark gray if in addition 
the masses of the two pseudojets are required to differ by less than 30\%)
}
	\label{fig:both}
\end{center}
\end{figure}

The clockwork models of flavour are endowed with a powerful flavour protection against FCNCs, very similar to the RS models.  The FCNCs with light quarks on the external legs are suppressed by the same small overlaps of the zero-modes, giving rise to hierarchies between the SM quark masses. This clockwork-GIM  mechanism, along with the constraints on $Y_{U,D}$ arising from the stability of the Higgs potential,   suffices to alleviate the flavour constraints to the level that TeV scale gear masses are compatible with experimental bounds~\cite{Alonso:2018bcg}.

TeV scale gears
can be searched for at the LHC, where they are produced through QCD pair production.
The collider signatures
depend on the gear decay patterns. The gears decay predominantly through their coupling to the Higgs doublet into gears from a different-chirality chain.
The lightest gears decay directly to SM fermions, mostly $t$ and $b$, via the emission of $W,Z$ or $h$, as do heavier gears for which these are the only kinematically allowed channels. 
The main existing collider constraints 
are from searches for pair production of vector-like quarks, in final states involving third generation SM quarks. Ref.~\cite{Alonso:2018bcg} found the two
35 fb${}^{-1}$ 13 TeV ATLAS 
searches for vector-like quarks decaying into 
$tW$ final states~\cite{Aaboud:2018uek}, as well as the analogous search employing the 
$tZ$ and $tH$ final states~\cite{Aaboud:2018xuw}, to be currently most sensitive, see 
Fig.~\ref{fig:both}.





The dense spectrum of gears and the potentially complex pattern of gear decays poses an experimental challenge.
In the conventional vector-like quark searches the clockwork signal will appear as an excess of events with high transverse energies or $H_T$, but without a dominant single peak in the invariant mass of any particular final state, such as $tH$ or $tW$. Ref.~\cite{Alonso:2018bcg} proposed a novel reconstruction strategy targeting pair production of heavy quarks with a-priori unknown but potentially long decay chains that result in a single heavy flavoured quark, $t$ or $b$, plus any number of massive weak or Higgs bosons per decay chain. The proposed search strategy uses a modified hemisphere clustering algorithm, with $t-$ and $b-$tagged jets as seeds for clustering into exactly two pseudo-jets (the invariant mass of these is shown as mid-gray distribution in Fig.~\ref{fig:both} right). The original hemisphere clustering uses instead the jets with highest invariant mass as seeds and shows no sharp features (light gray). Requiring that the masses of the two pseudo-jets differ by less then 30\% gives the dark grey distribution, with clearly visible gears (in the exploratory study of~\cite{Alonso:2018bcg} tops, $b$-quarks, $W$, $Z$ and the Higgs were not decayed).




\subsection{Flavour implications for high \pt new physics searches} 
In this subsection we collect several signatures of flavour models or models where nontrivial flavour structure is relevant for high \pt searches: the FCNC top decays to exotica, the (model dependent) implications for di-Higgs production, and the set of signatures that are related to neutrino mass models. 

\subsubsection{Top decays to exotica} 
{\it\small Authors (TH): S. Banerjee, M. Chala, M. Spannowsky.} 

The FCNC mediated processes are rare within the SM. However, LHC is a top factory and significant number of events are expected even for top decays with very small branching ratios. In light of this, studies of top FCNC decays to SM particles have garnered a strong interest in the community~\cite{Agashe:2009di,Mele:1998ag,Greljo:2014dka,Azatov:2014lha,Khachatryan:2015att,Botella:2015hoa,Bardhan:2016txk,Badziak:2017wxn,Khachatryan:2016sib,Gabrielli:2016cut,CMS-PAS-TOP-17-017,Aaboud:2018nyl,Aaboud:2017mfd,Aaboud:2018pob,Papaefstathiou:2017xuv,Abbas:2015cua}. The FCNC top decays to SM particles, $t\to q Z, q \gamma, qg, q H$, $q=u,c$, and the related constraints from FCNC production processes, were discussed in Sections \ref{sec:top:FCNC} and \ref{sec:8:EXP:FCNC}.

In the presence of light NP, other exotic top FCNC decays are possible. We highlight one such possibility, where the NP spectrum contains a light 
scalar singlet, $S$, with mass $m_S$ below the top quark mass.
Such a scalar
particle is predicted in a number of well-motivated NP models, e.g., in the NMSSM~\cite{Ellwanger:2009dp} and in
non-minimal composite Higgs models~\cite{Dimopoulos:1981xc,Kaplan:1983fs,Kaplan:1983sm,Panico:2015jxa}. Moreover, quite often the induced $t\to cS, uS$ FCNC decays are easier to probe than, for instance the ones involving the SM Higgs, $t\to ch, uh$~\cite{Zhang:2013xya}. 
The reason is three-fold; \textit{(i)} The top FCNCs mediated by 
$S$ are usually suppressed by one less power of the heavy physics scale; \textit{(ii)} $S$ may have a larger decay width into cleaner final states, such as $\ell^+\ell^-$, $b\overline{b}$ 
or $\gamma\gamma$; \textit{(iii)} $S$ can be much lighter than the Higgs, reducing the 
phase space suppression. 
Note that very light $S$, i.e., with $m_S < m_h/2\sim 62.5$ GeV, need not be
excluded by the LHC constraints on the Higgs width, $\Gamma(h\rightarrow S S) \lesssim 10$ MeV ~\cite{Khachatryan:2016ctc}. Indeed, 
for a quartic coupling $\lambda_{HS} S^2 |H|^2$, 
this bound is avoided for $\lambda_{HS} < 0.05$.


There are no direct experimental limits on $t \to q S$ from colliders. The indirect constraints from 1-loop box diagrams in 
$D^0-\bar{D}^0$ oscillations constrain the products of two $S$ Yukawas, $\tilde Y_{ut} \tilde Y_{ct(tc)}$, and $\tilde Y_{tu} \tilde Y_{ct(tc)}$, to be small~\cite{Bona:2007vi,Harnik:2012pb,Agashe:2013hma}. The $S\bar t c$ or $S\bar t u$ couplings can still be sizeable, but not both at the same time. Inspired by the CMS $t\to hc$ search~\cite{CMS:2017cck}, Ref.~\cite{Banerjee:2018fsx} developed a dedicated analysis for $t\to qS$, by varying $m_S$, the mass of $S$. The projections at the 14 TeV LHC for  $S \to b\bar{b}$ and $S \to \gamma \gamma$ decays  are shown in Fig.~\ref{fig:sbb2} .
Assuming that $S$ is the only light NP state, its couplings to the SM quarks are induced by dimension 5 effective operator (not displaying generational indices)
\begin{equation}\label{eq:lag}
 \mathcal{L} = -{\bar{q}_L}{\tilde{Y}}\frac{S}{f}  \tilde{H}{u_R} + \text{h.c.}\supset \tilde{g} \frac{m_t}{f} \bar{t}_L S c_R + \text{h.c.},
\end{equation}
where $f$ is the NP scale, and on the r.h.s. we introduced a new flavour violating coupling $\tilde g$. The three Benchmark Points (BP) shown in Figs. \ref{fig:sbb2} 
are
\begin{equation}
\label{eq:benchmarks:tqS}
\begin{split}
 \text{BP1(2,3)}:~~\tilde{g} = 1.0(1.0,0.1),~f=2(10,2)~\text{TeV}~~\Longrightarrow~~\BR(t\rightarrow Sc)\sim 10^{-3(4,5)}-10^{-2(3,4)}.
\end{split}
\end{equation}
If the flavour conserving couplings of $S$ to the SM fermions, $\psi$, are ${c_{\psi} m_\psi}
S(\bar{\psi}\psi)/f$, and to the photons ${c_{\gamma}\alpha} S F_{\mu\nu}\tilde{F}^{\mu\nu}/({4\pi f})$, then
$\mathcal{B}(S\rightarrow\gamma\gamma)/\mathcal{B}(S\rightarrow \overline{\psi}\psi) \sim 
({\alpha}/{\pi})^2 (m_S/m_\psi)^2$, 
taking $c_{\psi}\sim c_{\gamma}\sim \mathcal{O}(1)$.
 The suppression of $S\to \gamma\gamma$ can be partially 
compensated by scaling with $m_S$, so that $\mathcal{B}(S\to \gamma\gamma)$ can possibly be 
significantly larger than $\mathcal{B}(h\to \gamma\gamma)$. Searches should thus use both $S\to b\bar b$ and $S\to \gamma\gamma$. The details on how to reduce the backgrounds can be found in~\cite{Banerjee:2018fsx}.
Current searches for $S \rightarrow b\overline{b}$ in the gluon fusion 
channel~\cite{Sirunyan:2018ikr} constrain
only values of $c_\psi$ above $\sim 10$ for $f \sim 1$ TeV. Our analysis works
instead for very small values of $c_\psi$ provided the branching ratio 
is sizable.

\begin{figure}[t]
\includegraphics[width=0.45\textwidth]{\main/section10/img/bbChannel.pdf}~~~
\includegraphics[width=0.45\textwidth]{\main/section10/img/ggChannel.pdf}
 \caption{\label{fig:sbb2} Branching ratios that can be tested in the $b\overline{b}$ (left) and $\gamma\gamma$ (right) channels at 14 TeV HL-LHC with $3000$ fb$^{-1}$ (95\% CL upper limit is denoted by red dashed line). The three benchmark points, Eq. \eqref{eq:benchmarks:tqS}, are denoted with black lines. }
\end{figure}

For projections at future colliders we find that the increase in cross-section 
for the background at $\sqrt{s} = 27$ TeV ($100$ TeV) when compared to $\sqrt{s} = 14$ TeV is similar to 
that for the signal, and is $\sim 4(40)$. Assuming an integrated luminosity of $10$ ab$^{-1}$, 
we expect an increase in significance by a factor of $\sim 3.7$ ($\sim 11.5$). Similar results hold for the 
$b\overline{b}$ channel.

\subsubsection{Implications for di-Higgs production}
{\it\small Authors (TH): Martin Bauer, Marcela Carena and Adri\'an Carmona.}\\
Interestingly, in some models the flavour structure can feed back into nontrivial constraints on the scalar potential. This was demonstrated in the 2HDM model with FN charges, introduced in Sec.~\ref{sec:2HDM:flavour}.
The scalar couplings to gauge bosons are the same as in the normal type-I 2HDM while the scalar coupling between the heavy Higgs $H$ and two SM Higgs scalars $h$, as well as the triple Higgs coupling can be expressed as \cite{Boudjema:2001ii,Gunion:2002zf}
\begin{align}
\label{eq:mainn1}
g_{Hhh}&=\frac{c_{\beta-\alpha}}{v}\!\left[\big(1\!-\!f^h(\alpha,\beta)s_{\beta-\alpha}\big)\big(3M_A^2\!-\!2m_h^2\!-\!M_H^2\big)\!-\!M_A^2\right],\\
g_{hhh}&= -\frac{3}{v}\!\left[f^h(\alpha,\beta)c_{\beta-\alpha}^2(m_h^2-M_A^2)+m_h^2s_{\beta-\alpha}\right],\label{eq:mainn2}
\end{align}
where $M_A$ is the pseudoscalar mass. The $U(1)$ flavour symmetry restricts the number of allowed terms in the scalar potential forbidding, e.g., terms proportional to $H_1 H_2$. Interestingly, one can rewrite such self scalar interactions with the help of function $f^h(\alpha,\beta)$, since it is related to the combination $H_1 H_2^{\dagger}$ appearing in both the scalar potential and the higher dimensional operators generating different Yukawa couplings. Therefore, the parameter space for which $f^h(\alpha,\beta)\gg 1$ and  $c_{\beta-\alpha}\neq 0$ leads to maximally enhanced diagonal couplings of the SM Higgs to fermions \eqref{eq:diagocoup} as well as to enhanced trilinear couplings \eqref{eq:mainn1} and \eqref{eq:mainn2}. For maximally enhanced Yukawa couplings, the mass of the heavy Higgs $H$ cannot be taken arbitrarily large and resonant Higgs pair production has to be present. This correlation between the enhancement of the Higgs Yukawa couplings $\kappa^h_{\psi}$ and $\text{Br}(H \to hh)$ is illustrated for $M_H=M_A=M_{H^\pm}=500$ GeV in Fig. \ref{fig:BRvskappa} (left) where we plot the dependence of $\text{Br}(H \to hh)$ on $c_{\beta-\alpha} $ and $ t_\beta$ \cite{Bauer:2017cov}.  The dashed contours correspond to constant values of $|\kappa_{\psi}^h|$ for $n_{\psi}=1$. The correlation does not depend on the factor $n_\psi$, although  $n_\psi > 1$ leads to a larger enhancement. The two exceptions for which this correlation breaks down are the limits $c_{\beta-\alpha}\approx 0$ (disfavored in the flavour model) and $c_{\beta-\alpha}\approx \pm 1$ (disfavoured by SM Higgs couplings strength measurements). Depending on the structure of the Yukawa couplings, the value of $\kappa^h_\psi$ in Fig.~\ref{fig:BRvskappa} (left) can be larger or smaller than the value $n=1$ chosen to illustrate the relation between $g_{Hhh}$ and $\kappa^h_\psi$. Current experimental limits constrain this structure. For example, since $\kappa^h_\mu < 2.1$ \cite{ATLAS:2018kbw}, either $n=0$ for the muon, or one is constrained to the $\kappa^h_\mu < 2.1$ parameter space in Fig.~\ref{fig:BRvskappa} (left). 




%
 \begin{figure}
 \begin{center}
 \includegraphics[width=.52\textwidth]{\main/section10/img/BRvskap.pdf}~~~~~
 \includegraphics[width=.44\textwidth]{\main/section10/img/diff.pdf}
 \caption{\label{fig:BRvskappa} Left: $\text{Br}(H \to hh)$ as a function of $\cos(\beta-\alpha)$ and $ \tan\beta$ for $M_H=M_{H^\pm}=550$ GeV and $M_A=450$ GeV. The dashed contours correspond to constant $|\kappa_{\psi}^h|$ (we set $n_{\psi}=1$). Right: Invariant mass distribution for the different contributions to the $pp\to hh$ signal with $c_{\beta-\alpha}=-0.45$ and $\kappa^h_{\psi}=5$ (blue),  $\kappa_{\psi}^h=4$ (green) and $\kappa_{\psi}^h=3$ (red) at $\sqrt{s}=27 $ TeV, respectively.}
 \end{center}
  \vspace{-.6cm}
 \end{figure}
%


There is  a non-trivial interplay between resonant and non-resonant contributions to $pp\to hh$, as shown in Fig. \ref{fig:BRvskappa} (right), for $\sqrt{s}=27$ TeV, setting $M_A=450$ GeV and $M_{H}=M_{H^{\pm}}=550$ GeV,  $c_{\beta-\alpha}=-0.45$ and three different values of $\kappa_{\psi}^{h}=3, 4$ and $5$. When the enhancement in the Higgs Yukawa couplings is large enough, the interference between non-resonant and resonant contributions turns the broad peak into a shoulder in the $d\sigma/ dm_{hh}$ distribution for the total cross section, as shown for the case $\kappa_{\psi}^h=5$ by the blue line.

\subsubsection{Neutrino Mass Models at the HL/HE LHC}
{\it\small Authors (TH): T. Han, T. Li, X. Marcano, S. Pascoli, R. Ruiz, C. Weiland.}


The questions pertaining to neutrino masses: whether  or not  neutrinos are Majorana particles, the origin of smallness of the neutrino masses,  as well as the reason for  large mixing angles, remain some of the most pressing open issues in particle physics today.
A set of potential solutions is provided by the seesaw models. These postulate new particles that couple to SM fields via
mixing/Yukawa couplings, SM gauge currents, and/or new gauge symmetries.
If accessible, a plethora of rich physics can be studied in considerable detail at hadron colliders. 
This would complement low energy and oscillation probes of neutrinos~\cite{Atre:2009rg,Cai:2017mow}.
In the following, we summarize the discovery potential of seesaw models at hadron colliders with collision energies of $\sqrt{s} = 14$ and 27 TeV.
For a more comprehensive reviews on the sensitivity of colliders to neutrino mass models, see \cite{Atre:2009rg,Cai:2017mow,Weiland:2013wha,Ruiz:2015gsa,Marcano:2017ucg} and references therein.



\paragraph*{The Type I Seesaw and Variants}\label{sec:lhcSeesawtype1}
%
In Type I seesaw the light neutrino masses and mixing are generated from couplings of SM leptons to new fermionic gauge singlets with Majorana masses. 
For low-scale seesaw models with only fermionic singlets, 
lepton number has to be nearly conserved and light neutrino masses are proportional to small lepton number violating (LNV) parameters~\cite{Kersten:2007vk,Moffat:2017feq}.
For high-scale seesaws  with only fermionic singlets, light neutrino masses are inversely proportional to large LNV mass scales,
and again lepton number is approximately conserved at low energies.
Thus LNV processes are suppressed in type I seesaw models
(unless additional particles
are introduced to decouple the light neutrino mass generation from heavy neutrino production).





%
\begin{figure}[t]
\begin{center}
\includegraphics[width=0.7\textwidth]{\main/section10/img/lhcSeesaw_Feynman_HeavyNMultiProd_1706_02298.pdf}
\\[2mm]
\includegraphics[width=0.48\textwidth]{\main/section10/img/issVeto_Mixing_Vs_Mass_ETAU_tael_Global_FigL.pdf}~~~~~
\includegraphics[width=0.48\textwidth]{\main/section10/img/issVeto_Mixing_Vs_Mass_MUTAU_tamu_Global_FigR.pdf}
\end{center}
\caption{
Upper:
Born-level diagrams for heavy neutrino, $N$, production via (a) Drell-Yan (DY), (b) gluon fusion (GF), and (c) vector boson fusion (VBF). 
Lower: for the benchmark mixing hypotheses
 $\vert V_{e4}\vert = \vert V_{\tau4} \vert$ with $\vert V_{\mu4}\vert = 0$ (left panel) and
 $\vert V_{\mu4}\vert = \vert V_{\tau4} \vert$ with $\vert V_{e4}\vert = 0$ (right panel),
the projected sensitivity at $\sqrt{s}=14,~27$ and $100$ TeV using the tri-lepton + dynamic jet veto analysis of Ref.~\cite{Pascoli:2018rsg}.
}
\label{fig:lhcSeesaw_ISS}
\end{figure}
%

If kinematically accessible, heavy neutrinos $N$ can be produced in hadron collisions through neutral current and charged current processes,
as shown in Fig. \ref{fig:lhcSeesaw_ISS} (upper).
The expected suppression of LNV processes in type I seesaw models 
motivates the study of lepton number conserving (LNC) processes, such as 
the heavy neutrino $N$ production via DY and VBF, with subsequent decays to only leptons, 
\begin{equation}
 p p \to \ell_N N +X \to \ell_N \ell_W W +X \to \ell_N \ell_W \ell_\nu \nu +X,
 \label{eq:pp3lX}
\end{equation}
giving the trilepton final state, $\ell^\pm_i \ell^\mp_j \ell^\pm_k + \textrm{MET}$.
Projections from a new tri-lepton search strategy recently proposed in Ref.~\cite{Pascoli:2018rsg}, 
based on a dynamical jet veto selection cut, are shown in Fig.~\ref{fig:lhcSeesaw_ISS}, assuming the benchmark mixing hypotheses
 $\vert V_{e4}\vert = \vert V_{\tau4} \vert$ with $\vert V_{\mu4}\vert = 0$ (left panel) and
 $\vert V_{\mu4}\vert = \vert V_{\tau4} \vert$ with $\vert V_{e4}\vert = 0$ (right panel), for $\sqrt{s}=14,~27$ and $100$ TeV.
For benchmark luminosities, the colliders can probe active-sterile mixing as small as (approximately) $\vert V_{\ell 4}\vert^2 \sim 5\times10^{-4} - 9\times10^{-5}$
and masses as heavy as (approximately) $1.5-15$ TeV for $\vert V_{\ell 4}\vert^2 \sim 10^{-2}$.

Another possibility is to search for lepton flavor violating (LFV) final states such as
\begin{equation}
 q ~\overline{q}' \to  ~N ~\ell^\pm_1 \to ~\ell^\pm_1 ~\ell^\mp_2 ~ W^\mp  \to ~\ell^\pm_1 ~\ell^\mp_2 ~j ~j.
 \label{eq:LFVfinalstate}
\end{equation}
This was, e.g., studied in Ref~\cite{Arganda:2015ija} in the context of the inverse seesaw (ISS), a low-scale variant of the type I seesaw.
Due to strong experimental limits on $\mu\to e\gamma$ by
MEG~\cite{Adam:2013mnn}, the event rates involving taus are more promising
than those for $e^\pm \mu^\mp jj$.
After  $\mathcal L=3~\rm{ab}^{-1}$ of data taking, more than 100 LFV
events of $\tau^\pm \mu^\mp jj$ type could be produced for neutrino masses below 700 (1000)
GeV for $pp$ collisions at 14 (27)~TeV. 



%

%
\begin{figure}[t]
\begin{center}
\includegraphics[width=0.45\textwidth]{\main/section10/img/lhcSeesaw_PhenoType1_lumiVsMNMuMu14TeV_1411_7305.pdf}~~~~		
\includegraphics[width=0.45\textwidth]{\main/section10/img/lhcSeesaw_PhenoType1_sMuMuVsMN14TeV_1411_7305.pdf}		
\end{center}
\caption{
Left: required luminosity for $3~(5) \sigma$ evidence (discovery) using the LNV final state $\mu^\pm \mu^\pm jj$, as a function of the heavy neutrino mass, $m_N$, assuming optimistic (brown) 
and pessimistic (purple) mixing scenarios~\cite{Alva:2014gxa}. Right:
sensitivity to $N-\mu$ mixing~\cite{Alva:2014gxa} with the optimistic (pessimistic) mixing scenario is given by the horizontal dashed (full) line.
}
\label{fig:SeesawIDiscoveryPotential}
\end{figure}
%

In the presence of additional particles that can decouple the heavy neutrino production 
from the light neutrino mass generation, e.g., new but far off-shell gauge bosons~\cite{Ruiz:2017nip},
the Majorana nature of the heavy neutrinos can lead to striking LNV collider signatures,
such as the well-studied same-sign dilepton and jets process~\cite{Keung:1983uu}
\begin{equation}
 pp \rightarrow  ~N ~\ell^\pm_1 \rightarrow ~\ell^\pm_1 ~\ell^\pm_2 ~ W^\mp  \rightarrow ~\ell^\pm_1 ~\ell^\pm_2 +nj.
 \label{eq:LNVfinalstate}
\end{equation}
Assuming that a low-scale type I seesaw is responsible for the heavy neutrino production, 
Fig. \ref{fig:SeesawIDiscoveryPotential} displays the discovery potential and active-heavy
mixing sensitivity of the $\mu^\pm\mu^\pm$ channel~\cite{Alva:2014gxa}.
Assuming the  pessimistic/conservative mixing scenario of $S_{\mu\mu} = 1.1\times10^{-3}$~\cite{Alva:2014gxa}, 
the HL-LHC with 3 ab$^{-1}$ 
would be able to discover a heavy neutrino with a mass of $m_N\simeq400\,\mathrm{GeV}$ 
and is sensitive to masses up to $550\,\mathrm{GeV}$ at $3\sigma$. 
Using only 1 ab$^{-1}$, the HL-LHC can improve on the preexisting mixing constraints summarized in the pessimistic scenario 
for neutrino masses up to $500\,\mathrm{GeV}$.


%
\paragraph*{Heavy Neutrinos and the Left-Right Symmetric Model}\label{sec:lhcSeesawlrsmCollider}

%
\begin{figure}[!t]
\begin{center}
\includegraphics[scale=1,width=.4\textwidth]{\main/section10/img/lhcSeesaw_LRSM_ppWR_XSec_vs_Mass_1607_03504.pdf}~~~~~~~~~~ 	
\includegraphics[scale=1,width=.4\textwidth]{\main/section10/img/lhcSeesaw_LRSM_WR_jN_Disc_vs_Mass_1607_03504.pdf} 
\end{center}
\caption{
Left: The total $pp\to W_R$  cross section at NLO+NNLL(Threshold).
Right: $5(2)\sigma$ discovery potential (sensitivity) via $W_R$ decaying to an electron and neutrino jet $(j_N)$,
as a function of $W_R$ mass, at $\sqrt{s} = 14$ and 27 TeV~\cite{Mitra:2016kov}.
}
\label{fig:lrsm_HLvsHE_LHC}
\end{figure}
%

The Left-Right Symmetric Model (LRSM) 
addresses the origin of both tiny neutrino masses via a Type I+II seesaw hybrid mechanism 
as well as the SM $V-A$ chiral structure through spontaneous breaking of the $SU(2)_L\times SU(2)_R$ symmetry. 
The model predicts new heavy gauge bosons $(W_R^\pm,~Z_R')$,
heavy Majorana neutrinos $(N)$, and a plethora of neutral and electrically charged scalars $(H_i^0,~H_j^\pm,~H_k^{\pm\pm})$.
The LRSM gauge couplings are fixed to the SM Weak coupling constant, up to (small) RG-running corrections.
As a result, the Drell-Yan production mechanisms for $W_R$ and $Z_R$ result in large rates at hadron colliders.
 
One of the most promising discovery channels is the production of heavy Majorana neutrinos
from resonant $W_R$, with $N$ decaying via a lepton number-violating final state. At the partonic level, the process is 
~\cite{Keung:1983uu} (for details see \cite{Cai:2017mow} and references therein)
\begin{equation}
 q_1 \overline{q}_2 \to W_R \to N ~\ell^\pm_i \to \ell_i^\pm \ell_j^\pm W_R^{\mp *} \to \ell_i^\pm \ell_j^\pm q_1' \overline{q_2'}.
 \label{eq:seesawLRSMDY}
\end{equation}
Due to the ability to fully reconstruct the final state of Eq.~\eqref{eq:seesawLRSMDY}, many properties of $W_R$ and $N$ can be extracted,
including a complete determination of $W_R$ chiral couplings to quarks independent of leptons~\cite{Han:2012vk}.
Beyond the canonical $pp \to W_R \to N\ell \to 2\ell + 2j$ channel, it may be the case that the heavy neutrino is hierarchically lighter than the 
right-handed (RH) gauge bosons. 
Notably, for $(m_N/M_{W_R})\lesssim0.1$, $N$ is sufficiently Lorentz boosted that its decay products, particularly the charged lepton, 
are too collimated to be resolved experimentally~\cite{Ferrari:2000sp,Mitra:2016kov}.
%
Instead, one can consider the $(\ell_j^\pm q_1' \overline{q_2'})$-system as a single object, a \textit{neutrino jet}~\cite{Mitra:2016kov,Mattelaer:2016ynf}.
The hadronic process is then
 $ p p \to W_R \to N\ell_i^\pm \to j_N ~\ell_i^\pm$,
and inherits much of the desired properties of \eqref{eq:seesawLRSMDY}, 
such as the simultaneous presence of high-\pt charged leptons and lack of MET~\cite{Mitra:2016kov,Mattelaer:2016ynf},
resulting in a very strong discovery potential.
Fig. \ref{fig:lrsm_HLvsHE_LHC} (right) shows the requisite integrated luminosity for $5(2)\sigma$ discovery at $\sqrt{s}=14$ and 27 TeV.



%
\paragraph*{Type II Scalars}\label{sec:lhcSeesawtype2}
%

%
\begin{figure}[!t]
\begin{center}
\includegraphics[scale=1,width=.45\textwidth]{\main/section10/img/lhcSeesaw_Type2_L-MH-1tau-LHCX14_1802_0094.pdf}~~~~~~ 
\includegraphics[scale=1,width=.45\textwidth]{\main/section10/img/lhcSeesaw_Type2_L-MH-1tau-LHCX27_1802_0094.pdf}	
\end{center}
\caption{
Requisite luminosity  for $5(3)\sigma$ discovery (evidence) as a function of $M_{H^{\pm\pm}}$
for the process $pp\to H^{++}H^{--}\to \tau_h \ell^\pm \ell^\mp \ell^\mp$,
where $\tau^\pm\to \pi^\pm \nu $, 
for the NH and IH at $\sqrt{s}=14, 27$ TeV. 
}
\label{fig:lhcSeesawtypeiiScalars}
\end{figure}
%

Type II seesaw introduces a new scalar SU$(2)_L$ triplet that couples to SM leptons.
The light neutrinos obtain Majorana masses through  SU$(2)_L$ triplet vev, so that
type II scenario notably does not have sterile neutrinos.
The most appealing production mechanisms at hadron colliders of triplet Higgs bosons are
\begin{eqnarray}
pp\to Z^\ast/\gamma^\ast \to H^{++} H^{--}, \ \ \ pp\to W^\ast\to H^{\pm\pm}H^\mp,
\end{eqnarray}
followed, 
by lepton flavor- and lepton number-violating decays to the SM charged leptons.
In Type II scenarios, $H^{\pm\pm}$ decays to $\tau^\pm \tau^\pm$ and $\mu^\pm \mu^\pm$ pairs are comparable or greater than the
$e^\pm e^\pm$ channel by two orders of magnitude. 
Moreover, the $\tau \mu$ channel is typically dominant in decays involving different lepton flavors~\cite{Perez:2008ha,Li:2018jns}.
If such a seesaw is realized in nature, tau polarizations can help to determine the chiral property of triplet scalars.
One can discriminate between different heavy scalar mediated neutrino mass mechanisms, e.g., between Type II seesaw and Zee-Babu model, 
by studying the distributions of tau lepton decay products~\cite{Sugiyama:2012yw,Li:2018jns}. 
Due to the low $\tau_h$ identification efficiencies, 
future colliders with high energy and/or luminosity enables one to investigate and search for doubly charged Higgs decaying to $\tau_h$ pairs.
Accounting for constraints from neutrino oscillation data on the doubly charged Higgs branching ratios,
as well as  tau polarization effects~\cite{Li:2018jns},
Figs.~\ref{fig:lhcSeesawtypeiiScalars} displays
the 3$\sigma$ and 5$\sigma$ significance in the plane of integrated luminosity versus doubly charged Higgs mass 
for $pp\to H^{++}H^{--}\to \tau^\pm\ell^\pm\ell^\mp\ell^\mp$ at 
$\sqrt{s} = 14, 27$ TeV, for single $\tau$ channel with $\tau\to \pi {\nu}$, both for normal (NH) and inverted hierarchy (IH).



%
\paragraph*{Type III Leptons}\label{sec:lhcSeesawtype3}
%

\begin{figure}[!t]
\begin{center}
 \includegraphics[width=0.8\textwidth]{\main/section10/img/lhcSeesaw_Feynman_TypeIIIMultiProd_1711_02180.pdf}	
 \\[5mm]
  \includegraphics[scale=1,width=.40\textwidth]{\main/section10/img/lhcSeesaw_Type3_XSec_vs_Mass_1509_05416.pdf}~~~~~~~~~~~~~~~~ 
\includegraphics[scale=1,width=.40\textwidth]{\main/section10/img/lhcSeesaw_Type3_Disc_vs_Mass_1509_05416.pdf} 
\end{center}
\caption{
Upper:
Born level production of Type III leptons via (a) Drell-Yan, (b) gluon fusion, and (c) photon fusion; from \cite{Cai:2017mow}.
Lower left:
  the inclusive production cross section for  $pp\to NE^\pm + E^+E^-$, at NLO in QCD~\cite{Ruiz:2015zca} for $\sqrt{s}=14$ and 27 TeV, as a function of heavy triplet lepton mass. Lower right:
  the required integrated luminosity for 5(2)$\sigma$ discovery~(sensitivity) to $N E^\pm + E^+E^-$, 
 based on the analyses of~\cite{Arhrib:2009mz,Li:2009mw}.
}
\label{fig:lhcSeesawtype3}
\end{figure}

Low-scale Type III seesaws introduce heavy electrically charged $(E^\pm)$ and neutral $(N)$ leptons, 
part of SU$(2)_L$ triplet,
that couple to both SM charged and neutral leptons through mixing/Yukawa couplings.
Triplet leptons couple appreciably to EW gauge bosons,  
and thus do not have suppressed production cross section, contrary to seesaw scenarios with gauge singlet fermions.
Up to small (and potentially negligible) mixing effects 
%
the triplet lepton 
pair production cross sections are fully determined, see Fig. \ref{fig:lhcSeesawtype3} (upper) for relevant tree level diagrams for the production of heavy charged leptons.
%
%
%
%
Drell-Yan processes are the dominant production channel of triplet leptons when kinematically accessible~\cite{Cai:2017mow}.
%
Fig. \ref{fig:lhcSeesawtype3} (lower left) shows 
the summed cross sections for the Drell-Yan processes,
%
$ pp\to \gamma^*/Z^* \to E^+E^-$, and  
$ pp\to W^{\pm *} \to E^\pm N$,
%
%
 at NLO in QCD, following \cite{Ruiz:2015zca}, as a function of triplet masses (assuming $m_N = m_E$), at $\sqrt{s}=14$ and 27 TeV.
%
%


Another consequence of the triplet leptons coupling to all EW bosons is the adherence to the Goldstone Equivalence Theorem.
This implies that triplet leptons with masses well above the EW scale
will preferentially decay to longitudinal polarized $W$ and $Z$ bosons as well as to the Higgs bosons. 
For decays of EW boson to jets or charged lepton pairs, triplet lepton can be fully reconstructed from their final-state 
enabling their properties to be studied in detail.
%
For fully reconstructible final-states,
\begin{eqnarray}
 N E^\pm	\rightarrow ~\ell\ell' + WZ/Wh		&\rightarrow& ~\ell\ell' ~+~ nj+mb,
 \\
 E^+ E^- 	\rightarrow ~\ell\ell' + ZZ/Zh/hh 	&\rightarrow& ~\ell\ell' ~+~ nj+mb,
\end{eqnarray}
which correspond approximately to the branching fractions $\BR(NE)\approx0.115$ and $\BR(EE)\approx0.116$,
search strategies such as those considered in \cite{Arhrib:2009mz,Li:2009mw} can be enacted.
Assuming a fixed detector acceptance and efficiency of $\mathcal{A}=0.75$,
which is in line to those obtained by \cite{Arhrib:2009mz,Li:2009mw},
Fig. \ref{fig:lhcSeesawtype3}(lower right) shows as a function of triplet mass 
the requisite luminosity for a 5$\sigma$ discovery (solid) and 2$\sigma$ evidence (dash-dot)
of triplet leptons at $\sqrt{s}=14$ and 27 TeV.
%
With $\mathcal{L}=3-5$ ab$^{-1}$, the 14 TeV HL-LHC can discover states as heavy as $m_{N},m_{E}=1.6-1.8$ TeV.
For the same amount of data, the 27 TeV HE-LHC can discover heavy leptons  $m_{N},m_{E}=2.6-2.8$ TeV;
with $\mathcal{L}=15$ ab$^{-1}$, one can discover~(probe) roughly $m_{N},m_{E}=3.2~(3.8)$ TeV.

\subsection{Implications of TeV scale flavour models for electroweak baryogenesis} 
{\it\small Authors (TH): Oleksii Matsedonskyi, Geraldine Servant.}

%

In most solutions to the SM flavour puzzle, Yukawa couplings have a dynamical origin, which means that they potentially impact the cosmological evolution.
%
 In Refs.~\cite{Baldes:2016rqn,Baldes:2016gaf,vonHarling:2016vhf,Bruggisser:2017lhc,Bruggisser:2018mus,Bruggisser:2018mrt,Baldes:2018nel,Servant:2018xcs},  such connections between flavour and cosmology, in particular with the  electroweak baryogenesis,  have been investigated in detail.

Electroweak baryogenesis (EWBG) is a framework where the
matter-antimatter asymmetry of the universe was created during the electroweak phase transition. It 
 relies on a charge transport mechanism in the vicinity of bubble walls during a first-order electroweak (EW) phase transition. EWBG uses EW scale physics only and is therefore testable experimentally. It requires an extension of the Higgs sector, giving a first-order EW phase transition as well as new sources of \CP violation. Typically, there are stringent constraints on EWBG models from bounds on Electric Dipole Moments.
 In the following,  we consider situations where the source of \CP violation has changed with time, which is a natural way to evade constraints. 
 The main motivation is to link EWBG to low-scale flavour models.  If the physics responsible for the structure of the Yukawa couplings is linked to EW symmetry breaking, we can expect the Yukawa couplings to vary at the same time as the Higgs is acquiring a VEV, in particular if the flavour structure is controlled by a new scalar field which couples to the Higgs.
 This is  precisely what can happen in Composite Higgs (CH) models ~\cite{Kaplan:1983fs,Panico:2015jxa}, on which we focus in the following.


%

The  Composite Higgs (CH) models assume that the Higgs boson arises as a bound state of a new strong interaction, confining around the TeV scale $f$. Other composite resonances are naturally heavier than the Higgs, due to an approximate Goldstone symmetry suppressing the Higgs mass. The rest of the SM fields do not belong to the strong sector and are elementary. The nature of the EW phase transition in CH models can be substantially different with respect to the SM. One of the reasons is that the CH models naturally feature new scalar resonances that participate in the EW phase transition and change its properties (see Refs.~\cite{Espinosa:2011eu,Chala:2016ykx}). On the other hand, the EW transition may become strongly first order even without the help of such additional states, if the EW transition happens simultaneously with the deconfinement-confinement phase transition of the new strong sector (see Refs.~\cite{Bruggisser:2018mus,Bruggisser:2018mrt} and also Refs.~\cite{vonHarling:2016vhf,Megias:2018sxv} for the dual description in the warped 5D space).    

%

The flavour structure of the CH models is intimately tied with the viability of EWBG. The requirement of having a sizeable top quark Yukawa coupling, while suppressing the unwanted flavour-violating effects, suggests that the new sector is nearly scale-invariant
%
for a large range of energies above the confinement scale. As a result, one may expect that the transition dynamics is mostly determined by a single light field -- the dilaton $\chi$ (see e.g Ref.~\cite{Coradeschi:2013gda}). Depending on its mass, whose size can be related to the separation of the UV flavour scale and the EW scale, the EW phase transition may happen separately from the confinement, or simultaneously with it~\cite{Bruggisser:2018mus,Bruggisser:2018mrt}. %

Moreover, the mechanism for generating the SM flavour hierarchy in CH models may also be a source of \CP asymmetry during the EW phase transition. The hierarchy of SM Yukawas $\lambda_q$ 
%
is generated by the renormalization group running of the couplings $y_q$ between the elementary fermions and the strong sector operators, 
\begin{equation}
\lambda_q \propto y_q^2, \quad\text{with} \quad y_q=y_q^{\text{UV}} \left( \frac{\mu}{\Lambda_{\text{UV}}} \right)^{\gamma_{y_q}},
\end{equation}
where $\mu\sim \chi$ is the confinement scale, $\Lambda_{\text{UV}}$ is some large scale at which all the mixings are generated with a similar size, $y_q^{\text{UV}}$, and $\gamma_{y_q}$ is the anomalous dimension of the operator responsible for the mixing.
 This means that the size of the Yukawa couplings changes with the evolution of the confinement scale during the confinement phase transition. Such a change of the Yukawa interactions may efficiently source the CP-violation required for the baryogenesis~\cite{Bruggisser:2017lhc}.

%

\begin{figure}[!t]\centering
\includegraphics[scale=.44]{\main/section10/img/hvv.png}
\hspace{0.2cm}
\includegraphics[scale=.44]{\main/section10/img/Imhtt.png}
\caption{Relative deviation of the Higgs couplings to $W$ and $Z$ bosons (left panel) and the imaginary part of the correction to the top quark Yukawa coupling (right panel), as functions of the dilaton mass, $m_\chi$, the number of colours, $N$, in the new confining sector, and generic mass of composite states, $m_*$. Solid black (red dashed) contours correspond to the glueball-like (meson-like) dilaton. The current  and near future experimental sensitivities of electron EDM experiments  to the imaginary part of the top yukawa correspond respectively to approximately $2\times 10^{-3}$ \cite{Andreev:2018ayy,Brod:2013cka}
%
and $2\times 10^{-4}$~\cite{Kumar:2013qya}.
}
\label{fig:ewbg_observables}
\end{figure}

For both types of transitions  mentioned, one can expect to observe deviations of the Higgs couplings from the SM predictions. These deviations are a result of contributions generic to CH models, as well as those linked to the features of the phase transition and new sources of CP-violation. For concreteness we focus on the  more minimal example, the combined electroweak and strong sector phase transition. 
%
The potential of the Higgs boson can be parametrized in terms of trigonometric functions of $h/f$, with the overall size of the potential controlled by the mixings between the elementary and composite fermions~\cite{Matsedonskyi:2012ym,Panico:2012uw},
\begin{equation}\label{eq:ewbg_vh}
V = c_1 \sin^2 \frac h f + c_2 \sin^4 \frac h f,
\end{equation}
where $c_1 \sim c_2 \sim \sum_q ({3 y_q^2}/{(4 \pi)^2}) g_\star^2 f^4$. The dependence of the $y_q$ mixings on the dilaton field is responsible for the mass mixing between the dilaton and the Higgs field, which we parametrise by an angle $\delta$.
To generate sufficient amount of \CP violation, such mass mixing needs to be sizeable. Let us consider its effect on the quark Yukawa coupling:
\begin{equation}\label{eq:ewbg_yuk}
{\cal L}_{\rm Yukawa} = \lambda(\chi) \left(\chi \sin \frac h f \right) \bar q_L q_R 
= \bar q_L q_R h \left(\lambda(f) \frac {\chi}{f} + \beta_{\lambda} \frac {\chi - f }{f} \right) + \dots, 
\end{equation}
where we performed an expansion in $\chi$ around its present day value, $f$. Similar, but CP-preserving, modifications are also generated in the couplings of the Higgs boson to the SM gauge fields and the Higgs self-interactions.
The complex phase of the Yukawa beta-function, $\beta_\lambda$, in \eqref{eq:ewbg_yuk} has to be different from that of the quark mass $\lambda(f) h$, as the Yukawa phase changes with energy. 
Choosing the mass parameter to be real, the CP-violating interaction resides in the term $\propto \beta_\lambda$. Rotating the $h$ and $\chi$ fields to the mass basis, the CP-even and CP-odd corrections to the Higgs Yukawa  interaction are
\begin{equation}
\text{Re} [\delta \lambda] \sim \text{Re} [\beta_\lambda] \delta (v/f) + \lambda (\delta^2/2+\delta(v/f)) ,\quad
\text{Im} [\delta \lambda] \sim \text{Im} [\beta_\lambda]  \delta (v/f).
\end{equation} 
In Fig.~(\ref{fig:ewbg_observables}) we show the values of the CP-violating top quark Yukawa modification and the deviations of the Higgs couplings to the $W$ and $Z$ bosons.
Such couplings can be tested directly at the LHC, and also in the measurements of electric dipole moments.  
%
The strength of the phase transition can be tested in gravitational waves signals at the future space-based observatory LISA~\cite{Caprini:2015zlo}.

%
%
%
\subsection{High \pt searches in the context of flavour anomalies}

{\bf Authors:} Alejandro Celis, Admir Greljo, Lukas Mittnacht, Marco Nardecchia, Tevong You

Precision measurements of flavour transitions at low energies, such as flavour changing $B$, $D$ and $K$ decays, are sensitive probes of hypothetical dynamics at high energy scales.  These can provide the first evidence of new phenomena beyond the SM, even before direct discovery of new particles at high energy colliders. Indeed, the current anomalies observed in $B$-meson decays, in particular, the charged current one in $b\to c\tau \nu$ transitions, and neutral current one in $b\to s\ell^+\ell^-$,  may be the first hint of new dynamics which is still waiting to be discovered at high-\pt. When considering models that can accommodate the anomalies, it is crucial to analyse the constraints derived from high-\pt searches at the LHC, since these can often rule out significant regions of model parameter space. Below we review these constrains, and assess the impact of the High Luminosity and High Energy LHC upgrades (see also the discussion in Sections \ref{sec:7:bsll:NP},\ref{sec7:bsll:model:indep:fits}, and \ref{sec7:ModelsNP:bctaunu}).



\subsubsection{EFT analysis} 

If the dominant NP effects give rise to dimension-six SMEFT operators, the low-energy flavour measurements are sensitive to $C/\Lambda^2$, with $C$ the dimensionless NP Wilson coefficient and $\Lambda$ the NP scale.    The size of the Wilson coefficient is model dependent, and thus so is the NP scale required to explain the $R_{D^{(*)}}$and $R_{K^{(*)}}$ anomalies.    Perturbative unitarity sets an upper bound on the energy scale below which new dynamics need to appear~\cite{DiLuzio:2017chi}.    
The conservative bounds on the scale of unitarity violation are   $\Lambda_U = 9.2$ TeV and $84$ TeV for $R_{D^{(*)}}$and $R_{K^{(*)}}$, respectively, obtained when the flavour structure of NP operators is exactly aligned with what is needed to explain the anomalies.       More realistic frameworks for flavour structure, such as MFV, $U(2)$ flavour models, or partial compositeness, give rise to NP effective operators with largest effects for the third generation. This results in stronger unitarity bounds,
$\Lambda_U = 1.9$ TeV and $17$ TeV for $R_{D^{(*)}}$ and $R_{K^{(*)}}$, respectively.  
These results mean that: \textit{(i)} the mediators responsible for the $b\to c \tau \nu$ charged current anomalies are expected to be in the energy range of the LHC, \textit{(ii)}  the mediators responsible for the $b\to s \ell \ell$ neutral current anomalies could well be above the energy range of the LHC. However, in realistic flavour models also these mediators typically fall within the (HE-)LHC reach.       

If the neutrinos in $b \to c \tau \nu$ are part of a left-handed doublet, the NP responsible for $R_{D^{(*)}}$ anomaly
generically implies a sizeable signal in $p p \to \tau^+ \tau^-$ production at high-\pt. For realistic flavour structures, in which $b \to c$ transition is $\mathcal{O}(V_{cb})$ suppressed compared to $b \to b$, one expects rather large $b b \to \tau \tau$ NP amplitude. 
Schematically, $\Delta R_{D^{(*)}} \sim C_{b b \tau \tau} (1 + \lambda_{bs} / {V_{cb}})$, where $C_{b b \tau \tau}$ is the size of effective dim-6 interactions controlling $b b \to \tau \tau$, and $\lambda_{bs}$ is a dimensionless parameter controlling the size of flavour violation. Recasting ATLAS 13 TeV, $3.2$~fb$^{-1}$ search for $\tau^+ \tau^-$ \cite{Aaboud:2016cre}, Ref.~\cite{Faroughy:2016osc} showed that $\lambda_{bs} = 0$ scenario is already in slight tension with data. For $\lambda_{bs} \sim 5$, which is moderately large, but still compatible with FCNC constraints, HL (or even HE) upgrade of the LHC would be needed to cover the relevant parameter space implied by the anomaly (see Fig.~[5] in~\cite{Buttazzo:2017ixm}). 
For large $\lambda_{bs}$ the limits from $p p \to \tau^+ \tau^-$ become comparable with direct the limits on $p p \to \tau \nu$ from the bottom-charm fusion. The limits on the EFT coefficients from $p p \to \tau \nu$ were derived in Ref.~\cite{Greljo:2018tzh}, and the future LHC projections are  promising. The main virtue of this channel is that the same four-fermion interaction is compared in $b \to c \tau \nu$ at low energies and $b c \to \tau \nu$ at high-\pt. 
Since the effective NP scale in $R(D^{(*)})$ anomaly is low, the above EFT analyses are only indicative. 
For more quantitative statements we review below bounds on explicit models.

The hints of NP in $R_{K^{(*)}}$ require  a $(b s) (\ell \ell)$ interaction.
Correlated effects in high-\pt tails of $p p \to \mu^+ \mu^- (e^+ e^-)$ distributions are expected, if the numerators (denominators) of LFU ratios $R_{K^{(*)}}$ are affected. 
Ref.~\cite{Greljo:2017vvb} recast the  13~TeV $36.1$~fb$^{-1}$ ATLAS search \cite{ATLAS:2017wce} (see also \cite{Aaboud:2017buh}),
to set limits on a number of semi-leptonic four-fermion operators, and derive projections for HL-LHC (see Table 1 in~\cite{Greljo:2017vvb}). These show that direct limits on the $(b s) (\ell \ell)$ operator from the tails of distributions will never be competitive with those implied by the rare $B$-decays~\cite{Greljo:2017vvb,Afik:2018nlr}. On the other hand, flavour conserving operators, $(q q) (\ell \ell)$, are efficiently constrained by the high \pt tails of the distributions. The flavour structure of an underlining NP could thus be probed by constraining ratios $\lambda^q_{b s} = C_{b s} / C_{qq}$ with $C_{b s}$  fixed by the $R_{K^{(*)}}$ anomaly. For example, in models with MFV flavour structure, so that $\lambda^{u,d}_{b s} \sim V_{c b}$, the present high-\pt dilepton data is already in slight tension with the anomaly~\cite{Greljo:2017vvb}. Instead, if  couplings to valence quarks are suppressed, e.g., if NP dominantly couples to the 3rd family SM fermions, then $\lambda^b_{b s} \sim V_{c b}$. Such NP will hardly be probed even at the HL-LHC, and  it is possible that NP responsible for the neutral current anomaly might stay undetected in the high-\pt tails at HL-LHC and even at HE-LHC. Future data will cover a significant part of viable parameter space, though not completely, so that discovery is possible, but not guaranteed.


\subsubsection{Constraints on simplified models for $b\to c \tau \nu$}


Since the $b\to c \tau \nu$ decay is a tree-level process in the SM that receives no drastic suppression, models that can explain these anomalies necessarily require a mediator that contributes at tree-level:      


\noindent  $\bullet$ \, \textit{SM-like $W^{\prime}$}:  A SM-like $W^{\prime}$ boson, coupling to left-handed fermions, would explain the approximately equal enhancements observed in $R(D)$ and $R(D^{*})$.     A possible realization is a color-neutral real $SU(2)_L$ triplet of massive vector bosons~\cite{Greljo:2015mma}.  However, typical models encounter problems with current LHC data since they result in large contributions to $pp \to \tau^+ \tau^-$ cross-section, mediated by the neutral partner of the $W^{\prime}$~\cite{Greljo:2015mma,Boucenna:2016qad,Faroughy:2016osc}.    For $M_{W^{\prime}} \gtrsim 500$~GeV, solving the $R(D^{(*)})$ anomaly within the vector triplet model while being consistent with $\tau^+ \tau^-$ resonance searches at the LHC is only possible if the related $Z^{\prime}$ has a large total decay width~\cite{Faroughy:2016osc}. Focusing on the $W^{\prime}$, Ref.~\cite{Abdullah:2018ets} analyzed the production of this mediator via $gg$ and $gc$ fusion, decaying to
$\tau \nu_{\tau}$. Ref.~\cite{Abdullah:2018ets} concluded that a dedicated search using that $b$-jet is present in the final state would be effective in reducing the SM background compared to an inclusive analysis that relies on $\tau$-tagging and $E_T^{\rm{miss}}$. Nonetheless, relevant limits will be set by an inclusive search in the future~\cite{Greljo:2018tzh}.


\vspace{0.2cm}

\noindent  $\bullet$  \,  \textit{Right-handed $W^{\prime}$:} Refs.~\cite{Greljo:2018ogz,Asadi:2018wea} recently proposed that $W^{\prime}$ could mediate a right-handed interaction, with a light sterile right-handed neutrino carrying the missing energy in the $B$ decay. In this case, it is possible to completely decorelate FCNC constraints from $R(D^{(*)})$. The most constraining process in this case is instead $p p \to \tau \nu$. Ref.~\cite{Greljo:2018tzh} performed a recast of the latest ATLAS and CMS searches at 13 TeV and about 36 fb$^{-1}$ to constrain most of the relevant parameter space for the anomaly.

\vspace{0.2cm}

\noindent  $\bullet$\, \textit{Charged Higgs $H^{\pm}$:}       Models that introduce a charged Higgs, for instance a two-Higgs-doublet model, also contain additional neutral scalars.  Their masses are constrained by electroweak precision measurements to be close to that of the charged Higgs.  Accommodating the $R(D^{(*)})$ anomalies with a charged Higgs typically implies large new physics contributions to $pp \to \tau^+ \tau^-$ via the neutral scalar exchanges, so that current LHC data can challenge this option~\cite{Faroughy:2016osc}.     Note that a charged Higgs also presents an important tension between the current measurement of $R(D^*)$ and the measured lifetime of the $B_c$ meson~\cite{Li:2016vvp,Alonso:2016oyd,Celis:2016azn,Akeroyd:2017mhr}.


\begin{figure}[t]
\centering
\includegraphics[width=0.39 \textwidth]{section10/img/LQ-search-strategy.pdf}
\caption{Schematic of the LHC bounds on  LQ showing complementarity  in constraining the  $(m_{LQ},y_{q_{\ell}})$ parameters. The three cases are:
 pair production $\sigma \propto y_{q_{\ell}}^0 $, single production $\sigma \propto y_{q_{\ell}}^2 $ and Drell-Yan $\sigma \propto y_{q_{\ell}}^4$ (from~\cite{Dorsner:2018ynv}).}
\label{fig:LQstrategy}
\end{figure}

\begin{figure}[t]
\centering
\includegraphics[width=0.46 \textwidth]{section10/img/S3.pdf}~~~
\includegraphics[width=0.46 \textwidth]{section10/img/U1.pdf}
\caption{ \small \Blu{Present constraints and HL(HE)-LHC projections in the leptoquark mass versus coupling  plane for the scalar leptoquark $S_3$ (left), and vector leptoquark $U_1$ (right). The grey and dark grey solid regions are the current exclusions. The grey and black dashed lines are the projected reach for HL-LHC (pair and single leptoquark production prospects are based on the CMS simulation from Section XXX). The red dashed lines are the projected reach at HE-LHC (see Section \ref{sec:HE_LHC_LQ}). The green and yellow bands are the $1\sigma$ and $2\sigma$ preferred regions from the fit to $B$ physics anomalies. The second coupling required to fit the anomaly does not enter in the leading high-\pt diagrams but it is relevant for fixing the preferred region shown in green, for more details see Ref.~\cite{Buttazzo:2017ixm}. 
}
}
\label{fig:LQmediators}
\end{figure}

\vspace{0.2cm}

\noindent  $\bullet$ \, \textit{Leptoquarks:} 

The observed anomalies in charged and neutral currents appear in semileptonic decays of the $B$-mesons. This implies that the putative NP has to couple to both quarks and leptons at the fundamental level. A natural BSM option is to consider mediators that couple simultaneously quarks and leptons at the tree level. Such states are commonly referred as leptoquarks. Decay and production mechanisms of the LQ are directly linked to the physics required to explain the anomalous data.
\begin{itemize}
\item {\bf Leptoquark decays:} the fit to the $R(D^{*})$ observables suggest a rather light leptoquark (at the TeV scale) that couples predominately to the third generation fermions of the SM. A series of constraints from flavour physics, in particular the absence of BSM effects in kaon and charm mixing observables, reinforces this picture. 

\item {\bf Leptoquark production mechanism:}  The size of the couplings required to explain the anomaly is typically very large, roughly $y_{q\ell} \approx m_{LQ}$/ (1 TeV). Depending of the actual sizes of the leptoquark couplings and its mass we can distinguish three regimes that are relevant for the phenomenology at the LHC:
 	\begin{enumerate}
	\item LQ pair production due to strong interactions,
	\item Single LQ production plus lepton via a single insertion of the LQ coupling, and
	\item Non-resonant production of di-lepton through $t$-channel exchange of the leptoquark.
	\end{enumerate}
Interestingly all three regimes provide complementary bounds in the  $(m_{LQ},y_{q_{\ell}})$ plane, see Fig.~\ref{fig:LQstrategy}.

\end{itemize}




Several simplified models with leptoquark as a mediator were shown to be consistent with the low-energy data.
 A vector leptoquark with $SU(3)_c\times SU(2)_L\times U(1)_Y$ SM quantum numbers $U_\mu \sim ({\bf 3}, {\bf 1}, 2/3)$ was identified as the only single mediator model which can simultaneously fit the two anomalies (see e.g.~\cite{Buttazzo:2017ixm} for a recent fit including leading RGE effects). 
 In order to substantially cover the relevant parameter space, one needs future HL- (HE-) LHC, see Fig.~\ref{fig:LQmediators} (right) (see also Fig.~5 of~\cite{Buttazzo:2017ixm} for details on the present LHC constraints). A similar statement applies to an alternative model featuring two scalar leptoquarks, $S_1, S_3$ \cite{Crivellin:2017zlb}. The pair of plots in Fig. \ref{fig:LQmediators} summaries the current exclusion and the discovery reach for the HE and HL-LHC in the LQ coupling versus mass plane. 
 
 \Blu{Leptoquarks states are emerging as the most convincing mediators for the explanations of the flavour anomalies. It is then important to explore all the possible signatures at the HL- and HE-LHC.
 The experimental program should focus not only on final states containing quarks and leptons of the third generation, but also on the whole list of decay channels including the off-diagonal ones ($b \mu$, $s \tau, \dots$). The completeness of this approach would allow to shed light on the flavour structure of the putative New Physics. 
 
 Another aspects to be emphasized concerning leptoquark models for the anomalies is the fact of the possible presence of extra fields required to complete the UV lagrangian. The accompanying particles would leave more important signatures at high $p_T$ than the leptoquark, this is particularly true for vector leptoquark extensions (see for example \cite{DiLuzio:2017vat,DiLuzio:2018zxy})}  
    
 
%\Blu{As a final remark, it is important to remember that the anomalies are not yet experimentally established. Among others, this also means that the statements on whether or not  the high \pt LHC constraints rule out certain $R(D^{(*)})$ explanations assumes that the actual values of $R(D^{(*)})$ are given by their current global averages. If future measurements decrease the global average, the high \pt constraints can in some cases be greatly relaxed and HL- and/or HE-LHC may be essential for these, at present tightly constrained, cases.}


%%%%%%%%%%%
\subsubsubsection{HE-LHC sensitivity study for $p p \to \Phi \Phi \to (b \tau) (b \tau)$}
\label{sec:HE_LHC_LQ}
\begin{figure}[t]
	\centering
	\includegraphics[width=0.5\textwidth]{section10/img/soverb.pdf}
	\caption{Expected sensitivity for pair production of scalar (\Blu{$S_3$}) and vector ($U_1$) leptoquark at 27 TeV $p p$ collider with an integrated luminosity of $15~\text{ab}^{-1}$.} \label{fig:lukas}
\end{figure}

We analysed the sensitivity of the 27 TeV $pp$ collider with $15$~ab$^{-1}$ integrated luminosity to probe pair production of the scalar and vector leptoquarks decaying to $(b \tau)$ final state. We investigated events containing either one electron or muon, one hadronically decaying tau lepton, and  at least two jets. The signal events and the dominant background events ($t\bar{t}$) were generated with {\tt MadGraph5\_aMC@NLO} at leading-order. {\tt PYTHIA6} was used to shower and hadronise events and {\tt DELPHES3} was used to simulate the detector response. The scalar leptoquark (\Blu{$S_3$}) and the vector leptoquark ($U_1$) UFO model files were taken from~\cite{Dorsner:2018ynv}.

To verify the procedure, we simulated the $t\bar{t}$ background and the scalar leptoquark signal at 13 TeV and compared to the predicted shapes in the $S_T$ distribution from the existing CMS analysis~\cite{Sirunyan:2017yrk}. After we verified the 13 TeV analysis, we simulated the signal and the dominant background events at $27$ TeV. From these samples, we selected all events satisfying the particle content requirements and applied the lower cut in the $S_T$ variable. The cut threshold was chosen to maximize $s/\sqrt{b}$ while requiring at least 2 expected signal events at an integrated luminosity of $15~\text{ab}^{-1}$. In the case of the vector leptoquark we considered the Yang-Mills ($\kappa = 1$) and the minimal coupling ($\kappa = 0$) scenarios for the couplings to the gluon field strength, ${\cal L}\supset -ig \kappa U_{1\mu}^\dagger T^a U_{1\nu} G^a_{\mu\nu}$~\cite{Dorsner:2018ynv}. From the simulations of the scalar and vector leptoquark events, we found the ratio of the cross-sections, and assuming similar kinematics, we estimated the sensitivity also for the vector leptoquark $U_1$.

As shown in Fig.~\ref{fig:lukas}, the HE-LHC collider will be able to probe pair produced third generation scalar leptoquark (decaying exclusively to $b \tau$ final state) up to mass of $\sim 4$~TeV and vector leptoquark up to $\sim 4.5$~TeV and $\sim 5.2$~TeV for the minimal coupling and Yang-Mills scenarios, respectively. While this result is obtained by a rather crude analysis, it shows the impressive reach of the future high-energy $pp$-collider. In particular, the HE-LHC will cut deep into the relevant perturbative parameter space for $b \to c \tau \nu$ anomaly. As a final comment, this is a rather conservative estimate of the sensitivity to leptoquark models solving $R(D^{*})$ anomaly, since a dedicated single leptoquark production search is expected to yield even stronger bounds~\cite{Buttazzo:2017ixm,Dorsner:2018ynv}.
%%%%%%%%%%%







\subsubsection{Constraints on simplified models for $b\to s ll$}

The $b\to s ll$ transition is  both loop and CKM suppressed in the SM. The explanations of the $b\to s ll$ anomalies can thus have both tree level and loop level mediators.
% enter at tree-level or at the loop level. 
Loop-level explanations typically involve lighter particles.
% that enter at a lower UV cut-off scale. 
Tree-level mediators can also be light, if sufficiently weakly coupled. However, they can also be much heavier---possibly beyond the reach of the LHC. 

%\begin{figure}[t]
%\centering
%\includegraphics[width=0.45 %\textwidth]{section10/img/feynmandiagrams_LQZprime.PNG}
%\caption{Tree level Feynman diagrams for leptoquarks (left) and $Z^\prime$ (right)  mediating $b \to s \mu^+ \mu^-$ transition (Fig. taken from ~\cite{Allanach:2017bta}).}
%\label{fig:feynmanLQZprime}
%\end{figure}

\noindent $\bullet$ \textit{Tree-level mediators:}\\
%For the four-fermion operators involved in 
For $b \to sll$ anomaly there are two possible tree-level UV-completions, the $Z'$ vector boson and leptoquarks, either scalar or vector (see Fig.~\ref{fig:bsll_diagrams} in Sec.~\ref{sec:7:bsll:NP}).
% whose Feynman diagrams are depicted in Fig.~\ref{fig:feynmanLQZprime}. The $Z^\prime$ vector boson mediates an interaction between the $b$ and $s$ quark at one vertex and two muons at the other. A leptoquark is defined as having quantum numbers such that it may interact with a quark and lepton at one vertex, and can be either scalar or vector. In this case the minimal couplings required of the leptoquark must contain a $b$ or $s$ together with a muon at a vertex. 
%Ref.~\cite{Allanach:2017bta} 
%%initiated the study of implications of the anomalies for direct discovery at future colliders. They 
%estimated the sensitivity to simplified models of $Z^\prime$ and leptoquarks at the HL- and HE-LHC (as well as the FCC-hh) in the expected parameter space where the new particles may explain the neutral current flavour anomalies. A more detailed study for the $Z^\prime$ case was then made by Ref.~\cite{Allanach:2018odd}, extending in particular to wider widths. We summarise their results here. 
For leptoquarks, Fig.~\ref{fig:LQexclusion} shows the current 95\% CL limits from 8 TeV CMS with 19.6 fb$^{-1}$  in the $\mu\mu jj$ final state (solid black line), as well as the HL-LHC (dashed black line) and 1 (10) ab$^{-1}$ HE-LHC extrapolated limits (solid (dashed) cyan line). 
%is denoted by a dashed black line for HL-LHC and by a solid (dashed) cyan line for HE-LHC with 1 (10) ab$^{-1}$. 
Dotted lines give the cross-sections times branching ratio at the corresponding collider energy for pair production of scalar leptoquarks, calculated at NLO using the code of Ref.~\cite{Kramer:2004df}. We see that the sensitivity to a leptoquark with only the minimal $b-\mu$ and $s-\mu$ couplings reaches around 2.5 and 4.5 TeV at the HL-LHC and HE-LHC, respectively. This pessimistic estimate is a lower bound that will typically be improved in realistic models with additional flavour couplings. Moreover, the reach can be extended by single production searches~\cite{Allanach:2017bta}, albeit in a more model-dependent way than pair production. The cross section predictions for vector leptoquark are more model-dependent and are not shown in Fig.~\ref{fig:LQexclusion}. For $\mathcal{O}(1)$ couplings the corresponding limits are typically stronger than for scalar leptoquarks. 

\begin{figure}[t]
\centering
\includegraphics[width=0.45 \textwidth]{section10/img/LQ_exclusionplot_HLLHC_HELHC.png}
\caption{Current and projected 95\% CL limits on $\mu\mu jj$ final state at CMS (solid black) and HL-LHC (cyan) with 1 (10) ab$^{-1}$ in solid (dashed) lines. The pair production cross-section for scalar leptoquarks is shown in dotted lines for 14 (27) TeV in black (cyan) (Fig. taken from ~\cite{Allanach:2017bta}). \Blu{See also Section (XXX-->Cambridge group)}}
\label{fig:LQexclusion}
\end{figure}

\begin{figure}[t]
\centering
\includegraphics[width=0.39 \textwidth]{section10/img/14TeVMDM.pdf}
\caption{ HL-LHC 95\% (blue) and 99\% (orange) CL sensitivity contours to $Z^\prime$ in the ``mixed-down'' model for $g_{\mu\mu}$ vs $Z^\prime$ mass in TeV. The dashed grey contours give the $Z'$ width as a fraction of mass. The green and red regions are excluded by trident neutrino production and $B_s$ mixing, respectively. The dashed blue line is the stronger $B_s$ mixing constraint from Ref.~\cite{Bazavov:2016nty} (Fig. taken from~\cite{Allanach:2018odd}). \Blu{See also Section (XXX-->Cambridge group)} }
\label{fig:Zprime14TeV}
\end{figure}

For the $Z^\prime$ mediator the minimal couplings in the mass eigenstate basis are obtained by unitary transformations from the gauge eigenstate basis, which necessarily induces other couplings.  Ref.~\cite{Allanach:2018odd} defined the ``mixed-up'' model (MUM) and ``mixed-down'' model (MDM) such that the minimal couplings are obtained via CKM rotations in either the up or down sectors respectively. For MUM there is no sensitivity at the HL-LHC.  The predicted sensitivity at the HL-LHC for the MDM is shown in Fig.~\ref{fig:Zprime14TeV} as functions of $Z^\prime$ muon coupling $g_{\mu\mu}$ and the $Z^\prime$ mass, setting the $Z'$ coupling to $b$ and $s$ quarks such that it solves the $b\to s\ell\ell$ anomaly. The solid blue (orange) contours give the 95\% and 99\% CL sensitivity. The red and green regions are excluded by $B_s$ mixing~\cite{Arnan:2016cpy} and neutrino trident production~\cite{Falkowski:2017pss,Falkowski:2018dsl}, respectively. The more stringent $B_s$ mixing constraint from Ref.~\cite{Bazavov:2016nty} is denoted by the dashed blue line; see, however, Ref.~\cite{Kumar:2018kmr} for further discussion regarding the implications of this bound. The dashed grey contours denote the width as a fraction of the mass. We see that the HL-LHC will only be sensitive to $Z^\prime$ with narrow width, up to masses of 5 TeV. 

\begin{figure}[t]
\centering
\includegraphics[width=0.39 \textwidth]{section10/img/27TeVMUM.pdf}~~~~~~~~~
\includegraphics[width=0.39 \textwidth]{section10/img/27TeVMDM.pdf}
\caption{ HE-LHC 95\% (blue) and 99\% (orange) CL sensitivity contours to $Z^{\prime}$ in the ``mixed-up'' (left) and ``mixed-down'' (right) model in the parameter space of $g_{\mu\mu}$ vs $Z^{\prime}$ mass in TeV. The dashed grey contours are the width as a fraction of mass. The green and red regions are excluded by trident neutrino production and $B_s$ mixing. The dashed blue line is the stronger $B_s$ mixing constraint from Ref.~\cite{Bazavov:2016nty} (Figs. taken from~\cite{Allanach:2018odd}). \Blu{See also Section (XXX-->Cambridge group)}}
\label{fig:Zprime27TeV}
\end{figure}

At the HE-LHC, the reach for 10 ab$^{-1}$ is shown in Fig.~\ref{fig:Zprime27TeV} for the MUM and MDM on the left and right, respectively. In this case the sensitivity may reach a $Z^{\prime}$ with wider widths up to 0.25 and 0.5 of its mass, while the mass extends out to 10 to 12 TeV. We stress that this is a pessimistic estimate of the projected sensitivity, particular to the two minimal models; more realistic scenarios will typically be easier to discover. 


\vspace{0.2cm}

\noindent  $\bullet$ \textit{Explanations at the one-loop level:}\\
It is possible to accomodate the $b \to s \ell^+ \ell^-$ anomalies even if mediators only enter at one-loop.  One possibility is the mediators coupling to right-handed top quarks and to muons~\cite{Celis:2017doq,Becirevic:2017jtw,Kamenik:2017tnu,Fox:2018ldq,Camargo-Molina:2018cwu}.   Given the loop and CKM  suppression of the NP contribution to the $b \to s \ell^+ \ell^-$ amplitude, these models can explain the $b \to s \ell^+ \ell^-$ anomalies for a light mediator, with mass around $\mathcal{O}(1)$~TeV or lighter.       Constraints from the LHC and future projections for the HL-LHC were derived in~\cite{Camargo-Molina:2018cwu} by recasting di-muon resonance, $pp\to t\bar tt \bar t$ and SUSY searches.  Two scenarios were considered: \textit{i)} a scalar LQ $R_2(3, 2, 7/6)$ combined with a vector LQ $\widetilde U_{1 \alpha}(3, 1, 5/3)$, \textit{ii)} a vector boson $Z^{\prime}$ in the singlet representation of the SM gauge group.    Ref.~\cite{Fox:2018ldq} also analyzed the HL-LHC projections for the $Z^{\prime}$.     The constraints from the LHC already rule out part of the relevant parameter space and the HL-LHC will be able to cover much of the remaining regions.    Dedicated searches in the $pp\to t\bar tt \bar t$ channel and a dedicated search for $t \mu$ resonances in $t \bar t \mu^+ \mu^-$ final state can improve the sensitivity to these models~\cite{Camargo-Molina:2018cwu}. 

\subsubsection{Conclusions regarding high \pt probes of flavour anomalies}

The anomalous results in $B$-meson decays cannot be considered yet as a convincing evidence of New Physics. On the other hand, the number (and quality) of observables that are not in complete agreement with the SM prediction is growing with time in a coherent way. If true, the implications for HEP will be profound. 
%of the flavour anomalies will have a profound impact for the physics case of the future strategies for the HEP community. 
The conclusions we draw are the following:
\begin{itemize}
\item[$\bullet$] A conservative argument based on perturbative unitarity \cite{DiLuzio:2017chi} sets an upper bound on the New Physics scale to be 9 TeV for charged current anomalies and 80 TeV for neutral current ones. 
%This does not allow to formulate a no-loose theorem for the high-luminosity program. However from 
The analysis of explicit models show that the high-luminosity program has a clear potential to probe a large portion of the possible BSM options. 

\item[$\bullet$] The explanation of the anomalies in $b \to c \tau \nu$ transitions requires non trivial model building. In particular, it is not possible to simply isolate the physics that mediate the flavour anomalous transitions. In complete models other signature have to be considered. Typically it would be very difficult to escape direct detection at HL/HE-LHC.

\item[$\bullet$] Even though the naive scale associated with the $b \to s \ell \ell$ anomalies   is much higher than the energy accessible at HL/HE-LHC, in motivated models the flavour suppressions and weak couplings guarantee a large coverage of the parameter space for leptoquark and $Z'$ mediators \cite{Allanach:2018odd}.
\end{itemize}

More generally, the probes of lepton flavour universality such as the ratios of inclusive $\tau^+\tau^-$ vs. $\mu^+\mu^-$ (or $\mu^+\mu^-$ vs. $e^+e^-$) mass distributions, are important, theoretically clean, tests of the SM and are well motivated observables both at HL- and HE-LHC, whether or not the current $B$-meson anomalies become statistically significant.





\end{document}
