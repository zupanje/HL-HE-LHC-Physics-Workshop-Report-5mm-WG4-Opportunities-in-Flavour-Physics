% don't remove the folling lines, and edit the defintion of \main if needed
\documentclass[../report.tex]{subfiles}
\providecommand{\main}{..}
\IfEq{\jobname}{\currfilebase}{\AtEndDocument{\biblio}}{}
% until here

\begin{document}

\section{The high $p_T$ flavour physics program} 
%(anticipated length 15-20 pages)
\label{sec:highpT}

Flavour and high $p_T$ searches are intertwined in several ways. On the one hand the stringent bounds from low-energy constraints put severe bounds on the NP models that contain states with TeV masses with couplings to quarks.  On the other hand, TeV scale New Physics is suggested by several solutions of long standing problems of the SM, e.g., the hierarchy problem. Quite often the low energy constraints are avoided by assuming Minimal Flavour Violation (MFV), where the only flavour breaking, even in the NP sector, is due to the SM Yukawa matrices. However, more general flavour structures for the NP states are still allowed. In fact such non-MFV couplings can have interesting consequences. In general we can group the NP models into two broad classes: \textit{(i)} models that address outstanding problems of the SM, such at the SM flavour puzzle, the origin of dark matter, or the hierarchy problem, which may have non trivial flavour structure,  and \textit{(ii)} models designed to explain the $b\to s\ell\ell$ and $b\to c\tau \nu$ flavour anomalies, that almost inevitably have quite a distinct flavour structure.  The NP mediators potentially responsible for the anomalies, $Z'$, $W'$ or leptoquarks, could be found at high $p_T$ searches in the HL- or HE-LHC.

%\td{need to write a general introduction} Flavour and high $p_T$ searches are intertwined in several ways. On the one hand the stringent bounds from low energy constraints put severe bounds on the NP models that contain states with TeV masses with couplings to quarks. Quite often these bounds are avoided by assuming Minimal Flavour Violation (MFV), where the only flavour breaking, even in the NP sector, is due to the SM Yukawa matrices. However, more general flavour structures for the NP states are still allowed. In fact such non-MFV couplings can have interesting consequences. In general we can group the NP models into two broad classes: (i) NP models that aim to explain the SM flavour puzzle, and (ii) models that solve one of the outstanding problems particles physics, e.g., the origin of dark matter, or the hierarchy problem, but with nontrivial flavour structure in NP sector that leads to modified high $p_T$ searches. Such a nontrivial flavour structure is especially motivated in the models of NP that aim to explain the recent anomalies. 
%(not too many of these have states at TeV, though)

In the rest of this Section we first briefly review the models that address the SM flavour puzzle and have states that could be probed at the HL-/HE-LHC, and their implications for high $p_T$ searches. The second part of this Section is devoted to  high $p_T$ implications of $B$ physics anomalies. 
% ... \td{say what was done}

\subsection{Models of flavour and TeV Physics}
%(anticipated length 3 pages)
%{\bf Author(s): Martin Bauer, Adrian Carmona, Geraldine Servant?} 
%\subsubsection{Overview of TeV scale flavour models} 

%\td{quickly introduce the models}

\subsubsection{\bf Randall-Sundrum models of flavour}
%\td{need to update and shorten}
%One of the most interesting models of New Physics
%is based on the idea of a 
Models of flavour based on warped extra dimension~\cite{Randall:1999ee} attempt to simultaneously solve
% . 
%This notion has great appeal as it can lead to a simultaneous resolution 
%to
 the hierarchy problem as well as the SM flavour 
problem~\cite{Grossman:1999ra,Gherghetta:2000qt}.
 %problem of the SM
%by accomodating rather naturally
%the observed large disparity 
%of fermion masses
%For lack of space we do not discuss the implications of 
%universal extra dimensions, for which we refer the reader 
%to the recent review by \cite{Hooper:2007qk}. 
In the Randall-Sundrum (RS) models the 5-dimensional space-time has 
anti-de Sitter geometry (AdS$_5$), truncated by flat 4D boundaries, 
the Planck (UV) and the TeV (IR) branes.  
This setup gives a warped metric in the bulk~\cite{Randall:1999ee}  
%\begin{equation}
$ds^2 = \exp(-2 k r_c |\phi| ) \eta_{\mu\nu} dx^\mu dx^\nu - r_c^2 d\phi^2$,
%\label{metric}
%\end{equation}
where $k$ is the 5D curvature scale, $r_c$ the radius of compactification and  
$\phi\in [-\pi,\pi]$ the coordinate along the $5$-th dimension. 
The warp factor, $\exp({-2 k r_c |\phi|})$, leads to different length scales 
in different 4D slices along the $\phi$ direction, 
which provides a solution to the hierarchy problem.  
In particular, the Higgs field is assumed to be localized
near the TeV-brane so that the metric ``warps'' 
$\langle{H}\rangle_5 \sim M_5\sim M_P \sim 10^{19}$~GeV down to the weak scale, 
$\langle{H}\rangle_4 = \exp({-kr_c \pi}) \langle{H}\rangle_5$.  
For $kr_c \approx 12$ then $\langle{H}\rangle_{\rm SM} \equiv \langle{H}\rangle_4 \sim 1$~TeV.  

%Originally all the remaining SM fields were 
%assumed to also reside at the IR-brane~\cite{Davoudiasl:1999jd}.  
%However, the cutoff of the effective 4D theory is then also 
%red-shifted to the weak scale. 
%This in turn leads to unsuppressed higher dimensional operators 
%and thus large violations of the electroweak precision data and unacceptably 
%large FCNCs. 



The hierarchies among the quark masses can be realized by localizing the Higgs on the IR brane, while the fermions have different profiles in the $5$-th dimension. 
% problem can be solved by realizing that the points along the warped 
%$5^{\rm th}$ dimension correspond to different effective 4D cut-off scales. 
The first and second generation zero mode fermions are localized
close to the UV-brane and have small overlaps with the Higgs, giving small effective 4D SM Yukawa interactions, and thus small quark masses after electroweak symmetry breaking.
%the higher dimensional operators get suppressed 
%by effectively larger scales \cite{Gherghetta:2000qt}. 
%Note that this explains why first and second generation fermions are light: 
%the Yukawa interactions are small because of small overlap between
%IR localized Higgs and UV localized light fermion zero modes. 
The top quark, on the other hand, is localized near the TeV brane 
resulting in a large top Yukawa coupling. 


This configuration has a built in automatic suppression of FCNCs, which are suppressed by the 
same zero mode overlaps that gives the hierarchy of masses~\cite{Grossman:1999ra,Gherghetta:2000qt}. This is the RS analog of the SM Glashow-Iliopoulos-Maiani (GIM) mechanism, dubbed RS-GIM in  \cite{Agashe:2004cp,Agashe:2004ay}.
%
% substantially (however, see below) 
%and reproduces the fermion mass hierarchies 
%without invoking large disparities in the Yukawa couplings of the fundamental
%5D action. 
%It thus has a built-in analog of the SM Glashow-Iliopoulos-Maiani (GIM) mechanism 
%(the RS GIM), and reproduces the approximate flavor symmetry among the light fermions.
Similarly to the SM GIM, the RS GIM is violated by the large top quark mass. 
In particular, $(t,b)_L$ needs to be localized near the TeV brane, 
otherwise the 5D Yukawa coupling becomes too large and makes the theory
strongly coupled at the scale of the first Kaluza-Klein (KK) excitation. In general this 
leads to sizeable corrections to electroweak precision observables, such as the $Z b_L b_L$ couplings. 
Such problems can be largely ameliorated by enlarging the bulk symmetry such that it contains a custodial $SU(2)_L\times SU(2)_R$ symmetry \cite{Agashe:2003zs}, which for instance lowers the KK scale bounds from EW precision tests from 5 TeV to about 2TeV \cite{Agashe:2013kxa}. The consequences for flavour phenomenology have been worked out in a series of papers, see, e.g., \cite{Blanke:2008yr,Albrecht:2009xr,Casagrande:2010si,Casagrande:2008hr}, with $K-\bar K$ mixing for instance requiring the KK scale to be above 8 TeV \cite{Agashe:2013kxa}. With flavour alignment the scale of KK modes could be substantially lowered \cite{Csaki:2009wc} and could be reachable by HL/HE-LHC.

The KK gluon resonances cannot be produced from gluons \cite{Allanach:2009vz}, so that the LHC production is restricted to the quark-antiquark fusion, even though this is suppressed by the flavor dependent zero mode overlaps. This means that the LHC cross section for the first KK gluon resonances are small, suppressed also by the quark-anti-quark parton density functions (PDFs). The dominant decay mode is into $t\bar t$ final state, due to the large zero mode overlaps \cite{Agashe:2006hk}. Using the benchmark RS model from \cite{Lillie:2007yh}, the most recent CMS analysis for $t\bar t$ resonance searches, using both hadronic and leptonic tops, sets a bound of 4.55 TeV on the mass of the KK gluon  \cite{Sirunyan:2018ryr}. The projected reach for 33TeV and 100TeV $pp$ colliders can be found in \cite{Agashe:2013kxa}.

%mostly decay to top quarks because of the large overlaps 
%
%At leading order, a single KK gluon resonance cannot be produced from gluon initial states. The production at the
%        s = 14 TeV. The couplings of a KK gluon to quarks is also flavor-dependent and can be described by the same wavefunction overlap integrals
%LHC is therefore suppressed by the quark-anti-quark parton density functions (PDFs) at
%which appear in equation
%
%
%This has two consequences:
%(1) in the interaction basis, the coupling of $b_L$ to gauge KK modes
%(say the gluons),
%$g_{ G^{ \rm KK } }^b$, is
%large compared to the couplings of the lighter quarks.
%This is a source of flavor violation 
%leading to FCNCs. 
%(2) The Higgs vev mixes the zero mode of $Z$ and its KK modes, 
%leading to a non-universal shift 
%$\delta g_Z^{ b}  \sim  g_{Z^{\rm KK }}^{b}
%\sqrt{ \log \left( M_{\rm Pl} / \hbox{TeV} \right) }
%{ m_Z^2 }/{ m_{\rm KK }^2 }$
%in the coupling of
%$b_L$ to the physical $Z$~\cite{Agashe:2003zs,Burdman:2003ya}. Here $g_{Z^{\rm KK }}^{b}$ is the  
%coupling
%between $b_L$ and a KK $Z$ state before electroweak symmetry breaking.
%The factor $\sqrt{ \log \left( M_{\rm Pl} / \hbox{TeV} \right) }$ comes from
%enhanced Higgs coupling to gauge KK modes, 
%which are also localized near the TeV brane.
%Electroweak precision measurements of $Z\to b_L\bar b_L$ require that
%this shift is smaller than $\sim 1 \%$.
%Using $g^b_{Z^{\rm KK} } \sim g_Z$ this is satisfied for $m_{ KK } \sim 3 \ \hbox{TeV}$. 
%In passing we also note that with enhanced bulk electroweak gauge symmetry, 
%$SU(2)_L \times SU(2)_R \times U(1)_{B-L}$, and KK masses of $\approx$ 3 TeV,
%consistency with constraints from electroweak precision measurements 
%are achieved~\cite{Agashe:2003zs}.  
%
%The tension between obtaining a large top Yukawa coupling 
%and not introducing too large flavor violation and disagreement with EWP data~\cite{Agashe:2003zs,Burdman:2003ya} 
%is solved in all models by assuming
%(1) a close to maximal 5D Yukawa coupling, $\lambda_{\rm 5D} \sim 4$, 
%so that the weakly coupled effective theory contains $3$--$4$ KK modes, 
%and (2) by localizing $(t,b)_L$ as close
%to the TeV brane as allowed by  $\delta g_Z^{ b}\sim 1 \%$. 
%This almost unavoidable setup leads to sizeable NP contributions 
%in the following three types of FCNC processes that are top quark dominated:
%(i) $\Delta F = 2$ transitions, 
%(ii) $\Delta F = 1$ decays governed by box and EW penguin diagrams; 
%(iii) radiative decays. 
%
%Sizeable modifications of $\Delta F=2$ processes are possible 
%from tree-level KK gluon exchanges.
%The $\Delta F=1$ processes receive contributions from 
%tree level exchange of KK $Z$ modes. 
%These tend to give smaller effects than KK gluon exchanges. 
%Nevertheless it can lead to appreciable effects in the branching ratio,
%direct CP asymmetry and the spectrum of $b \to s \ell^+ \ell^-$~\cite{Burdman:2003ya,Agashe:2004ay,Agashe:2004cp}. 
%In $b \rightarrow s \bar{q} q$ QCD penguin dominated 
%$B\to (\phi, \eta^\prime, \pi^0, \omega, \rho^0) K_s$ decays 
%on the other hand the RS contributions from flavor-violating $Z$ vertex
%are at least $\sim g_Z^2 / g_s^2 \sim 20 \%$ suppressed and thus
%subleading~\cite{Agashe:2004ay,Agashe:2004cp}. 
%Consequently, RS models can accommodate only mild deviations
%from the SM in the corresponding time dependent CP asymmetries.
%
%We should emphasise that these models are not fully developed yet so 
%there can be appreciable uncertainties in the specific predictions. 
%For instance, the particular framework outlined above runs into at least 
%two problems unless the relevant KK-masses are much larger than $~3$ TeV:
%(i) the presence of right-handed couplings can cause enhanced contributions 
%to $\Delta S=2$ processes, $K$--$\bar K$ mixing and $\epsilon_K$~\cite{Beall:1981ze,Bona:2007vi}, 
%and (ii) the simple framework with ${\cal O}(1)$ complex phases tends to give 
%an electron electric dipole moment about a factor $\sim 20$ above the 
%experimental bound~\cite{Agashe:2004ay,Agashe:2004cp}. 
%An interesting proposal for the flavor dynamics in the
%RS setup was recently put forward by~\cite{Fitzpatrick:2007sa} who
%introduced 5D anarchic minimal flavor violation in the quark sector 
%(see also ~\cite{Cacciapaglia:2007fw}). 
%This gives a low energy effective theory that falls in the NMFV class, 
%consistent with both FCNC and dipole moment constraints 
%(see section~\ref{Sec:MFV}).
%In this picture new flavor and CP violation phases are present, 
%however, their dominant effect occurs only in the up type quark sector.  

\subsubsection{Partial compositeness}
Partial compositeness as the origin of the flavour hierarchies in composite Higgs models \cite{Kaplan:1991dc} is the holographic dual to the RS models of flavour. While the Higgs is the lightest state of the composite sector, usually a pseudo-Nambu Goldstone boson from global symmetry breaking, the SM fermions and gauge bosons are elementary (for a review, see, e.g, \cite{Panico:2015jxa}). The elementary fermions, $Q,U,D$, are coupled to the composite sector through linear mixing with the composite operators, ${\mathcal O}_Q$, ${\mathcal O}_U$, ${\mathcal O}_D$,
\begin{equation}
{\cal L}\supset \epsilon_Q \bar Q_L {\mathcal O}_Q +\epsilon_U \bar U_R{\mathcal O}_U+\epsilon_D \bar D_R {\mathcal O}_D.
\end{equation}
The mixing parameters $\epsilon_a$ exhibit exponentially large hierarchies because of large, yet still ${\mathcal O}(1)$, differences in anomalous dimensions of the corresponding composite operators. This is the analog of the zero mode overlaps in the RS models. The SM Yukawa are given by $(Y_{U(D)})_{ij}\sim \epsilon_{Q}^i \epsilon_{U(D)}^j$. For 
$\epsilon_Q^1\ll \epsilon_Q^2\ll \epsilon_Q^3\sim 1$, $\epsilon_U^1\ll \epsilon_U^2\ll \epsilon_U^3\sim 1$, $\epsilon_D^1\ll \epsilon_D^2\ll \epsilon_D^3\ll 1$ one can obtain the SM structure of quark masses and CKM mixings. 

The composite Higgs models are described by the compositeness scale $f$ and the mass of the first composite resonances, $M_*\sim g_* f$, with $g_*$ the typical strength of the resonances in the composite sector. The searches at the HL-/HE-LHC consist of Higgs coupling measurements, including deviations in Higgs Yukawa couplings (for further details see the discussion in the WG2 write-up \td{cite}),
%\td{cite WG2}, 
and searches for composite resonances preferably coupled to third generation fermions, electroweak gauge bosons, or the Higgs (further details can be found in WG3 write-up \td{cite}). Flavour observables put strong bounds on $M_*$, if the flavour structure is assumed to be generic. Such bounds can be relaxed in the case of approximate flavour symmetries, see, e.g., Ref. \cite{Panico:2015jxa} for a review.
%\td{cite WG3}.


%\subsubsection{U(2) models of flavour}
%\td{to be done}

\subsubsection{Low scale gauge flavour  symmetries}
The SM has in the limit of vanishing Yukawa couplings a large global symmetry. In the quark sector this is $G_F=SU(3)_Q\times SU(3)_U\times SU(3)_D$. Ref. \cite{Grinstein:2010ve} showed that the SM flavor symmetry group $G_F$ is anomaly free, if one adds a set of fermions that are vector-like under the SM gauged group, but chiral under the $G_F$. This means that $G_F$ can be gauged. It is broken by a set of scalar fields that have hiearchical vevs and lead to hierarchy of SM quark masses. This also implies a hierarchy  for the masses of the flavoured gauge bosons, with gauge bosons that more strongly couple to third generation being lighter, while the flavoured gauge bosons that couple more strongly to the first two generations are significantly heavier. This pattern in the spectrum of flavoured gauge bosons then avoids too large contributions to FCNCs \cite{Buras:2011wi}.  
%\td{to be done}

At the LHC one searches for the lightest flavoured gauge bosons, with ${\mathcal O}({\rm TeV})$ masses, which couple mostly to $b$ quarks and $t$ quarks, but could also have non-negligible couplings to the first two generations. The di-jet and $t\bar t$ resonance searches are thus sensitive probes. A signal could also come from production of the lightest vectorlike fermions, $t'$, or $b'$ \cite{Grinstein:2010ve}. For several further benchmarks see, e.g., Ref. \cite{Bishara:2015mha}, where also a connection with dark matter was explored.

\subsubsection{2HDM and low scale flavour models}
\label{sec:2HDM:flavour}
%taken from WG2, should cite them
{\it\small Author(s) (TH): Martin Bauer, Marcela Carena and Adri\'an Carmona.}\\
In 2 Higgs Doublet Models (2HDMs), the two Higgs doublets, $H_1$ and $H_2$ are usually assumed not to carry flavour quantum numbers. The collider phenomenology, on the other hand, changes substantially, if they do. 
% the term $H_1 H_2\equiv H_1^T (i\sigma_2) H_2 $ is a SM singlet which can however be charged under an additional $U(1)$ flavour symmetry. 
 This is an interesting possibility that could solve the SM flavour puzzle via the Froggatt-Nielsen (FN) mechanism where the flavon is replaced by the $H_1 H_2\equiv H_1^T (i\sigma_2) H_2$ operator. In this way, the NP scale $\Lambda$ where the higher dimensional FN operators are generated is tied to the electroweak scale, leading to much stronger phenomenological consequences. Let us assume for concreteness a type-I like 2DHM with the following Yukawa Lagrangian in the quark sector \cite{Bauer:2015kzy,Bauer:2015fxa}
\begin{equation}
\label{eq:yuk1}
\mathcal{L}_Y\supset  y^u_{ij} \left(\frac{H_1 H_2}{\Lambda^2}\right)^{n_{u_{ij}}}\bar{q}_L^{i}H_1 u_{R}^j+y^d_{ij} \left(\frac{H_1^{\dagger} H_2^{\dagger}}{\Lambda^2}\right)^{n_{d_{ij}}}\bar{q}_L^{i}\tilde{H}_1 d_{R}^j
%+y_{ij}^{\ell}\left(\frac{H_1^{\dagger}H_2^{\dagger}}{\Lambda^2}\right)^{n_{e_{ij}}}\bar{\ell}_{L}^i\tilde{H}_1 e_R^j
+\mathrm{h.c.}\,,
\end{equation}
where $\tilde{H}_1\equiv i\sigma_2 H^{\ast}_1$ as usual, and the charges $n_{u,d,e}$ are a combination of the $U(1)$ charges of $H_1$, $(H_1H_2)$ and the different SM fermion fields (for an alternative discussion, where $H_1, H_2$ carry flavour charges, but the Yukawa interactions are taken to be renormalizable, see \cite{Dery:2016fyj}). For simplicity, we set the flavour charges of $H_1$ and $H_2$ to $0$ and $1$, respectively, such that  
%\begin{align}
$n_{u_{ij}}=a_{q_i}-a_{u_j}, n_{d_{ij}}=-a_{q_i}+a_{d_j}$,
%\end{align}
if we denote by $a_{q_i},a_{u_i}, \ldots,$ the $U(1)$ charges of the SM fermions. In general, the fermion masses are given by
\begin{equation}\label{eq:epsilon}
m_\psi=y_{\psi} \varepsilon^{n_\psi} \frac{v}{\sqrt{2}},  \qquad \varepsilon = \frac{v_1 v_2}{2\Lambda^2}=\frac{t_\beta}{1+t_\beta^2}\frac{v^2}{2\Lambda^2},
\end{equation}
with the vacuum expectation values $\langle H_{1, 2}\rangle=v_{1, 2}$ and $t_\beta \equiv v_1/v_2$. For the right assignment of flavour charges one is able to accommodate the observed hierarchy of SM fermion masses and mixing angles.  This framework also leads to enhanced diagonal Yukawa couplings between the Higgs and the SM fermions, while FCNCs are suppressed. If we denote by $h$ and $H$ the two neutral scalar mass eigenstates, with $h$ the observed $125$ GeV Higgs, the couplings between the scalars $\varphi=h,H$ and SM fermions $\psi_{L_i, R_i}= P_{L,R} \psi_i$ in the mass eigenbasis read 
\begin{equation}\label{eq:newlagrangian}
\mathcal{L}= g_{\varphi \psi_{L_i} \psi_{R_j} }\, \varphi \,\bar \psi_{L_i} \psi_{R_j}+\mathrm{h.c.}
\end{equation}
with $i$, such that $u_i=u, c,t$,\,  $d_i=d,s,b$ and $e_i=e,\mu,\tau$. This induces 
flavour-diagonal couplings 
\begin{align}\label{eq:diagocoup}
g_{\varphi \psi_{L_i}\psi_{R_i}}= \kappa^\varphi_{\psi_i}\, \frac{m_{\psi_i}}{v} =\left(g^{\varphi}_{\psi_i}(\alpha,\beta)+n_{\psi_i}\, f^\varphi(\alpha, \beta)\right)\frac{m_{\psi_i}}{v},
\end{align}
as well as flavour off-diagonal couplings
 \begin{align}\label{eq:foffc}
g_{\varphi \psi_{L_i}\psi_{R_j}}&=  f^\varphi(\alpha, \beta)\left(\mathcal{A}_{ij}\frac{m_{\psi_j}}{v}-\frac{m_{\psi_i}}{v}\mathcal{B}_{ij} \right)\,.
\end{align}
The flavour universal functions in \eqref{eq:diagocoup} and \eqref{eq:foffc} are
$g^h_{\psi_i}={c_{\beta-\alpha}}/{t_\beta}+s_{\beta-\alpha}$, $ g^H_{\psi_i}=c_{\beta-\alpha}-{s_{\beta-\alpha}}/{t_\beta}$,
and
$ f^h(\alpha,\beta)=c_{\beta-\alpha}\big({1}/{t_\beta}-t_\beta\big)+2s_{\beta-\alpha}$, $f^H(\alpha,\beta)=-s_{\beta-\alpha}\big({1}/{t_\beta}-t_\beta\big)+2c_{\beta-\alpha}$,
where $c_{x}\equiv \cos x$, $s_x\equiv \sin x$. The entries in matrices $\mathcal{A}$ and $\mathcal{B}$ are proportional to the flavour charges of the corresponding fermions that define the coefficients in \eqref{eq:yuk1}. Unless all flavour charges for a given type of fermions are equal, the off-diagonal elements in matrices $\mathcal{A}$ and $\mathcal{B}$ lead to FCNCs which are chirally suppressed by powers of the ratio~$\varepsilon$, see \cite{Bauer:2017cov} for more details and explicit examples for scalings of matrix elements in $\mathcal{A}$ and $\mathcal{B}$.


\subsubsection{\bf A Clockwork solution to the flavour puzzle}
%\td{need to shorten}
{\it\small Author(s) (TH): Adri\'an Carmona.}\\
The clockwork mechanism, introduced in \cite{Choi:2015fiu, Kaplan:2015fuy} and later generalized to a broader context in Ref.~\cite{Giudice:2016yja}, allows one to obtain large hierarchies in couplings or mass scales. Ref. \cite{Alonso:2018bcg} showed that it can also be used to generate the observed hierarchy of quark masses and mixing angles with anarchic Yukawa couplings, providing a solution to the flavour puzzle. 

In the clockwork solution to the flavour puzzle, each SM chiral fermion $\psi$ is accompanied with a $N_\psi$-node chain of vector-like fermions, $\psi_{L,j},\psi_{R,j}$, with masses $m$, and
%In the simple case of clockworking just a right-handed chiral field $\psi_R\equiv \psi_{R,0}$, one adds a $N$-node chain of vector like fermions $\psi_{L,j},\psi_{R,j}$, with masses $m\bar{\psi}_{L,j}\psi_{R,j}$, as well as 
a series of nearest neighbour mass terms, $qm\bar{\psi}_{L,j}\psi_{R,j-1}$, between the nodes, where  $j=1,\ldots,N_\psi$. The mass spectrum of each chain has one chiral zero mode, the would-be SM fermion, 
%chains of fermions $\psi_{R,j}$ and $\psi_{L,j}$  carry the same gauge quantum numbers as $\psi_{R,0}$. The covariant derivatives are thus the same for fermions on all the nodes. After diagonalization, one obtains a chiral massless state, $\psi_{R,0}^{\prime}$, 
and $N_\psi$  heavy Dirac fermion mass-eigenstates -- the gears.
%, with masses $M_{k}^2=m^2(1+q^2-2q\cos(\frac{k\pi}{N_\psi+1}))$.
For $q\gg 1$,  the spectrum of the gears is compressed in a $2m$ band around $qm$, with $(M_{N_\psi}-M_1)\ll M_1$. The massless zero-mode interacts with the SM Higgs, which is on the $0$-th node, through a set of Yukawa interactions described by ${\mathcal O}(1)$ Yukawa matrices, $Y_{U,D}$. 
The component of the massless mode on the $0$-th node is, 
%$\psi_{R,0}^{\prime}$ has a component $f_{\psi}$ on the $0$-th node which, 
in the large $q$ limit, given by $1/q^{N_\psi}$. This suppression is the origin of the SM Yukawa hierarchies, 
%\td{check Higgs section notation}
%In order to explain the hierarchy of SM quark masses through the clockworking mechanism we introduce for each of the SM fermions, $\psi_i$, where $i=1,2,3$, is the generation index, an $N_{\psi_i}$-node chain of 
%vector-like fermions with the same quantum numbers. In addition, the SM Higgs resides on the $0$-th node, coupling the fermions on the $0$-th nde through Yukawa interactions.  After electroweak symmetry breaking the Yukawa interactions lead to a mass term for the zero modes, which  are identified with the SM fermions. To leading order in $v^2/M^2$ expansion, the SM Higgs Yukawa matrices are given by the products of zero mode overlaps with the $0$-th node, $f_\psi$,
\begin{align}
\label{eqa:YuSM}
\left(Y_u^{\rm SM}\right)_{ij}&\sim q^{-N_{Q(i)}}_{Q(i)}\left(Y_U\right)_{ij}q^{-N_{u(j)}}_{u(j)},  \quad \left(Y_d^{\rm SM}\right)_{ij} \sim q^{-N_{Q(i)}}_{Q(i)}\left(Y_D\right)_{ij}q^{-N_{d(j)}}_{d(j)}.
\end{align}
The hierarchy of quark masses can be then naturally obtained for anarchical $Y_U$ and $Y_D$ Yukawa matrices if $q^{-N_{Q(i)}}\ll q^{-N_{Q(j)}}$, $q^{-N_{u(i)}}_{u(i)}\ll q^{-N_{u(j)}}_{u(j)}$, $q^{-N_{d(i)}}_{d(i)}\ll q^{-N_{d(j)}}_{d(j)}$, for $i<j$ (in the benchmark below we take $q_i$ to be universal and equal to $q$).
% For the sake of concreteness,  we will only consider the clockwork universal-$q$ limit, which occurs  when all the clockwork factors are the same, $q\equiv q_{Q(i)}=q_{u(i)}=q_{d(i)}\sim {\mathcal O}({\rm few})$, while $N_{Q(1)}\gg N_{Q(2)} \gg N_{Q(3)}$   and similarly for up and down right-handed quarks.\footnote{In particular, we consider  $q=1/\lambda$, with $\lambda=|V_{us}|\simeq0.23$ as well as $N_{Q(1)}=3, N_{Q(2)}=2$, $N_{u(1)}=4$, $N_{u(2)}=1$, $N_{d(1)}=4$, $N_{d(2)}=3$, $N_{d(3)}=2$ and $N_{Q(3)}=N_{u(3)}=0$.} The large multiplicity of colored fermionic degrees of freedom  destroys asymptotic freedom for the strong interaction, putting an upper bound on the maximum number of  gears.  On the other hand, quark loops will also provide a negative contribution to the one-loop beta function of the Higgs quartic self-coupling $\lambda |H|^4/2$. Requiring that the beta function remains perturbative constraints the Yukawa entries in $Y_U$ and $Y_D$ to be smaller than 1,  with the possible exception of $\left(Y_U\right)_{33}$.

%Clockworking  a  left-handed chiral fermion, $\psi_L$, proceeds along exactly the same lines, but exchanging  $L\leftrightarrow R$ everywhere. 

%Unlike the zero mode, the profiles of the gears are not exponentially suppressed on the $0$-th node, even when $q,N\gg 1$. 

\begin{figure}
\begin{center}
\includegraphics[width=0.4\textwidth]{\main/section10/img/GearXsec_all_fn.pdf}
	\includegraphics[width=0.59\textwidth]{\main/section10/img/GearSpheres_all_fn.pdf}
\caption{\textit{Left:} The total gear pair production cross-sections in the final states $tW+X$, $tH+X$ and $tZ+X$ for the benchmark model from Ref. \cite{Alonso:2018bcg}, with contributions from individual gears shown stacked.
% and ordered top down by their increasing mass (decreasing cross-section). 
The currently most stringent upper bounds~\cite{Aaboud:2018uek,Aaboud:2018xuw} are denoted with dashed lines (for the $tZ+X$~\cite{Aaboud:2018xuw} final state the bound is too weak to be shown). 
%, obtained by recasting of searches for vector-like quarks in the $tW+X$~\cite{Aaboud:2018uek} and $tH+X$ (the 1-lepton channel)~\cite{Aaboud:2018xuw} final states, are denoted with dashed lines. The corresponding bound on the $tZ+X$~\cite{Aaboud:2018xuw} final state is much weaker and is not shown. 	
\textit{Right:} Invariant mass spectrum of individual pseudojets clustered using the hemisphere algorithm applied to partonic gear pair production and decay at the 13 TeV LHC. The original hemisphere clustering results are shown in light gray, and the modified hemisphere clustering results in mid-gray (dark gray if in addition 
%. Finally, the modified hemisphere clustering results, where in addition 
the masses of the two pseudojets are required to differ by less than 30\%)
%,  are shown in dark gray. 
}
	\label{fig:both}
\end{center}
\end{figure}


%If one integrates out all the new heavy particles, one can use the generated dimension-6 operators of the SM effective field theory  to analyze the constraints from low-energy experiments: from weak boson decays, rare meson decays and neutral meson mixings. 
The clockwork models of flavour are endowed with a powerful flavour protection against FCNCs, very similar to the RS models.  The FCNCs with light quarks on the external legs are suppressed by the same small overlaps of the zero-modes, giving rise to hierarchies between the SM quark masses. This clockwork-GIM  mechanism, along with the constraints on $Y_{U,D}$ arising from the stability of the Higgs potential,   suffices to alleviate the flavour constraints to the level that TeV scale gear masses are compatible with experimental bounds~\cite{Alonso:2018bcg}.

TeV scale gears
%, as allowed by the present low energy constraints, 
can be searched for at the LHC, where they are produced through QCD pair production.
% and at future high energy colliders. 
%The main production channel for the gears is the QCD pair production with the corresponding cross sections precisely calculable~\cite{Czakon:2017wor}. 
The collider signatures
%, on the other hand, do 
depend on the gear decay patterns. The gears decay predominantly through their coupling to the Higgs doublet into gears from a different-chirality chain.
The lightest gears decay directly to SM fermions, mostly $t$ and $b$, via the emission of $W,Z$ or $h$, as do heavier gears for which these are the only kinematically allowed channels. 
%Given the overlap suppression the decays are predominantly to $t$ and $b$. 
The main existing collider constraints 
%on clockwork flavour models are  expected to arise 
are from searches for pair production of vector-like quarks, in final states involving third generation SM quarks. Ref.~\cite{Alonso:2018bcg} found the two
%In particular, we find the searches for down-like gears decaying to the $tW$ channel, as well as searches for up-like gears decaying to the $tH$ and $tZ$ final states, to be most sensitive.  To perform a more detailed analysis, we recast the recent 
35 fb${}^{-1}$ 13 TeV ATLAS 
searches for vector-like quarks decaying into 
$tW$ final states~\cite{Aaboud:2018uek}, as well as the analogous search employing the 
$tZ$ and $tH$ final states~\cite{Aaboud:2018xuw}, to be currently most sensitive, see 
%, both using 35 fb${}^{-1}$ of LHC data at 13~TeV for the two benchmarks defined in Ref.~\cite{Alonso:2018bcg}. The results for the last of these benchmarks are shown in the left panel of 
Fig.~\ref{fig:both}.





The dense spectrum of gears and the potentially complex pattern of gear decays poses an experimental challenge.
% also in the case a signal is discovered. 
In the conventional vector-like quark searches the clockwork signal will appear as an excess of events with high transverse energies or $H_T$, but without a dominant single peak in the invariant mass of any particular final state, such as $tH$ or $tW$. Ref.~\cite{Alonso:2018bcg} proposed a novel reconstruction strategy targeting pair production of heavy quarks with a-priori unknown but potentially long decay chains that result in a single heavy flavoured quark, $t$ or $b$, plus any number of massive weak or Higgs bosons per decay chain. The proposed search strategy uses a modified hemisphere clustering algorithm, with $t-$ and $b-$tagged jets as seeds for clustering into exactly two pseudo-jets (the invariant mass of these is shown as mid-gray distribution in Fig.~\ref{fig:both} right). The original hemisphere clustering uses instead the jets with highest invariant mass as seeds and shows no sharp features (light gray). Requiring that the masses of the two pseudo-jets differ by less then 30\% gives the dark grey distribution, with clearly visible gears (in the exploratory study of~\cite{Alonso:2018bcg} tops, $b$-quarks, $W$, $Z$ and the Higgs were not decayed).

%Our procedure is based on the so-called hemisphere clustering algorithm, defined in Section 13.4 of Ref.~\cite{Ball:2007zza}. Jets, as well as isolated leptons and photons, are clustered into exactly two pseudojets, where the clustering is performed by minimizing the Lund distance measure~\cite{Sjostrand:1982am}. The original hemisphere algorithm is seeded by the two objects with the largest combined invariant mass. Since each gear decay chain results in exactly one heavy flavoured quark ($t$ or $b$) we instead seed our algorithm with $t$- and $b$-tagged jets. The idea is that, at least for the moderately boosted pair produced gears, the two pseudojets will predominatly capture the decay products of the individual gears. Finally, we select events, where the invariant masses of the two pseudojets are comparable. 
%
%We demonstrate the usefulness of our procedure by simulating gear production and decay for the benchmark model at parton level (where tops, $b$-quarks, $W$, $Z$ and the Higgs are not decayed). The results are shown in the right panel of Fig.~\ref{fig:both}, where we plot the invariant mass distributions of individual pseudojets and overlay the spectral lines of the gears in the benchmark model. The resulting spectrum with the original hemisphere clustering does not exhibit any sharp features, with the bulk of the invariant mass distribution lying well below the mass of the lightest gear. The invariant mass distribution for the modified hemisphere algorithm already exhibits clear spectral line features. The pseudojets with masses of the top and $b$-quarks are abundantly identified, but also those of a few lowest lying gears. Finally, when one only keeps the events for which the masses of the two pseudojets differ by less than 30\%, the low invariant mass peaks corresponding to the pseudojets containing only a single top or $b$-quark  are rejected by this requirement. In addition, the gear peaks are even more pronounced after this cut, with little loss in the number of signal events in the peaks. 




\subsection{Flavour implications for high $p_T$ new physics searches} 
%{\bf Author(s): Martin Bauer, Adrian Carmona} (anticipated length 2 pages)
%\td{a few sentences to tie over}
In this subsection we collect several signatures of flavour models or models where nontrivial flavour structure is relevant for high $p_T$ searches: the FCNC top decays to exotica, the (model dependent) implications for di-Higgs production, and the set of signatures that are related to neutrino mass models. 

\subsubsection{Top decays to exotica} 
{\it\small Author(s) (TH): S. Banerjee, M. Chala, M. Spannowsky} 
%(4 pages) \td{JZ: should be less, probably}
% S. Banerjee, M. Chala, M. Spannowsky



The FCNC mediated processes are rare within the SM. However, LHC is a top factory and significant number of events are expected even for top decays with very small branching ratios. In light of this, studies of top FCNC decays to SM particles have garnered a strong interest in the community~\cite{Agashe:2009di,Mele:1998ag,Greljo:2014dka,Azatov:2014lha,Khachatryan:2015att,Botella:2015hoa,Bardhan:2016txk,Badziak:2017wxn,Khachatryan:2016sib,Gabrielli:2016cut,CMS-PAS-TOP-17-017,Aaboud:2018nyl,Aaboud:2017mfd,Aaboud:2018pob,Papaefstathiou:2017xuv,Abbas:2015cua}. The FCNC top decays to SM particles, $t\to q Z, q \gamma, qg, q H$, $q=u,c$, and the related constraints from FCNC production processes, were discussed in Sections \ref{sec:top:FCNC} and \ref{sec:8:EXP:FCNC}.

%\td{need to say something more about classic $t\to c Z$, etc signals, i.e., reference to section \ref{sec:top:quark}, once that is finished}

In the presence of light NP, other exotic top FCNC decays are possible. We highlight one such possibility, where the NP spectrum contains a light 
%extend previous works by studying the top decay
%to a 
scalar singlet, $S$, with mass $m_S$ below the top quark mass.
%around the Electroweak (EW) scale and an up-type light quark.
Such scalar
particle is predicted in a number of well-motivated NP models, e.g., in the NMSSM~\cite{Ellwanger:2009dp} and in
non-minimal composite Higgs models~\cite{Dimopoulos:1981xc,Kaplan:1983fs,Kaplan:1983sm,Panico:2015jxa}. Moreover, quite often the induced $t\to cS, uS$ FCNC decays are easier to probe than, for instance the ones involving the SM Higgs, $t\to ch, uh$~\cite{Zhang:2013xya}. 
%such a scalar can usually induce FCNCs to a degree significantly larger than what can be mediated by an SM-like Higgs 
%boson
The reason is three-fold; \textit{(i)} The top FCNCs mediated by 
$S$ are usually suppressed by one less power of the heavy physics scale; \textit{(ii)} $S$ may have a larger decay width into cleaner final states, such as $\ell^+\ell^-$, $b\overline{b}$ 
or $\gamma\gamma$; \textit{(iii)} $S$ can be much lighter than the Higgs, reducing the 
%corresponding top decay
%being therefore 
phase space suppression. %Thus, top FCNCs mediated by new singlet scalars might be seen at the 
%future runs of the LHC.
Note that very light $S$, i.e., with $m_S < m_h/2\sim 62.5$ GeV, need not be
excluded by the LHC constraints on the Higgs width, $\Gamma(h\rightarrow S S) \lesssim 10$ MeV ~\cite{Khachatryan:2016ctc}. Indeed, 
for a quartic coupling $\lambda_{HS} S^2 |H|^2$, 
%we have
%$\Gamma(h\rightarrow S S)\sim \lambda_{HS}^2/(32\pi) \times v^2/m_h$, so we can evade the aforementioned 
this bound is avoided for $\lambda_{HS} < 0.05$.

There are no direct experimental limits on $t \to q S$ from colliders. The indirect constraints from 1-loop box diagrams in 
$D^0-\bar{D}^0$ oscillations constrain the products of two $S$ Yukawas, $\tilde Y_{ut} \tilde Y_{ct(tc)}$, and $\tilde Y_{tu} \tilde Y_{ct(tc)}$, to be small~\cite{Bona:2007vi,Harnik:2012pb,Agashe:2013hma}. The $S\bar t c$ or $S\bar t u$ couplings can still be sizeable, but not both at the same time. Inspired by the CMS $t\to hc$ search~\cite{CMS:2017cck}, Ref.~\cite{Banerjee:2018fsx} developed a dedicated analysis for $t\to qS$, by varying the mass of $S$, $m_S$. The projections at the 14 TeV LHC for  $S \to b\bar{b}$ and $S \to \gamma \gamma$ decays  are shown in Figs. \ref{fig:sbb2} .
%and \ref{fig:sgg2}, respectively.  
%However, one can have constraints from flavour 
%observables, \textit{viz.}, $D^0-\bar{D}^0$ oscillations~\cite{Bona:2007vi,Harnik:2012pb,Agashe:2013hma}. These
%constraints are always in the form of a product of two $S$ Yukawas, $Y_{ct}$ and $Y_{ut}$ (and also $Y_{uc}$). One 
%can always fall back upon scenarios where $Y_{ut}$ is negligibly small, rendering these constraints unimportant.
%%
%Therefore, inspired by the $t\to hc$ search by CMS~\cite{CMS:2017cck}, we develop dedicated analyses for $t\to qS$,
%by varying $m_S$.
%We first discuss the model-independent framework and then discuss the analyses and results in two different 
%final states, \textit{viz.}, $S \to b\bar{b}$ and $S \to \gamma \gamma$ at the 14 TeV LHC. Finally, we quote naive 
%estimations at $\sqrt{s} = 27$ TeV and $\sqrt{s} = 100$ TeV. \\ \\
%
%{\bf Effective Lagrangian} \\
Assuming that $S$ is the only light NP state, its couplings to the SM quarks are induced by dimension 5 effective operator (not displaying generational indices)
\begin{equation}\label{eq:lag}
 \mathcal{L} = -{\bar{q}_L}{\tilde{Y}}\frac{S}{f}  \tilde{H}{u_R} + \text{h.c.}\supset \tilde{g} \frac{m_t}{f} \bar{t}_L S c_R + \text{h.c.},
\end{equation}
where $f$ is the NP scale, and on the r.h.s. we introduced a new flavour violating coupling $\tilde g$. The three Benchmark Points (BP) shown in Figs. \ref{fig:sbb2} 
%and \ref{fig:sgg2} 
are
%, each including $m_S = 20, 50, 80, 100, 120$ and $150$ GeV, are as follows
%
\begin{equation}
\label{eq:benchmarks:tqS}
\begin{split}
 \text{BP1(2,3)}:~~\tilde{g} = 1.0(1.0,0.1),~f=2(10,2)~\text{TeV}~~\Longrightarrow~~\BR(t\rightarrow Sc)\sim 10^{-3(4,5)}-10^{-2(3,4)}.
% \\
% \text{BP 2}: \quad \tilde{g} = 1.0~, \quad f = 10~\text{TeV}\quad\Longrightarrow\quad\mathcal{B}(t\rightarrow Sc)\sim 10^{-4}-10^{-3},\\
% \text{BP 3}: \quad \tilde{g} = 0.1~, \quad f =~\, 2~\text{TeV}\quad\Longrightarrow\quad\mathcal{B}(t\rightarrow Sc)\sim 10^{-5}-10^{-4}. 
\end{split}
\end{equation}
%
%We study a scenario where the SM Higgs sector is extended by a gauge singlet, $S$, with mass $m_S$. 
%At low energies, the relevant Yukawa Lagrangian can be written as~
%
%\begin{equation}\label{eq:lag}
% \mathcal{L} = -\mathbf{\overline{q_L}}\bigg(\mathbf{Y} + \mathbf{Y'}\frac{|H|^2}{f^2} + \mathbf{\tilde{Y}}\frac{S}{f}\bigg)  \tilde{H} \mathbf{u_R} + \text{h.c.}~,
%\end{equation}
%%
%where $H = [\phi^+, (h + \phi^0)/\sqrt{2}]^t$, is the SM-like Higgs doublet, $\mathbf{q_L}$ ($\mathbf{u_R}$) denotes 
%the left-handed (right-handed) quarks, $\mathbf{Y}, \mathbf{Y'}, \; \textrm{and} \; \mathbf{\tilde{Y}}$ are arbitrary 
%flavour matrices, $v\sim 246$ GeV, is the Higgs vacuum expectation value (\textit{vev}) and $f\gtrsim 
%\mathcal{O}(\textrm{TeV})$ is the new physics scale. Normally, the flavour matrices are not aligned, and hence the
%FCNCs can arise in the EW phase. This induces various new physics effects including the top flavour-violating decays, 
%\textit{viz.}, $t\rightarrow h c$ or $t\rightarrow S c$. As discussed before, the latter dominates. (In CHMs, Higgs mediated
%top FCNCs are even absent in first approximation~\cite{Agashe:2009di}, \textit{i.e.} $\mathbf{Y}\sim\mathbf{Y^\prime}$, whereas this does not need to be the case for $S$~\cite{Banerjee:2018fsx}).
%
%After the EWSB, one obtains
%%
%\begin{align}
%\label{lag:lag1}
% \mathcal{L} &= -\frac{v}{\sqrt{2}}\bigg[\mathbf{\overline{q_L}}\mathbf{Y}\left(1+\frac{h}{v}\right)\mathbf{u_R} + \frac{S}{f}\mathbf{\overline{q_L}}\mathbf{\tilde{Y}}\mathbf{u_R} + \mathcal{O}\left(\frac{1}{f^2}\right)\bigg]\supset \tilde{g} \frac{m_t}{f} \overline{t_L} S c_R + \text{h.c.},
%\end{align}
%%
%where $m_t\sim 173$ GeV is the top mass and $g$ is an $\mathcal{O}(1)$ coupling. The partial width of $t \to Sc$ 
%is
%%
%\begin{equation}
% \Gamma(t\rightarrow Sc) = \frac{\tilde{g}^2}{32\pi}\frac{v^2}{f^2}m_t\bigg(1-\frac{m_S^2}{m_t^2}\bigg)^2.
%\end{equation}
%
If the flavour conserving couplings of $S$ to the SM fermions, $\psi$, are ${c_{\psi} m_\psi}
S(\bar{\psi}\psi)/f$, and to the photons ${c_{\gamma}\alpha} S F_{\mu\nu}\tilde{F}^{\mu\nu}/({4\pi f})$, then
$\mathcal{B}(S\rightarrow\gamma\gamma)/\mathcal{B}(S\rightarrow \overline{\psi}\psi) \sim 
({\alpha}/{\pi})^2 (m_S/m_\psi)^2$, 
taking $c_{\psi}\sim c_{\gamma}\sim \mathcal{O}(1)$.
% couplings and $\alpha$ the fine-structure 
%constant. Thus, at leading order,
%%
%\begin{equation}
% \Gamma(S\rightarrow\psi\psi) = \frac{N_c}{8\pi}\frac{c_{\psi}^2 m_\psi^2}{f^2}m_S~ \; \quad \textrm{and}~ \; \quad \Gamma(S\rightarrow\gamma\gamma) = \frac{c_\gamma^2 \alpha^2}{64 \pi^3 f^2}m_S^3~.
%\end{equation}
%%
%One finds the ratio 
 The suppression of $S\to \gamma\gamma$ can be partially 
compensated by scaling with $m_S$, so that $\mathcal{B}(S\to \gamma\gamma)$ can possibly be 
significantly larger than $\mathcal{B}(h\to \gamma\gamma)$. Searches should thus use both $S\to b\bar b$ and $S\to \gamma\gamma$. The details on how to reduce the backgrounds can be found in~\cite{Banerjee:2018fsx}.
%\\ \\



%%The Lagrangian in 
%
%%{\bf LHC prospects for $t\rightarrow Sc, S\rightarrow b\overline{b}$} \\
%
%The reach for $t\to c S(\to b\bar b)$ was obtained in~\cite{Banerjee:2018fsx} by requiring 
%%Here, we focus on the scenario where the scalar singlet decays to a pair of $b$-quarks, yielding a final state 
%%comprised of 
%at least four jets, three of them $b$-tagged, and exactly one isolated lepton. 
%%Our goal here is to derive an upper bound on $\mathcal{B}(t \to S c, S \to b \bar{b})$ at 95~\% Confidence Level 
%%(CL). We use fixed $b$-tagging efficiency of 70\%. The $c \; (\textrm{light jet}) \to b$ mistag rate is 10\% (1\%). 
%The dominant real background is the semi-leptonic $t\bar{t} b \bar{b}$ production. 
%%Besides, we also have the fully leptonic 
%%$t\bar{t}b\bar{b}$. 
%The major fake backgrounds are the semi-leptonic (and leptonic) $t\bar{t}$, $Wb\bar{b}$ and 
%$Zb\bar{b}$. 
%%These are all merged (MLM merging~\cite{Mangano:2006rw}) with up to extra partons.
%These are reduced by making cuts on reconstructed top mass, $S$ mass, and on $m_T$, see~\cite{Banerjee:2018fsx} for details. 
%%
%%%Here, we focus on the scenario where the scalar singlet decays to a pair of $b$-quarks, yielding a final state 
%%%comprised of at least four jets, three of them required to be $b$-tagged, and exactly one isolated lepton. 
%%%Our goal here is to derive an upper bound on $\mathcal{B}(t \to S c, S \to b \bar{b})$ at 95~\% Confidence Level 
%%%(CL). We use fixed $b$-tagging efficiency of 70\%. The $c \; (\textrm{light jet}) \to b$ mistag rate is 10\% (1\%). The most 
%%%dominant real background is the semi-leptonic $t\bar{t} b \bar{b}$ production. Besides, we also have the fully leptonic 
%%%$t\bar{t}b\bar{b}$. The major fake backgrounds are the semi-leptonic (and leptonic) $t\bar{t}$, $Wb\bar{b}$ and 
%%%$Zb\bar{b}$. These are all merged (MLM merging~\cite{Mangano:2006rw}) with up to extra partons.
%%
%%The details of the
%%event selection can be found in the original paper~\cite{Banerjee:2018fsx}. Once the events are selected, we look for the 
%%closest pair (in $\Delta R$) of $b$-tagged jets and reconstruct the top mass, $m_t^{\Delta R}$ with the additional
%%hardest jet, not $b$-tagged. We demand that this variable lies within 50 GeV from $m_t$. With the remaining $b$-jet,
%%we construct the transverse mass variable, $m_T$ and demand $m_T < 200$ GeV. 
%%We finally impose a benchmark-dependent cut, 
%%$0.8\, m_S < m_{S}^{\Delta R} < m_S + 10$ GeV. Fig.~\ref{fig:sbb2} shows our results. 
%The left panel in Fig.~\ref{fig:sbb2} shows  the $3$ ab${}^{-1}$ 
%%\td{correct?}
%95~\% CL upper limit on $\mathcal B(t \to S c, S \to b \bar{b})$. The right panel shows the minimum integrated luminosity 
%required to test a branching ratio of $10^{-4}$ at 95~\% CL. The highest reach is attained for $m_S \sim 80$ GeV, for which one can probe
%$\mathcal{B}(t\rightarrow Sc, S\rightarrow b\overline{b}) > 10^{-4}$ at $95$ \% CL with $\mathcal{L} = 3$ ab$^{-1}$ of integrated luminosity. At lower masses, this reach is smaller as it is harder to resolve two $b$-jets, but a boosted analysis with fat jets may be useful. 
%%For larger masses, 
%%the sensitivity gets also reduced because $m_t^{\Delta R}$ does not always peak around $m_t$~\cite{Banerjee:2018fsx}. \\ \\
%%
%
%%{\bf LHC prospects for $t\rightarrow Sc, S\rightarrow \gamma\gamma$} \\

\begin{figure}[t]
\includegraphics[width=0.45\textwidth]{\main/section10/img/bbChannel.pdf}~~~
\includegraphics[width=0.45\textwidth]{\main/section10/img/ggChannel.pdf}
 %\includegraphics[width=0.45\textwidth]{\main/section10/img/LbbChannel.pdf}
 %
 \caption{\label{fig:sbb2} \textit{Left:} Branching ratios that can be tested in the $b\overline{b}$ (left) and $\gamma\gamma$ (right) channels at 14 TeV HL-LHC with $3000$ fb$^{-1}$ (95\% CL upper limit is denoted by red dashed line). The three benchmark points, Eq. \eqref{eq:benchmarks:tqS}, are denoted with black lines. }
\end{figure}
%
%
%

%For $t\to cS(\to \gamma)$ the analysis 
%%decays to a pair of photons. We consider a final state comprising 
%required at least two 
%jets (one $b$-tagged), one isolated lepton and two isolated photons, with a 
%% The reconstructions are similar to the previous
%%case. Because of a much sharper diphoton mass resolution, we require a 
%3 GeV mass window about the reconstructed scalar mass. 
%%The 
%%other cuts remain as in the previous case. 
%We show the 95\% CL upper limit on the branching ratio $\mathcal{B}(t 
%\to S c, S \to \gamma \gamma)$ in Fig.~\ref{fig:sgg2}, along with the minimum integrated luminosity necessary to 
%probe a branching ratio of $10^{-6}$. The effect of the 125 GeV Higgs can be seen where the background is much 
%larger.
%We find that,
%with the same integrated luminosity, $\mathcal{B}(t\rightarrow Sc, S\rightarrow \gamma\gamma) > 10^{-7}$ can be 
%tested at the 95 \% C.L. Consequently, if $\mathcal{B}(S\rightarrow\gamma\gamma) \sim 1\%$, we can indirectly probe NP scales as large as $\sim 50$ TeV. In this particular channel, the major background is 
%$t\overline{t}\gamma\gamma$ for $m_S$, well separated from $m_h$. 
 
%%
%%%%%%%%%%%%%%%%%%%%%%%%%%%%%%%%
%\begin{figure}[t]
% \includegraphics[width=0.45\textwidth]{\main/section10/img/ggChannel.pdf}
% \includegraphics[width=0.45\textwidth]{\main/section10/img/LggChannel.pdf}
% %
% \caption{\label{fig:sgg2} \textit{Left:} Branching ratios that can be tested in the $\gamma\gamma$ channel. Superimposed are the theoretical expectations in the three BPs. \textit{Right:} Luminosity required to test $\mathcal{B}(t\rightarrow St, S\rightarrow \gamma\gamma) = 10^{-6}$. Superimposed are $\mathcal{L} = 300$ fb$^{-1}$ and $3000$ fb$^{-1}$.}
%\end{figure}

For projections at future colliders we find that the increase in cross-section 
for the background at $\sqrt{s} = 27$ TeV ($100$ TeV) when compared to $\sqrt{s} = 14$ TeV is similar to 
that for the signal, and is $\sim 4(40)$. Assuming an integrated luminosity of $10$ ab$^{-1}$, 
we expect an increase in significance by a factor of $\sim 3.7$ ($\sim 11.5$). Similar results hold for the 
$b\overline{b}$ channel.

\subsubsection{Implications for di-Higgs production}
%taken from WG2, should cite them
{\it\small Author(s) (TH): Martin Bauer, Marcela Carena and Adri\'an Carmona}\\
Interestingly, in some models the flavour structure can feed back into nontrivial constraints on the scalar potential. This was demonstrated in the 2HDM model with FN charges, introduced in Section \ref{sec:2HDM:flavour}.
The scalar couplings to gauge bosons are the same as in the normal type-I 2HDM while the scalar coupling between the heavy Higgs $H$ and two SM Higgs scalars $h$, as well as the triple Higgs coupling can be expressed as \cite{Boudjema:2001ii,Gunion:2002zf}
\begin{align}
\label{eq:mainn1}
g_{Hhh}&=\frac{c_{\beta-\alpha}}{v}\!\left[\big(1\!-\!f^h(\alpha,\beta)s_{\beta-\alpha}\big)\big(3M_A^2\!-\!2m_h^2\!-\!M_H^2\big)\!-\!M_A^2\right],\\
g_{hhh}&= -\frac{3}{v}\!\left[f^h(\alpha,\beta)c_{\beta-\alpha}^2(m_h^2-M_A^2)+m_h^2s_{\beta-\alpha}\right],\label{eq:mainn2}
\end{align}
where $M_A$ is the pseudoscalar mass. The $U(1)$ flavour symmetry restricts the number of allowed terms in the scalar potential forbidding, e.g., terms proportional to $H_1 H_2$. Interestingly, one can rewrite such self scalar interactions with the help of function $f^h(\alpha,\beta)$, since it is related to the combination $H_1 H_2^{\dagger}$ appearing in both the scalar potential and the higher dimensional operators generating different Yukawa couplings. Therefore, the parameter space for which $f^h(\alpha,\beta)\gg 1$ and  $c_{\beta-\alpha}\neq 0$ leads to maximally enhanced diagonal couplings of the SM Higgs to fermions \eqref{eq:diagocoup} as well as to enhanced trilinear couplings \eqref{eq:mainn1} and \eqref{eq:mainn2}. For maximally enhanced Yukawa couplings, the mass of the heavy Higgs $H$ cannot be taken arbitrarily large and resonant Higgs pair production has to be present. This correlation between the enhancement of the Higgs Yukawa couplings $\kappa^h_{\psi}$ and $\text{Br}(H \to hh)$ is illustrated for $M_H=M_A=M_{H^\pm}=500$ GeV in Fig. \ref{fig:BRvskappa} (left) where we plot the dependence of $\text{Br}(H \to hh)$ on $c_{\beta-\alpha} $ and $ t_\beta$ \cite{Bauer:2017cov}.  The dashed contours correspond to constant values of $|\kappa_{\psi}^h|$ for $n_{\psi}=1$. The correlation does not depend on the factor $n_\psi$, although  $n_\psi > 1$ leads to a larger enhancement. The two exceptions for which this correlation breaks down are the limits $c_{\beta-\alpha}\approx 0$ (disfavored in the flavour model) and $c_{\beta-\alpha}\approx \pm 1$ (disfavoured by SM Higgs couplings strength measurements). Depending on the structure of the Yukawa couplings, the value of $\kappa^h_\psi$ in Figure~\ref{fig:BRvskappa} (left) can be larger or smaller than the value $n=1$ chosen to illustrate the relation between $g_{Hhh}$ and $\kappa^h_\psi$. Current experimental limits constrain this structure. For example, since $\kappa^h_\mu < 2.1$ \cite{ATLAS:2018kbw}, either $n=0$ for the muon, or one is constrained to the $\kappa^h_\mu < 2.1$ parameter space in Figure~\ref{fig:BRvskappa} (left). 
%, the first one (which corresponds to the decoupling limit) is at odds with the flavour model, for it requires large values of the spurion $\mu_3\propto M_A$ which softly breaks the $U(1)$ flavour symmetry. 




%
 \begin{figure}
 \begin{center}
 \includegraphics[width=.52\textwidth]{\main/section10/img/BRvskap.pdf}~~~~~
 \includegraphics[width=.44\textwidth]{\main/section10/img/diff.pdf}
 \caption{\label{fig:BRvskappa} Left: $\text{Br}(H \to hh)$ as a function of $\cos(\beta-\alpha)$ and $ \tan\beta$ for $M_H=M_{H^\pm}=550$ GeV and $M_A=450$ GeV. The dashed contours correspond to constant $|\kappa_{\psi}^h|$ (we set $n_{\psi}=1$). Right: Invariant mass distribution for the different contributions to the $pp\to hh$ signal with $c_{\beta-\alpha}=-0.45$ and $\kappa^h_{\psi}=5$ (blue),  $\kappa_{\psi}^h=4$ (green) and $\kappa_{\psi}^h=3$ (red) at $\sqrt{s}=27 $ TeV, respectively.}
 \end{center}
  \vspace{-.6cm}
 \end{figure}
%
%
%\begin{figure*}
%%\includegraphics[width=.465\textwidth]{\main/section10/img/xsec_3.pdf}
%\includegraphics[width=.49\textwidth]{\main/section10/img/diff.pdf}
%	\caption{\label{fig:xsecc} Left: Cross section for Higgs pair production in units of the SM prediction as a function of $\kappa_{\psi}^h$ for $c_{\beta-\alpha}=-0.45~(-0.4)$ and $M_A=450$ GeV , $M_H=M_{H^\pm}=550$ GeV in blue (green) at $\sqrt{s}=27 $ TeV. Right: Invariant mass distribution for the different contributions to the signal with $c_{\beta-\alpha}=-0.45$ and $\kappa^h_{\psi}=5$ (blue),  $\kappa_{\psi}^h=4$ (green) and $\kappa_{\psi}^h=3$ (red) at $\sqrt{s}=27 $ TeV, respectively.} 
%\end{figure*}
%

%The enhancement in $\text{Br}(H\to hh)$ shown in Figure~\ref{fig:BRvskappa} is partially cancelled in the production cross section $\sigma(gg\to H)$ for large values of $t_{\beta}$ due to the fact that $\sigma(gg\to H)\propto 1+1/t_\beta^2-(\kappa_t^h)^2$, with $\kappa_t^h \approx 1$. However, the cross-section $\sigma(gg\to h\to hh)$ is not suppressed for such values of $t_{\beta}$ and the combination of both contributions leads to a continuous enhancement in the di-Higgs cross-section. 
There is  a non-trivial interplay between resonant and non-resonant contributions to $pp\to hh$, as shown in Fig. \ref{fig:BRvskappa} (right), for $\sqrt{s}=27$ TeV, setting $M_A=450$ GeV and $M_{H}=M_{H^{\pm}}=550$ GeV,  $c_{\beta-\alpha}=-0.45$ and three different values of $\kappa_{\psi}^{h}=3, 4$ and $5$. When the enhancement in the Higgs Yukawa couplings is large enough, the interference between non-resonant and resonant contributions turns the broad peak into a shoulder in the $d\sigma/ dm_{hh}$ distribution for the total cross section, as shown for the case $\kappa_{\psi}^h=5$ by the blue line.
% Resolving such shape in the  invariant mass distribution can be quite challenging. We encourage a dedicated analysis considering the corresponding $d\sigma/dm_{hh}$ templates to maximize the sensitivity to features in the di-Higgs invariant mass distribution from the simultaneous enhancement of  $g_{hhh}$, $g_{Hhh}$ and $\kappa^h_{\psi}$.

\subsubsection{Neutrino Mass Models at the HL/HE LHC}
%\td{needs to be shortened}
%{\bf Author(s): Cedric Weiland?}(anticipated length 2 pages)
%\td{missing}

%subsection{\theoryC{Neutrino mass models at the High Luminosity and High Energy LHC}}
{\it\small Author(s) (TH): T. Han, T. Li, X. Marcano, S. Pascoli, R. Ruiz, C. Weiland}
%need to cite WG3
%{\bf Author(s): 
%T. Han$^a$ (than@pitt.edu), 
%T. Li$^{b,c}$ (tong.li@monash.edu), 
%X. Marcano$^d$ (xabier.marcano@th.u-psud.fr), 
%S. Pascoli$^e$ (silvia.pascoli@durham.ac.uk), 
%R. Ruiz$^e$ (richard.ruiz@durham.ac.uk), 
%C. Weiland$^e$ (cedric.weiland@durham.ac.uk)cew64@pitt.edu
%}
%
%\begin{flushleft}
%{\small
%$^a$Department of Physics and Astronomy, University of Pittsburgh, Pittsburgh, PA 15260, USA\\
%$^b$School of Physics, Nankai University, Tianjin 300071, China\\
%$^c$ARC Centre of Excellence for Particle Physics at the Terascale,
%School of Physics and Astronomy,\\ Monash University, Melbourne, Victoria 3800, Australia\\
%$^d$Laboratoire de Physique Th\'eorique, CNRS, \\
%Univ. Paris-Sud, Universit\'e Paris-Saclay, 91405 Orsay, France\\
%$^e$Institute for Particle Physics Phenomenology {\rm(IPPP)},\\
%Department of Physics, Durham University, Durham, DH1 3LE, UK
%}
%\end{flushleft}


%%%%%%%%%%%%%%%%%%%%%%%%%%%%%%%%%%%%%%%%%%%%%%%%%%%%%%%%%%%%%%%%%%
%\subsubsection{Introduction}\label{sec:lhcSeesawIntro}
%\paragraph*{Introduction}\label{sec:lhcSeesawIntro}
%%%%%%%%%%%%%%%%%%%%%%%%%%%%%%%%%%%%%%%%%%%%%%%%%%%%%%%%%%%%%%%%%%

%\pf{Some issues to be addressed: e.g. need to be consistent about HE lumi, plots with 100 TeV curves and no 27TeV curve need to be fixed}

The questions pertaining to neutrino masses: whether  or not  neutrinos are Majorana particles, the origin of smallness of the neutrino masses,  as well as the reason for  large mixing angles, remain some of the most pressing open issues in particle physics today.
A set of potential solutions is provided by the seesaw models. These postulate new particles that couple to SM fields via
mixing/Yukawa couplings, SM gauge currents, and/or new gauge symmetries.
If accessible, a plethora of rich physics can be studied in considerable detail at hadron colliders. 
This would complement low energy and oscillation probes of neutrinos~\cite{Atre:2009rg,Cai:2017mow}.
In the following, we summarize the discovery potential of seesaw models at hadron colliders with collision energies of $\sqrt{s} = 14$ and 27 TeV.
%In particular, we will discuss models featuring heavy neutrinos, both pseudo-Dirac and Majorana as well as those with new gauge interaction, and models featuring scalar and fermion electroweak triplets.
%In Sec.~\ref{sec:lhcSeesawtype1}, models featuring heavy neutrinos, both pseudo-Dirac and Majorana as well as those with new gauge interactions, are discussed.
%Similarly, in Sec.~\ref{sec:lhcSeesawtype2} and Sec.~\ref{sec:lhcSeesawtype3}, models featuring scalar and fermion electroweak triplets are assessed, respectively.
For a more comprehensive reviews on the sensitivity of colliders to neutrino mass models, see \cite{Atre:2009rg,Cai:2017mow,Weiland:2013wha,Ruiz:2015gsa,Marcano:2017ucg} and references therein.


%%%%%%%%%%%%%%%%%%%%%%%%%%%%%%%%%%%%%%%%%%%%%%%%%%%%%%%%%%%%%%%%%%
%\subsubsection{The Type I Seesaw and Variants Discovery Potential at the HL- and HE-LHC}\label{sec:lhcSeesawtype1}
\paragraph*{The Type I Seesaw and Variants}\label{sec:lhcSeesawtype1}
%%%%%%%%%%%%%%%%%%%%%%%%%%%%%%%%%%%%%%%%%%%%%%%%%%%%%%%%%%%%%%%%
In Type I seesaw the light neutrino masses and mixing are generated from couplings of SM leptons to new fermionic gauge singlets with Majorana masses. 
%In \cite{Kersten:2007vk,Moffat:2017feq}, it was proved that requiring all three light neutrinos to be massless is equivalent 
%to the conservation of lepton number at all orders in perturbation theory. 
%In other words, f
For low-scale seesaw models with only fermionic singlets, 
lepton number has to be nearly conserved and light neutrino masses are proportional to small lepton number violating (LNV) parameters~\cite{Kersten:2007vk,Moffat:2017feq}.
% in these models. 
For high-scale seesaws  with only fermionic singlets, light neutrino masses are inversely proportional to large LNV mass scales,
%,
%
and again lepton number is approximately conserved at low energies.
%This in turn leads to the expectation that 
Thus LNV processes are suppressed in type I seesaw models
(unless additional particles
%whether they be 
%fermions or scalars 
%charged under the SM gauge couplings or 
%new gauge interactions,
are introduced to decouple the light neutrino mass generation from heavy neutrino production).
%The updated discovery potential of heavy, SM singlet neutrinos at $pp$ colliders is now summarized.



%%%%%%%%%%%%%%%%%%%%%%%%%%%%%%%%%%%%%%%%%%%%%%%%%%%%%%%%%%%%%%%%%%
%\paragraph*{Heavy Neutrino Production through EW Bosons at Hadron Colliders}
%
%%-----------------------------------------
%\begin{figure}[!t]
%\begin{center}
%%\includegraphics[width=0.9\textwidth]{\main/section6FlavorRelatedStudies/img/lhcSeesaw_Feynman_HeavyNMultiProd_1706_02298.pdf} \label{fig:heavyNDiagrams}
%%\\
%%\includegraphics[width=0.45\textwidth]{\main/section6FlavorRelatedStudies/img/lhcSeesaw_Type1_XSec_vs_mN_LHCX14_1805_09335.pdf} \label{fig:heavyNxsection14}
%%\includegraphics[width=0.45\textwidth]{\main/section6FlavorRelatedStudies/img/lhcSeesaw_Type1_XSec_vs_mN_LHCX27_1805_09335.pdf} \label{fig:heavyNxsection27}
%%\end{center}
%\caption{
%Upper:
%Born-level diagrams for heavy neutrino $N$ production via (a) Drell-Yan, (b) gluon fusion, and (c) vector boson fusion; 
%from \cite{Ruiz:2017yyf}.
%Lower:
%Production cross sections, divided by active-heavy mixing $\vert V_{\ell N}\vert^2$, as a function of the heavy neutrino mass
%at (L) $\sqrt{s}=14\,\mathrm{TeV}$ and (R) 27 TeV~\cite{Degrande:2016aje,Ruiz:2017yyf}.
%}
%\label{fig:lhcSeesaw_HeavyNProd}
%\end{figure}
%%-------------------------------------------------------------------



%Following the prescriptions of \cite{Degrande:2016aje,Ruiz:2017yyf}, 
%for $m_N >M_W$ and at various accuracies,
%the corresponding $\sqrt{s}=14$ TeV heavy neutrino production cross sections are presented in Fig. \ref{fig:lhcSeesaw_HeavyNProd}(bottom left).
%While the Drell-Yan (DY) process dominates at low masses, 
%$W\gamma$ boson fusion (VBF) dominates for $m_N\gtrsim 900-1000\,\mathrm{GeV}$~\cite{Alva:2014gxa,Ruiz:2017yyf},
%with gluon fusion (GF) remaining a sub-leading channel throughout~\cite{Hessler:2014ssa,Ruiz:2017yyf}.
%The situation is quite different at 27 TeV as shown in Fig. \ref{fig:lhcSeesaw_HeavyNProd}(lower right). 
%Indeed, GF is the leading production mode for $m_N\gtrsim 450\,\mathrm{GeV}$ until $m_N\approx 940\,\mathrm{GeV} $ where VBF takes over.
%For $m_N\approx 1$ TeV, the GF, DY, and VBF mechanisms all possess fiducial cross sections in excess of 10 fb.



%%%%%%%%%%%%%%%%%%%%%%%%%%%%%%%%%%%%%%%%%%%%%%%%%%%%%%%%%%%%%%%%%%
%\paragraph*{Discovery Potential of Heavy Pseudo-Dirac Neutrinos in Low Scale Seesaws}

%-------------------------------------------------------------------
\begin{figure}[t]
\begin{center}
\includegraphics[width=0.7\textwidth]{\main/section10/img/lhcSeesaw_Feynman_HeavyNMultiProd_1706_02298.pdf}
\\[2mm]
\includegraphics[width=0.48\textwidth]{\main/section10/img/issVeto_Mixing_Vs_Mass_ETAU_tael_Global_FigL.pdf}~~~~~
%	\label{fig:jetvetotael}
\includegraphics[width=0.48\textwidth]{\main/section10/img/issVeto_Mixing_Vs_Mass_MUTAU_tamu_Global_FigR.pdf}%	\label{fig:jetvetotata}
%\\
%\includegraphics[width=0.45\textwidth]{\main/section6FlavorRelatedStudies/img/lhcSeesaw_ISS_MuTaujj_14TeV.pdf}	\label{fig:MuTaujj_14TeV}
%\includegraphics[width=0.45\textwidth]{\main/section6FlavorRelatedStudies/img/lhcSeesaw_ISS_MuTaujj_27TeV.pdf}	\label{fig:MuTaujj_27TeV}
\end{center}
\caption{
Upper:
Born-level diagrams for heavy neutrino, $N$, production via (a) Drell-Yan (DY), (b) gluon fusion (GF), and (c) vector boson fusion (VBF). 
%from \cite{Ruiz:2017yyf}.
Lower: for the benchmark mixing hypotheses
 $\vert V_{e4}\vert = \vert V_{\tau4} \vert$ with $\vert V_{\mu4}\vert = 0$ (left panel) and
 $\vert V_{\mu4}\vert = \vert V_{\tau4} \vert$ with $\vert V_{e4}\vert = 0$ (right panel),
the projected sensitivity at $\sqrt{s}=14,~27$ and $100$ TeV using the tri-lepton + dynamic jet veto analysis of Ref.~\cite{Pascoli:2018rsg}.
%Lower:
%Number of $\mu^\pm \tau^\mp jj$ events as a function of the seesaw scale $M_R$ for representative scaling factors f in the ISS 
%(L) at the HL-LHC with $3\,\mathrm{ab}^{-1}$
%and 
%(R) at the HE-LHC with $3\,\mathrm{ab}^{-1}$.
%Triangles respect constraints from LFV radiative decays while crosses do not respect the $\tau \rightarrow \mu \gamma$ BaBar upper limit~\cite{Aubert:2009ag}.
}
\label{fig:lhcSeesaw_ISS}
\end{figure}
%-------------------------------------------------------------------

If kinematically accessible, heavy neutrinos $N$ can be produced in hadron collisions through neutral current and charged current processes,
as shown in Fig. \ref{fig:lhcSeesaw_ISS} (upper).
The expected suppression of LNV processes in type I seesaw models 
%that contain only fermionic gauge singlets
motivates the study of lepton number conserving (LNC) processes, such as 
the heavy neutrino $N$ production via DY and VBF, with subsequent decays to only leptons, 
\begin{equation}
 p p \to \ell_N N +X \to \ell_N \ell_W W +X \to \ell_N \ell_W \ell_\nu \nu +X,
 \label{eq:pp3lX}
\end{equation}
giving the trilepton final state, $\ell^\pm_i \ell^\mp_j \ell^\pm_k + \textrm{MET}$.
Projections from a new tri-lepton search strategy recently proposed in Ref.~\cite{Pascoli:2018rsg}, 
based on a dynamical jet veto selection cut, are shown in Fig.~\ref{fig:lhcSeesaw_ISS}, assuming the benchmark mixing hypotheses
 $\vert V_{e4}\vert = \vert V_{\tau4} \vert$ with $\vert V_{\mu4}\vert = 0$ (left panel) and
 $\vert V_{\mu4}\vert = \vert V_{\tau4} \vert$ with $\vert V_{e4}\vert = 0$ (right panel), for $\sqrt{s}=14,~27$ and $100$ TeV.
For benchmark luminosities, the colliders can probe active-sterile mixing as small as (approximately) $\vert V_{\ell 4}\vert^2 \sim 5\times10^{-4} - 9\times10^{-5}$
and masses as heavy as (approximately) $1.5-15$ TeV for $\vert V_{\ell 4}\vert^2 \sim 10^{-2}$.

Another possibility is to search for lepton flavor violating (LFV) final states such as
\begin{equation}
 q ~\overline{q}' \to  ~N ~\ell^\pm_1 \to ~\ell^\pm_1 ~\ell^\mp_2 ~ W^\mp  \to ~\ell^\pm_1 ~\ell^\mp_2 ~j ~j.
 \label{eq:LFVfinalstate}
\end{equation}
This was, e.g., studied in Ref~\cite{Arganda:2015ija} in the context of the inverse seesaw (ISS), a low-scale variant of the type I seesaw.
Due to strong experimental limits on $\mu\to e\gamma$ by
MEG~\cite{Adam:2013mnn}, the event rates involving taus are more promising
than those for $e^\pm \mu^\mp jj$.
%Following~\cite{Arganda:2015ija}, the number of $\tau^\pm \mu^\mp jj$ can
%be estimated using the $\mu_X$-parametrization~\cite{Arganda:2014dta}
%with the neutrino Yukawa coupling
%\begin{equation}
% Y_{\nu}=f \left(\begin{array}{ccc}
%-1&1&0\\
%1&1&0.9\\
%1&1&1
%\end{array}\right)\,.
%\label{eq:Ytmmax1}
%\end{equation}
%as a representative example, and considering that only the lightest
%pseudo-Dirac pair is kinematically available.
%The number of events for this channel 
%is shown in Fig. \ref{fig:lhcSeesaw_ISS}
%(L) and (R) for $\sqrt{s} = 14$ and 27~TeV, respectively, although similar
%numbers would be expected for $\tau^\pm e^\mp jj$.
After  $\mathcal L=3~\rm{ab}^{-1}$ of data taking, more than 100 LFV
events of $\tau^\pm \mu^\mp jj$ type could be produced for neutrino masses below 700 (1000)
GeV for $pp$ collisions at 14 (27)~TeV. 
%\pf{Why was lumi for HE-LHC only 3/ab, not 15/ab?}



%%%%%%%%%%%%%%%%%%%%%%%%%%%%%%%%%%%%%%%%%%%%%%%%%%%%%%%%%%%%%%%%%%
%\paragraph*{Discovery Potential of Heavy Majorana Neutrinos in Phenomenological Type I Seesaw}

%-------------------------------------------------------------------
\begin{figure}[t]
\begin{center}
\includegraphics[width=0.45\textwidth]{\main/section10/img/lhcSeesaw_PhenoType1_lumiVsMNMuMu14TeV_1411_7305.pdf}~~~~	
%\label{fig:seesawIMassReach}	
\includegraphics[width=0.45\textwidth]{\main/section10/img/lhcSeesaw_PhenoType1_sMuMuVsMN14TeV_1411_7305.pdf}	
%\label{fig:seesawIMixingReach}	
\end{center}
\caption{
Left: required luminosity for $3~(5) \sigma$ evidence (discovery) using the LNV final state $\mu^\pm \mu^\pm jj$, as a function of the heavy neutrino mass, $m_N$, assuming optimistic (brown) 
and pessimistic (purple) mixing scenarios~\cite{Alva:2014gxa}. Right:
sensitivity to $N-\mu$ mixing~\cite{Alva:2014gxa} with the optimistic (pessimistic) mixing scenario is given by the horizontal dashed (full) line.
}
\label{fig:SeesawIDiscoveryPotential}
\end{figure}
%-------------------------------------------------------------------

In the presence of additional particles that can decouple the heavy neutrino production 
from the light neutrino mass generation, e.g., new but far off-shell gauge bosons~\cite{Ruiz:2017nip},
the Majorana nature of the heavy neutrinos can lead to striking LNV collider signatures,
such as the well-studied same-sign dilepton and jets process~\cite{Keung:1983uu}
\begin{equation}
 pp \rightarrow  ~N ~\ell^\pm_1 \rightarrow ~\ell^\pm_1 ~\ell^\pm_2 ~ W^\mp  \rightarrow ~\ell^\pm_1 ~\ell^\pm_2 +nj.
 \label{eq:LNVfinalstate}
\end{equation}
Assuming that a low-scale type I seesaw is responsible for the heavy neutrino production, 
Fig. \ref{fig:SeesawIDiscoveryPotential} displays the discovery potential and active-heavy
mixing sensitivity of the $\mu^\pm\mu^\pm$ channel~\cite{Alva:2014gxa}.
Assuming the  pessimistic/conservative mixing scenario of $S_{\mu\mu} = 1.1\times10^{-3}$~\cite{Alva:2014gxa}, %\pf{pessimistic mixing not defined} 
the HL-LHC with 3 ab$^{-1}$ 
would be able to discover a heavy neutrino with a mass of $m_N\simeq400\,\mathrm{GeV}$ 
and is sensitive to masses up to $550\,\mathrm{GeV}$ at $3\sigma$. 
Using only 1 ab$^{-1}$, the HL-LHC can improve on the preexisting mixing constraints summarized in the pessimistic scenario 
for neutrino masses up to $500\,\mathrm{GeV}$.


%%%%%%%%%%%%%%%%%%%%%%%%%%%%%%%%%%%%%%%%%%%%%%%%%%%%%%%%%%%%%%%%%%
\paragraph*{Heavy Neutrinos and the Left-Right Symmetric Model}\label{sec:lhcSeesawlrsmCollider}

%-------------------------------------------------------------------
\begin{figure}[!t]
\begin{center}
\includegraphics[scale=1,width=.4\textwidth]{\main/section10/img/lhcSeesaw_LRSM_ppWR_XSec_vs_Mass_1607_03504.pdf}~~~~~~~~~~ 
%\label{fig:lrsm_HLvsHL_LHC_WRXSec_vs_Mass}		
\includegraphics[scale=1,width=.4\textwidth]{\main/section10/img/lhcSeesaw_LRSM_WR_jN_Disc_vs_Mass_1607_03504.pdf} 
%\label{fig:lrsm_HLvsHL_LHC_WRXDisc_vs_Mass}		
%\\
%\includegraphics[scale=1,width=.45\textwidth]{\main/section6FlavorRelatedStudies/img/lhcSeesaw_lnvWithoutVR_WR_Excl_1703_04669.pdf}	\label{fig:lnvWithoutVR_WR_Excl_1703_04669}
%\includegraphics[scale=1,width=.45\textwidth]{\main/section6FlavorRelatedStudies/img/lhcSeesaw_neftLimits_vs_mN_1703_04669.pdf} 	\label{fig:neftLimits_vs_mN_1703_04669}	
\end{center}
\caption{
Left: The total $pp\to W_R$  cross section at NLO+NNLL(Threshold).
Right: $5(2)\sigma$ discovery potential (sensitivity) via $W_R$ decaying to an electron and neutrino jet $(j_N)$,
as a function of $W_R$ mass, at $\sqrt{s} = 14$ and 27 TeV~\cite{Mitra:2016kov}.
%
%Lower:
%Observed and expected sensitivity to heavy Majorana neutrinos through the process $pp\to \mu^\pm N \to 2\mu^\pm+2j$ and produced
%(L) via non-resonant $W_R$ as well as (R) dimension-six NEFT operators~\cite{Ruiz:2017nip}.
}
\label{fig:lrsm_HLvsHE_LHC}
\end{figure}
%-------------------------------------------------------------------

% \pf{add 27TeV curves to plots that have 100 TeV curves}
The Left-Right Symmetric Model (LRSM) 
%remains one of the best motivated high-energy completions of the SM.
%It 
addresses the origin of both tiny neutrino masses via a Type I+II seesaw hybrid mechanism 
as well as the SM $V-A$ chiral structure through spontaneous breaking of the $SU(2)_L\times SU(2)_R$ symmetry. 
%amongst other low-energy phenomena.
The model predicts new heavy gauge bosons $(W_R^\pm,~Z_R')$,
heavy Majorana neutrinos $(N)$, and a plethora of neutral and electrically charged scalars $(H_i^0,~H_j^\pm,~H_k^{\pm\pm})$.
%Unlike U$(1)_{BL}$ neutrino mass models, t
The LRSM gauge couplings are fixed to the SM Weak coupling constant, up to (small) RG-running corrections.
As a result, the Drell-Yan production mechanisms for $W_R$ and $Z_R$ result in large rates at hadron colliders.
%Following the procedure of \cite{Mitra:2016kov}, the $pp\to W_R^\pm$ cross section at NLO+NNLL(Threshold) 
%is shown in Fig. \ref{fig:lrsm_HLvsHE_LHC}(upper left) as a function of mass $M_{W_R}$ at $\sqrt{s} = 14$ and 27 TeV.
%At $\sqrt{s}=14~(27)$ TeV, one sees that production cross sections for masses as large as $M_{W_R}\approx 5.5~(9)$ TeV are in excess of 1 fb.
%For $M_{W_R}\approx 7.5~(12.5)$ TeV, rates exceed 100 ab, 
%and indicate $\mathcal{O}(10^2-10^3)$ events can be collected with $\mathcal{L}=1-15$ ab$^{-1}$ of data.
%Such computations up to NLO in QCD with parton shower matching, including for more generic coupling,
%are also publicly available following \cite{Mattelaer:2016ynf,Fuks:2017vtl}.
 

%Of the many collider predictions the LRSM, o
One of the most promising discovery channels is the production of heavy Majorana neutrinos
from resonant $W_R$, with $N$ decaying via a lepton number-violating final state. At the partonic level, the process is 
%this is given by
~\cite{Keung:1983uu} (for details see \cite{Cai:2017mow} and references therein)
\begin{equation}
 q_1 \overline{q}_2 \to W_R \to N ~\ell^\pm_i \to \ell_i^\pm \ell_j^\pm W_R^{\mp *} \to \ell_i^\pm \ell_j^\pm q_1' \overline{q_2'}.
 \label{eq:seesawLRSMDY}
\end{equation}
%and has been extensively studied throughout the literature. .
Due to the ability to fully reconstruct \eqref{eq:seesawLRSMDY}, many properties of $W_R$ and $N$ can be extracted,
including a complete determination of $W_R$ chiral couplings to quarks independent of leptons~\cite{Han:2012vk}.
Beyond the canonical $pp \to W_R \to N\ell \to 2\ell + 2j$ channel, it may be the case that the heavy neutrino is hierarchically lighter than the 
right-handed (RH) gauge bosons. 
Notably, for $(m_N/M_{W_R})\lesssim0.1$, $N$ is sufficiently Lorentz boosted that its decay products, particularly the charged lepton, 
are too collimated to be resolved experimentally~\cite{Ferrari:2000sp,Mitra:2016kov}.
%
Instead, one can consider the $(\ell_j^\pm q_1' \overline{q_2'})$-system as a single object, a \textit{neutrino jet}~\cite{Mitra:2016kov,Mattelaer:2016ynf}.
The hadronic process is then
%\begin{equation}
 $ p p \to W_R \to N\ell_i^\pm \to j_N ~\ell_i^\pm$,
 %
and inherits much of the desired properties of \eqref{eq:seesawLRSMDY}, 
such as the simultaneous presence of high-$p_T$ charged leptons and lack of MET~\cite{Mitra:2016kov,Mattelaer:2016ynf},
resulting in a very strong discovery potential.
%Assuming conservative detector efficiency $(\varepsilon)$ and selection acceptance $(\mathcal{A})$ rates
%of $(\varepsilon,\mathcal{A})\approx(0.33,0.64)$ based on the realistic analysis of \cite{Mitra:2016kov},
%and a branching fraction of BR$(W_R \to Ne \to eeq\overline{q'})\approx 10\%$ for $(m_N/M_{W_R})<0.1$. 
%\pf{do you mean $e$ or $\ell$ for the 10\% BR of N?}
Fig. \ref{fig:lrsm_HLvsHE_LHC} (right) shows the requisite integrated luminosity for $5(2)\sigma$ discovery at $\sqrt{s}=14$ and 27 TeV.
%
%With $\mathcal{L}=3~(5)$ ab$^{-1}$, $W_R$ as heavy as $6~(6.5)$ TeV and $10~(10.5)$ TeV, respectively, can be discovered at $\sqrt{s}=14$ (27) TeV.
%With $\mathcal{L}=15$ ab$^{-1}$, mass scales as heavy 16 TeV can be probed at the $2\sigma$ level at $\sqrt{s}=27$ TeV.

%For such heavy $W_R$ and $Z_R$ that may be kinematically outside the reach of the $\sqrt{s}=14$ TeV LHC,
%one can still produce EW- and sub-TeV scale via off-shell $W_R$ and $Z_R$ bosons~\cite{Ruiz:2017nip}.
%As a result, the $pp \to W_R^* \to N\ell \to 2\ell + 2j$ process occurs instead at a hard scale $Q\sim m_N$ and 
%cannot be distinguished from the phenomenological Type I Seesaw without a 
%detailed analysis of the heavy neutrino's chiral couplings~\cite{Han:2012vk,Ruiz:2017nip}.
%However, this also means that searches for heavy $N$ in the context of the phenomenological Type I can be recast/reinterpreted in the context of the LRSM.
%Subsequently, as shown in Fig. \ref{fig:lrsm_HLvsHE_LHC}(lower left), 
%$W_R$ as heavy as $8-9$ TeV can be probed indirectly with $\mathcal{L}=1$ ab$^{-1}$ at $\sqrt{s}=14$ TeV~\cite{Ruiz:2017nip}.
%%
%A similar argument can be applied to heavy neutrinos produced through dimension-six
%Heavy Neutrino Effective Field Theory (NEFT) operators, revealing sensitivity to mass scales up to $\Lambda \sim \mathcal{O}(10)$ TeV
%over the $\sqrt{s}=14$ TeV LHC's lifetime~\cite{Ruiz:2017nip},
%as shown in Fig. \ref{fig:lrsm_HLvsHE_LHC}(lower right).


%%%%%%%%%%%%%%%%%%%%%%%%%%%%%%%%%%%%%%%%%%%%%%%%%%%%%%%%%%%%%%%%%%
%\subsubsection{Type II Scalars Discovery Potential at the HL- and HE-LHC}\label{sec:lhcSeesawtype2}
\paragraph*{Type II Scalars}\label{sec:lhcSeesawtype2}
%%%%%%%%%%%%%%%%%%%%%%%%%%%%%%%%%%%%%%%%%%%%%%%%%%%%%%%%%%%%%%%%

%-------------------------------------------------------------------
\begin{figure}[!t]
\begin{center}
%\includegraphics[scale=1,width=.45\textwidth]{\main/section6FlavorRelatedStudies/img/lhcSeesaw_Type2_Triplet_XSec_vs_Mass.pdf} 		\label{fig:typeii-xsec}
\includegraphics[scale=1,width=.45\textwidth]{\main/section10/img/lhcSeesaw_Type2_L-MH-1tau-LHCX14_1802_0094.pdf}~~~~~~ 
%	\label{fig:LMtau-typeii-LHCX14}
%\\
\includegraphics[scale=1,width=.45\textwidth]{\main/section10/img/lhcSeesaw_Type2_L-MH-1tau-LHCX27_1802_0094.pdf}	
%\label{fig:LMtau-typeii-LHCX27}
%\includegraphics[scale=1,width=.45\textwidth]{\main/section6FlavorRelatedStudies/img/lhcSeesaw_Type2_L-MH-1tau-LHC100_1802_0094.jpg}	\label{fig:LMtau-typeii-LHC100}
\end{center}
\caption{
%Upper (L): The total cross section for $pp\to H^{++} H^{--} and H^{\pm\pm}H^\mp$ at $\sqrt{s}$ = 14, 27, and 100 TeV.
%Upper (R)- Lower (R):
Requisite luminosity  for $5(3)\sigma$ discovery (evidence) as a function of $M_{H^{\pm\pm}}$
for the process $pp\to H^{++}H^{--}\to \tau_h \ell^\pm \ell^\mp \ell^\mp$,
where $\tau^\pm\to \pi^\pm \nu $, 
for the NH and IH at $\sqrt{s}=14, 27$ TeV. 
%and 100 TeV, 
%respectively.
}
\label{fig:lhcSeesawtypeiiScalars}
\end{figure}
%-------------------------------------------------------------------

Type II seesaw introduces a new scalar SU$(2)_L$ triplet that couples to SM leptons.
% in order to 
%reproduce t
The light neutrinos obtain Majorana masses through  SU$(2)_L$ triplet vev, so that
%, giving 
%This is done by the 
%spontaneous generation of a 
%the left handed Majorana mass for the light neutrinos.
%Moreover, the 
type II scenario notably does not have sterile neutrinos.
%, 
%demonstrating that light neutrino masses themselves do not imply the existence of additional fermions.
The most appealing production mechanisms at hadron colliders of triplet Higgs bosons are
% the pair production
%of doubly charged Higgs and the associated production of doubly charged Higgs and singly charged Higgs,
\begin{eqnarray}
pp\to Z^\ast/\gamma^\ast \to H^{++} H^{--}, \ \ \ pp\to W^\ast\to H^{\pm\pm}H^\mp,
\end{eqnarray}
followed, 
%in the most general situation, 
by lepton flavor- and lepton number-violating decays to the SM charged leptons.
%In Fig. \ref{fig:lhcSeesawtypeiiScalars}(upper left), we show the total cross section of the 
%$pp\to H^{++} H^{--}$ and $pp\to  H^{\pm\pm}H^\mp$ processes as a function of triplet mass (in the degenerate limit with $M_{H^\pm}=M_{H^{\pm\pm}})$,
%in collisions at $\sqrt{s} = 14$, 27, and 100 TeV.
In Type II scenarios, $H^{\pm\pm}$ decays to $\tau^\pm \tau^\pm$ and $\mu^\pm \mu^\pm$ pairs are comparable or greater than the
$e^\pm e^\pm$ channel by two orders of magnitude. 
Moreover, the $\tau \mu$ channel is typically dominant in decays involving different lepton flavors~\cite{Perez:2008ha,Li:2018jns}.
If such a seesaw is realized in nature, tau polarizations can help to determine the chiral property of triplet scalars.
One can discriminate between different heavy scalar mediated neutrino mass mechanisms, e.g., between Type II seesaw and Zee-Babu model, 
by studying the distributions of tau lepton decay products~\cite{Sugiyama:2012yw,Li:2018jns}. 
Due to the low $\tau_h$ identification efficiencies, 
future colliders with high energy and/or luminosity enables one to investigate and search for doubly charged Higgs decaying to $\tau_h$ pairs.
Accounting for constraints from neutrino oscillation data on the doubly charged Higgs branching ratios,
as well as  tau polarization effects~\cite{Li:2018jns},
Figs.~\ref{fig:lhcSeesawtypeiiScalars} displays
the 3$\sigma$ and 5$\sigma$ significance in the plane of integrated luminosity versus doubly charged Higgs mass 
for $pp\to H^{++}H^{--}\to \tau^\pm\ell^\pm\ell^\mp\ell^\mp$ at 
$\sqrt{s} = 14, 27$ TeV, for single $\tau$ channel with $\tau\to \pi {\nu}$, both for normal (NH) and inverted hierarchy (IH).
% 
%the sensitivity to doubly charged Higgs mass at HL-LHC can reach 655 GeV and 695 GeV for NH and IH respectively with a luminosity of 3 ab$^{-1}$. 
%Higher masses, 1380-1930 GeV for NH and 1450-2070 GeV for IH, can be probed at 27 TeV with 15 ab$^{-1}$ and 100 TeV with 3 ab$^{-1}$.



%%%%%%%%%%%%%%%%%%%%%%%%%%%%%%%%%%%%%%%%%%%%%%%%%%%%%%%%%%%%%%%%%%
%\subsubsection{Type III Leptons Discovery Potential at HL- and HE-LHC}\label{sec:lhcSeesawtype3}
\paragraph*{Type III Leptons}\label{sec:lhcSeesawtype3}
%%%%%%%%%%%%%%%%%%%%%%%%%%%%%%%%%%%%%%%%%%%%%%%%%%%%%%%%%%%%%%%%

\begin{figure}[!t]
\begin{center}
 \includegraphics[width=0.8\textwidth]{\main/section10/img/lhcSeesaw_Feynman_TypeIIIMultiProd_1711_02180.pdf}	%\label{fig:type3Seesaw}
 \\[5mm]
  \includegraphics[scale=1,width=.40\textwidth]{\main/section10/img/lhcSeesaw_Type3_XSec_vs_Mass_1509_05416.pdf}~~~~~~~~~~~~~~~~ 
  %\label{fig:type3_HLvsHL_LHC_XSec_vs_Mass}
\includegraphics[scale=1,width=.40\textwidth]{\main/section10/img/lhcSeesaw_Type3_Disc_vs_Mass_1509_05416.pdf} %\label{fig:lhcSeesawtype3_HLvsHL_LHC_Disc_vs_Mass}
\end{center}
\caption{
Upper:
Born level production of Type III leptons via (a) Drell-Yan, (b) gluon fusion, and (c) photon fusion; from \cite{Cai:2017mow}.
Lower left:
  the inclusive production cross section for  $pp\to NE^\pm + E^+E^-$, at NLO in QCD~\cite{Ruiz:2015zca} for $\sqrt{s}=14$ and 27 TeV, as a function of heavy triplet lepton mass. Lower right:
  the required integrated luminosity for 5(2)$\sigma$ discovery~(sensitivity) to $N E^\pm + E^+E^-$, 
 based on the analyses of~\cite{Arhrib:2009mz,Li:2009mw}.
}
\label{fig:lhcSeesawtype3}
\end{figure}

Low-scale Type III seesaws introduce heavy electrically charged $(E^\pm)$ and neutral $(N)$ leptons, 
part of SU$(2)_L$ triplet,
that couple to both SM charged and neutral leptons through mixing/Yukawa couplings.
Triplet leptons couple appreciably to EW gauge bosons,  
and thus do not have suppressed production cross section, contrary to seesaw scenarios with gauge singlet fermions.
Up to small (and potentially negligible) mixing effects 
%The presence of electroweak gauge couplings also implies that once a collider energy and mass are stipulated, 
the triplet lepton 
pair production cross sections are fully determined, see Fig. \ref{fig:lhcSeesawtype3} (upper) for relevant tree level diagrams for the production of heavy charged leptons.
%Production mechanisms  include charged current and neutral current Drell-Yan, photon fusion, and gluon fusion if one considers heavy-light charged lepton associated production, see Fig. \ref{fig:lhcSeesawtype3} (upper).
%A recent assessment of triplet production modes found~\cite{Cai:2017mow} that despite the sizable luminosities
%afforded to gluon fusion $(gg\to E^\pm\ell^\mp)$, including its large QCD corrections~\cite{Ruiz:2017yyf},
%and photon fusion $(\gamma\gamma\to E^+E^-)$, the 
Drell-Yan processes are the dominant production channel of triplet leptons when kinematically accessible~\cite{Cai:2017mow}.
%In light of this, in 
Fig. \ref{fig:lhcSeesawtype3} (lower left) shows 
the summed cross sections for the Drell-Yan processes,
%\begin{equation}
$ pp\to \gamma^*/Z^* \to E^+E^-$, and  
$ pp\to W^{\pm *} \to E^\pm N$,
%\end{equation}
%are shown
 at NLO in QCD, following \cite{Ruiz:2015zca}, as a function of triplet masses (assuming $m_N = m_E$), at $\sqrt{s}=14$ and 27 TeV.
%For $m_{N},m_{E} \approx 1.2~(1.8)$ TeV,  the production rate reaches $\sigma(pp\to NE+EE)\approx 1$ fb at $\sqrt{s}=14~(27)$ TeV; and for heavier leptons with $m_{N},m_{E} \approx 2.5~(4.2)$ TeV, 
%one sees that $\sigma(pp\to NE+EE)\approx 1$ ab.


Another consequence of the triplet leptons coupling to all EW bosons is the adherence to the Goldstone Equivalence Theorem.
This implies that triplet leptons with masses well above the EW scale
will preferentially decay to longitudinal polarized $W$ and $Z$ bosons as well as to the Higgs bosons. 
For decays of EW boson to jets or charged lepton pairs, triplet lepton can be fully reconstructed from their final-state 
enabling their properties to be studied in detail.
%
For fully reconstructible final-states,
\begin{eqnarray}
 N E^\pm	\rightarrow ~\ell\ell' + WZ/Wh		&\rightarrow& ~\ell\ell' ~+~ nj+mb,
 \\
 E^+ E^- 	\rightarrow ~\ell\ell' + ZZ/Zh/hh 	&\rightarrow& ~\ell\ell' ~+~ nj+mb,
\end{eqnarray}
which correspond approximately to the branching fractions $\BR(NE)\approx0.115$ and $\BR(EE)\approx0.116$,
search strategies such as those considered in \cite{Arhrib:2009mz,Li:2009mw} can be enacted.
Assuming a fixed detector acceptance and efficiency of $\mathcal{A}=0.75$,
which is in line to those obtained by \cite{Arhrib:2009mz,Li:2009mw},
Fig. \ref{fig:lhcSeesawtype3}(lower right) shows as a function of triplet mass 
the requisite luminosity for a 5$\sigma$ discovery (solid) and 2$\sigma$ evidence (dash-dot)
of triplet leptons at $\sqrt{s}=14$ and 27 TeV.
%
With $\mathcal{L}=3-5$ ab$^{-1}$, the 14 TeV HL-LHC can discover states as heavy as $m_{N},m_{E}=1.6-1.8$ TeV.
For the same amount of data, the 27 TeV HE-LHC can discover heavy leptons  $m_{N},m_{E}=2.6-2.8$ TeV;
with $\mathcal{L}=15$ ab$^{-1}$, one can discover~(probe) roughly $m_{N},m_{E}=3.2~(3.8)$ TeV.

\subsection{Implications of TeV scale flavour models for electroweak baryogenesis} 
{\it\small Author(s) (TH): Oleksii Matsedonskyi, Geraldine Servant}

%Intro: EWBG

In most solutions to the SM flavour puzzle, Yukawa couplings have a dynamical origin, which means that they potentially impact the cosmological evolution.
% should be studied.
 In Refs.~\cite{Baldes:2016rqn,Baldes:2016gaf,vonHarling:2016vhf,Bruggisser:2017lhc,Bruggisser:2018mus,Bruggisser:2018mrt,Baldes:2018nel,Servant:2018xcs},  such connections between flavour and cosmology, in particular with the  electroweak baryogenesis,  have been investigated in detail.

Electroweak baryogenesis (EWBG) is a mechanism to explain the matter antimatter asymmetry of the universe  using baryon number violation already present in the SM. It 
 relies on a charge transport mechanism in the vicinity of bubble walls during a first-order electroweak (EW) phase transition. EWBG uses EW scale physics only and is therefore testable experimentally. It requires an extension of the Higgs sector, giving a first-order EW phase transition as well as new sources of CP violation. Typically, there are stringent constraints on EWBG models from bounds on Electric Dipole Moments.
 In the following,  we consider that the source of CP violation has changed with time, which is a natural way to evade constraints. 
 The main motivation is to link EWBG to low-scale flavour models.  If the physics responsible for the structure of the Yukawa couplings is linked to EW symmetry breaking, we can expect the Yukawa couplings to vary at the same time as the Higgs is acquiring a VEV, in particular if the flavour structure is controlled by a new scalar field which couples to the Higgs.
 This is  precisely what can happen in Composite Higgs (CH) models ~\cite{Kaplan:1983fs,Panico:2015jxa}, on which we focus in the following.


%Intro: CH

The  Composite Higgs (CH) models assume that the Higgs boson arises as a bound state of a new strong interaction, confining around the TeV scale $f$. Other composite resonances are naturally heavier than the Higgs, due to an approximate Goldstone symmetry suppressing the Higgs mass. The rest of the SM fields do not belong to the strong sector and are elementary. The nature of the EW phase transition in CH models can be substantially different with respect to the SM. One of the reasons is that the CH models naturally feature new scalar resonances that participate in the EW phase transition and change its properties (see Refs.~\cite{Espinosa:2011eu,Chala:2016ykx}). On the other hand, the EW transition may become strongly first order even without the help of such additional states, if the EW transition happens simultaneously with the deconfinement-confinement phase transition of the new strong sector (see Refs.~\cite{Bruggisser:2018mus,Bruggisser:2018mrt} and also Refs.~\cite{vonHarling:2016vhf,Megias:2018sxv} for the dual description in the warped 5D space).    

%Flavour in CH

The flavour structure of the CH models is intimately tied with the viability of EWBG. The requirement of having a sizeable top quark Yukawa coupling, while suppressing the unwanted flavour-violating effects, suggests that the new sector is nearly scale-invariant
% and strongly coupled 
for a large range of energies above the confinement scale. As a result, one may expect that the transition dynamics is mostly determined by a single light field -- the dilaton $\chi$ (see e.g Ref.~\cite{Coradeschi:2013gda}). Depending on its mass, whose size can be related to the separation of the UV flavour scale and the EW scale, the EW phase transition may happen separately from the confinement, or simultaneously with it~\cite{Bruggisser:2018mus,Bruggisser:2018mrt}. %The two-step transition is expected to happen for the heavier dilaton, while the one-step -- for the dilaton lighter than about $\sim 0.5$~TeV. 

Moreover, the mechanism for generating the SM flavour hierarchy in CH models may also be a source of CP asymmetry during the EW phase transition. The hierarchy of SM Yukawas $\lambda_q$ 
%\td{does it conform to Higgs section notation} 
is generated by the renormalization group running of the couplings $y_q$ between the elementary fermions and the strong sector operators, 
\begin{equation}
\lambda_q \propto y_q^2, \quad\text{with} \quad y_q=y_q^{\text{UV}} \left( \frac{\mu}{\Lambda_{\text{UV}}} \right)^{\gamma_{y_q}},
\end{equation}
where $\mu\sim \chi$ is the confinement scale, $\Lambda_{\text{UV}}$ is some large scale at which all the mixings are generated with a similar size, $y_q^{\text{UV}}$, and $\gamma_{y_q}$ is the anomalous dimension of the operator responsible for the mixing.
 This means that the size of the Yukawa couplings changes with the evolution of the confinement scale during the confinement phase transition. Such a change of the Yukawa interactions may efficiently source the CP-violation required for the baryogenesis~\cite{Bruggisser:2017lhc}.

%Pheno

\begin{figure}[!t]\centering
\includegraphics[scale=.44]{\main/section10/img/hvv.png}
\hspace{0.2cm}
\includegraphics[scale=.44]{\main/section10/img/Imhtt.png}
\caption{Relative deviation of the Higgs couplings to $W$ and $Z$ bosons (left panel) and the imaginary part of the correction to the top quark Yukawa coupling (right panel), as functions of the dilaton mass, $m_\chi$, the number of colours, $N$, in the new confining sector, and generic mass of composite states, $m_*$. Solid black (red dashed) contours correspond to the glueball-like (meson-like) dilaton. The current  and near future experimental sensitivities of electron EDM experiments  to the imaginary part of the top yukawa correspond respectively to approximately $2\times 10^{-3}$ \cite{Andreev:2018ayy,Brod:2013cka}
%~\cite{Cirigliano:2016nyn} \td{update?} 
and $2\times 10^{-4}$~\cite{Kumar:2013qya}.
}
\label{fig:ewbg_observables}
\end{figure}

For both types of transitions  mentioned, one can expect to observe deviations of the Higgs couplings from the SM predictions. These deviations are result of contributions generic to CH models, as well as those linked to the features of the phase transition and new sources of CP-violation. For concreteness we focus on the  more minimal example, the combined electroweak and strong sector phase transition. 
%
The potential of the Higgs boson can be parametrized in terms of trigonometric functions of $h/f$, with the overall size of the potential controlled by the mixings between the elementary and composite fermions~\cite{Matsedonskyi:2012ym,Panico:2012uw},
\begin{equation}\label{eq:ewbg_vh}
V = c_1 \sin^2 \frac h f + c_2 \sin^4 \frac h f,
\end{equation}
where $c_1 \sim c_2 \sim \sum_q ({3 y_q^2}/{(4 \pi)^2}) g_\star^2 f^4$. The dependence of the $y_q$ mixings on the dilaton field is responsible for the mass mixing between the dilaton and the Higgs field, which we parametrise by an angle $\delta$.
To generate sufficient amount of CP violation, such mass mixing needs to be sizeable. Let us consider its effect on the quark Yukawa coupling:
\begin{equation}\label{eq:ewbg_yuk}
{\cal L}_{\rm Yukawa} = \lambda(\chi) \left(\chi \sin \frac h f \right) \bar q_L q_R 
= \bar q_L q_R h \left(\lambda(f) \frac {\chi}{f} + \beta_{\lambda} \frac {\chi - f }{f} \right) + \dots, 
\end{equation}
where we performed an expansion in $\chi$ around its present day value, $f$. Similar, but CP-preserving, modifications are also generated in the couplings of the Higgs boson to the SM gauge fields and the Higgs self-interactions.
The complex phase of the Yukawa beta-function, $\beta_\lambda$, in \eqref{eq:ewbg_yuk} has to be different from that of the quark mass $\lambda(f) h$, as the Yukawa phase changes with energy. 
Choosing the mass parameter to be real, the CP-violating interaction resides in the term $\propto \beta_\lambda$. Rotating the $h$ and $\chi$ fields to the mass basis, the CP-even and CP-odd corrections to the Higgs Yukawa  interaction are
\begin{equation}
\text{Re} [\delta \lambda] \sim \text{Re} [\beta_\lambda] \delta (v/f) + \lambda (\delta^2/2+\delta(v/f)) ,\quad
\text{Im} [\delta \lambda] \sim \text{Im} [\beta_\lambda]  \delta (v/f).
\end{equation} 
In Fig.~(\ref{fig:ewbg_observables}) we show the values of the CP-violating top quark Yukawa modification and the deviations of the Higgs couplings to the $W$ and $Z$ bosons.
Such couplings can be tested directly at the LHC, and also in the measurements of electric dipole moments.  
%In addition to the observables directly linked to the CP violation, t
The strength of the phase transition can be tested in gravitational waves signals at the future space-based observatory LISA~\cite{Caprini:2015zlo}.

%%%%%%%%%%%%
%%%  Anomalies   %%%
%%%%%%%%%%%%
\subsection{High \pt searches in the context of flavour anomalies}

{\bf Authors:} Alejandro Celis, Admir Greljo, Lukas Mittnacht, Marco Nardecchia, Tevong You

Precision measurements of flavour transitions at low energies, such as flavour changing $B$, $D$ and $K$ decays, are sensitive probes of hypothetical dynamics at high energy scales.  These can provide the first evidence of new phenomena beyond the SM, even before direct discovery of new particles at high energy colliders. Indeed, the current anomalies observed in $B$-meson decays, in particular, the charged current one in $b\to c\tau \nu$ transitions, and neutral current one in $b\to s\ell^+\ell^-$,  may be the first hint of new dynamics which is still waiting to be discovered at high-\pt. When considering models that can accommodate the anomalies, it is crucial to analyse the constraints derived from high-\pt searches at the LHC, since these can often rule out significant regions of model parameter space. Below we review these constrains, and assess the impact of the High Luminosity and High Energy LHC upgrades (see also the discussion in Sections \ref{sec:7:bsll:NP},\ref{sec7:bsll:model:indep:fits}, and \ref{sec7:ModelsNP:bctaunu}).



\subsubsection{EFT analysis} 

If the dominant NP effects give rise to dimension-six SMEFT operators, the low-energy flavour measurements are sensitive to $C/\Lambda^2$, with $C$ the dimensionless NP Wilson coefficient and $\Lambda$ the NP scale.    The size of the Wilson coefficient is model dependent, and thus so is the NP scale required to explain the $R_{D^{(*)}}$and $R_{K^{(*)}}$ anomalies.    Perturbative unitarity sets an upper bound on the energy scale below which new dynamics need to appear~\cite{DiLuzio:2017chi}.    
The conservative bounds on the scale of unitarity violation are   $\Lambda_U = 9.2$ TeV and $84$ TeV for $R_{D^{(*)}}$and $R_{K^{(*)}}$, respectively, obtained when the flavour structure of NP operators is exactly aligned with what is needed to explain the anomalies.       More realistic frameworks for flavour structure, such as MFV, $U(2)$ flavour models, or partial compositeness, give rise to NP effective operators with largest effects for the third generation. This results in stronger unitarity bounds,
$\Lambda_U = 1.9$ TeV and $17$ TeV for $R_{D^{(*)}}$ and $R_{K^{(*)}}$, respectively.  
These results mean that: \textit{(i)} the mediators responsible for the $b\to c \tau \nu$ charged current anomalies are expected to be in the energy range of the LHC, \textit{(ii)}  the mediators responsible for the $b\to s \ell \ell$ neutral current anomalies could well be above the energy range of the LHC. However, in realistic flavour models also these mediators typically fall within the (HE-)LHC reach.       

If the neutrinos in $b \to c \tau \nu$ are part of a left-handed doublet, the NP responsible for $R_{D^{(*)}}$ anomaly
generically implies a sizeable signal in $p p \to \tau^+ \tau^-$ production at high-\pt. For realistic flavour structures, in which $b \to c$ transition is $\mathcal{O}(V_{cb})$ suppressed compared to $b \to b$, one expects rather large $b b \to \tau \tau$ NP amplitude. 
Schematically, $\Delta R_{D^{(*)}} \sim C_{b b \tau \tau} (1 + \lambda_{bs} / {V_{cb}})$, where $C_{b b \tau \tau}$ is the size of effective dim-6 interactions controlling $b b \to \tau \tau$, and $\lambda_{bs}$ is a dimensionless parameter controlling the size of flavour violation. Recasting ATLAS 13 TeV, $3.2$~fb$^{-1}$ search for $\tau^+ \tau^-$ \cite{Aaboud:2016cre}, Ref.~\cite{Faroughy:2016osc} showed that $\lambda_{bs} = 0$ scenario is already in slight tension with data. For $\lambda_{bs} \sim 5$, which is moderately large, but still compatible with FCNC constraints, HL (or even HE) upgrade of the LHC would be needed to cover the relevant parameter space implied by the anomaly (see Fig.~[5] in~\cite{Buttazzo:2017ixm}). 
For large $\lambda_{bs}$ the limits from $p p \to \tau^+ \tau^-$ become comparable with direct the limits on $p p \to \tau \nu$ from the bottom-charm fusion. The limits on the EFT coefficients from $p p \to \tau \nu$ were derived in Ref.~\cite{Greljo:2018tzh}, and the future LHC projections are  promising. The main virtue of this channel is that the same four-fermion interaction is compared in $b \to c \tau \nu$ at low energies and $b c \to \tau \nu$ at high-\pt. 
Since the effective NP scale in $R(D^{(*)})$ anomaly is low, the above EFT analyses are only indicative. 
For more quantitative statements we review below bounds on explicit models.

The hints of NP in $R_{K^{(*)}}$ require  a $(b s) (\ell \ell)$ interaction.
Correlated effects in high-\pt tails of $p p \to \mu^+ \mu^- (e^+ e^-)$ distributions are expected, if the numerators (denominators) of LFU ratios $R_{K^{(*)}}$ are affected. 
Ref.~\cite{Greljo:2017vvb} recast the  13~TeV $36.1$~fb$^{-1}$ ATLAS search \cite{ATLAS:2017wce} (see also \cite{Aaboud:2017buh}),
to set limits on a number of semi-leptonic four-fermion operators, and derive projections for HL-LHC (see Table 1 in~\cite{Greljo:2017vvb}). These show that direct limits on the $(b s) (\ell \ell)$ operator from the tails of distributions will never be competitive with those implied by the rare $B$-decays~\cite{Greljo:2017vvb,Afik:2018nlr}. On the other hand, flavour conserving operators, $(q q) (\ell \ell)$, are efficiently constrained by the high \pt tails of the distributions. The flavour structure of an underlining NP could thus be probed by constraining ratios $\lambda^q_{b s} = C_{b s} / C_{qq}$ with $C_{b s}$  fixed by the $R_{K^{(*)}}$ anomaly. For example, in models with MFV flavour structure, so that $\lambda^{u,d}_{b s} \sim V_{c b}$, the present high-\pt dilepton data is already in slight tension with the anomaly~\cite{Greljo:2017vvb}. Instead, if  couplings to valence quarks are suppressed, e.g., if NP dominantly couples to the 3rd family SM fermions, then $\lambda^b_{b s} \sim V_{c b}$. Such NP will hardly be probed even at the HL-LHC, and  it is possible that NP responsible for the neutral current anomaly might stay undetected in the high-\pt tails at HL-LHC and even at HE-LHC. Future data will cover a significant part of viable parameter space, though not completely, so that discovery is possible, but not guaranteed.


\subsubsection{Constraints on simplified models for $b\to c \tau \nu$}


Since the $b\to c \tau \nu$ decay is a tree-level process in the SM that receives no drastic suppression, models that can explain these anomalies necessarily require a mediator that contributes at tree-level:      


\noindent  $\bullet$ \, \textit{SM-like $W^{\prime}$}:  A SM-like $W^{\prime}$ boson, coupling to left-handed fermions, would explain the approximately equal enhancements observed in $R(D)$ and $R(D^{*})$.     A possible realization is a color-neutral real $SU(2)_L$ triplet of massive vector bosons~\cite{Greljo:2015mma}.  However, typical models encounter problems with current LHC data since they result in large contributions to $pp \to \tau^+ \tau^-$ cross-section, mediated by the neutral partner of the $W^{\prime}$~\cite{Greljo:2015mma,Boucenna:2016qad,Faroughy:2016osc}.    For $M_{W^{\prime}} \gtrsim 500$~GeV, solving the $R(D^{(*)})$ anomaly within the vector triplet model while being consistent with $\tau^+ \tau^-$ resonance searches at the LHC is only possible if the related $Z^{\prime}$ has a large total decay width~\cite{Faroughy:2016osc}. Focusing on the $W^{\prime}$, Ref.~\cite{Abdullah:2018ets} analyzed the production of this mediator via $gg$ and $gc$ fusion, decaying to
$\tau \nu_{\tau}$. Ref.~\cite{Abdullah:2018ets} concluded that a dedicated search using that $b$-jet is present in the final state would be effective in reducing the SM background compared to an inclusive analysis that relies on $\tau$-tagging and $E_T^{\rm{miss}}$. Nonetheless, relevant limits will be set by an inclusive search in the future~\cite{Greljo:2018tzh}.


\vspace{0.2cm}

\noindent  $\bullet$  \,  \textit{Right-handed $W^{\prime}$:} Refs.~\cite{Greljo:2018ogz,Asadi:2018wea} recently proposed that $W^{\prime}$ could mediate a right-handed interaction, with a light sterile right-handed neutrino carrying the missing energy in the $B$ decay. In this case, it is possible to completely decorelate FCNC constraints from $R(D^{(*)})$. The most constraining process in this case is instead $p p \to \tau \nu$. Ref.~\cite{Greljo:2018tzh} performed a recast of the latest ATLAS and CMS searches at 13 TeV and about 36 fb$^{-1}$ to constrain most of the relevant parameter space for the anomaly.

\vspace{0.2cm}

\noindent  $\bullet$\, \textit{Charged Higgs $H^{\pm}$:}       Models that introduce a charged Higgs, for instance a two-Higgs-doublet model, also contain additional neutral scalars.  Their masses are constrained by electroweak precision measurements to be close to that of the charged Higgs.  Accommodating the $R(D^{(*)})$ anomalies with a charged Higgs typically implies large new physics contributions to $pp \to \tau^+ \tau^-$ via the neutral scalar exchanges, so that current LHC data can challenge this option~\cite{Faroughy:2016osc}.     Note that a charged Higgs also presents an important tension between the current measurement of $R(D^*)$ and the measured lifetime of the $B_c$ meson~\cite{Li:2016vvp,Alonso:2016oyd,Celis:2016azn,Akeroyd:2017mhr}.


\begin{figure}[t]
\centering
\includegraphics[width=0.39 \textwidth]{section10/img/LQ-search-strategy.pdf}
\caption{Schematic of the LHC bounds on  LQ showing complementarity  in constraining the  $(m_{LQ},y_{q_{\ell}})$ parameters. The three cases are:
 pair production $\sigma \propto y_{q_{\ell}}^0 $, single production $\sigma \propto y_{q_{\ell}}^2 $ and Drell-Yan $\sigma \propto y_{q_{\ell}}^4$ (from~\cite{Dorsner:2018ynv}).}
\label{fig:LQstrategy}
\end{figure}

\begin{figure}[t]
\centering
\includegraphics[width=0.46 \textwidth]{section10/img/S3.pdf}~~~
\includegraphics[width=0.46 \textwidth]{section10/img/U1.pdf}
\caption{ \small \Blu{Present constraints and HL(HE)-LHC projections in the leptoquark mass versus coupling  plane for the scalar leptoquark $S_3$ (left), and vector leptoquark $U_1$ (right). The grey and dark grey solid regions are the current exclusions. The grey and black dashed lines are the projected reach for HL-LHC (pair and single leptoquark production prospects are based on the CMS simulation from Section XXX). The red dashed lines are the projected reach at HE-LHC (see Section \ref{sec:HE_LHC_LQ}). The green and yellow bands are the $1\sigma$ and $2\sigma$ preferred regions from the fit to $B$ physics anomalies. The second coupling required to fit the anomaly does not enter in the leading high-\pt diagrams but it is relevant for fixing the preferred region shown in green, for more details see Ref.~\cite{Buttazzo:2017ixm}. 
}
}
\label{fig:LQmediators}
\end{figure}

\vspace{0.2cm}

\noindent  $\bullet$ \, \textit{Leptoquarks:} 

The observed anomalies in charged and neutral currents appear in semileptonic decays of the $B$-mesons. This implies that the putative NP has to couple to both quarks and leptons at the fundamental level. A natural BSM option is to consider mediators that couple simultaneously quarks and leptons at the tree level. Such states are commonly referred as leptoquarks. Decay and production mechanisms of the LQ are directly linked to the physics required to explain the anomalous data.
\begin{itemize}
\item {\bf Leptoquark decays:} the fit to the $R(D^{*})$ observables suggest a rather light leptoquark (at the TeV scale) that couples predominately to the third generation fermions of the SM. A series of constraints from flavour physics, in particular the absence of BSM effects in kaon and charm mixing observables, reinforces this picture. 

\item {\bf Leptoquark production mechanism:}  The size of the couplings required to explain the anomaly is typically very large, roughly $y_{q\ell} \approx m_{LQ}$/ (1 TeV). Depending of the actual sizes of the leptoquark couplings and its mass we can distinguish three regimes that are relevant for the phenomenology at the LHC:
 	\begin{enumerate}
	\item LQ pair production due to strong interactions,
	\item Single LQ production plus lepton via a single insertion of the LQ coupling, and
	\item Non-resonant production of di-lepton through $t$-channel exchange of the leptoquark.
	\end{enumerate}
Interestingly all three regimes provide complementary bounds in the  $(m_{LQ},y_{q_{\ell}})$ plane, see Fig.~\ref{fig:LQstrategy}.

\end{itemize}




Several simplified models with leptoquark as a mediator were shown to be consistent with the low-energy data.
 A vector leptoquark with $SU(3)_c\times SU(2)_L\times U(1)_Y$ SM quantum numbers $U_\mu \sim ({\bf 3}, {\bf 1}, 2/3)$ was identified as the only single mediator model which can simultaneously fit the two anomalies (see e.g.~\cite{Buttazzo:2017ixm} for a recent fit including leading RGE effects). 
 In order to substantially cover the relevant parameter space, one needs future HL- (HE-) LHC, see Fig.~\ref{fig:LQmediators} (right) (see also Fig.~5 of~\cite{Buttazzo:2017ixm} for details on the present LHC constraints). A similar statement applies to an alternative model featuring two scalar leptoquarks, $S_1, S_3$ \cite{Crivellin:2017zlb}. The pair of plots in Fig. \ref{fig:LQmediators} summaries the current exclusion and the discovery reach for the HE and HL-LHC in the LQ coupling versus mass plane. 
 
 \Blu{Leptoquarks states are emerging as the most convincing mediators for the explanations of the flavour anomalies. It is then important to explore all the possible signatures at the HL- and HE-LHC.
 The experimental program should focus not only on final states containing quarks and leptons of the third generation, but also on the whole list of decay channels including the off-diagonal ones ($b \mu$, $s \tau, \dots$). The completeness of this approach would allow to shed light on the flavour structure of the putative New Physics. 
 
 Another aspects to be emphasized concerning leptoquark models for the anomalies is the fact of the possible presence of extra fields required to complete the UV lagrangian. The accompanying particles would leave more important signatures at high $p_T$ than the leptoquark, this is particularly true for vector leptoquark extensions (see for example \cite{DiLuzio:2017vat,DiLuzio:2018zxy})}  
    
 
%\Blu{As a final remark, it is important to remember that the anomalies are not yet experimentally established. Among others, this also means that the statements on whether or not  the high \pt LHC constraints rule out certain $R(D^{(*)})$ explanations assumes that the actual values of $R(D^{(*)})$ are given by their current global averages. If future measurements decrease the global average, the high \pt constraints can in some cases be greatly relaxed and HL- and/or HE-LHC may be essential for these, at present tightly constrained, cases.}


%%%%%%%%%%%
\subsubsubsection{HE-LHC sensitivity study for $p p \to \Phi \Phi \to (b \tau) (b \tau)$}
\label{sec:HE_LHC_LQ}
\begin{figure}[t]
	\centering
	\includegraphics[width=0.5\textwidth]{section10/img/soverb.pdf}
	\caption{Expected sensitivity for pair production of scalar (\Blu{$S_3$}) and vector ($U_1$) leptoquark at 27 TeV $p p$ collider with an integrated luminosity of $15~\text{ab}^{-1}$.} \label{fig:lukas}
\end{figure}

We analysed the sensitivity of the 27 TeV $pp$ collider with $15$~ab$^{-1}$ integrated luminosity to probe pair production of the scalar and vector leptoquarks decaying to $(b \tau)$ final state. We investigated events containing either one electron or muon, one hadronically decaying tau lepton, and  at least two jets. The signal events and the dominant background events ($t\bar{t}$) were generated with {\tt MadGraph5\_aMC@NLO} at leading-order. {\tt PYTHIA6} was used to shower and hadronise events and {\tt DELPHES3} was used to simulate the detector response. The scalar leptoquark (\Blu{$S_3$}) and the vector leptoquark ($U_1$) UFO model files were taken from~\cite{Dorsner:2018ynv}.

To verify the procedure, we simulated the $t\bar{t}$ background and the scalar leptoquark signal at 13 TeV and compared to the predicted shapes in the $S_T$ distribution from the existing CMS analysis~\cite{Sirunyan:2017yrk}. After we verified the 13 TeV analysis, we simulated the signal and the dominant background events at $27$ TeV. From these samples, we selected all events satisfying the particle content requirements and applied the lower cut in the $S_T$ variable. The cut threshold was chosen to maximize $s/\sqrt{b}$ while requiring at least 2 expected signal events at an integrated luminosity of $15~\text{ab}^{-1}$. In the case of the vector leptoquark we considered the Yang-Mills ($\kappa = 1$) and the minimal coupling ($\kappa = 0$) scenarios for the couplings to the gluon field strength, ${\cal L}\supset -ig \kappa U_{1\mu}^\dagger T^a U_{1\nu} G^a_{\mu\nu}$~\cite{Dorsner:2018ynv}. From the simulations of the scalar and vector leptoquark events, we found the ratio of the cross-sections, and assuming similar kinematics, we estimated the sensitivity also for the vector leptoquark $U_1$.

As shown in Fig.~\ref{fig:lukas}, the HE-LHC collider will be able to probe pair produced third generation scalar leptoquark (decaying exclusively to $b \tau$ final state) up to mass of $\sim 4$~TeV and vector leptoquark up to $\sim 4.5$~TeV and $\sim 5.2$~TeV for the minimal coupling and Yang-Mills scenarios, respectively. While this result is obtained by a rather crude analysis, it shows the impressive reach of the future high-energy $pp$-collider. In particular, the HE-LHC will cut deep into the relevant perturbative parameter space for $b \to c \tau \nu$ anomaly. As a final comment, this is a rather conservative estimate of the sensitivity to leptoquark models solving $R(D^{*})$ anomaly, since a dedicated single leptoquark production search is expected to yield even stronger bounds~\cite{Buttazzo:2017ixm,Dorsner:2018ynv}.
%%%%%%%%%%%







\subsubsection{Constraints on simplified models for $b\to s ll$}

The $b\to s ll$ transition is  both loop and CKM suppressed in the SM. The explanations of the $b\to s ll$ anomalies can thus have both tree level and loop level mediators.
% enter at tree-level or at the loop level. 
Loop-level explanations typically involve lighter particles.
% that enter at a lower UV cut-off scale. 
Tree-level mediators can also be light, if sufficiently weakly coupled. However, they can also be much heavier---possibly beyond the reach of the LHC. 

%\begin{figure}[t]
%\centering
%\includegraphics[width=0.45 %\textwidth]{section10/img/feynmandiagrams_LQZprime.PNG}
%\caption{Tree level Feynman diagrams for leptoquarks (left) and $Z^\prime$ (right)  mediating $b \to s \mu^+ \mu^-$ transition (Fig. taken from ~\cite{Allanach:2017bta}).}
%\label{fig:feynmanLQZprime}
%\end{figure}

\noindent $\bullet$ \textit{Tree-level mediators:}\\
%For the four-fermion operators involved in 
For $b \to sll$ anomaly there are two possible tree-level UV-completions, the $Z'$ vector boson and leptoquarks, either scalar or vector (see Fig.~\ref{fig:bsll_diagrams} in Sec.~\ref{sec:7:bsll:NP}).
% whose Feynman diagrams are depicted in Fig.~\ref{fig:feynmanLQZprime}. The $Z^\prime$ vector boson mediates an interaction between the $b$ and $s$ quark at one vertex and two muons at the other. A leptoquark is defined as having quantum numbers such that it may interact with a quark and lepton at one vertex, and can be either scalar or vector. In this case the minimal couplings required of the leptoquark must contain a $b$ or $s$ together with a muon at a vertex. 
%Ref.~\cite{Allanach:2017bta} 
%%initiated the study of implications of the anomalies for direct discovery at future colliders. They 
%estimated the sensitivity to simplified models of $Z^\prime$ and leptoquarks at the HL- and HE-LHC (as well as the FCC-hh) in the expected parameter space where the new particles may explain the neutral current flavour anomalies. A more detailed study for the $Z^\prime$ case was then made by Ref.~\cite{Allanach:2018odd}, extending in particular to wider widths. We summarise their results here. 
For leptoquarks, Fig.~\ref{fig:LQexclusion} shows the current 95\% CL limits from 8 TeV CMS with 19.6 fb$^{-1}$  in the $\mu\mu jj$ final state (solid black line), as well as the HL-LHC (dashed black line) and 1 (10) ab$^{-1}$ HE-LHC extrapolated limits (solid (dashed) cyan line). 
%is denoted by a dashed black line for HL-LHC and by a solid (dashed) cyan line for HE-LHC with 1 (10) ab$^{-1}$. 
Dotted lines give the cross-sections times branching ratio at the corresponding collider energy for pair production of scalar leptoquarks, calculated at NLO using the code of Ref.~\cite{Kramer:2004df}. We see that the sensitivity to a leptoquark with only the minimal $b-\mu$ and $s-\mu$ couplings reaches around 2.5 and 4.5 TeV at the HL-LHC and HE-LHC, respectively. This pessimistic estimate is a lower bound that will typically be improved in realistic models with additional flavour couplings. Moreover, the reach can be extended by single production searches~\cite{Allanach:2017bta}, albeit in a more model-dependent way than pair production. The cross section predictions for vector leptoquark are more model-dependent and are not shown in Fig.~\ref{fig:LQexclusion}. For $\mathcal{O}(1)$ couplings the corresponding limits are typically stronger than for scalar leptoquarks. 

\begin{figure}[t]
\centering
\includegraphics[width=0.45 \textwidth]{section10/img/LQ_exclusionplot_HLLHC_HELHC.png}
\caption{Current and projected 95\% CL limits on $\mu\mu jj$ final state at CMS (solid black) and HL-LHC (cyan) with 1 (10) ab$^{-1}$ in solid (dashed) lines. The pair production cross-section for scalar leptoquarks is shown in dotted lines for 14 (27) TeV in black (cyan) (Fig. taken from ~\cite{Allanach:2017bta}). \Blu{See also Section (XXX-->Cambridge group)}}
\label{fig:LQexclusion}
\end{figure}

\begin{figure}[t]
\centering
\includegraphics[width=0.39 \textwidth]{section10/img/14TeVMDM.pdf}
\caption{ HL-LHC 95\% (blue) and 99\% (orange) CL sensitivity contours to $Z^\prime$ in the ``mixed-down'' model for $g_{\mu\mu}$ vs $Z^\prime$ mass in TeV. The dashed grey contours give the $Z'$ width as a fraction of mass. The green and red regions are excluded by trident neutrino production and $B_s$ mixing, respectively. The dashed blue line is the stronger $B_s$ mixing constraint from Ref.~\cite{Bazavov:2016nty} (Fig. taken from~\cite{Allanach:2018odd}). \Blu{See also Section (XXX-->Cambridge group)} }
\label{fig:Zprime14TeV}
\end{figure}

For the $Z^\prime$ mediator the minimal couplings in the mass eigenstate basis are obtained by unitary transformations from the gauge eigenstate basis, which necessarily induces other couplings.  Ref.~\cite{Allanach:2018odd} defined the ``mixed-up'' model (MUM) and ``mixed-down'' model (MDM) such that the minimal couplings are obtained via CKM rotations in either the up or down sectors respectively. For MUM there is no sensitivity at the HL-LHC.  The predicted sensitivity at the HL-LHC for the MDM is shown in Fig.~\ref{fig:Zprime14TeV} as functions of $Z^\prime$ muon coupling $g_{\mu\mu}$ and the $Z^\prime$ mass, setting the $Z'$ coupling to $b$ and $s$ quarks such that it solves the $b\to s\ell\ell$ anomaly. The solid blue (orange) contours give the 95\% and 99\% CL sensitivity. The red and green regions are excluded by $B_s$ mixing~\cite{Arnan:2016cpy} and neutrino trident production~\cite{Falkowski:2017pss,Falkowski:2018dsl}, respectively. The more stringent $B_s$ mixing constraint from Ref.~\cite{Bazavov:2016nty} is denoted by the dashed blue line; see, however, Ref.~\cite{Kumar:2018kmr} for further discussion regarding the implications of this bound. The dashed grey contours denote the width as a fraction of the mass. We see that the HL-LHC will only be sensitive to $Z^\prime$ with narrow width, up to masses of 5 TeV. 

\begin{figure}[t]
\centering
\includegraphics[width=0.39 \textwidth]{section10/img/27TeVMUM.pdf}~~~~~~~~~
\includegraphics[width=0.39 \textwidth]{section10/img/27TeVMDM.pdf}
\caption{ HE-LHC 95\% (blue) and 99\% (orange) CL sensitivity contours to $Z^{\prime}$ in the ``mixed-up'' (left) and ``mixed-down'' (right) model in the parameter space of $g_{\mu\mu}$ vs $Z^{\prime}$ mass in TeV. The dashed grey contours are the width as a fraction of mass. The green and red regions are excluded by trident neutrino production and $B_s$ mixing. The dashed blue line is the stronger $B_s$ mixing constraint from Ref.~\cite{Bazavov:2016nty} (Figs. taken from~\cite{Allanach:2018odd}). \Blu{See also Section (XXX-->Cambridge group)}}
\label{fig:Zprime27TeV}
\end{figure}

At the HE-LHC, the reach for 10 ab$^{-1}$ is shown in Fig.~\ref{fig:Zprime27TeV} for the MUM and MDM on the left and right, respectively. In this case the sensitivity may reach a $Z^{\prime}$ with wider widths up to 0.25 and 0.5 of its mass, while the mass extends out to 10 to 12 TeV. We stress that this is a pessimistic estimate of the projected sensitivity, particular to the two minimal models; more realistic scenarios will typically be easier to discover. 


\vspace{0.2cm}

\noindent  $\bullet$ \textit{Explanations at the one-loop level:}\\
It is possible to accomodate the $b \to s \ell^+ \ell^-$ anomalies even if mediators only enter at one-loop.  One possibility is the mediators coupling to right-handed top quarks and to muons~\cite{Celis:2017doq,Becirevic:2017jtw,Kamenik:2017tnu,Fox:2018ldq,Camargo-Molina:2018cwu}.   Given the loop and CKM  suppression of the NP contribution to the $b \to s \ell^+ \ell^-$ amplitude, these models can explain the $b \to s \ell^+ \ell^-$ anomalies for a light mediator, with mass around $\mathcal{O}(1)$~TeV or lighter.       Constraints from the LHC and future projections for the HL-LHC were derived in~\cite{Camargo-Molina:2018cwu} by recasting di-muon resonance, $pp\to t\bar tt \bar t$ and SUSY searches.  Two scenarios were considered: \textit{i)} a scalar LQ $R_2(3, 2, 7/6)$ combined with a vector LQ $\widetilde U_{1 \alpha}(3, 1, 5/3)$, \textit{ii)} a vector boson $Z^{\prime}$ in the singlet representation of the SM gauge group.    Ref.~\cite{Fox:2018ldq} also analyzed the HL-LHC projections for the $Z^{\prime}$.     The constraints from the LHC already rule out part of the relevant parameter space and the HL-LHC will be able to cover much of the remaining regions.    Dedicated searches in the $pp\to t\bar tt \bar t$ channel and a dedicated search for $t \mu$ resonances in $t \bar t \mu^+ \mu^-$ final state can improve the sensitivity to these models~\cite{Camargo-Molina:2018cwu}. 

\subsubsection{Conclusions regarding high \pt probes of flavour anomalies}

The anomalous results in $B$-meson decays cannot be considered yet as a convincing evidence of New Physics. On the other hand, the number (and quality) of observables that are not in complete agreement with the SM prediction is growing with time in a coherent way. If true, the implications for HEP will be profound. 
%of the flavour anomalies will have a profound impact for the physics case of the future strategies for the HEP community. 
The conclusions we draw are the following:
\begin{itemize}
\item[$\bullet$] A conservative argument based on perturbative unitarity \cite{DiLuzio:2017chi} sets an upper bound on the New Physics scale to be 9 TeV for charged current anomalies and 80 TeV for neutral current ones. 
%This does not allow to formulate a no-loose theorem for the high-luminosity program. However from 
The analysis of explicit models show that the high-luminosity program has a clear potential to probe a large portion of the possible BSM options. 

\item[$\bullet$] The explanation of the anomalies in $b \to c \tau \nu$ transitions requires non trivial model building. In particular, it is not possible to simply isolate the physics that mediate the flavour anomalous transitions. In complete models other signature have to be considered. Typically it would be very difficult to escape direct detection at HL/HE-LHC.

\item[$\bullet$] Even though the naive scale associated with the $b \to s \ell \ell$ anomalies   is much higher than the energy accessible at HL/HE-LHC, in motivated models the flavour suppressions and weak couplings guarantee a large coverage of the parameter space for leptoquark and $Z'$ mediators \cite{Allanach:2018odd}.
\end{itemize}

More generally, the probes of lepton flavour universality such as the ratios of inclusive $\tau^+\tau^-$ vs. $\mu^+\mu^-$ (or $\mu^+\mu^-$ vs. $e^+e^-$) mass distributions, are important, theoretically clean, tests of the SM and are well motivated observables both at HL- and HE-LHC, whether or not the current $B$-meson anomalies become statistically significant.


%\subsubsection{Experimental prospects for high $p_T$ searches at HL-LHC relevant for $B$ anomalies}
\label{sec:exprospects}

%

We give next  the experimental prospects for leptoquark searches, with leptoquarks decaying to the final states relevant for $B$ physics anomalies. 

\subsubsubsection{Prospects for leptoquark searches assuming $t$+$\tau$ and/or $t$+$\mu$ decays}
\label{sec:LQ:t+tau}
The reach of searches for pair production of leptoquarks (LQs) with decays to $t+\mu$ and $t+\tau$ is studied for the HL-LHC with target integrated luminosities of 
$\mathcal{L}_{\text{int}}^{\text{target}} = 300$ and $3000\,\mathrm{fb}^{-1}$~\cite{CMS-PAS-FTR-18-008}. The studies are based on published CMS results of the $t$+$\mu$~\cite{Sirunyan:2018ruf} and $t$+$\tau$~\cite{Sirunyan:2018nkj} LQ decay channels which use data of proton-proton collisions at $\sqrt{s}=13\,\mathrm{TeV}$ corresponding to $\mathcal{L}_{\text{int}} = 35.9\,\mathrm{fb}^{-1}$ recorded in 2016. While the analysis strategies are kept unchanged with respect to the ones in Refs.~\cite{Sirunyan:2018nkj,Sirunyan:2018ruf}, different total integrated luminosities, the higher center-of-mass energy of 14\,TeV, and different scenarios of systematic uncertainties are considered. In the first scenario (denoted ''w/ YR18 syst. uncert.''), the relative experimental systematic uncertainties are scaled by a factor of $1 / \sqrt{f}$, with 
$f=\mathcal{L}_{\text{int}}^{\text{target}}/35.9\,\mathrm{fb}^{-1}$, 
until they reach a defined lower limit based on estimates of the achievable accuracy with the upgraded detector~\cite{Collaboration:2650976}. The relative theoretical systematic uncertainties are halved. In the second scenario (denoted ''w/ stat. uncert. only''), no systematic uncertainties are considered. The relative statistical uncertainties in both scenarios are scaled by $1 / \sqrt{f}$.

Figure \ref{fig:LQpairs:significances} presents the expected signal significances of the analyses as a function of the LQ mass for different assumed integrated luminosities in the ''w/ YR18 syst. uncert.'' and ''w/ stat. uncert. only'' scenarios. Increasing the target integrated luminosity to $\mathcal{L}_{\text{int}}^{\text{target}} = 3000\,\mathrm{fb}^{-1}$ greatly increases the discovery potential of both analyses. The LQ mass corresponding to a discovery at $5\sigma$ significance with a dataset corresponding to $3000\,\mathrm{fb}^{-1}$ increases by more than 500\,GeV compared to the situation at $\mathcal{L}_{\text{int}}^{\text{target}} = 35.9\,\mathrm{fb}^{-1}$, from about 1200\,GeV to roughly 1700\,GeV, in the $\text{LQ}\rightarrow t\mu$ decay channel. For LQs decaying exclusively to top quarks and $\tau$ leptons, a gain of 400\,GeV is expected, pushing the LQ mass in reach for a $5\sigma$ discovery from 800\,GeV to 1200\,GeV. 

\begin{figure}[t]
\centering
\includegraphics[width=0.49\textwidth]{section10/img/LQpairs_Sign_tmu.pdf}
\includegraphics[width=0.49\textwidth]{section10/img/LQpairs_Sign_ttau.pdf}
\caption{Expected significances for an LQ decaying exclusively to top quarks and muons (left) or top quarks and $\tau$ leptons (right).}
\label{fig:LQpairs:significances}
\end{figure}

In Fig.~\ref{fig:LQpairs:limits1d}, the expected projected exclusion limits on the LQ pair production cross section are shown. Leptoquarks decaying only to top quarks and muons are expected to be excluded below masses of 1900\,GeV for $3000\,\mathrm{fb}^{-1}$, which is a gain of 500\,GeV compared to the limit of 1420\,GeV obtained in the published analysis of the 2016 dataset~\cite{Sirunyan:2018ruf}. The mass exclusion limit for LQs decaying exclusively to top quarks and $\tau$ leptons are expected to be increased by 500\,GeV, from 900\,GeV to approximately 1400\,GeV.

\begin{figure}[t]
\centering
\includegraphics[width=0.49\textwidth]{section10/img/LQpairs_Limits_tmu.pdf}
\includegraphics[width=0.49\textwidth]{section10/img/LQpairs_Limits_ttau.pdf}
\caption{Expected upper limits on the LQ pair production cross section at the 95\% CL for an LQ decaying exclusively to top quarks and muons (left) or $\tau$ leptons (right).}
\label{fig:LQpairs:limits1d}
\end{figure}

Figure~\ref{fig:LQpairs:limits2d} shows the expected signal significances and upper exclusion limits on the pair production cross section of scalar LQs allowed to decay to top quarks and muons or $\tau$ leptons at the 95\% CL as a function of the LQ mass and a variable branching fraction $\mathcal{B}(\mathrm{LQ}\rightarrow t\mu) = 1 - \mathcal{B}(\mathrm{LQ}\rightarrow t\tau)$ for an integrated luminosity of $3000\,\mathrm{fb}^{-1}$ in the two different scenarios. For all values of $\mathcal{B}$, LQ masses up to approximately 1200\,GeV and 1400\,GeV are expected to be in reach for a discovery at the 5$\sigma$ level and a 95\% CL exclusion, respectively.

\begin{figure}[t]
\centering
\includegraphics[width = 0.49\textwidth]{section10/img/LQpairs_Sign_2d.pdf}
\includegraphics[width = 0.49\textwidth]{section10/img/LQpairs_Limits_2d.pdf}
\caption{Expected significances (left) and expected upper limits on the LQ pair-production cross section at the 95\% CL (right) as a function of the LQ mass and the branching fraction. Color-coded lines represent lines of a constant expected significance or cross section limit, respectively. The red lines indicate the 5$\sigma$ discovery level (left) and the mass exclusion limit (right).}
\label{fig:LQpairs:limits2d}
\end{figure}


\subsubsubsection{CMS Search for leptoquarks decaying to $\tau$ and $b$}\label{sec:leptoqbtau}

This analysis from CMS~\cite{CMS-PAS-FTR-18-028} presents future discovery and exclusion prospects for singly and pair produced 
third-generation scalar LQs, each decaying to $\tauh$ and a bottom quark. Here, $\tauh$ denotes a hadronically decaying 
$\tau$ lepton. 
The relevant Feynman diagrams of the signal processes at leading order (LO) are shown in Fig.~\ref{LQ3-diagram}.

\begin{figure}[t]
\centering
\includegraphics[width=0.25\textwidth]{\main/section10/img/diag1.pdf}
\hspace{0.2cm}
\includegraphics[width=0.25\textwidth]{\main/section10/img/diag2.pdf}
\hspace{0.2cm}
\includegraphics[width=0.25\textwidth]{\main/section10/img/diag3.pdf}
\caption{Leading order Feynman diagrams for the production of a third-generation LQ in the single production s-channel (left) and the pair production channel via gluon fusion (center) and quark fusion (right).}
\label{LQ3-diagram}
\end{figure}
\vspace{-5pt}

The analysis uses DELPHES~\cite{deFavereau:2013fsa} event samples of simulated pp collisions at a center-of-mass energy of 14 TeV, corresponding to integrated luminosities of 300 and 3000\fbinv. 
The matrix elements of LQ signals for both single and pair LQ production are generated at LO using version 2.6.0 of
$ {\tt MadGraph5\_aMC@NLO}$~\cite{Alwall:2014hca} for $m_{LQ}=$500, 1000, 1500, and 2000 \GeV. The branching fraction $\beta$ of the LQ to a charged
lepton and a quark, in this case LQ$\to \tau b$, is assumed to be $\beta=1$. The unknown Yukawa coupling $\lambda$ of the LQ to a
$\tau$ lepton and a bottom quark is set to $\lambda=1$. The width $\Gamma$ is calculated using $\Gamma = m_{LQ} \lambda^2 /
(16\pi)$~\cite{Plehn:1997az}, and is less than 10\% of the LQ mass for most of the considered search range. The signal samples are
normalized to the cross section calculated at LO, multiplied by a $K$ factor to account for higher order contributions~\cite{Dorsner:2018ynv}.

Similar event selections are used in both the singly and pair produced LQ searches, except for the requirement on the number of jets. 
In both channels, two reconstructed $\tauh$ with opposite-sign charge are required, each with transverse momentum $p_{T,\tau}>50$\GeV\ and a maximum pseudorapidity  ${|\eta_{\tau}|}<2.3$. 
In the search for single production, the presence of at least one reconstructed jet with $p_T > 50\GeV$ is required, while at least
two are required in the search for pair production. Jets are reconstructed with FASTJET~\cite{Cacciari:2011ma}, using the anti-\kt algorithm~\cite{Cacciari:2008gp}, with a distance parameter of 0.4. 

To reduce background due to Drell-Yan (particularly Z$\to\tau\tau$) events, the invariant mass of the two selected $\tauh$,
$m_{\tau\tau}$, is required to be $>95$\GeV.
In addition, at least one of the previously selected jets is required to be b-tagged to reduce QCD multijet backgrounds. 
Finally, an event is rejected if it contains identified and isolated electrons (muons), with $p_T>10$\GeV, $|\eta|<$2.4 (2.5).
The acceptance of the signal events is 4.9\%  (11\%) for single (pair) production, where the branching ratio of two $\tau$ leptons decaying hadronically is included in the numerator of the acceptance.

Signal extraction is based on a binned maximum likelihood fit to the distribution of the scalar $p_T$ sum S$_{\text{T}}$, which 
is defined as the sum of the transverse momenta of the two $\tauh$ and either the highest-$p_T$ jet in the 
case of single LQ production, or the two highest-$p_T$ jets in the case of LQ pair production. These distributions are shown 
in Fig.~\ref{fig:scalarptsums} for the HL-LHC 3000 $\fbinv$ scenario.

\begin{figure}[t]
\centering
\includegraphics[width=0.48\textwidth]{\main/section10/img/scalarpt1sumfinal.png}
\includegraphics[width=0.48\textwidth]{\main/section10/img/scalarpt2sumfinal.png}
\caption{Left: scalar sum of the \pt\ of the two selected $\tau$ leptons and the highest-$p_T$ jet in the single LQ selection region. Right: scalar sum of the $p_T$ of the two selected $\tau$ leptons and the two highest-$p_T$ jets in the LQ pair search region. The considered backgrounds are shown as stacked histograms, while empty histograms for signals for the single LQ and LQ pair channels (for $m_{LQ}=1000$\GeV) are overlaid to illustrate the sensitivity. Both signal and backround are normalised to a luminosity of 3000 $\fbinv$.} %Signal is normalized to $\lambda=1$, $\beta=1$ and $\sigma=1$ pb.}
\label{fig:scalarptsums}
\end{figure}

Systematic uncertainties are 
calculated by scaling the current experimental uncertainties. For uncertainties limited by statistics, including the uncertainty 
on the DY (3.3\%) and QCD (3.3\%) cross sections, a scale factor of $1/\sqrt{L}$ is applied, for an integrated luminosity ratio $L$. 
For uncertainties 
coming from theoretical calculations, a scale factor of 1/2 is applied with respect to current uncertainties, as is the case for 
the uncertainties on the cross section for top (2.8\%) or diboson (3\%) events. Other experimental systematic uncertainties are 
scaled by the 
square root of the integrated luminosity ratio until the uncertainty reaches a minimum value, including uncertainties on the integrated 
luminosity (1\%), 
$\tau$ identification (5\%) and b-tagging/misidentification (1\%-5\%).

Figure~\ref{fig:LQ3-projection} shows an upper limit at 95\% CL on the cross section times branching fraction $\beta$ as a 
function of $m_{LQ}$ by using the asymptotic CLs modified frequentist 
criterion~\cite{Cowan:2010js,Junk:1999kv,Read:2002hq, ATLAS:2011tau}.
Upper limits are calculated considering 
two different scenarios. The first one, hereafter abbreviated as "stat. only" considers only statistical uncertainties, to 
observe how the results are affected by the increase of the integrated luminosity. The second scenario, hereafter 
abbreviated as "stat.+syst.," also includes the estimate of the systematic uncertainties at the HL-LHC. 
For the single LQ production search, the theoretical prediction for the cross section assumes $\lambda = 1$ and $\beta = 1$. 

\begin{figure}[t]
\vspace{-5pt}
\centering
\includegraphics[width=0.49\textwidth]{\main/section10/img/projection_single.pdf}
\includegraphics[width=0.49\textwidth]{\main/section10/img/projection_pair.pdf}
\caption{Expected limits at 95\% CL on the product of the cross section $\sigma$ and the branching fraction $\beta$, as a function
of the LQ mass, for the two high luminosity projections, 300 \fbinv (red) and 3000 \fbinv (orange), for both the stat. only  (dashed
lines) and the stat.+syst. scenarios (solid lines). This is shown in conjunction with the theoretical predictions at NLO~\cite{Dorsner:2018ynv} in cyan. Projections are calculated for both the single LQ (left) and LQ pair production (right).}
\label{fig:LQ3-projection}
\end{figure}

Comparing the limits with theoretical predictions assuming unit Yukawa coupling $\lambda = 1$, third-generation scalar 
leptoquarks are expected to be excluded at 95\% confidence level for LQ masses below 732 (1249) GeV for a luminosity 
of 300 \fbinv, and below 1130 (1518) GeV for 3000 \fbinv in the single (pair) production channel, considering both 
statistical and systematic uncertainties. 

Since the single-LQ signal cross section scales with $\lambda^2$, it is straightforward to recast the results presented in 
Fig.~\ref{fig:LQ3-projection} in terms of expected upper limits on $m_{LQ}$  as a function of $\lambda$, as shown in 
Fig.~\ref{fig:lambda_mass}. 
The blue band shows the parameter space (95\% CL) for the scalar LQ preferred by the $\PB$ physics 
anomalies: $\lambda = (0.95 \pm 0.50) m_{LQ} (\TeV)$~\cite{Buttazzo:2017ixm}. 
For the 300 (3000) \fbinv luminosity scenario, the leptoquark pair production channel is more sensitive if  
$\lambda<2.7$ $(2.3)$, while the single leptoquark production is dominant otherwise.
 
\begin{figure}[t]
\centering
\includegraphics[width=0.5\textwidth]{\main/section10/img/exclusion.pdf}
\caption{Expected exclusion limits at 95\% CL on the Yukawa coupling $\lambda$ at the LQ-lepton-quark vertex, as a function of the
LQ mass. A unit branching fraction $\beta$ of the LQ to a $\tau$ lepton and a bottom quark is assumed. Future projections for 300
and 3000 \fbinv are shown for both the stat. only and stat.+syst. scenarios, shown as dashed and
filled lines respectively, and for both the single LQ and LQ pair production, where the latter corresponds to the vertical line
(since it does not depend on $\lambda$). The left hand side of the lines  represents the exclusion region for each of the
projections, whereas the region with diagonal blue hatching shows the parameter space preferred by one of the models proposed to
explain anomalies observed in $\PB$ physics~\cite{Buttazzo:2017ixm}.}
\label{fig:lambda_mass}
\end{figure}

Using the predicted cross section~\cite{Dorsner:2018ynv} of the signal, it is also possible to estimate the maximal LQ mass expected to be in reach for a 5$\sigma$ discovery. 
Figure~\ref{fig:LQ3-discovery} shows the expected local significance of a signal-like excess as a function of the LQ mass hypothesis.

\begin{figure}[t]
\vspace{-5pt}
\centering
\includegraphics[width=0.49\textwidth]{\main/section10/img/significance_LQ.pdf}
\includegraphics[width=0.49\textwidth]{\main/section10/img/significance_LQLQ.pdf}
\caption{Expected local significance of a signal-like excess as a function of the LQ mass, for the two high luminosity projections,
300 \fbinv (red) and 3000 \fbinv (orange), assuming the theoretical prediction for the LQ cross section at NLO~\cite{Dorsner:2018ynv}, calculated with $\lambda = 1$ and $\beta = 1$. Projections are calculated for both single LQ (left) and LQ pair production (right).}
\label{fig:LQ3-discovery}
\end{figure}

In summary, this study shows that future LQ searches under higher luminosity conditions are promising, as they are expected 
to greatly increase the reach of the search. They also show that the pair production channel is  expected to be the most 
sensitive. A significance of 5$\sigma$ is within reach for LQ masses below 800 (1200) FeV for the single (pair) production 
channels in the 300\fbinv scenario and 1000 (1500) GeV for the 3000 \fbinv scenario.


\subsubsubsection{CMS Searches for $W^\prime \to \tau + \met$}

New heavy gauge bosons are predicted by various SM extensions. The charged version of such heavy gauge bosons
is generally referred to as \PWpr.
The decay $\PWpr \rightarrow \tau \nu$ yields a single hadronically decaying tau
($\tau_h$) as the only detectable object and missing energy due to the neutrinos.
Hadronically decaying tau leptons are selected since the corresponding branching fraction,
about 60\%, is the largest among all $\tau$ decays.
Tau-jets are experimentally distinctive because of their low charged hadron
multiplicity, unlike QCD multi-jets,
which have high charged hadron multiplicity, or other leptonic \PWpr boson decays, which yield no jet. 
This Phase-2 study~\cite{CMS-PAS-FTR-18-030} follows closely the recently published Run~2 result~\cite{Sirunyan:2018lbg},
using hadronically decaying tau leptons.

The signature of a \PWpr boson (see Fig.~\ref{fig:feynmanWprime}), is considered
similar to a high-mass W boson. 
It could be observed in the distribution of the
transverse mass ($M_T$) of the transverse momentum of the $\tau$ ($p_T^{\tau}$)
and the missing transverse momentum (\ptmiss):
$M_T = \sqrt{2 p_T^{\tau}p_T^{miss}(1-cos\Delta\phi(\tau,p_T^{miss}))}$.
Unlike the leptonic search channels, the signal shape of $W^\prime$ bosons with hadronically
decaying tau leptons does
not show a Jacobian peak structure because of the presence of two neutrinos in the final
state. Despite the multi-particle final state,
the decay appears as a typical two-body decay; the axis of the
hadronic tau jet is back to
back with \ptmiss and the magnitude of both is comparable such that their ratio is about unity.


\begin{figure}[htp]
\begin{center}

\includegraphics[width=.45\textwidth]{\main/section10/img/WprimeToTau.pdf}
\includegraphics[width=.49\textwidth]{\main/section10/img/MT_final_3000Wprime.pdf}

\caption{Left: Illustration of the studied channel $W^\prime \rightarrow \tau \nu$ with the
subsequent hadronic decay of the tau ($\tau_h$).
Right: The discriminating variable, $M_T$, after all selection criteria for the HL-LHC conditions of
\intLumi and 200~PU.
The relevant SM backgrounds are shown according to the labels in
the legend. Signal examples for \PWpr boson masses of 1\TeV, 5\TeV and 6\TeV are scaled to their SSM
LO cross section and \intLumi.}
\label{fig:feynmanWprime}
\end{center}
\end{figure}


The results are interpreted in the context of the sequential standard model (SSM)
~\cite{Altarelli:1989ff} in terms of \PWpr mass and coupling strength. 
A model-independent cross section limit allows interpretations in other models.
The signal was simulated in LO and the detector performance simulated with \delphes{}.
The \PWpr boson coupling strength, $g_{\PWpr}$, is given in terms of the SM weak coupling strength $g_{\PW} = e/\sin^2\theta_W\approx 0.65$.
Here, $\theta_{\PW}$ is the weak mixing angle.
If the \PWpr boson is a heavier copy of the SM \PW\ boson, their coupling ratio is $g_{\PWpr}/g_{\PW}=1$ and the 
SSM \PWpr boson theoretical cross sections, signal shapes, and widths apply.
However, different couplings are possible. Because of the dependence of the width of a particle on its couplings
the consequent effect on the transverse mass distribution, a limit can also be set on the coupling strength.

The dominant background appears in the off-shell tail of the $M_T$ distribution of
the SM \PW\ boson.
Subleading background contributions arise from $t\bar{t}$ and QCD multijet events. The number of background events
is reduced by the event selection.
These backgrounds primarily arise as a consequence of jets misidentified as $\tau_h$ candidates
and populate the lower transverse masses while the signal exhibits an excess of events at high $M_T$.
Events with one hadronically decaying $\tau$
and \ptmiss are selected if the ratio of $\pt^{\Pgt}$ to \ptmiss satisfies $0.7 < \pt^{\Pgt}/\ptmiss <
1.3$ and the
angle $\Delta\phi(\vec{p_T}^{\Pgt}, \vec{p_T}_{miss})$ is greater than 2.4 radians.

The physics sensitivity is studied based on the $M_T$ distribution in Fig.~\ref{fig:feynmanWprime}-right.
Signal events are expected to be particularly prominent at the upper end of the $M_T$
distribution, where the expected SM background is low.
So far, there are no indications for the existence of a SSM \PWpr\ boson~\cite{Sirunyan:2018lbg}. With the
high luminosity during Phase-2, the \PWpr mass reach for potential observation increases
to 6.9\TeV\ and 6.4\TeV\
for 3~$\sigma$ evidence and 5~$\sigma$ discovery, respectively, as shown in Fig.~\ref{fig:WprimeResults}-left. Alternatively, in case of no observation, one can exclude SSM \PWpr boson masses up to 7.0\TeV
with \intLumi.
These are multi-bin limits taking into account the full $M_T$ shape.

While the SSM model assumes SM-like couplings of the fermions, the couplings could well be weaker if
further decays occur. The HL-LHC has good sensitivity to study these couplings.
The sensitivity to weaker couplings extends significantly.
A model-independent cross section limit for new physics with $\tau$+MET
in the final state is depicted in Fig.~\ref{fig:WprimeResults}-right, calculated as a single-bin limit
by counting the number of events above a sliding threshold $M_T^{\rm min}$.

\begin{figure}[hbtp]\centering
    \includegraphics[width=0.49\textwidth]{\main/section10/img/SignificanceWprime.pdf}
    \includegraphics[width=0.49\textwidth]{\main/section10/img/MILimit_3000fbinvWprime.pdf}
\caption{Left: Discovery significance for SSM \PWpr to tau leptons.
Right: Model-independent cross section limit. For this, a single-bin limit is
calculated for increasing $M_T^{\rm min}$
while keeping the signal yield constant in order to avoid including any signal shape
information on this limit calculation.}
    \label{fig:WprimeResults}
\end{figure}








\end{document}
