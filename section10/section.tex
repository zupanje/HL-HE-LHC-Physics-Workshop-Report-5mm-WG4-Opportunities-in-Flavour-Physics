% don't remove the folling lines, and edit the defintion of \main if needed
\documentclass[../report.tex]{subfiles}
\providecommand{\main}{..}
\IfEq{\jobname}{\currfilebase}{\AtEndDocument{\biblio}}{}
% until here

\begin{document}

\section{The high $p_T$ flavour physics program} (anticipated length 15-20 pages)

\td{need to write a general introduction} Flavour and high $p_T$ searches are intertwined in several ways. On the one hand the stringent bounds from low energy constraints put severe bounds on the NP models that contain states with TeV masses with couplings to quarks. Quite often these bounds are avoided by assuming Minimal Flavour Violation (MFV), where the only flavour breaking, even in the NP sector, is due to the SM Yukawa matrices. However, more general flavour structures for the NP states are still allowed. In fact such non-MFV couplings can have interesting consequences. In general we can group the NP models into two broad classes: (i) NP models that aim to explain the SM flavor puzzle, and (ii) models that solve one of the outstanding problems particles physics, e.g., the origin of dark matter, or the hierarchy problem, but with nontrivial flavor structure in NP sector that leads to modified high $p_T$ searches. Such a nontrivial flavour structure is especially motivated in the models of NP that aim to explain the recent anomalies. 
%(not too many of these have states at TeV, though)

In the rest of this section we briefly review... \td{say what was done}

\subsection{Models of flavour and TeV Physics}
{\bf Author(s): Martin Bauer, Adrian Carmona, Geraldine Servant?} (anticipated length 3 pages)
%\subsubsection{Overview of TeV scale flavour models} 

\td{quickly introduce the models}

\subsubsection{\bf Randall-Sundrum models of flavour}
\td{to be done}

\subsubsection{\bf A Clockwork solution to the flavour puzzle}
%\td{need to shorten}
{\bf Author: Adrian Carmona}
The clockwork mechanism, introduced in \cite{Choi:2015fiu, Kaplan:2015fuy} and later generalized to a broader context in Ref.~\cite{Giudice:2016yja}, provides a natural mechanism for obtaining large hierarchies in couplings or mass scales. Ref. \cite{Alonso:2018bcg} showed that it can also be used to generate the observed hierarchy of quark masses and mixing angles with anarchic Yukawa couplings, providing thus a solution to the flavor puzzle. 

In clockwork solution to the flavour puzzle, each SM chiral fermion $\psi$ is accompanied with a $N_\psi$-node chain of vector-like fermions, $\psi_{L,j},\psi_{R,j}$, with masses $m$, and
%In the simple case of clockworking just a right-handed chiral field $\psi_R\equiv \psi_{R,0}$, one adds a $N$-node chain of vector like fermions $\psi_{L,j},\psi_{R,j}$, with masses $m\bar{\psi}_{L,j}\psi_{R,j}$, as well as 
a series of nearest neighbour mass terms, $qm\bar{\psi}_{L,j}\psi_{R,j-1}$, between the nodes, where  $j=1,\ldots,N_\psi$. The mass spectrum of each chain has one chiral zero mode, the would-be SM fermion, 
%chains of fermions $\psi_{R,j}$ and $\psi_{L,j}$  carry the same gauge quantum numbers as $\psi_{R,0}$. The covariant derivatives are thus the same for fermions on all the nodes. After diagonalization, one obtains a chiral massless state, $\psi_{R,0}^{\prime}$, 
and $N_\psi$  heavy Dirac fermion mass-eigenstates -- the gears, with masses $M_{k}^2=m^2(1+q^2-2q\cos(\frac{k\pi}{N_\psi+1}))$.
For $q\gg 1$,  the spectrum of the gears is compressed in a $2m$ band around $qm$, with $(M_{N_\psi}-M_1)\ll M_1$. The massless zero-mode interacts with the SM Higgs, which is on the $0$-th node, through a set of Yukawa interactions described by ${\mathcal O}(1)$ Yukawa matrices, $Y_{U,D}$. 
The component of the massless mode on the $0$-th node is, 
%$\psi_{R,0}^{\prime}$ has a component $f_{\psi}$ on the $0$-th node which, 
in the large $q$ limit, given by $1/q^{N_\psi}$. This suppression is the origin of the SM Yukawa hierarchies, \td{check Higgs section notation}
%In order to explain the hierarchy of SM quark masses through the clockworking mechanism we introduce for each of the SM fermions, $\psi_i$, where $i=1,2,3$, is the generation index, an $N_{\psi_i}$-node chain of 
%vector-like fermions with the same quantum numbers. In addition, the SM Higgs resides on the $0$-th node, coupling the fermions on the $0$-th nde through Yukawa interactions.  After electroweak symmetry breaking the Yukawa interactions lead to a mass term for the zero modes, which  are identified with the SM fermions. To leading order in $v^2/M^2$ expansion, the SM Higgs Yukawa matrices are given by the products of zero mode overlaps with the $0$-th node, $f_\psi$,
\begin{align}
\label{eqa:YuSM}
\left(Y_u^{\rm SM}\right)_{ij}&\sim q^{-N_{Q(i)}}_{Q(i)}\left(Y_U\right)_{ij}q^{-N_{u(j)}}_{u(j)},  \quad \left(Y_d^{\rm SM}\right)_{ij} \sim q^{-N_{Q(i)}}_{Q(i)}\left(Y_D\right)_{ij}q^{-N_{d(j)}}_{d(j)}.
\end{align}
The hierarchy of quark masses can be then naturally obtained for anarchical $Y_U$ and $Y_D$ Yukawa matrices if $q^{-N_{Q(i)}}\ll q^{-N_{Q(j)}}$, $q^{-N_{u(i)}}_{u(i)}\ll q^{-N_{u(j)}}_{u(j)}$, $q^{-N_{d(i)}}_{d(i)}\ll q^{-N_{d(j)}}_{d(j)}$, for $i<j$ (in the benchmark below we take $q_i$ to be universal and equal to $q$).
% For the sake of concreteness,  we will only consider the clockwork universal-$q$ limit, which occurs  when all the clockwork factors are the same, $q\equiv q_{Q(i)}=q_{u(i)}=q_{d(i)}\sim {\mathcal O}({\rm few})$, while $N_{Q(1)}\gg N_{Q(2)} \gg N_{Q(3)}$   and similarly for up and down right-handed quarks.\footnote{In particular, we consider  $q=1/\lambda$, with $\lambda=|V_{us}|\simeq0.23$ as well as $N_{Q(1)}=3, N_{Q(2)}=2$, $N_{u(1)}=4$, $N_{u(2)}=1$, $N_{d(1)}=4$, $N_{d(2)}=3$, $N_{d(3)}=2$ and $N_{Q(3)}=N_{u(3)}=0$.} The large multiplicity of colored fermionic degrees of freedom  destroys asymptotic freedom for the strong interaction, putting an upper bound on the maximum number of  gears.  On the other hand, quark loops will also provide a negative contribution to the one-loop beta function of the Higgs quartic self-coupling $\lambda |H|^4/2$. Requiring that the beta function remains perturbative constraints the Yukawa entries in $Y_U$ and $Y_D$ to be smaller than 1,  with the possible exception of $\left(Y_U\right)_{33}$.

%Clockworking  a  left-handed chiral fermion, $\psi_L$, proceeds along exactly the same lines, but exchanging  $L\leftrightarrow R$ everywhere. 

%Unlike the zero mode, the profiles of the gears are not exponentially suppressed on the $0$-th node, even when $q,N\gg 1$. 



%If one integrates out all the new heavy particles, one can use the generated dimension-6 operators of the SM effective field theory  to analyze the constraints from low-energy experiments: from weak boson decays, rare meson decays and neutral meson mixings. 
The clockwork models of flavor are endowed with a powerful flavor protection against FCNCs, very similar to the RS models.  The FCNCs with light quarks on the external legs are suppressed by the small overlaps of the zero-modes, which is the same suppression that gives rise to hierarchies between the SM quark masses. This clockwork-GIM  mechanism, along with the constraints on $Y_{U,D}$ arising from the stability of the Higgs potential,   suffices to alleviate the flavor constraints to the level that TeV scale gear masses are compatible with experimental bounds~\cite{Alonso:2018bcg}.

TeV scale gears
%, as allowed by the present low energy constraints, 
can be searched for at the LHC, where they are produced through QCD pair production.
% and at future high energy colliders. 
%The main production channel for the gears is the QCD pair production with the corresponding cross sections precisely calculable~\cite{Czakon:2017wor}. 
The collider signatures
%, on the other hand, do 
depend on the gear decay patterns. The gears decay predominantly through their coupling to the Higgs doublet into gears from a different-chirality chain.
The lightest gears decay directly to SM fermions, mostly $t$ and $b$, via the emission of $W,Z$ or $h$, as do heavier gears for which these are the only kinematically allowed channels. 
%Given the overlap suppression the decays are predominantly to $t$ and $b$. 
The main existing collider constraints 
%on clockwork flavor models are  expected to arise 
are from searches for pair production of vector-like quarks, in final states involving third generation SM quarks. Ref.~\cite{Alonso:2018bcg} found the two
%In particular, we find the searches for down-like gears decaying to the $tW$ channel, as well as searches for up-like gears decaying to the $tH$ and $tZ$ final states, to be most sensitive.  To perform a more detailed analysis, we recast the recent 
35 fb${}^{-1}$ 13 TeV ATLAS 
searches for vector-like quarks decaying into 
$tW$ final states~\cite{Aaboud:2018uek}, as well as the analogous search employing the 
$tZ$ and $tH$ final states~\cite{Aaboud:2018xuw}, to be currently most sensitive, see 
%, both using 35 fb${}^{-1}$ of LHC data at 13~TeV for the two benchmarks defined in Ref.~\cite{Alonso:2018bcg}. The results for the last of these benchmarks are shown in the left panel of 
Fig.~\ref{fig:both}.

\begin{figure}
\begin{center}
\includegraphics[width=0.4\textwidth]{\main/section10/img/GearXsec_all_fn.pdf}
	\includegraphics[width=0.59\textwidth]{\main/section10/img/GearSpheres_all_fn.pdf}
\caption{	Left: The total gear pair production cross-sections in the final states $tW+X$, $tH+X$ and $tZ+X$ for the benchmark model from Ref. \cite{Alonso:2018bcg}, with contributions from individual gears shown stacked.
% and ordered top down by their increasing mass (decreasing cross-section). 
The currently most stringent upper bounds~\cite{Aaboud:2018uek,Aaboud:2018xuw} are denoted with dashed lines (for the $tZ+X$~\cite{Aaboud:2018xuw} final state the bound is too weak to be shown). 
%, obtained by recasting of searches for vector-like quarks in the $tW+X$~\cite{Aaboud:2018uek} and $tH+X$ (the 1-lepton channel)~\cite{Aaboud:2018xuw} final states, are denoted with dashed lines. The corresponding bound on the $tZ+X$~\cite{Aaboud:2018xuw} final state is much weaker and is not shown. 	
Right: Invariant mass spectrum of individual pseudojets clustered using the hemisphere algorithm applied to partonic gear pair production and decay at the 13 TeV LHC. The original hemisphere clustering results are shown in light gray, and the modified hemisphere clustering results in mid-gray (dark gray if in addition 
%. Finally, the modified hemisphere clustering results, where in addition 
the masses of the two pseudojets are required to differ by less than 30\%)
%,  are shown in dark gray. 
}
	\label{fig:both}
\end{center}
\end{figure}




The dense spectrum of gears and the potentially complex pattern of gear decays poses an experimental challenge.
% also in the case a signal is discovered. 
In the conventional vector-like quark searches the clockwork signal will appear as an excess of events with high transverse energies or $H_T$, but without a dominant single peak in the invariant mass of any particular final state, such as $tH$ or $tW$. Ref.~\cite{Alonso:2018bcg} proposed a novel reconstruction strategy targeting pair production of heavy quarks with a-priori unknown but potentially long decay chains that result in a single heavy flavored quark, $t$ or $b$, plus any number of massive weak or Higgs bosons per decay chain. The proposed search strategy uses a modified hemisphere clustering algorithm, with $t-$ and $b-$tagged jest as seeds for clustering into exactly two pseudo-jets (the invariant mass of these is shown as mid-gray distribution in Fig.~\ref{fig:both} right). The original hemisphere clustering uses instead the jets with highest invariant mass as seeds and shows no sharp features (light gray). Requiring that the masses of the two pseudo-jets differ by less then 30\% gives the dark grey distribution, with clearly visible gears (in the exploratory study of~\cite{Alonso:2018bcg} tops, $b$-quarks, $W$, $Z$ and the Higgs were not decayed).

%Our procedure is based on the so-called hemisphere clustering algorithm, defined in Section 13.4 of Ref.~\cite{Ball:2007zza}. Jets, as well as isolated leptons and photons, are clustered into exactly two pseudojets, where the clustering is performed by minimizing the Lund distance measure~\cite{Sjostrand:1982am}. The original hemisphere algorithm is seeded by the two objects with the largest combined invariant mass. Since each gear decay chain results in exactly one heavy flavored quark ($t$ or $b$) we instead seed our algorithm with $t$- and $b$-tagged jets. The idea is that, at least for the moderately boosted pair produced gears, the two pseudojets will predominatly capture the decay products of the individual gears. Finally, we select events, where the invariant masses of the two pseudojets are comparable. 
%
%We demonstrate the usefulness of our procedure by simulating gear production and decay for the benchmark model at parton level (where tops, $b$-quarks, $W$, $Z$ and the Higgs are not decayed). The results are shown in the right panel of Fig.~\ref{fig:both}, where we plot the invariant mass distributions of individual pseudojets and overlay the spectral lines of the gears in the benchmark model. The resulting spectrum with the original hemisphere clustering does not exhibit any sharp features, with the bulk of the invariant mass distribution lying well below the mass of the lightest gear. The invariant mass distribution for the modified hemisphere algorithm already exhibits clear spectral line features. The pseudojets with masses of the top and $b$-quarks are abundantly identified, but also those of a few lowest lying gears. Finally, when one only keeps the events for which the masses of the two pseudojets differ by less than 30\%, the low invariant mass peaks corresponding to the pseudojets containing only a single top or $b$-quark  are rejected by this requirement. In addition, the gear peaks are even more pronounced after this cut, with little loss in the number of signal events in the peaks. 

\subsubsection{U(2) models of flavour}
\td{to be done}

\subsubsection{Low scale flavour gauge symmetries}
\td{to be done}

\subsection{Implications for high $p_T$ new physics searches} 
{\bf Author(s): Martin Bauer, Adrian Carmona} (anticipated length 2 pages)
\td{a few sentences to tie over}

\subsubsection{Top decays to exotica} 
{\bf Authors: S. Banerjee, M. Chala, M. Spannowsky} 
%(4 pages) \td{JZ: should be less, probably}
% S. Banerjee, M. Chala, M. Spannowsky



FCNC mediated processes are rare within the SM. However, LHC is a top factory, producing top quarks in copious amounts,  significant number of events are expected even for top decays with very small branching ratios. In light of this fact, studies of top FCNC decays to SM particles have garnered a strong interest in the community~\cite{Agashe:2009di,Mele:1998ag,Greljo:2014dka,Azatov:2014lha,Khachatryan:2015att,Botella:2015hoa,Bardhan:2016txk,Badziak:2017wxn,Khachatryan:2016sib,Gabrielli:2016cut,CMS-PAS-TOP-17-017,Aaboud:2018nyl,Aaboud:2017mfd,Aaboud:2018pob,Papaefstathiou:2017xuv}.

\td{need to say something more about classic $t\to c Z$, etc signals, i.e., reference to section \ref{sec:top:quark}, once that is finished}

In presence of NP other exotic top FCNC decays are possible. We highlight one such possibility, where the NP spectrum contains a light 
%extend previous works by studying the top decay
%to a 
scalar singlet, $S$, with mass $m_S$ below the top quark mass.
%around the Electroweak (EW) scale and an up-type light quark.
Such scalar
particle is predicted in a number of well-motivated NP models, e.g., in the NMSSM~\cite{Ellwanger:2009dp} and in
non-minimal composite Higgs models~\cite{Dimopoulos:1981xc,Kaplan:1983fs,Kaplan:1983sm,Panico:2015jxa}. Moreover, quite often the induced $t\to cS, uS$ FCNC decays are easier to probe than, for instance the ones involving the SM Higgs, $t\to ch, uh$~\cite{Zhang:2013xya}. 
%such a scalar can usually induce FCNCs to a degree significantly larger than what can be mediated by an SM-like Higgs 
%boson
The reason is three-fold; \textit{(i)} The top FCNCs mediated by 
$S$ are usually suppressed by one less power of the heavy physics scale; \textit{(ii)} in principle, $S$ can possibly have a larger decay width into cleaner final states, such as $\ell^+\ell^-$, $b\overline{b}$ 
or $\gamma\gamma$; \textit{(iii)} $m_S$ can be much smaller than the Higgs mass, reducing the 
%corresponding top decay
%being therefore 
phase space suppression. %Thus, top FCNCs mediated by new singlet scalars might be seen at the 
%future runs of the LHC.

There are no direct experimental limits on $t \to q S$ from colliders. The indirect constraints from 1-loop box diagrams in 
$D^0-\bar{D}^0$ oscillations constrain the products of two $S$ Yukawas, $\tilde Y_{ut} \tilde Y_{ct(tc)}$, and $\tilde Y_{tu} \tilde Y_{ct(tc)}$, to be small~\cite{Bona:2007vi,Harnik:2012pb,Agashe:2013hma}. This means that $S\bar t c$ or $S\bar t u$ couplings can be sizeable, but not both at the same time. Therefore, inspired by the CMS $t\to hc$ search~\cite{CMS:2017cck}, authors of Ref.~\cite{Banerjee:2018fsx} developed a dedicated analyses for $t\to qS$, by varying $m_S$. The projections for  $S \to b\bar{b}$ and $S \to \gamma \gamma$ decays, at the 14 TeV LHC, are shown in Figs. \ref{fig:sbb2} and \ref{fig:sgg2}, respectively.  Note that very light $S$, i.e., with $m_S < m_h/2\sim 62.5$ GeV, need not be
excluded by the LHC constraints on the Higgs width, $\Gamma(h\rightarrow S S) \lesssim 10$ MeV ~\cite{Khachatryan:2016ctc}. Indeed, 
for a quartic coupling $\lambda_{HS} S^2 |H|^2$, 
%we have
%$\Gamma(h\rightarrow S S)\sim \lambda_{HS}^2/(32\pi) \times v^2/m_h$, so we can evade the aforementioned 
this bound is avoided for $\lambda_{HS} < 0.05$.

%However, one can have constraints from flavour 
%observables, \textit{viz.}, $D^0-\bar{D}^0$ oscillations~\cite{Bona:2007vi,Harnik:2012pb,Agashe:2013hma}. These
%constraints are always in the form of a product of two $S$ Yukawas, $Y_{ct}$ and $Y_{ut}$ (and also $Y_{uc}$). One 
%can always fall back upon scenarios where $Y_{ut}$ is negligibly small, rendering these constraints unimportant.
%%
%Therefore, inspired by the $t\to hc$ search by CMS~\cite{CMS:2017cck}, we develop dedicated analyses for $t\to qS$,
%by varying $m_S$.
%We first discuss the model-independent framework and then discuss the analyses and results in two different 
%final states, \textit{viz.}, $S \to b\bar{b}$ and $S \to \gamma \gamma$ at the 14 TeV LHC. Finally, we quote naive 
%estimations at $\sqrt{s} = 27$ TeV and $\sqrt{s} = 100$ TeV. \\ \\
%
%{\bf Effective Lagrangian} \\

Assuming that $S$ is the only light NP state, its couplings to the SM quarks can be written in terms of a dimension 5 effective operator (not displaying generational indices)
\begin{equation}\label{eq:lag}
 \mathcal{L} = -{\bar{q}_L}{\tilde{Y}}\frac{S}{f}  \tilde{H}{u_R} + \text{h.c.}\supset \tilde{g} \frac{m_t}{f} \bar{t}_L S c_R + \text{h.c.},
\end{equation}
where $f$ is the NP scale, and on r.h.s. we introduced a new flavor violating coupling $\tilde g$. The three Benchmark Points (BP) shown in Figs. \ref{fig:sbb2} and \ref{fig:sgg2} are
%, each including $m_S = 20, 50, 80, 100, 120$ and $150$ GeV, are as follows
%
\begin{equation}
\begin{split}
 \text{BP 1}: \quad \tilde{g} = 1.0~, \quad f =~\, 2~\text{TeV}\quad\Longrightarrow\quad\mathcal{B}(t\rightarrow Sc)\sim 10^{-3}-10^{-2}~;\\
 \text{BP 2}: \quad \tilde{g} = 1.0~, \quad f = 10~\text{TeV}\quad\Longrightarrow\quad\mathcal{B}(t\rightarrow Sc)\sim 10^{-4}-10^{-3}~;\\
 \text{BP 3}: \quad \tilde{g} = 0.1~, \quad f =~\, 2~\text{TeV}\quad\Longrightarrow\quad\mathcal{B}(t\rightarrow Sc)\sim 10^{-5}-10^{-4}~. 
\end{split}
\end{equation}
%
%We study a scenario where the SM Higgs sector is extended by a gauge singlet, $S$, with mass $m_S$. 
%At low energies, the relevant Yukawa Lagrangian can be written as~
%
%\begin{equation}\label{eq:lag}
% \mathcal{L} = -\mathbf{\overline{q_L}}\bigg(\mathbf{Y} + \mathbf{Y'}\frac{|H|^2}{f^2} + \mathbf{\tilde{Y}}\frac{S}{f}\bigg)  \tilde{H} \mathbf{u_R} + \text{h.c.}~,
%\end{equation}
%%
%where $H = [\phi^+, (h + \phi^0)/\sqrt{2}]^t$, is the SM-like Higgs doublet, $\mathbf{q_L}$ ($\mathbf{u_R}$) denotes 
%the left-handed (right-handed) quarks, $\mathbf{Y}, \mathbf{Y'}, \; \textrm{and} \; \mathbf{\tilde{Y}}$ are arbitrary 
%flavour matrices, $v\sim 246$ GeV, is the Higgs vacuum expectation value (\textit{vev}) and $f\gtrsim 
%\mathcal{O}(\textrm{TeV})$ is the new physics scale. Normally, the flavour matrices are not aligned, and hence the
%FCNCs can arise in the EW phase. This induces various new physics effects including the top flavour-violating decays, 
%\textit{viz.}, $t\rightarrow h c$ or $t\rightarrow S c$. As discussed before, the latter dominates. (In CHMs, Higgs mediated
%top FCNCs are even absent in first approximation~\cite{Agashe:2009di}, \textit{i.e.} $\mathbf{Y}\sim\mathbf{Y^\prime}$, whereas this does not need to be the case for $S$~\cite{Banerjee:2018fsx}).
%
%After the EWSB, one obtains
%%
%\begin{align}
%\label{lag:lag1}
% \mathcal{L} &= -\frac{v}{\sqrt{2}}\bigg[\mathbf{\overline{q_L}}\mathbf{Y}\left(1+\frac{h}{v}\right)\mathbf{u_R} + \frac{S}{f}\mathbf{\overline{q_L}}\mathbf{\tilde{Y}}\mathbf{u_R} + \mathcal{O}\left(\frac{1}{f^2}\right)\bigg]\supset \tilde{g} \frac{m_t}{f} \overline{t_L} S c_R + \text{h.c.},
%\end{align}
%%
%where $m_t\sim 173$ GeV is the top mass and $g$ is an $\mathcal{O}(1)$ coupling. The partial width of $t \to Sc$ 
%is
%%
%\begin{equation}
% \Gamma(t\rightarrow Sc) = \frac{\tilde{g}^2}{32\pi}\frac{v^2}{f^2}m_t\bigg(1-\frac{m_S^2}{m_t^2}\bigg)^2.
%\end{equation}
%
If the flavor conserving couplings of $S$ to the SM fermions, $\psi$, as ${c_{\psi} m_\psi}
S(\bar{\psi}\psi)/f$, and to the photons ${c_{\gamma}\alpha} S F_{\mu\nu}\tilde{F}^{\mu\nu}/({4\pi f})$, then
$\mathcal{B}(S\rightarrow\gamma\gamma)/\mathcal{B}(S\rightarrow \overline{\psi}\psi) \sim 
({\alpha}/{\pi})^2 (m_S/m_\psi)^2$, 
taking $c_{\psi}$ and $c_{\gamma}$ to be $\mathcal{O}(1)$.
% couplings and $\alpha$ the fine-structure 
%constant. Thus, at leading order,
%%
%\begin{equation}
% \Gamma(S\rightarrow\psi\psi) = \frac{N_c}{8\pi}\frac{c_{\psi}^2 m_\psi^2}{f^2}m_S~ \; \quad \textrm{and}~ \; \quad \Gamma(S\rightarrow\gamma\gamma) = \frac{c_\gamma^2 \alpha^2}{64 \pi^3 f^2}m_S^3~.
%\end{equation}
%%
%One finds the ratio 
 The small electromagnetic coupling driven suppression of $S\to \gamma\gamma$ can be partially 
compensated by scaling with $m_S$, so that $\mathcal{B}(S\to \gamma\gamma)$ can possibly be 
significantly larger than $\mathcal{B}(h\to \gamma\gamma)$. This motivates that both $S\to b\bar b$ and $S\to \gamma\gamma$ are searched for.
%\\ \\



%The Lagrangian in 

%{\bf LHC prospects for $t\rightarrow Sc, S\rightarrow b\overline{b}$} \\

The reach for $t\to c S(\to b\bar b)$ was obtained in~\cite{Banerjee:2018fsx} by requiring 
%Here, we focus on the scenario where the scalar singlet decays to a pair of $b$-quarks, yielding a final state 
%comprised of 
at least four jets, three of them $b$-tagged, and exactly one isolated lepton. 
%Our goal here is to derive an upper bound on $\mathcal{B}(t \to S c, S \to b \bar{b})$ at 95~\% Confidence Level 
%(CL). We use fixed $b$-tagging efficiency of 70\%. The $c \; (\textrm{light jet}) \to b$ mistag rate is 10\% (1\%). 
The dominant real background is the semi-leptonic $t\bar{t} b \bar{b}$ production. 
%Besides, we also have the fully leptonic 
%$t\bar{t}b\bar{b}$. 
The major fake backgrounds are the semi-leptonic (and leptonic) $t\bar{t}$, $Wb\bar{b}$ and 
$Zb\bar{b}$. 
%These are all merged (MLM merging~\cite{Mangano:2006rw}) with up to extra partons.
These are reduced by making cuts on reconstructed top mass, $S$ mass, and on $m_T$, see~\cite{Banerjee:2018fsx} for details. 
%
%%Here, we focus on the scenario where the scalar singlet decays to a pair of $b$-quarks, yielding a final state 
%%comprised of at least four jets, three of them required to be $b$-tagged, and exactly one isolated lepton. 
%%Our goal here is to derive an upper bound on $\mathcal{B}(t \to S c, S \to b \bar{b})$ at 95~\% Confidence Level 
%%(CL). We use fixed $b$-tagging efficiency of 70\%. The $c \; (\textrm{light jet}) \to b$ mistag rate is 10\% (1\%). The most 
%%dominant real background is the semi-leptonic $t\bar{t} b \bar{b}$ production. Besides, we also have the fully leptonic 
%%$t\bar{t}b\bar{b}$. The major fake backgrounds are the semi-leptonic (and leptonic) $t\bar{t}$, $Wb\bar{b}$ and 
%%$Zb\bar{b}$. These are all merged (MLM merging~\cite{Mangano:2006rw}) with up to extra partons.
%
%The details of the
%event selection can be found in the original paper~\cite{Banerjee:2018fsx}. Once the events are selected, we look for the 
%closest pair (in $\Delta R$) of $b$-tagged jets and reconstruct the top mass, $m_t^{\Delta R}$ with the additional
%hardest jet, not $b$-tagged. We demand that this variable lies within 50 GeV from $m_t$. With the remaining $b$-jet,
%we construct the transverse mass variable, $m_T$ and demand $m_T < 200$ GeV. 
%We finally impose a benchmark-dependent cut, 
%$0.8\, m_S < m_{S}^{\Delta R} < m_S + 10$ GeV. Fig.~\ref{fig:sbb2} shows our results. 
The left panel in Fig.~\ref{fig:sbb2} shows  the $3$ab${}^{-1}$ \td{corect?}
95~\% CL upper limit on $BR(t \to S c, S \to b \bar{b})$. The right panel shows the minimum integrated luminosity 
required to test a branching ratio of $10^{-4}$ at 95~\% CL. The highest reach is attained for $m_S \sim 80$ GeV, for which one can probe
$\mathcal{B}(t\rightarrow Sc, S\rightarrow b\overline{b}) > 10^{-4}$ at $95$ \% CL with $\mathcal{L} = 3$ ab$^{-1}$ of integrated luminosity. At lower masses, this reach is smaller as it is harder to resolve two $b$-jets, but a boosted analysis with fat jets may be useful. 
%For larger masses, 
%the sensitivity gets also reduced because $m_t^{\Delta R}$ does not always peak around $m_t$~\cite{Banerjee:2018fsx}. \\ \\
%

%{\bf LHC prospects for $t\rightarrow Sc, S\rightarrow \gamma\gamma$} \\

\begin{figure}[t]
 \includegraphics[width=0.45\textwidth]{\main/section10/img/bbChannel.pdf}
 \includegraphics[width=0.45\textwidth]{\main/section10/img/LbbChannel.pdf}
 %
 \caption{\label{fig:sbb2} \it Left) Branching ratios that can be tested in the $b\overline{b}$ channel. Superimposed are the theoretical expectations in the three BPs. Right) Luminosity required to test $\mathcal{B}(t\rightarrow S c, S\rightarrow b\overline{b}) = 10^{-4}$. Superimposed are $\mathcal{L} = 300$ fb$^{-1}$ and $3000$ fb$^{-1}$.}
\end{figure}
%
%
%

For $t\to cS(\to \gamma)$ the analysis 
%decays to a pair of photons. We consider a final state comprising 
required at least two 
jets (one $b$-tagged), one isolated lepton and two isolated photons, with a 
% The reconstructions are similar to the previous
%case. Because of a much sharper diphoton mass resolution, we require a 
3 GeV mass window about the reconstructed scalar mass. 
%The 
%other cuts remain as in the previous case. 
We show the 95\% CL upper limit on the branching ratio $\mathcal{B}(t 
\to S c, S \to \gamma \gamma)$ in Fig.~\ref{fig:sgg2}, along with the minimum integrated luminosity necessary to 
probe a branching ratio of $10^{-6}$. The effect of the 125 GeV Higgs can be seen where the background is much 
larger.
We find that,
with the same integrated luminosity, $\mathcal{B}(t\rightarrow Sc, S\rightarrow \gamma\gamma) > 10^{-7}$ can be 
tested at the 95 \% CL. Consequently, if $\mathcal{B}(S\rightarrow\gamma\gamma) \sim 1\%$, we can indirectly probe new 
physics scales as large as $\sim 50$ TeV. In this particular channel, the major background is 
$t\overline{t}\gamma\gamma$ for $m_S$, well separated from $m_h$. 
 
%
%%%%%%%%%%%%%%%%%%%%%%%%%%%%%%%
\begin{figure}[t]
 \includegraphics[width=0.45\textwidth]{\main/section10/img/ggChannel.pdf}
 \includegraphics[width=0.45\textwidth]{\main/section10/img/LggChannel.pdf}
 %
 \caption{\label{fig:sgg2} \it Left) Branching ratios that can be tested in the $\gamma\gamma$ channel. Superimposed are the theoretical expectations in the three BPs. Right) Luminosity required to test $\mathcal{B}(t\rightarrow St, S\rightarrow \gamma\gamma) = 10^{-6}$. Superimposed are $\mathcal{L} = 300$ fb$^{-1}$ and $3000$ fb$^{-1}$.}
\end{figure}



In regard of the possibilities af future colliders, we find that the increase in cross-section 
for the background at $\sqrt{s} = 27$ TeV ($100$ TeV) when compared to $\sqrt{s} = 14$ TeV is similar to 
that in the signal, and is of order $\sim 4$ ($\sim 40$). Assuming an integrated luminosity of $10$ ab$^{-1}$, 
we expect an increase in significance by a factor of $\sim 3.7$ ($\sim 11.5$). Similar results will hold for the 
$b\overline{b}$ channel.

\subsubsection{Neutrino Mass Models at the HL/HE LHC}
{\bf Author(s): Cedric Weiland?}(anticipated length 2 pages)
\td{missing}

\subsection{Implications of TeV scale flavour models for electroweak baryogenesis} 
{\bf Author(s): Oleksii Matsedonskyi, Geraldine Servant} (anticipated length 2 pages)

%Intro: EWBG

In most solutions to the SM flavour puzzle, Yukawa couplings have a dynamical origin, which means that they potentially impact the cosmological evolution.
% should be studied.
 In Refs.~\cite{Baldes:2016rqn,Baldes:2016gaf,vonHarling:2016vhf,Bruggisser:2017lhc,Bruggisser:2018mus,Bruggisser:2018mrt,Baldes:2018nel,Servant:2018xcs},  such connections between flavour and cosmology, in particular with the  electroweak baryogenesis,  have been investigated in detail.

Electroweak baryogenesis (EWBG) is a mechanism to explain the matter antimatter asymmetry of the universe  using baryon number violation already present in the SM. It 
 relies on a charge transport mechanism in the vicinity of bubble walls during a first-order electroweak (EW) phase transition. EWBF uses EW scale physics only and is therefore testable experimentally. It requires an extension of the Higgs sector, giving a first-order EW phase transition as well as new sources of CP violation. Typically, there are stringent constraints on EWBG models from bounds on Electric Dipole Moments.
 In the following,  we consider that the source of CP violation has changed with time, which is a natural way to evade constraints. 
 The main motivation is to link EWBG to low-scale flavour models.  If the physics responsible for the structure of the Yukawa couplings is linked to EW symmetry breaking, we can expect the Yukawa couplings to vary at the same time as the Higgs is acquiring a VEV, in particular if the flavour structure is controlled by a new scalar field which couples to the Higgs.
 This is  precisely what can happen in Composite Higgs (CH) models ~\cite{Kaplan:1983fs,Panico:2015jxa}, on which we focus in the following.


%Intro: CH

The  Composite Higgs (CH) models assume that the Higgs boson arises as a bound state of a new strong interaction, confining around the TeV scale $f$. Other composite resonances are naturally heavier than the Higgs, due to an approximate Goldstone symmetry suppressing the Higgs mass. The rest of the SM fields do not belong to the strong sector and are elementary. The nature of the EW phase transition in CH models can be substantially different with respect to the SM. One of the reasons is that the CH models naturally feature new scalar resonances that participate in the EW phase transition and change its properties (see Refs.~\cite{Espinosa:2011eu,Chala:2016ykx}). On the other hand, the EW transition may become strongly first order even without the help of such additional states, if the EW transition happens simultaneously with the deconfinement-confinement phase transition of the new strong sector (see Refs.~\cite{Bruggisser:2018mus,Bruggisser:2018mrt} and also Refs.~\cite{vonHarling:2016vhf,Megias:2018sxv} for the dual description in the warped 5D space).    

%Flavour in CH

The flavour structure of the CH models is intimately tied with the viability of EWBG. The requirement of having a sizeable top quark Yukawa coupling, while suppressing the unwanted flavour-violating effects, suggests that the new sector is nearly scale-invariant
% and strongly coupled 
for a large range of energies above the confinement scale. As a result, one may expect that the transition dynamics is mostly determined by a single light field -- the dilaton $\chi$ (see e.g Ref.~\cite{Coradeschi:2013gda}). Depending on its mass, whose size can be related to the separation of the UV flavour scale and the EW scale, the EW phase transition may happen separately from the confinement, or simultaneously with it~\cite{Bruggisser:2018mus,Bruggisser:2018mrt}. %The two-step transition is expected to happen for the heavier dilaton, while the one-step -- for the dilaton lighter than about $\sim 0.5$~TeV. 

Moreover, the mechanism for generating the SM flavour hierarchy in CH models may also be a source of CP asymmetry during the EW phase transition. The hierarchy of SM Yukawas $\lambda_q$ \td{does it conform to Higgs section notation} is generated by the renormalization group running of the couplings $y_q$ between the elementary fermions and the strong sector operators, 
\begin{equation}
\lambda_q \propto y_q^2, \quad\text{with} \quad y_q=y_q^{\text{UV}} \left( \frac{\mu}{\Lambda_{\text{UV}}} \right)^{\gamma_{y_q}},
\end{equation}
where $\mu\sim \chi$ is the confinement scale, $\Lambda_{\text{UV}}$ is some large scale at which all the mixings are generated with a similar size, $y_q^{\text{UV}}$, and $\gamma_{y_q}$ is the anomalous dimension of the operator responsible for the mixing.
 This means that the size of the Yukawa couplings changes with the evolution of the confinement scale during the confinement phase transition. Such a change of the Yukawa interactions may efficiently source the CP-violation required for the baryogenesis~\cite{Bruggisser:2017lhc}.

%Pheno

\begin{figure}[!t]\centering
\includegraphics[scale=.44]{\main/section10/img/hvv.png}
\hspace{0.2cm}
\includegraphics[scale=.44]{\main/section10/img/Imhtt.png}
\caption{Relative deviation of the Higgs couplings to $W$ and $Z$ bosons (left panel) and the imaginary part of the correction to the top quark Yukawa coupling (right panel), in terms of the dilaton mass $m_\chi$, the number of colours of the new confining sector $N$, and the mass of generic composite states $m_*$. Black contours correspond to the glueball-like dilaton and red dashed to the meson-like dilaton. The current  and near future experimental sensitivities of electron EDM experiments  to the imaginary part of the top yukawa correspond respectively to approximately $2\times 10^{-2}$~\cite{Cirigliano:2016nyn} and $2\times 10^{-4}$~\cite{Kumar:2013qya}.
}
\label{fig:ewbg_observables}
\end{figure}

For both types of transitions  mentioned, one can expect to observe deviations of the Higgs couplings from the SM predictions. These deviations are result of contributions generic to CH models, as well as those linked to the features of the phase transition and new sources of CP-violation. For concreteness we focus on the  more minimal example, the combined electroweak and strong sector phase transition. 
%
The potential of the Higgs boson can be parametrized in terms of trigonometric functions of $h/f$, with the overall size of the potential controlled by the mixings between the elementary and composite fermions~\cite{Matsedonskyi:2012ym,Panico:2012uw},
\begin{equation}\label{eq:ewbg_vh}
V = c_1 \sin^2 \frac h f + c_2 \sin^4 \frac h f,
\end{equation}
where $c_1 \sim c_2 \sim \sum_q ({3 y_q^2}/{(4 \pi)^2}) g_\star^2 f^4$. The dependence of the $y_q$ mixings on the dilaton field is responsible for the mass mixing between the dilaton and the Higgs field, which we parametrise by an angle $\delta$.
To generate sufficient amount of CP violation, such mass mixing needs to be sizeable. Let us consider its effect on the quark Yukawa coupling:
\begin{equation}\label{eq:ewbg_yuk}
{\cal L}_{\rm Yukawa} = \lambda(\chi) \left(\chi \sin \frac h f \right) \bar q_L q_R 
= \bar q_L q_R h \left(\lambda(f) \frac {\chi}{f} + \beta_{\lambda} \frac {\chi - f }{f} \right) + \dots, 
\end{equation}
where we performed an expansion in $\chi$ around its present day value, $f$. Similar, but CP-preserving, modifications are also generated in the couplings of the Higgs boson to the SM gauge fields and the Higgs self-interactions.
The complex phase of the Yukawa beta-function, $\beta_\lambda$, in \eqref{eq:ewbg_yuk} has to be different from that of the quark mass $\lambda(f) h$, as the Yukawa phase changes with energy. 
Choosing the mass parameter to be real, the CP-violating interaction resides in the term $\propto \beta_\lambda$. Rotating the $h$ and $\chi$ fields to the mass basis, the CP-even and CP-odd corrections to the Higgs Yukawa  interaction are
\begin{equation}
\text{Re} [\delta \lambda] \sim \text{Re} [\beta_\lambda] \delta (v/f) + \lambda (\delta^2/2+\delta(v/f)) ,\quad
\text{Im} [\delta \lambda] \sim \text{Im} [\beta_\lambda]  \delta (v/f).
\end{equation} 
In Fig.~(\ref{fig:ewbg_observables}) we show the values of the CP-violating top quark Yukawa modification and the deviations of the Higgs couplings to the $W$ and $Z$ bosons.
Such couplings can be tested directly at the LHC, and also in the measurements of electric dipole moments.  
In addition to the observables directly linked to the CP violation, the strength of the phase transition can be tested in gravitational waves signals at the future space-based observatory LISA~\cite{Caprini:2015zlo}.


\subsection{High \pt searches in the context of flavour anomalies}

{\bf Authors:} Alejandro Celis, Admir Greljo, Lukas Mittnacht, Marco Nardecchia, Tevong You

Precision measurements of flavour transitions at low energies, such as flavour changing $B$, $D$ and $K$ decays, are sensitive probes of hypothetical dynamics at high energy scales.  These can provide the first evidence of new phenomena beyond the SM, even before direct discovery of new particles at high energy colliders. Indeed, the current anomalies observed in $B$-meson decays, in particular, the charged current one in $b\to c\tau \nu$ transitions, and neutral current one in $b\to s\ell^+\ell^-$,  may be the first hint of new dynamics which is still waiting to be discovered at high-\pt. When considering models that can accommodate the anomalies, it is crucial to analyse the constraints derived from high-\pt searches at the LHC, since these can often rule out significant regions of model parameter space. Below we review these constrains, and assess the impact of the High Luminosity and High Energy LHC upgrades (see also the discussion in Sections \ref{sec:7:bsll:NP},\ref{sec7:bsll:model:indep:fits}, and \ref{sec7:ModelsNP:bctaunu}).



\subsubsection{EFT analysis} 

If the dominant NP effects give rise to dimension-six SMEFT operators, the low-energy flavour measurements are sensitive to $C/\Lambda^2$, with $C$ the dimensionless NP Wilson coefficient and $\Lambda$ the NP scale.    The size of the Wilson coefficient is model dependent, and thus so is the NP scale required to explain the $R_{D^{(*)}}$and $R_{K^{(*)}}$ anomalies.    Perturbative unitarity sets an upper bound on the energy scale below which new dynamics need to appear~\cite{DiLuzio:2017chi}.    
The conservative bounds on the scale of unitarity violation are   $\Lambda_U = 9.2$ TeV and $84$ TeV for $R_{D^{(*)}}$and $R_{K^{(*)}}$, respectively, obtained when the flavour structure of NP operators is exactly aligned with what is needed to explain the anomalies.       More realistic frameworks for flavour structure, such as MFV, $U(2)$ flavour models, or partial compositeness, give rise to NP effective operators with largest effects for the third generation. This results in stronger unitarity bounds,
$\Lambda_U = 1.9$ TeV and $17$ TeV for $R_{D^{(*)}}$ and $R_{K^{(*)}}$, respectively.  
These results mean that: \textit{(i)} the mediators responsible for the $b\to c \tau \nu$ charged current anomalies are expected to be in the energy range of the LHC, \textit{(ii)}  the mediators responsible for the $b\to s \ell \ell$ neutral current anomalies could well be above the energy range of the LHC. However, in realistic flavour models also these mediators typically fall within the (HE-)LHC reach.       

If the neutrinos in $b \to c \tau \nu$ are part of a left-handed doublet, the NP responsible for $R_{D^{(*)}}$ anomaly
generically implies a sizeable signal in $p p \to \tau^+ \tau^-$ production at high-\pt. For realistic flavour structures, in which $b \to c$ transition is $\mathcal{O}(V_{cb})$ suppressed compared to $b \to b$, one expects rather large $b b \to \tau \tau$ NP amplitude. 
Schematically, $\Delta R_{D^{(*)}} \sim C_{b b \tau \tau} (1 + \lambda_{bs} / {V_{cb}})$, where $C_{b b \tau \tau}$ is the size of effective dim-6 interactions controlling $b b \to \tau \tau$, and $\lambda_{bs}$ is a dimensionless parameter controlling the size of flavour violation. Recasting ATLAS 13 TeV, $3.2$~fb$^{-1}$ search for $\tau^+ \tau^-$ \cite{Aaboud:2016cre}, Ref.~\cite{Faroughy:2016osc} showed that $\lambda_{bs} = 0$ scenario is already in slight tension with data. For $\lambda_{bs} \sim 5$, which is moderately large, but still compatible with FCNC constraints, HL (or even HE) upgrade of the LHC would be needed to cover the relevant parameter space implied by the anomaly (see Fig.~[5] in~\cite{Buttazzo:2017ixm}). 
For large $\lambda_{bs}$ the limits from $p p \to \tau^+ \tau^-$ become comparable with direct the limits on $p p \to \tau \nu$ from the bottom-charm fusion. The limits on the EFT coefficients from $p p \to \tau \nu$ were derived in Ref.~\cite{Greljo:2018tzh}, and the future LHC projections are  promising. The main virtue of this channel is that the same four-fermion interaction is compared in $b \to c \tau \nu$ at low energies and $b c \to \tau \nu$ at high-\pt. 
Since the effective NP scale in $R(D^{(*)})$ anomaly is low, the above EFT analyses are only indicative. 
For more quantitative statements we review below bounds on explicit models.

The hints of NP in $R_{K^{(*)}}$ require  a $(b s) (\ell \ell)$ interaction.
Correlated effects in high-\pt tails of $p p \to \mu^+ \mu^- (e^+ e^-)$ distributions are expected, if the numerators (denominators) of LFU ratios $R_{K^{(*)}}$ are affected. 
Ref.~\cite{Greljo:2017vvb} recast the  13~TeV $36.1$~fb$^{-1}$ ATLAS search \cite{ATLAS:2017wce} (see also \cite{Aaboud:2017buh}),
to set limits on a number of semi-leptonic four-fermion operators, and derive projections for HL-LHC (see Table 1 in~\cite{Greljo:2017vvb}). These show that direct limits on the $(b s) (\ell \ell)$ operator from the tails of distributions will never be competitive with those implied by the rare $B$-decays~\cite{Greljo:2017vvb,Afik:2018nlr}. On the other hand, flavour conserving operators, $(q q) (\ell \ell)$, are efficiently constrained by the high \pt tails of the distributions. The flavour structure of an underlining NP could thus be probed by constraining ratios $\lambda^q_{b s} = C_{b s} / C_{qq}$ with $C_{b s}$  fixed by the $R_{K^{(*)}}$ anomaly. For example, in models with MFV flavour structure, so that $\lambda^{u,d}_{b s} \sim V_{c b}$, the present high-\pt dilepton data is already in slight tension with the anomaly~\cite{Greljo:2017vvb}. Instead, if  couplings to valence quarks are suppressed, e.g., if NP dominantly couples to the 3rd family SM fermions, then $\lambda^b_{b s} \sim V_{c b}$. Such NP will hardly be probed even at the HL-LHC, and  it is possible that NP responsible for the neutral current anomaly might stay undetected in the high-\pt tails at HL-LHC and even at HE-LHC. Future data will cover a significant part of viable parameter space, though not completely, so that discovery is possible, but not guaranteed.


\subsubsection{Constraints on simplified models for $b\to c \tau \nu$}


Since the $b\to c \tau \nu$ decay is a tree-level process in the SM that receives no drastic suppression, models that can explain these anomalies necessarily require a mediator that contributes at tree-level:      


\noindent  $\bullet$ \, \textit{SM-like $W^{\prime}$}:  A SM-like $W^{\prime}$ boson, coupling to left-handed fermions, would explain the approximately equal enhancements observed in $R(D)$ and $R(D^{*})$.     A possible realization is a color-neutral real $SU(2)_L$ triplet of massive vector bosons~\cite{Greljo:2015mma}.  However, typical models encounter problems with current LHC data since they result in large contributions to $pp \to \tau^+ \tau^-$ cross-section, mediated by the neutral partner of the $W^{\prime}$~\cite{Greljo:2015mma,Boucenna:2016qad,Faroughy:2016osc}.    For $M_{W^{\prime}} \gtrsim 500$~GeV, solving the $R(D^{(*)})$ anomaly within the vector triplet model while being consistent with $\tau^+ \tau^-$ resonance searches at the LHC is only possible if the related $Z^{\prime}$ has a large total decay width~\cite{Faroughy:2016osc}. Focusing on the $W^{\prime}$, Ref.~\cite{Abdullah:2018ets} analyzed the production of this mediator via $gg$ and $gc$ fusion, decaying to
$\tau \nu_{\tau}$. Ref.~\cite{Abdullah:2018ets} concluded that a dedicated search using that $b$-jet is present in the final state would be effective in reducing the SM background compared to an inclusive analysis that relies on $\tau$-tagging and $E_T^{\rm{miss}}$. Nonetheless, relevant limits will be set by an inclusive search in the future~\cite{Greljo:2018tzh}.


\vspace{0.2cm}

\noindent  $\bullet$  \,  \textit{Right-handed $W^{\prime}$:} Refs.~\cite{Greljo:2018ogz,Asadi:2018wea} recently proposed that $W^{\prime}$ could mediate a right-handed interaction, with a light sterile right-handed neutrino carrying the missing energy in the $B$ decay. In this case, it is possible to completely decorelate FCNC constraints from $R(D^{(*)})$. The most constraining process in this case is instead $p p \to \tau \nu$. Ref.~\cite{Greljo:2018tzh} performed a recast of the latest ATLAS and CMS searches at 13 TeV and about 36 fb$^{-1}$ to constrain most of the relevant parameter space for the anomaly.

\vspace{0.2cm}

\noindent  $\bullet$\, \textit{Charged Higgs $H^{\pm}$:}       Models that introduce a charged Higgs, for instance a two-Higgs-doublet model, also contain additional neutral scalars.  Their masses are constrained by electroweak precision measurements to be close to that of the charged Higgs.  Accommodating the $R(D^{(*)})$ anomalies with a charged Higgs typically implies large new physics contributions to $pp \to \tau^+ \tau^-$ via the neutral scalar exchanges, so that current LHC data can challenge this option~\cite{Faroughy:2016osc}.     Note that a charged Higgs also presents an important tension between the current measurement of $R(D^*)$ and the measured lifetime of the $B_c$ meson~\cite{Li:2016vvp,Alonso:2016oyd,Celis:2016azn,Akeroyd:2017mhr}.


\begin{figure}[t]
\centering
\includegraphics[width=0.39 \textwidth]{section10/img/LQ-search-strategy.pdf}
\caption{Schematic of the LHC bounds on  LQ showing complementarity  in constraining the  $(m_{LQ},y_{q_{\ell}})$ parameters. The three cases are:
 pair production $\sigma \propto y_{q_{\ell}}^0 $, single production $\sigma \propto y_{q_{\ell}}^2 $ and Drell-Yan $\sigma \propto y_{q_{\ell}}^4$ (from~\cite{Dorsner:2018ynv}).}
\label{fig:LQstrategy}
\end{figure}

\begin{figure}[t]
\centering
\includegraphics[width=0.46 \textwidth]{section10/img/S3.pdf}~~~
\includegraphics[width=0.46 \textwidth]{section10/img/U1.pdf}
\caption{ \small \Blu{Present constraints and HL(HE)-LHC projections in the leptoquark mass versus coupling  plane for the scalar leptoquark $S_3$ (left), and vector leptoquark $U_1$ (right). The grey and dark grey solid regions are the current exclusions. The grey and black dashed lines are the projected reach for HL-LHC (pair and single leptoquark production prospects are based on the CMS simulation from Section XXX). The red dashed lines are the projected reach at HE-LHC (see Section \ref{sec:HE_LHC_LQ}). The green and yellow bands are the $1\sigma$ and $2\sigma$ preferred regions from the fit to $B$ physics anomalies. The second coupling required to fit the anomaly does not enter in the leading high-\pt diagrams but it is relevant for fixing the preferred region shown in green, for more details see Ref.~\cite{Buttazzo:2017ixm}. 
}
}
\label{fig:LQmediators}
\end{figure}

\vspace{0.2cm}

\noindent  $\bullet$ \, \textit{Leptoquarks:} 

The observed anomalies in charged and neutral currents appear in semileptonic decays of the $B$-mesons. This implies that the putative NP has to couple to both quarks and leptons at the fundamental level. A natural BSM option is to consider mediators that couple simultaneously quarks and leptons at the tree level. Such states are commonly referred as leptoquarks. Decay and production mechanisms of the LQ are directly linked to the physics required to explain the anomalous data.
\begin{itemize}
\item {\bf Leptoquark decays:} the fit to the $R(D^{*})$ observables suggest a rather light leptoquark (at the TeV scale) that couples predominately to the third generation fermions of the SM. A series of constraints from flavour physics, in particular the absence of BSM effects in kaon and charm mixing observables, reinforces this picture. 

\item {\bf Leptoquark production mechanism:}  The size of the couplings required to explain the anomaly is typically very large, roughly $y_{q\ell} \approx m_{LQ}$/ (1 TeV). Depending of the actual sizes of the leptoquark couplings and its mass we can distinguish three regimes that are relevant for the phenomenology at the LHC:
 	\begin{enumerate}
	\item LQ pair production due to strong interactions,
	\item Single LQ production plus lepton via a single insertion of the LQ coupling, and
	\item Non-resonant production of di-lepton through $t$-channel exchange of the leptoquark.
	\end{enumerate}
Interestingly all three regimes provide complementary bounds in the  $(m_{LQ},y_{q_{\ell}})$ plane, see Fig.~\ref{fig:LQstrategy}.

\end{itemize}




Several simplified models with leptoquark as a mediator were shown to be consistent with the low-energy data.
 A vector leptoquark with $SU(3)_c\times SU(2)_L\times U(1)_Y$ SM quantum numbers $U_\mu \sim ({\bf 3}, {\bf 1}, 2/3)$ was identified as the only single mediator model which can simultaneously fit the two anomalies (see e.g.~\cite{Buttazzo:2017ixm} for a recent fit including leading RGE effects). 
 In order to substantially cover the relevant parameter space, one needs future HL- (HE-) LHC, see Fig.~\ref{fig:LQmediators} (right) (see also Fig.~5 of~\cite{Buttazzo:2017ixm} for details on the present LHC constraints). A similar statement applies to an alternative model featuring two scalar leptoquarks, $S_1, S_3$ \cite{Crivellin:2017zlb}. The pair of plots in Fig. \ref{fig:LQmediators} summaries the current exclusion and the discovery reach for the HE and HL-LHC in the LQ coupling versus mass plane. 
 
 \Blu{Leptoquarks states are emerging as the most convincing mediators for the explanations of the flavour anomalies. It is then important to explore all the possible signatures at the HL- and HE-LHC.
 The experimental program should focus not only on final states containing quarks and leptons of the third generation, but also on the whole list of decay channels including the off-diagonal ones ($b \mu$, $s \tau, \dots$). The completeness of this approach would allow to shed light on the flavour structure of the putative New Physics. 
 
 Another aspects to be emphasized concerning leptoquark models for the anomalies is the fact of the possible presence of extra fields required to complete the UV lagrangian. The accompanying particles would leave more important signatures at high $p_T$ than the leptoquark, this is particularly true for vector leptoquark extensions (see for example \cite{DiLuzio:2017vat,DiLuzio:2018zxy})}  
    
 
%\Blu{As a final remark, it is important to remember that the anomalies are not yet experimentally established. Among others, this also means that the statements on whether or not  the high \pt LHC constraints rule out certain $R(D^{(*)})$ explanations assumes that the actual values of $R(D^{(*)})$ are given by their current global averages. If future measurements decrease the global average, the high \pt constraints can in some cases be greatly relaxed and HL- and/or HE-LHC may be essential for these, at present tightly constrained, cases.}


%%%%%%%%%%%
\subsubsubsection{HE-LHC sensitivity study for $p p \to \Phi \Phi \to (b \tau) (b \tau)$}
\label{sec:HE_LHC_LQ}
\begin{figure}[t]
	\centering
	\includegraphics[width=0.5\textwidth]{section10/img/soverb.pdf}
	\caption{Expected sensitivity for pair production of scalar (\Blu{$S_3$}) and vector ($U_1$) leptoquark at 27 TeV $p p$ collider with an integrated luminosity of $15~\text{ab}^{-1}$.} \label{fig:lukas}
\end{figure}

We analysed the sensitivity of the 27 TeV $pp$ collider with $15$~ab$^{-1}$ integrated luminosity to probe pair production of the scalar and vector leptoquarks decaying to $(b \tau)$ final state. We investigated events containing either one electron or muon, one hadronically decaying tau lepton, and  at least two jets. The signal events and the dominant background events ($t\bar{t}$) were generated with {\tt MadGraph5\_aMC@NLO} at leading-order. {\tt PYTHIA6} was used to shower and hadronise events and {\tt DELPHES3} was used to simulate the detector response. The scalar leptoquark (\Blu{$S_3$}) and the vector leptoquark ($U_1$) UFO model files were taken from~\cite{Dorsner:2018ynv}.

To verify the procedure, we simulated the $t\bar{t}$ background and the scalar leptoquark signal at 13 TeV and compared to the predicted shapes in the $S_T$ distribution from the existing CMS analysis~\cite{Sirunyan:2017yrk}. After we verified the 13 TeV analysis, we simulated the signal and the dominant background events at $27$ TeV. From these samples, we selected all events satisfying the particle content requirements and applied the lower cut in the $S_T$ variable. The cut threshold was chosen to maximize $s/\sqrt{b}$ while requiring at least 2 expected signal events at an integrated luminosity of $15~\text{ab}^{-1}$. In the case of the vector leptoquark we considered the Yang-Mills ($\kappa = 1$) and the minimal coupling ($\kappa = 0$) scenarios for the couplings to the gluon field strength, ${\cal L}\supset -ig \kappa U_{1\mu}^\dagger T^a U_{1\nu} G^a_{\mu\nu}$~\cite{Dorsner:2018ynv}. From the simulations of the scalar and vector leptoquark events, we found the ratio of the cross-sections, and assuming similar kinematics, we estimated the sensitivity also for the vector leptoquark $U_1$.

As shown in Fig.~\ref{fig:lukas}, the HE-LHC collider will be able to probe pair produced third generation scalar leptoquark (decaying exclusively to $b \tau$ final state) up to mass of $\sim 4$~TeV and vector leptoquark up to $\sim 4.5$~TeV and $\sim 5.2$~TeV for the minimal coupling and Yang-Mills scenarios, respectively. While this result is obtained by a rather crude analysis, it shows the impressive reach of the future high-energy $pp$-collider. In particular, the HE-LHC will cut deep into the relevant perturbative parameter space for $b \to c \tau \nu$ anomaly. As a final comment, this is a rather conservative estimate of the sensitivity to leptoquark models solving $R(D^{*})$ anomaly, since a dedicated single leptoquark production search is expected to yield even stronger bounds~\cite{Buttazzo:2017ixm,Dorsner:2018ynv}.
%%%%%%%%%%%







\subsubsection{Constraints on simplified models for $b\to s ll$}

The $b\to s ll$ transition is  both loop and CKM suppressed in the SM. The explanations of the $b\to s ll$ anomalies can thus have both tree level and loop level mediators.
% enter at tree-level or at the loop level. 
Loop-level explanations typically involve lighter particles.
% that enter at a lower UV cut-off scale. 
Tree-level mediators can also be light, if sufficiently weakly coupled. However, they can also be much heavier---possibly beyond the reach of the LHC. 

%\begin{figure}[t]
%\centering
%\includegraphics[width=0.45 %\textwidth]{section10/img/feynmandiagrams_LQZprime.PNG}
%\caption{Tree level Feynman diagrams for leptoquarks (left) and $Z^\prime$ (right)  mediating $b \to s \mu^+ \mu^-$ transition (Fig. taken from ~\cite{Allanach:2017bta}).}
%\label{fig:feynmanLQZprime}
%\end{figure}

\noindent $\bullet$ \textit{Tree-level mediators:}\\
%For the four-fermion operators involved in 
For $b \to sll$ anomaly there are two possible tree-level UV-completions, the $Z'$ vector boson and leptoquarks, either scalar or vector (see Fig.~\ref{fig:bsll_diagrams} in Sec.~\ref{sec:7:bsll:NP}).
% whose Feynman diagrams are depicted in Fig.~\ref{fig:feynmanLQZprime}. The $Z^\prime$ vector boson mediates an interaction between the $b$ and $s$ quark at one vertex and two muons at the other. A leptoquark is defined as having quantum numbers such that it may interact with a quark and lepton at one vertex, and can be either scalar or vector. In this case the minimal couplings required of the leptoquark must contain a $b$ or $s$ together with a muon at a vertex. 
%Ref.~\cite{Allanach:2017bta} 
%%initiated the study of implications of the anomalies for direct discovery at future colliders. They 
%estimated the sensitivity to simplified models of $Z^\prime$ and leptoquarks at the HL- and HE-LHC (as well as the FCC-hh) in the expected parameter space where the new particles may explain the neutral current flavour anomalies. A more detailed study for the $Z^\prime$ case was then made by Ref.~\cite{Allanach:2018odd}, extending in particular to wider widths. We summarise their results here. 
For leptoquarks, Fig.~\ref{fig:LQexclusion} shows the current 95\% CL limits from 8 TeV CMS with 19.6 fb$^{-1}$  in the $\mu\mu jj$ final state (solid black line), as well as the HL-LHC (dashed black line) and 1 (10) ab$^{-1}$ HE-LHC extrapolated limits (solid (dashed) cyan line). 
%is denoted by a dashed black line for HL-LHC and by a solid (dashed) cyan line for HE-LHC with 1 (10) ab$^{-1}$. 
Dotted lines give the cross-sections times branching ratio at the corresponding collider energy for pair production of scalar leptoquarks, calculated at NLO using the code of Ref.~\cite{Kramer:2004df}. We see that the sensitivity to a leptoquark with only the minimal $b-\mu$ and $s-\mu$ couplings reaches around 2.5 and 4.5 TeV at the HL-LHC and HE-LHC, respectively. This pessimistic estimate is a lower bound that will typically be improved in realistic models with additional flavour couplings. Moreover, the reach can be extended by single production searches~\cite{Allanach:2017bta}, albeit in a more model-dependent way than pair production. The cross section predictions for vector leptoquark are more model-dependent and are not shown in Fig.~\ref{fig:LQexclusion}. For $\mathcal{O}(1)$ couplings the corresponding limits are typically stronger than for scalar leptoquarks. 

\begin{figure}[t]
\centering
\includegraphics[width=0.45 \textwidth]{section10/img/LQ_exclusionplot_HLLHC_HELHC.png}
\caption{Current and projected 95\% CL limits on $\mu\mu jj$ final state at CMS (solid black) and HL-LHC (cyan) with 1 (10) ab$^{-1}$ in solid (dashed) lines. The pair production cross-section for scalar leptoquarks is shown in dotted lines for 14 (27) TeV in black (cyan) (Fig. taken from ~\cite{Allanach:2017bta}). \Blu{See also Section (XXX-->Cambridge group)}}
\label{fig:LQexclusion}
\end{figure}

\begin{figure}[t]
\centering
\includegraphics[width=0.39 \textwidth]{section10/img/14TeVMDM.pdf}
\caption{ HL-LHC 95\% (blue) and 99\% (orange) CL sensitivity contours to $Z^\prime$ in the ``mixed-down'' model for $g_{\mu\mu}$ vs $Z^\prime$ mass in TeV. The dashed grey contours give the $Z'$ width as a fraction of mass. The green and red regions are excluded by trident neutrino production and $B_s$ mixing, respectively. The dashed blue line is the stronger $B_s$ mixing constraint from Ref.~\cite{Bazavov:2016nty} (Fig. taken from~\cite{Allanach:2018odd}). \Blu{See also Section (XXX-->Cambridge group)} }
\label{fig:Zprime14TeV}
\end{figure}

For the $Z^\prime$ mediator the minimal couplings in the mass eigenstate basis are obtained by unitary transformations from the gauge eigenstate basis, which necessarily induces other couplings.  Ref.~\cite{Allanach:2018odd} defined the ``mixed-up'' model (MUM) and ``mixed-down'' model (MDM) such that the minimal couplings are obtained via CKM rotations in either the up or down sectors respectively. For MUM there is no sensitivity at the HL-LHC.  The predicted sensitivity at the HL-LHC for the MDM is shown in Fig.~\ref{fig:Zprime14TeV} as functions of $Z^\prime$ muon coupling $g_{\mu\mu}$ and the $Z^\prime$ mass, setting the $Z'$ coupling to $b$ and $s$ quarks such that it solves the $b\to s\ell\ell$ anomaly. The solid blue (orange) contours give the 95\% and 99\% CL sensitivity. The red and green regions are excluded by $B_s$ mixing~\cite{Arnan:2016cpy} and neutrino trident production~\cite{Falkowski:2017pss,Falkowski:2018dsl}, respectively. The more stringent $B_s$ mixing constraint from Ref.~\cite{Bazavov:2016nty} is denoted by the dashed blue line; see, however, Ref.~\cite{Kumar:2018kmr} for further discussion regarding the implications of this bound. The dashed grey contours denote the width as a fraction of the mass. We see that the HL-LHC will only be sensitive to $Z^\prime$ with narrow width, up to masses of 5 TeV. 

\begin{figure}[t]
\centering
\includegraphics[width=0.39 \textwidth]{section10/img/27TeVMUM.pdf}~~~~~~~~~
\includegraphics[width=0.39 \textwidth]{section10/img/27TeVMDM.pdf}
\caption{ HE-LHC 95\% (blue) and 99\% (orange) CL sensitivity contours to $Z^{\prime}$ in the ``mixed-up'' (left) and ``mixed-down'' (right) model in the parameter space of $g_{\mu\mu}$ vs $Z^{\prime}$ mass in TeV. The dashed grey contours are the width as a fraction of mass. The green and red regions are excluded by trident neutrino production and $B_s$ mixing. The dashed blue line is the stronger $B_s$ mixing constraint from Ref.~\cite{Bazavov:2016nty} (Figs. taken from~\cite{Allanach:2018odd}). \Blu{See also Section (XXX-->Cambridge group)}}
\label{fig:Zprime27TeV}
\end{figure}

At the HE-LHC, the reach for 10 ab$^{-1}$ is shown in Fig.~\ref{fig:Zprime27TeV} for the MUM and MDM on the left and right, respectively. In this case the sensitivity may reach a $Z^{\prime}$ with wider widths up to 0.25 and 0.5 of its mass, while the mass extends out to 10 to 12 TeV. We stress that this is a pessimistic estimate of the projected sensitivity, particular to the two minimal models; more realistic scenarios will typically be easier to discover. 


\vspace{0.2cm}

\noindent  $\bullet$ \textit{Explanations at the one-loop level:}\\
It is possible to accomodate the $b \to s \ell^+ \ell^-$ anomalies even if mediators only enter at one-loop.  One possibility is the mediators coupling to right-handed top quarks and to muons~\cite{Celis:2017doq,Becirevic:2017jtw,Kamenik:2017tnu,Fox:2018ldq,Camargo-Molina:2018cwu}.   Given the loop and CKM  suppression of the NP contribution to the $b \to s \ell^+ \ell^-$ amplitude, these models can explain the $b \to s \ell^+ \ell^-$ anomalies for a light mediator, with mass around $\mathcal{O}(1)$~TeV or lighter.       Constraints from the LHC and future projections for the HL-LHC were derived in~\cite{Camargo-Molina:2018cwu} by recasting di-muon resonance, $pp\to t\bar tt \bar t$ and SUSY searches.  Two scenarios were considered: \textit{i)} a scalar LQ $R_2(3, 2, 7/6)$ combined with a vector LQ $\widetilde U_{1 \alpha}(3, 1, 5/3)$, \textit{ii)} a vector boson $Z^{\prime}$ in the singlet representation of the SM gauge group.    Ref.~\cite{Fox:2018ldq} also analyzed the HL-LHC projections for the $Z^{\prime}$.     The constraints from the LHC already rule out part of the relevant parameter space and the HL-LHC will be able to cover much of the remaining regions.    Dedicated searches in the $pp\to t\bar tt \bar t$ channel and a dedicated search for $t \mu$ resonances in $t \bar t \mu^+ \mu^-$ final state can improve the sensitivity to these models~\cite{Camargo-Molina:2018cwu}. 

\subsubsection{Conclusions regarding high \pt probes of flavour anomalies}

The anomalous results in $B$-meson decays cannot be considered yet as a convincing evidence of New Physics. On the other hand, the number (and quality) of observables that are not in complete agreement with the SM prediction is growing with time in a coherent way. If true, the implications for HEP will be profound. 
%of the flavour anomalies will have a profound impact for the physics case of the future strategies for the HEP community. 
The conclusions we draw are the following:
\begin{itemize}
\item[$\bullet$] A conservative argument based on perturbative unitarity \cite{DiLuzio:2017chi} sets an upper bound on the New Physics scale to be 9 TeV for charged current anomalies and 80 TeV for neutral current ones. 
%This does not allow to formulate a no-loose theorem for the high-luminosity program. However from 
The analysis of explicit models show that the high-luminosity program has a clear potential to probe a large portion of the possible BSM options. 

\item[$\bullet$] The explanation of the anomalies in $b \to c \tau \nu$ transitions requires non trivial model building. In particular, it is not possible to simply isolate the physics that mediate the flavour anomalous transitions. In complete models other signature have to be considered. Typically it would be very difficult to escape direct detection at HL/HE-LHC.

\item[$\bullet$] Even though the naive scale associated with the $b \to s \ell \ell$ anomalies   is much higher than the energy accessible at HL/HE-LHC, in motivated models the flavour suppressions and weak couplings guarantee a large coverage of the parameter space for leptoquark and $Z'$ mediators \cite{Allanach:2018odd}.
\end{itemize}

More generally, the probes of lepton flavour universality such as the ratios of inclusive $\tau^+\tau^-$ vs. $\mu^+\mu^-$ (or $\mu^+\mu^-$ vs. $e^+e^-$) mass distributions, are important, theoretically clean, tests of the SM and are well motivated observables both at HL- and HE-LHC, whether or not the current $B$-meson anomalies become statistically significant.



\end{document}
