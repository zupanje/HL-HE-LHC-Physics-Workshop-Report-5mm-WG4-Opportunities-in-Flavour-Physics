\subsection{High \pt searches in the context of flavour anomalies}

{\bf Authors:} Alejandro Celis, Admir Greljo, Lukas Mittnacht, Marco Nardecchia, Tevong You

Precision measurements of flavour transitions at low energies, such as flavour changing $B$, $D$ and $K$ decays, are sensitive probes of hypothetical dynamics at high energy scales.  These can provide the first evidence of new phenomena beyond the SM, even before direct discovery of new particles at high energy colliders. Indeed, the current anomalies observed in $B$-meson decays, in particular, the charged current one in $b\to c\tau \nu$ transitions, and neutral current one in $b\to s\ell^+\ell^-$,  may be the first hint of new dynamics which is still waiting to be discovered at high-\pt. When considering models that can accommodate the anomalies, it is crucial to analyse the constraints derived from high-\pt searches at the LHC, since these can often rule out significant regions of model parameter space. Below we review these constrains, and assess the impact of the High Luminosity and High Energy LHC upgrades (see also the discussion in Sections \ref{sec:7:bsll:NP},\ref{sec7:bsll:model:indep:fits}, and \ref{sec7:ModelsNP:bctaunu}).



\subsubsection{EFT analysis} 

If the dominant NP effects give rise to dimension-six SMEFT operators, the low-energy flavour measurements are sensitive to $C/\Lambda^2$, with $C$ the dimensionless NP Wilson coefficient and $\Lambda$ the NP scale.    The size of the Wilson coefficient is model dependent, and thus so is the NP scale required to explain the $R_{D^{(*)}}$and $R_{K^{(*)}}$ anomalies.    Perturbative unitarity sets an upper bound on the energy scale below which new dynamics need to appear~\cite{DiLuzio:2017chi}.    
The conservative bounds on the scale of unitarity violation are   $\Lambda_U = 9.2$ TeV and $84$ TeV for $R_{D^{(*)}}$and $R_{K^{(*)}}$, respectively, obtained when the flavour structure of NP operators is exactly aligned with what is needed to explain the anomalies.       More realistic frameworks for flavour structure, such as MFV, $U(2)$ flavour models, or partial compositeness, give rise to NP effective operators with largest effects for the third generation. This results in stronger unitarity bounds,
$\Lambda_U = 1.9$ TeV and $17$ TeV for $R_{D^{(*)}}$ and $R_{K^{(*)}}$, respectively.  
These results mean that: \textit{(i)} the mediators responsible for the $b\to c \tau \nu$ charged current anomalies are expected to be in the energy range of the LHC, \textit{(ii)}  the mediators responsible for the $b\to s \ell \ell$ neutral current anomalies could well be above the energy range of the LHC. However, in realistic flavour models also these mediators typically fall within the (HE-)LHC reach.       

If the neutrinos in $b \to c \tau \nu$ are part of a left-handed doublet, the NP responsible for $R_{D^{(*)}}$ anomaly
generically implies a sizeable signal in $p p \to \tau^+ \tau^-$ production at high-\pt. For realistic flavour structures, in which $b \to c$ transition is $\mathcal{O}(V_{cb})$ suppressed compared to $b \to b$, one expects rather large $b b \to \tau \tau$ NP amplitude. 
Schematically, $\Delta R_{D^{(*)}} \sim C_{b b \tau \tau} (1 + \lambda_{bs} / {V_{cb}})$, where $C_{b b \tau \tau}$ is the size of effective dim-6 interactions controlling $b b \to \tau \tau$, and $\lambda_{bs}$ is a dimensionless parameter controlling the size of flavour violation. Recasting ATLAS 13 TeV, $3.2$~fb$^{-1}$ search for $\tau^+ \tau^-$ \cite{Aaboud:2016cre}, Ref.~\cite{Faroughy:2016osc} showed that $\lambda_{bs} = 0$ scenario is already in slight tension with data. For $\lambda_{bs} \sim 5$, which is moderately large, but still compatible with FCNC constraints, HL (or even HE) upgrade of the LHC would be needed to cover the relevant parameter space implied by the anomaly (see Fig.~[5] in~\cite{Buttazzo:2017ixm}). 
For large $\lambda_{bs}$ the limits from $p p \to \tau^+ \tau^-$ become comparable with direct the limits on $p p \to \tau \nu$ from the bottom-charm fusion. The limits on the EFT coefficients from $p p \to \tau \nu$ were derived in Ref.~\cite{Greljo:2018tzh}, and the future LHC projections are  promising. The main virtue of this channel is that the same four-fermion interaction is compared in $b \to c \tau \nu$ at low energies and $b c \to \tau \nu$ at high-\pt. 
Since the effective NP scale in $R(D^{(*)})$ anomaly is low, the above EFT analyses are only indicative. 
For more quantitative statements we review below bounds on explicit models.

The hints of NP in $R_{K^{(*)}}$ require  a $(b s) (\ell \ell)$ interaction.
Correlated effects in high-\pt tails of $p p \to \mu^+ \mu^- (e^+ e^-)$ distributions are expected, if the numerators (denominators) of LFU ratios $R_{K^{(*)}}$ are affected. 
Ref.~\cite{Greljo:2017vvb} recast the  13~TeV $36.1$~fb$^{-1}$ ATLAS search \cite{ATLAS:2017wce} (see also \cite{Aaboud:2017buh}),
to set limits on a number of semi-leptonic four-fermion operators, and derive projections for HL-LHC (see Table 1 in~\cite{Greljo:2017vvb}). These show that direct limits on the $(b s) (\ell \ell)$ operator from the tails of distributions will never be competitive with those implied by the rare $B$-decays~\cite{Greljo:2017vvb,Afik:2018nlr}. On the other hand, flavour conserving operators, $(q q) (\ell \ell)$, are efficiently constrained by the high \pt tails of the distributions. The flavour structure of an underlining NP could thus be probed by constraining ratios $\lambda^q_{b s} = C_{b s} / C_{qq}$ with $C_{b s}$  fixed by the $R_{K^{(*)}}$ anomaly. For example, in models with MFV flavour structure, so that $\lambda^{u,d}_{b s} \sim V_{c b}$, the present high-\pt dilepton data is already in slight tension with the anomaly~\cite{Greljo:2017vvb}. Instead, if  couplings to valence quarks are suppressed, e.g., if NP dominantly couples to the 3rd family SM fermions, then $\lambda^b_{b s} \sim V_{c b}$. Such NP will hardly be probed even at the HL-LHC, and  it is possible that NP responsible for the neutral current anomaly might stay undetected in the high-\pt tails at HL-LHC and even at HE-LHC. Future data will cover a significant part of viable parameter space, though not completely, so that discovery is possible, but not guaranteed.


\subsubsection{Constraints on simplified models for $b\to c \tau \nu$}


Since the $b\to c \tau \nu$ decay is a tree-level process in the SM that receives no drastic suppression, models that can explain these anomalies necessarily require a mediator that contributes at tree-level:      


\noindent  $\bullet$ \, \textit{SM-like $W^{\prime}$}:  A SM-like $W^{\prime}$ boson, coupling to left-handed fermions, would explain the approximately equal enhancements observed in $R(D)$ and $R(D^{*})$.     A possible realization is a color-neutral real $SU(2)_L$ triplet of massive vector bosons~\cite{Greljo:2015mma}.  However, typical models encounter problems with current LHC data since they result in large contributions to $pp \to \tau^+ \tau^-$ cross-section, mediated by the neutral partner of the $W^{\prime}$~\cite{Greljo:2015mma,Boucenna:2016qad,Faroughy:2016osc}.    For $M_{W^{\prime}} \gtrsim 500$~GeV, solving the $R(D^{(*)})$ anomaly within the vector triplet model while being consistent with $\tau^+ \tau^-$ resonance searches at the LHC is only possible if the related $Z^{\prime}$ has a large total decay width~\cite{Faroughy:2016osc}. Focusing on the $W^{\prime}$, Ref.~\cite{Abdullah:2018ets} analyzed the production of this mediator via $gg$ and $gc$ fusion, decaying to
$\tau \nu_{\tau}$. Ref.~\cite{Abdullah:2018ets} concluded that a dedicated search using that $b$-jet is present in the final state would be effective in reducing the SM background compared to an inclusive analysis that relies on $\tau$-tagging and $E_T^{\rm{miss}}$. Nonetheless, relevant limits will be set by an inclusive search in the future~\cite{Greljo:2018tzh}.


\vspace{0.2cm}

\noindent  $\bullet$  \,  \textit{Right-handed $W^{\prime}$:} Refs.~\cite{Greljo:2018ogz,Asadi:2018wea} recently proposed that $W^{\prime}$ could mediate a right-handed interaction, with a light sterile right-handed neutrino carrying the missing energy in the $B$ decay. In this case, it is possible to completely decorelate FCNC constraints from $R(D^{(*)})$. The most constraining process in this case is instead $p p \to \tau \nu$. Ref.~\cite{Greljo:2018tzh} performed a recast of the latest ATLAS and CMS searches at 13 TeV and about 36 fb$^{-1}$ to constrain most of the relevant parameter space for the anomaly.

\vspace{0.2cm}

\noindent  $\bullet$\, \textit{Charged Higgs $H^{\pm}$:}       Models that introduce a charged Higgs, for instance a two-Higgs-doublet model, also contain additional neutral scalars.  Their masses are constrained by electroweak precision measurements to be close to that of the charged Higgs.  Accommodating the $R(D^{(*)})$ anomalies with a charged Higgs typically implies large new physics contributions to $pp \to \tau^+ \tau^-$ via the neutral scalar exchanges, so that current LHC data can challenge this option~\cite{Faroughy:2016osc}.     Note that a charged Higgs also presents an important tension between the current measurement of $R(D^*)$ and the measured lifetime of the $B_c$ meson~\cite{Li:2016vvp,Alonso:2016oyd,Celis:2016azn,Akeroyd:2017mhr}.


\begin{figure}[t]
\centering
\includegraphics[width=0.39 \textwidth]{section10/img/LQ-search-strategy.pdf}
\caption{Schematic of the LHC bounds on  LQ showing complementarity  in constraining the  $(m_{LQ},y_{q_{\ell}})$ parameters. The three cases are:
 pair production $\sigma \propto y_{q_{\ell}}^0 $, single production $\sigma \propto y_{q_{\ell}}^2 $ and Drell-Yan $\sigma \propto y_{q_{\ell}}^4$ (from~\cite{Dorsner:2018ynv}).}
\label{fig:LQstrategy}
\end{figure}

\begin{figure}[t]
\centering
\includegraphics[width=0.46 \textwidth]{section10/img/S3.pdf}~~~
\includegraphics[width=0.46 \textwidth]{section10/img/U1.pdf}
\caption{ \small \Blu{Present constraints and HL(HE)-LHC projections in the leptoquark mass versus coupling  plane for the scalar leptoquark $S_3$ (left), and vector leptoquark $U_1$ (right). The grey and dark grey solid regions are the current exclusions. The grey and black dashed lines are the projected reach for HL-LHC (pair and single leptoquark production prospects are based on the CMS simulation from Section XXX). The red dashed lines are the projected reach at HE-LHC (see Section \ref{sec:HE_LHC_LQ}). The green and yellow bands are the $1\sigma$ and $2\sigma$ preferred regions from the fit to $B$ physics anomalies. The second coupling required to fit the anomaly does not enter in the leading high-\pt diagrams but it is relevant for fixing the preferred region shown in green, for more details see Ref.~\cite{Buttazzo:2017ixm}. 
}
}
\label{fig:LQmediators}
\end{figure}

\vspace{0.2cm}

\noindent  $\bullet$ \, \textit{Leptoquarks:} 

The observed anomalies in charged and neutral currents appear in semileptonic decays of the $B$-mesons. This implies that the putative NP has to couple to both quarks and leptons at the fundamental level. A natural BSM option is to consider mediators that couple simultaneously quarks and leptons at the tree level. Such states are commonly referred as leptoquarks. Decay and production mechanisms of the LQ are directly linked to the physics required to explain the anomalous data.
\begin{itemize}
\item {\bf Leptoquark decays:} the fit to the $R(D^{*})$ observables suggest a rather light leptoquark (at the TeV scale) that couples predominately to the third generation fermions of the SM. A series of constraints from flavour physics, in particular the absence of BSM effects in kaon and charm mixing observables, reinforces this picture. 

\item {\bf Leptoquark production mechanism:}  The size of the couplings required to explain the anomaly is typically very large, roughly $y_{q\ell} \approx m_{LQ}$/ (1 TeV). Depending of the actual sizes of the leptoquark couplings and its mass we can distinguish three regimes that are relevant for the phenomenology at the LHC:
 	\begin{enumerate}
	\item LQ pair production due to strong interactions,
	\item Single LQ production plus lepton via a single insertion of the LQ coupling, and
	\item Non-resonant production of di-lepton through $t$-channel exchange of the leptoquark.
	\end{enumerate}
Interestingly all three regimes provide complementary bounds in the  $(m_{LQ},y_{q_{\ell}})$ plane, see Fig.~\ref{fig:LQstrategy}.

\end{itemize}




Several simplified models with leptoquark as a mediator were shown to be consistent with the low-energy data.
 A vector leptoquark with $SU(3)_c\times SU(2)_L\times U(1)_Y$ SM quantum numbers $U_\mu \sim ({\bf 3}, {\bf 1}, 2/3)$ was identified as the only single mediator model which can simultaneously fit the two anomalies (see e.g.~\cite{Buttazzo:2017ixm} for a recent fit including leading RGE effects). 
 In order to substantially cover the relevant parameter space, one needs future HL- (HE-) LHC, see Fig.~\ref{fig:LQmediators} (right) (see also Fig.~5 of~\cite{Buttazzo:2017ixm} for details on the present LHC constraints). A similar statement applies to an alternative model featuring two scalar leptoquarks, $S_1, S_3$ \cite{Crivellin:2017zlb}. The pair of plots in Fig. \ref{fig:LQmediators} summaries the current exclusion and the discovery reach for the HE and HL-LHC in the LQ coupling versus mass plane. 
 
 \Blu{Leptoquarks states are emerging as the most convincing mediators for the explanations of the flavour anomalies. It is then important to explore all the possible signatures at the HL- and HE-LHC.
 The experimental program should focus not only on final states containing quarks and leptons of the third generation, but also on the whole list of decay channels including the off-diagonal ones ($b \mu$, $s \tau, \dots$). The completeness of this approach would allow to shed light on the flavour structure of the putative New Physics. 
 
 Another aspects to be emphasized concerning leptoquark models for the anomalies is the fact of the possible presence of extra fields required to complete the UV lagrangian. The accompanying particles would leave more important signatures at high $p_T$ than the leptoquark, this is particularly true for vector leptoquark extensions (see for example \cite{DiLuzio:2017vat,DiLuzio:2018zxy})}  
    
 
%\Blu{As a final remark, it is important to remember that the anomalies are not yet experimentally established. Among others, this also means that the statements on whether or not  the high \pt LHC constraints rule out certain $R(D^{(*)})$ explanations assumes that the actual values of $R(D^{(*)})$ are given by their current global averages. If future measurements decrease the global average, the high \pt constraints can in some cases be greatly relaxed and HL- and/or HE-LHC may be essential for these, at present tightly constrained, cases.}


%%%%%%%%%%%
\subsubsubsection{HE-LHC sensitivity study for $p p \to \Phi \Phi \to (b \tau) (b \tau)$}
\label{sec:HE_LHC_LQ}
\begin{figure}[t]
	\centering
	\includegraphics[width=0.5\textwidth]{section10/img/soverb.pdf}
	\caption{Expected sensitivity for pair production of scalar (\Blu{$S_3$}) and vector ($U_1$) leptoquark at 27 TeV $p p$ collider with an integrated luminosity of $15~\text{ab}^{-1}$.} \label{fig:lukas}
\end{figure}

We analysed the sensitivity of the 27 TeV $pp$ collider with $15$~ab$^{-1}$ integrated luminosity to probe pair production of the scalar and vector leptoquarks decaying to $(b \tau)$ final state. We investigated events containing either one electron or muon, one hadronically decaying tau lepton, and  at least two jets. The signal events and the dominant background events ($t\bar{t}$) were generated with {\tt MadGraph5\_aMC@NLO} at leading-order. {\tt PYTHIA6} was used to shower and hadronise events and {\tt DELPHES3} was used to simulate the detector response. The scalar leptoquark (\Blu{$S_3$}) and the vector leptoquark ($U_1$) UFO model files were taken from~\cite{Dorsner:2018ynv}.

To verify the procedure, we simulated the $t\bar{t}$ background and the scalar leptoquark signal at 13 TeV and compared to the predicted shapes in the $S_T$ distribution from the existing CMS analysis~\cite{Sirunyan:2017yrk}. After we verified the 13 TeV analysis, we simulated the signal and the dominant background events at $27$ TeV. From these samples, we selected all events satisfying the particle content requirements and applied the lower cut in the $S_T$ variable. The cut threshold was chosen to maximize $s/\sqrt{b}$ while requiring at least 2 expected signal events at an integrated luminosity of $15~\text{ab}^{-1}$. In the case of the vector leptoquark we considered the Yang-Mills ($\kappa = 1$) and the minimal coupling ($\kappa = 0$) scenarios for the couplings to the gluon field strength, ${\cal L}\supset -ig \kappa U_{1\mu}^\dagger T^a U_{1\nu} G^a_{\mu\nu}$~\cite{Dorsner:2018ynv}. From the simulations of the scalar and vector leptoquark events, we found the ratio of the cross-sections, and assuming similar kinematics, we estimated the sensitivity also for the vector leptoquark $U_1$.

As shown in Fig.~\ref{fig:lukas}, the HE-LHC collider will be able to probe pair produced third generation scalar leptoquark (decaying exclusively to $b \tau$ final state) up to mass of $\sim 4$~TeV and vector leptoquark up to $\sim 4.5$~TeV and $\sim 5.2$~TeV for the minimal coupling and Yang-Mills scenarios, respectively. While this result is obtained by a rather crude analysis, it shows the impressive reach of the future high-energy $pp$-collider. In particular, the HE-LHC will cut deep into the relevant perturbative parameter space for $b \to c \tau \nu$ anomaly. As a final comment, this is a rather conservative estimate of the sensitivity to leptoquark models solving $R(D^{*})$ anomaly, since a dedicated single leptoquark production search is expected to yield even stronger bounds~\cite{Buttazzo:2017ixm,Dorsner:2018ynv}.
%%%%%%%%%%%







\subsubsection{Constraints on simplified models for $b\to s ll$}

The $b\to s ll$ transition is  both loop and CKM suppressed in the SM. The explanations of the $b\to s ll$ anomalies can thus have both tree level and loop level mediators.
% enter at tree-level or at the loop level. 
Loop-level explanations typically involve lighter particles.
% that enter at a lower UV cut-off scale. 
Tree-level mediators can also be light, if sufficiently weakly coupled. However, they can also be much heavier---possibly beyond the reach of the LHC. 

%\begin{figure}[t]
%\centering
%\includegraphics[width=0.45 %\textwidth]{section10/img/feynmandiagrams_LQZprime.PNG}
%\caption{Tree level Feynman diagrams for leptoquarks (left) and $Z^\prime$ (right)  mediating $b \to s \mu^+ \mu^-$ transition (Fig. taken from ~\cite{Allanach:2017bta}).}
%\label{fig:feynmanLQZprime}
%\end{figure}

\noindent $\bullet$ \textit{Tree-level mediators:}\\
%For the four-fermion operators involved in 
For $b \to sll$ anomaly there are two possible tree-level UV-completions, the $Z'$ vector boson and leptoquarks, either scalar or vector (see Fig.~\ref{fig:bsll_diagrams} in Sec.~\ref{sec:7:bsll:NP}).
% whose Feynman diagrams are depicted in Fig.~\ref{fig:feynmanLQZprime}. The $Z^\prime$ vector boson mediates an interaction between the $b$ and $s$ quark at one vertex and two muons at the other. A leptoquark is defined as having quantum numbers such that it may interact with a quark and lepton at one vertex, and can be either scalar or vector. In this case the minimal couplings required of the leptoquark must contain a $b$ or $s$ together with a muon at a vertex. 
%Ref.~\cite{Allanach:2017bta} 
%%initiated the study of implications of the anomalies for direct discovery at future colliders. They 
%estimated the sensitivity to simplified models of $Z^\prime$ and leptoquarks at the HL- and HE-LHC (as well as the FCC-hh) in the expected parameter space where the new particles may explain the neutral current flavour anomalies. A more detailed study for the $Z^\prime$ case was then made by Ref.~\cite{Allanach:2018odd}, extending in particular to wider widths. We summarise their results here. 
For leptoquarks, Fig.~\ref{fig:LQexclusion} shows the current 95\% CL limits from 8 TeV CMS with 19.6 fb$^{-1}$  in the $\mu\mu jj$ final state (solid black line), as well as the HL-LHC (dashed black line) and 1 (10) ab$^{-1}$ HE-LHC extrapolated limits (solid (dashed) cyan line). 
%is denoted by a dashed black line for HL-LHC and by a solid (dashed) cyan line for HE-LHC with 1 (10) ab$^{-1}$. 
Dotted lines give the cross-sections times branching ratio at the corresponding collider energy for pair production of scalar leptoquarks, calculated at NLO using the code of Ref.~\cite{Kramer:2004df}. We see that the sensitivity to a leptoquark with only the minimal $b-\mu$ and $s-\mu$ couplings reaches around 2.5 and 4.5 TeV at the HL-LHC and HE-LHC, respectively. This pessimistic estimate is a lower bound that will typically be improved in realistic models with additional flavour couplings. Moreover, the reach can be extended by single production searches~\cite{Allanach:2017bta}, albeit in a more model-dependent way than pair production. The cross section predictions for vector leptoquark are more model-dependent and are not shown in Fig.~\ref{fig:LQexclusion}. For $\mathcal{O}(1)$ couplings the corresponding limits are typically stronger than for scalar leptoquarks. 

\begin{figure}[t]
\centering
\includegraphics[width=0.45 \textwidth]{section10/img/LQ_exclusionplot_HLLHC_HELHC.png}
\caption{Current and projected 95\% CL limits on $\mu\mu jj$ final state at CMS (solid black) and HL-LHC (cyan) with 1 (10) ab$^{-1}$ in solid (dashed) lines. The pair production cross-section for scalar leptoquarks is shown in dotted lines for 14 (27) TeV in black (cyan) (Fig. taken from ~\cite{Allanach:2017bta}). \Blu{See also Section (XXX-->Cambridge group)}}
\label{fig:LQexclusion}
\end{figure}

\begin{figure}[t]
\centering
\includegraphics[width=0.39 \textwidth]{section10/img/14TeVMDM.pdf}
\caption{ HL-LHC 95\% (blue) and 99\% (orange) CL sensitivity contours to $Z^\prime$ in the ``mixed-down'' model for $g_{\mu\mu}$ vs $Z^\prime$ mass in TeV. The dashed grey contours give the $Z'$ width as a fraction of mass. The green and red regions are excluded by trident neutrino production and $B_s$ mixing, respectively. The dashed blue line is the stronger $B_s$ mixing constraint from Ref.~\cite{Bazavov:2016nty} (Fig. taken from~\cite{Allanach:2018odd}). \Blu{See also Section (XXX-->Cambridge group)} }
\label{fig:Zprime14TeV}
\end{figure}

For the $Z^\prime$ mediator the minimal couplings in the mass eigenstate basis are obtained by unitary transformations from the gauge eigenstate basis, which necessarily induces other couplings.  Ref.~\cite{Allanach:2018odd} defined the ``mixed-up'' model (MUM) and ``mixed-down'' model (MDM) such that the minimal couplings are obtained via CKM rotations in either the up or down sectors respectively. For MUM there is no sensitivity at the HL-LHC.  The predicted sensitivity at the HL-LHC for the MDM is shown in Fig.~\ref{fig:Zprime14TeV} as functions of $Z^\prime$ muon coupling $g_{\mu\mu}$ and the $Z^\prime$ mass, setting the $Z'$ coupling to $b$ and $s$ quarks such that it solves the $b\to s\ell\ell$ anomaly. The solid blue (orange) contours give the 95\% and 99\% CL sensitivity. The red and green regions are excluded by $B_s$ mixing~\cite{Arnan:2016cpy} and neutrino trident production~\cite{Falkowski:2017pss,Falkowski:2018dsl}, respectively. The more stringent $B_s$ mixing constraint from Ref.~\cite{Bazavov:2016nty} is denoted by the dashed blue line; see, however, Ref.~\cite{Kumar:2018kmr} for further discussion regarding the implications of this bound. The dashed grey contours denote the width as a fraction of the mass. We see that the HL-LHC will only be sensitive to $Z^\prime$ with narrow width, up to masses of 5 TeV. 

\begin{figure}[t]
\centering
\includegraphics[width=0.39 \textwidth]{section10/img/27TeVMUM.pdf}~~~~~~~~~
\includegraphics[width=0.39 \textwidth]{section10/img/27TeVMDM.pdf}
\caption{ HE-LHC 95\% (blue) and 99\% (orange) CL sensitivity contours to $Z^{\prime}$ in the ``mixed-up'' (left) and ``mixed-down'' (right) model in the parameter space of $g_{\mu\mu}$ vs $Z^{\prime}$ mass in TeV. The dashed grey contours are the width as a fraction of mass. The green and red regions are excluded by trident neutrino production and $B_s$ mixing. The dashed blue line is the stronger $B_s$ mixing constraint from Ref.~\cite{Bazavov:2016nty} (Figs. taken from~\cite{Allanach:2018odd}). \Blu{See also Section (XXX-->Cambridge group)}}
\label{fig:Zprime27TeV}
\end{figure}

At the HE-LHC, the reach for 10 ab$^{-1}$ is shown in Fig.~\ref{fig:Zprime27TeV} for the MUM and MDM on the left and right, respectively. In this case the sensitivity may reach a $Z^{\prime}$ with wider widths up to 0.25 and 0.5 of its mass, while the mass extends out to 10 to 12 TeV. We stress that this is a pessimistic estimate of the projected sensitivity, particular to the two minimal models; more realistic scenarios will typically be easier to discover. 


\vspace{0.2cm}

\noindent  $\bullet$ \textit{Explanations at the one-loop level:}\\
It is possible to accomodate the $b \to s \ell^+ \ell^-$ anomalies even if mediators only enter at one-loop.  One possibility is the mediators coupling to right-handed top quarks and to muons~\cite{Celis:2017doq,Becirevic:2017jtw,Kamenik:2017tnu,Fox:2018ldq,Camargo-Molina:2018cwu}.   Given the loop and CKM  suppression of the NP contribution to the $b \to s \ell^+ \ell^-$ amplitude, these models can explain the $b \to s \ell^+ \ell^-$ anomalies for a light mediator, with mass around $\mathcal{O}(1)$~TeV or lighter.       Constraints from the LHC and future projections for the HL-LHC were derived in~\cite{Camargo-Molina:2018cwu} by recasting di-muon resonance, $pp\to t\bar tt \bar t$ and SUSY searches.  Two scenarios were considered: \textit{i)} a scalar LQ $R_2(3, 2, 7/6)$ combined with a vector LQ $\widetilde U_{1 \alpha}(3, 1, 5/3)$, \textit{ii)} a vector boson $Z^{\prime}$ in the singlet representation of the SM gauge group.    Ref.~\cite{Fox:2018ldq} also analyzed the HL-LHC projections for the $Z^{\prime}$.     The constraints from the LHC already rule out part of the relevant parameter space and the HL-LHC will be able to cover much of the remaining regions.    Dedicated searches in the $pp\to t\bar tt \bar t$ channel and a dedicated search for $t \mu$ resonances in $t \bar t \mu^+ \mu^-$ final state can improve the sensitivity to these models~\cite{Camargo-Molina:2018cwu}. 

\subsubsection{Conclusions regarding high \pt probes of flavour anomalies}

The anomalous results in $B$-meson decays cannot be considered yet as a convincing evidence of New Physics. On the other hand, the number (and quality) of observables that are not in complete agreement with the SM prediction is growing with time in a coherent way. If true, the implications for HEP will be profound. 
%of the flavour anomalies will have a profound impact for the physics case of the future strategies for the HEP community. 
The conclusions we draw are the following:
\begin{itemize}
\item[$\bullet$] A conservative argument based on perturbative unitarity \cite{DiLuzio:2017chi} sets an upper bound on the New Physics scale to be 9 TeV for charged current anomalies and 80 TeV for neutral current ones. 
%This does not allow to formulate a no-loose theorem for the high-luminosity program. However from 
The analysis of explicit models show that the high-luminosity program has a clear potential to probe a large portion of the possible BSM options. 

\item[$\bullet$] The explanation of the anomalies in $b \to c \tau \nu$ transitions requires non trivial model building. In particular, it is not possible to simply isolate the physics that mediate the flavour anomalous transitions. In complete models other signature have to be considered. Typically it would be very difficult to escape direct detection at HL/HE-LHC.

\item[$\bullet$] Even though the naive scale associated with the $b \to s \ell \ell$ anomalies   is much higher than the energy accessible at HL/HE-LHC, in motivated models the flavour suppressions and weak couplings guarantee a large coverage of the parameter space for leptoquark and $Z'$ mediators \cite{Allanach:2018odd}.
\end{itemize}

More generally, the probes of lepton flavour universality such as the ratios of inclusive $\tau^+\tau^-$ vs. $\mu^+\mu^-$ (or $\mu^+\mu^-$ vs. $e^+e^-$) mass distributions, are important, theoretically clean, tests of the SM and are well motivated observables both at HL- and HE-LHC, whether or not the current $B$-meson anomalies become statistically significant.

