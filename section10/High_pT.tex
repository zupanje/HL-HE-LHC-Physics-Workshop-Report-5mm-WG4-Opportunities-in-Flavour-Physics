\subsection{High $p_T$ searches in the context of flavour anomalies}
%in collaboration with WG3
{\bf Author(s): Alejandro Celis, Tevong You, Marco Nardecchia, Admir Greljo} {\it (anticipated length 10 pages)}

Precision measurements of flavour transitions at low energies, such as flavour changing $B$, $D$ and $K$ decays, are sensitive probes of hypothetical dynamics at high energy scales.  These can provide the first evidence of new phenomena beyond the SM, even before direct discovery of new particles at high energy colliders. Indeed, the current anomalies observed in $B$-meson decays, the charge current one in $b\to c\tau \nu$ transitions, and neutral current one in $b\to s\ell^+\ell^-$,  may be the first hint of new dynamics which is still waiting to be discovered at high-$p_T$. When considering models that can accommodate the anomalies, it is crucial to analyse the constraints derived from high-$p_T$ searches at the LHC, since these can often rule out significant regions of model parameter space. Below we reviewed these constrains, and assess the impact of the High Luminosity and High Energy LHC upgrades.
% in this context.


\subsubsection{EFT analysis} 

Assuming that the dominant NP effects give rise to dimension six operators of the SMEFT, the low energy flavour measurements are sensitive to 
%the NP through combinations of the form 
$C/\Lambda^2$, with $C$ the dimensionless NP Wilson coefficient and $\Lambda$ the NP scale.    The size of the Wilson coefficients is model dependent, and thus so is the NP scale required to explain the $R_{D^{(*)}}$ and $R_{K^{(*)}}$ anomalies.    However, perturbative unitarity does set an upper bound on the energy scale below which new dynamics needs to appear~\cite{DiLuzio:2017chi}.    
%When the NP operators responsible for explaining the $R_{D^{(*)}}$ and $R_{K^{(*)}}$ anomalies are aligned in the direction of the relevant flavour eigenstates one derives a 
The conservative bounds on the scale of unitarity violation are   $\Lambda_U = 9.2$ TeV and $84$ TeV for $R_{D^{(*)}}$ and $R_{K^{(*)}}$, respectively, obtained when the flavour structure NP operators is exactly aligned with what is needed to explain the anomalies.       More realistic frameworks for flavour structure, such as minimal flavor violation, $U(2)$ flavour models, or partial compositeness, give rise to NP effective operators with largest effects for the third generation. This results in stronger unitarity bounds,
%  In the case of third generation alignment in the quark sector one derives 
$\Lambda_U = 1.9$ TeV and $17$ TeV for $R_{D^{(*)}}$ and $R_{K^{(*)}}$, respectively.  
%Two lessons can be drawn from these results:  
These results mean that: \textit{i)} the mediators responsible for the charged current anomalies are expected to be in the energy range of the LHC, \textit{ii)}  the mediators responsible for the neutral current anomalies could well be above the energy range of the LHC. However, in realistic flavour models also these mediators typically fall within the (HE-)LHC reach.       

The first step towards explicit mediator models is to estimate the sensitivity of measuring the effects of $SU(2)_L$ gauge invariant dimension-6 operators on the tails of di-leptons produced at HL/HE LHC, i.e. $p p \to e^+ e^-$, $\mu^+ \mu^-$, and $\tau^+ \tau^-$. Assuming the mediator mass to be kinematically inaccessible for an on-shell production, this complementary test of $B$-anomalies is rather general, as it directly compares semi-leptonic four-fermion operators affecting the high-$p_T$ di-lepton production to those involved in $B$-decays. Such a comparison is possible thanks to the well-known properties of the effective theory (RGE, etc.). Apart from the flavour structure, it is not required to specify the details of an underlining UV completion.

In this context, it is instructive to start with an argument put forward in Ref.~\cite{Faroughy:2016osc}. Quite generically, $SU(2)_L$ gauge symmetry, with neutrinos being a part of the left-handed lepton doublet, implies that the new physics responsible for $R_{D^{(*)}}$ anomaly (i.e. $b \to c \tau \nu$ transitions) is also expected to induce a sizeable signal in $p p \to \tau^+ \tau^-$ production at high-$p_T$. In fact, assuming the realistic flavour structure, in which $b \to c$ transition is $\mathcal{O}(V_{cb})$ suppressed compared to $b \to b$, one expects rather large $b b \to \tau \tau$. 
Schematically, $\Delta R(D^{(*)}) \sim C_{b b \tau \tau} (1 + \lambda_{bs} / {V_{cb}})$, where $C_{b b \tau \tau}$ is the size of effective dim-6 interactions controlling $b b \to \tau \tau$, and $\lambda_{bs}$ is a dimensionless parameter controlling the size of flavour violation. Recasting ATLAS 13 TeV, $3.2$~fb$^{-1}$ search for $\tau^+ \tau^-$, Ref.~\cite{Faroughy:2016osc} showed that $\lambda_{bs} = 0$ scenario is already in slight tension with the data. However, as shown in Ref.~\cite{Buttazzo:2017ixm}, for $\lambda_{bs} \sim 5$ (moderately large, but still compatible with FCNC), HL (or even HE) upgrade of the LHC would be needed to cover the relevant parameter space implied by the anomaly (see Fig.~[5] in~\cite{Buttazzo:2017ixm}). 
Large $\lambda_{bs}$ limit diminishes the dominance of $p p \to \tau^+ \tau^-$ with respect to direct $p p \to \tau \nu$ from the bottom-charm fusion. The limits on the EFT coefficients for the latter process have been derived in Ref.~\cite{Jose-Jorge-Admir}, and the future LHC projections are quite promising. The main virtue of this channel is that the same four-fermion interaction is compared in $b \to c \tau \nu$ at low energies and $b c \to \tau \nu$ at high-$p_T$. As a final comment, since the implied effective NP scale in $R(D^{(*)})$ anomaly is quite low, the EFT analyses are only an indication of sensitivity. In fact, explicit models should be considered for more quantitative statements (see discussion below).

Hinted new physics in the neutral current anomaly (i.e. $R_{K^{(*)}}$) at low-energies materialises as a local $(b s) (\ell \ell)$ interaction. Correlated effects in the high-p$_T$ tails of $p p \to \mu^+ \mu^-$ or $e^+ e^-$ are expected depending on whether the numerator or denominator of the LFU ratios are affected. This connection has been investigated in Ref.~\cite{Greljo:2017vvb}, where ATLAS search, performed at 13~TeV with $36.1$~fb$^{-1}$ of data, was recast to set limits on a number of semi-leptonic four-fermion operators, as well as, to derive projections for HL-LHC (see Table 1 in~\cite{Greljo:2017vvb}). To start with, direct limits on the $(b s) (\ell \ell)$ operator from the tails will never be competitive to those implied by the rare $B$-decays~\cite{Greljo:2017vvb,Afik:2018nlr}. On the other hand, flavour conserving operators, $(q q) (\ell \ell)$, are efficiently constrained in the tails. Therefore, flavour structure of an underlining new physics could be probed by constraining ratios $\lambda^q_{b s} = C_{b s} / C_{qq}$ where $C_{b s}$ is assumed to be fixed by the anomaly. For example, in models which follow MFV, $\lambda^{u,d}_{b s} \sim V_{c b}$, the present high-$p_T$ dilepton data is already in a slight tension with the anomaly~\cite{Greljo:2017vvb}. However, if instead couplings to valence quarks are suppressed, e.g. dominant new physics in the 3rd family fermions $\lambda^b_{b s} \sim V_{c b}$, will hardly be probed even at the HL-LHC. It is possible that new physics in the neutral current anomaly might stay undetected in the high-$p_T$ tails at HL- (even HE-) LHC. Nonetheless, the future data will cover a significant (but not complete) part of viable parameter space.


\subsubsection{Constraints on simplified models for $b\to c \tau \nu$}


Since the $b\to c \tau \nu$ decay is a tree-level process in the SM that receives no particular suppression, models that accomodate these anomalies introduce a mediator that contributes at tree-level.      


\noindent  $\bullet$ \, \textit{SM-like $W^{\prime}$}:  Explaining the $R(D^{(*)})$ anomalies with a SM-like $W^{\prime}$ boson, coupling to left-handed fermions, is an interesting proposal as it would explain the similar enhancement observed in $R(D)$ and $R(D^{*})$.     A possible realization is to have a color-neutral real $\mathrm{SU(2)}_L$ triplet of massive vector bosons~\cite{Greljo:2015mma}.  However, typical models encounter problems with current LHC data as they introduce a large cross-section for $pp \to \tau^+ \tau^-$ mediated by the neutral partner of the $W^{\prime}$~\cite{Greljo:2015mma,Boucenna:2016qad,Faroughy:2016osc}.    For $M_{W^{\prime}} \gtrsim 500$~GeV, solving the $R(D^{(*)})$ anomaly within the vector triplet model while being consistent with $\tau^+ \tau^-$ resonance searches at the LHC is only possible for a very large $Z^{\prime}$ total decay width~\cite{Faroughy:2016osc}. Focusing on the $W^{\prime}$; Ref.~\cite{Abdullah:2018ets} analyzed the production of this mediator via $gg$ and $gc$ fusion, decaying to
$\tau \nu_{\tau}$. Ref.~\cite{Abdullah:2018ets} concludes that a dedicated search using information of the $b$-jet present in the final state would be effective in reducing the SM background compared to an inclusive analysis that relies on $\tau$-tagging and $E_T^{\rm{miss}}$.


\vspace{0.2cm}

\noindent  $\bullet$  \,  \textit{Right-handed $W^{\prime}$:} It has been recently proposed in Refs.~\cite{Greljo:2018ogz,Asadi:2018wea} that $W^{\prime}$ could instead mediate right-handed interaction assuming a light sterile right-handed neutrino carrying the missing energy in the $B$ decay. In this case, it is possible to completely decorelate FCNC constraints from $R(D^{(*)})$. The most constraining process in this case is instead $p p \to \tau \nu$. Future LHC data will be needed to completely cover the relevant parameter space (see Fig.~[3] in~\cite{Greljo:2018ogz}).

\vspace{0.2cm}


\noindent  $\bullet$\, \textit{Charged Higgs $H^{\pm}$:}       Models that introduce a charged Higgs, for instance a two-Higgs-doublet model, also contain additional neutral scalars.  The mass of the latter is constrained by electroweak precision measurements to be close to that of the charged Higgs.  Accommodating the $R(D^{(*)})$ anomalies with a charged Higgs typically implies large new physics contributions to $pp \to \tau^+ \tau^-$ via the neutral scalar exchanges, ruling out these models with current LHC data~\cite{Faroughy:2016osc}.     Note that a charged Higgs also presents an important tension between the current measurement of $R(D^*)$ and the measured lifetime of the $B_c$ meson~\cite{Li:2016vvp,Alonso:2016oyd,Celis:2016azn,Akeroyd:2017mhr}. 



\vspace{0.2cm}

\noindent  $\bullet$ \, \textit{Leptoquarks:} 

The observed anomalies in charged and neutral currents appear in semileptonic decays of the B-mesons. This implies that the putative NP has to couple to both quarks and leptons at the fundamental level. A natural BSM option is to consider mediators that couple simultaneously quarks and leptons at the tree level. Such states are commonly referred as leptoquarks. Decay and production mechanisms of the LQ are directly linked to the physics required to explain the anomalous data.
\begin{itemize}
\item {\bf Leptoquark decays:} the fit to the $R(D^{*})$ observables suggest a rather light leptoquark (at the TeV scale) that couples predominately to third generation fermions of the SM. A series of constraints from flavour physics (in particular the absence of BSM effects in kaon and charm mixing observables) reinforces this picture. 

\item {\bf Leptoquark production mechanism:}  The size of the couplings required to explain the anomaly is typically very large, qualitatively is given by $y_{q\ell} \approx m_{LQ}$/ (1 TeV). Depending of the actual size of the leptoquark coupling and mass value we can distinguish three regimes that are relevant for the phenomenology at the LHC:
 	\begin{enumerate}
	\item LQ pair production due to strong interactions,
	\item Single LQ production plus lepton via a single insertion of the LQ coupling, and
	\item Non-resononant production of di-lepton through t-channel exchange of the leptoquark.
	\end{enumerate}
Interestingly all the three regimes provide complementary bounds in the  $(m_{LQ},y_{q_{\ell}})$ plane (see discussion in Sec.~4 of Ref.~\cite{Dorsner:2018ynv}).

\begin{figure}[t]
\centering
\includegraphics[width=0.39 \textwidth]{section10/img/LQ-search-strategy.pdf}
\caption{Complementary of the bounds from LQ processes at the LHC on the $(m_{LQ},y_{q_{\ell}})$ parameter space. The three cases are:
 Pair production $\sigma \propto y_{q_{\ell}}^0 $, Single production $\sigma \propto y_{q_{\ell}}^2 $ and Drell-Yann $\sigma \propto y_{q_{\ell}}^4$.}
\label{fig:LQstrategy}
\end{figure}
\end{itemize}

Few simplified models with leptoquark as a mediator were shown to be consistent with the low-energy data. In addition, the high-$p_T$ phenomenology of these models including projections was discussed. In Ref.~\cite{Buttazzo:2017ixm}, a vector leptoquark with SM quantum numbers $U_\mu \sim ({\bf 3}, {\bf 1}, 2/3)$ was identified as an exceptional single mediator model which can fit the two anomalies simultaneously. The LHC limits from complementary processes are shown in Fig.~[4] of~\cite{Buttazzo:2017ixm}. In order to substantially cover the relevant parameter space, one needs future HL- (HE-) LHC. A similar statement is valid for a competitor model featuring two scalar leptoquarks (see Fig.~[2] in~\cite{Marzocca:2018wcf}. A comparison between the reach of CLIC and LHC is shown Fig.~\ref{fig:LQmediators}.

\begin{figure}[t]
\centering
\includegraphics[width=0.39 \textwidth]{section10/img/S3.pdf}
\includegraphics[width=0.39 \textwidth]{section10/img/U1.pdf}
\caption{ \small Comparison between the reach of CLIC and LHC in the $m_{\rm LQ}-g_3$ plane. The gray region represents the parameter space where the expected total number of events for leptoquark production in the $b\tau$ channel will be larger than 10 at 3~TeV CLIC with $3\, {\rm ab}^{-1}$ of luminosity; the red region is the present exclusion from LHC, combining pair production (vertical bound) and $\tau\tau$ searches; the red dashed line is the reach of HL-LHC. The green and yellow bands are the $1\sigma$ and $2\sigma$ preferred regions from the flavour fit. 
Left: scalar leptoquark $S_3$, Right: vector leptoquark $U_1$.}
\label{fig:LQmediators}
\end{figure}

\subsubsection{Constraints on simplified models for $b\to s ll$}

As $b\to s ll$ in the SM is loop and CKM suppressed, it is possible in principle to have explanations of these anomalies with mediators that enter at tree-level or at the loop level. Loop-level explanations typically involve lighter particles that enter at a lower UV cut-off scale. Tree-level mediators can also be light if sufficiently weakly coupled, or they can be much heavier---possibly beyond the reach of the LHC. 


\noindent $\bullet$ \textit{Tree-level mediators:}\\

\begin{figure}[h!]
\centering
\includegraphics[width=0.39 \textwidth]{section10/img/feynmandiagrams_LQZprime.PNG}
\caption{Feynman diagrams for $Z^\prime$ and leptoquarks mediating $b \to s \mu^+ \mu^-$ interactions.}
\label{fig:feynmanLQZprime}
\end{figure}

For the four-fermion operator involved in $b \to sll$, there are two possible UV-completions at tree-level, whose Feynman diagrams are depicted in Fig.~\ref{fig:feynmanLQZprime}. The $Z^\prime$ vector boson mediates an interaction between the $b$ and $s$ quark at one vertex and two muons at the other. A leptoquark is defined as having quantum numbers such that it may interact with a quark and lepton at one vertex, and can be either scalar or vector. In this case the minimal couplings required of the leptoquark must contain a $b$ or $s$ together with a muon at a vertex. 

Ref.~\cite{Allanach:2017bta} initiated the study of implications of the anomalies for direct discovery at future colliders. They estimated the sensitivity to simplified models of $Z^\prime$ and leptoquarks at the high-luminosity and high-energy upgrades of the LHC (as well as the FCC-hh) in the expected parameter space where the new particles may explain the neutral current flavour anomalies. A more detailed study for the $Z^\prime$ case was then made by Ref.~\cite{Allanach:2018odd}, extending in particular to wider widths. We summarise their results here. 

\begin{figure}[t]
\centering
\includegraphics[width=0.39 \textwidth]{section10/img/LQ_exclusionplot_HLLHC_HELHC.png}
\caption{Current and projected 95\% CL limits on $\mu\mu jj$ final state at CMS (solid black) and HL-LHC (cyan) with 1 (10) ab$^{-1}$ in solid (dashed) lines. The pair production cross-section for scalar leptoquarks is shown in dotted lines for 14 (27) TeV in black (cyan). }
\label{fig:LQexclusion}
\end{figure}

For leptoquarks, the current 95\% CL limits in the $\mu\mu jj$ final state by CMS at 8 TeV with 19.6 fb$^{-1}$ is shown in solid black in Fig.~\ref{fig:LQexclusion}. The extrapolated limit is denoted by a dashed black line for HL-LHC and by a solid (dashed) cyan line for HE-LHC with 1 (10) ab$^{-1}$. We plot the cross-section times branching ratio in dotted lines at the corresponding collider energy for pair production of scalar leptoquarks, calculated at NLO using the code of Ref.~\cite{Kramer:2004df}. We see that the sensitivity to a leptoquark with only the minimal $b-\mu$ and $s-\mu$ couplings reaches around 2.5 and 4.5 TeV at the HL-LHC and HE-LHC, respectively. This pessimistic estimate is a lower bound that will typically be improved in realistic models with additional flavour couplings. Moreover, the reach can be extended by single production searches~\cite{Allanach:2017bta}, albeit in a more model-dependent way than pair production. The vector leptoquark case is also more model-dependent and is not shown here, but for $\mathcal{O}(1)$ couplings the corresponding limits are typically stronger than for scalar leptoquarks. 

\begin{figure}[t]
\centering
\includegraphics[width=0.39 \textwidth]{section10/img/14TeVMDM.pdf}
\caption{ HL-LHC 95\% (blue) and 99\% (orange) CL sensitivity contours to $Z^\prime$ in the ``mixed-down'' model for $g_{\mu\mu}$ vs $Z^\prime$ mass in TeV. The dashed grey contours are the width as a fraction of mass. The green and red regions are excluded by trident neutrino production and $B_s$ mixing. The dashed blue line is the stronger $B_s$ mixing constraint from Ref.~\cite{Bazavov:2016nty}. }
\label{fig:Zprime14TeV}
\end{figure}

For the $Z^\prime$ case the minimal couplings in the mass eigenstate basis must be obtained by unitary transformations from the gauge eigenstate that would necessarily induce other couplings. Ref.~\cite{Allanach:2018odd} defined the ``mixed-up'' model (MUM) and ``mixed-down'' model (MDM) such that the minimal couplings are obtained via CKM rotations in either the up or down sectors respectively. The predicted sensitivity at the HL-LHC for the MDM is shown in Fig.~\ref{fig:Zprime14TeV} for varying the $Z^\prime$ muon coupling $g_{\mu\mu}$ and the $Z^\prime$ mass. The solid blue (orange) contours give the 95\% and 99\% CL sensitivity. The red and green regions are excluded by $B_s$ mixing~\cite{Arnan:2016cpy} and neutrino trident production~\cite{Falkowski:2017pss,Falkowski:2018dsl}, respectively. The more stringent $B_s$ mixing constraint from Ref.~\cite{Bazavov:2016nty} is shown in dashed blue; see, however, Ref.~\cite{Kumar:2018kmr} for further discussion regarding the implications of this bound. The dashed grey contours denote the width as a fraction of the mass. We see that the HL-LHC will only be sensitive to $Z^\prime$'s with narrow width, up to masses of 5 TeV. The corresponding plot for the MUM is not shown as there is no sensitivity at the HL-LHC. 

\begin{figure}[t]
\centering
\includegraphics[width=0.39 \textwidth]{section10/img/27TeVMUM.pdf}
\includegraphics[width=0.39 \textwidth]{section10/img/27TeVMDM.pdf}
\caption{ HE-LHC 95\% (blue) and 99\% (orange) CL sensitivity contours to $Z^\prime$ in the ``mixed-up'' (left) and ``mixed-down'' (right) model in the parameter space of $g_{\mu\mu}$ vs $Z^\prime$ mass in TeV. The dashed grey contours are the width as a fraction of mass. The green and red regions are excluded by trident neutrino production and $B_s$ mixing. The dashed blue line is the stronger $B_s$ mixing constraint from Ref.~\cite{Bazavov:2016nty}.}
\label{fig:Zprime27TeV}
\end{figure}

At the HE-LHC, the reach for 10 ab$^{-1}$ is shown in Fig.~\ref{fig:Zprime27TeV} for the MUM and MDM on the left and right respectively. In this case the sensitivity may reach a $Z^\prime$ with wider widths up to 0.25 and 0.5 of its mass, while the mass extends out to 10 to 12 TeV. Again, this is a pessimistic estimate of the projected sensitivity to a particularly minimal model; more realistic scenarios will typically be easier to discover. 


\vspace{0.2cm}

\noindent  $\bullet$ \textit{Explanations at the one-loop level:}\\
Some models can accomodate the $b \to s \ell^+ \ell^-$ anomalies with mediators that enter only at one-loop.  One possibility is that of mediators coupling to right-handed top quarks and to muons~\cite{Celis:2017doq,Becirevic:2017jtw,Kamenik:2017tnu,Fox:2018ldq,Camargo-Molina:2018cwu}.   Given the loop and CKM  suppression of the new physics contribution to the $b \to s \ell^+ \ell^-$ amplitude, these models can explain the $b \to s \ell^+ \ell^-$ anomalies for a light mediator (around $\mathcal{O}(1)$~TeV or lighter).       Constraints from the LHC and future projections for the HL-LHC were derived in~\cite{Camargo-Molina:2018cwu} by recasting di-muon resonance, $pp\to t\bar tt \bar t$ and SUSY searches.  Two scenarios were considered: \textit{i)} a scalar LQ $R_2(3, 2, 7/6)$ combined with a vector LQ $\widetilde U_{1 \alpha}(3, 1, 5/3)$, \textit{ii)} a vector boson $Z^{\prime}$ in the singlet representation of the Standard Model gauge group.    Ref.~\cite{Fox:2018ldq} also analyzed the HL-LHC projections for the $Z^{\prime}$.     The constraints from the LHC already rule out part of the relevant parameter space and the HL-LHC will be able to cover much of the remaining regions.    Dedicated searches in the $pp\to t\bar tt \bar t$ channel and a dedicated search for $t \mu$ resonances in $t \bar t \mu^+ \mu^-$ final state can improve the sensitivity to these models~\cite{Camargo-Molina:2018cwu}. 

\subsubsection{Conclusions}

The anomalous results in $B$-meson decays cannot be considered yet as a convincing evidence of New Physics. On the other hand the number (and quality) of observables that are not in complete agreement with the SM prediction is growing with time in a coherent way. If true, the implications of the flavour anomalies will have a profound impact for the physics case of the future strategies for the HEP community. 
The conclusions we draw are the following:
\begin{itemize}
\item A conservative argument based on perturbative unitarity \cite{DiLuzio:2017chi} sets an upper bound on the New Physics scale to be 9 TeV for charged current anomalies and 80 TeV for neutral current ones. This does not allow to formulate a no-loose theorem for the high-luminosity program. However from the analysis of explicit models it is clear that the high-luminosity program has a clear potential to probe large portion of the possible BSM options. 

\item The explanation of the anomalies in $b \to c \tau \nu$ transitions requires non trivial model building. In particular, it is not possible to simply isolate the physics that mediate the flavour anomalous transitions. In complete models other signature have to be considered. Typically it would be very difficult to escape direct detection at HL/HE-LHC.

\item Even if the naive scale of the flavour anomalies $b \to s \ell \ell$  is much higher than the energy accessible at HL/HE-LHC, in motivated models, flavour suppressions and weak couplings guarantee a large coverage of the parameter space for leptoquark and $Z'$ mediators \cite{Allanach:2018odd}.
\end{itemize}