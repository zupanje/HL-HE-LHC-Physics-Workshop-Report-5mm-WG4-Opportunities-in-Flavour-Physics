\subsubsection{Experimental prospects for high $p_T$ searches at HL-LHC relevant for $B$ anomalies}
\label{sec:exprospects}

%\Blu{WE AGREED TO NOT HAVE SUBSUBSECTION THAT IS WHY I PROMOTED IT AS SUBSECTION. ALSO YOU SHOULD ADD THE EXP AUTHORS HERE: THEY DESERVE THE CREDIT FOR THIS (SUB)SECTION. HVALA, MARCO.}

We give next  the experimental prospects for leptoquark searches, with leptoquarks decaying to the final states relevant for $B$ physics anomalies. 

\subsubsubsection{Prospects for leptoquark searches assuming $t$+$\tau$ and/or $t$+$\mu$ decays}
\label{sec:LQ:t+tau}
The reach of searches for pair production of leptoquarks (LQs) with decays to $t+\mu$ and $t+\tau$ is studied for the HL-LHC with target integrated luminosities of 
$\mathcal{L}_{\text{int}}^{\text{target}} = 300$ and $3000\,\mathrm{fb}^{-1}$~\cite{CMS-PAS-FTR-18-008}. The studies are based on published CMS results of the $t$+$\mu$~\cite{Sirunyan:2018ruf} and $t$+$\tau$~\cite{Sirunyan:2018nkj} LQ decay channels which use data of proton-proton collisions at $\sqrt{s}=13\,\mathrm{TeV}$ corresponding to $\mathcal{L}_{\text{int}} = 35.9\,\mathrm{fb}^{-1}$ recorded in 2016. While the analysis strategies are kept unchanged with respect to the ones in Refs.~\cite{Sirunyan:2018nkj,Sirunyan:2018ruf}, different total integrated luminosities, the higher center-of-mass energy of 14\,TeV, and different scenarios of systematic uncertainties are considered. In the first scenario (denoted ''w/ YR18 syst. uncert.''), the relative experimental systematic uncertainties are scaled by a factor of $1 / \sqrt{f}$, with 
$f=\mathcal{L}_{\text{int}}^{\text{target}}/35.9\,\mathrm{fb}^{-1}$, 
until they reach a defined lower limit based on estimates of the achievable accuracy with the upgraded detector~\cite{Collaboration:2650976}. The relative theoretical systematic uncertainties are halved. In the second scenario (denoted ''w/ stat. uncert. only''), no systematic uncertainties are considered. The relative statistical uncertainties in both scenarios are scaled by $1 / \sqrt{f}$.

Figure \ref{fig:LQpairs:significances} presents the expected signal significances of the analyses as a function of the LQ mass for different assumed integrated luminosities in the ''w/ YR18 syst. uncert.'' and ''w/ stat. uncert. only'' scenarios. Increasing the target integrated luminosity to $\mathcal{L}_{\text{int}}^{\text{target}} = 3000\,\mathrm{fb}^{-1}$ greatly increases the discovery potential of both analyses. The LQ mass corresponding to a discovery at $5\sigma$ significance with a dataset corresponding to $3000\,\mathrm{fb}^{-1}$ increases by more than 500\,GeV compared to the situation at $\mathcal{L}_{\text{int}}^{\text{target}} = 35.9\,\mathrm{fb}^{-1}$, from about 1200\,GeV to roughly 1700\,GeV, in the $\text{LQ}\rightarrow t\mu$ decay channel. For LQs decaying exclusively to top quarks and $\tau$ leptons, a gain of 400\,GeV is expected, pushing the LQ mass in reach for a $5\sigma$ discovery from 800\,GeV to 1200\,GeV. 

\begin{figure}[t]
\centering
\includegraphics[width=0.49\textwidth]{section10/img/LQpairs_Sign_tmu.pdf}
\includegraphics[width=0.49\textwidth]{section10/img/LQpairs_Sign_ttau.pdf}
\caption{Expected significances for an LQ decaying exclusively to top quarks and muons (left) or top quarks and $\tau$ leptons (right).}
\label{fig:LQpairs:significances}
\end{figure}

In Fig.~\ref{fig:LQpairs:limits1d}, the expected projected exclusion limits on the LQ pair production cross section are shown. Leptoquarks decaying only to top quarks and muons are expected to be excluded below masses of 1900\,GeV for $3000\,\mathrm{fb}^{-1}$, which is a gain of 500\,GeV compared to the limit of 1420\,GeV obtained in the published analysis of the 2016 dataset~\cite{Sirunyan:2018ruf}. The mass exclusion limit for LQs decaying exclusively to top quarks and $\tau$ leptons are expected to be increased by 500\,GeV, from 900\,GeV to approximately 1400\,GeV.

\begin{figure}[t]
\centering
\includegraphics[width=0.49\textwidth]{section10/img/LQpairs_Limits_tmu.pdf}
\includegraphics[width=0.49\textwidth]{section10/img/LQpairs_Limits_ttau.pdf}
\caption{Expected upper limits on the LQ pair production cross section at the 95\% CL for an LQ decaying exclusively to top quarks and muons (left) or $\tau$ leptons (right).}
\label{fig:LQpairs:limits1d}
\end{figure}

Figure~\ref{fig:LQpairs:limits2d} shows the expected signal significances and upper exclusion limits on the pair production cross section of scalar LQs allowed to decay to top quarks and muons or $\tau$ leptons at the 95\% CL as a function of the LQ mass and a variable branching fraction $\mathcal{B}(\mathrm{LQ}\rightarrow t\mu) = 1 - \mathcal{B}(\mathrm{LQ}\rightarrow t\tau)$ for an integrated luminosity of $3000\,\mathrm{fb}^{-1}$ in the two different scenarios. For all values of $\mathcal{B}$, LQ masses up to approximately 1200\,GeV and 1400\,GeV are expected to be in reach for a discovery at the 5$\sigma$ level and a 95\% CL exclusion, respectively.

\begin{figure}[t]
\centering
\includegraphics[width = 0.49\textwidth]{section10/img/LQpairs_Sign_2d.pdf}
\includegraphics[width = 0.49\textwidth]{section10/img/LQpairs_Limits_2d.pdf}
\caption{Expected significances (left) and expected upper limits on the LQ pair-production cross section at the 95\% CL (right) as a function of the LQ mass and the branching fraction. Color-coded lines represent lines of a constant expected significance or cross section limit, respectively. The red lines indicate the 5$\sigma$ discovery level (left) and the mass exclusion limit (right).}
\label{fig:LQpairs:limits2d}
\end{figure}

\subsubsubsection{Search for leptoquarks decaying to $\tau$ and b}
%\contributors{Y. Takahashi, P. Matorras, CMS}
%{\bf Author(s): Y. Takahashi et al., CMS}

%Third-generation scalar LQs have recently received considerable interest from the theory community, as the existence of leptoquarks with large couplings can explain the anomaly in the $\overline{\PB} \to \PD \tau \overline{\nu}$ and $\overline{\PB} \to \PD^{\ast} \tau \overline{\nu}$ decay rates reported by the
%BaBar~\cite{Lees:2012xj, Lees:2013uzd}, Belle~\cite{Matyja:2007kt, Bozek:2010xy, Huschle:2015rga, Sato:2016svk, Hirose:2016wfn, Hirose:2017dxl},
%and LHCb~\cite{Aaij:2015yra} Collaborations. 

The analysis from CMS~\cite{CMS-PAS-FTR-18-028} presents future discovery and exclusion prospects for singly and pair produced 
third-generation scalar LQs, each decaying to $\tau_h$ and a bottom quark. Here, $\tau_h$ denotes a hadronically decaying 
$\tau$ lepton. 
The relevant Feynman diagrams of the signal processes at leading order (LO) are shown in Fig. \ref{LQ3-diagram}.

\begin{figure}[t]
\centering
\includegraphics[width=0.42\textwidth]{\main/section10/img/LQsingleProduction.png}
\hspace{0.2cm}
\includegraphics[width=0.38\textwidth]{\main/section10/img/gluonLQLQ.png}
\caption{Leading order Feynman diagrams for the production of a third-generation LQ in the single production $s$-channel (left) and the pair production channel via gluon fusion (right).}
\label{LQ3-diagram}
\end{figure}
\vspace{-5pt}

The analysis uses {\tt Delphes}~\cite{deFavereau:2013fsa} event samples of simulated $pp$ collisions at a center-of-mass energy of 14 TeV, corresponding to integrated luminosities of 300 and 3000\fbinv. %Additional pp interactions (pileup or PU) within the same or adjacent bunch crossings are included, with an average of 200 interactions per event. 
The matrix elements of LQ signals for both single and pair LQ production are generated at LO using version 2.6.0 of
{\tt MadGraph5\_aMC@NLO}~\cite{aMCatNLO} for $m_{LQ}=$500, 1000, 1500, and 2000 GeV. The branching fraction $\beta$ of the LQ to a charged
lepton and a quark, in this case LQ$\to \tau b$, is assumed to be $\beta=1$. The unknown Yukawa coupling $\lambda$ of the LQ to a
$\tau$ lepton and a bottom quark is set to $\lambda=1$. The width $\Gamma$ is calculated using $\Gamma = m_{LQ} \lambda^2 /
(16\pi)$~\cite{Plehn:1997az}, and is less than 10\% of the LQ mass for most of the considered search range. The signal samples are
normalized to the cross section calculated at LO, multiplied by a $K$ factor to account for higher order contributions~\cite{Dorsner:2018ynv}.

Similar event selections are used in both the singly and pair produced LQ searches, except for the requirement on the number of jets. 
In both channels, two reconstructed $\tau_h$ with opposite-sign charge are required, each with transverse momentum $p_{T,\tau}>50$~GeV and a maximum pseudorapidity  $|\eta_{\tau}|<2.3$. 
In the search for single production, the presence of at least one reconstructed jet with $p_T > 50$~GeV is required, while at least
two are required in the search for pair production. Jets are reconstructed with {\tt Fastjet}~\cite{Cacciari:2011ma}, using the anti-$k_T$ algorithm~\cite{Cacciari:2008gp}, with a distance parameter of 0.4. 

To reduce background due to Drell-Yan events (particularly Z$\to\tau\tau$), the invariant mass of the two selected $\tau_h$,
$m_{\tau\tau}$, is required to be $>95$~GeV.
In addition, at least one of the previously selected jets is required to be $b$-tagged to reduce QCD multijet backgrounds. 
Finally, an event is rejected, if it contains identified and isolated electrons (muons), with $p_T>10$~GeV, $|\eta|<2.4 (2.5)$.
The acceptance of the signal events is $4.9\%$  ($11\%$) for single (pair) production, where the branching ratio of two $\tau$ leptons decaying hadronically is included in the numerator of the acceptance.
Signal extraction is based on a binned maximum likelihood fit to the distribution of the scalar $p_T$ sum $S_{\text{T}}$, which 
is defined as the sum of the transverse momenta of the two $\tau_h$ and either the highest-$p_T$ jet in the 
case of single LQ production, or the two highest-$p_T$ jets in the case of LQ pair production. These distributions are shown 
in Fig. \ref{fig:scalarptsums} for the HL-LHC 3000 $\fbinv$ scenario.

\begin{figure}[t]
\centering
\includegraphics[width=0.48\textwidth]{\main/section10/img/scalarpt1sum.png}
\includegraphics[width=0.48\textwidth]{\main/section10/img/scalarpt2sum.png}
\caption{Left: scalar sum of the \pt\ of the two selected $\tau$ leptons and the highest-$p_T$ jet in the single LQ selection region. Right: scalar sum of the $p_T$ of the two selected $\tau$ leptons and the two highest-$p_T$ jets in the LQ pair search region. The considered backgrounds are shown as stacked histograms, while empty histograms for signals for the single LQ and LQ pair channels (for $m_{LQ}=1000$~GeV) are overlaid to illustrate the sensitivity. Both signal and backround are normalised to a luminosity of 3000 $\fbinv$.} %Signal is normalized to $\lambda=1$, $\beta=1$ and $\sigma=1$ pb.}
\label{fig:scalarptsums}
\end{figure}

Systematic uncertainties are 
calculated by scaling the current experimental uncertainties. For uncertainties limited by statistics, including the uncertainty 
on the DY ($3.3\%$) and QCD ($3.3\%$) cross sections, a scale factor of $1/\sqrt{L}$ is applied, for an integrated luminosity ratio $L$. 
For uncertainties 
coming from theoretical calculations, a scale factor of 1/2 is applied with respect to current uncertainties, as is the case for 
the uncertainties on the cross section for top ($2.8\%$) or diboson ($3\%$) events. Other experimental systematic uncertainties are 
scaled by the 
square root of the integrated luminosity ratio until the uncertainty reaches a minimum value, including uncertainties on the integrated 
luminosity ($1\%$), 
$\tau$ identification ($5\%$) and $b$-tagging/misidentification ($1\%$/$5\%$).

Fig.~\ref{fig:LQ3-projection} shows an upper limit at 95\% CL on the cross section, $\sigma$, times branching fraction, $\beta$, as a 
function of $m_{LQ}$, by using the asymptotic CLs modified frequentist 
criterion~\cite{Cowan:2010js, Junk:1999kv,Read:2002hq, CMS-NOTE-2011-005}.
Upper limits are calculated considering 
two different scenarios. The first one, hereafter abbreviated as ``stat. only'' considers only statistical uncertainties, to 
observe how the results are affected by the increase of the integrated luminosity. The second scenario, hereafter 
abbreviated as ``stat.+syst.'', also includes the estimate of the systematic uncertainties at the HL-LHC. 
For the single LQ production search, the theoretical prediction for the cross section assumes $\lambda = 1$ and $\beta = 1$. 

\begin{figure}[t]
\vspace{-5pt}
\centering
\includegraphics[width=0.45\textwidth]{\main/section10/img/projection_single.pdf}
\includegraphics[width=0.45\textwidth]{\main/section10/img/projection_pair.pdf}
\caption{Expected limits at 95\% CL on the product of the cross section, $\sigma$, and the branching fraction, $\beta$, as a function
of the LQ mass, for the two high luminosity projections, 300 \fbinv (red) and 3000 \fbinv (orange), for both the stat. only  (dashed
lines) and the stat.+syst. scenarios (solid lines). This is shown in conjunction with the theoretical predictions at NLO~\cite{Dorsner:2018ynv} in cyan. Projections are calculated for both the single LQ (left) and LQ pair production (right).}
\label{fig:LQ3-projection}
\end{figure}

Comparing the limits with theoretical predictions assuming unit Yukawa coupling, $\lambda = 1$, third-generation scalar 
leptoquarks are expected to be excluded at $95\%$ confidence level for LQ masses below 1.35 (1.71) TeV for a luminosity 
of 300 \fbinv, and below 1.60 (1.93) TeV for 3000 \fbinv in the single (pair) production channel, considering both 
statistical and systematic uncertainties. 
 Since the single-LQ signal cross section scales with $\lambda^2$, it is straightforward to recast the results presented in 
Fig. \ref{fig:LQ3-projection} in terms of expected upper limits on $m_{LQ}$  as a function of $\lambda$, which is shown in 
Fig. \ref{fig:lambda_mass}. 
The blue band shows the parameter space (95\% CL) for the scalar LQ preferred by the $\PB$ physics 
anomalies: $\lambda = (0.95 \pm 0.50) m_{LQ} ({\rm TeV})$~\cite{Buttazzo:2017ixm} \td{JZ: is this for $S_1$ or $S_3$ or both?}. 
For the 300 (3000) \fbinv luminosity scenario, the leptoquark pair production channel is more sensitive if  
$\lambda<2.5$ $(1.5)$, while the single leptoquark production is dominant otherwise. Using the predicted cross section~\cite{Dorsner:2018ynv} of the signal, it is also possible to estimate the maximal LQ mass expected to be in reach for a 5$\sigma$ discovery. 
Fig.~\ref{fig:LQ3-discovery} shows the expected local significance of a signal-like excess as a function of the LQ mass hypothesis.
 
\begin{figure}[tb]
\centering
\includegraphics[width=0.45\textwidth]{\main/section10/img/exclusion.pdf}
\caption{Expected 95\% CL exclusion limits on the Yukawa coupling $\lambda$, entering the LQ-lepton-quark vertex, as a function of the
LQ mass. A unit branching fraction $\beta$ of the LQ to a $\tau$ lepton and a bottom quark is assumed. Future projections for 300
and 3000 \fbinv are shown for both the stat. only and stat.+syst. scenarios, shown as dashed and
filled lines respectively, and for both the single LQ and LQ pair production, where the latter corresponds to the vertical line
(since it does not depend on $\lambda$). The left hand side of the lines  represents the exclusion region for each of the
projections, whereas the region with diagonal blue hatching shows the parameter space preferred by one of the models proposed to
explain anomalies observed in $\PB$ physics~\cite{Buttazzo:2017ixm}.}
\label{fig:lambda_mass}
\end{figure}



\begin{figure}[tb]
\vspace{-5pt}
\centering
\includegraphics[width=0.45\textwidth]{\main/section10/img/significance_LQ.pdf}~~~~
\includegraphics[width=0.45\textwidth]{\main/section10/img/significance_LQLQ.pdf}
\caption{Expected local significance of a signal-like excess as a function of the LQ mass, for the two high luminosity projections,
300 \fbinv (red) and 3000 \fbinv (orange), assuming the theoretical prediction for the LQ cross section at NLO~\cite{Dorsner:2018ynv}, calculated with $\lambda = 1$ and $\beta = 1$. Projections are calculated for both single LQ (left) and LQ pair production (right).}
\label{fig:LQ3-discovery}
\end{figure}

In summary, this study shows that the future LQ searches under higher luminosity conditions are promising, as they are expected 
to greatly increase the experimental reach. It also shows that the pair production channel is  expected to be the most 
sensitive. A significance of 5$\sigma$ is within reach for LQ masses below 1.25 (1.35) TeV for the single (pair) production 
channels in the 300\fbinv scenario and 1.50 (1.50) TeV for the 3000 \fbinv scenario.



