\subsubsection{Experimental prospects}
We next give the experimenta prospects for leptoquark and $W'$ searches decaying to final states relevant for the $B$ physics anomalies. 

\subsubsubsection{Prospects for leptoquark searches in $t$+$\tau$ and $t$+$\mu$ decays} 
The reach of searches for pair production of leptoquarks (LQs) with decays to $t+\mu$ and $t+\tau$ is studied for the HL-LHC with target integrated luminosities of 
$\mathcal{L}_{\text{int}}^{\text{target}} = 300$ and $3000\,\mathrm{fb}^{-1}$~\cite{CMS-PAS-FTR-18-008}. The studies are based on published CMS results of the $t$+$\mu$~\cite{Sirunyan:2018ruf} and $t$+$\tau$~\cite{Sirunyan:2018nkj} LQ decay channels which use data of proton-proton collisions at $\sqrt{s}=13\,\mathrm{TeV}$ corresponding to $\mathcal{L}_{\text{int}} = 35.9\,\mathrm{fb}^{-1}$ recorded in 2016. While the analysis strategies are kept unchanged with respect to the ones in Refs.~\cite{Sirunyan:2018nkj, Sirunyan:2018ruf}, different total integrated luminosities, the higher center-of-mass energy of 14\,TeV, and different scenarios of systematic uncertainties are considered. In the first scenario (denoted ``w/ YR18 syst. uncert.''), the relative experimental systematic uncertainties are scaled by a factor of $1 / \sqrt{f}$, with 
$f=\mathcal{L}_{\text{int}}^{\text{target}}/35.9\,\mathrm{fb}^{-1}$, 
until they reach a defined lower limit based on estimates of the achievable accuracy with the upgraded detector~\cite{CMS:FTR-18-012}. The relative theoretical systematic uncertainties are halved. In the second scenario (denoted ``w/ stat. uncert. only''), no systematic uncertainties are considered. The relative statistical uncertainties in both scenarios are scaled by $1 / \sqrt{f}$.

Figure \ref{fig:LQpairs:significances} presents the expected signal significances of the analyses as a function of the LQ mass for different assumed integrated luminosities in the ``w/ YR18 syst. uncert.'' and ``w/ stat. uncert. only'' scenarios. Increasing the target integrated luminosity to $\mathcal{L}_{\text{int}}^{\text{target}} = 3000\,\mathrm{fb}^{-1}$ greatly increases the discovery potential of both analyses. The LQ mass corresponding to a discovery at $5\sigma$ significance with a dataset corresponding to $3000\,\mathrm{fb}^{-1}$ increases by more than 500\,GeV compared to the situation at $\mathcal{L}_{\text{int}}^{\text{target}} = 35.9\,\mathrm{fb}^{-1}$, from about 1200\,GeV to roughly 1700\,GeV, in the $\text{LQ}\rightarrow t\mu$ decay channel. For LQs decaying exclusively to top quarks and $\tau$ leptons, a gain of 400\,GeV is expected, pushing the LQ mass in reach for a $5\sigma$ discovery from 800\,GeV to 1200\,GeV. 

\begin{figure}[t]
\centering
\includegraphics[width=0.49\textwidth]{section10/img/LQpairs_Sign_tmu.pdf}
\includegraphics[width=0.49\textwidth]{section10/img/LQpairs_Sign_ttau.pdf}
\caption{Expected significances for a LQ decaying exclusively to top quarks and muons (left) or $\tau$ leptons (right).}
\label{fig:LQpairs:significances}
\end{figure}

In Fig.~\ref{fig:LQpairs:limits1d}, the expected projected exclusion limits on the LQ pair production cross section are shown. Leptoquarks decaying only to top quarks and muons are expected to be excluded below masses of 1900\,GeV for $3000\,\mathrm{fb}^{-1}$, which is a gain of 500\,GeV compared to the limit of 1420\,GeV obtained in the published analysis of the 2016 dataset~\cite{Sirunyan:2018ruf}. The mass exclusion limit for LQs decaying exclusively to top quarks and $\tau$ leptons are expected to be increased by 500\,GeV, from 900\,GeV to approximately 1400\,GeV.

\begin{figure}[t]
\centering
\includegraphics[width=0.49\textwidth]{section10/img/LQpairs_Limits_tmu.pdf}
\includegraphics[width=0.49\textwidth]{section10/img/LQpairs_Limits_ttau.pdf}
\caption{Expected upper limits on the LQ pair production cross section at the 95\% CL for a LQ decaying exclusively to top quarks and muons (left) or $\tau$ leptons (right).}
\label{fig:LQpairs:limits1d}
\end{figure}

Figure~\ref{fig:LQpairs:limits2d} shows the expected signal significances and upper exclusion limits on the pair production cross section of scalar LQs allowed to decay to top quarks and muons or $\tau$ leptons at the 95\% CL as a function of the LQ mass and a variable branching fraction $\mathcal{B}(\mathrm{LQ}\rightarrow t\mu) = 1 - \mathcal{B}(\mathrm{LQ}\rightarrow t\tau)$ for an integrated luminosity of $3000\,\mathrm{fb}^{-1}$ in the two different scenarios. For all values of $\mathcal{B}$, LQ masses up to approximately 1200\,GeV and 1400\,GeV are expected to be in reach for a discovery at the 5$\sigma$ level and a 95\% CL exclusion, respectively.

\begin{figure}[t]
\centering
\includegraphics[width = 0.49\textwidth]{section10/img/LQpairs_Sign_2d.pdf}
\includegraphics[width = 0.49\textwidth]{section10/img/LQpairs_Limits_2d.pdf}
\caption{Expected significances (left) and expected upper limits on the LQ pair-production cross section at the 95\% CL (right) as a function of the LQ mass and the branching fraction. Color-coded lines represent lines of a constant expected significance or cross section limit, respectively. The red lines indicate the 5$\sigma$ discovery level (left) and the mass exclusion limit (right).}
\label{fig:LQpairs:limits2d}
\end{figure}

\subsubsubsection{Searches for $W^\prime \to \tau + MET$}

New heavy gauge bosons are predicted by various SM extensions. The charged version of such heavy gauge bosons is generally referred to as $W'$. Recently decays to third generation fermions have gained interest as possible explanations for some of the observed flavour anomalies. 
The decay $W'  \rightarrow \tau \nu$ yields a single hadronically decaying tau
($\tau_h$) as the only detectable object and missing energy due to the neutrinos.
Hadronically decaying tau leptons are selected since the corresponding branching fraction,
about 60\%, is the largest among all $\tau$ decays.
Tau-jets are experimentally distinctive because of their low charged hadron
multiplicity, unlike QCD multi-jets,
which have high charged hadron multiplicity, or other leptonic \PWpr boson decays, which yield no jet. 
This Phase-2 study~\cite{CMS-PAS-FTR-18-030} follows closely the recently published Run~2 result~\cite{Run2Tau},

The signature of a $W^\prime$ is considered similar to a high-mass $W$ boson (see Fig.~\ref{fig:feynman2}). 
It could be observed in the distribution of the
transverse mass ($M_T$) of the transverse momentum of the $\tau$ ($p_T^{\tau}$)
and the missing transverse momentum):
$M_T = \sqrt{2 p_T^{\tau}p_T^{\rm miss}(1-\cos\Delta\phi(\tau,p_T^{\rm miss}))}$.

The results are interpreted in the context of the sequential standard model (SSM)
~\cite{Altarelli:1989ff} in terms of  $W^\prime$ mass and coupling strength. 
A model-independent cross section limit allows interpretations in other models.
The signal was simulated at LO and the detector performance simulated with {\tt DELPHES}.
The $W^\prime$ boson coupling strength, $g_{W^\prime}$, is given in terms of the SM weak coupling strength $g_{\PW} = e/\sin^2\theta_W\approx 0.65$.
Here, $\theta_{W}$ is the weak mixing angle.
If the $W^\prime$ boson is a heavier copy of the SM \PW\ boson, their coupling ratio is $g_{W^\prime}/g_{W}=1$ and the SSM W boson theoretical cross sections, signal shapes, and widths apply.
However, different couplings are possible. Because of the dependence of the width of a particle on its couplings
the consequent effect on the transverse mass distribution, a limit can also be set on the coupling strength.


\begin{figure}[t]
\begin{center}
\includegraphics[width=.49\textwidth]{section10/img/MT_final_3000.pdf}
\caption{The discriminating variable, $M_T$, after all selections for HL-LHC conditions of
3000~fb$^{-1}$ and 200~PU.
The relevant SM backgrounds are shown according to the labels in
the legend. Signal examples for $W^\prime$ boson masses of 1\tev, 5\tev and 6\tev are scaled to their SSM LO cross section and 3000~fb$^{-1}$.}
\label{fig:feynman2}
\end{center}
\end{figure}

\begin{figure}[t]\centering
\includegraphics[width=0.49\textwidth]{section10/img/Significance.pdf}
\includegraphics[width=0.49\textwidth]{section10/img/MILimit_3000fbinv.pdf}
\caption{\textit{Left:} Discovery significance for SSM  $W^\prime$ to tau leptons.
\textit{Right:} Model-independent cross section limit. For this, a single-bin limit is
calculated for increasing $M_T^{\rm min}$
while keeping the signal yield constant in order to avoid including any signal shape
information on this limit calculation.}
    \label{fig:WprimeResults}
\end{figure}

The physics sensitivity is studied based on the $M_T$ distribution in Fig.~\ref{fig:feynman2}.
Signal events are expected to be particularly prominent at the upper end of the $M_T$
distribution, where the expected SM background is low.
So far, there are no indications for the existence of a SSM $W^\prime$ boson~\cite{Run2Tau}. With the
high luminosity during Phase-2, the $W^\prime$ mass reach for potential observation increases
to 6.8\tev\ and 6.4\tev\
for 3~$\sigma$ evidence and 5~$\sigma$ discovery, respectively, as shown in Fig.~\ref{fig:WprimeResults}-left. Alternatively, in case of no observation, one can exclude SSM $W'$ boson masses up to 7.4\tev
with $3000$~fb$^{-1}$.
These are multi-bin limits taking into account the full $M_T$ shape.


While the SSM model assumes SM-like couplings of the fermions, the couplings could well be weaker if
further decays occur. The HL-LHC has a good sensitivity to study these couplings.
The sensitivity to weaker couplings extends significantly.

A model-independent cross section limit for new physics with $\tau$+MET
in the final state is depicted in Fig.~\ref{fig:WprimeResults}-right, calculated as a single-bin limit
by counting the number of events above a sliding threshold $M_T^{\rm min}$.



