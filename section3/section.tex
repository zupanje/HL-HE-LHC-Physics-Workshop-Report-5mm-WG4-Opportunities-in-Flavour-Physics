% don't remove the folling lines, and edit the defintion of \main if needed
\documentclass[../report.tex]{subfiles}
\providecommand{\main}{..}
\IfEq{\jobname}{\currfilebase}{\AtEndDocument{\biblio}}{}
% until here
\newcommand{\inlineComplexmix}{\ensuremath{\nicefrac{q}{p}}}
\newcommand{\inlineInvcomplexmix}{\ensuremath{\nicefrac{p}{q}}}
\newcommand{\inlineRmix}{\ensuremath{\left | \inlineComplexmix \right |}}
\newcommand{\gp}{\ensuremath{g_{+}}}
\newcommand{\gm}{\ensuremath{g_{-}}}
\newcommand{\phimix}{\ensuremath{\phi_{\mathit{mix}}}}
\def\brdmumu{\ensuremath{\mathcal{B}(D^0 \to \mu^{+} \mu^{-})}\xspace}
\def\dmumu{\ensuremath{D^0 \to \mu^{+} \mu^{-}}\xspace}
\def\dgammagamma{\ensuremath{D^0 \to \gamma \gamma}\xspace}
\def\bea{\begin{eqnarray}}
\def\eea{\end{eqnarray}}
\def\ba{\begin{eqnarray}}
\def\ea{\end{eqnarray}}
\def\be{\begin{equation}}
\def\ee{\end{equation}}
\def\beq{\begin{equation}}
\def\eeq{\end{equation}}
\def\beq{\begin{equation}}
\def\eeq{\end{equation}}
\def\bea{\begin{eqnarray}}
\def\eea{\end{eqnarray}}
%\def\barD{\overline D{}^0}
%\def\D0bar{\overline D{}^0}
\def\dzbar{\bar D{}^0}
\def\dbar{\bar D{}^0}
\def\DDbar{D{}^0-\overline D{}^0}
\def\lsim{\lesssim}
\def\gsim{\gtrsim}

\begin{document}

\section{Charm quark probes of new physics}
{\bf Author(s): J. Brod, A. Lenz, S. Fajfer, A. Kagan, L. Silvestrini} (15 pages, TH:EXP= 50\%:50\%)

%In the SM, the dynamics of FCNC processes is determined by the interplay between loop functions
%and CKM matrix elements. 
In the SM, the FCNC processes involving charmed hadrons are
suppressed compared to those involving strange or beauty hadrons since they are proportional to 
the soft breaking of the GIM mechanism by the bottom quark mass. This also means that 
 the contributions from long-distance physics, due to intermediate $d$, $s$ quarks, are relatively more important, complicating the predictions. 
%same reason, the prediction of these processes is generally
%complicated by a large contribution of long-distance
% * <kaganalexander@gmail.com> 2018-10-17T18:12:15.422Z:
%
% ^.
% * <kaganalexander@gmail.com> 2018-10-17T18:12:12.135Z:
%
% ^.
%physics. 
Moreover, due to small off-diagonal CKM matrix elements the third generation approximately factorizes from the first two generations, 
leading to additional suppression of the CP violating effects in charmed hadrons.
% is additionally
%suppressed by an approximate factorization of ,
%. 
The charmed
hadrons can then be used as sensitive probes of new physics in the up-quark
sector, to the extent that theoretical uncertainties can be brought
under control, or circumvented by using experimental data.

\subsection{Charm mixing}
\label{sec:Dmixingth}
Weak interactions mix $D^0$ and $\bar D^0$ mesons, so that the mass 
%Diagonalisation of the 2x2 Hamiltonian matrix, $H$, describing the mixing of the neutral $D$ mesons
%gives the same eigenvalue equations as in the neutral $B$ systems.  
%The mass 
eigenstates are
%denoted as 
$|D_{1,2}\rangle = p |D^0 \rangle \pm q |\dzbar\rangle$. By convention $|D_{2}\rangle$
is CP-even in the absence of CP violation.  The mass and width differences, 
$\Delta M = m_2 - m_1$, and $\Delta \Gamma = \Gamma_2 - \Gamma_1$,  are parametrized as 
\begin{eqnarray}\label{xyexp}
x \equiv \Delta M^D / \Gamma^D = 0.46 \% \pm 0.13 \% \, ,
&&
y \equiv \Delta \Gamma^D / (2 \Gamma^D) = 0.62 \% \pm 0.07 \%.
\end{eqnarray}
Here $\Gamma$ is the total decay rate of the neutral $D$ mesons, while the numerical values are from the fits to the experimental measurements \cite{Amhis:2016xyh}.
% are also included. 
% While a decay rate difference in the neutral $D$ system is by now firmly established, the possibility of 
% having a vanishing mass difference is not yet experimentally excluded (the future sensitivity for $x$
% will be at the order of $0.001 \%$).  
% However, as discussed below, $x\sim y$ is naturally expected
% in the SM \cite{Falk:2001hx,Falk:2004wg,Cheng:2010rv}.
This is to be contrasted with the $B$ system, where $|\Gamma_{12}/M_{12}| \ll 1$ holds.
We also define the ``theoretical'' mixing parameters, 
\beq 
x_{12}\, \equiv\, \frac{2|M^D_{12}|}{\Gamma^D}\,
,~~ y_{12}\, \equiv\,\frac{|\Gamma^D_{12}|}{\Gamma^D}\,,~~\phi_{12} \,\equiv\, {\rm arg}\left(\frac{M^D_{12}}{\Gamma^D_{12} }\right)\,,
\label{x12yi12phi12}
\eeq
where $M_{12}^D $ and $\Gamma_{12}^D$ are the dispersive and absorptive contributions to the $\DDbar$ mixing amplitude, $\langle  D^0 | H | \dzbar \rangle \ =\  M^D_{12} - \frac{i}{2} \Gamma^D_{12}$. 
The phase $\phi_{12}$ gives rise to CP violation in mixing, cf. Sec.~\ref{sec:CPVDmixth}. Its magnitude is currently bounded to lie below $\sim 100$ mrad at 95\% CL~\cite{LHCbimplications,Amhis:2016xyh}.
These parameters are related to $x$ and $y$ as,
\beq (x- i y)^2 = x_{12}^2 - y_{12}^2  - 2 \,i\, x_{12} y_{12} \cos\phi_{12}\,,\label{xyx12y12}\eeq
so that, up to negligible corrections quadratic in $\sin\phi_{12}$ (in general, $|y| \le y_{12}$ \cite{Nierste:2009wg,Jubb:2016mvq}),
\beq \begin{split} |x|=x_{12} ,~~|y| = y_{12}\,.\label{x12vsx}
\end{split}
\eeq 


It is convenient to begin the discussion of the SM mixing amplitudes 
%(and indirect CPV, cf. Sec.~\ref{sec:CPVDmixth}) 
with their $U$-spin flavor symmetry decomposition \cite{Falk:2001hx,Gronau:2012kq}.
Employing CKM unitarity, $\lambda_d + \lambda_s +\lambda_b = 0$, with  $\lambda_x = V_{cx} V^*_{ux}$, 
$\Gamma^D_{12}$ can be written as 
\beq
 \Gamma^D_{12 } = \frac{(\lambda_s - \lambda_d )^2}{4 }\, \Gamma_2 + \frac{(\lambda_s - \lambda_d ){\lambda_b} }{ 2 }\, \Gamma_1 + 
\frac{{ \lambda_b^2} }{4} \, \Gamma_0 \,,  \label{Uspindecomp}\eeq
where
\beq \begin{split} \label{Uspinamps}
\Gamma_2  &= \Gamma_{ss} + \Gamma_{dd} - 2 \Gamma_{sd} ~\sim~ (\bar s s - \bar d d )^2 ={\mathcal O}(\epsilon^2)  \,,\cr
\Gamma_1 & = \Gamma_{ss} - \Gamma_{dd}~ \sim~ (\bar s s - \bar d d ) (\bar s s + \bar d d) ={\mathcal O}(\epsilon)\,,\cr
\Gamma_0 & = \Gamma_{ss} +\Gamma_{dd}+2 \Gamma_{sd} ~ \sim~ (\bar s s+ \bar d d )^2 ={\mathcal O}(1)  \,,
\end{split}
\eeq
with $\epsilon$ the $U$-spin breaking parameter. 
%and similarly for the dispersive parts with $\Gamma\to M$ replacements everywhere.
 The $\Gamma_{2,1,0}$ are the $\Delta U_3 =0$ elements of the $U$-spin 5-plet, triplet, and singlet, respectively.
%$\Delta U$= 2, 1, 0 (5-plet, triplet, and singlet)  multiplets, respectively.  
%This can be seen from their quark flavor structure, given by
%where t
The individual $\Gamma_{ij}$ are identified, at the quark level, with box diagrams containing on-shell internal $i$ and $j$ quarks. They possess flavor structure $\Gamma_{ij}\sim  (\bar i i) (\bar j j) (\bar u c)^2$, where
the external $u $ and $c$ quarks are irrelevant for the $U$-spin decomposition.
%The orders in the $U$-spin breaking parameter $\epsilon$ at which they enter are shown on the right (corresponding to the power of the 
%$U$-spin breaking spurion $ \epsilon (\bar s s - \bar d d)$ required to construct each $\Gamma_i$).  
The $U$-spin decomposition of $M^D_{12}$ is analogous to \eqref{Uspindecomp}, with $\Gamma\to M$ replacement everywhere. 
%with the $M_i$ given by 
%\beq \begin{split} \label{Mspinamps}
%M_2  &= M_{ss} +M_{dd} - 2 M_{sd} ~\sim~ (\bar s s - \bar d d )^2 
%=O(\epsilon^2)  \,,\cr
%M_1 & =M_{ss} - M_{dd} + M_{sb} - M_{db} ~ \sim~ (\bar s s - \bar d d ) (\bar s s + \bar d d) =O(\epsilon)\,,\cr
%M_0 & =M_{ss} +M_{dd}+2 M_{sd} + M_{sb} + M_{db} + M_{bb} ~ \sim~ (\bar s s+ \bar d d )^2 =O(1)  \,.
%\end{split}
%\eeq
%The $M_{ij} $ are ``off-shell" analogs of the $\Gamma_{ij}$.
At the level of quark box diagrams, $M_{1,0}$ also receive contributions containing internal $b$ quarks.
The small value of $\lambda_b$ implies that we can neglect the $\Delta U= 1, 0$ contributions to the mass and width differences, even though the $\Delta U=2$ piece is higher order in $\epsilon$.



Evaluation of the SM mixing amplitudes is very challenging,
because the charm quark mass lies at an intermediate scale between the masses of the light quarks and the bottom quark.
%the heavy quark expansion (HQE) applies, and and chiral perturbation theory.  
%, too light to be considered a heavy quark not much larger than $\Lambda_{QCD}$.  
Broadly, there are two approaches: (i) an inclusive one employing the operator product expansion (OPE), which expands in powers of $\Lambda_{\rm QCD}/m_c$, as in the heavy quark expansion (HQE), and assumes that local quark-hadron duality holds \cite{Georgi:1992as,Ohl:1992sr,Bigi:2000wn,Bobrowski:2010xg}; and (ii) 
an exclusive one in which $y$ is estimated by summing over contributions of exclusive states, and $x$ is estimated via a dispersion relation which relates
it to $y$.
In the first approach, 
the HQE applied to $\Gamma_{12}^D$, combined with the relevant non-perturbative dimension six operator matrix elements evaluated in 
\cite{Carrasco:2014uya,Carrasco:2015pra,Bazavov:2017weg,Kirk:2017juj}, yields
contributions of the individual $\Gamma_{ij}$ %, e.g. $\Gamma_{ss}$, 
to $y$ that are five times larger than the experimental value \cite{Lenz:2016fcv}.
This corresponds to $\Gamma_{ij} \sim \Gamma^D$, which is not surprising, given that the HQE can accommodate the charm meson lifetimes
\cite{Lenz:2013aua,Kirk:2017juj}.
However, the result for $\Gamma_2$, cf. \eqref{Uspindecomp}, \eqref{Uspinamps}, yields a value of $y$
%$s\bar s the whole expression of Eq.(\ref{Gamma12D}) one lies 
lying about four orders 
of magnitude below experiment, due to large GIM-cancellations between the $\Gamma_{ij} $ contributions.
Evidently, the inclusive approach is not well suited for analyzing the $U$-spin breaking responsible for $\DDbar$ mixing, i.e., the charm quark is not sufficiently massive.
First estimates 
of the dimension nine contribution in the HQE \cite{Bobrowski:2012jf} indicate an enhancement compared to the leading dimension six terms, but do not alter this conclusion.  The HQE result $\theta_c^2 \Gamma_{ij}\sim 5 \,y $ would require 
large $U$-spin violation, i.e. 





In a fit to the data which supplements the OPE with a $U$-spin breaking model for duality violation \cite{Jubb:2016mvq}, the latter yields ${\mathcal O}(20\%)$ corrections to the $\Gamma_{ij}$, corresponding to $\epsilon^2 = O(20\%)$ in $\Gamma_2$, cf. \eqref{Uspinamps}.


further supporting the need for a non-perturbative description of $\DDbar$ mixing. % in the SM.
One possibility is the second (exclusive) approach.



The starting point for the exclusive approach\cite{Falk:2001hx,Falk:2004wg,Cheng:2010rv,Gronau:2012kq,Jiang:2017zwr} 
%aims to determine $M_{12}^D$ and  $\Gamma_{12}^D$ at the hadron level. A potential starting point 
is a sum over the decay modes contributing to the absorptive and dispersive mixing amplitudes,
\beq
\Gamma_{12}^D =  \sum_f \rho_f                                            A_f^* \, \bar A_f  \, ;
~~~~~M_{12}^D  =   \langle {D}^0 | H |\dzbar \rangle 
+ P \sum_f{}^{\prime} \frac{    A_f^* \, \bar A_f  }
                                          {m_{D^0}^2 - E_f^2} \, ,
\label{Dmix_exclusive}
\eeq
where $A_f $ and $\bar A_f $ are the $D^0 \to f$ and $\dzbar \to f$ decay amplitudes, respectively, $\sum$ ($\sum^\prime$) denotes a sum over all on-shell (off-shell) hadronic states $f$ into which both the $D^0$ and $\bar{D}^0$ can decay, $\rho_f$ is the 
density of the state $f$, and $P$ is the principal value.   Unfortunately, the charm quark mass is not sufficiently light
for $D$ meson decays to be dominated by a few final states.  Moreover, the off-shell decay amplitudes
and strong phase differences entering the above expressions %are not directly measurable, and 
are not calculable from first principles.  
Thus, simplified treatments of $SU(3)_F$ flavor symmetry breaking have been utilized.
%(recall that the mixing amplitudes vanish in the symmetric limit, cf.~\eqref{Uspindecomp}--\eqref{Mspinamps}).  
A rough $U$-spin based estimate for $y$ is simply obtained from the first term in \eqref{Uspindecomp}, 
\beq y \,= \, \sin^2 \theta_c \times  \Gamma_2/ \Gamma^D\, \sim \,\sin^2\theta_C \,\times \, \epsilon^2  \sim (0.2 -5)\%,\eeq
where we have taken $\Gamma_2 \sim \Gamma^D \epsilon^2 $; and 
$\epsilon \sim 0.2 - 1$, corresponding to variation from nominal to maximal $U$-spin breaking.
The authors of \cite{Falk:2001hx,Falk:2004wg} only took $SU(3)_F$-breaking phase space effects into account in the exclusive sum.  They found that 
$y\lsim 1\%$ could naturally be realized, where a value at the high end would require contributions from higher multiplicity final states, due to larger $SU(3)_F$ breaking effects near threshold.
This conclusion was subsequently supported in \cite{Cheng:2010rv,Jiang:2017zwr}, which added experimental branching ratio inputs together with factorization based models for dynamical $SU(3)_F$ breaking effects, e.g., in the form factors and strong phases.
Rough dispersion relation treatments relating $x$ to $y$ in \cite{Falk:2004wg,Cheng:2010rv } suggested that $|x/y| \sim 0.1 -1$.


We conclude that the estimates of $x$ and $y$ in the SM are consistent with their measured values, cf. \eqref{xyexp}.  
Unfortunately, the large theoretical uncertainties eliminate the window for NP in these quantities. % is currently closed due to the large theoretical uncertainties.
The situation is markedly different for the CPV mixing observables, due to their large suppression in the SM, as discussed below.
On a very long time-scale, direct lattice calculations might be able to predict the
SM values of $x$ and $y$ by building on the methods described in \cite{Hansen:2012tf}.


% Diagonalisation of the 2x2 matrix describing the mixing of the neutral $D$ mesons
% gives the same Eigenwert equations like in the neutral $B$ systems:
% \begin{eqnarray}
% \Delta M_D^2 - \frac14 \Delta \Gamma_D^2 
% =  
% 4 \left| M_{12}^D \right|^2 - \left| \Gamma_{12}^D \right|^2 \, ,
% &&
% \Delta M_D \Delta \Gamma_D 
%  =  
% 4 \left| M_{12}^D \right| \left| \Gamma_{12}^D \right| \cos (\phi_{12}^D) \, ,
% \label{xyrelns} \end{eqnarray}
% with the mass difference $\Delta M_D$ and decay rate difference $ \Delta \Gamma_D$ of the mass 
% eigenstates of the neutral $D$ mesons. The box diagrams giving rise to $D$ mixing can have internal
% $d$, $s$ and $b$ quarks - compared to $u$, $c$, $t$ in the $B$ sector. $M_{12}^D$ denotes the 
% dispersive part of the box diagram, $\Gamma_{12}^D$ the absorptive part and the relative phase 
% of the two is given by $\phi_{12}^D = - \arg (-M_{12}^D / \Gamma_{12}^D)$. Unlike in the $B$ 
% system, where $|\Gamma_{12}/M_{12}| \ll 1$ holds, the expressions for  $\Delta M_D$ and 
% $\Delta \Gamma_D$ in terms of $M_{12}^D$ and $\Gamma_{12}^D$ cannot be simplified,  
% and for the knowledge of $\Delta M_D$ both $M_{12}^D$ and $\Gamma_{12}^D$ have to be determined. 
% On the other hand it is well-known that bounds like $\Delta \Gamma_D \leq 2 |\Gamma_{12}^D |$ 
% hold \cite{Nierste:2009wg,Jubb:2016mvq}.
% The experimental measurements \cite{Amhis:2016xyh} of the mass and decay rate differences 
% yield very small values
% \begin{eqnarray}\label{xyexp}
% x := \Delta M_D / \Gamma_D = 0.32 \% \pm 0.14 \% \, ,
% &&
% y := \Delta \Gamma_D / (2 \Gamma_D) = 0.61 \% \pm 0.07 \% \, ,
% \end{eqnarray}
% where $\Gamma_D$ denotes the total decay rate of the neutral $D$ mesons. While
% a decay rate difference in the neutral $D$ system is by now firmly established, the possibility of 
% having a vanishing mass difference is still not excluded - the future experimental sensitivity for $x$
% will be at the order of $0.001 \%$.
% \\
% $\Gamma_{12}^D$ can be expressed in terms of on-shell box diagrams differing in the 
% internal quarks - $(s \bar{s})$, $(s \bar{d})$, $(d \bar{s})$ and $(d \bar{d})$.
% Using the unitarity ($\lambda_d + \lambda_s +\lambda_b = 0$ with  $\lambda_x = V_{cx} V^*_{ux}$) of the CKM matrix \cite{Cabibbo:1963yz,Kobayashi:1973fv} one gets
% \begin{eqnarray}
% \Gamma_{12}^D & = & -   \lambda_s^2 \left(\Gamma_{ss}^D - 2 \Gamma_{sd}^D + \Gamma_{dd}^D\right)
%                     + 2 \lambda_s \lambda_b \left(\Gamma_{sd}^D - \Gamma_{dd}^D \right)
%                     -   \lambda_b^2 \Gamma_{dd}^D \, .
% \label{Gamma12D}
% \end{eqnarray}
% Eq.(\ref{Gamma12D}) shows a very pronounced CKM hierarchy: 
% expressed in terms of the Wolfenstein parameter \cite{Wolfenstein:1983yz} $\lambda \approx 0.225$ 
% (web-updates of \cite{Charles:2004jd,Bona:2006ah}) one has $\lambda_s \propto \lambda$ 
% and $\lambda_b \propto \lambda^5$. 
% In the exact $SU(3)_F$ limit $\Gamma_{ss}^D = \Gamma_{sd}^D = \Gamma_{dd}^D$ holds and 
% the first two terms of the r.h.s. of Eq.(\ref{Gamma12D}) vanish and only the tiny contribution from 
% the third term survives. 
% The determination of $M_{12}^D$ involves in addition box diagrams with internal $b$ 
% quarks and in contrast to $\Gamma_{12}^D$ now the dispersive part of the diagrams has 
% to be determined. Denoting the dispersive part of a box diagram with internal $x$ and $y$ 
% quarks by $M_{xy}^D$ and using again CKM unitarity one gets the following structure
% \begin{eqnarray}
% M_{12}^D \! & \! = \! &  \! \! \! \lambda_s^2          \left[ M_{ss}^D \! - 2 M_{sd}^D + M_{dd}^D \right]
%               \! + 2 \lambda_s \lambda_b  \left[ M_{bs}^D \! -   M_{bd}^D \! - M_{sd}^D + M_{dd}^D \right]
%               \!+   \lambda_b^2          \left[ M_{bb}^D \! - 2 M_{bd}^D + M_{dd}^D \right]\, .
% \label{M12D}
% \end{eqnarray} 
% In the case of neutral $B$ mesons the third term (replacing $b,s,d \to t,c,u$) is clearly dominant, while in the case 
% of $D$ mesons the extreme CKM suppression of $\lambda_b$ might be compensated by a less pronounced 
% GIM cancellation \cite{Glashow:1970gm} and in the end all three contributions of 
% Eq.(\ref{M12D}) could have a similar size.
% For the theoretical determination of $M_{12}^D$ and $\Gamma_{12}^D$ one can use a 
% quark-level (inclusive) or a hadron-level (exclusive) description. 
% The inclusive approach for $\Gamma_{12}^D$ is based on the heavy quark expansion (HQE) 
% \cite{Khoze:1983yp,Shifman:1984wx,Bigi:1991ir,Bigi:1992su,Blok:1992hw,Blok:1992he,Beneke:1998sy} and 
% works very well for the $B$ system \cite{Lenz:2014jha,Artuso:2015swg,Kirk:2017juj}.
% Surprisingly the HQE can successfully explain the large lifetime ratio of charm mesons 
% \cite{Lenz:2013aua,Kirk:2017juj} indicating an expansion parameter $\Lambda/m_c \approx 0.3$. 
% Lifetimes have the advantages of being free from GIM cancellations - to make statements concerning 
% the convergence of the HQE in the charm system more sound, future experimental and theoretical 
% studies of charmed baryon lifetimes will be necessary.
% Applying the HQE (the appearing non-perturbative matrix elements of dimension six operators have been
% determined in \cite{Carrasco:2014uya,Carrasco:2015pra,Bazavov:2017weg,Kirk:2017juj}) to 
% a single diagram contributing to $\Gamma_{12}^D$ - e.g. only internal
% $s \bar{s}$ quark - one gets five times the experimental value of $y$ \cite{Lenz:2016fcv}, 
% applying the HQE to the whole expression of Eq.(\ref{Gamma12D}) one lies about four orders 
% of magnitude below the experiment! Due to the success of the HQE in the $B$ system (where the 
% expansion parameter is only a factor of three smaller) and for $D$ meson lifetimes 
% it is unlikely that the HQE fails by four orders of magnitude in charm mixing - the problem seems
% to be rooted in severe GIM cancellations.
% In \cite{Jubb:2016mvq} it was shown that duality violations of the order of $20 \%$ could be 
% sufficient to explain the discrepancy. Another interesting idea 
% \cite{Georgi:1992as,Ohl:1992sr,Bigi:2000wn,Bobrowski:2010xg}
% is a lifting of the severe GIM cancellation in the first and second term of Eq.(\ref{Gamma12D})
% by higher terms in the HQE - overcompensating the $\Lambda/m_c$ suppression. First estimates 
% of the dimension nine contribution in the HQE for $D$ mixing \cite{Bobrowski:2012jf}
% indicate an enhancement compared to the leading dimension six terms, unfortunately not 
% large enough to explain the experimental value. Here a full theory determination of the HQE
% terms of dimension nine and twelve will provide further insight.
% It is interesting to note that the lifting of GIM cancellation in $D$-mixing by higher orders in the 
% HQE \cite{Georgi:1992as,Ohl:1992sr,Bigi:2000wn,Bobrowski:2010xg} could also yield a sizeable 
% CP violating phase in $\Gamma_{12}$, stemming from the second term on the r.h.s. of Eq.(\ref{Gamma12D}).
% According to \cite{Bobrowski:2010xg} values of up to $1 \%$ for $\phi_{12}^D$ can currently not be excluded.
% After settling the issues with the inclusive theory prediction for $\Gamma_{12}^D$ one could aim for
% a quark level determination of $M_{12}^D$.
% \\
% The exclusive approach\cite{Falk:2001hx,Falk:2004wg,Cheng:2010rv,Jiang:2017zwr} 
% aims to determine $M_{12}^D$ and  $\Gamma_{12}^D$ at the hadron level. A potential starting point are the expressions
% \begin{eqnarray}
% \Gamma_{12}^D & = & \sum \limits_n \rho_n \langle \bar{D}^0 | {\cal H}_{eff.}^{\Delta C =1} | n   \rangle
%                                           \langle  n        | {\cal H}_{eff.}^{\Delta C =1} | D^0 \rangle \, ,
% \\
% M_{12}^D & = & \sum \limits_n \langle \bar{D}^0 | {\cal H}_{eff.}^{\Delta C =2} |D^0 \rangle 
% + P \sum \limits_n \frac{ \langle \bar{D}^0 | {\cal H}_{eff.}^{\Delta C =1} | n   \rangle
%                                           \langle  n        | {\cal H}_{eff.}^{\Delta C =1} | D^0 \rangle }
%                                           {m_D^2 - E_n^2} \, ,
% \label{Dmix_exclusive}
% \end{eqnarray}
% where $n$ denotes all possible hadronic states in which both $D^0$ and$\bar{D}^0$ can decay into, $\rho_n$ is the 
% density of the state $n$ and $P$ is the principal value. Unfortunately a first principle calculation 
% of the arising matrix elements is beyond our current abilities, thus we have to make simplifying 
% assumptions like taking only the $SU(3)_F$-breaking effects due to phase space effects into account
% and neglecting any other hadronic effects. Doing so the authors of \cite{Falk:2001hx,Falk:2004wg} 
% found that $y$ and $x$ could naturally be of the order of a per cent. On the other hand such a treatment 
% clearly does not allow to draw strong conclusions about the existence of BSM effects, if the 
% measurement would disagree with these expectations.  The exclusive approach can be improved by 
% using experimental input, as done in \cite{Cheng:2010rv}, or by trying to take into account 
% additional dynamical effects compared to the phase space. In  \cite{Jiang:2017zwr} the
% factorization-assisted topological-amplitude approach was used for this purpose.
% On a very long time-scale also direct lattice calculations might also be able to predict the
% SM values for $D$-mixing by building up on methods described in \cite{Hansen:2012tf}.

\subsection{CP violation in $\DDbar$ mixing}
\label{sec:CPVDmixth}


% mention window for NP here unlike in the CP conserving obseravbles x, y
In the SM, CP violation (CPV) in $\DDbar$ mixing is highly suppressed, entering at 
{ ${\mathcal O}(|V_{cb} V_{ub} / V_{cs} V_{us}| ) \sim 10^{-3}$}. 
This raises several questions, which we briefly address, based on work to appear in \cite{GKLPPS}:
What is the resulting theoretical uncertainty on the indirect CPV 
observables? How large is the current window for New Physics (NP)?  
What is an appropriate parametrization for indirect CPV effects, 
given the expected sensitivity in the LHCb/Belle-II era?   


There are two types of CPV due to mixing; both are referred to 
as ``indirect'' CPV.  The first is CPV due to interference between the dispersive and absorptive mixing amplitudes (``CPVMIX''), which arises when $\phi_{12}\ne 0$.  CPVMIX can be directly measured via 
the semileptonic CP asymmetry 
\beq \label{ASL}
A_{\rm SL} \ \equiv\  
\frac{\Gamma(D^0\to K^+\ell^-\nu) - \Gamma(\dzbar\to K^-\ell^+\nu)}
{\Gamma(D^0\to K^+\ell^-\nu) + \Gamma(\dzbar\to K^-\ell^+\nu)} \ =\ 
 \frac{|q/p|^4 -1}{|q/p|^4 +1}\ = \frac{2 x_{12} y_{12}}{x_{12}^2 + y_{12}^2 } \sin\phi_{12}\,.
\eeq
The second type of indirect CPV is due to interference between a direct decay 
amplitude and a ``mixed'' amplitude followed by decay (``CPVINT''), 
i.e., interference between $D^0 \to f$ and $D^0 \to \dzbar \to f $. 
For decays to a CP eigenstate final state, there are two CPVINT observables
%, introduced in
\cite{Branco:1999fs}, 
\beq 
\lambda^M_{f} \equiv \,\frac{M_{12}}{|M_{12} |}\,
\frac{A_f } { \overline A_f} = \,\eta_{f}^{CP}\,\left|\frac{A_f }{ \overline A_f }\right| \, e^{ i  \phi_f^M }\,,~~~
\lambda^\Gamma_{f} \equiv \,\frac{\Gamma_{12} }{|\Gamma_{12} |}\,
\frac{A_f } {\overline A_f} = \,\eta_{f}^{CP}\,\left|\frac{A_f } {\overline A_f}\right| \, e^{ i  \phi_f^\Gamma }
   \,,\label{lambdaMf}\eeq   
parametrizing the interference for a dispersive and absorptive mixing amplitudes, respectively.
The $\phi^{M}_f$ and $\phi^{\Gamma}_f$ are CPV weak phases, with $\phi_{12} = \phi_{f}^M - \phi_{f}^\Gamma$,  while $\eta_f^{CP} = + (-)$ for CP even (odd) final states. 
%For decays to a CP-conjugate pair of final states $f$ and $\bar f$,
%there are two pairs of observables analogous to \eqref{lambdaMf}, which depend on the weak phases $\phi^{M}_f$ and $\phi^{\Gamma}_f$, as well as the strong phase difference between $A_f $ and $ \overline A_f $.
%The absorptive and dispersive phases satisfy $\phi_{12} = \phi_{f}^M - \phi_{f}^\Gamma$.
%\beq \phi_{12} = \phi_{f}^M - \phi_{f}^\Gamma \,.\label{phiMfmphiGf}\eeq
In general, $\phi^{M}_f$ and $\phi^{\Gamma}_f$ 
are final-state specific due 
to non-universal weak and strong phases entering the CKM suppressed SM 
(and potential NP) contributions to the subleading decay amplitudes. 


Non-vanishing $\phi^{M}_f$ and $\phi^\Gamma_f$
yield {\it time-dependent} CP asymmetries. For example, in singly Cabibbo suppressed (SCS) decays to 
CP-eigenstates, $f=K^+K^-,\,\pi^+ \pi^-,...$, the effective decay widths, $\hat{\Gamma}$, for $D^0$ and
$\dzbar$ decays will differ,
\beq 
\Delta Y_f \ \equiv\ 
\frac{\hat \Gamma^{}_{\dzbar \to f} - \hat \Gamma^{}_{D^0 \to f}} {2 \Gamma_D}
\ = \,- x_{12} \sin\phi_f^M  + a_f^d \, y_{12} \,.
\label{eqn:scs_asymm1}
\eeq
%where $\Delta Y_f = - A_\Gamma$.
The first and second terms on the RHS are the dispersive CPVINT and direct CPV contributions, respectively, % (which is subleading, as we will see below), 
where the direct CP asymmetry is defined as $a_f^d = 1 - \left|\bar A_f / A_f \right|$. 
%\beq a_f^d = 1 - \left|\bar A_f / A_f \right| . \label{afdCP}\eeq
They can, in principle, be disentangled
via measurements of the corresponding time-integrated CP asymmetries \eqref{ACPhh}, which satisfy $A_{CP} (D^0 \to h^+ h^-) = \Delta Y_{h^+ h^-} \,\langle t\rangle/ \tau^{}_D   +\ a_{h^+ h^-}^{d}$.
Examples of time-dependent CP asymmetries in decays to non-CP eigenstates include
the SCS final states $f=K^* K$ or $f=\rho\pi$, and the Cabibbo favored/doubly Cabibbo suppressed
(CF/DCS) final states $f=K^\pm \pi^\mp$.  These asymmetries generally depend on both $\phi_f^M$ and $\phi_f^\Gamma$, unlike decays to CP eigenstates, due to the additional strong phases \cite{GKLPPS}.
%difference between $A_f$ and $\overline A_f$.%$\Delta_f$ \cite{GKLPPS}.

The dispersive and absorptive observables are simply related \cite{GKLPPS} 
to the more familiar parametrization of indirect CPV, see, e.g., \cite{Nir:1992uv}. The latter consists 
of $ |q/p|-1 $, 
and  
\beq{\lambda_f} \equiv \frac{q}{p} \frac{ \bar A_f }{A_f} = - \eta_f^{ CP} \left|\lambda_f \right| e^{i \,\phi_{\lambda_f}},\label{lambdaf}\eeq
for CP eigenstate final states, and pairs of observables ${\lambda_f} $, $\lambda_{\bar f} $ 
for non-CP eigenstate final states.
The same number of independent parameters is employed in each case (recall that $\phi_{12} = \phi_f^M - \phi_f^\Gamma$).


% With the continuing improvements in experimental sensitivity at LHCb,  
Single mrad precision for $\phi_{12}$ could become a realistic target at LHCb, see below and Sec.~ \ref{sec:expcharmmix}, to be 
compared with the current $\sim 50-100$ mrad bounds 
[95\% CL] quoted by HFLAV and UTfit Collaboration~\cite{Amhis:2016xyh,LHCbimplications}. 
Thus, we must estimate the final state dependence in $\phi_f^M$, $\phi_f^\Gamma$
%(recall that $\phi_{12} = \phi_f^M-\phi_f^\Gamma$) 
due to the subleading decay amplitudes, and consider how to best parameterize this.
This is accomplished via the $U$-spin flavor symmetry decomposition of the $\DDbar$ mixing amplitudes given in Eq. \eqref{Uspindecomp}.
%--\eqref{Mspinamps}.
We define three theoretical CPV phases 
%with respect to ${\rm arg}[(\lambda_s - \lambda_d)^2]$, the direction of the $\Delta U=2$ contributions $\Gamma_2$, $M_2$ in the mixing amplitude complex plane,
\beq \label{theorphasesdef}
{ \phi_2^\Gamma }\equiv  {\rm arg}\left(\frac{\Gamma_{12}} {\Gamma_{12}^{\Delta U=2} }\right),~~{ \phi_2^M }\equiv  {\rm arg}\left(\frac{M_{12} }{\Gamma_{12}^{\Delta U=2} }\right),~~\phi_2 \equiv  {\rm arg}\left(\frac{q} { p} \frac{1}{\Gamma_{12}^{\Delta U=2} } \right)\,,
\eeq
%The phases $\phi_2^M$, $\phi_2^\Gamma$, and $\phi_2$ 
which are the theoretical analogs of the final state dependent phases $\phi_f^M$, $\phi_f^\Gamma$, and $\phi_{\lambda_f}
$, respectively.
The $U$-spin breaking hierarchy $\Gamma_1/ \Gamma_2 = {\mathcal O}(1/\epsilon)$ %cf. \eqref{Gamma12decomp}, 
yields
the estimate
\beq \begin{split} \label{theorphases}
{ \phi_2^\Gamma }&  \approx {\rm Im}\left(  \frac{2 \lambda_b }{ \lambda_s - \lambda_d}  \frac{\Gamma_1 }{\Gamma_2} \right)
\sim \left| \frac{\lambda_b }{ \theta_c } \right| \sin \gamma  \times \frac{1} { \epsilon},
\end{split}
\eeq
and similarly for $\phi_2^M$ (the $O(\lambda_b^2)$ contributions are negligible).
Taking the nominal value $\epsilon \sim 0.2$ for $U$-spin breaking, 
%and noting that $\phi_{12} =  \phi_2^M - \phi_2^\Gamma$,
% in \eqref{theorphases}, 
we obtain the rough 
SM estimates
\beq  \phi_{12} \sim \phi_2^\Gamma \sim \phi_2^M \sim 3 \times 10^{-3} 
\,.\label{phiGphiMnaive}\eeq
Comparison with the current 95\% CL bounds on $\phi_{12}$ 
implies that there is an ${\mathcal O}(10)$ window for NP in indirect CPV.
An alternative expression for $\phi_2^\Gamma$ follows from \eqref{theorphases} via the relation $\Gamma_2  \cong y \,\Gamma^D$, 
 \beq |\phi_2^\Gamma| =  \left|\frac{\sin\gamma\,\, \lambda_b\,\lambda_s 
} { y }\right| \,\frac{|\Gamma_1 |} { \Gamma^D } \approx 0.005 \, \frac{|\Gamma_1 |} { \Gamma^D } \sim 0.005\, \epsilon  \,,\label{refined}\eeq
where in the last relation we have taken $\Gamma_1 \sim \Gamma^D \epsilon$.
In principle, $\Gamma_1$ can be estimated via the exclusive approach as more data on SCS $D^0$ decay branching ratios and direct CP asymmetries becomes available.



The misalignments between $\phi_f^M$, $\phi_f^\Gamma$, $\phi_{\lambda_f}$ in \eqref{lambdaMf}, \eqref{lambdaf}, and their theoretical counterparts satisfy  %these phases, for given final state $f$, satisfy
\beq {{ \delta \phi_f } \equiv \phi_{f}^\Gamma - \phi_{2}^\Gamma = \phi_{f}^M - \phi_{2}^M  = \phi_2 -\phi_{\lambda_f}} \,.\eeq
We can characterize the magnitude of the misalignment $\delta \phi_f$ in the SM as follows:
(i) For CF/DCS decays it is precisely known and negligible, i.e., $\delta \phi_f = {\mathcal O}(\lambda_b^2 / \theta_c^2 )$;
(ii) In SCS decays, $\delta \phi_f$ is related to direct CPV as $\delta \phi_f = a_f^d  \,\cot \delta $
 via the $U$-spin decomposition of the decay amplitudes \cite{Brod:2012ud}, where
a strong phase $\delta = {\mathcal O}(1)$ is expected due to 
large rescattering at the charm mass scale.  Thus, for $f = \pi^+ \pi^-,\, K^+ K^-$,
the experimental bounds $a_f^d \lsim {\mathcal O}(10^{-3})$ imply that $\delta \phi_f \lsim {\mathcal O}(10^{-3})$; 
(iii) In SCS decays, $\delta \phi_f = {\mathcal O}( \lambda_b \sin\gamma /\theta_c ) \times \cot\delta$, i.e., it is ${\mathcal O}(1)$ in $SU(3)_F$ breaking.  Thus,
%at which quantities arise, find $\phi_{12}^\Gamma = O(1/\epsilon) $.  
\eqref{theorphases} yields ${\delta \phi_f  /\phi_{2}^\Gamma } = {\mathcal O}({ \epsilon})$,
implying an order of magnitude suppression of the misalignment. % relative to the 


%We learn that
In the LHCb era, 
% and assuming $\lsim 10$ mrad sensitivity
% , 
a single pair of dispersive and absorptive phases, identified with $\phi_2^M$ and  $\phi_2^\Gamma$, respectively, should suffice to parametrize all indirect CPV effects. We refer to this fortunate circumstance as {\it approximate universality}.  Moreover, approximate universality generalizes beyond the SM under the following conservative assumptions about NP %decay amplitudes containing new weak phases: 
contributions:
(i) they can be neglected in CF/DCS decays (a highly exotic NP flavor structure would otherwise be required in order to evade the $\epsilon_K$ constraint~\cite{Bergmann:1999pm}); (ii) in SCS decays they are of similar size or  
%magnitudes are similar to, or 
smaller than the SM QCD penguin amplitudes, as already hinted at by the experimental bounds on the direct CP asymmetries $a^d_{K^+K^-},\,a^d_{\pi^+\pi^-}$. These assumptions can be tested by future direct CPV measurements.


Under approximate universality, 
%the phases $\phi_f^M$ and $\phi_f^\Gamma$ are replaced with $\phi_2^M$ and $\phi_2^\Gamma$, respectively, 
$\phi_f^M\to \phi_2^M$ and $\phi_f^\Gamma\to\phi_2^\Gamma$
in expressions for time-dependent CP asymmetries. 
A global fit to the CPV data for any two of the three phases, 
$\phi_2^M$, $\phi_2^\Gamma$, $\phi_{12}$, is equivalent to the traditional
two-parameter fit for 
%the parameters 
$|q/p|$ and $ \phi$
, where $ \phi$ is identified with $\phi_2$~\cite{GKLPPS}.
The relations 
\beq
\left| \frac{q} { p}\right| -1   \approx  \frac{|x| |y| } { x^2 + y^2 } \sin{ \phi_{12}} ,~~~~~
\tan 2 (\phi_2 + { \phi^\Gamma})  \approx  - \frac{x_{12}^2 } { x_{12}^2 + y_{12}^2  }   
\sin 2{ \phi_{12}} \,,
\label{beyondsuperweak}
\eeq
together with $\phi_{12} = \phi_2^M - \phi_2^\Gamma$, allow one to translate between 
$(\phi_2, |q/p|)$ and $(\phi_2^M, \phi_2^\Gamma)$. 
To illustrate the potential reach of LHCb in $\phi_2^M$ and $\phi_2^\Gamma$ from prompt charm production at 300 fb$^{-1}$,
the projected statistical errors on $\phi$, $|q/p|$, $x$, $y$ from $D^0 \to K_s \pi^+ \pi^-$ (combined with the Belle 
error correlation matrix \cite{Peng:2014oda}), and on $\Delta Y_f = -A_\Gamma$,  
given in the last rows of Tables \ref{tab:D0KSPiPi_yields} and \ref{tab:Agamma_yields}, and assuming the central values $|q/p|=1$, $\phi=0$, $x=0.57\%$, and $y=0.7\%$, yield 
\beq \sigma (\phi_2^M ) = 2~\text{mrad},~~~ \sigma (\phi_2^\Gamma ) = 5~\text{mrad}\,.\eeq 
The projected statistical errors are  smaller for $D^0 \to K^+ \pi^-\pi^-\pi^+$ than for $D^0 \to K_s \pi^+ \pi^-$, cf. Tab.~\ref{tab:WS-D2K3pi}.
Thus, if the systematic errors are not prohibitively large, cf. Sec.~\ref{sec:expcharmmix},
LHCb could probe indirect CPV in the SM.


\subsection{Direct CP violating probes}
%Two-body hadronic $D$-meson decays can be classified
Direct \CP asymmetries,
\begin{equation}\label{ACPhh}
A_{\CP}(\Dz\to h^-h^+)\equiv \frac{\Gamma(D^0\rightarrow h^-h^+)-\Gamma(\Dzb\rightarrow h^-h^+)}{\Gamma(D^0\rightarrow h^-h^+)+\Gamma(\Dzb\rightarrow h^-h^+)},
\end{equation}
with $h$ a light meson, are suppressed in the SM but could be enhanced by NP.
%The sensitivity to direct \CP violation is enhanced through a measurement of the difference in \CP asymmetries between \Dz\to\Km\Kp and \Dz\to\pim\pip decays, $\Delta A_{\CP}=A_{\CP}(\Km\Kp)-A_{\CP}(\pim\pip)$.
%The individual asymmetries $A_{\CP}(\Km\Kp)$ and $A_{\CP}(\pim\pip)$ can also be measured. 
Quite likely the most prominent test of direct CPV in charm is the
observable 
$
\Delta A_{\CP}=A_{\CP}(\Km\Kp)-A_{\CP}(\pim\pip),
$
%(see Sec.~\ref{sec:directCP:exp} for the
%definition). 
which measure the difference between direct CPV in SCS modes $D^0 \to K^+ K^-$ and $D^0 \to \pi^+ \pi^-$. This is despite that the
SM prediction is hard to obtain.  However, assuming nominal breaking of
the $SU(3)$ flavor symmetry, $\epsilon\approx f_K/f_\pi - 1 \approx
{\mathcal O}(20\,\%)$, one can infer from the observed branching ratios of the
CF decay$D^0 \to \pi^+ K^-$ and DCS decay $D^0
\to \pi^- K^+$ a consistent picture involving large
 matrix elements for $U$-spin breaking penguin~\cite{Brod:2012ud,
  Savage:1991wu, Pirtskhalava:2011va, Feldmann:2012js, Franco:2012ck,
  Hiller:2012xm}. These can account, in the presence of large strong
phases,  for values of $\Delta A_{\CP}
\lesssim 0.2\%$~\cite{Brod:2011re}, in accordance with the current
measurements. LHCb is expected to probe far into this region after
Upgrade~II (see Sec.~\ref{sec:directCP:exp}).

SCS $D$ decay modes are sensitives probes of CP violation in and
beyond the SM~\cite{Grossman:2006jg}. Other decay modes can also be
sensitive to CP asymmetries of the same order, both in the SM and in
NP extensions that modify the QCD penguin operators. Besides the modes
$D^+ \to K^+ \bar K^0$ and $D_s^+ \to \pi^+ K^0$, obtained from the
above via exchange of the spectator quark, the mode $D^0 \to K_S K_S$
is particularly interesting, because a large CP asymmetry $\lesssim
1\,\%$ can be expected~\cite{Brod:2011re, Nierste:2015zra}. The
related $D \to K K^*$ modes have smaller asymmetries, but this can be
compensated by the higher experimental
efficiency~\cite{Nierste:2017cua}.

Another interesting class of observables are the semileptonic $D$
decays, $D \to P \ell^+ \ell^-$ and $D \to P_1 P_2 \ell^+ \ell^-$,
where $P, P_1, P_2$ are light pseudoscalars. In analogy to the
well-studied corresponding semileptonic $B$ decay modes, interesting
null test of CP violation can be constructed~\cite{deBoer:2015boa,
  deBoer:2017que, deBoer:2018buv}. In certain regions of phase space,
the CP asymmetries can be largely enhanced via interferences with
resonances, which makes these decays an interesting class of
observables for LHCb.


\subsection{Null tests from isospin sum rules}
\label{sec:isospin:null:test}
SM predictions for hadronic $D$ decays are notoriously difficult. In
some cases it is possible to obtain strong indications for the
presence of NP by relating various modes using the
approximate flavor symmetry $SU(3)$ (invariance under interchange of
up, down, and strange quarks) or the more precise isospin (invariance under interchange of
up and down quarks). 

Probably the simplest example is the $D^+ \to \pi^+
\pi^0$ decay~\cite{Brod:2011re}. The final state has isospin $I=2$ which
cannot be reached from the $I = 1/2$ initial state via the $\Delta I =
1/2$ QCD penguin operators, predicting very suppressed direct CPV in the SM. An important question in this context is
the size of isospin-breaking effects. Isospin-breaking due to QED and
the difference of up- and down-quark mass is CP conserving and can be
safely neglected. The electroweak penguin contribution is relatively
suppressed by $\alpha/\alpha_s$ in the SM. Thus, enhanced direct CPV in $D^+ \to \pi^+ \pi^0$ would signal the presence of
isospin-violating NP.

Another example is the sum of rate differences~\cite{Grossman:2012eb}
\begin{equation}
\begin{split}
  |A_{\pi^+ \pi^-}|^2 - |\bar A_{\pi^+ \pi^-}|^2 & + |A_{\pi^0
    \pi^0}|^2 - |\bar A_{\pi^0 \pi^0}|^2 \\ & - \frac{3}{2} \big(
  |A_{\pi^+ \pi^0}|^2 - |\bar A_{\pi^- \pi^0}|^2 \big) = 3 \big(
  |A_{1/2}|^2 - |\bar A_{1/2}|^2 \big),
\end{split}
\end{equation}
which depends only on $\Delta I = 1/2$ amplitudes. There are two
possibilities. If the sum is non-zero, there are $\Delta I = 1/2$
contributions to CP violation; they can be due to SM or NP. If the sum
is zero, but the individual asymmetries are non-zero, the CP
asymmetries are likely dominated by $\Delta I = 3/2$ NP
contributions. More sum rules, involving also vector meson final
states, can be devised for the decay modes $D \to \rho \pi$, $D \to
K^{(*)} \bar K^{(*)} \pi (\rho)$, $D_s^+ \to K^* \pi
(\rho)$~\cite{Grossman:2012eb}. For an exhaustive list of sum rules
based on the flavor $SU(3)$ or its subgroups, 
%including
%also vector meson final states, 
see Refs.~\cite{Grossman:2012ry,
  Grossman:2013lya, Muller:2015rna}.

\subsection{Radiative and leptonic charm decays}
In the down-type quark sector the GIM suppression is less effective than in the up-quark sector, due to the presence of $top$-quarks running in the loops which leads to an effective decoupling with the other diagrams, owing to the large top quark mass. In the \emph{charm} sector FCNCs are instead more effectively suppressed due to the absence of a large mass down-type quark and therefore, within the SM, branching fractions of rare $D$ decays have very small values. However, these processes  can be enhanced in NP scenarios by up to several orders of magnitude when compared to the SM.  

\subsubsection{Radiative Decays}
The  $c \to u$ transition  occurs  in the SM  only at loop level, providing an option to 
study effects of NP in the up-like quark sector. 
The  GIM  mechanism leads to the  strong suppression in the 
branching ratios  of   inclusive $c\to u \gamma$. However, in the exclusive charm decays,  non-perturbative QCD effects become very important. %\section{The decay $D\to V\gamma$}
The effective weak Lagrangian for $D\to V\gamma$ is, see, e.g., \cite{deBoer:2017que},
\begin{align}
\label{eq:Leff:e1}
 \mathcal L_\text{eff}^\text{weak}=\frac{4G_F}{\sqrt 2}\left(\sum_{q\in\{d,s\}}V_{cq}^*V_{uq}\sum_{i=1}^2C_iO_i^{(q)}+\sum_{i=3}^6C_iO_i+\sum_{i=7}^8\left(C_i)_i+C_i'O_i'\right)\right)\,,
\end{align}
with the operators ${\cal O}_{2(1)}=(\bar u_L\gamma_\mu(T^a)q_L)(\overline q_L\gamma^\mu(T^a)c_L)$, ${ \cal O}_7^{(\prime)}={e\, }{m_c}/{16\pi^2}(\bar u_{L(R)}\sigma^{\mu\nu}c_{R(L)})F_{\mu\nu}$ and $ { \cal O}_8^{(\prime)}={g_s\,m_c}/{16\pi^2}(\bar u_{L(R)}\sigma^{\mu\nu}T^ac_{R(L)})G^a_{\mu\nu}$. 
%\begin{align}
%\label{e2}
% & { \cal O}_{2(1)}=(\bar u_L\gamma_\mu(T^a)q_L)(\overline q_L\gamma^\mu(T^a)c_L)\,,\nonumber\\
% & { \cal O}_7^{(\prime)}=\frac{e\,m_c}{16\pi^2}(\bar u_{L(R)}\sigma^{\mu\nu}c_{R(L)})F_{\mu\nu}\,,\nonumber\\
% & { \cal O}_8^{(\prime)}=\frac{g_s\,m_c}{16\pi^2}(\bar u_{L(R)}\sigma^{\mu\nu}T^ac_{R(L)})G^a_{\mu\nu}\,,
%\end{align}
%as explained in details in Ref. \cite{deBoer:2017que}. 
The SM effective Wilson coefficients $C_i^\text{(eff)}$ are known at  two-loop level in QCD \cite{deBoer:2017que,deBoer:2016dcg,Greub:1996wn,Fajfer:2002gp}.
The authors of Ref. \cite{deBoer:2017que}  improved the SM prediction for the branching ratios of $D\to V\gamma$, by  including power corrections and updating the hybrid model predictions. The hybrid model combines   the heavy quark effective theory and chiral perturbation theory using experimentally measured parameters \cite{Fajfer:1997bh,Fajfer:1998dv}. 
They also  included corrections to the perturbative Wilson coefficients by employing  a QCD based approach, worked out for $B$ physics as reviewed in  \cite{deBoer:2017que}. Updated values for the  SM Wilson coefficients %of $Q_{1,2}$ and the effective coefficient of the chromomagnetic dipole operator
at leading order in $\alpha_s$ are given in  \cite{deBoer:2017que,deBoer:2015boa,deBoer:2016dcg}. 
%\begin{align}
 %&C_1^{(0)}\in[-1.28,-0.83]\,,\quad C_2^{(0)}\in[1.14,1.06]\,,\nonumber\\
 %&C_8^{(0)\text{eff}}\in[0.47\cdot10^{-5}-1.33\cdot10^{-5}i,0.21\cdot10^{-5}-0.61\cdot10^{-5}]\,,\nonumber\\
  %&C_7^\text{eff}\in[-0.00151-(0.00556i)_\text{s}+(0.00005i)_\text{CKM},-0.00088-(0.00327i)_\text{s}+(0.00002i)_\text{CKM}]\,.
%\end{align}
%In \cite{deBoer:2017que} authors varied $\mu_c$ i within $[m_c/\sqrt2,\sqrt2m_c]$.
The GIM mechanism  suppresses strongly  $C_8^{\text{eff}}$ in the SM. 
%The  effective  Wilson coefficient $C_7^{\prime\,\text{eff}}\sim m_u/m_c\simeq 0$ \cite{deBoer:2017que}.   
%the data for  \cite{deBoer:2017que,deBoer:2016dcg,deBoer:2015boa} 
The experimental branching ratios  for the Cabibbo allowed decay  $BR(D^0\to K^*(892) \gamma) = (3.31\pm 0.34)\times 10^{-4}$ and Cabibbo suppressed  decays 
$BR(D^0\to\phi\gamma) = (2.73\pm 0.35)\times 10^{-5}$    \cite{Patrignani:2016xqp} %\td {should this be $D^0\to \rho^0\gamma$?:} 
%\td{what is meant here?:} 
$BR(D^0 \to D^0\to \rho^0 \gamma) = (1.77\pm 0.30\pm 0.07)\times 10^{-5}$ \cite{Abdesselam:2016yvr}. The hybrid model accompanied with  the weak annihilation  predicts %the rates 
$BR(D^0 \to \rho^0 \gamma)_{SM}= (0.11-3.8)\times 10^{-5}$, $BR(D^0\to\phi\gamma)_{SM} = (0.24-2.8))\times 10^{-5}$,   $BR(D^0\to\bar K^{*0} \gamma)_{SM} = (0.26-4.6)\times 10^{-4}$    \cite{deBoer:2017rzd} \cite{deBoer:2017que}. The NP 
scenarios discussed in Ref.   \cite{deBoer:2017rzd} can contribute at the loop level to $C_{7,8}^\text{eff}$, and cannot significantly modify the branching ratios.
% making the modification of the branching ratio insignificant.
 %\td{please check the rewrite:} 
 However, the NP induced CPV asymmetry was found to still be modest, e.g.,  $\sim {\mathcal O}(10\%)$ for $D^0 \to \rho^0 \gamma$. In the case of baryonic mode $\Lambda_c \to p \gamma$ the rate is estimated to be $\sim {\cal O} (10^{-5})$. The forward-backward asymmetry of photon momentum relative to $\Lambda_c$ boost can be $0.2$ in the SM, and somewhat smaller  within scenarios of NP discussed in \cite{deBoer:2017rzd}. 
 
 
\subsubsection{Rare charm leptonic decays} 
To describe NP effects the effective Lagrangian for the $c\to u \bar \ell \ell $ has to be extended by the following operators
\begin{equation}
  \label{eq:operators}
  \begin{split}
  \qquad \quad {\cal O}_{9(10)} &= \frac{e^2}{(4\pi)^2}\, (\bar u \gamma^\mu P_L c)
  (\bar\ell \gamma_\mu (\gamma_5)\ell) \,, \qquad \quad  
%    {\cal O}_{10} = \frac{e^2}{(4\pi)^2}\, (\bar u \gamma^\mu P_L c)
%  (\bar\ell \gamma_\mu \gamma_5 \ell)\, , \\
%   \qquad \quad 
   {\cal O}_{S(P)}  = \frac{e^2}{(4\pi)^2}\, (\bar u P_R c)
  (\bar\ell (\gamma_5) \ell)\,, 
% \qquad \quad \qquad   {\cal O}_P = \frac{e^2}{(4\pi)^2}\, (\bar u P_R c) (\bar\ell \gamma_5 \ell)\, ,
\\
    \qquad \quad  {\cal O}_{T} &= \frac{e^2}{(4\pi)^2}\, (\bar u \sigma_{\mu\nu} c)
  (\bar\ell \sigma^{\mu\nu}  \ell)\,,
 \qquad \qquad\quad   {\cal O}_{T5} = \frac{e^2}{(4\pi)^2}\, (\bar u \sigma_{\mu\nu} c)
  (\bar\ell \sigma^{\mu\nu} \gamma_5  \ell).
\end{split}
\end{equation}
%\begin{equation}
%  \label{eq:operators}
%  \begin{split}
%  \qquad \quad {\cal O}_9 &= \frac{e^2}{(4\pi)^2}\, (\bar u \gamma^\mu P_L c)
%  (\bar\ell \gamma_\mu \ell) \,, \qquad \quad  
%    {\cal O}_{10} = \frac{e^2}{(4\pi)^2}\, (\bar u \gamma^\mu P_L c)
%  (\bar\ell \gamma_\mu \gamma_5 \ell)\, , \\
%   \qquad \quad {\cal O}_{S} & = \frac{e^2}{(4\pi)^2}\, (\bar u P_R c)
%  (\bar\ell \ell)\,, 
% \qquad \quad \qquad   {\cal O}_P = \frac{e^2}{(4\pi)^2}\, (\bar u P_R c) (\bar\ell \gamma_5 \ell)\, ,\\
%    \qquad \quad  {\cal O}_{T} &= \frac{e^2}{(4\pi)^2}\, (\bar u \sigma_{\mu\nu} c)
%  (\bar\ell \sigma^{\mu\nu}  \ell)\,,
% \qquad \quad   {\cal O}_{T5} = \frac{e^2}{(4\pi)^2}\, (\bar u \sigma_{\mu\nu} c)
%  (\bar\ell \sigma^{\mu\nu} \gamma_5  \ell)\,.\\ 
%\end{split}
%\end{equation} 
Among these only the $C_9$ is nonzero in the SM. For each ${\cal O}_i$ one can introduce ${\cal O}_i^\prime$ with the corresponding Wilson coefficient $C_i^\prime$ by replacing the chirality operator $P_L= 1/2(1-\gamma^5)$ by $P_R= 1/2(1+\gamma^5)$ \cite{Fajfer:2015mia}. 
%the Wilson coefficient of  operator ${\cal O}_9$ differs from 0. 
In Ref. \cite{deBoer:2015boa} authors obtained (N)NLO QCD SM Wilson coefficients at $\mu_c=m_c$, %\td{these should be dimensionful, if the definition in \eqref{eq:Leff:e1} applies} 
$C_{7} \simeq( -0.0011-0.0041 i)$ and $C_9 \simeq  -0.021 X_{ds}$, where $X_{ds}=V_{cd}^*V_{ud}L(m_d^2,q^2)+V_{cs}^*V_{us}L(m_s^2,q^2)$,  with $L(m^2, q^2)$ defined in eq. (B1) of \cite{deBoer:2015boa}. 
%\begin{align}\label{eq:one_loop_qbarq}
 %L(m^2,q^2)=\frac53+\ln\frac{\mu_c^2}{m^2}+x-\frac12(2+x)|1-x|^{1/2}
% \begin{cases}
 % \ln\frac{1+\sqrt{1-x}}{1-\sqrt{1-x}}-i\pi\quad x\equiv\frac{(2m)^2}{q^2}<1\\
%  2\tan^{-1}\left[\frac1{\sqrt{x-1}}\right]\quad x\equiv\frac{(2m)^2}{q^2}>1. 
% \end{cases}
%\end{align}
%\td{What is $L$?}.
 In the range of $m_c/\sqrt2\le\mu_c\le\sqrt2m_c$ the  effective Wilson coefficients  were found to be $(-0.0014-0.0054i)\le C_7\le(-0.00087-0.0033i)$ and 
 $-0.060X_{ds}(\mu_c=\sqrt2m_c)\le C_9\le0.030 X_{ds}(\mu_c=m_c/\sqrt2)$. In the small 
 $q^2 \gtrsim 1 \, \mbox{GeV}^2$ region  $|C_{9}| \lesssim 5\cdot10^{-4}$.  

One can consider contribution of this effective Lagrangian in the exclusive decay channels. For the $D$ meson di-leptonic decays the best upper bound to date is obtained by the LHCb collaboration at  the $90\%$ CL 
~\cite{Aaij:2013cza}
%\begin{equation}
 % \label{eq:Dmumu-exp}
$BR(D^0 \to \mu^+ \mu^-) < 6.2 \times 10^{-9}$.
%\end{equation}
In the decay
$D^+ \to \pi^+ \mu^+ \mu^-$ the LHCb experiment %cused on kinematic
determined bounds on the branching ratio in the two kinematic regions of the di-lepton mass, %$q^2 = (k_- + k_+)^2$, 
 chosen to be either below or
above the dominant resonant contributions. The measured
total branching ratio, obtained by extrapolating spectra over the
non-resonant region, is~\cite{Aaij:2013sua}
%\begin{equation}
%  \label{eq:Dpimumu-exp}
$BR(D^+ \to \pi^+ \mu^+ \mu^-) < 8.3 \ 10^{-8}$,
%\end{equation}
while the separate branching fractions in the low- and high-$q^2$ bins are bounded to be below 
$BR(\pi^+ \mu^+ \mu^-)_{I}<  2.5\times 10^{-8}$ for region I, ${q^2 \in [0.0625,0.276]\,{GeV}^2} $, and
    $BR(\pi^+ \mu^+ \mu^-)_{II}  <  2.9 \times 10^{-8}$ for region II, ${q^2 \in [1.56,4.00]{GeV}^2}$~\cite{Aaij:2013sua}. 
%\footnote{Note that the high-$q^2$ bin quoted by
 % the experiment extends beyond the maximal allowed $q^2_\mrm{max} = (m_D-m_\pi)^2 = 2.99\e{GeV}^2$.}:
%\begin{equation}
%  \label{eq:Dpimumu-bins}
%  \begin{split}
%    $BR(\pi^+ \mu^+ \mu^-)_{I} \equiv BR(D^+ \to \pi^+ \mu^+
%      \mu^-)_{q^2 \in [0.0625,0.276]\,{GeV}^2} <  2.5\times 10^{-8}$  and
%    $BR(\pi^+ \mu^+ \mu^-)_{II} \equiv BR(D^+ \to \pi^+ \mu^+
%      \mu^-)_{q^2 \in [1.56,4.00]{GeV}^2} <  2.9 \times 10^{-8}$. 
 % \end{split}
%\end{equation}
These can be used to put bounds on Wilson coefficients. Allowing for NP contributions to only one 
 Wilson coefficient at a time gives the upper bounds listed in Table~\ref{tab:BR}, where $\tilde C_i = V_{ub} V_{cb}^* C_i$\cite{Fajfer:2015mia}.
% to have a real nonzero
%value one can extract its upper bound  as presented in \cite{Fajfer:2015mia}. %This is repeated for each choice of random phases and moduli of Breit-Wigner parameters $a_{\eta,\rho,\omega,\phi}$, where the latter are sampled in their$1\sigma$ regions (90\% CL bound for $|a_\omega|$),c.f. Tab.~\ref{tab:bw}. 
%The most relaxed bound obtained in this way
%are presented in Tab.~\ref{tab:BR}, where 
%$\tilde C_i = V_{ub} V_{cb}^* C_i$ notation is being used.  At the same time
The bound on $BR(D^0 \to \mu^+ \mu^-)$ 
%can give bound on Wilson coefficients $C_{10}$, $C_{S}$, and
%$C_P$. It turns out that the upper bound on $BR(D^0\to \mu^+ \mu^-)$ is
gives the most stringent bounds on 
%more restrictive for 
$C_{S,P,10}$ Wilson coefficients.
% than any of the
%invariant dilepton mass bins of $D^+ \to \pi^+ \mu^+ \mu^-$.

For baryonic $c\to u \ell^+ \ell^-$ transitions the relevant  form factors are known from lattice QCD calculations for the %produced relevant  for the
 $\Lambda_c \to p \ell^+\ell^-$ decay \cite{Meinel:2017ggx}. The dominant contributions to the branching ratio come from resonant regions of $\Lambda_c  \to p \rho, (\omega, \phi)$ with $\rho$, $\omega$ and $\phi$ decaying to $\mu^+ \mu^-$. %\td{not sure about the meaning of this sentence. Is the search in the resonance region?:} 
 This enables to  investigate the impact of NP on the differential branching ratio, the  fraction of longitudinally polarized di-muons and the forward-backward asymmetry. %\td{please check:}
 %The measured value $BR(\Lambda_c \to p \mu^+ \mu^-)_{res}  = (3.7\pm 1.2) \times 10^{-7}$, or rather t
 The upper 90\% CL bound $BR(\Lambda_c \to p  \mu^+ \mu^-)_{exp}  <7.7 \times 10^{-8}$  \cite{Aaij:2017nsd}, obtained by excluding the $\pm40$ MeV intervals around resonances still allows NP contribution in $C_9$ and $ C_{10}$. 
The $\Lambda_c \to p \mu^+ \mu^-$ forward-backward asymmetry appears as a result of  a nonzero $C_{10}$ Wilson coefficient, generated by the NP.  Therefore  this observable  provides a clean null test of the SM as suggested in Ref. \cite{Meinel:2017ggx}.%The case of  
 %NP  contributing to $C_9= - C_{10} \simeq 0.6$ in \cite{Meinel:2017ggx},  still weaker NP bounds than the $D^+ \to \pi^+ \mu^+ \mu^-$ analysis. 
% 
% and current bound obtained from the differential branching ratio for $D^+ \to \pi^+ \mu^+ \mu^-$,  the author of \cite{Meinel:2017ggx}  found that  the current upper bound of the LHCb $BR(\Lambda_c \to p  \mu^+ \mu^-)_{exp}  <7.7 \times 10^{-8}$  (at 90 \% confidence level) in the dimuon mass region,  excluding $\pm40$ MeV intervals around resonances,  \cite{Aaij:2017nsd}   
%still allows the presence of NP with bounds obtained  in $D^+ \to \pi^+ \mu^+ \mu^-$. 

\begin{table}[t]
\caption{%\td{Mass dimensions do not work} 
Maximal experimentally allowed magnitudes of the Wilson coefficients,
  $|\tilde C_i| = |V_{ub}  V_{cb}^* C_i|$, obtained from 
  non-resonant part of $D^+ \to \pi^+ \mu^+ \mu^-$ decay, low $q^2$ region I: $q^2 \in [0.0625,0.276]\,{\rm GeV}^2$; high $q^2$ region: $q^2 \in
  [1.56,4.00] {\rm GeV}^2$); and from $BR (D^0 \to \mu^+ \mu^-) < 7.6 \times 10^{-9}$~\cite{Aaij:2013cza}.
  The last row applies to the case $\tilde C_9= \pm\tilde C_{10}$. All the bounds assume real
  $C_{i}$, and also apply to coefficients with flipped chirality, $\tilde C_j^\prime$.%\td{the definition for these is missing}
  }
\label{tab:BR}
\begin{center}
\begin{tabular}{cr@{.}lr@{.}lr@{.}l} 
\hline\hline\\[-4mm]
\multicolumn{1}{c}{}&\multicolumn{6}{c}{$|\tilde C_i|_\mathrm{max}$}\\
%\hline
 \multicolumn{1}{c}{}  & \multicolumn{2}{c}{$BR(\pi \mu \mu)_{I}$} &  \multicolumn{2}{c}{$BR(\pi \mu \mu)_{II}$} &   \multicolumn{2}{c}{$BR(D^0\to \mu \mu)$} \\ 
\hline\\[-4mm] 
$\tilde C_7$ & \hspace{0.8cm} 2&4 \hspace{0.8cm} & \hspace{0.8cm} 1&6 \hspace{0.8cm} & \multicolumn{2}{c}{-} \\
$\tilde C_9$ & 2&1 & 1&3 & \multicolumn{2}{c}{-}  \\
$\tilde C_{10}$ & 1&4  & 0&92 & \hspace{.7cm}0&63\\
$\tilde C_S$ & 4&5 & 0&38& 0&049\\
$\tilde C_P$ & 3&6 & 0&37 & 0&049\\
$\tilde C_T$ & 4&1 & 0&76& \multicolumn{2}{c}{-}  \\
$\tilde C_{T5}$ & 4&4& 0&74 & \multicolumn{2}{c}{-}  \\
$\tilde C_9= \pm \tilde C_{10}$ & 1&3& 0&81&  0&63\\
 \hline\hline
\end{tabular}
\end{center}
\end{table}
%\begin{table}[t]
%\caption{Maximal allowed values of the Wilson coefficient moduli,
%  $|\tilde C_i| = |V_{ub}  V_{cb}^* C_i|$, calculated in the
%  nonresonant regions of $D^+ \to \pi^+ \mu^+ \mu^-$ in the low lepton
%  invariant mass region ($q^2 \in [0.0625,0.276]\,{\rm GeV}^2$), denoted by ${I}$,
%  in the high invariant mass region ($q^2 \in
%  [1.56,4.00] {\rm GeV}^2$), denoted by ${II}$, and from the upper
%  bound $BR (D^0 \to \mu^+ \mu^-) < 7.6 \times 10^{-9}$~\cite{Aaij:2013cza}.
%  The last row gives the maximal value for the case where $\tilde
%  C_9= \pm\tilde C_{10}$. All the quoted bounds have been derived for real
%  $C_{i}$. The bounds for $\tilde C_i$ apply also to the chiraly
%  flipped coefficients $\tilde C_j^\prime$.}
%\label{tab:BR}
%\begin{center}
%\begin{tabular}{||c||r@{.}l|r@{.}l|r@{.}l||} 
%\cline{2-7}
%\multicolumn{1}{c||}{}&\multicolumn{6}{c||}{$|\tilde C_i|_\mathrm{max}$}\\
%\cline{2-7}
% \multicolumn{1}{c||}{}  & \multicolumn{2}{c|}{$BR(\pi \mu \mu)_{I}$} &  \multicolumn{2}{c|}{$BR(\pi \mu \mu)_{II}$} &   \multicolumn{2}{c||}{$BR(D^0\to \mu \mu)$} \\ 
%\hline \hline
%$\tilde C_7$ & \hspace{0.8cm} 2&4 \hspace{0.8cm} & \hspace{0.8cm} 1&6 \hspace{0.8cm} & \multicolumn{2}{c||}{-} \\
%$\tilde C_9$ & 2&1 & 1&3 & \multicolumn{2}{c||}{-}  \\
%$\tilde C_{10}$ & 1&4  & 0&92 & \hspace{.7cm}0&63\\
%$\tilde C_S$ & 4&5 & 0&38& 0&049\\
%$\tilde C_P$ & 3&6 & 0&37 & 0&049\\
%$\tilde C_T$ & 4&1 & 0&76& \multicolumn{2}{c||}{-}  \\
%$\tilde C_{T5}$ & 4&4& 0&74 & \multicolumn{2}{c||}{-}  \\
%$\tilde C_9= \pm \tilde C_{10}$ & 1&3& 0&81&  0&63\\
% \hline
%\end{tabular}
%\end{center}
%\end{table}

The above bounds allow for appreciable  NP contributions. An example are leptoquark mediators \cite{Dorsner:2016wpm}, which are well motivated as the NP explanations of  
%are possible  of NP mediators \cite{Dorsner:2016wpm} which are rather successful in explaining 
the $B$-meson anomalies, see, e.g., \cite{Buttazzo:2017ixm}.
Refs. \cite{deBoer:2015boa} and \cite{Fajfer:2015mia} showed that leptoquark exchanges do not affect much 
%it was found that the impact of leptoquarks is rather insignificant in the case of 
the branching ratios, but can lead to CP asymmetries in $D\to \pi l^+ l^-$ and $D_s \to K l^+ l^-$  of a few percent \cite{Fajfer:2012nr,deBoer:2015boa}. Namely, such CP asymmetries are 
%give effects of the order few percent if the case of  
%\td{can we be more concrete? give one example or cite somebody for an example?}
defined close to the $\phi$  resonance that couples to the lepton pair and they  can be generated by imaginary parts of the $C_{7}$ Wilson coefficients in the effective Lagrangian  for $c \to  u l^+l^−$ processes. 
In the case of any NP scenario one cannot consider any charm rare decays without considering bounds from the $D^0- \bar D^0 $ oscillation as pointed out in \cite{Golowich:2009ii}. In some particular NP scenarios, bounds on NP are stronger if from  the $D^0- \bar D^0 $ oscillation. 
 In addition, if the NP is realized by the left-handed doublets of the weak isospin, then NP present in $B$ physics also inevitably appears in charm physics, accompanied with the appropriate CKM matrix elements.  



\subsection{Inputs for $B$ physics}
If discovered, the presence of direct CP violation in $D$ decays can
affect the extraction of the CKM angle $\gamma$ from the
``tree-level'' decays $B \to DK$~\cite{Gronau:1990ra, Gronau:1991dp,
  Atwood:1996ci, Giri:2003ty}. If one includes the $B \to D \pi$
modes, the effect can be of order one~\cite{Martone:2012nj}. One of
the advantages of obtaining $\gamma$ from tree decays is that all
hadronic parameters can be fit from data. This remains essentially
true even if direct CPV is present in the decays of the final-state
$D$ mesons~\cite{Martone:2012nj, Wang:2012ie, Bhattacharya:2013vc}, as
long as one includes all direct CP asymmetries in the fit -- there are
still more observables than parameters in the fit. However, one can
show that there remains an ambiguity: A shift in the angle $\gamma$
can be compensated by a corresponding, unobservable shift in the
contributing strong phases. This shift symmetry can be broken by
assuming the absence of CPV in one of the $D$ decay modes (for
instance, in the SM this is the case for CF and DCS decay
modes). Alternatively, one could consider ratios of $B \to f_D K$ and
$B \to f_D \pi$ modes where the strong phase
cancels~\cite{LHCb-CONF-2012-032}, or use information on the relative
strong phases, e.g., by measuring entangled decays at $D$
factories~\cite{Martone:2012nj}.



%\td{SF will also write something}

\subsection{Experimental prospects}% (including interplay with Belle II and BES III)}
\label{sec:expcharmmix}
\subsubsection{Mixing and time-dependent CPV in two-body decays} 
The mixing and CPV parameters in \Dz--\Dzb oscillations can be accessed by comparing the decay-time-dependent ratio of \decay{\Dz}{K^+\pi^-} to \decay{\Dz}{K^-\pi^+} rates with the corresponding ratio for the charge-conjugate processes.

The latest measurement from LHCb~\cite{LHCb-PAPER-2017-046} uses Run~1
and early Run~2 (2015--2016) data, corresponding to a total sample of
about $\mathcal{L}=5\invfb$ of integrated luminosity.  Assuming \CP
conservation, the mixing parameters are measured to be
$x_{K\pi}'^2=(3.9 \pm 2.7) \times10^{-5}$,
$y_{K\pi}'= (5.28 \pm 0.52) \times 10^{-3}$, and
$R_D^{K\pi} = (3.454 \pm 0.031)\times10^{-3}$.  Studying \Dz and \Dzb decays
separately shows no evidence for \CP violation and provides the
current most stringent bounds on the parameters $A_D^{K\pi}$ and $|q/p|$ from
a single measurement, $A_D^{K\pi} =(-0.1\pm9.1)\times10^{-3}$ and
$1.00< |q/p| <1.35$ at the $68.3\%$ confidence level.  

\begin{table}[tb]
\centering
\caption{Extrapolated signal yields, and statistical precision on the mixing and \CP-violation parameters, from the analysis of promptly produced  DCS \decay{\Dstarp}{\Dz(\to K^+\pi^-)\pi^+} decays.
Signal yields of promptly produced CF \decay{\Dstarp}{\Dz(\to K^-\pi^+)\pi^+} decays are typically $250$ times larger.\label{tab:WS-D2Kpi-yields}}
\begin{tabular}{lcccccc} 
\hline\hline\\[-4mm]
Sample ($\mathcal{L}$)  & Yield ($\times10^6$) & $\sigma(x_{K\pi}'^2)$ & $\sigma(y_{K\pi}')$ & $\sigma(A_D)$ & $\sigma(|q/p|)$ & $\sigma(\phi)$ \\
\hline
Run 1--2 (9\invfb)   &  1.8   & $1.5\times10^{-5}$ & $2.9\times10^{-4}$ & 0.51\% & 0.12 & 10\degrees \\
Run 1--3 (23\invfb)  &   10   & $6.4\times10^{-6}$ & $1.2\times10^{-4}$ & 0.22\% & 0.05 &  4\degrees \\
Run 1--4 (50\invfb)  &   25   & $3.9\times10^{-6}$ & $7.6\times10^{-5}$ & 0.14\% & 0.03 &  3\degrees \\
Run 1--5 (300\invfb) &  170   & $1.5\times10^{-6}$ & $2.9\times10^{-5}$ & 0.05\% & 0.01 &  1\degrees \\
\hline\hline
\end{tabular}
\end{table}

In Table~\ref{tab:WS-D2Kpi-yields} the signal yields and the statistical precision from Ref.~\cite{LHCb-PAPER-2017-046} are extrapolated to the end of Run 2
and to the end of \upgradetwo, assuming that the central values of the measurements stay the same. This assumption is particularly important for the \CP-violation parameters, as their precision may depend on the measured values.

Systematic uncertainties are estimated using control samples of data and none of them are foreseen to have irreducible contributions that exceed the ultimate statistical precision, if the detector performance (particularly in terms of vertexing/tracking and particle identification capabilities) is kept at least in line with what is currently achieved at LHCb.

\subsubsection{Mixing and time-dependent CPV in $\Dz\to\KS\pip\pim$} 
The self-conjugate decay \decay{D^0}{\KS \pi^+ \pi^-} includes both CF and  DCS, 
as well as $\CP$-eigenstate processes reconstructed in the same final state.
This allows for the relative strong phase between different contributions to be determined from data, and, in turn, enables both the mixing parameters $x$ and $y$, as well as the \CP-violation parameters $|q/p|$ and $\phi$ to be directly measured without need for external input.
As a result, this channel provides the dominant constraint on the parameter $x$ in the global fits.

%\td{please check the rewrite:} 
The mixing and CPV parameters modulate the time-dependence of the complex amplitudes, and these amplitudes themselves vary over the two-dimensional final state phase-space. 
%which also depend on the two-dimensional phase-space of the decay.
As such, the measurement relies both on the precise understanding of the detector acceptance as a function of phase-space and decay time, and on the accurate description of the evolution of the underlying decay amplitudes over the Dalitz plane. Both model-dependent and model-independent approaches using quantum-correlated $D\bar{D}$ pairs from $\psi(3770)$ decays can be applied.

Previous measurements from the CLEO~\cite{Asner:2005sz}, BaBar~\cite{delAmoSanchez:2010xz}, and Belle~\cite{Peng:2014oda} collaborations have used the model-dependent approach, with the Belle measurement having the best precision to date, $x = (0.56 ^{+0.20}_{-0.23})\%$, $y = (0.30 ^{+0.16}_{-0.17})\%$ (assuming \CP symmetry), and $|q/p| = 0.90 ^{+0.18}_{-0.16}$, $\phi = (-6 \pm 12)\degrees$.
The one published LHCb result was based on 1~fb$^{-1}$ of Run 1 data~\cite{LHCb-PAPER-2015-042}, and used a model-independent approach with strong phases taken from the CLEO measurement~\cite{Libby:2010nu} to determine $x = (-0.86 \pm 0.56)\%$, $y = (0.03 \pm 0.48)\%$.
This analysis used around $2\times10^5$ \decay{D^{*+}}{D^0 \pi^+}, \decay{D^0}{\KS \pi^+ \pi^-} decays from 2011, which suffered from low \KS trigger efficiencies that were significantly increased for 2012 and beyond, and will benefit further from software trigger innovations at the LHCb in the upgrade era.

At LHCb these decays can be reconstructed either through semileptonic decays, for instance \decay{B^-}{D^0 \mu^- \bar{\nu}_{\mu}}, where the muon charge is used to tag the initial $D^0$ flavour, or through prompt charm production, where the charge of the slow pion in the decay \decay{D^{*+}}{D^0 \pi^+} tags the initial flavour.
The two channels have complementary properties and both will be important components of future mixing and \CP violation analyses at LHCb.

The prompt charm yields are significantly larger than for the semileptonic channel, due to the increased production cross-section. However, for the semileptonic channel the triggering on signal candidates is much more efficient, and introduces fewer non-uniformities in the acceptance.
The estimated future yields are presented in Table~\ref{tab:D0KSPiPi_yields}.
Also shown are projected statistical precisions on the four mixing and CPV parameters, which have been extrapolated from complete analyses of the Run 1 data for both the semileptonic and prompt cases. 

\begin{table}[tb]
 \caption{Extrapolated signal yields at LHCb, together with statistical precision on the mixing and \CP violation parameters, for the analysis of the decay \decay{D^0}{\KS \pi^+ \pi^-}.
Candidates tagged by semileptonic $B$ decay (SL) and those from prompt charm meson production are shown separately.}
 \label{tab:D0KSPiPi_yields}
\begin{center}
  \renewcommand{\arraystretch}{1.2}
 \begin{tabular}{lcccccc} 
 \hline\hline
 Sample (lumi $\mathcal{L}$)              & Tag    & Yield & $\sigma(x)$ & $\sigma(y)$ & $\sigma(|q/p|)$ & $\sigma(\phi)$ \\ \hline
\multirow{2}{*}{Run 1--2 (9~fb$^{-1})$}   & SL     & 10M   & 0.07\%      & 0.05\%      & 0.07            & 4.6\degrees    \\
                                          & Prompt & 36M   & 0.05\%      & 0.05\%      & 0.04            & 1.8\degrees    \\ %\hline
\multirow{2}{*}{Run 1--3 (23~fb$^{-1})$}  & SL     & 33M   & 0.036\%     & 0.030\%     & 0.036           & 2.5\degrees    \\
                                          & Prompt & 200M  & 0.020\%     & 0.020\%     & 0.017           & 0.77\degrees   \\ %\hline
\multirow{2}{*}{Run 1--4 (50~fb$^{-1})$} & SL     & 78M   & 0.024\%     & 0.019\%     & 0.024           & 1.7\degrees    \\
                                         & Prompt & 520M  & 0.012\%     & 0.013\%     & 0.011           & 0.48\degrees   \\% \hline
\multirow{2}{*}{Run 1--5 (300~fb$^{-1})$} & SL     & 490M  & 0.009\%     & 0.008\%     & 0.009           & 0.69\degrees   \\
                                          & Prompt & 3500M & 0.005\%     & 0.005\%     & 0.004           & 0.18\degrees   \\
 \hline\hline
\end{tabular}
\renewcommand{\arraystretch}{1.0}
\end{center}
\end{table}

For this channel the dominant systematic uncertainties on mixing parameters  come from two main sources.
First is the precision with which the non-uniformities in detector acceptance  can be determined versus as a function of phase space and decay time. Second is the knowledge of the strong-phase variation across the Dalitz plane.
For the LHCb Run 2 analysis, both contributions are significantly smaller than the statistical precision. In the longer term new approaches will be necessary to further reduce these systematic uncertainties.
Trigger and event selection techniques should be adapted to emphasise uniform acceptance, a task made easier by the removal of the calorimeter-based hardware trigger. New techniques, such as the bin-flip method~\cite{Pilar:2014jfg}, 
can further reduce dependence on the non-uniform acceptance, although at the cost of
degraded statistical precision on the mixing and \CP-violation parameters.
In the model-dependent approach many of the model systematic uncertainties may reduce or vanish with increased integrated luminosity, as currently fixed parameters are incorporated into the data fit, and the data become increasingly capable of rejecting unsuitable models provided that there is suitable evolution in the model descriptions.
For the model-independent approach, the uncertainty from external inputs (currently from CLEO-c, later with 50\% reduction from BESIII) will also reduce with luminosity as the LHCb data starts to provide constraining power.
There are no systematic uncertainties which are known to have irreducible contributions that exceed the ultimate statistical precision.

For the \CP violation parameters additional sources of systematic uncertainty come from the knowledge of detector-induced asymmetries.
In particular, there is a known asymmetry between $K^0$ and $\bar{K}^0$ in their interactions with material.
The limitation here will be the precision with which the material traversed by each $\KS$ meson can be determined.
The LHCb \upgradetwo detector will be constructed to minimise material, and to allow precise evaluation of the remaining contributions. In summary, this channel has comparable power on \CP violating parameters, but with a simpler two-dimensional phase space and complementary detector systematic uncertainties, as the four-body decays that we discuss next. 
% in the previous section.

\subsubsection{Mixing and time-dependent CPV in four-body decays} 
Like \decay{\Dz}{\Km\pip} and \decay{\Dz}{\Kp\pim}, the decays
\decay{\Dz}{\Km\pip\pim\pip} and \decay{\Dz}{\Kp\pim\pim\pip} are a
pair of CF and DCS decays with high sensitivity to charm mixing.
However, the rich amplitude structure across the five dimensional phase
space of the latter decays offers unique opportunities (and
challenges) in these four-body modes.

In the phase-space integrated analysis using $3\mathrm{fb}^{-1}$ of data,
LHCb made the first observation of mixing in this decay mode, and
measured quantities $R_D^{K3\pi} = (3.21 \pm 0.014)\cdot 10^{-3}$,
as well as
$R^{K3\pi}_{\mathrm{coher}}y'_{K3\pi} = (0.3\pm 1.8)\cdot 10^{-3}$, and
$\frac{1}{4}(x^2 + y^2) = (4.8\pm 1.8) \cdot
10^{-5}$~\cite{LHCb-PAPER-2015-057}. The coherence factor,
$R^{K3\pi}_{\mathrm{coher}}$, 
%is the coherence factor, which takes into account the 
measures the effect of integrating
over the entire four-body phase
space~\cite{Atwood:2003mj,Evans:2016tlp}. \td{check refs}
%~\cite{Atwood:coherenceFactor,Evans:2016tlp}.

The unique power of multibody decays lies to a large extent in the
fact that the strong phase difference between the interfering \Dz and
\Dzb amplitudes varies across the phase space. This can be fully
exploited only by moving away from the phase-space-integrated approach to the analyses of phase space distributions, either in
bins or unbinned.
%, rather than a phase-space-integrated approach. 
Such
a ``phase space resolved'' approach allows a direct measurement of
$x_{K\pi\pi\pi}'$ and $y_{K\pi\pi\pi}'$ (rather than only the
$x'^2$ and $y'$ as in the 2-body case), and,
crucially, provides high sensitivity to the \CP\ violating variables $\phi$ and
$|q/p|$.

On the other hand, the same phase variations that make multibody decays so
powerful, are also a major challenge, as they need to be known precisely in
order to cleanly extract the mixing and \CP violation parameters of
interest. In principle, the relevant phases can be inferred from an
amplitude model such as that obtained from $3\mathrm{fb}^{-1}$ of LHCb
data~\cite{LHCb-PAPER-2017-040}. Such models may introduce  theoretical uncertainties that are unacceptably large for
the precision era of \lhcb \upgradetwo,  unless there are significant innovations
in the theoretical description of four-body amplitudes. The alternative is to use 
model-independent approaches.
% will also be applied. 
These use 
quantum-correlated events at the charm threshold to infer
the required phase information in a model-unbiased way. BESIII is
working closely with \lhcb~\cite{Malde:2223391} to provide the necessary model-independent
inputs for \decay{\Dz}{\Kp\pim\pim\pip} across different regions of
phase space for measurements of the $\gamma$ angle as well as charm mixing and
\CP violation measurements.

Sensitivity studies with model-dependent approaches give a
useful indication of the precision that can be achieved. A recent such
study, Ref.~\cite{Muller:2297069}, uses LHCb's latest
\decay{\Dz}{\Kp\pim\pim\pip} amplitude
model~\cite{LHCb-PAPER-2017-040}.  Table~\ref{tab:WS-D2K3pi} gives the
yields and sensitivities scaled from the study in~\cite{Muller:2297069},  illustrating the
impressive sensitivity of this decay mode.
The study is based on
promptly produced \Dstarp mesons, decaying in the flavour-conserving
 \decay{\Dstarp}{\Dz\pip} channel.  Several systematic uncertainties
require improvements in the analysis method in order to scale with
increasing sample sizes. However, given the huge potential of this
channel, sufficient effort is expected to be dedicated to this challenge, such
that adequate methods can be developed, and that the necessary input from
threshold measurements is both generated at BESIII and exploited
optimally at \lhcb. Indeed, once these are in place, this channel has
the potential for probing \CP violation at the $\order(10^{-5})$ level, given the
current world average value of $x$. 
\begin{table}[tb]
\caption{Extrapolated signal yields for LHCb, and sensitivity to the mixing and
  \CP-violation parameters, from the analysis of
  \decay{\Dz}{\Kp\pim\pim\pip} decays (statistical uncertainties only).\label{tab:WS-D2K3pi}}
\centering
\begin{tabular}{lccccc} 
\hline\hline\\[-4mm]
Sample ($\mathcal{L}$)  & Yield ($\times10^6$) & $\sigma(x_{K\pi\pi\pi}')$ & $\sigma(y_{K\pi\pi\pi}')$ & $\sigma(|q/p|)$ & $\sigma(\phi)$ \\
\hline
Run 1-2 (9\invfb)       &  0.22                & $2.3\times10^{-4}$       & $2.3\times10^{-4}$       & 0.020           & 1.2\degrees \\
Run 1-3 (23\invfb)      &  1.29                & $0.9\times10^{-4}$
                                                                           & $0.9\times10^{-4}$       & 0.008           & 0.5\degrees \\
Run 1-4 (50\invfb)      &  3.36                & $0.6\times10^{-4}$       & $0.6\times10^{-4}$       & 0.005           & 0.3\degrees \\
Run 1-5 (300\invfb)     & 22.5\phantom{00}     & $0.2\times10^{-4}$       & $0.2\times10^{-4}$       & 0.002           & 0.1\degrees \\
\hline\hline
\end{tabular}
\end{table}

\subsubsection{Measurement of $A_\Gamma$}
\label{sec:AGamma}
The large yields available in the SCS modes, $f=\pi^+\pi^-$ or $f=K^+K^-$, together with tagging from the \Dstarpm decay, allow for a precise measurement of $A_\Gamma$, provided the systematic uncertainties can be controlled with high degree of precision.
Tagging based on semileptonic decays of a parent bottom hadron is also possible and has been used in a published LHCb measurement~\cite{LHCb-PAPER-2014-069}, but contributes significantly lower yields. 

Most potential systematic effects are essentially constant in $t$ and
therefore cause little uncertainties in the observed decay time
evolution of the asymmetry.  However, second-order effects and
detector--induced correlation between momentum and proper decay time
are sufficient to produce spurious asymmetries, that must be
appropriately corrected. In addition, contamination from secondary
decays is a first-order effect in time that must be suppressed, and
its residual bias accounted for.  Both corrections are dependent on
the availability of a large number of CF $\Dz\to\Km\pip$  decays
as calibration, and can be expected to scale with statistics;
collection of this sample with the same trigger as for the signal
modes is therefore a crucial tool for performing this measurement with high precision in the future.

The Run~1 LHCb measurement of this quantity gave consistent results in the two $h^+h^-$ modes, averaging  $\agamma= (- 0.13 \pm 0.28 \pm 0.10)\times 10^{-3}$ \cite{LHCb-PAPER-2016-063}, which is still statistically dominated. For the reasons mentioned above, this precision is at the threshold of becoming physically interesting, making it a worthy target to pursue with more data. It seems highly unlikely that any experiment built in the foreseeable future will be able to do this, except for an upgrade of LHCb to higher luminosity. 

Table~\ref{tab:Agamma_yields} shows expected yields and precisions attainable in  \lhcb \upgradetwo, under the same assumptions on efficiencies adopted in the previous sections; this must include provisions for acquiring and storing $5\times 10^{10}$ CF decays. The ultimate combined precision is $1\times 10^{-5}$.

\begin{table}[t]
\caption{Extrapolated signal yields at LHCb, and statistical precision on
  indirect \CP violation from \agamma. 
}
 \label{tab:Agamma_yields}
\begin{center}
  \renewcommand{\arraystretch}{1.2}
 \begin{tabular}{lccccc} 
 \hline\hline
 Sample ($\mathcal{L}$)   & Tag   & Yield $K^+K^-$ & $\sigma(\agamma)_{K^+K^-}$  & Yield  $\pi^+\pi^-$  & $\sigma(\agamma)_{\pi^+\pi^-}$ \\[1mm] \hline
Run 1--2 (9~fb$^{-1})$    & Prompt   & 60M    & 0.013\% & 18M & 0.024\%       \\ %\hline
Run 1--3 (23~fb$^{-1})$  & Prompt   & 310M   & 0.0056\% &92M& 0.0104 \%       \\ %\hline
Run 1--4 (50~fb$^{-1})$  & Prompt   & 793M   & 0.0035\% &236M& 0.0065 \%       \\ %\hline
Run 1--5 (300~fb$^{-1})$ & Prompt  & 5.3G   & 0.0014\% & 1.6G & 0.0025 \%         \\
 \hline\hline
\end{tabular}
\renewcommand{\arraystretch}{1.0}
\end{center}
\end{table}

\subsubsection{Combined mixing and time dependent CPV sensitivity}
%A comparison of t
The projected precisions of the analyses presented in
the previous sections are shown in
Fig.~\ref{fig:charm_indirect_channels}, and are compared with the expected
precisions at Belle~II. The expected LHCb constraints on $\phi$ are
translated into asymmetry constraints ($A_{\CP}^{ind.} \approx x \sin(\phi)$) by multiplying by the current
HFLAV average of $x$ and neglecting the uncertainty on this under the
assumption that $x$ will be comparatively well determined in the
future. This comparison neglect additional constraining power from $|q/p|$.
The relative values of these asymmetry constraints with those from
\agamma is indicative only.

\begin{figure}
  \centering
  \includegraphics[width=1.0\textwidth]{section3/figures/CharmIndirectCPVSummary.pdf}
  % 
  \caption{The predicted constraints on the indirect \CP violation
    asymmetry in
    charm from the decay channels indicated in the labels at the
    bottom of the columns. Predictions are shown in LS2
    (2020) from LHCb, LS3 (2025) from LHCb, at the end of Belle~II (2025), and at the end
    of the HL-LHC LHCb \upgradetwo\ programme. } 
\label{fig:charm_indirect_channels}
\end{figure}

The analyses presented in the previous sections are also combined to
establish the sensitivity to the \CP-violating parameters $|q/p|$ and
$\phi$. The combination is performed using the method described in Ref.~\cite{LHCb-PAPER-2016-032}.
At an integrated luminosity of $300\invfb$ the sensitivity to $|q/p|$ is expected to be $0.001$ and that to $\phi$ to be $0.1^\circ$.
This remarkable sensitivity is contrasted in Fig.~\ref{fig:charm_indirect} with the HFLAV world average
as of 2017. We can conclude that the
LHCb \upgradetwo will have impressive power to characterise NP
contributions to \CP violation and is the only foreseen facility with strong potential of
probing the SM contribution.

\begin{figure}
  \centering
  \includegraphics[width=0.6\textwidth]{section3/figures/charm_indirect_CPV}
  % 
  \caption{The estimated constraints for LHCb \upgradetwo\ on $\phi$, $|q/p|$ from the combination of the analyses (red), see main text for details, compared to the current world-average precision (light blue).} 
\label{fig:charm_indirect}
\end{figure}

\subsubsection{Direct CP violation}\label{sec:directCP:exp}

The SCS decays $D^0 \rightarrow \Km\Kp$ and $D^0 \rightarrow \pim\pip$ 
play a critical role in the measurement of time-integrated direct \CP violation
through time-integrated \CP asymmetry in the $h^-h^+$ decay rates, Eq. \eqref{ACPhh}.
%\begin{equation}\label{ACPhh}
%A_{\CP}(\Dz\to h^-h^+)\equiv \frac{\Gamma(D^0\rightarrow h^-h^+)-\Gamma(\Dzb\rightarrow h^-h^+)}{\Gamma(D^0\rightarrow h^-h^+)+\Gamma(\Dzb\rightarrow h^-h^+)}.
%\end{equation}
%
The sensitivity to direct \CP violation is enhanced through a measurement of the difference in \CP asymmetries between \Dz\to\Km\Kp and \Dz\to\pim\pip decays, $\Delta A_{\CP}=A_{\CP}(\Km\Kp)-A_{\CP}(\pim\pip)$.
The individual asymmetries $A_{\CP}(\Km\Kp)$ and $A_{\CP}(\pim\pip)$ can also be measured. 

A measurement of the time-integrated \CP asymmetry in $D^0 \rightarrow \Km\Kp$ has been performed in \lhcb\ with 3\invfb 
collected at centre-of-mass energies of 7 and 8\tev. The flavour of the charm meson at production is determined from the charge of the pion in
$\Dstarp\to\Dz\pi^+$ decays, or via the charge of the muon in semileptonic \bquark-hadron decays, $\Bb\to\Dz\mun\neumb X$. 
The analysis strategy so far relies on the $D^+\rightarrow K^-\pi^+\pi^- $, $D^+\rightarrow K_s^0\pi^+$ and $\Dstarp\to\Dz (\to K^-\pi^+)\pi^+$ decays as control samples~\cite{LHCb-PAPER-2016-035}. In this case, due to the weighting procedures aiming to fully cancel the production and reconstruction asymmetries, the effective prompt signal yield for $A_{\CP}(\Km\Kp)$ is reduced. The expected signal yields and the corresponding statistical precision in LHCb \upgradetwo are summarised in Table~\ref{tab:expectedyields}.

\begin{table}[t]
\caption{\small Extrapolated signal yields at LHCb and statistical precision on direct \CP violation observables for the promptly produced samples.}
\centering
\begin{tabular}{l c  c  c  c  c }
\hline\hline\\[-4mm]
Sample ($\mathcal{L}$) & Tag & Yield & Yield  & $\sigma(\Delta A_{\CP})$ &  $\sigma(A_{\CP}(hh))$ \\
 &  & \Dz\to\Km\Kp & \Dz\to\pim\pip & [$\%$] &  [\%] \\
\hline
Run 1-2 (9 \invfb)  & Prompt      &  52M &   17M & $0.03$  & $0.07$ \\   
Run 1-3 (23 \invfb) & Prompt      & 280M &   94M & $0.013$  & $0.03$ \\   
Run 1-4 (50 \invfb) & Prompt     &  1G  &  305M & $0.01$ & $0.03$ \\ 
Run 1-5  (300 \invfb) & Prompt    & 4.9G &  1.6G & $0.003$ & $0.007$ \\ 
\hline\hline
\end{tabular}
\label{tab:expectedyields}
\end{table}
 
The observable $\Delta A_{\CP}$  is robust against systematic uncertainties. The main sources of systematic uncertainties are inaccuracies in the fit model, the weighting procedure, the contamination of the prompt sample with secondary $\Dz$ mesons and the presence of peaking backgrounds.
There are no systematic uncertainties with expected irreducible contributions above the ultimate statistical precision.  This channel is already entering the upper range of the physically interesting sensitivities, and will likely continue to provide the world's best sensitivity to direct \CP violation in charm at LHCb \upgradetwo. The power of these two-body \CP eigenstates at LHCb \upgradetwo is illustrated in Fig.~\ref{fig:agammadeltaacp}, which shows the indirect (see Sect.~\ref{sec:AGamma}) and direct \CP constraints that will come from these modes.

\begin{figure}
  \centering
  \includegraphics[width=0.7\textwidth]{section3/figures/direct_indirect_cpv_LHCb_300fb}
  % 
  \caption{The estimated constraints for LHCb \upgradetwo\ on indirect and direct charm \CP violation from the analysis of two-body \CP eigenstates. The current world-average precision~\cite{HFLAV16} is $\pm2.6 \times 10^{-4}$ for indirect and $\pm 18 \times 10^{-4}$ for direct \CP violation and thus larger than the full scale of this plot.} 
\label{fig:agammadeltaacp}
\end{figure}

There are a number of other two-body modes of strong physics interest for which \upgradetwo will also make important contributions. These include the decay modes $\Dz\to\KS\KS$ (0.28\%), 
$\Dz\to\KS\overline{K^{0*}}$ (0.21\%), $\Dz\to\KS{K^{0*}}$ (0.15\%), $\Dsp\to\KS\pip$ $(3.2\times 10^{-4})$, $\Dp\to\KS\Kp$ $(1.2\times 10^{-4})$, $\Dp\to \phi \pip$ $(6\times 10^{-5})$, $\Dp \to \eta'\pip$ $(3.2\times 10^{-5})$, $\Dsp \to \eta'\pip$ $(3.2\times 10^{-4})$, where the projected statistical only \CP asymmetry sensitivities are given in brackets after the decay mode. The first three modes mentioned are notable as they receive sizeable contributions from exchange amplitudes at tree-level and could have a relatively enhanced contribution from penguin annihilation diagrams which are sensitive to NP. Consequently, 
%some authors have highlighted them as 
they could be potential \CP violation discovery channels~\cite{Nierste:2015zra},\cite{Nierste:2017cua}.

Searches for direct \CP violation in the phase space of SCS $D^+\to h_1h_2h_3$ 
decays, hereafter referred to as $D\to 3h$, are complementary to that of $D^{(0,+)}\to h_1h_2$ 
($h_i = \pi, K$). In charged $D$ systems, only \CP violation in the decay is possible. The main observable 
is the \CP asymmetry, which, in the case of two-body decays, is a single number. In contrast,  $D\to 3h$ 
decays allow to study the variation of the \CP asymmetry across the two-dimensional phase space 
(usually represented by the Dalitz plot). 

The estimated  signal yields in future LHCb upgrades  are summarised in Table~\ref{tab:Dhhh-yields}. The yields are based on
an extrapolation of the Run 2 yields per unit luminosity, made under the same assumptions as in Section 6.1. The
estimated sensitivities to observation of \CP violation, using $D^+ \to \pi^-\pi^+\pi^+$ as an example, are presented 
in Table~\ref{tab:Dhhh-sensitivity}.

\begin{table}[tb]
\centering
\caption{Extrapolated signal yields at LHCb, in units of 10$^6$, for the SCS decays $D^+ \to K^-K^+\pi^+$, $D^+ \to \pi^-\pi^+\pi^+$,
and for the DCS decays $D^+ \to K^-K^+K^+$, $D^+ \to \pi^-K^+\pi^+$. \label{tab:Dhhh-yields}}
\begin{tabular}{lcccc}
\hline \hline
Sample ($\mathcal{L}$)  & $D^+ \to K^-K^+\pi^+$  & $D^+ \to \pi^-\pi^+\pi^+$ &  $D^+ \to K^-K^+K^+$ & $D^+ \to \pi^-K^+\pi^+$ \\
\hline
Run 1--2 (9\invfb)     &   200       & 100     & 14       & 8   \\
Run 1--4 (23\invfb)   &   1,000    & 500  & 70     & 40   \\
Run 1--4 (50\invfb)   &   2,600    & 1,300  & 182     & 104   \\
Run 1--6 (300\invfb) &  17,420   &  8,710 & 1,219  &  697  \\
\hline \hline
\end{tabular}
\end{table}

\begin{table}[bt]
\centering
\caption{Sensitivities to several illustrative \CP-violation scenarios in $D^+ \to \pi^-\pi^+\pi^+$ decay. Simulated $D^+$ and $D^-$
Dalitz plots are generated with relative changes in the phase of the $R\pi^{\pm}$ amplitude, $R=\rho^0(770)$, 
$f_0(500)$ or $f_2(1270)$. The values of the phase difference are given in degrees and correspond to 
a 5$\sigma$  \CP-violation effect. Simulations are performed with $3\invfb$ and extrapolated to the 
expected luminosities.}
\begin{tabular}{lc c c c}
\hline\hline
  resonant channel           & $9\invfb$   & $23\invfb$     & $50\invfb$     & $300\invfb$\\
\hline
$f_0(500)\pi$                    & 0.30 &  0.13   &  0.083   &  0.032 \\
$\rho^0(770\pi$                & 0.50 &  0.22   &  0.14     &  0.054\\
$f_2(1270)\pi$                  & 1.0  &  0.45   &   0.28    & 0.11\\
\hline\hline
\end{tabular}
\label{tab:Dhhh-sensitivity}
\end{table}

The SM generated \CP violation could be observed in the SCS decays, such as \decay{\Dz}{\pip\pim\pip\pim} and \decay{\Dz}{\Kp\Km\pip\pim}, while NP is needed to justify any observation of \CP violation in the DCS decays, such as \decay{\Dz}{\Kp\pim\pip\pim}.
Many techniques can be adopted to search for \CP violation, all of them exploiting the rich resonant structure of the decays.
The methods used so far  at \lhcb are based on ${\widehat{T}}$-odd asymmetries and the energy test, while studies are ongoing to measure model-dependent \CP asymmetries in the decay amplitudes.

The study of ${\widehat{T}}$-odd asymmetries exploits potential $P$-odd \CP violation from the interference between different amplitude structures in the decay, as described in Ref.~\cite{Durieux:2015zwa}.  This uses a triple product $C_T=\vec{p}_A\cdot (\vec{p}_B
\times \vec{p}_C)$ constructed from the momenta of three of the final state particles $\vec{p}_A,\vec{p}_B,\vec{p}_C $.
\lhcb has studied ${\widehat{T}}$-odd asymmetries using 3\invfb data from the Run~1 dataset, and obtained a sensitivity of $2.9\times10^{-3}$ with very small systematic uncertainties~\cite{LHCB-PAPER-2014-046}.
The peculiarity of this measurement is the absence of instrumental asymmetries, since it is given by the difference of two asymmetries measured separately on \Dz and \Dzb decays, $a_{\CP}=(A_T-\bar{A}_T)/2$, where
\begin{align}
  A_T = \frac{(\Gamma(\Dz,C_T>0)-\Gamma(\Dz,C_T<0))}{(\Gamma(\Dz,C_T>0)+\Gamma(\Dz,C_T<0))},\quad \bar{A}_T = \frac{(\Gamma(\Dzb,\bar{C}_T>0)-\Gamma(\Dzb,\bar{C}_T<0))}{(\Gamma(\Dzb,\bar{C}_T>0)+\Gamma(\Dzb,\bar{C}_T<0))}.
\end{align}
One can therefore expect the errors to scale with luminosity to reach a sensitivity down to $2.9\times10^{-5}$ ($9.4\times10^{-5}$) for $\Dz\to\pip\pim\pip\pim$ ($\Dz\to\Kp\Km\pip\pim$) decays, as detailed in Table~\ref{tab:CS-D24h-yields}.
\begin{table}[tb]
\centering
\caption{ Extrapolated signal yields, and statistical precision on ${\widehat{T}}$-odd \CP-violation observables at LHCb.\label{tab:CS-D24h-yields}}
\begin{tabular}{lcccc}
\hline\hline
  & \multicolumn{2}{c}{$\Dz\to\pip\pim\pip\pim$} & \multicolumn{2}{c}{$\Dz\to\Kp\Km\pip\pim$} \\
Sample ($\mathcal{L}$)  & Yield ($\times10^6$) & $\sigma(a_{\CP}^{\widehat{T}\text{-odd}})$  & Yield ($\times10^6$) & $\sigma(a_{\CP}^{\widehat{T}\text{-odd}})$ \\
\hline
Run 1--2 (9\invfb)   & 13.5   & $2.4\times10^{-4}$ &  4.7   & $5.4\times10^{-4}$ \\
Run 1--3 (23\invfb)  &   69   & $1.1\times10^{-4}$ &   12   & $3.4\times10^{-4}$ \\
Run 1--5 (300\invfb) &  900   & $2.9\times10^{-5}$ &  156   & $9.4\times10^{-5}$ \\
\hline\hline
\end{tabular}
\end{table}

The energy test method is insensitive to global asymmetries. However, it is expected that it will become sensitive to variations in phase space of production and detection asymmetries.
These can be controlled in data by application of the method to CF decays, such as \decay{\Dz}{\Km\pip\pip\pim}.
Assuming scaling with the square-root of the ratio of sample sizes, the same p-values can be expected for the \CP asymmetries given in Table~\ref{tab:charm_623_energy_test}.

\begin{table}[bt]
\centering
\caption{Overview of sensitivities to various \CP-violation scenarios for \decay{\Dz}{\pip\pim\pip\pim} decays as extrapolated from Ref.~\cite{LHCb-PAPER-2016-044}. The relative changes in magnitude and phase of the amplitude of the resonance $R$ to which sensitivity is expected are given in $\%$ and $^\circ$, respectively. The $P$-wave $\rho^0(770)$ is a $P$-odd component. The phase change in this resonance is tested with the $P$-odd \CP-violation test. Results for all the other scenarios are given with the standard $P$-even test.}
\begin{tabular}{lc c c}
\hline\hline
$R$ (partial wave)        & $9\invfb$   & $23\invfb$ & $300\invfb$\\
\hline
\decay{a_1}{\rhoz\pi} (S) & $1.4\%$     & $0.6\%$      & $0.17\%$\\
\decay{a_1}{\rhoz\pi} (S) & $0.8^\circ$ & $0.35^\circ$ & $0.10^\circ$\\
\rhoz\rhoz (D)            & $1.4\%$     & $0.6\%$      & $0.17\%$\\
\hline
\rhoz\rhoz (P)            & $0.8^\circ$ & $0.35^\circ$ & $0.10^\circ$\\
\hline\hline
\end{tabular}
\label{tab:charm_623_energy_test}
\end{table}

Charm decays with neutrals in the final state can help to shed light on the SM or beyond-SM origin of possible \CP-violation signals by testing correlations between \CP asymmetries measured in various flavour-\grpsuthree or isosping related decays, see Section \ref{sec:isospin:null:test} and Refs. \cite{Bhattacharya:2012ah,Pirtskhalava:2011va,Feldmann:2012js}. These modes are, however, particularly challenging in hadronic collisions, where the calorimeter background for low energy clusters is high, while the trigger retention rate needs to be kept low to allow for affordable rates. 

Nevertheless, good performances are achieved when considering decays with at least two charged particles in the final states, such as \decay{\Dz}{\pi^+\pi^-\pi^0}, since the charge particles help to identify the displaced decay vertex of the charm meson. In only 2\invfb of data collected during 2012, LHCb has reconstructed about 660,000 \decay{\Dz}{\pi^+\pi^-\pi^0} decays~\cite{LHCb-PAPER-2014-054}, \ie about five times more than Babar from its full data set~\cite{TheBABAR:2016gom}, with comparable purity. Preliminary estimates for Run 2 data, give about 500\,000 signal decays per \invfb, making future \CP-violation searches in this channel very promising. Similarly, large samples of \decay{\D_{(\squark)}^+}{\eta^{(\prime)}\pi^+} decays, with \decay{\eta^{(\prime)}}{\pi^+\pi^-\gamma}, or \decay{\Dp}{\pi^+\pi^0} decays, with \decay{\pi^0}{e^+e^-\gamma}, are already possible with the current detector. The \decay{\D_{(\squark)}^+}{\eta^{\prime}\pi^{+}} mode, as an example, has been used by LHCb during Run 1 to perform the most precise measurement of \CP asymmetries in these channels to date, with uncertainties below the 1\% level~\cite{LHCb-PAPER-2016-041}.

More challenging final states consisting only of neutral particles, such as $\pi^0 \pi^0$ or $\eta \eta$, can still be reconstructed with \decay{\pi^0}{\gamma\gamma} or \decay{\eta}{\gamma\gamma} candidates made of photons which, after interacting with the detector material, have converted into an $e^+e^-$ pair. Such conversions must occur before the tracking system to have electron tracks reconstructed. Although the reconstruction efficiency of these ``early'' converted photons in the current detector reaches only a few percent of the calorimetric photon efficiency, their purity is much higher. This approach may become interesting only with the large data sets that are expected to be collected by the end of \upgradetwo. The $\eta$ decays can also be reconstructed through the $\pi^+\pi^-\gamma$ final state.

Unlike the \Dz decays which are usually tagged with a soft pion from \Dstarp decays, there is no easy tagging of the \Dp modes, which thus often suffer from a high combinatorial background. Employing a $\pi^0$ tag using \decay{\Dstarp}{\Dp\pi^0} decays could facilitate future studies of \Dp decays, in particular those with challenging and/or high multiplicity final states.

The study of these modes may be challenging in Run~3 due to the cluster pile-up at higher luminosities and radiation damage of the current calorimeter. The new calorimeter proposed for LHCb \upgradetwo would have an improved granularity. It would therefore improve the efficiency for \decay{\pi^0}{\gamma\gamma} decays and, in particular, could make the $\pi^0$ tag feasible.

\subsubsection{Rare leptonic and radiative charm decays}
The most experimentally accessible very rare charm decay is \dmumu. The world's best limit on \brdmumu was obtained by LHCb with $0.9\invfb$ of 2011 data~\cite{LHCb-PAPER-2013-013}, resulting in
\begin{equation}
\brdmumu< 6.2 \times 10^{-9}\text{ at 90\% CL. }
\end{equation}
Extrapolating the current detector performance, the expected limit is about $5.9 \times 10^{-10}$ with $23~\invfb$ and $1.8 \times 10^{-10}$ with $300~\invfb$ of integrated luminosity,
covering a large part of the unambiguous space to search for NP without being affected by long distance uncertainties in the SM predictions.

The next class of rare charm decays which are particularly suited for experiments at hadron colliders 
are three-body decays with a pair of leptons in the final state, such as $D^{+}_{(s)}/ \Lc \to h^+ \ellell$ and $D^0 \to h^+h^- \ellell$. In some NP scenarios the short distance contributions can be enhanced by several orders of magnitude allowing NP to manifest as an enhancement of the branching fraction. An example of such a model is shown in Fig.~\ref{fig:LC2PMUMU}, where the Wilson coefficients were assumed to obtained NP contributions, $C_9^{\rm NP} =-0.6$ and $C_{10}^{\rm NP} =0.6$ (cf. Eqs. \eqref{eq:Leff:e1}, \eqref{eq:operators}).
Outside of the resonance regions the short distance contributions are 
%enhanced to  similar levels as 
comparable to the long distance effects. The chosen NP benchmark point leads to branching ratios just below the current LHCb limit~\cite{LHCb-PAPER-2017-039} \td{this branching ratio cuts out resonances, right? Should we denote this explicitly?}
\begin{equation}
\BF( \Lc \to p \mumu) < 5.9 \times 10^{-8}~{\rm at}~90\%~{\rm CL}.
\end{equation}
In the \upgradetwo LHCb is expected to improve the limit to
\begin{equation}
\BF( \Lc \to p \mumu) < 4.4 \times 10^{-9}~{\rm at}~90\%~{\rm CL}.
\end{equation}

\begin{figure}[tb]
\includegraphics[width=0.5\textwidth]{{section3/figures/Lambdac_proton_mu_mu_dB_new}.pdf}
\includegraphics[width=0.5\textwidth]{{section3/figures/Lambdac_proton_mu_mu_dB_C9NP_m0p6_C10_0p6_new}.pdf}
\caption{Differential branching fraction for the decay $\Lc \to p \mumu$ as a function of $q^2$ (left) in the SM and (right) in a NP scenario with ${\cal C}_9^{\rm NP}=-0.6$ and ${\cal C}_{10}^{\rm NP}=-0.6$~\cite{Meinel:2017ggx}. The current LHCb limit is marked with green. The extrapolated upper limit with the $23$ and $300~\invfb$ data sets is marked in red.\label{fig:LC2PMUMU}}
\end{figure}

An order of magnitude improvement is expected for the limit on $\BF(D^{+}_{(s)}\to\pi^+\mu^+\mu^-)$ decays, which is currently at $7.3\times10^{-8}$
at 90\% CL \cite{LHCb-PAPER-2012-051}. The expected upper limits are about $1.3\times10^{-8}$ with $23 \invfb$ and $0.37\times10^{-8}$  with $300 \invfb$.
In addition, LHCb will have the ability to measure angular observables such as the forward-background asymmetry, $A_{\rm FB}$, or time integrated $A_{\CP}$, 
which will provide additional handles to separate the long distance from the short distance contributions, and for which some theoretical predictions in NP scenarios already exist \cite{deBoer:2015boa},
as shown in Fig.~\ref{fig:D2PIMUMU}.

\begin{figure}[tb]
\includegraphics[width=0.33\textwidth]{section3/figures/dBQDppiR2mu.pdf}
\includegraphics[width=0.66\textwidth,clip=true,trim=0mm 60mm 0mm 0mm]{section3/figures/ACPD2pimumu.pdf}
\caption{
    (Left) Differential branching fraction of $\D^+ \to \pi^+ \mumu$ as a function of $q^2$ in the SM. The current LHCb limit is marked with a dashed black line. (Middle and right) Predictions for $A_{\CP}$ in a particular NP scenario of Ref.~\cite{deBoer:2015boa} with different strong phases $\delta_{\rho,\phi}$.\label{fig:D2PIMUMU}
}
\end{figure}

Last but not least with a sample corresponding to an integrated luminosity of $300~\invfb$ LHCb will be able to perform searches for the LFV decays $D^{+}_{(s)}/ \Lc\to h^+ \ell^+ \ell^{\prime -}$
and perform tests of lepton universality via the ratios ${\cal B} (\decay{D^{+}_{(s)}/\Lc}{h^+ \mu^+ \mu^{-}})/{\cal B}(\decay{D^{+}_{(s)}/\Lc}{h^+ e^+ e^{-}})$. 

The decays $D^0\to h^+h^-\ellell$ have richer dynamics than the 
two- and three-body decays, allowing for a variety of differential distributions to be investigated.
Due to the huge charm production cross-section at the LHC, and LHCb's ability to trigger on low \pt\ dimuons, 
LHCb has unique physics reach in studying these decays. 
In fact, significant progress has already been made with the observation of the CF decay $\decay{\Dz}{\Km\pip\mun\mup}$ (with the dimuon mass in the $\rho/\omega$ region) with a branching fraction $(4.17\pm0.42)\times10^{-6}$~\cite{LHCb-PAPER-2015-043}  and the SCS decays $\decay{\Dz}{\pip\pim\mup\mun}$ and $\decay{\Dz}{\Kp\Km\mup\mun}$ with branching fractions of $(9.64\pm1.20)\times10^{-7}$ and $(1.54\pm0.33)\times10^{-7}$, respectively~\cite{LHCb-PAPER-2017-019}. For the latter, also the differential branching fraction as a function of the dimuon mass squared, $q^2$, was measured.
Furthermore, LHCb has performed the first measurement of \CP- and angular asymmetries in $\decay{\Dz}{\pip\pim\mumu}$ and $\decay{\Dz}{\Kp\Km\mumu}$ decays using Run~2 data. 
This resulted in the first determination of the forward-backward asymmetry $A_{\rm FB}$, the triple-product asymmetry $A_{2\phi}$ and the \CP-asymmetry $A_{\rm CP}$ with uncertainties at the percent-level~\cite{LHCb-PAPER-2018-020}. 
Moreover, the implementation of triggers for dielectron modes 
opens the possibility of measuring branching-fraction ratios between di-muon and di-electron modes.
Since the main limit for these studies comes from the available statistics, LHCb will already have excellent prospects in Run~3 and 4. 
Projected signal yields for the muonic modes of ${\mathcal O}(10^4)$ will allow more sensitive studies of angular asymmetry and the first amplitude analyses to attempt to disentangle short distance and long distance components.
However,  the full potential for these decays will be exploited only with the $300$\invfb LHCb upgrade.

\subsubsection{Prospects with charm baryons}

In general, charmed baryon decays offer a rich laboratory within which to study 
matter-antimatter asymmetries.
The most easily accessible modes for \lhcb are multibody decays containing one 
proton and several kaons or pions, such as the CF $\decay{\Lc}{\proton\Km\pip}$ decay and the SCS 
$\decay{\Lc}{\proton\Km\Kp}$ and $\proton\pim\pip$ decays.
Unlike the analogous final states of the spinless neutral and charged \PD mesons, such 
charmed baryon final states have at least five degrees of freedom, allowing for 
a complex variation of the strength of \CP violation across the phase space.
A complete description of such a space is experimentally challenging.
This is compounded by the relatively short lifetime of the \Lc baryon with 
respect to those of the \PD mesons, which decreases the power of selections 
based on the displacement of the charm vertex, increasing the prompt 
combinatorial background, which mainly comprises low-momentum pions and kaons.
Selections of charm baryons then heavily rely on good discrimination between 
proton, kaon, and pion hypotheses in the reconstruction of charged tracks, and 
on a precise secondary vertex reconstruction.
The latter is also necessary when reconstructing charm baryons originating from 
semileptonic \bquark-hadron decays, which has the advantage of both a simple 
trigger path (a high-\pt, displaced muon) and a cleaner experimental signature 
due to the large \bquark-hadron lifetime.
The presence of the proton in the final state, whilst a useful handle for 
selections, poses experimental challenges for \CP-violation measurements in 
addition to those present for measurements with \PD mesons, as the 
proton-antiproton detection asymmetry must be accounted for.
This has not yet been measured at \lhcb due to the lack of a suitable control 
mode.

The most precise measurement of \CP violation in charm baryons was made 
recently by the \lhcb collaboration using Run 1 data, corresponding to 
$3$\invfb of integrated luminosity~\cite{LHCb-PAPER-2017-044}.
The difference between the phase-space-averaged \CP asymmetries in 
$\decay{\Lc}{\proton\Km\Kp}$ and $\decay{\Lc}{\proton\pim\pip}$ decays was 
found to be consistent with \CP symmetry to a statistical precision of 
$0.9\,\%$.
This difference is largely insensitive to the proton detection asymmetry, but 
masks like-sign \CP asymmetries between the two modes.
Further studies must then gain a precise understanding of the proton detection 
asymmetry, in addition to measuring the variation of \CP violation across the 
decay phase space.

Although there is little literature on the subject, the magnitude of direct
\CP violation in charm baryon decays is expected to be similar to that for 
charmed mesons~\cite{Bigi:2012ev}, and so the first step in furthering our 
understanding is to reach a precision of $0.5\times10^{-4}$. This can be met by LHCb
given the $300$\invfb of integrated luminosity collected by the end of Run 5.
The acquisition of more data is vital in enabling studies of states heavier 
than the \Lc baryon, as their alternate compositions may permit considerably different dynamics. In this respect it is interesting to note that the \Xiccpp baryon was discovered~\cite{LHCb-PAPER-2018-026} by LHCb using the $\Lc\Km\pip\pip$ final state with 
data taken in 2016, corresponding to an integrated luminosity of $1.7$\invfb.
A signal yield of $313 \pm 33$ was determined, which can be 
extrapolated to around $100,000$ such decays obtainable with data corresponding to
$300$\invfb, allowing asymmetry measurements with a precision of $0.4$\,\%. The flexible trigger and real-time analysis concept of LHCb's \upgradetwo will however enable not only this measurement but a detailed mapping of the entire landscape of multiply-heavy charmed hadrons.

%In conclusion, the high statistics of LHCb \upgradetwo offers the possibility
%to open up the detailed study of this new research area.


%\subsection{Combined th/exp perspective on charm mixing/CPV global fits and charm as %input to $B$ physics}
%\td{missing}

\end{document}
