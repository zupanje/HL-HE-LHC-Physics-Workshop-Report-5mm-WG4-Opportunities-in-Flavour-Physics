\subsection{Experimental prospects}% (including interplay with Belle II and BES III)}
\label{sec:expcharmmix}
\subsubsection{Mixing and time-dependent CPV in two-body decays} 
The mixing and CPV parameters in \Dz--\Dzb oscillations can be accessed by comparing the decay-time-dependent ratio of \decay{\Dz}{K^+\pi^-} to \decay{\Dz}{K^-\pi^+} rates with the corresponding ratio for the charge-conjugate processes.

The latest measurement from LHCb~\cite{LHCb-PAPER-2017-046} uses Run~1
and early Run~2 (2015--2016) data, corresponding to a total sample of
about $\mathcal{L}=5\invfb$ of integrated luminosity.  Assuming \CP
conservation, the mixing parameters are measured to be
$x_{K\pi}'^2=(3.9 \pm 2.7) \times10^{-5}$,
$y_{K\pi}'= (5.28 \pm 0.52) \times 10^{-3}$, and
$R_D^{K\pi} = (3.454 \pm 0.031)\times10^{-3}$.  Studying \Dz and \Dzb decays
separately shows no evidence for \CP violation and provides the
current most stringent bounds on the parameters $A_D^{K\pi}$ and $|q/p|$ from
a single measurement, $A_D^{K\pi} =(-0.1\pm9.1)\times10^{-3}$ and
$1.00< |q/p| <1.35$ at the $68.3\%$ confidence level.  

\begin{table}[tb]
\centering
\caption{Extrapolated signal yields, and statistical precision on the mixing and \CP-violation parameters, from the analysis of promptly produced  DCS \decay{\Dstarp}{\Dz(\to K^+\pi^-)\pi^+} decays.
Signal yields of promptly produced CF \decay{\Dstarp}{\Dz(\to K^-\pi^+)\pi^+} decays are typically $250$ times larger.\label{tab:WS-D2Kpi-yields}}
\begin{tabular}{lcccccc} 
\hline\hline\\[-4mm]
Sample ($\mathcal{L}$)  & Yield ($\times10^6$) & $\sigma(x_{K\pi}'^2)$ & $\sigma(y_{K\pi}')$ & $\sigma(A_D)$ & $\sigma(|q/p|)$ & $\sigma(\phi)$ \\
\hline
Run 1--2 (9\invfb)   &  1.8   & $1.5\times10^{-5}$ & $2.9\times10^{-4}$ & 0.51\% & 0.12 & 10\degrees \\
Run 1--3 (23\invfb)  &   10   & $6.4\times10^{-6}$ & $1.2\times10^{-4}$ & 0.22\% & 0.05 &  4\degrees \\
Run 1--4 (50\invfb)  &   25   & $3.9\times10^{-6}$ & $7.6\times10^{-5}$ & 0.14\% & 0.03 &  3\degrees \\
Run 1--5 (300\invfb) &  170   & $1.5\times10^{-6}$ & $2.9\times10^{-5}$ & 0.05\% & 0.01 &  1\degrees \\
\hline\hline
\end{tabular}
\end{table}

In Table~\ref{tab:WS-D2Kpi-yields} the signal yields and the statistical precision from Ref.~\cite{LHCb-PAPER-2017-046} are extrapolated to the end of Run 2
and to the end of \upgradetwo, assuming that the central values of the measurements stay the same. This assumption is particularly important for the \CP-violation parameters, as their precision may depend on the measured values.

Systematic uncertainties are estimated using control samples of data and none of them are foreseen to have irreducible contributions that exceed the ultimate statistical precision, if the detector performance (particularly in terms of vertexing/tracking and particle identification capabilities) is kept at least in line with what is currently achieved at LHCb.

\subsubsection{Mixing and time-dependent CPV in $\Dz\to\KS\pip\pim$} 
The self-conjugate decay \decay{D^0}{\KS \pi^+ \pi^-} includes both CF and  DCS, 
as well as $\CP$-eigenstate processes reconstructed in the same final state.
This allows for the relative strong phase between different contributions to be determined from data, and, in turn, enables both the mixing parameters $x$ and $y$, as well as the \CP-violation parameters $|q/p|$ and $\phi$ to be directly measured without need for external input.
As a result, this channel provides the dominant constraint on the parameter $x$ in the global fits.

%\td{please check the rewrite:} 
The mixing and CPV parameters modulate the time-dependence of the complex amplitudes, and these amplitudes themselves vary over the two-dimensional final state phase-space. 
%which also depend on the two-dimensional phase-space of the decay.
As such, the measurement relies both on the precise understanding of the detector acceptance as a function of phase-space and decay time, and on the accurate description of the evolution of the underlying decay amplitudes over the Dalitz plane. Both model-dependent and model-independent approaches using quantum-correlated $D\bar{D}$ pairs from $\psi(3770)$ decays can be applied.

Previous measurements from the CLEO~\cite{Asner:2005sz}, BaBar~\cite{delAmoSanchez:2010xz}, and Belle~\cite{Peng:2014oda} collaborations have used the model-dependent approach, with the Belle measurement having the best precision to date, $x = (0.56 ^{+0.20}_{-0.23})\%$, $y = (0.30 ^{+0.16}_{-0.17})\%$ (assuming \CP symmetry), and $|q/p| = 0.90 ^{+0.18}_{-0.16}$, $\phi = (-6 \pm 12)\degrees$.
The one published LHCb result was based on 1~fb$^{-1}$ of Run 1 data~\cite{LHCb-PAPER-2015-042}, and used a model-independent approach with strong phases taken from the CLEO measurement~\cite{Libby:2010nu} to determine $x = (-0.86 \pm 0.56)\%$, $y = (0.03 \pm 0.48)\%$.
This analysis used around $2\times10^5$ \decay{D^{*+}}{D^0 \pi^+}, \decay{D^0}{\KS \pi^+ \pi^-} decays from 2011, which suffered from low \KS trigger efficiencies that were significantly increased for 2012 and beyond, and will benefit further from software trigger innovations at the LHCb in the upgrade era.

At LHCb these decays can be reconstructed either through semileptonic decays, for instance \decay{B^-}{D^0 \mu^- \bar{\nu}_{\mu}}, where the muon charge is used to tag the initial $D^0$ flavour, or through prompt charm production, where the charge of the slow pion in the decay \decay{D^{*+}}{D^0 \pi^+} tags the initial flavour.
The two channels have complementary properties and both will be important components of future mixing and \CP violation analyses at LHCb.

The prompt charm yields are significantly larger than for the semileptonic channel, due to the increased production cross-section. However, for the semileptonic channel the triggering on signal candidates is much more efficient, and introduces fewer non-uniformities in the acceptance.
The estimated future yields are presented in Table~\ref{tab:D0KSPiPi_yields}.
Also shown are projected statistical precisions on the four mixing and CPV parameters, which have been extrapolated from complete analyses of the Run 1 data for both the semileptonic and prompt cases. 

\begin{table}[tb]
 \caption{Extrapolated signal yields at LHCb, together with statistical precision on the mixing and \CP violation parameters, for the analysis of the decay \decay{D^0}{\KS \pi^+ \pi^-}.
Candidates tagged by semileptonic $B$ decay (SL) and those from prompt charm meson production are shown separately.}
 \label{tab:D0KSPiPi_yields}
\begin{center}
  \renewcommand{\arraystretch}{1.2}
 \begin{tabular}{lcccccc} 
 \hline\hline
 Sample (lumi $\mathcal{L}$)              & Tag    & Yield & $\sigma(x)$ & $\sigma(y)$ & $\sigma(|q/p|)$ & $\sigma(\phi)$ \\ \hline
\multirow{2}{*}{Run 1--2 (9~fb$^{-1})$}   & SL     & 10M   & 0.07\%      & 0.05\%      & 0.07            & 4.6\degrees    \\
                                          & Prompt & 36M   & 0.05\%      & 0.05\%      & 0.04            & 1.8\degrees    \\ %\hline
\multirow{2}{*}{Run 1--3 (23~fb$^{-1})$}  & SL     & 33M   & 0.036\%     & 0.030\%     & 0.036           & 2.5\degrees    \\
                                          & Prompt & 200M  & 0.020\%     & 0.020\%     & 0.017           & 0.77\degrees   \\ %\hline
\multirow{2}{*}{Run 1--4 (50~fb$^{-1})$} & SL     & 78M   & 0.024\%     & 0.019\%     & 0.024           & 1.7\degrees    \\
                                         & Prompt & 520M  & 0.012\%     & 0.013\%     & 0.011           & 0.48\degrees   \\% \hline
\multirow{2}{*}{Run 1--5 (300~fb$^{-1})$} & SL     & 490M  & 0.009\%     & 0.008\%     & 0.009           & 0.69\degrees   \\
                                          & Prompt & 3500M & 0.005\%     & 0.005\%     & 0.004           & 0.18\degrees   \\
 \hline\hline
\end{tabular}
\renewcommand{\arraystretch}{1.0}
\end{center}
\end{table}

For this channel the dominant systematic uncertainties on mixing parameters  come from two main sources.
First is the precision with which the non-uniformities in detector acceptance  can be determined versus as a function of phase space and decay time. Second is the knowledge of the strong-phase variation across the Dalitz plane.
For the LHCb Run 2 analysis, both contributions are significantly smaller than the statistical precision. In the longer term new approaches will be necessary to further reduce these systematic uncertainties.
Trigger and event selection techniques should be adapted to emphasise uniform acceptance, a task made easier by the removal of the calorimeter-based hardware trigger. New techniques, such as the bin-flip method~\cite{Pilar:2014jfg}, 
can further reduce dependence on the non-uniform acceptance, although at the cost of
degraded statistical precision on the mixing and \CP-violation parameters.
In the model-dependent approach many of the model systematic uncertainties may reduce or vanish with increased integrated luminosity, as currently fixed parameters are incorporated into the data fit, and the data become increasingly capable of rejecting unsuitable models provided that there is suitable evolution in the model descriptions.
For the model-independent approach, the uncertainty from external inputs (currently from CLEO-c, later with 50\% reduction from BESIII) will also reduce with luminosity as the LHCb data starts to provide constraining power.
There are no systematic uncertainties which are known to have irreducible contributions that exceed the ultimate statistical precision.

For the \CP violation parameters additional sources of systematic uncertainty come from the knowledge of detector-induced asymmetries.
In particular, there is a known asymmetry between $K^0$ and $\bar{K}^0$ in their interactions with material.
The limitation here will be the precision with which the material traversed by each $\KS$ meson can be determined.
The LHCb \upgradetwo detector will be constructed to minimise material, and to allow precise evaluation of the remaining contributions. In summary, this channel has comparable power on \CP violating parameters, but with a simpler two-dimensional phase space and complementary detector systematic uncertainties, as the four-body decays that we discuss next. 
% in the previous section.

\subsubsection{Mixing and time-dependent CPV in four-body decays} 
Like \decay{\Dz}{\Km\pip} and \decay{\Dz}{\Kp\pim}, the decays
\decay{\Dz}{\Km\pip\pim\pip} and \decay{\Dz}{\Kp\pim\pim\pip} are a
pair of CF and DCS decays with high sensitivity to charm mixing.
However, the rich amplitude structure across the five dimensional phase
space of the latter decays offers unique opportunities (and
challenges) in these four-body modes.

In the phase-space integrated analysis using $3\mathrm{fb}^{-1}$ of data,
LHCb made the first observation of mixing in this decay mode, and
measured quantities $R_D^{K3\pi} = (3.21 \pm 0.014)\cdot 10^{-3}$,
as well as
$R^{K3\pi}_{\mathrm{coher}}y'_{K3\pi} = (0.3\pm 1.8)\cdot 10^{-3}$, and
$\frac{1}{4}(x^2 + y^2) = (4.8\pm 1.8) \cdot
10^{-5}$~\cite{LHCb-PAPER-2015-057}. The coherence factor,
$R^{K3\pi}_{\mathrm{coher}}$, 
%is the coherence factor, which takes into account the 
measures the effect of integrating
over the entire four-body phase
space~\cite{Atwood:2003mj,Evans:2016tlp}. \td{check refs}
%~\cite{Atwood:coherenceFactor,Evans:2016tlp}.

The unique power of multibody decays lies to a large extent in the
fact that the strong phase difference between the interfering \Dz and
\Dzb amplitudes varies across the phase space. This can be fully
exploited only by moving away from the phase-space-integrated approach to the analyses of phase space distributions, either in
bins or unbinned.
%, rather than a phase-space-integrated approach. 
Such
a ``phase space resolved'' approach allows a direct measurement of
$x_{K\pi\pi\pi}'$ and $y_{K\pi\pi\pi}'$ (rather than only the
$x'^2$ and $y'$ as in the 2-body case), and,
crucially, provides high sensitivity to the \CP\ violating variables $\phi$ and
$|q/p|$.

On the other hand, the same phase variations that make multibody decays so
powerful, are also a major challenge, as they need to be known precisely in
order to cleanly extract the mixing and \CP violation parameters of
interest. In principle, the relevant phases can be inferred from an
amplitude model such as that obtained from $3\mathrm{fb}^{-1}$ of LHCb
data~\cite{LHCb-PAPER-2017-040}. Such models may introduce  theoretical uncertainties that are unacceptably large for
the precision era of \lhcb \upgradetwo,  unless there are significant innovations
in the theoretical description of four-body amplitudes. The alternative is to use 
model-independent approaches.
% will also be applied. 
These use 
quantum-correlated events at the charm threshold to infer
the required phase information in a model-unbiased way. BESIII is
working closely with \lhcb~\cite{Malde:2223391} to provide the necessary model-independent
inputs for \decay{\Dz}{\Kp\pim\pim\pip} across different regions of
phase space for measurements of the $\gamma$ angle as well as charm mixing and
\CP violation measurements.

Sensitivity studies with model-dependent approaches give a
useful indication of the precision that can be achieved. A recent such
study, Ref.~\cite{Muller:2297069}, uses LHCb's latest
\decay{\Dz}{\Kp\pim\pim\pip} amplitude
model~\cite{LHCb-PAPER-2017-040}.  Table~\ref{tab:WS-D2K3pi} gives the
yields and sensitivities scaled from the study in~\cite{Muller:2297069},  illustrating the
impressive sensitivity of this decay mode.
The study is based on
promptly produced \Dstarp mesons, decaying in the flavour-conserving
 \decay{\Dstarp}{\Dz\pip} channel.  Several systematic uncertainties
require improvements in the analysis method in order to scale with
increasing sample sizes. However, given the huge potential of this
channel, sufficient effort is expected to be dedicated to this challenge, such
that adequate methods can be developed, and that the necessary input from
threshold measurements is both generated at BESIII and exploited
optimally at \lhcb. Indeed, once these are in place, this channel has
the potential for probing \CP violation at the $\order(10^{-5})$ level, given the
current world average value of $x$. 
\begin{table}[tb]
\caption{Extrapolated signal yields for LHCb, and sensitivity to the mixing and
  \CP-violation parameters, from the analysis of
  \decay{\Dz}{\Kp\pim\pim\pip} decays (statistical uncertainties only).\label{tab:WS-D2K3pi}}
\centering
\begin{tabular}{lccccc} 
\hline\hline\\[-4mm]
Sample ($\mathcal{L}$)  & Yield ($\times10^6$) & $\sigma(x_{K\pi\pi\pi}')$ & $\sigma(y_{K\pi\pi\pi}')$ & $\sigma(|q/p|)$ & $\sigma(\phi)$ \\
\hline
Run 1-2 (9\invfb)       &  0.22                & $2.3\times10^{-4}$       & $2.3\times10^{-4}$       & 0.020           & 1.2\degrees \\
Run 1-3 (23\invfb)      &  1.29                & $0.9\times10^{-4}$
                                                                           & $0.9\times10^{-4}$       & 0.008           & 0.5\degrees \\
Run 1-4 (50\invfb)      &  3.36                & $0.6\times10^{-4}$       & $0.6\times10^{-4}$       & 0.005           & 0.3\degrees \\
Run 1-5 (300\invfb)     & 22.5\phantom{00}     & $0.2\times10^{-4}$       & $0.2\times10^{-4}$       & 0.002           & 0.1\degrees \\
\hline\hline
\end{tabular}
\end{table}

\subsubsection{Measurement of $A_\Gamma$}
\label{sec:AGamma}
The large yields available in the SCS modes, $f=\pi^+\pi^-$ or $f=K^+K^-$, together with tagging from the \Dstarpm decay, allow for a precise measurement of $A_\Gamma$, provided the systematic uncertainties can be controlled with high degree of precision.
Tagging based on semileptonic decays of a parent bottom hadron is also possible and has been used in a published LHCb measurement~\cite{LHCb-PAPER-2014-069}, but contributes significantly lower yields. 

Most potential systematic effects are essentially constant in $t$ and
therefore cause little uncertainties in the observed decay time
evolution of the asymmetry.  However, second-order effects and
detector--induced correlation between momentum and proper decay time
are sufficient to produce spurious asymmetries, that must be
appropriately corrected. In addition, contamination from secondary
decays is a first-order effect in time that must be suppressed, and
its residual bias accounted for.  Both corrections are dependent on
the availability of a large number of CF $\Dz\to\Km\pip$  decays
as calibration, and can be expected to scale with statistics;
collection of this sample with the same trigger as for the signal
modes is therefore a crucial tool for performing this measurement with high precision in the future.

The Run~1 LHCb measurement of this quantity gave consistent results in the two $h^+h^-$ modes, averaging  $\agamma= (- 0.13 \pm 0.28 \pm 0.10)\times 10^{-3}$ \cite{LHCb-PAPER-2016-063}, which is still statistically dominated. For the reasons mentioned above, this precision is at the threshold of becoming physically interesting, making it a worthy target to pursue with more data. It seems highly unlikely that any experiment built in the foreseeable future will be able to do this, except for an upgrade of LHCb to higher luminosity. 

Table~\ref{tab:Agamma_yields} shows expected yields and precisions attainable in  \lhcb \upgradetwo, under the same assumptions on efficiencies adopted in the previous sections; this must include provisions for acquiring and storing $5\times 10^{10}$ CF decays. The ultimate combined precision is $1\times 10^{-5}$.

\begin{table}[t]
\caption{Extrapolated signal yields at LHCb, and statistical precision on
  indirect \CP violation from \agamma. 
}
 \label{tab:Agamma_yields}
\begin{center}
  \renewcommand{\arraystretch}{1.2}
 \begin{tabular}{lccccc} 
 \hline\hline
 Sample ($\mathcal{L}$)   & Tag   & Yield $K^+K^-$ & $\sigma(\agamma)_{K^+K^-}$  & Yield  $\pi^+\pi^-$  & $\sigma(\agamma)_{\pi^+\pi^-}$ \\[1mm] \hline
Run 1--2 (9~fb$^{-1})$    & Prompt   & 60M    & 0.013\% & 18M & 0.024\%       \\ %\hline
Run 1--3 (23~fb$^{-1})$  & Prompt   & 310M   & 0.0056\% &92M& 0.0104 \%       \\ %\hline
Run 1--4 (50~fb$^{-1})$  & Prompt   & 793M   & 0.0035\% &236M& 0.0065 \%       \\ %\hline
Run 1--5 (300~fb$^{-1})$ & Prompt  & 5.3G   & 0.0014\% & 1.6G & 0.0025 \%         \\
 \hline\hline
\end{tabular}
\renewcommand{\arraystretch}{1.0}
\end{center}
\end{table}

\subsubsection{Combined mixing and time dependent CPV sensitivity}
%A comparison of t
The projected precisions of the analyses presented in
the previous sections are shown in
Fig.~\ref{fig:charm_indirect_channels}, and are compared with the expected
precisions at Belle~II. The expected LHCb constraints on $\phi$ are
translated into asymmetry constraints ($A_{\CP}^{ind.} \approx x \sin(\phi)$) by multiplying by the current
HFLAV average of $x$ and neglecting the uncertainty on this under the
assumption that $x$ will be comparatively well determined in the
future. This comparison neglect additional constraining power from $|q/p|$.
The relative values of these asymmetry constraints with those from
\agamma is indicative only.

\begin{figure}
  \centering
  \includegraphics[width=1.0\textwidth]{section3/figures/CharmIndirectCPVSummary.pdf}
  % 
  \caption{The predicted constraints on the indirect \CP violation
    asymmetry in
    charm from the decay channels indicated in the labels at the
    bottom of the columns. Predictions are shown in LS2
    (2020) from LHCb, LS3 (2025) from LHCb, at the end of Belle~II (2025), and at the end
    of the HL-LHC LHCb \upgradetwo\ programme. } 
\label{fig:charm_indirect_channels}
\end{figure}

The analyses presented in the previous sections are also combined to
establish the sensitivity to the \CP-violating parameters $|q/p|$ and
$\phi$. The combination is performed using the method described in Ref.~\cite{LHCb-PAPER-2016-032}.
At an integrated luminosity of $300\invfb$ the sensitivity to $|q/p|$ is expected to be $0.001$ and that to $\phi$ to be $0.1^\circ$.
This remarkable sensitivity is contrasted in Fig.~\ref{fig:charm_indirect} with the HFLAV world average
as of 2017. We can conclude that the
LHCb \upgradetwo will have impressive power to characterise NP
contributions to \CP violation and is the only foreseen facility with strong potential of
probing the SM contribution.

\begin{figure}
  \centering
  \includegraphics[width=0.6\textwidth]{section3/figures/charm_indirect_CPV}
  % 
  \caption{The estimated constraints for LHCb \upgradetwo\ on $\phi$, $|q/p|$ from the combination of the analyses (red), see main text for details, compared to the current world-average precision (light blue).} 
\label{fig:charm_indirect}
\end{figure}

\subsubsection{Direct CP violation}\label{sec:directCP:exp}

The SCS decays $D^0 \rightarrow \Km\Kp$ and $D^0 \rightarrow \pim\pip$ 
play a critical role in the measurement of time-integrated direct \CP violation
through time-integrated \CP asymmetry in the $h^-h^+$ decay rates, Eq. \eqref{ACPhh}.
%\begin{equation}\label{ACPhh}
%A_{\CP}(\Dz\to h^-h^+)\equiv \frac{\Gamma(D^0\rightarrow h^-h^+)-\Gamma(\Dzb\rightarrow h^-h^+)}{\Gamma(D^0\rightarrow h^-h^+)+\Gamma(\Dzb\rightarrow h^-h^+)}.
%\end{equation}
%
The sensitivity to direct \CP violation is enhanced through a measurement of the difference in \CP asymmetries between \Dz\to\Km\Kp and \Dz\to\pim\pip decays, $\Delta A_{\CP}=A_{\CP}(\Km\Kp)-A_{\CP}(\pim\pip)$.
The individual asymmetries $A_{\CP}(\Km\Kp)$ and $A_{\CP}(\pim\pip)$ can also be measured. 

A measurement of the time-integrated \CP asymmetry in $D^0 \rightarrow \Km\Kp$ has been performed in \lhcb\ with 3\invfb 
collected at centre-of-mass energies of 7 and 8\tev. The flavour of the charm meson at production is determined from the charge of the pion in
$\Dstarp\to\Dz\pi^+$ decays, or via the charge of the muon in semileptonic \bquark-hadron decays, $\Bb\to\Dz\mun\neumb X$. 
The analysis strategy so far relies on the $D^+\rightarrow K^-\pi^+\pi^- $, $D^+\rightarrow K_s^0\pi^+$ and $\Dstarp\to\Dz (\to K^-\pi^+)\pi^+$ decays as control samples~\cite{LHCb-PAPER-2016-035}. In this case, due to the weighting procedures aiming to fully cancel the production and reconstruction asymmetries, the effective prompt signal yield for $A_{\CP}(\Km\Kp)$ is reduced. The expected signal yields and the corresponding statistical precision in LHCb \upgradetwo are summarised in Table~\ref{tab:expectedyields}.

\begin{table}[t]
\caption{\small Extrapolated signal yields at LHCb and statistical precision on direct \CP violation observables for the promptly produced samples.}
\centering
\begin{tabular}{l c  c  c  c  c }
\hline\hline\\[-4mm]
Sample ($\mathcal{L}$) & Tag & Yield & Yield  & $\sigma(\Delta A_{\CP})$ &  $\sigma(A_{\CP}(hh))$ \\
 &  & \Dz\to\Km\Kp & \Dz\to\pim\pip & [$\%$] &  [\%] \\
\hline
Run 1-2 (9 \invfb)  & Prompt      &  52M &   17M & $0.03$  & $0.07$ \\   
Run 1-3 (23 \invfb) & Prompt      & 280M &   94M & $0.013$  & $0.03$ \\   
Run 1-4 (50 \invfb) & Prompt     &  1G  &  305M & $0.01$ & $0.03$ \\ 
Run 1-5  (300 \invfb) & Prompt    & 4.9G &  1.6G & $0.003$ & $0.007$ \\ 
\hline\hline
\end{tabular}
\label{tab:expectedyields}
\end{table}
 
The observable $\Delta A_{\CP}$  is robust against systematic uncertainties. The main sources of systematic uncertainties are inaccuracies in the fit model, the weighting procedure, the contamination of the prompt sample with secondary $\Dz$ mesons and the presence of peaking backgrounds.
There are no systematic uncertainties with expected irreducible contributions above the ultimate statistical precision.  This channel is already entering the upper range of the physically interesting sensitivities, and will likely continue to provide the world's best sensitivity to direct \CP violation in charm at LHCb \upgradetwo. The power of these two-body \CP eigenstates at LHCb \upgradetwo is illustrated in Fig.~\ref{fig:agammadeltaacp}, which shows the indirect (see Sect.~\ref{sec:AGamma}) and direct \CP constraints that will come from these modes.

\begin{figure}
  \centering
  \includegraphics[width=0.7\textwidth]{section3/figures/direct_indirect_cpv_LHCb_300fb}
  % 
  \caption{The estimated constraints for LHCb \upgradetwo\ on indirect and direct charm \CP violation from the analysis of two-body \CP eigenstates. The current world-average precision~\cite{HFLAV16} is $\pm2.6 \times 10^{-4}$ for indirect and $\pm 18 \times 10^{-4}$ for direct \CP violation and thus larger than the full scale of this plot.} 
\label{fig:agammadeltaacp}
\end{figure}

There are a number of other two-body modes of strong physics interest for which \upgradetwo will also make important contributions. These include the decay modes $\Dz\to\KS\KS$ (0.28\%), 
$\Dz\to\KS\overline{K^{0*}}$ (0.21\%), $\Dz\to\KS{K^{0*}}$ (0.15\%), $\Dsp\to\KS\pip$ $(3.2\times 10^{-4})$, $\Dp\to\KS\Kp$ $(1.2\times 10^{-4})$, $\Dp\to \phi \pip$ $(6\times 10^{-5})$, $\Dp \to \eta'\pip$ $(3.2\times 10^{-5})$, $\Dsp \to \eta'\pip$ $(3.2\times 10^{-4})$, where the projected statistical only \CP asymmetry sensitivities are given in brackets after the decay mode. The first three modes mentioned are notable as they receive sizeable contributions from exchange amplitudes at tree-level and could have a relatively enhanced contribution from penguin annihilation diagrams which are sensitive to NP. Consequently, 
%some authors have highlighted them as 
they could be potential \CP violation discovery channels~\cite{Nierste:2015zra},\cite{Nierste:2017cua}.

Searches for direct \CP violation in the phase space of SCS $D^+\to h_1h_2h_3$ 
decays, hereafter referred to as $D\to 3h$, are complementary to that of $D^{(0,+)}\to h_1h_2$ 
($h_i = \pi, K$). In charged $D$ systems, only \CP violation in the decay is possible. The main observable 
is the \CP asymmetry, which, in the case of two-body decays, is a single number. In contrast,  $D\to 3h$ 
decays allow to study the variation of the \CP asymmetry across the two-dimensional phase space 
(usually represented by the Dalitz plot). 

The estimated  signal yields in future LHCb upgrades  are summarised in Table~\ref{tab:Dhhh-yields}. The yields are based on
an extrapolation of the Run 2 yields per unit luminosity, made under the same assumptions as in Section 6.1. The
estimated sensitivities to observation of \CP violation, using $D^+ \to \pi^-\pi^+\pi^+$ as an example, are presented 
in Table~\ref{tab:Dhhh-sensitivity}.

\begin{table}[tb]
\centering
\caption{Extrapolated signal yields at LHCb, in units of 10$^6$, for the SCS decays $D^+ \to K^-K^+\pi^+$, $D^+ \to \pi^-\pi^+\pi^+$,
and for the DCS decays $D^+ \to K^-K^+K^+$, $D^+ \to \pi^-K^+\pi^+$. \label{tab:Dhhh-yields}}
\begin{tabular}{lcccc}
\hline \hline
Sample ($\mathcal{L}$)  & $D^+ \to K^-K^+\pi^+$  & $D^+ \to \pi^-\pi^+\pi^+$ &  $D^+ \to K^-K^+K^+$ & $D^+ \to \pi^-K^+\pi^+$ \\
\hline
Run 1--2 (9\invfb)     &   200       & 100     & 14       & 8   \\
Run 1--4 (23\invfb)   &   1,000    & 500  & 70     & 40   \\
Run 1--4 (50\invfb)   &   2,600    & 1,300  & 182     & 104   \\
Run 1--6 (300\invfb) &  17,420   &  8,710 & 1,219  &  697  \\
\hline \hline
\end{tabular}
\end{table}

\begin{table}[bt]
\centering
\caption{Sensitivities to several illustrative \CP-violation scenarios in $D^+ \to \pi^-\pi^+\pi^+$ decay. Simulated $D^+$ and $D^-$
Dalitz plots are generated with relative changes in the phase of the $R\pi^{\pm}$ amplitude, $R=\rho^0(770)$, 
$f_0(500)$ or $f_2(1270)$. The values of the phase difference are given in degrees and correspond to 
a 5$\sigma$  \CP-violation effect. Simulations are performed with $3\invfb$ and extrapolated to the 
expected luminosities.}
\begin{tabular}{lc c c c}
\hline\hline
  resonant channel           & $9\invfb$   & $23\invfb$     & $50\invfb$     & $300\invfb$\\
\hline
$f_0(500)\pi$                    & 0.30 &  0.13   &  0.083   &  0.032 \\
$\rho^0(770\pi$                & 0.50 &  0.22   &  0.14     &  0.054\\
$f_2(1270)\pi$                  & 1.0  &  0.45   &   0.28    & 0.11\\
\hline\hline
\end{tabular}
\label{tab:Dhhh-sensitivity}
\end{table}

The SM generated \CP violation could be observed in the SCS decays, such as \decay{\Dz}{\pip\pim\pip\pim} and \decay{\Dz}{\Kp\Km\pip\pim}, while NP is needed to justify any observation of \CP violation in the DCS decays, such as \decay{\Dz}{\Kp\pim\pip\pim}.
Many techniques can be adopted to search for \CP violation, all of them exploiting the rich resonant structure of the decays.
The methods used so far  at \lhcb are based on ${\widehat{T}}$-odd asymmetries and the energy test, while studies are ongoing to measure model-dependent \CP asymmetries in the decay amplitudes.

The study of ${\widehat{T}}$-odd asymmetries exploits potential $P$-odd \CP violation from the interference between different amplitude structures in the decay, as described in Ref.~\cite{Durieux:2015zwa}.  This uses a triple product $C_T=\vec{p}_A\cdot (\vec{p}_B
\times \vec{p}_C)$ constructed from the momenta of three of the final state particles $\vec{p}_A,\vec{p}_B,\vec{p}_C $.
\lhcb has studied ${\widehat{T}}$-odd asymmetries using 3\invfb data from the Run~1 dataset, and obtained a sensitivity of $2.9\times10^{-3}$ with very small systematic uncertainties~\cite{LHCB-PAPER-2014-046}.
The peculiarity of this measurement is the absence of instrumental asymmetries, since it is given by the difference of two asymmetries measured separately on \Dz and \Dzb decays, $a_{\CP}=(A_T-\bar{A}_T)/2$, where
\begin{align}
  A_T = \frac{(\Gamma(\Dz,C_T>0)-\Gamma(\Dz,C_T<0))}{(\Gamma(\Dz,C_T>0)+\Gamma(\Dz,C_T<0))},\quad \bar{A}_T = \frac{(\Gamma(\Dzb,\bar{C}_T>0)-\Gamma(\Dzb,\bar{C}_T<0))}{(\Gamma(\Dzb,\bar{C}_T>0)+\Gamma(\Dzb,\bar{C}_T<0))}.
\end{align}
One can therefore expect the errors to scale with luminosity to reach a sensitivity down to $2.9\times10^{-5}$ ($9.4\times10^{-5}$) for $\Dz\to\pip\pim\pip\pim$ ($\Dz\to\Kp\Km\pip\pim$) decays, as detailed in Table~\ref{tab:CS-D24h-yields}.
\begin{table}[tb]
\centering
\caption{ Extrapolated signal yields, and statistical precision on ${\widehat{T}}$-odd \CP-violation observables at LHCb.\label{tab:CS-D24h-yields}}
\begin{tabular}{lcccc}
\hline\hline
  & \multicolumn{2}{c}{$\Dz\to\pip\pim\pip\pim$} & \multicolumn{2}{c}{$\Dz\to\Kp\Km\pip\pim$} \\
Sample ($\mathcal{L}$)  & Yield ($\times10^6$) & $\sigma(a_{\CP}^{\widehat{T}\text{-odd}})$  & Yield ($\times10^6$) & $\sigma(a_{\CP}^{\widehat{T}\text{-odd}})$ \\
\hline
Run 1--2 (9\invfb)   & 13.5   & $2.4\times10^{-4}$ &  4.7   & $5.4\times10^{-4}$ \\
Run 1--3 (23\invfb)  &   69   & $1.1\times10^{-4}$ &   12   & $3.4\times10^{-4}$ \\
Run 1--5 (300\invfb) &  900   & $2.9\times10^{-5}$ &  156   & $9.4\times10^{-5}$ \\
\hline\hline
\end{tabular}
\end{table}

The energy test method is insensitive to global asymmetries. However, it is expected that it will become sensitive to variations in phase space of production and detection asymmetries.
These can be controlled in data by application of the method to CF decays, such as \decay{\Dz}{\Km\pip\pip\pim}.
Assuming scaling with the square-root of the ratio of sample sizes, the same p-values can be expected for the \CP asymmetries given in Table~\ref{tab:charm_623_energy_test}.

\begin{table}[bt]
\centering
\caption{Overview of sensitivities to various \CP-violation scenarios for \decay{\Dz}{\pip\pim\pip\pim} decays as extrapolated from Ref.~\cite{LHCb-PAPER-2016-044}. The relative changes in magnitude and phase of the amplitude of the resonance $R$ to which sensitivity is expected are given in $\%$ and $^\circ$, respectively. The $P$-wave $\rho^0(770)$ is a $P$-odd component. The phase change in this resonance is tested with the $P$-odd \CP-violation test. Results for all the other scenarios are given with the standard $P$-even test.}
\begin{tabular}{lc c c}
\hline\hline
$R$ (partial wave)        & $9\invfb$   & $23\invfb$ & $300\invfb$\\
\hline
\decay{a_1}{\rhoz\pi} (S) & $1.4\%$     & $0.6\%$      & $0.17\%$\\
\decay{a_1}{\rhoz\pi} (S) & $0.8^\circ$ & $0.35^\circ$ & $0.10^\circ$\\
\rhoz\rhoz (D)            & $1.4\%$     & $0.6\%$      & $0.17\%$\\
\hline
\rhoz\rhoz (P)            & $0.8^\circ$ & $0.35^\circ$ & $0.10^\circ$\\
\hline\hline
\end{tabular}
\label{tab:charm_623_energy_test}
\end{table}

Charm decays with neutrals in the final state can help to shed light on the SM or beyond-SM origin of possible \CP-violation signals by testing correlations between \CP asymmetries measured in various flavour-\grpsuthree or isosping related decays, see Section \ref{sec:isospin:null:test} and Refs. \cite{Bhattacharya:2012ah,Pirtskhalava:2011va,Feldmann:2012js}. These modes are, however, particularly challenging in hadronic collisions, where the calorimeter background for low energy clusters is high, while the trigger retention rate needs to be kept low to allow for affordable rates. 

Nevertheless, good performances are achieved when considering decays with at least two charged particles in the final states, such as \decay{\Dz}{\pi^+\pi^-\pi^0}, since the charge particles help to identify the displaced decay vertex of the charm meson. In only 2\invfb of data collected during 2012, LHCb has reconstructed about 660,000 \decay{\Dz}{\pi^+\pi^-\pi^0} decays~\cite{LHCb-PAPER-2014-054}, \ie about five times more than Babar from its full data set~\cite{TheBABAR:2016gom}, with comparable purity. Preliminary estimates for Run 2 data, give about 500\,000 signal decays per \invfb, making future \CP-violation searches in this channel very promising. Similarly, large samples of \decay{\D_{(\squark)}^+}{\eta^{(\prime)}\pi^+} decays, with \decay{\eta^{(\prime)}}{\pi^+\pi^-\gamma}, or \decay{\Dp}{\pi^+\pi^0} decays, with \decay{\pi^0}{e^+e^-\gamma}, are already possible with the current detector. The \decay{\D_{(\squark)}^+}{\eta^{\prime}\pi^{+}} mode, as an example, has been used by LHCb during Run 1 to perform the most precise measurement of \CP asymmetries in these channels to date, with uncertainties below the 1\% level~\cite{LHCb-PAPER-2016-041}.

More challenging final states consisting only of neutral particles, such as $\pi^0 \pi^0$ or $\eta \eta$, can still be reconstructed with \decay{\pi^0}{\gamma\gamma} or \decay{\eta}{\gamma\gamma} candidates made of photons which, after interacting with the detector material, have converted into an $e^+e^-$ pair. Such conversions must occur before the tracking system to have electron tracks reconstructed. Although the reconstruction efficiency of these ``early'' converted photons in the current detector reaches only a few percent of the calorimetric photon efficiency, their purity is much higher. This approach may become interesting only with the large data sets that are expected to be collected by the end of \upgradetwo. The $\eta$ decays can also be reconstructed through the $\pi^+\pi^-\gamma$ final state.

Unlike the \Dz decays which are usually tagged with a soft pion from \Dstarp decays, there is no easy tagging of the \Dp modes, which thus often suffer from a high combinatorial background. Employing a $\pi^0$ tag using \decay{\Dstarp}{\Dp\pi^0} decays could facilitate future studies of \Dp decays, in particular those with challenging and/or high multiplicity final states.

The study of these modes may be challenging in Run~3 due to the cluster pile-up at higher luminosities and radiation damage of the current calorimeter. The new calorimeter proposed for LHCb \upgradetwo would have an improved granularity. It would therefore improve the efficiency for \decay{\pi^0}{\gamma\gamma} decays and, in particular, could make the $\pi^0$ tag feasible.

\subsubsection{Rare leptonic and radiative charm decays}
The most experimentally accessible very rare charm decay is \dmumu. The world's best limit on \brdmumu was obtained by LHCb with $0.9\invfb$ of 2011 data~\cite{LHCb-PAPER-2013-013}, resulting in
\begin{equation}
\brdmumu< 6.2 \times 10^{-9}\text{ at 90\% CL. }
\end{equation}
Extrapolating the current detector performance, the expected limit is about $5.9 \times 10^{-10}$ with $23~\invfb$ and $1.8 \times 10^{-10}$ with $300~\invfb$ of integrated luminosity,
covering a large part of the unambiguous space to search for NP without being affected by long distance uncertainties in the SM predictions.

The next class of rare charm decays which are particularly suited for experiments at hadron colliders 
are three-body decays with a pair of leptons in the final state, such as $D^{+}_{(s)}/ \Lc \to h^+ \ellell$ and $D^0 \to h^+h^- \ellell$. In some NP scenarios the short distance contributions can be enhanced by several orders of magnitude allowing NP to manifest as an enhancement of the branching fraction. An example of such a model is shown in Fig.~\ref{fig:LC2PMUMU}, where the Wilson coefficients were assumed to obtained NP contributions, $C_9^{\rm NP} =-0.6$ and $C_{10}^{\rm NP} =0.6$ (cf. Eqs. \eqref{eq:Leff:e1}, \eqref{eq:operators}).
Outside of the resonance regions the short distance contributions are 
%enhanced to  similar levels as 
comparable to the long distance effects. The chosen NP benchmark point leads to branching ratios just below the current LHCb limit~\cite{LHCb-PAPER-2017-039} \td{this branching ratio cuts out resonances, right? Should we denote this explicitly?}
\begin{equation}
\BF( \Lc \to p \mumu) < 5.9 \times 10^{-8}~{\rm at}~90\%~{\rm CL}.
\end{equation}
In the \upgradetwo LHCb is expected to improve the limit to
\begin{equation}
\BF( \Lc \to p \mumu) < 4.4 \times 10^{-9}~{\rm at}~90\%~{\rm CL}.
\end{equation}

\begin{figure}[tb]
\includegraphics[width=0.5\textwidth]{{section3/figures/Lambdac_proton_mu_mu_dB_new}.pdf}
\includegraphics[width=0.5\textwidth]{{section3/figures/Lambdac_proton_mu_mu_dB_C9NP_m0p6_C10_0p6_new}.pdf}
\caption{Differential branching fraction for the decay $\Lc \to p \mumu$ as a function of $q^2$ (left) in the SM and (right) in a NP scenario with ${\cal C}_9^{\rm NP}=-0.6$ and ${\cal C}_{10}^{\rm NP}=-0.6$~\cite{Meinel:2017ggx}. The current LHCb limit is marked with green. The extrapolated upper limit with the $23$ and $300~\invfb$ data sets is marked in red.\label{fig:LC2PMUMU}}
\end{figure}

An order of magnitude improvement is expected for the limit on $\BF(D^{+}_{(s)}\to\pi^+\mu^+\mu^-)$ decays, which is currently at $7.3\times10^{-8}$
at 90\% CL \cite{LHCb-PAPER-2012-051}. The expected upper limits are about $1.3\times10^{-8}$ with $23 \invfb$ and $0.37\times10^{-8}$  with $300 \invfb$.
In addition, LHCb will have the ability to measure angular observables such as the forward-background asymmetry, $A_{\rm FB}$, or time integrated $A_{\CP}$, 
which will provide additional handles to separate the long distance from the short distance contributions, and for which some theoretical predictions in NP scenarios already exist \cite{deBoer:2015boa},
as shown in Fig.~\ref{fig:D2PIMUMU}.

\begin{figure}[tb]
\includegraphics[width=0.33\textwidth]{section3/figures/dBQDppiR2mu.pdf}
\includegraphics[width=0.66\textwidth,clip=true,trim=0mm 60mm 0mm 0mm]{section3/figures/ACPD2pimumu.pdf}
\caption{
    (Left) Differential branching fraction of $\D^+ \to \pi^+ \mumu$ as a function of $q^2$ in the SM. The current LHCb limit is marked with a dashed black line. (Middle and right) Predictions for $A_{\CP}$ in a particular NP scenario of Ref.~\cite{deBoer:2015boa} with different strong phases $\delta_{\rho,\phi}$.\label{fig:D2PIMUMU}
}
\end{figure}

Last but not least with a sample corresponding to an integrated luminosity of $300~\invfb$ LHCb will be able to perform searches for the LFV decays $D^{+}_{(s)}/ \Lc\to h^+ \ell^+ \ell^{\prime -}$
and perform tests of lepton universality via the ratios ${\cal B} (\decay{D^{+}_{(s)}/\Lc}{h^+ \mu^+ \mu^{-}})/{\cal B}(\decay{D^{+}_{(s)}/\Lc}{h^+ e^+ e^{-}})$. 

The decays $D^0\to h^+h^-\ellell$ have richer dynamics than the 
two- and three-body decays, allowing for a variety of differential distributions to be investigated.
Due to the huge charm production cross-section at the LHC, and LHCb's ability to trigger on low \pt\ dimuons, 
LHCb has unique physics reach in studying these decays. 
In fact, significant progress has already been made with the observation of the CF decay $\decay{\Dz}{\Km\pip\mun\mup}$ (with the dimuon mass in the $\rho/\omega$ region) with a branching fraction $(4.17\pm0.42)\times10^{-6}$~\cite{LHCb-PAPER-2015-043}  and the SCS decays $\decay{\Dz}{\pip\pim\mup\mun}$ and $\decay{\Dz}{\Kp\Km\mup\mun}$ with branching fractions of $(9.64\pm1.20)\times10^{-7}$ and $(1.54\pm0.33)\times10^{-7}$, respectively~\cite{LHCb-PAPER-2017-019}. For the latter, also the differential branching fraction as a function of the dimuon mass squared, $q^2$, was measured.
Furthermore, LHCb has performed the first measurement of \CP- and angular asymmetries in $\decay{\Dz}{\pip\pim\mumu}$ and $\decay{\Dz}{\Kp\Km\mumu}$ decays using Run~2 data. 
This resulted in the first determination of the forward-backward asymmetry $A_{\rm FB}$, the triple-product asymmetry $A_{2\phi}$ and the \CP-asymmetry $A_{\rm CP}$ with uncertainties at the percent-level~\cite{LHCb-PAPER-2018-020}. 
Moreover, the implementation of triggers for dielectron modes 
opens the possibility of measuring branching-fraction ratios between di-muon and di-electron modes.
Since the main limit for these studies comes from the available statistics, LHCb will already have excellent prospects in Run~3 and 4. 
Projected signal yields for the muonic modes of ${\mathcal O}(10^4)$ will allow more sensitive studies of angular asymmetry and the first amplitude analyses to attempt to disentangle short distance and long distance components.
However,  the full potential for these decays will be exploited only with the $300$\invfb LHCb upgrade.

\subsubsection{Prospects with charm baryons}

In general, charmed baryon decays offer a rich laboratory within which to study 
matter-antimatter asymmetries.
The most easily accessible modes for \lhcb are multibody decays containing one 
proton and several kaons or pions, such as the CF $\decay{\Lc}{\proton\Km\pip}$ decay and the SCS 
$\decay{\Lc}{\proton\Km\Kp}$ and $\proton\pim\pip$ decays.
Unlike the analogous final states of the spinless neutral and charged \PD mesons, such 
charmed baryon final states have at least five degrees of freedom, allowing for 
a complex variation of the strength of \CP violation across the phase space.
A complete description of such a space is experimentally challenging.
This is compounded by the relatively short lifetime of the \Lc baryon with 
respect to those of the \PD mesons, which decreases the power of selections 
based on the displacement of the charm vertex, increasing the prompt 
combinatorial background, which mainly comprises low-momentum pions and kaons.
Selections of charm baryons then heavily rely on good discrimination between 
proton, kaon, and pion hypotheses in the reconstruction of charged tracks, and 
on a precise secondary vertex reconstruction.
The latter is also necessary when reconstructing charm baryons originating from 
semileptonic \bquark-hadron decays, which has the advantage of both a simple 
trigger path (a high-\pt, displaced muon) and a cleaner experimental signature 
due to the large \bquark-hadron lifetime.
The presence of the proton in the final state, whilst a useful handle for 
selections, poses experimental challenges for \CP-violation measurements in 
addition to those present for measurements with \PD mesons, as the 
proton-antiproton detection asymmetry must be accounted for.
This has not yet been measured at \lhcb due to the lack of a suitable control 
mode.

The most precise measurement of \CP violation in charm baryons was made 
recently by the \lhcb collaboration using Run 1 data, corresponding to 
$3$\invfb of integrated luminosity~\cite{LHCb-PAPER-2017-044}.
The difference between the phase-space-averaged \CP asymmetries in 
$\decay{\Lc}{\proton\Km\Kp}$ and $\decay{\Lc}{\proton\pim\pip}$ decays was 
found to be consistent with \CP symmetry to a statistical precision of 
$0.9\,\%$.
This difference is largely insensitive to the proton detection asymmetry, but 
masks like-sign \CP asymmetries between the two modes.
Further studies must then gain a precise understanding of the proton detection 
asymmetry, in addition to measuring the variation of \CP violation across the 
decay phase space.

Although there is little literature on the subject, the magnitude of direct
\CP violation in charm baryon decays is expected to be similar to that for 
charmed mesons~\cite{Bigi:2012ev}, and so the first step in furthering our 
understanding is to reach a precision of $0.5\times10^{-4}$. This can be met by LHCb
given the $300$\invfb of integrated luminosity collected by the end of Run 5.
The acquisition of more data is vital in enabling studies of states heavier 
than the \Lc baryon, as their alternate compositions may permit considerably different dynamics. In this respect it is interesting to note that the \Xiccpp baryon was discovered~\cite{LHCb-PAPER-2018-026} by LHCb using the $\Lc\Km\pip\pip$ final state with 
data taken in 2016, corresponding to an integrated luminosity of $1.7$\invfb.
A signal yield of $313 \pm 33$ was determined, which can be 
extrapolated to around $100,000$ such decays obtainable with data corresponding to
$300$\invfb, allowing asymmetry measurements with a precision of $0.4$\,\%. The flexible trigger and real-time analysis concept of LHCb's \upgradetwo will however enable not only this measurement but a detailed mapping of the entire landscape of multiply-heavy charmed hadrons.

%In conclusion, the high statistics of LHCb \upgradetwo offers the possibility
%to open up the detailed study of this new research area.
