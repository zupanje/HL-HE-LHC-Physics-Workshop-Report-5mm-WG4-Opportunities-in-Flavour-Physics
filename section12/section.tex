% don't remove the folling lines, and edit the defintion of \main if needed
\documentclass[../report.tex]{subfiles}
\providecommand{\main}{..}
\IfEq{\jobname}{\currfilebase}{\AtEndDocument{\biblio}}{}
% until here

\begin{document}

\section{Conclusions}
Flavour physics has many faces. This is manifest already in the Standard Model (SM), where the origin of the flavour structure are the Higgs Yukawa couplings, but which exhibits itself in the flavour non-diagonal couplings of $W$ boson, once we rotate to the quark mass eigenbases. As a result we can search for signs of New Physics (NP) using flavour probes in many sectors: the Higgs couplings to the SM fermions, be it deviations in the flavour diagonal Yukawa couplings or in searches for flavour violating couplings, or in the flavour changing transitions of top, bottom, charm and strange quarks or of leptons. 

HL-LHC and HE-LHC experiments are set to make great strides in almost all of these probes, as exhibited in great detail in this document. The high luminosity programs at ATLAS and CMS are unmatched in their ability to push the precision frontier  in measuring Higgs couplings and searching for rare top decays, at the same time expanding their flavour-physics contribution. LHCb with its planned \upgradetwo can, quite impressively, cover a very wide range of low-energy probes in bottom, charm and strange flavor transitions. 

These low-energy measurements will bloom in competition with the upgraded Belle 2 super $B$ factory, which is coming on line at this very moment.  There is also large complementarity between the two programs: The LHC hadronic environment gives access to a larger set of hadronic states, such as heavy baryons, and a larger sample of $B_s$ mesons. The HL/HE-LHC
upgrades will push well past the benchmarks set by Belle 2 for many measurements in the near future. 
%, while Belle 2 is likely to be the frontrunner in the measurements with missing energy. 

In turn, this complementarity is very timely, given the tantalizing hints for NP surfacing in measurements of semi-tauonic and rare semi-muonic $B$-decays, which suggest that Lepton Universality may be broken by new dynamics in the TeV range. The flavor anomalies thus lead to a different type of complementarity, between the low-energy flavor probes and the direct searches of NP at high \pt in ATLAS and CMS. This interplay is especially prolific in the context of model building, where the anomalies can be fitted into the more ambitious BSM program. Therefore, the HL/HE-LHC programme has the potential to shape the NP to come, in case the anomalies were confirmed in the coming years.    

Finally, it is important to note that the projected advancements in experiments are set to be accompanied by improvements in theory, most notably, in the calculations of the hadronic matrix elements by lattice QCD simulations. These matrix elements are required for the precision flavour-physics program, to convert the measurements in constraints or to lead to unambiguous discoveries of NP through the effects of their virtual corrections. In addition, the capabilities of LHCb to chart the spectrum and properties of resonances and exotic states in QCD, will assist a better understanding of the nonperturbative phenomena. The combined improvements in the theoretical predictions, along with HL/HE-LHC experimental achievements are set to improve the current reach on the NP physics mass scale by at least  a factor of two. This represents quite an impressive advance, with real discovery potential. 


 %with focus on a global analysis of the scale of NP whether the anomalies turn out to be real or not.\td{missing}

\end{document}
