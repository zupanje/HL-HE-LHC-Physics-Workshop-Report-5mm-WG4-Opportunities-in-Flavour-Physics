% don't remove the folling lines, and edit the defintion of \main if needed
\documentclass[../report.tex]{subfiles}
\providecommand{\main}{..}
\IfEq{\jobname}{\currfilebase}{\AtEndDocument{\biblio}}{}
% until here

\begin{document}
\setcounter{section}{0}
\section{Instructions}
\setcounter{section}{0}
%\section{Introduction}
The main repository of the report material is Overleaf. To use it you register for free on {\tt www.overleaf.com}. This allows people to work simultaneously on the report (as long as they do not edit the same sentence at the same time). There are two main ways you can use Overleaf:
\begin{enumerate}
\item You can work directly on the Overleaf version on-line, in which case you will see ``instantaneous compilation of latex'' through the web interface. You can see the file structure after you click on the ``Project'' button. 
\item 
You can use git, which we assume will be most convenient for the majority of people (instructions below).
\end{enumerate}

Please let us know with which e-mail you register to Overleaf so that we can share the editing rights of the projects with you (by default we used the email address that this instructions were sent to). Make sure to click on the link the overleaf system sends you - in that way you accept the share of the repository. If you use your {\tt cern.ch} mail
account to register, this will give you, for free, the Professional license. This
is needed if you want to access the full functionality, including the
ability to grant editing rights to many people.

To use Overleaf through git you first download locally the whole package  from the git repository. You can then use the
standard git tools to commit updated versions.

In order to clone the repository of the WG4 chapter, type in the terminal (when in directory where you want the report to be):
\begin{verbatim}
$>$ git clone https://git.overleaf.com/14797298htvwtqymfvhs 
\end{verbatim}
%, with several people working simultaneously or 

Before you make changes first pull the latest version
\begin{verbatim}
$>$ git pull
\end{verbatim}

To commit changes:
\begin{verbatim}
$>$  git commit -am "comment"
$>$  git push
\end{verbatim}

This document can be compiled in its entirety, using the
\begin{verbatim}
$>$ pdflatex report
\end{verbatim}
command from the main directory. Alternatively, you can compile
individual sections, with the command
\begin{verbatim}
$>$ pdflatex section
\end{verbatim}
issued from the section subdirectory. This will generate the complete
section, including its 
bibliography.
I am not sure this is possible when working directly on the web with
Overleaf. But this works in your local version, if you input the
pdflatex commands yourself.

\subsection{Conventions when writing}
If you work on a subsection, please add your name as a comment below the title, so we can track the authorship. For instance:
\begin{verbatim}
\section{Vision}
{\bf Author(s): G. Isidori, Z. Ligeti} (5pages)
\% G. Isidori, Z. Ligeti
\end{verbatim}

 The file paths to figures
(Fig.~\ref{fig:higgs}) must all include this absolute
reference, as in
\begin{verbatim}
\includegraphics[width=0.45\textwidth]{\main/section_instructions/img/hgg.pdf} .
\end{verbatim}

\begin{figure}[ht]
\centering
\includegraphics[width=0.75\textwidth]{\main/section0/img/hgg.pdf}
\caption{Caption of the figure.}
\label{fig:higgs}
\end{figure}

As for tables, please use the standard tabular environment, with the
caption on top of the table. For cross referencing, try use lables
such as
\verb|\label{eq:eqname}|, \verb|\label{tab:tabname}| and
\verb|\label{fig:figname}|. 

If you need subsections, you can include them here directly, or input
them as files, which can live in their own subdirectory. If including
a subsection from a file, import them with the command
\begin{verbatim}
\subfile{\main/section0/sub1/mysubsec}
\end{verbatim}
where mysubsec.tex would start of course with the statement
\verb|\begin{subsection}|.  In this case, make sure you define
properly the path to figures and other files. This is crucial to
guarantee compatibility of compilation of the whole Chapter and or the
individual section. The fact that a Section compiles doesn't imply
that the Chapter will compile, if paths are entered inaccurately. At
this time I have not foreseen the ability to compile subsections
standalone, I hope it is not necessary.

\subfile{\main/section0/sub1/mysubsec}

\subsection{If you want to create your Overleaf project}
You can create the project for your section by cloning this
project. It's important that you do it using your professional
Overleaf account (namely with your cern.ch email credentials), since
the project inherits the functionalities available to its owner.
After creating the project, make sure you set the proper
administration rights, listing people who will have editing
rights. You can do this clicking on the "share" button on top of the overleaf project page. You are given various options, such as making the project read/write accessible to anyone who has the url, or just readable to anyone who has a (different) url, or restricted to users selected in a list. The same button gives you the git repository url, to download the project locally. Notice that those you give rights to cannot further assign
rights to others. Only the owner of the project can grant access
rights. Of course section editors can clone a new version of the whole
chapter, leave alone the other sections to focus on their own, and
grant access to it to their co-editors. At the end they just put
things back into the master project.  I understood that it should be
possible to grant rights to a CERN e-group, but I am not sure this is
working already, I'm waiting for confirmation.

You can work directly on the Overleaf version, or you can download the
whole package locally from the git repository. You can then use the
standard git tools to commit updated versions. Working locally is
almost essential at the beginning of the project, when you want to
create a complete subfolder structure to contain the various sections
of your report. Some scripts are available (see below) to faciliate
this task.

You must preserve the template
structure of the section.tex file in the directory of each
section. You should also make sure that all style files contained in
the main directory are present in the section directories. The
simplest way to create a new section is to issue the command
\begin{verbatim}
> make newsection newfolder=new_folder_name
\end{verbatim}
from the main directory. Here \verb|new_folder_name| is the name of the
directory you want to create to host the new section. A template section
file will be created there, together with the img and bib directories
(for figures and for the section.bib file, resp).
All necessary style files are copied over. Now you just need to input
the \verb|new_folder_name/section.tex| file into the main driver file,
report.tex, with the syntax:
\begin{verbatim}
\newpage
\subfile{\main/newdirectory/section}
\end{verbatim}
When editing the new \texttt{section.tex} file, please pay attention to the
header of the template. There are a few weirds commands, and an
apparently odd
\verb|\begin{document}|
statement. These are required to be able to compile the section as a
stand alone object. To this end, it is important that the command
\verb|\providecommand{\main}{..}|
points \verb|\main| to the directory containing the main driver file,
\texttt{report.tex}.
\subsection{CMS authors' list}
J.~Alimena$^1$, P.~Azzi$^2$, R.~Bhattacharya$^3$, E.~Boos$^4$, A.~Bragagnolo$^5$, L.~Cadamuro$^6$, F.R.~Cavallo$^7$, M.~Cepeda$^{8,9}$, K.F.~Chen$^{10}$, M.~Dinardo$^{11,12}$, P.~Dini$^{12}$, L.~Dudko$^4$, S.~Dutta$^3$, G.~Fedi$^{13}$, S.~Fiorendi$^{11,12}$, F.~Fiori$^{14}$, A.~Gilbert$^9$, L.~Guzzi$^{11,12}$, J.~Haller$^{15}$, K.~Hoepfner$^{16}$, J.M.~Hogan$^{17,18}$,
Sa.~Jain$^{19}$,
J.~Kieseler$^9$, T.~Klijnsma$^{20}$, R.~Kogler$^{15}$, A.~Korytov$^6$, U.~Langenegger$^{21}$, F.~Ligabue$^{14}$, S.~Malvezzi$^{12}$,
P.~Mandrik$^{22}$, P.~Matorras Cuevas$^{23}$, K.~Mazumdar$^{19}$, A.B.~Meyer$^{24}$, G.~Mitselmakher$^6$, M.~Narain$^{17}$, O.~Ozcelik$^{25,26}$, F.~Palla$^{13}$, K.~Pedro$^{27}$, M.~Perfilov$^4$, A.~Reimers$^{15}$, S.~Sarkar$^3$, A.~Savin$^{28}$, S.~Sawant$^{19}$, A.~Schmidt$^{16}$, B.~Schneider$^{27}$, K.~Skovpen$^{29}$, Y.~Takahashi$^{23}$, D~Tsiakkouri$^{30}$, R.~Venditti$^{31}$, P.~Volkov$^4$, G.~Vorotnikov$^4$, J.~Wang$^6$, W.~Zhang$^{17}$

\noindent
$^1$ The Ohio State University, Columbus, USA \\
$^2$ INFN Sezione di Padova, Padova ,Italy\\
$^3$ Saha Institute of Nuclear Physics, HBNI, Kolkata, India \\
$^4$ Skobeltsyn Institute of Nuclear Physics, Lomonosov Moscow State University, Moscow, Russia\\
$^5$ Universit\`{a} di Padova, Padova, Italy\\
$^6$ University of Florida, Gainesville, USA \\
$^7$ INFN Sezione di Bologna, Bologna, Italy\\
$^8$ Centro de Investigaciones Energ\'{e}ticas Medioambientales 
y Tecnol\'{o}gicas , Madrid, Spain\\
$^9$ CERN, European Organization for Nuclear Research, Geneva, Switzerland\\
$^{10}$ National Taiwan University (NTU), Taipei, Taiwan\\
$^{11}$ Universit\`{a} di Milano-Bicocca, Milano, Italy\\
$^{12}$ INFN Sezione di Milano-Bicocca, Milano, Italy\\
$^{13}$ INFN Sezione di Pisa, Pisa, Italy\\
$^{14}$ INFN Sezione di Pisa, Scuola Normale Superiore di Pisa, Pisa, Italy\\
$^{15}$ University of Hamburg, Hamburg, Germany\\
$^{16}$ RWTH Aachen University, III. Physikalisches Institut A, Aachen, Germany\\
$^{17}$ Brown University, Providence, and Bethel University, St. Paul, USA\\
$^{18}$ Bethel University, St. Paul, USA\\
$^{19}$ Tata Institute of Fundamental Research-B, Mumbai, India\\
$^{20}$ ETH Zurich - Institute for Particle Physics and 
Astrophysics (IPA), Zurich, Switzerland\\
$^{21}$ Paul Scherrer Institut, Villigen, Switzerland\\
$^{22}$ Institute for High Energy Physics of National Research Centre 'Kurchatov Institute', Protvino, Russia\\
$^{23}$ Universit\"{a}t Z\"{u}rich, Zurich, Switzerland\\
$^{24}$ Deutsches Elektronen-Synchrotron, Hamburg, Germany\\
$^{25}$ Bogazici University, Istanbul, Turkey\\
$^{26}$ Laborat\'{o}rio de Instrumenta\c{c}\~{a}o e F\'{i}sica Experimental de Part\'{i}culas, Lisboa, Portugal\\
$^{27}$ Fermi National Accelerator Laboratory, Batavia, USA\\
$^{28}$ University of Wisconsin - Madison, Madison, USA\\ 
$^{29}$ Vrije Universiteit Brussel, Brussel, Belgium\\
$^{30}$ University of Cyprus, Nicosia, Cyprus\\
$^{31}$ INFN Sezione di Bari, Bari, Italy\\

\end{document}
