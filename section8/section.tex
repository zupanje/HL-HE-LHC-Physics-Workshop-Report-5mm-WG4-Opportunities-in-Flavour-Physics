% mainfile: ../report.tex

% don't remove the folling lines, and edit the defintion of \main if needed
\documentclass[../report.tex]{subfiles}
\providecommand{\main}{..}
\IfEq{\jobname}{\currfilebase}{\AtEndDocument{\biblio}}{}
% until here
\begin{document}

%\section{Flavor aspects of top physics}
\section{The top quark and flavor physics}
\label{sec:top:quark}
{\bf Author(s): Wouter Dekens, Frederic Deliot, Gauthier Durieux, Mariel Estevez, Darius Faroughy, Jernej F. Kamenik, Fabio Maltoni,  Eleni Vryonidou, Cen Zhang} (10 pages,TH:EXP= 50\%:50\%)

Top quark is characteristic for its large mass and its ${\mathcal O}(1)$ coupling to the Higgs, quite distinct from any other SM fermion. Studying top quark  properties may shed light on the resolution of the SM flavour puzzle, or at least as to why one and only one Yukawa coupling is large. At the same time, the large coupling of top to the Higgs is also the reason for the weak scale hierarchy problem to be so acute -- the quadratic divergent corrections to the Higgs mass are driven almost completely by this coupling. BSM models addressing the hierarchy problem may thus well leave an imprint in the top quark properties and decays.  For instance, the top FCNCs, $t\to c\gamma, cZ, cg$ are null tests of the SM and are used as BSM probes.

Top quark also directly enters the flavour phenomenology. Loops with the top quark are 
responsible for the largest short-distance contributions to the down-quark FCNCs. 
The SM predictions are thus controlled by the flavor couplings of the top -- with $B$ and $K$ transitions determining the CKM matrix elements $V_{tb}, V_{ts}, V_{td}$ through this virtual effects. Determining these and the structure of the $Wti$ vertices  directly, from high $p_T$ transitions, can then serve as independent consistency checks of the SM.


\subsection{Determinations of $V_{tx}$} 
%Jernej Kamenik
%Mariel Estevez, 
%Darius Faroughy 
%Fabio Maltoni

%%%%%%%%%%%%%%%%%%%%%%%%%%%%%%%%%
%
\subsubsection{Measuring $|V_{td}|$ at (HL)LHC}
%Jernej Kamenik
%Mariel Estevez,
%
%%%%%%%%%%%%%%%%%%%%%%%%%%%%%%%%%

A possible experimental strategy to probe the $|V_{td}|$ matrix element directly at the (HL/HE)LHC is using single top production associated with a $W$ boson ($pp \rightarrow tW$). The idea is to exploit the production cross-section enhancement, as well as boosts of the top quarks coming from initial state valence $d$-partons. The $d$-quark is a valence constituent of the proton and there is an imbalance with the $\bar{d}$-quark that motivates to explore $V_{td}$-sensitive observables in the form of charge asymmetries.
The main backgrounds, contrary to $t$-channel single top production, are charge symmetric or have very small charge asymmetries. 
The $dg \to tW$ associated production process is interesting because of its sizeable charge asymmetry in proton collisions and also because its kinematics predicts a characteristic angular distribution. We expect a relatively large incoming momentum on average from the valence $d$-quark, so consequently a forward $W^-$ is preferred in the lab frame, which is supposed to produce a forward $\ell ^-$ in signal events. The main two backgrounds to this $\ell ^+ \ell ^- b E_T^{\rm{miss}}$ final state would be the dileptonic $t \bar{t}$ production (missing one of the b-jets from top decays) and $g b \to tW$ associated production proportional to $|V_{tb}|^2$. Both backgrounds have very small charge asymmetries. In order to increase the sensitivity and enhance the signal we impose cuts that reduce the cross-sections of the backgrounds. The details of all the considered backgrounds, the events selection, and simulations employed can be found in Ref.~\cite{Alvarez:2017ybk}. We observe that the most important difference comes from the $\eta (\ell ^-)$ distribution, where the signal clearly prefers forward negatively charged leptons. We propose the following asymmetry
\begin{equation}
A(\eta,p_T) = \frac{N^+ - N^-}{N^+ + N^-},
\label{A}
\end{equation}
where
\begin{equation}
N^\pm = N\left( \Delta |\eta(\ell)| \gtrless 0 \,\& \, \Delta p_T(\ell) \gtrless 0 \right),
\label{i-th contribution}
\end{equation}
as a $|V_{td}|$ sensitive observable, where each $i$-th process contributes with
\begin{equation}
N^+_i - N^-_i = \sigma_i \cdot {\cal A}_i \cdot \epsilon_i \cdot A_i(\eta,p_T),
\end{equation}
and the factors on the right hand side are the cross-section, acceptance, selection efficiency and asymmetry, respectively. 
%We parameterize eventual departures from SM in $V_{td}$ through the ratio
%\begin{equation}
%r \equiv \left| \frac{V_{td}}{V_{td}^{\rm SM}}\right|\,,
%\end{equation}
%in order to classify processes according to their leading power in $r$. 
To quantify the versatility of the proposed charge asymmetry we study the prospective experimental reach in $r\equiv |V_{td}/V_{td}^{\rm SM}|$ by computing the difference of $A(\eta, p_T)$ to its SM expectation in units of the uncertainty. Based on existing experimental studies of charge asymmetries in top production~\cite{Khachatryan:2016ysn} we include an estimation for the systematic uncertainty of $\Delta _{sys}=0.2 \%$. 
We define the significance as
\begin{equation}
\mbox{significance} = \frac{\left| A(\eta,p_T) - A(\eta,p_T)^{\rm SM}\right| }{\sqrt{(N^+ + N^-)^{-1} + \Delta_{\rm syst}^2}} .
\end{equation}

In Fig.~\ref{figure} we plot the contours of the expected significance in measuring $A(\eta,p_T)$ as a function of $r$ and the luminosity, for $13$ TeV and $27$ TeV LHC. As a rough estimation, we also show the significance for the case where the dominant $t \bar{t}$ background could be reduced by a further factor of $1/2$ using e.g. multivariate discrimination techniques as done in existing single top analyses~\cite{Chatrchyan:2014tua, Aad:2015eto, Aaboud:2016lpj} (in dashed lines). 
Values of $r<10$ could be directly accessible at the (HL)LHC@13TeV, improving the existing best direct constraint~\cite{Khachatryan:2014nda} by roughly a factor of three. In particular, the current direct bound on $|V_{td}|$ can be surpassed with the Run II LHC dataset, and with $3000\, {\rm fb}^{-1}$ it could be possible to probe $|V_{td}|\sim |V_{ts}^{\rm SM}| \simeq 0.04$. 

The dominant $t \bar{t}$ background is mostly generated via gluon fusion, which is a charge symmetric process. There are subleading contributions from processes like $u \bar{u} \rightarrow t \bar{t}$ which are charge asymmetric but only at higher orders in QCD. This is the main reason the dominant $t \bar{t}$ background has a strongly suppressed charge asymmetry. At higher collision energies, the probability of finding energetic-enough gluons in the proton increases faster than that of valence quarks. Consequently the fraction of $t\bar t$ events from the quark-antiquark initial state is reduced~\cite{Khachatryan:2016ysn}.  This leads to a shrinking charge asymmetry with growing collision energy. %A further significant reduction in systematic uncertainties below our current estimate of $0.2 \%$ could allow the (HL)LHC eventually to probe values as low as $|V_{td}| \sim 0.06$ at $13$ TeV. 
Unfortunately, the same happens to the signal, this time because at higher energies the asymmetry between $d$ and $\bar{d}$ partons inside the proton is reduced. Combined with the growing contribution of the (mostly charge symmetric) $t\bar t$ background to the denominator, the end effect is severely diminishing the end significance as observed at $27$\,TeV compared to 13\,TeV at comparable luminosities. 
%On the other hand, at $7$ TeV the significance does increase and with $3000 fb^{-1}$, values of $r<10$ could be accesible with $3 \sigma$ and values of $r \sim 10$ with $5 \sigma$.
\begin{figure}[t]
\centering
\includegraphics[width=0.45\textwidth]{section8/figs/contour_r-L.pdf}
\hspace{7mm}
\includegraphics[width=0.45\textwidth]{section8/figs/contour_r-L_27.pdf}
%\hspace{3mm}
\caption{Contour lines for the projected $2\sigma$, $3\sigma$ and $5\sigma$ upper bounds on $|V_{td}|$, parameterized as a function of $r\equiv|V_{td}/V_{td}^{\rm SM}|$ and the LHC luminosity. The $13$ TeV energy results are shown on the left hand side and $27$ TeV on the right. Dashed lines correspond to the estimation of the same analysis assuming a further reduction of the dominant $t\bar t$ background by half. See text for details.} 
\label{figure}
\end{figure}

%%%%%%%%%%%%%%%%%%%%%%%%%%%%%%%%%
%
\subsubsection{Measuring Cabibbo-suppressed decays of the top quark at (HL, HE)LHC}
%Jernej Kamenik
%Darius Faroughy,
%
%%%%%%%%%%%%%%%%%%%%%%%%%%%%%%%%%

	With the large $t\bar t$ statistics at the LHC, one might attempt a direct measurement of Cabibbo-suppressed decays of the top quark, $t\to (s,d)\,W^\pm$ in leptonic $t\bar t$ events. Given that there is no practical way of distinguishing between strange and down quark jets at the detector level (without dedictated PID systems), the method described here could only provide a direct measurement of $\sqrt{|V_{ts}|^2+|V_{td}|^2}$ by measuring the ratio of branching fractions
\begin{equation}
\rho\equiv\sqrt{\frac{\mathcal{B}(t\to sW)+\mathcal{B}(t\to dW)}{\mathcal{B}(t\to bW)}}\,.
\end{equation}
This quantity measures the hierarchy between diagonal and off-diagonal entries in the third row of the CKM matrix and is supposed to be approximately given by $\rho\approx |V_{ts}|\approx0.04$ in the SM. 

In order to measure top decays into light quarks, it is necessary to discriminate between heavy-flavor jets, e.g. $b$-jets arising from the $t\to b W$ background, gluons from ISR/FSR contamination, and the signal light-quark jets (denoted in the following as {\it $q$-jets}) from the Cabibbo-suppressed top decays. This can be achieved through a {\it $q$-tagger} based on existing techniques used for $b$-tagging and quark/gluon jet discrimination~\cite{Gallicchio:2011xq}. There are several known useful observables which can be used in a $q$-tagger. In the following we use: (i) The multiplicity $N_{\mathrm{SV}}$ of secondary vertices (SV) in the jet within a fiducial volume of the tracker, (ii) the fraction of longitudinal momentum of the hardest prompt charged track (i.e. originating from the primary vertex of the collision) with respect to the total momentum of the jet 
\begin{equation}\label{zmax}
 z_{\mathrm{max}}\equiv\mathrm{max}\Big[\frac{\vec{p}_{x}\cdot\vec{p}_{\text{jet}}}{|\vec{p}_\text{jet}|^2}\Big]_{x\in\mathrm{jet}}\,,
\end{equation}
$x$ being a reconstructed prompt track and (iii) the 2-point energy correlation function \cite{Larkoski:2013eya,Moult:2016cvt} defined as
\begin{align}
U^\beta_1=\sum_{i,j\in \mathrm{jet}} z_T^i z_T^j\,(R_{ij})^\beta\ ,\ \ z_T^i\equiv p_T^i /p_T^{\mathrm{jet}}\,,
\end{align}
where $R_{ij}^2=(\eta_i-\eta_j)^2+(\phi_i-\phi_j)^2$ and $\beta$ is a free real parameter for which quark/gluon discrimination is optimized at $\beta=0.2$. For more details on these observables in the context of the $q$-tagger we refer the reader to~Ref.\cite{thePaper}. The resulting performance of the $q$-tagger in simulated samples of $pp\to Zj$ for $j=s,b,c,g$ and $Z\to\mu\bar\mu$ in a realistic LHC environment at $13$\,TeV is presented in figure~\ref{figROC}, where we show the ROC curves for the $q$-tagger obtained by sliding the values of the cuts $z_{\mathrm{max}}$ (solid lines) or $U_1$ (dashed lines), independently. 
\begin{figure}[t]
 % \vspace{-0.3cm}
\begin{center}{
\includegraphics[width=8cm]{section8/figs/quark-gluon_ROC.pdf}
%  \vspace{-0.3cm}
\caption{{ ROC curves for the two $q$-taggers}} \label{figROC}}
\end{center}
\end{figure}
%\paragraph{light-quark content in top decays}
%Equipped with the light-quark tagger we are now in position to study the flavor content of top quark decays from $t\bar t$ production. 
A reference working point for the utilized taggers is shown in Table~\ref{tab:effs}.
%%%%%%%%%%%%%%%%%%%%%
%\begin{table}[htbp!]
\begin{table}[t!]
\begin{center}
% \small
 {
%\renewcommand{\arraystretch}{1}
\centering
%\footnotesize
\begin{tabular}{cccccccc}
\hline\hline 
(t) type & Cuts &\  $\epsilon^{\mathrm t}_{q}$\ &\ $\epsilon^{\mathrm t}_{b}$\ &\ $\epsilon^{\mathrm t}_{c}$\ &\ $\epsilon^{\mathrm t}_{g}$
\\ 
\hline
$q$-tagger &$N_{\mathrm SV}=0$ \& $z_{\mathrm max}>0.3$    &  $0.18$ & $0.0031$ & $0.038$ & $ 0.049$ \\
$b$-tagger & $N_{\mathrm SV}>3$     		                          &  $0.0091$ & $0.64$ & $0.09$ & $0.016$ \\ 
\hline\hline
\end{tabular}
}
\end{center}
\caption{ Tagging and mis-tagging efficiencies for the $q$-tagging working point used in the analysis.}
\label{tab:effs} 
\end{table}

	%First we generated MC samples of $t\bar t$ and $tW$ plus jets in {\tt Madgraph 5} at LO and decayed the tops into both $t\to bW$ and $t\to (s,d)W$ channels with the W-boson decaying leptonically and SM values for the CKM matrix. Jet matching, parton showering and hadronization were performed in {\tt pythia8} and jet clustering with {\tt FastJet}. 
	
	A possible extraction of $\rho$ using leptonic $t\bar t$ events then proceeds as follows.
	After pre-selection requirements, two selected highest $p_T$ jets in an event are subjected to the $b$- and $q$-taggers (for the working points in Table \ref{tab:effs}). The results are binned into one of the six tagged dijet categories $\{\jmath\jmath,\jmath b,\jmath q, qb,qq,bb\}$ where $q$, $b$ and $\jmath$ represent $q\,$-jets, $b\,$-jets and non-tagged jets (i.e. a jet failing both taggers), respectively. 
%	A fit to event yields in all the categories can then be used to extract $\rho$ together with the ISR/FSR contamination fractions. 
	Solely for the purpose of illustration we present  in Figure \ref{fig:master} the hypothetical upper limits on $\rho$ taking into account only statistical uncertainties in the $qb$ category for the signal significance $S/\sqrt{B}$ at $2\sigma$ (solid boundary) and $5\sigma$ (dashed boundary) as a function of the LHC luminosity, compared to the current best limit by CMS shown by the gray dashed curve. These results suggest that the HL-LHC could find evidence of Cabbibo-suppressed top decays and provide an additional experimental handle for the direct determination of the $V_{tx}$ CKM elements. %For a more detailed study of these prospects we refer the reader to Ref.~\cite{thePaper}


\begin{figure}[h!]
%  \vspace{-0.3cm}
\begin{center}{
\includegraphics[width=8cm]{section8/figs/LHC_Resolved_lep_tops.pdf} 
%\vspace{-0.3cm}
\caption{{Illustration of statistical limits on the ratio $\rho \simeq \sqrt{|V_{ts}|^2 + |V_{td}|^2}/|V_{tb}|$ from an analysis of Cabbibo-suppressed top decays in leptonic $t\bar t$ events at 13TeV LHC. See text for details.}} \label{fig:master}}
\end{center}
\end{figure}


%\paragraph{Outlook}
For more reliable (HL/HE-)LHC projections it is necessary to carefully categorize and estimate the main sources of systematic uncertainties. The dominating source of uncertainties is expected to come from the extraction and mapping of the $q$- and $b$-tagging efficiencies from $Zj$ production onto $t\bar t$ events. A complete assessment of these uncertainties is outside the scope of this report. Nevertheless, once these uncertainties are extracted from experimental data the following step is to perform a fit of the categorized dijet data $\{\jmath\jmath,\jmath b,\jmath q, qb,qq,bb\}$ in inclusive dilepton events to a probabilistic model taking into account the tagging efficiencies, similar to that performed in \cite{Silva:2010qt,Khachatryan:2014nda} for the extraction of $V_{tb}$. For further details on this search strategy see Ref.~\cite{thePaper}.

Another interesting direction, specially relevant for the HE-LHC, is to measure $t\to (s,d)W$ from boosted semi-leptonic $t\bar t$ events. Events with one top-tagged fat jet, one narrow jet and one lepton (both daughter candidates of the leptonic top) can be categorized by the flavor content of the narrow jet. A preliminary analysis~\cite{thePaper} suggests comparable sensitivity to the leptonic $t\bar t$ dataset already at the 13TeV LHC. As a final note, given the small signal-to-background ratio of Cabibbo-suppressed top decays, advanced machine learning approaches to data analysis should be able to further improve upon these simple search strategies.


\subsection{Determinations of anomalous $Wtb$ vertices including CP-violation effects from $T$-odd kinematic distributions}
% Frederic Deliot
% Wouter Dekens

Beyond the SM contributions to the $Wtb$ vertex have been widely studied in the context of various processes in the literature  \cite{Grzadkowski:2008mf,Drobnak:2010ej,AguilarSaavedra:2008zc,GonzalezSprinberg:2011kx,Drobnak:2011aa,Cao:2015doa,Hioki:2015env,Schulze:2016qas,Kamenik:2011dk,Zhang:2012cd,deBlas:2015aea,Buckley:2015nca,Buckley:2015lku,Bylund:2016phk, Castro:2016jjv,Deliot:2017byp}.
%
Assuming that new physics arises at a scale well above the electroweak scale, these corrections to the $Wtb$ vertex can be described within an effective field theory. The first contributions then appear at dimension six and, in the conventions of \cite{AguilarSaavedra:2018nen}, take the following form
\begin{align}\label{eq:Wtb}
\mathcal L_{tb} =-\bar t\left[ \frac{g}{\sqrt{2}}\,\gamma^\mu (V_{tb}(1+\frac{v^2}{2}c_{\varphi q}^{-})P_L+\frac{v^2}{2} C_{\varphi tb}P_R)W_\mu^+-v\sigma^{\mu\nu}W_{\mu\nu}^+\left(C_{bW}P_R+C_{tW}^*P_L\right)\right]b+{\rm h.c.}\,\,,
\end{align}
where $C_i = (c_{i}+ic_i^{ [I]})/\Lambda^2$.~\footnote{Here we followed Ref.~\cite{Grzadkowski:2008mf} and assumed that no flavor-changing neutral currents are induced at tree level, which implies $C_{\varphi q}^{1(33)}+C_{\varphi q}^{3(33)} = 0$ such that only the (real) combination $c_{\varphi q}^-=C_{\varphi q}^{1(33)}-C_{\varphi q}^{3(33)} = 0$ appears in the Eq.\ \eqref{eq:Wtb}.}
%
These Wilson coefficients are related to the often-used anomalous couplings \cite{AguilarSaavedra:2008zc,AguilarSaavedra:2006fy} as follows
\begin{align}
V_L = V_{tb}^*(1+\frac{v^2}{2}c_{\varphi q}^{-}),\qquad V_R = \frac{v^2}{2} C_{\varphi tb}^*,\qquad g_L = -\sqrt{2}v^2 C_{bW}^*,\qquad g_R = -\sqrt{2}v^2 C_{tW}\,.
\end{align}
It should be mentioned that, within the SM-EFT, gauge invariance links the above vertices to additional interactions, the full forms of the relevant operators are given in Sec.~8.3.

The above interactions give tree-level contributions to single-top production and observables in $t\to W^+b$, in particular, the $W$-boson helicity fractions. Apart from these processes that are `directly' sensitive the $Wtb$ vertices, there are `indirect' observables that receive contributions from the same operators through loop diagrams; examples are $h\to \bar bb$, $B\to X_s\gamma$, and searches for electric dipole moments (EDMs).
Here we briefly discuss both the direct and indirect limits on the operators in Eq.\ \eqref{eq:Wtb}, as well as the projections for the HE/HL-LHC.

\subsubsection{Direct probes}
%

\begin{table}\label{tab:directWtb}
\center
\caption{$95\%$ CL constraints from direct observables on the real and imaginary parts of the Wilson coefficients that contribute to the $Wtb$ vertex. The current  and projected limits are shown in the left and right panel, respectively. We assumed $\Lambda = 1$ TeV and $c_{\varphi q}^-=0$ in both panels.}
\vspace{.2cm}
\begin{tabular}{lccclccc}
\hline\hline
Current &$c^{}_{tW}$& $c^{}_{bW}$& $c^{}_{\varphi tb}$ & ~~~~$3000$ fb$^{-1}$ &$c^{}_{tW}$& $c^{}_{bW}$& $c^{}_{\varphi tb}$\\\hline
Re$(\ldots)$ &$[-0.70,\,0.82]$ &$[-2.2,\,2.2]$ &$[-3.9,\,3.2]$ & Re$(\ldots)$ &
\\
Im$(\ldots)$ &$[-1.5,\,2.2]$ &$[-2.2,\,2.1]$ &$[-3.5,\,3.5]$ & Im$(\ldots)$ &
%\end{tabular} \hfill
%\begin{tabular}{lccc}
\\
\hline\hline
\end{tabular}
\end{table}
%\begin{table}\label{tab:directWtb}
%\center\small
%\caption{$95\%$ CL constraints from direct observables on the real and imaginary parts of the Wilson coefficients that contribute to the $Wtb$ vertex. The current  and projected limits are shown in the left and right panel, respectively. We assumed $\Lambda = 1$ TeV and $c_{\varphi q}^-=0$ in both panels.}
%\vspace{.2cm}
%\begin{tabular}{l|ccc}
% &$c^{}_{tW}$& $c^{}_{bW}$& $c^{}_{\varphi tb}$\\\hline
%Re &$[-0.70,\,0.82]$ &$[-2.2,\,2.2]$ &$[-3.9,\,3.2]$\\
%Im &$[-1.5,\,2.2]$ &$[-2.2,\,2.1]$ &$[-3.5,\,3.5]$
%\end{tabular} \hfill
%\begin{tabular}{l|ccc}
%$3000$ fb$^{-1}$ &$c^{}_{tW}$& $c^{}_{bW}$& $c^{}_{\varphi tb}$\\\hline
%Re &\\
%Im &
%\end{tabular}
%\end{table}
%
As mentioned above, the $Wtb$ interactions can be probed directly in single-top production and the $W$ boson helicity fractions in top decays. Furthermore, in the case of polarized top quarks, it is possible to construct additional observables that are sensitive to the $Wtb$ couplings, and in particular to their $T$-odd effects. Here we briefly discuss these observables and the resulting (projected) constraints.

\smallskip
{\bf Single top production~~}
The single-top production cross section has been measured in the $t$ and $s$ channels at the LHC at $\sqrt{s} = 7$, $8$, and $13$ TeV \cite{Aad:2014fwa,Aaboud:2016ymp,Aaboud:2017pdi,Chatrchyan:2012ep,Khachatryan:2014iya,Sirunyan:2016cdg}. These cross sections in principle receive contributions from all $Wtb$ couplings and the measurements can be compared with the SM prediction \cite{Brucherseifer:2014ama} at NNLO in QCD, while BSM contributions have been evaluated at NLO in \cite{Alioli:2017ce,Cirigliano:2016nyn}.
%
It should be mentioned that the single-top cross sections in principle receive contributions from other operators in the SM EFT, in particular from several four-quark operators. Thus, in order to perform an exhaustive global analysis one would have to include these additional as well. For discussions on such operators in the context of single-top production, see e.g.\ Refs.\ \cite{Buckley:2015lku,Buckley:2015nca}.

\smallskip
{\bf Top quark decays~~}
The helicity fractions of the $W$ boson in top decays are mostly sensitive to $c_{tW}$, $c_{bW}$, and $c_{\varphi tb}$, and have been measured at both the Tevatron and the LHC \cite{Aaltonen:2012rz,Aad:2012ky,Chatrchyan:2013jna,Aad:2015yem,Khachatryan:2014vma,Aaboud:2016hsq}. In addition, the phase $\delta^-$ between the amplitudes of longitudinally and transversely polarized $W$ bosons recoiling against a left-handed $b$ quark carries information on the imaginary part of $C_{tW}$ \cite{Aad:2015yem,Boudreau:2013yna}.
The SM predictions for the helicity fractions are known to NNLO in QCD \cite{Czarnecki:2010gb}, while the BSM contributions have been computed to NLO \cite{Drobnak:2010ej}.  

As mentioned above, in the case of polarized top-quark decays, it becomes possible to construct additional observables that are sensitive to both the real and imaginary parts of the $Wtb$ couplings. In particular, both the asymmetries constructed in Ref.\ \cite{Aguilar-Saavedra:2015yza,AguilarSaavedra:2010nx} and the triple-differential measurements of Ref.\ \cite{Aaboud:2017yqf} are sensitive to $c_{tW}^{[I]}$. As a result, the limits on $c_{tW}^{[I]}$ already improve noticeably when including current experimental measurements of these angular asymmetries \cite{Deliot:2017byp}.
\newline

After taking into account the experimental results on single-top production, the helicity fraction, and angular asymmetries one obtains the limits in Table \ref{tab:current_limits} \cite{Deliot:2017byp}. 


\subsubsection{Indirect probes}
%
\begin{table}\label{tabindirectWtb}
\center\small
\caption{Indicative constraints on the SM-EFT operators contributing to the $Wtb$ vertex from indirect observables. A single Wilson coefficient was turned on at a time and we assumed $\Lambda = 1$ TeV.}
\vspace{.2cm}
\begin{tabular}{l|llll}
&$h\to \bar bb$ & EWPT&$B\to X_s\gamma$ &EDMs\\\hline
$c^{-}_{\varphi q}$&-&-&$[-2.0,2.2]$ &- \\
$c^{}_{\varphi tb}$&$[-0.3,4.3]$ & -&$[-4.6,4.9]\cdot 10^{-2}$&-\\
$c^{}_{tW}$&-&$[-1.1,0.7]$&$[-0.22,0.89]$&  -\\
 $c^{}_{bW}$ &-&-& $[-13,3.5]\cdot 10^{-3}$& -\\\hline
$c^{[I]}_{\varphi tb}$&$[-7.3,7.3]$ &-&$[-0.20,0.20]$&$[-0.019,0.019]$ \\
$c^{[I]}_{tW}$&-&-& $[-2.4, 4.5]$ & $[-8.1,8.1]\cdot 10^{-3}$\\
 $c^{[I]}_{bW}$ &-&-& $[-4.3, 2.3] \cdot 10^{-2}$& $[-5.5,5.5] \cdot 10^{-4}$\\
\end{tabular}
\end{table}
%
As mentioned above, the anomalous $Wtb$ interactions can contribute to processes without top quarks through loop diagrams.  Although this can give rise to stringent constraints, their interpretation requires some care. As indirect observables receive contributions from additional operators apart from the $Wtb$ couplings, cancellations between different Wilson coefficients are possible. 
We will nevertheless consider the limits in the case that a single dimension-six operator is generated at the BSM scale, $\Lambda$, but note that these limits could be weakened significantly if this assumption is relaxed.
%
Here we give a short summary of the most relevant observables that are induced in this way, starting with EDMs.

\smallskip
{\bf Electric dipole moments~~}
Electric dipole moments are probes of CP violation and therefore receive contributions from the imaginary parts of the $Wtb$ couplings. 
The most stringent experimental limits have been obtained on the EDMs of the neutron, mercury, and the ThO molecule, the latter of which can be interpreted as a limit on the electron EDM for our purposes.
To obtain the contributions to EDMs one first has to evolve the $Wtb$ operators to low energies, $\mu\sim 2$ GeV. QCD becomes non-perturbative below this scale, and one has to match to Chiral perturbation theory, which describes the CP-odd interactions in terms of the low-energy degrees of freedom (pions, nucleons, photons, and electrons). These interaction can then be used to calculate the EDMs of nucleons, atoms, and molecules.

Of the operators in Eq.\ \eqref{eq:Wtb}, the $c_{\varphi tb}^{[I]}$ and $c_{bW}^{[I]}$ couplings are mainly constrained by the neutron EDM, while $c_{ tW}^{[I]}$ contributes to the electron EDM. The contributions to the neutron EDM arise through the Weinberg three-gluon operator, $O_{\tilde G}$, and the bottom-quark chromo EDM, $O_{dG}^{(33)}$, in the notation of \cite{Grzadkowski:2008mf}. In the first step $c_{\varphi tb}^{[I]}$ and $c_{bW}^{[I]}$  generate the bottom-quark chromo EDM through one-loop diagrams \cite{Dekens:2013zca,Alonso:2013hga,Alioli:2017ce}, which induces the Weinberg operators ater integrating out the bottom quark \cite{Dicus:1989va,Weinberg:1989dx,BraatenPRL,Boyd:1990bx}. The hadronic matrix element of the contribution of the Weinberg operator to the neutron EDM is poorly known. Combining naive-dimensional-analysis and sum-rule estimates \cite{Pospelov_Weinberg,Weinberg:1989dx,deVries:2010ah} one has $ |d_n|  = 6|e\,(50\, {\rm MeV})\, C_{\tilde G}(1\, {\rm GeV})|$, with an $\mathcal O(100\%)$ uncertainty. 

The contributions to the electron EDM also arise from a two-step process: $c_{ tW}^{[I]}$ first induces CP-odd operators of the form $X_{\mu\nu }\tilde X^{\mu\nu }\varphi^\dagger\varphi$, with $X_{\mu\nu}$ a $SU(2)_L$ or $U(1)_Y$ fieldstrength, as well as semileptonic interactions of the form, $(\bar e_L \sigma_{\mu\nu}\,e_R) \, (\bar t_L\sigma^{\mu\nu}t_R)$, through the renormalization group equations \cite{Cirigliano:2016nyn,Cirigliano:2016njn,Fuyuto:2017xup}. These additional operators then induce the electron EDM at one loop. 
%
Using the above contributions and the current experimental limits \cite{Afach:2015sja,Baker:2006ts,Baron:2013eja}, we obtain the constraints in Table \ref{tabindirectWtb}.

\smallskip
{\bf Rare B decays~~}
Unlike EDMs, measurements of $B\to X_s\gamma$ are sensitive to both the real imaginary parts of the couplings. Although all four of the $Wtb$ vertices give rise to  flavor-changing $b\to s$ transitions through one-loop diagrams \cite{Grzadkowski:2008mf,Aebischer:2015fzz}, the largest effects are due to $C_{\varphi tb}$ and $C_{bW}$. Both of these couplings induce contributions proportional to $m_t$ instead of a factor of $m_b$ that appears in SM (as well as for $c^-_{\varphi q}$ and $C_{tW}$), leading to a relative enhancement of $m_t/m_b$. Here we consider the constraints from measurements of the branching ratio and the CP asymmetry \cite{Amhis:2016xyh}, for which we use the theoretical expressions of \cite{Hurth:2003dk} and \cite{Benzke:2010tq}, respectively. This leads to the constraints  in Table \ref{tab:indirectWtb}.

\smallskip
{\bf Electroweak precision tests~~}
The $Wtb$ operators also modify the self energies of the SM gauge bosons through one-loop diagrams, which are often parametrized by the $S$, $T$, and $U$ parameters \cite{Peskin:1990zt,Peskin:1991sw,Barbieri:2004qk}.
Taking into account the RGE contributions, only $c_{tW}$ induces the $S$ parameter by mixing with the $O_{\varphi WB}$ operator ($c_{bW}$ gives rise to contributions that are proportional to $m_b$), which gives rise to the limit in Table \ref{tab:indirectWtb}. Here we assumed only a single dimension-six operator is present at $\mu = \Lambda$, however, it is possible to perform a more conservative analysis in this case. This can be done by including the couplings of the operators that induce $S$ and $T$ at tree level, $O_{\varphi WB}$ and $O_{HD}$, and marginalizing over them. This is made possible by the fact that the electroweak precision observables carry more information than is captured by the $S$, $T$, and $U$ parameters alone. This approach is discussed Ref.\ \cite{Greiner:2011tt,Zhang:2012cd} and gives rise to weaker limits, $c_{tW}\in [-1.6,0.8]$, $c_{bW}\in [-2,24]$, $c_{\varphi q}^-\in [-0.7,4.7]$ for $\Lambda= 1\,{\rm TeV}$, which, however, they are based on more conservative assumptions than those shown in Table \ref{tab:indirectWtb}.


\smallskip
{\bf Higgs decays~~}
Finally, the $Wtb$ interactions contribute to the process $h\to \bar bb$ by inducing corrections to the SM bottom-quark Yukawa coupling \cite{Alioli:2017ce}. In particular, the $C_{\varphi tb}$ coupling generates a contribution to the Yukawa coupling that scales as $y_b\sim \frac{y_t}{(4\pi)^2} v^2C_{\varphi tb}$. Thus, although this contribution only appears at one loop, the suppression is offset by the appearance of the top-quark Yukawa instead of that of the bottom quark. Using the combined ATLAS and CMS analysis \cite{Khachatryan:2016vau} of the Higgs decays to $\gamma\gamma$, $WW$, $ZZ$, $\tau\tau$, $\mu\mu$, and $\bar bb$ signal strengths we obtain the limits in Table \ref{tab:indirectWtb}.

\subsection{Global effective-field-theory interpretation of top-quark FCNCs}
%Cen Zhang, Gauthier Durieux
{
\newcommand{\hc}[1]{#1}%{}^\ddagger #1}%{\hspace{+2mm}\raisebox{.5mm}{$\cdot$}\hspace{-2mm}#1}}

\newcommand{\ifb}{\ensuremath{\text{fb}^{-1}}}
\newcommand{\iab}{\ensuremath{\text{ab}^{-1}}}

\newcommand{\FDF}{(\varphi^\dagger i\!\!\overleftrightarrow{D}_\mu\varphi)}
\newcommand{\FDFI}{(\varphi^\dagger i\!\!\overleftrightarrow{D}^I_\mu\varphi)}

\newcommand*{\tmp}[4]{\ensuremath{%
	{#4%
	\ifx\empty#3\empty\ifx\empty#1\empty\else^{#1}\fi\else^{#1(#3)}\fi%
	\ifx\empty#2\empty\else_{#2}\fi}}}%
\newcommand{\qq }[4][]{\tmp{#2}{#3}{#4}{#1{O}}}
%\newcommand*{\cc }[4][]{\tmp{#2}{#3}{#4}{#1{C}}}
\newcommand*{\ccc}[4][]{\tmp{#2}{#3}{#4}{#1{c}}}

\newcommand{\ReIm}{{}_{\Re}^{[\Im]}\!}
\newcommand*{\sw}{s_W}
\newcommand*{\cw}{c_W}

Following the methodology employed in Ref.\,\cite{Durieux:2014xla}, we provide an updated global effective-field-theory interpretation of current constraints on top-quark FCNCs before using the HL-LHC prospects presented in Sec.~\ref{WG1:top_fcnc}\textcolor{red}{[top fcnc experimental section]}.

\subsubsection{Effective operators}

The effective field theory (EFT) approach~\cite{Weinberg:1978kz,
Weinberg:1980wa, Georgi:1994qn} is a powerful tool for parametrizing physics beyond
the SM. In a low-energy limit, it incorporates all possible deviations from the SM in a
model-independent way, and in the meantime allows to consistently take into
account higher-order radiative corrections. We first introduce the EFT
description of top-quark FCNC processes, and present all relevant effective
operators.

Assuming the full $SU(3)_C\times SU(2)_L\times U(1)_Y$ gauge symmetry and matter content of the SM,
one can show that odd-dimensional operators all violate baryon or lepton 
numbers~\cite{Degrande:2012wf}. Imposing the conservation of these quantum numbers, the leading higher dimensional operators of the SM arise at dimension six.
We follow the top-quark EFT standards set by the LHC TOP WG in 
Ref.~\cite{AguilarSaavedra:2018nen}. They employ the linear
combinations of Warsaw-basis operators~\cite{Grzadkowski:2010es} which appear in
interactions with physical fields after electroweak symmetry breaking.

Among the operators contributing to top-quark FCNC processes,
several categories can be distinguished. We consider operators involving
exactly two quarks, as well as those involving two quarks and two leptons.
Operators containing four quarks only start contributing at next-to-leading order in QCD in most of the measurements we consider ($pp\to tj$ is the exception).
The corresponding Warsaw-basis operators have been listed in Eq.\,(12-20) and (21-28) of Ref.~\cite{AguilarSaavedra:2018nen}.
They are:
\begin{equation}
	\begin{aligned}
	\hc{\qq{}{u\varphi}{ij}}
	&=\bar{q}_i u_j\tilde\varphi\: (\varphi^{\dagger}\varphi)
	,\\
	\qq{1}{\varphi q}{ij}
	&=\FDF (\bar{q}_i\gamma^\mu q_j)
	,\\
	\qq{3}{\varphi q}{ij}
	&=\FDFI (\bar{q}_i\gamma^\mu\tau^I q_j)
	,\\
	\qq{}{\varphi u}{ij}
	&=\FDF (\bar{u}_i\gamma^\mu u_j)
	,\\
	\hc{\qq{}{\varphi ud}{ij}}
	&=(\tilde\varphi^\dagger iD_\mu\varphi)
	  (\bar{u}_i\gamma^\mu d_j)
	,\\
	\hc{\qq{}{uW}{ij}}
	&=(\bar{q}_i\sigma^{\mu\nu}\tau^Iu_j)\:\tilde{\varphi}W_{\mu\nu}^I
	,\\
	\hc{\qq{}{dW}{ij}}
	&=(\bar{q}_i\sigma^{\mu\nu}\tau^Id_j)\:{\varphi} W_{\mu\nu}^I
	,\\
	\hc{\qq{}{uB}{ij}}
	&=(\bar{q}_i\sigma^{\mu\nu} u_j)\quad\:\tilde{\varphi}B_{\mu\nu}
	,\\
	\hc{\qq{}{uG}{ij}}
	&=(\bar{q}_i\sigma^{\mu\nu}T^Au_j)\:\tilde{\varphi}G_{\mu\nu}^A
\end{aligned}\quad
\begin{aligned}
	\qq{1}{lq}{ijkl}
	&=(\bar l_i\gamma^\mu l_j)
	  (\bar q_k\gamma^\mu q_l)
	,\\
	\qq{3}{lq}{ijkl}
	&=(\bar l_i\gamma^\mu \tau^I l_j)
	  (\bar q_k\gamma^\mu \tau^I q_l)
	,\\
	\qq{}{lu}{ijkl}
	&=(\bar l_i\gamma^\mu l_j)
	  (\bar u_k\gamma^\mu u_l)
	,\\
	\qq{}{eq}{ijkl}
	&=(\bar e_i\gamma^\mu e_j)
	  (\bar q_k\gamma^\mu q_l)
	,\\
	\qq{}{eu}{ijkl}
	&=(\bar e_i\gamma^\mu e_j)
	  (\bar u_k\gamma^\mu u_l)
	,\\
	\hc{\qq{1}{lequ}{ijkl}}
	&=(\bar l_i e_j)\;\varepsilon\;
	  (\bar q_k u_l)
	,\\
	\hc{\qq{3}{lequ}{ijkl}}
	&=(\bar l_i \sigma^{\mu\nu} e_j)\;\varepsilon\;
	  (\bar q_k \sigma_{\mu\nu} u_l)
	,\\
	\hc{\qq{}{ledq}{ijkl}}
	&=(\bar l_i e_j)
	  (\bar d_k q_l)
  \end{aligned}
\end{equation}
Note however that the $\qq{}{\varphi ud}{}$, $\qq{}{dW}{}$ and $\qq{}{ledq}{}$ operators only contribute to charged top-quark currents and are therefore not relevant for our purpose.
The EFT degrees of freedom appearing in top-quark FCNC processes were defined in Appendices~E.1-2 of Ref.\,\cite{AguilarSaavedra:2018nen}. They are:
\begin{gather}
\begin{aligned}
\ccc{[I]}{t\varphi}{3a} &\equiv \ReIm\{\cc{}{u\varphi}{3a}\},\\
\ccc{[I]}{t\varphi}{a3} &\equiv \ReIm\{\cc{}{u\varphi}{a3}\},\\[2mm]
\ccc{-[I]}{\varphi q}{3+a}  &\equiv \ReIm\{\cc{1}{\varphi q}{3a}-\cc{3}{\varphi q}{3a}\},\\
%\ccc{3[I]}{\varphi q}{3+a}  &\equiv \ReIm\{\cc{3}{\varphi q}{3a}\},\\
\ccc{[I]}{\varphi u}{3+a}  &\equiv \ReIm\{\cc{}{\varphi u}{3a}\},\\
%\ccc{[I]}{\varphi ud}{3a} &\equiv \ReIm\{\cc{}{\varphi ud}{3a}\},\\
%\ccc{[I]}{\varphi ud}{a3} &\equiv \ReIm\{\cc{}{\varphi ud}{a3}\},\\
\end{aligned}
\qquad
\begin{aligned}
\ccc{[I]}{uA}{3a} &\equiv \ReIm\{\cw\cc{}{uB}{3a}+\sw\cc{}{uW}{3a}\},\\
\ccc{[I]}{uA}{a3} &\equiv \ReIm\{\cw\cc{}{uB}{a3}+\sw\cc{}{uW}{a3}\},\\
\ccc{[I]}{uZ}{3a} &\equiv \ReIm\{-\sw\cc{}{uB}{3a}+\cw\cc{}{uW}{3a}\},\\
\ccc{[I]}{uZ}{a3} &\equiv \ReIm\{-\sw\cc{}{uB}{a3}+\cw\cc{}{uW}{a3}\},\\
%\ccc{[I]}{dW}{3a} &\equiv \ReIm\{\cc{}{dW}{3a}\},\\
%\ccc{[I]}{dW}{a3} &\equiv \ReIm\{\cc{}{dW}{a3}\},\\
\ccc{[I]}{uG}{3a} &\equiv \ReIm\{\cc{}{uG}{3a}\},\\
\ccc{[I]}{uG}{a3} &\equiv \ReIm\{\cc{}{uG}{a3}\},\\
\end{aligned}
\\
\begin{aligned}
%\ccc{3[I]}{lq}{\ell,3+a} &\equiv \ReIm\{\cc{3}{lq}{\ell\ell 3a}\},\\
\ccc{-[I]}{lq}{\ell,3+a} &\equiv \ReIm\{\cc{-}{lq}{\ell\ell 3a}\},\\
\ccc{[I]}{eq}{\ell,3+a} &\equiv \ReIm\{\cc{}{eq}{\ell\ell 3a}\},\\
\ccc{[I]}{lu}{\ell,3+a} &\equiv \ReIm\{\cc{}{lu}{\ell\ell 3a}\},\\
\ccc{[I]}{eu}{\ell,3+a} &\equiv \ReIm\{\cc{}{eu}{\ell\ell 3a}\},\\
\end{aligned}
\qquad
\begin{aligned}
\ccc{S[I]}{lequ}{\ell,3a} &\equiv \ReIm\{\cc{1}{lequ}{\ell\ell 3a}\},\\
\ccc{S[I]}{lequ}{\ell,a3} &\equiv \ReIm\{\cc{1}{lequ}{\ell\ell a3}\},\\
\ccc{T[I]}{lequ}{\ell,3a} &\equiv \ReIm\{\cc{3}{lequ}{\ell\ell 3a}\},\\
\ccc{T[I]}{lequ}{\ell,a3} &\equiv \ReIm\{\cc{3}{lequ}{\ell\ell a3}\}.
%\ccc{S[I]}{ledq}{\ell,3a} &\equiv \ReIm\{\cc{}{ledq}{\ell\ell 3a}\},\\
%\ccc{S[I]}{ledq}{\ell,a3} &\equiv \ReIm\{\cc{}{ledq}{\ell\ell a3}\},\\
\end{aligned}
\end{gather}


\begin{figure}[tb]\centering
\includegraphics[width=.4\linewidth]{section8/figs/4f.png}
\caption{Effective operators give rise to four-point contact interactions that
are overlooked in the anomalous coupling approach although they contribute in
FCNC processes at the same order in $1/\Lambda^2$ as three-point ones.
Representative diagrams are shown for $ug\to th$ production (or radiative $t\to
hug$ decay), and $e^+e^-\to t\,\bar u$ (or $t\to u\,e^+e^-$ decay).}
\label{fig:missing_four_point_operators}
\end{figure}

Compared to the anomalous coupling parametrization, the EFT approach has two
features that are worth emphasizing. First, it includes four-fermion operators,
which have unduly been neglected in most
experimental analyses (Ref.~\cite{Achard:2002vv} excepted). These operators
of dimension six could for instance arise at tree level in models where heavy
mediators couple to two fermionic currents. 
Second, the EFT approach captures the correlations between interaction terms that derive from electroweak gauge invariance. For instance, the $\bar
t\sigma^{\mu\nu}T^Aq \; h\; G_{\mu\nu}^A$ and $\bar t\sigma^{\mu\nu}T^Aq\;
G_{\mu\nu}^A$ interactions arise from the same
$O_{uG}$ operator and their coefficients are thus related. This kind
of correlation is due to the Higgs field $\varphi$ having physical excitations
around a vacuum expectation value,
and thus occurs in processes involving a Higgs particle.
In Fig.~\ref{fig:missing_four_point_operators}, we show examples of four-point interactions contributing to single top-quark FCNC production.
Correlations also arise from the fact that left-handed down- and up-type quarks
belong to a single
gauge-eigenstate doublet. Operator coefficients measurable in $B$-meson physics
are actually related to those relevant to top-quark physics.
\textcolor{red}{\dots
The impact of the flavour physics constraints will be discussed\dots(make connection with Sec.~\ref{sec:8.4}?)}

%(see Ref.~\cite{Alonso:2014csa, Buras:2014fpa} for recent EFT analyses
%exploiting the full standard-model gauge invariance) 
%on top FCNC operators has for instance be studied in
%Ref.~\cite{Fox:2007in, Li:2011fza, Li:2011af, Gong:2013sh} with one single
%operator switched on at the time, though. A truly global analysis in a fully
%gauge-invariant EFT framework remains to be carried out. It should take advantage
%of all types of correlations and use several processes in which a closed set of
%operators contributes through different combinations. Only by doing this, is it
%possible to disentangle the effects coming from each of them.


\subsubsection{Theory predictions}
Because the LHC is a hadron collider, theory predictions at LO are
often not sufficient when an accurate interpretation of observables in terms of
theory parameters is needed. Typical NLO QCD
corrections in top-decay processes~\cite {Kidonakis:2003sc, Zhang:2008yn,
Drobnak:2010wh, Drobnak:2010by, Zhang:2010bm, Zhang:2014rja}
amount to approximately $10\%$, while in production processes they can reach
between about $30\%$ and $80\%$~\cite{Liu:2005dp, Gao:2009rf, Zhang:2011gh,
Li:2011ek, Wang:2012gp}. 
Theory predictions for top-quark FCNC processes are in general available at NLO
accuracy in QCD.  Complete results at NLO in QCD for top-quark FCNC decays
through two-quark and two-quark--two-lepton operators can be found in
Ref.~\cite{Zhang:2014rja}.
Single top-quark production associated with a neutral gauge boson, $\gamma$, $Z$, or
the scalar boson $h$ have also been studied.  Two-quark operators have been
implemented in the {\sc FeynRules}/{\sc MadGraph5\_aMC@NLO} simulation
chain~\cite{Alloul:2013bka,Degrande:2011ua,Alwall:2014hca}, allowing for
automated NLO QCD predictions matched to parton shower. 
Details on this implementation have been presented in Ref.~\cite{Degrande:2014tta}.
Two-quark-two-lepton operators are now also
available in {\sc MadGraph5\_aMC@NLO}. Finally, the direct top-quark
production with decay process, $pp\to bW^+$, involves some additional technical
difficulty due to the intermediate top-quark resonance. It is now being
studied, and the corresponding NLO generator matched to parton shower will be
available in the future \cite{directtop}.

\begin{table}[tbh]\centering
\begin{tabular*}{\textwidth}{@{\extracolsep{\fill}}c@{}r@{\,}*{4}{c@{\,}}c}
	& $\br^{95\%\text{CL}}$
	& Ref.
	& exp.
	& $\sqrt{s}$
	& $\mathcal{L}$
	& remarks
\\
$t\to q Z$
% ATLAS t>qll
\\\hline\noalign{\vskip1mm}
$u$ 	& $\bf 1.7\times 10^{-4}$
	& \cite{Aaboud:2018nyl}		& ATLAS	& $13\,\tev$	& $36.1\,\ifb$
	& decay, $|m_{\ell\ell}-m_Z|<15\,\gev$
\\
$c$	& $\bf 2.4\times 10^{-4}$
\\
% CMS t>qll (prod + decay)
$u$	& $2.4\times 10^{-4}$
	& \cite{CMS:2017twu}		& CMS	& $13\,\tev$	& $35.9\,\ifb$
	& production plus decay
\\
$c$	& $4.5\times 10^{-4}$
% CMS t>qll (production only)
\\
$u$	& $2.2\times 10^{-4}$
	& \cite{Sirunyan:2017kkr}	& CMS	& $8\,\tev$	& $19.7\,\ifb$
	& production, $76 < m_{\ell\ell} < 106\,\gev$
\\
$c$	& $4.9\times 10^{-4}$
\\[2mm]
% ATLAS t>jj
$t\to qg$
\\\hline\noalign{\vskip1mm}
$u$	& $0.40\times 10^{-4}$
	& \cite{Aad:2015gea}		& ATLAS	& $8\,\tev$	& $20.3\,\ifb$
	& $\sigma(pp\to t)\times \br(t\to bW) < 3.4\,\pb$
\\
$c$	& $\bf 2.0\times 10^{-4}$
\\
% CMS t>jj
$u$	&  $\bf 0.20\times 10^{-4}$
	& \cite{Khachatryan:2016sib}	& CMS	& $7,8\,\tev$	& $5.0,17.9\,\ifb$
	& in $pp\to tj$
\\
$c$	& $4.1\times 10^{-4}$
\\[2mm]
% CMS t>qA
$t\to q\gamma$
\\\hline\noalign{\vskip1mm}
$u$	& $\bf 1.3\times 10^{-4}$
	& \cite{Khachatryan:2015att}	& CMS	& $8\,\tev$	& $19.8\,\ifb$
	& $\sigma(pp\to t\gamma)\times \br(t\to bl\nu)<26\,\fb$
\\
$c$	& $\bf 17\times 10^{-4}$
	&&&&
	& $\sigma(pp\to t\gamma)\times \br(t\to bl\nu)<37\,\fb$
\\[2mm]
$t\to qh$
\\\hline\noalign{\vskip1mm}
$u$	& $\bf 19\times 10^{-4}$
	& \cite{Aaboud:2018pob}		& ATLAS	& $13\,\tev$	& $36.1\,\ifb$
	& multilepton channel
\\
$c$	& $\bf 16\times 10^{-4}$
	&
\\
$u$	& $55\times 10^{-4}$
	& \cite{Khachatryan:2016atv}	& CMS	& $8\,\tev$	& $19.7\,\ifb$
	& multilepton, $\gamma\gamma$, $b\bar b$
\\
$c$	& $40\times 10^{-4}$
	&
\\
$u$	& $47\times 10^{-4}$
	& \cite{Sirunyan:2017uae}	& CMS	& $13\,\tev$	& $35.9\,\ifb$
	& $b\bar{b}$
\\
$c$	& $47\times 10^{-4}$
	&
\\[1mm]\hline
\end{tabular*}
\caption{%
Summary of the existing $95\%$~C.L.\ limits on top-quark FCNC branching fractions obtained at the LHC.
A summary plot is made available at \protect\url{https://twiki.cern.ch/twiki/bin/view/LHCPhysics/LHCTopWGSummaryPlots\#table21} by the LHC TOP WG. 
Interestingly, Ref.\,\cite{Khachatryan:2015att} has set limits in a fiducial volume at the particle level for $pp\to t\gamma$ (not displayed in this table).
Bold numbers are used as input to our global effective-field-theory analysis.}
\label{tab:current_limits}
\end{table}


\subsubsection{Existing limits}
In Tab.~\ref{tab:current_limits}, we list the existing limits on FCNC processes.
%
We follow Ref.\,\cite{Durieux:2014xla} and interpret them in
a global effective field theory. Few additional remarks are in order:
%
\begin{itemize}

\item For $t\to q\ell\ell$ we use the predictions at NLO in QCD provided in Ref.\,\cite{Durieux:2014xla} for the $m_{\ell\ell}\in[78,102]\,\gev$ range although the most stringent constraints by ATLAS are set using $|m_{\ell\ell}-m_Z|<15\,\gev$. The bounds obtained by CMS by combining production and decay process cannot be reinterpreted to include four-fermion operators.

\item For single top-quark production through the $tqg$ interaction, we use the best constraints in the up-quark channel by CMS and in the charm-quark channel by ATLAS and naively combine them together using the numerical value at NLO in QCD for the $t\to jj$ branching fraction provided in Ref.\,\cite{Durieux:2014xla} (Sec.V.B).

\item For single top-quark production in association with a photon, we note the very interesting fiducial limit provided by Ref.\,\cite{Khachatryan:2015att} which allows for an accurate reinterpretation. Here, however, we use the simplified approach of Ref.\,\cite{Durieux:2014xla} based on the limit quoted on the total cross section and use the numerical values computed at NLO in QCD for $pp\to t\gamma+\bar{t}\gamma$ with a $30\,\gev$ cut on the photon $p_T$ although a $50\,\gev$ cut was applied in Ref.\,\cite{Khachatryan:2015att}.

\item For top-quark decay to a Higgs boson and a jet, we use the most stringent limits set by ATLAS in the multilepton channel. The dependence on all operator coefficients but $C_{t\phi}$ and $C_{tG}$ is assumed negligible.

\item Limits on $e^+e^-\to tj+\bar{t}j$ obtained at LEP~II~\cite{Aleph:2001dzz} still set the dominant constraints on four-fermion operators involving electrons while $t\to q\ell\ell$ at hadron colliders are the only measurements constraining four-fermion operators featuring a pair of muons.
Limits on the latter are however not explicitly shown below.
We employ as input, the limit set at the highest centre-of-mass energy of $\sqrt{s}=207\,\gev$ which is the most sensitive to four-fermion operators: $\sigma(e^+e^-\to tj+\bar tj)<170\,\fb$.

\end{itemize}

The results of a global analysis based on existing data are displayed
in Fig.~\ref{fig:limits_current} (left panel). They are presented as $95\%\,$C.L.\ limits
on the EFT degrees of freedom defined in Ref.\,\cite{AguilarSaavedra:2018nen}.
As explained in Ref.\,\cite{Durieux:2014xla}, no statistical combination is
attempted, i.e.~limits from different measurements are only overlaid.
In addition, we show two-dimensional constraints in the $c_{\varphi q,\varphi
u}^-$,$c_{eq,eu}$ plane in Fig.~\ref{fig:limits_pu_eu_current} (left panel)
to illustrate the complementarity between the LHC and the LEP~II measurements:
the former gives better constraints on two-fermion operators, while the latter
on four-fermion operators.


\begin{figure}\centering
\hfil\includegraphics[scale=1]{section8/figs/limits.mps}%
\hfil\includegraphics[scale=1]{section8/figs/limits_prospects.mps}
\hfil
\caption{Current (left) and prospective HL-LHC (right) $95\%\,$C.L.\ limits on top-quark FCNC operator coefficients in the conventions of Ref.\,\cite{AguilarSaavedra:2018nen}. Red and blue bars stand for $a=1$ and $2$, respectively, i.e.\ for top-up and top-charm FCNCs. White marks indicate individual limits, obtained under the unrealisitic assumption that all other operator coefficients vanish.}
\label{fig:limits_current}
\label{fig:limits_prospects}
\end{figure}


\begin{figure}\centering
\includegraphics[width=.49\textwidth]{section8/figs/limits_pu_eu.mps}\hfill
\includegraphics[width=.49\textwidth]{section8/figs/limits_pu_eu_prospects.mps}
\caption{Current (left) and prospective HL-LHC (right) limits on top-quark FCNC operator coefficients in a two-dimensional plane formed by two- ($x$ axis) and four-fermion ($y$ axis) operator coefficients. Other parameter are marginalized over, within the constraints obtained when all measurements are included. Red and blue regions stand for $a=1$ and $2$, respectively, i.e.\ for top-up and top-charm FCNCs. The impact of $t\to j\ell^+\ell^-$ and $e^+e^-\to tj,\bar{t}j$ measurements is displayed separately.}
\label{fig:limits_pu_eu_current}
\label{fig:limits_pu_eu_prospects}
\end{figure}

\subsubsection{Prospective limits}
We use the prospects presented in Sec.~\ref{WG1:top_fcnc}\textcolor{red}{[top fcnc experimental section]} to derive
approximate global constraints for the HL-LHC scenario. Few remarks are in order:
\begin{itemize}
\item As previously, we assume that the limits quoted on the $\br(t\to qZ)$ branching fraction are derived using the dilepton decays of the $Z$ boson only and in a $m_{\ell\ell}\in[78,102]\,\gev$ window for the dilepton invariant mass. This determines the sensitivity to four-fermion operators.

\item The limits on the $tq\gamma$ interaction have been derived by a combination of production and decay processes~\cite{CMS:2017gvo}. As the only prospect provided is on the $\br(t\to q\gamma)$ branching fraction, we approximate it as arising that arise from the measurement of the decay process only. The difference would lie primarily in the dependence on $tqg$ interactions.

\end{itemize}
Again, the results of a global fit based on the HL-LHC prospects are displayed
in Fig.~\ref{fig:limits_prospects} (right panel). Comparing with the left panel,
constraints on two-fermion operators are typically improved by a factor of a few,
while those on four-fermion operators are only marginally improved, mostly
indirectly through the improvement of the limits on other operator
coefficients.  The two dimensional plane of
Fig.~\ref{fig:limits_pu_eu_prospects} (right panel) shows that LEP~II
constraints on $e^+e^-\to tj,\bar{t}j$ production will start having little impact, even
on the four-fermion operators, after the HL-LHC phase.
}

\subsection{SMEFT analysis and complementarity with B-decays}
\label{sec:8.4}
%Eleni Vryonidou

\subsection{Experimental perspectives}


\end{document}
