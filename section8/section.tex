% mainfile: ../report.tex

% don't remove the folling lines, and edit the defintion of \main if needed
\documentclass[../report.tex]{subfiles}
\providecommand{\main}{..}
\IfEq{\jobname}{\currfilebase}{\AtEndDocument{\biblio}}{}
% until here
\begin{document}

%\section{Flavour aspects of top physics}
\section{The top quark and flavour physics}
\label{sec:top:quark}
%{\bf Author(s): Wouter Dekens, Frederic Deliot, Gauthier Durieux, Mariel Estevez, Darius Faroughy, Jernej F. Kamenik, Fabio Maltoni,  Eleni Vryonidou, Cen Zhang} (10 pages,TH:EXP= 50\%:50\%)

Among the SM fermions the top quark stands out. It has a large mass and an ${\mathcal O}(1)$ coupling to the Higgs, quite distinct from any other SM fermion. Studying top quark  properties may shed light on the resolution of the SM flavour puzzle, or at least as to why one and only one Yukawa coupling is large. The large Higgs-top coupling is also the reason for the weak scale hierarchy problem to be so acute -- the quadratic divergent corrections to the Higgs mass are driven almost completely by this coupling. BSM models addressing the hierarchy problem may thus well leave an imprint in the top quark properties and decays.  For instance, the FCNC top decays, $t\to c\gamma, cZ, cg$, are null tests of the SM and are used as BSM probes.

Top quark also directly enters the flavour phenomenology. Loops with the top quark are 
responsible for the largest short-distance contributions to the down-quark FCNCs. 
The SM predictions are thus controlled by the flavour couplings of the top -- with $B$ and $K$ transitions determining the CKM matrix elements $V_{tb}, V_{ts}, V_{td}$ through these virtual effects. Determining $V_{tx}$ directly from high \pt transitions, as well as the structure of the $Wtb$ vertices, can then serve as independent consistency checks of the SM. 

The LHC is already a top factory and the currently available statistics has allowed ATLAS and CMS to perform a vast campaign of top related measurements. However, the larger number of top quarks at HL-LHC and HE-LHC will open new possibilities for precise measurements of top-quark properties and for significant 
improvements probing NP, such as the rare FCNC decays. 
%those involving flavour changing neutral currents.

\subsection{Global effective-field-theory interpretation of top-quark FCNCs} \label{sec:top:FCNC} 
{\it\small Authors (TH): Gauthier Durieux, Teppei Kitahara, Cen Zhang}
{
\newcommand{\hc}[1]{#1}%{}^\ddagger #1}%{\hspace{+2mm}\raisebox{.5mm}{$\cdot$}\hspace{-2mm}#1}}

\newcommand{\ifb}{\ensuremath{\text{fb}^{-1}}}
\newcommand{\iab}{\ensuremath{\text{ab}^{-1}}}

\newcommand{\FDF}{(\varphi^\dagger i\!\!\overleftrightarrow{D}_\mu\varphi)}
\newcommand{\FDFI}{(\varphi^\dagger i\!\!\overleftrightarrow{D}^I_\mu\varphi)}

\newcommand*{\tmp}[4]{\ensuremath{%
	{#4%
	\ifx\empty#3\empty\ifx\empty#1\empty\else^{#1}\fi\else^{#1(#3)}\fi%
	\ifx\empty#2\empty\else_{#2}\fi}}}%
\newcommand{\qq }[4][]{\tmp{#2}{#3}{#4}{#1{O}}}
%\newcommand*{\cc }[4][]{\tmp{#2}{#3}{#4}{#1{C}}}
\newcommand*{\ccc}[4][]{\tmp{#2}{#3}{#4}{#1{c}}}

\newcommand{\ReIm}{{}_{\Re}^{[\Im]}\!}
\newcommand*{\sw}{s_W}
\newcommand*{\cw}{c_W}

%Following the methodology employed in Ref.\,\cite{Durieux:2014xla}, we provide an updated global effective-field-theory interpretation of current constraints on top-quark FCNCs before using the HL-LHC prospects presented in Sec.~\ref{WG1:top_fcnc}\textcolor{red}{[top fcnc experimental section]}.

\subsubsection{Effective operators}

Starting from an Effective Field Theory (EFT) with full $SU(3)_C\times SU(2)_L\times U(1)_Y$ gauge symmetry and matter content of the SM,
one can show that odd-dimensional operators all violate baryon or lepton 
numbers~\cite{Degrande:2012wf}. Imposing the conservation of these quantum numbers, the leading higher dimensional operators of the SM arise at dimension six.
We follow the top-quark EFT conventions set by the LHC TOP WG in 
Ref.~\cite{AguilarSaavedra:2018nen}. The LHC TOP WG employs linear
combinations of Warsaw-basis operators~\cite{Grzadkowski:2010es} which appear in
interactions with physical fields after electroweak symmetry breaking.

The operators contributing to top-quark FCNC processes fall into
several categories. We consider operators involving
exactly two quarks, as well as those involving two quarks and two leptons.
Operators containing four quarks only start contributing at next-to-leading order in QCD in most of the measurements we consider ($pp\to tj$ is the exception).
The corresponding Warsaw-basis operators are~\cite{AguilarSaavedra:2018nen} 
%(see also Eqs.\,(12-20) and (21-28) in Ref.~\cite{AguilarSaavedra:2018nen})
\begin{equation}
	\begin{aligned}
	\hc{\qq{}{u\varphi}{ij}}
	&=\bar{q}_i u_j\tilde\varphi\: (\varphi^{\dagger}\varphi)
	,\\
	\qq{1}{\varphi q}{ij}
	&=\FDF (\bar{q}_i\gamma^\mu q_j)
	,\\
	\qq{3}{\varphi q}{ij}
	&=\FDFI (\bar{q}_i\gamma^\mu\tau^I q_j)
	,\\
	\qq{}{\varphi u}{ij}
	&=\FDF (\bar{u}_i\gamma^\mu u_j)
	,\\
	\hc{\qq{}{\varphi ud}{ij}}
	&=(\tilde\varphi^\dagger iD_\mu\varphi)
	  (\bar{u}_i\gamma^\mu d_j)
	,\\
	\hc{\qq{}{uW}{ij}}
	&=(\bar{q}_i\sigma^{\mu\nu}\tau^Iu_j)\:\tilde{\varphi}W_{\mu\nu}^I
	,\\
	\hc{\qq{}{dW}{ij}}
	&=(\bar{q}_i\sigma^{\mu\nu}\tau^Id_j)\:{\varphi} W_{\mu\nu}^I
	,\\
	\hc{\qq{}{uB}{ij}}
	&=(\bar{q}_i\sigma^{\mu\nu} u_j)\quad\:\tilde{\varphi}B_{\mu\nu}
	,\\
	\hc{\qq{}{uG}{ij}}
	&=(\bar{q}_i\sigma^{\mu\nu}T^Au_j)\:\tilde{\varphi}G_{\mu\nu}^A
\end{aligned}\quad
\begin{aligned}
	\qq{1}{lq}{ijkl}
	&=(\bar l_i\gamma^\mu l_j)
	  (\bar q_k\gamma^\mu q_l)
	,\\
	\qq{3}{lq}{ijkl}
	&=(\bar l_i\gamma^\mu \tau^I l_j)
	  (\bar q_k\gamma^\mu \tau^I q_l)
	,\\
	\qq{}{lu}{ijkl}
	&=(\bar l_i\gamma^\mu l_j)
	  (\bar u_k\gamma^\mu u_l)
	,\\
	\qq{}{eq}{ijkl}
	&=(\bar e_i\gamma^\mu e_j)
	  (\bar q_k\gamma^\mu q_l)
	,\\
	\qq{}{eu}{ijkl}
	&=(\bar e_i\gamma^\mu e_j)
	  (\bar u_k\gamma^\mu u_l)
	,\\
	\hc{\qq{1}{lequ}{ijkl}}
	&=(\bar l_i e_j)\;\varepsilon\;
	  (\bar q_k u_l)
	,\\
	\hc{\qq{3}{lequ}{ijkl}}
	&=(\bar l_i \sigma^{\mu\nu} e_j)\;\varepsilon\;
	  (\bar q_k \sigma_{\mu\nu} u_l)
	,\\
	\hc{\qq{}{ledq}{ijkl}}
	&=(\bar l_i e_j)
	  (\bar d_k q_l)
  \end{aligned}
\end{equation}
The operators $\qq{}{\varphi ud}{}$, $\qq{}{dW}{}$, and $\qq{}{ledq}{}$,  only contribute to charged top-quark currents (not considering SM electroweak corrections) and are therefore not relevant for our purposes.
The EFT degrees of freedom appearing in top-quark FCNC processes were defined in Appendices~E.1-2 of Ref.\,\cite{AguilarSaavedra:2018nen}. They are:%
{%
\renewcommand*{\cc }[4][]{\tmp{#2}{#3}{#4}{#1{C}}}%
\begin{align}
\ccc{[I]}{t\varphi}{3a} &\equiv \ReIm\{\cc{}{u\varphi}{3a}\}, &\ccc{[I]}{uA}{3a} &\equiv \ReIm\{\cw\cc{}{uB}{3a}+\sw\cc{}{uW}{3a}\},
\\
\ccc{[I]}{t\varphi}{a3} &\equiv \ReIm\{\cc{}{u\varphi}{a3}\},&\ccc{[I]}{uA}{a3} &\equiv \ReIm\{\cw\cc{}{uB}{a3}+\sw\cc{}{uW}{a3}\},
\\
\ccc{-[I]}{\varphi q}{3+a}  &\equiv \ReIm\{\cc{1}{\varphi q}{3a}-\cc{3}{\varphi q}{3a}\},
&\ccc{[I]}{uZ}{3a} &\equiv \ReIm\{-\sw\cc{}{uB}{3a}+\cw\cc{}{uW}{3a}\},
\\
%\ccc{3[I]}{\varphi q}{3+a}  &\equiv \ReIm\{\cc{3}{\varphi q}{3a}\},\\
\ccc{[I]}{\varphi u}{3+a}  &\equiv \ReIm\{\cc{}{\varphi u}{3a}\},
&\ccc{[I]}{uZ}{a3} &\equiv \ReIm\{-\sw\cc{}{uB}{a3}+\cw\cc{}{uW}{a3}\},
%\ccc{[I]}{\varphi ud}{3a} &\equiv \ReIm\{\cc{}{\varphi ud}{3a}\},\\
%\ccc{[I]}{\varphi ud}{a3} &\equiv \ReIm\{\cc{}{\varphi ud}{a3}\},\\
\end{align}
as well as 
$\ccc{[I]}{uG}{3a} \equiv \ReIm\{\cc{}{uG}{3a}\}$, 
$\ccc{[I]}{uG}{a3} \equiv \ReIm\{\cc{}{uG}{a3}\},$ and 
\begin{align}
%\ccc{3[I]}{lq}{\ell,3+a} &\equiv \ReIm\{\cc{3}{lq}{\ell\ell 3a}\},\\
\ccc{-[I]}{lq}{\ell,3+a} &\equiv \ReIm\{\cc{-}{lq}{\ell\ell 3a}\}, &\ccc{S[I]}{lequ}{\ell,3a} &\equiv \ReIm\{\cc{1}{lequ}{\ell\ell 3a}\},
\\
\ccc{[I]}{eq}{\ell,3+a} &\equiv \ReIm\{\cc{}{eq}{\ell\ell 3a}\},& \ccc{S[I]}{lequ}{\ell,a3} &\equiv \ReIm\{\cc{1}{lequ}{\ell\ell a3}\},\\
\ccc{[I]}{lu}{\ell,3+a} &\equiv \ReIm\{\cc{}{lu}{\ell\ell 3a}\},& \ccc{T[I]}{lequ}{\ell,3a} &\equiv \ReIm\{\cc{3}{lequ}{\ell\ell 3a}\},\\
\ccc{[I]}{eu}{\ell,3+a} &\equiv \ReIm\{\cc{}{eu}{\ell\ell 3a}\},& \ccc{T[I]}{lequ}{\ell,a3} &\equiv \ReIm\{\cc{3}{lequ}{\ell\ell a3}\}.
\end{align}
}%

%{%
%\renewcommand*{\cc }[4][]{\tmp{#2}{#3}{#4}{#1{C}}}%
%\begin{gather}
%\begin{aligned}
%\ccc{[I]}{t\varphi}{3a} &\equiv \ReIm\{\cc{}{u\varphi}{3a}\},\\
%\ccc{[I]}{t\varphi}{a3} &\equiv \ReIm\{\cc{}{u\varphi}{a3}\},\\[2mm]
%\ccc{-[I]}{\varphi q}{3+a}  &\equiv \ReIm\{\cc{1}{\varphi q}{3a}-\cc{3}{\varphi q}{3a}\},\\
%%\ccc{3[I]}{\varphi q}{3+a}  &\equiv \ReIm\{\cc{3}{\varphi q}{3a}\},\\
%\ccc{[I]}{\varphi u}{3+a}  &\equiv \ReIm\{\cc{}{\varphi u}{3a}\},\\
%%\ccc{[I]}{\varphi ud}{3a} &\equiv \ReIm\{\cc{}{\varphi ud}{3a}\},\\
%%\ccc{[I]}{\varphi ud}{a3} &\equiv \ReIm\{\cc{}{\varphi ud}{a3}\},\\
%\end{aligned}
%\qquad
%\begin{aligned}
%\ccc{[I]}{uA}{3a} &\equiv \ReIm\{\cw\cc{}{uB}{3a}+\sw\cc{}{uW}{3a}\},\\
%\ccc{[I]}{uA}{a3} &\equiv \ReIm\{\cw\cc{}{uB}{a3}+\sw\cc{}{uW}{a3}\},\\
%\ccc{[I]}{uZ}{3a} &\equiv \ReIm\{-\sw\cc{}{uB}{3a}+\cw\cc{}{uW}{3a}\},\\
%\ccc{[I]}{uZ}{a3} &\equiv \ReIm\{-\sw\cc{}{uB}{a3}+\cw\cc{}{uW}{a3}\},\\
%%\ccc{[I]}{dW}{3a} &\equiv \ReIm\{\cc{}{dW}{3a}\},\\
%%\ccc{[I]}{dW}{a3} &\equiv \ReIm\{\cc{}{dW}{a3}\},\\
%\ccc{[I]}{uG}{3a} &\equiv \ReIm\{\cc{}{uG}{3a}\},\\
%\ccc{[I]}{uG}{a3} &\equiv \ReIm\{\cc{}{uG}{a3}\},\\
%\end{aligned}
%\\
%\begin{aligned}
%%\ccc{3[I]}{lq}{\ell,3+a} &\equiv \ReIm\{\cc{3}{lq}{\ell\ell 3a}\},\\
%\ccc{-[I]}{lq}{\ell,3+a} &\equiv \ReIm\{\cc{-}{lq}{\ell\ell 3a}\},\\
%\ccc{[I]}{eq}{\ell,3+a} &\equiv \ReIm\{\cc{}{eq}{\ell\ell 3a}\},\\
%\ccc{[I]}{lu}{\ell,3+a} &\equiv \ReIm\{\cc{}{lu}{\ell\ell 3a}\},\\
%\ccc{[I]}{eu}{\ell,3+a} &\equiv \ReIm\{\cc{}{eu}{\ell\ell 3a}\},\\
%\end{aligned}
%\qquad
%\begin{aligned}
%\ccc{S[I]}{lequ}{\ell,3a} &\equiv \ReIm\{\cc{1}{lequ}{\ell\ell 3a}\},\\
%\ccc{S[I]}{lequ}{\ell,a3} &\equiv \ReIm\{\cc{1}{lequ}{\ell\ell a3}\},\\
%\ccc{T[I]}{lequ}{\ell,3a} &\equiv \ReIm\{\cc{3}{lequ}{\ell\ell 3a}\},\\
%\ccc{T[I]}{lequ}{\ell,a3} &\equiv \ReIm\{\cc{3}{lequ}{\ell\ell a3}\}.
%%\ccc{S[I]}{ledq}{\ell,3a} &\equiv \ReIm\{\cc{}{ledq}{\ell\ell 3a}\},\\
%%\ccc{S[I]}{ledq}{\ell,a3} &\equiv \ReIm\{\cc{}{ledq}{\ell\ell a3}\},\\
%\end{aligned}
%\end{gather}%
%}%


\begin{figure}[tb]\centering
\includegraphics[width=.6\linewidth]{section8/figs/4f}
%\includegraphics[width=.49\linewidth]{section8/figs/4fb}
\caption{Effective operators give rise to four-point contact interactions that
are overlooked in the approach with anomalous couplings,  although they contribute to
FCNC processes at the same order in $1/\Lambda^2$ as the three-point ones.
Representative diagrams are shown for $ug\to th$ production (or radiative $t\to
hug$ decay), and $e^+e^-\to t\,\bar u$ (or $t\to u\,e^+e^-$ decay). Figure taken from Ref.~\cite{Durieux:2014xla}.}
\label{fig:missing_four_point_operators}
\end{figure}

Compared to the anomalous coupling parametrization, the EFT approach has two
features that are worth emphasizing. First, it includes four-fermion operators,
which have  been unduly neglected in most
experimental analyses (apart from Ref.~\cite{Achard:2002vv}).  
Second, the EFT approach captures the correlations between interaction terms that derive from electroweak gauge invariance. For instance, the $\bar
t\sigma^{\mu\nu}T^Aq \; h\; G_{\mu\nu}^A$ and $\bar t\sigma^{\mu\nu}T^Aq\;
G_{\mu\nu}^A$ interactions arise from the same
$O_{uG}$ operator and their coefficients are thus related. 
In Fig.~\ref{fig:missing_four_point_operators}, we show examples of four-point interactions contributing to single top-quark FCNC production.
Correlations also arise from the fact that left-handed down- and up-type quarks
belong to a single
gauge-eigenstate doublet. Operator coefficients measurable in $B$-meson physics
are thus related to those relevant to top-quark physics (see Secs.~\ref{sec:B:K:top} and \ref{sec:8:2:Indirect}).




\subsubsection{Theory predictions}
Because the LHC is a hadron collider, theory predictions at LO are
often not sufficient when an accurate interpretation of observables in terms of
theory parameters is needed. Typical NLO QCD
corrections in top-decay processes~\cite {Kidonakis:2003sc, Zhang:2008yn,
Drobnak:2010wh, Drobnak:2010by, Zhang:2010bm, Zhang:2014rja}
amount to approximately $10\%$, while in production processes they can reach
between about $30\%$ and $80\%$~\cite{Liu:2005dp, Gao:2009rf, Zhang:2011gh,
Li:2011ek, Wang:2012gp}. 
Theory predictions for top-quark FCNC processes are in general available at NLO
accuracy in QCD.  Complete results at NLO in QCD for top-quark FCNC decays
through two-quark and two-quark--two-lepton operators can be found in
Ref.~\cite{Zhang:2014rja}.
Single top-quark production associated with a neutral gauge boson, $\gamma$, $Z$, or
the scalar boson $h$ have also been studied.  Two-quark operators have been
implemented in the {\tt FeynRules}/{\tt MadGraph5\_aMC@NLO} simulation
chain~\cite{Alloul:2013bka,Degrande:2011ua,Alwall:2014hca}, allowing for
automated NLO QCD predictions matched to parton shower. 
Details on this implementation have been presented in Ref.~\cite{Degrande:2014tta}.
Two-quark-two-lepton operators are now also
available in {\tt MadGraph5\_aMC@NLO}. Finally, the direct top-quark
production with decay process, $pp\to bW^+$, involves additional technical
difficulties due to the intermediate top-quark resonance. It is now being
studied, and the corresponding NLO generator matched to parton shower will be
available in the future \cite{directtop}.

\begin{table}[t]\centering
\caption{%
Summary of the existing $95\%$~C.L.\ limits on top-quark FCNC branching fractions obtained at the LHC.
A summary plot is available at \protect\url{https://twiki.cern.ch/twiki/bin/view/LHCPhysics/LHCTopWGSummaryPlots\#table21}.
% by the LHC TOP WG. 
%Interestingly,
 Ref.\,\cite{Khachatryan:2015att} has set limits in a fiducial volume at the particle level for $pp\to t\gamma$ (not displayed in this table).
Numbers in bold are used as inputs to the global EFT analysis.}
\begin{tabular*}{\textwidth}{@{\extracolsep{\fill}}c@{}r@{\,}*{4}{c@{\,}}c}
\hline\hline
Mode	& $\br^{95\%\text{CL}}$
	& Ref.
	& exp.
	& $\sqrt{s}$
	& $\mathcal{L}$
	& remarks
\\
\hline
$t\to q Z$
% ATLAS t>qll
\\
\noalign{\vskip1mm}
$u$ 	& $\bf 1.7\times 10^{-4}$
	& \cite{Aaboud:2018nyl}		& ATLAS	& $13\,\tev$	& $36.1\,\ifb$
	& decay, $|m_{\ell\ell}-m_Z|<15\,\gev$
\\
$c$	& $\bf 2.4\times 10^{-4}$
\\
% CMS t>qll (prod + decay)
$u$	& $2.4\times 10^{-4}$
	& \cite{CMS:2017twu}		& CMS	& $13\,\tev$	& $35.9\,\ifb$
	& production plus decay
\\
$c$	& $4.5\times 10^{-4}$
% CMS t>qll (production only)
\\
$u$	& $2.2\times 10^{-4}$
	& \cite{Sirunyan:2017kkr}	& CMS	& $8\,\tev$	& $19.7\,\ifb$
	& production, $76 < m_{\ell\ell} < 106\,\gev$
\\
$c$	& $4.9\times 10^{-4}$
\\[2mm]
% ATLAS t>jj
\hline
$t\to qg$
\\\noalign{\vskip1mm}
$u$	& $0.40\times 10^{-4}$
	& \cite{Aad:2015gea}		& ATLAS	& $8\,\tev$	& $20.3\,\ifb$
	& $\sigma(pp\to t)\times \br(t\to bW) < 3.4\,\pb$
\\
$c$	& $\bf 2.0\times 10^{-4}$
\\
% CMS t>jj
$u$	&  $\bf 0.20\times 10^{-4}$
	& \cite{Khachatryan:2016sib}	& CMS	& $7,8\,\tev$	& $5.0,17.9\,\ifb$
	& in $pp\to tj$
\\
$c$	& $4.1\times 10^{-4}$
\\[2mm]
% CMS t>qA
\hline
$t\to q\gamma$
\\\noalign{\vskip1mm}
$u$	& $\bf 1.3\times 10^{-4}$
	& \cite{Khachatryan:2015att}	& CMS	& $8\,\tev$	& $19.8\,\ifb$
	& $\sigma(pp\to t\gamma)\times \br(t\to bl\nu)<26\,\fb$
\\
$c$	& $\bf 17\times 10^{-4}$
	&&&&
	& $\sigma(pp\to t\gamma)\times \br(t\to bl\nu)<37\,\fb$
\\[2mm]
\hline
$t\to qh$
\\\noalign{\vskip1mm}
$u$	& $\bf 19\times 10^{-4}$
	& \cite{Aaboud:2018pob}		& ATLAS	& $13\,\tev$	& $36.1\,\ifb$
	& multilepton channel
\\
$c$	& $\bf 16\times 10^{-4}$
	&
\\
$u$	& $55\times 10^{-4}$
	& \cite{Khachatryan:2016atv}	& CMS	& $8\,\tev$	& $19.7\,\ifb$
	& multilepton, $\gamma\gamma$, $b\bar b$
\\
$c$	& $40\times 10^{-4}$
	&
\\
$u$	& $47\times 10^{-4}$
	& \cite{Sirunyan:2017uae}	& CMS	& $13\,\tev$	& $35.9\,\ifb$
	& $b\bar{b}$
\\
$c$	& $47\times 10^{-4}$
	&
\\[1mm]\hline\hline
\end{tabular*}
\label{tab:current_limits}
\end{table}


\subsubsection{Existing limits}
Table~\ref{tab:current_limits} lists the existing limits on FCNC processes.
%
We follow Ref.\,\cite{Durieux:2014xla} and interpret them in
a global EFT analysis. Several additional remarks are in order:
%
\begin{itemize}

\item[$\bullet$] For $t\to q\ell\ell$ we use the predictions at NLO in QCD provided in Ref.\,\cite{Durieux:2014xla} for the $m_{\ell\ell}\in[78,102]\,\gev$ range although the most stringent constraints by ATLAS are set using $|m_{\ell\ell}-m_Z|<15\,\gev$. The CMS bounds obtained by combining production and decay process cannot be reinterpreted to include four-fermion operators.

\item[$\bullet$] For single top-quark production through the $tqg$ interaction, we use the best constraints: in the up-quark channel by CMS and in the charm-quark channel by ATLAS. We naively combine them using the numerical value at NLO in QCD for the $t\to jj$ branching fraction provided in Sec.~V.B of Ref.\,\cite{Durieux:2014xla}.

\item[$\bullet$] For single top-quark production in association with a photon, we note the very interesting fiducial limit provided by Ref.\,\cite{Khachatryan:2015att} which allows for an accurate reinterpretation. 
%Here, however, w
However, we use the simplified approach of Ref.\,\cite{Durieux:2014xla} based on the limit quoted on the total cross section and use the numerical values computed at NLO in QCD for $pp\to t\gamma+\bar{t}\gamma$ with a $30\,\gev$ cut on the photon \pt although a $50\,\gev$ cut was applied in Ref.\,\cite{Khachatryan:2015att}.

\item[$\bullet$] For $t\to h j$ decay we use the most stringent limits set by ATLAS in the multilepton channel. The dependence on all operator coefficients, except $C_{t\phi}$ and $C_{tG}$, is assumed to be negligible.

\item[$\bullet$] Limits on $e^+e^-\to tj+\bar{t}j$ obtained at LEP~II~\cite{Aleph:2001dzz} still dominate constraints on four-fermion operators involving electrons, while $t\to q\ell\ell$ at hadron colliders are the only measurements constraining four-fermion operators featuring a pair of muons. However, 
the latter limits are not explicitly shown below.
We use the limit from the highest LEP II centre-of-mass energy, $\sqrt{s}=207\,\gev$, which is the most sensitive to four-fermion operators, $\sigma(e^+e^-\to tj+\bar tj)<170\,\fb$.
\end{itemize}

The global analysis based on existing data gives $95\%\,$C.L.\ limits
on the EFT Wilsons coefficients in the notation of Ref.\,\cite{AguilarSaavedra:2018nen}, shown 
in Fig.~\ref{fig:limits_current} (left panel). 
%They are presented as 
As explained in Ref.\,\cite{Durieux:2014xla}, 
no statistical combination is
attempted, i.e.,~limits from different measurements are only overlaid.
 Fig.~\ref{fig:limits_pu_eu_current} (left panel) show two-dimensional constraints in the $c_{\varphi q,\varphi
u}^-$,$c_{eq,eu}$ plane. 
This illustrates the complementarity between the LHC and the LEP~II measurements;
the former gives better constraints on two-fermion operators, the latter
on four-fermion operators.


\begin{figure}\centering
\hfil\includegraphics[scale=1]{section8/figs/limits.mps}%
\hfil\includegraphics[scale=1]{section8/figs/limits_prospects.mps}
\hfil
\caption{Current (left) and projected HL-LHC (right) $95\%\,$C.L.\ limits on top-quark FCNC operator coefficients in the conventions of Ref.\,\cite{AguilarSaavedra:2018nen}. Red and blue bars denote  top-up and top-charm FCNCs, respectively. White marks indicate individual limits, obtained under the unrealistic assumption that all the other operator coefficients vanish.}
\label{fig:limits_current}
\label{fig:limits_prospects}
\end{figure}


\begin{figure}\centering
\includegraphics[width=.49\textwidth]{section8/figs/limits_pu_eu.mps}\hfill
\includegraphics[width=.49\textwidth]{section8/figs/limits_pu_eu_prospects.mps}
\caption{Current (left) and prospective HL-LHC (right) $95\%\,$C.L.\ limits on top-quark FCNC operator coefficients in a two-dimensional plane formed by two- ($x$ axis) and four-fermion ($y$ axis) operator coefficients. Other parameters are marginalized over, within the constraints obtained when all measurements are included. Red and blue regions are the combined constraints for top-up and top-charm FCNCs. The impact of $t\to j\ell^+\ell^-$ and $e^+e^-\to tj,\bar{t}j$ measurements is displayed separately in dark and light gray colors for top-up and top-charm FCNCs, respectively.}
\label{fig:limits_pu_eu_current}
\label{fig:limits_pu_eu_prospects}
\end{figure}

\subsubsection{Future limits}
We use the prospects presented in Sec.~\ref{sec:8:EXP:FCNC} to estimate the future reach of global constraints for the HL-LHC scenario.  As previously, we assume that the limits quoted on the $\br(t\to qZ)$ branching fraction are derived using the dilepton decays of the $Z$ boson, in a $m_{\ell\ell}\in[78,102]\,\gev$ window for the dilepton invariant mass. This determines the sensitivity to four-fermion operators.
The limits on the $tq\gamma$ interaction were derived by a combination of production and decay processes~\cite{CMS:2017gvo}. Since the only prospect provided is on the $\BR(t\to q\gamma)$ branching fraction, we approximate it as though it is from the measurement of the decay process only. This assumption affects primarily the dependence of bounds on $tqg$ interactions.

The results of a global fit based on the HL-LHC prospects are displayed
in Fig.~\ref{fig:limits_prospects} (right panel). Comparing with the left panel,
constraints on two-fermion operators are typically improved by a factor of a few,
while those on four-fermion operators are only marginally improved, mostly
indirectly through the improvement of the limits on other operator
coefficients.  The two dimensional plane of
Fig.~\ref{fig:limits_pu_eu_prospects} (right panel) shows that LEP~II
constraints on $e^+e^-\to tj,\bar{t}j$ production will start having little impact, even
on the four-fermion operators, after the HL-LHC phase.
}

\subsubsection{Complementarity with $B$-meson and kaon rare processes}
\label{sec:B:K:top}
%Teppei Kitahara
The NP effective operators which include the top quark also contribute to
the low-scale effective Hamiltonian for meson decays, 
and hence, precision measurements of the $B$-meson and kaon rare processes provide complementary constraints to the NP top-quark operators~\cite{Fox:2007in,Drobnak:2011aa,Brod:2014hsa}.
 

In the framework of SMEFT, 
the $SU(2)_L$ gauge symmetry between $t_L$ and $b_L$ provides 
a direct constraint to the NP top-quark operators arising from the tree-level matching onto the $B$ physics Hamiltonian \cite{Drobnak:2011aa,Aebischer:2015fzz}.
For instance, in the flavour basis, the following operators
\begin{align}
O_{\ell q}^{1(ij32)} &= (\bar{\ell}_i \gamma_{\mu} P_L \ell_j)(\bar{q}_3 \gamma^{\mu} P_L q_2) = 
(\bar{\ell}_i \gamma_{\mu} P_L \ell _j)(\bar{t} \gamma^{\mu} P_L c) + (\bar{\ell}_i \gamma_{\mu} P_L \ell _j)(\bar{b} \gamma^{\mu} P_L s)
\,,\\
%
O_{\ell e q u}^{1( i j 3 2)} &= ( \bar{\ell}_i P_R e_j ) \varepsilon (\bar{q}_3 P_R c)  =
- ( \bar{\ell}_i{}^{+} P_R e_j ) (\bar{t} P_R c)+( \bar{\nu}_i P_R e_j ) (\bar{b} P_R c)  \,,
\end{align}
are
constrained from $B_s \to \ell^+ \ell^-$, $B \to K^{(\ast)} \overline{\nu} \nu$, $b \to s \ell^+ \ell^-$, and $b \to c\ell^- \bar{\nu}$ observables.
Each of the constraints significantly depends on the lepton-flavour dependence
(\textit{e.g.}, see Ref.~\cite{Fajfer:2015ycq}).


Also,
the NP operators which include single top quark, 
\textit{e.g.},  
\begin{align}
O^{1(32)}_{\varphi q} = (\varphi^{\dag} \overleftrightarrow{i D}_{\mu} \varphi)(\bar{t} \gamma^{\mu} P_L c) + (\varphi^{\dag} \overleftrightarrow{i D}_{\mu} \varphi)(\bar{b} \gamma^{\mu} P_L s)\,,
\end{align}
and two top quarks, 
\textit{e.g.},  
\begin{align}
O_{qd}^{1 (33kl)} = (\bar{t}\gamma_{\mu} P_L t )(\bar{d}^k \gamma^{\mu} P_R d^l) 
+ (\bar{b}\gamma_{\mu} P_L b)(\bar{d}^k \gamma^{\mu} P_R d^l) \quad \textrm{with} ~ k\neq l \,,
\end{align}
can contribute to the low-scale effective Hamiltonian 
through the one-loop matching conditions at the electroweak symmetry breaking  scale by integrating out the top quark, $W$, $Z$ and the SM Higgs boson.
These one-loop contributions are enhanced by the top-quark mass.
Although such a two top-quark operator does not contribute to 
the single top-quark production mentioned in this section, 
once a UV completion is considered, 
single and two top-quarks operators could be related. 
The one-loop matching conditions onto $\Delta F= 1$ processes are given in Ref.~\cite{Aebischer:2015fzz}, while 
the conditions onto  $\Delta F= 2$ ones are
 given in Ref.~\cite{Endo:2018gdn}.
Besides,
 one-loop matching conditions to $\Delta F = 0$, \textit{e.g.}, $h \to \tau^+ \tau^-$ and the  leptonic dipole moments, are investigated in Ref.~\cite{Feruglio:2018fxo}.


\subsubsection{Experimental perspectives}
\label{sec:8:EXP:FCNC}

\subsubsubsection{Signal modeling}

%%CHECK really experiment dependent? 

The generation of signal events at ATLAS is done at NLO with {\tt MadGraph5\_aMC@NLO}~\cite{Alwall:2014hca, Mangano:2006rw} and the effective field theory framework developed in the TopFCNC model is used~\cite{Degrande:2014tta,Durieux:2014xla}. In the case of $gqt$ coupling, the MEtop generator is used instead~\cite{Coimbra:2012ys}. At CMS signal events are simulated at LO with {\tt MadGraph5\_aMC@NLO} with the effective lagrangian implemented by means of the {\tt FEYNRULES} package, except in the simulation of signal events for $gqt$ and $\gamma qt$ couplings where {\tt CompHEP}~\cite{Boos:2004kh} and {\tt PROTOS 2.0}~\cite{AguilarSaavedra:2010rx} are used, respectively. In both experiments {\tt Pythia 8} is used to simulate the parton showering and hadronization. The generation of signal events is done under the assumption of only one non-vanishing FCNC coupling at a time. 

\begin{table}[t]\center
\caption{Summary of the projected reach for the 95\% C.L. limits on the branching ratio for anomalous flavor changing top couplings.}
\begin{tabular}{cccc}
\hline\hline
$\mathcal{B}$ limit at 95\%C.L. & 3 ab$^{-1}$, 14 TeV & 15ab$^{-1}$, 27 TeV & Ref. \\
\hline
$t\to gu $ & $3.8\times 10^{-6}$ & $5.6\times 10^{-7}$ & \cite{CMS:2018kwi}\\
$t\to gc $ & $32.1\times 10^{-6}$ &  $19.1\times 10^{-7}$ & \cite{CMS:2018kwi}\\
$t\to qZ $ & $2.4\times 10^{-5}$ &  & \cite{ATL-PHYS-PUB-2016-019}\\
$t\to \gamma u $ & $8.6\times 10^{-6}$ & & \cite{Collaboration:2293646}\\
$t\to \gamma c $ & $7.4\times 10^{-5}$ & & \cite{Collaboration:2293646} \\
$t\to Hq $ & 10$^{-4}$ & & \cite{ATL-PHYS-PUB-2016-019} \\
\hline\hline
\end{tabular}
\end{table}


%%%
\subsubsubsection{Top-gluon}

The $gqt$ FCNC process was studied by ATLAS~\cite{Aad:2015gea} and CMS~\cite{Khachatryan:2016sib} in single top quark events. The event signature includes the
requirement of one isolated lepton and the presence of a significant amount of transverse
missing energy ($E^{miss}_{T}$). The analysis at CMS requires exactly one $b$ and one non-$b$ jet to be present in the final state with the dominant background arising from the $t\bar{t}$+jets
production, while the analysis at ATLAS vetoes any additional jets resulting in the dominant source of background associated with the $W$+jets production. A neural network-based
technique is used to separate signal from background events.
The observed~(expected) 95\% CL upper limits in the CMS analysis are $\mathcal{B}(t \to gu) < 2.0~(2.8) \times 10^{-5}$ and
$\mathcal{B}(t \to gc) < 4.1~(2.8) \times 10^{-4}$, while the resultant limits in case of ATLAS are $\mathcal{B}(t \to gu) < 4.0~(3.5) \times 10^{-5}$ 
and $\mathcal{B}(t \to gc) < 2.0~(1.8) \times 10^{-4}$. The projected limits for 3 ab$^{-1}$ are $\mathcal{B}(t \to gu) < 3.8 \times 10^{-6}$ and $\mathcal{B}(t \to gc) < 32.1 \times 10^{-6}$~\cite{CMS:2018kwi}. \td{Refer to new fig~\ref{fig:topglue}}

\begin{figure}[t]
%  \vspace{-0.3cm}
\begin{center}{
\includegraphics[width=10cm]{section8/figs/CMS-topglueFCNC.png}
%\vspace{-0.3cm}
\caption{{Two-dimensional expected limits on the FCNC couplings and the corresponding branching fractions at 68\% and 95\% C.L. for an integrated luminosity of 3000 fb$^{-1}$}} \label{fig:topglue}}
\end{center}
\end{figure}


%%%
\subsubsubsection{Top-Z}

%The $t \rightarrow Z q$ FCNC decays are probed by ATLAS in top %quark
%pair events at 13 TeV~\cite{Aaboud:2018nyl}. The event topology %includes 
%the presence of two same-sign or three isolated leptons, %$E^{miss}_{T}$, exactly
%one $b$ jet and at least one non-$b$ jet. Several control regions %are
%defined for each of the dominant background processes: $WZ$+jets,
%$ZZ$+jets, $ttZ$, and non-prompt leptons. The signal is extracted %from a simultaneous maximum likelihood fit over control and signal %regions using various event kinematic variables. The resultant %limits are $\mathcal{B}(t \to Zu) < 1.7~(2.4) \times 10^{-4}$ and %$\mathcal{B}(t \to Zc) < 2.3~(3.2) \times 10^{-4}$. The analysis of %the 13 TeV data at CMS explores a similar final state and %additionally considers a single top associated FCNC production with %a Z boson in the simulation of signal events~\cite{CMS:2017twu}. %Only the three lepton final state is considered. A boosted decision %tree (BDT) discriminant is defined to suppress background events. %The resultant limits are $\mathcal{B}(t \to Zu) < 2.4~(1.5) \times %10^{-4}$ and $\mathcal{B}(t \to Zc) < 4.5~(3.7) \times 10^{-4}$. %Preliminary projection studies suggest the expected upper limits of %$\mathcal{B}(t \to Zq) < 2.4 \times %10^{-5}$~\cite{ATL-PHYS-PUB-2016-019}.


ATLAS studied the sensitivity to the $tqZ$ interaction, by performing an analysis~\cite{TOPQ-2018-35} based on simulated samples and following both the strategy of its 13~TeV data analysis on the same subject~\cite{TOPQ-2017-06} and of the general recommendations for this HL-LHC study. The study is performed in the three charged lepton final state of $t\bar t$ events, in which one of the top quarks decays to $qZ$, $(q=u,c)$ and the other one decays to $bW$ ($t\bar t\to bWqZ\to b\ell\nu q\ell\ell$). The kinematics of the events are reconstructed through a $\chi^2$ minimisation and dedicated control regions are used to normalize the main backgrounds and constrain systematic uncertainties. The main uncertainties, in both the background and signal estimations, are expected to come from theoretical normalization uncertainties and uncertainties in the modeling of background processes in the simulation. Different scenarios for the systematic uncertainties are considered, ranging from the full estimations obtained with the 13~TeV data analysis, to the ones expected with improvements in theoretical predictions, which should be half of the former ones. A binned likelihood function $L(\mu, \theta)$ is used to do the statistical analysis and extract the signal normalisation. An improvement by a factor of five is expected in relation to the current 13~TeV data analysis results. Obtained branching ratio limits are at the level of 4 to 5 $\times 10^{-5}$ depending on the considered scenarios for the systematic uncertainties.


%%%
\subsubsubsection{Top-gamma}
The $t\gamma q$ anomalous interactions are probed by CMS at 8 TeV in
events with single top quarks produced in association with a photon~\cite{Khachatryan:2015att}.
Event selection criteria includes the presence of one isolated lepton,
one isolated photon, $E^{miss}_{T}$, and up to one b jet. The
dominant $W\gamma$ and $W$+jets backgrounds are suppressed with a BDT. The resultant exclusion limits are $\mathcal{B}(t \to \gamma u) < 1.3~(1.9) \times 10^{-4}$ and
$\mathcal{B}(t \to \gamma c) < 2.0~(1.7) \times 10^{-3}$. Preliminary projection studies yield $\mathcal{B}(t \to \gamma u) < 8.6 \times 10^{-6}$, $\mathcal{B}(t \to \gamma c) < 7.4 \times 10^{-5}$~\cite{Collaboration:2293646}.

%%%
\subsubsubsection{Top-Higgs}

The $tHq$ interactions are studied by ATLAS in top quark
pair events with $t \rightarrow qH, H \rightarrow \gamma\gamma$~\cite{Aaboud:2017mfd} and $H \rightarrow WW$~\cite{Aaboud:2018pob} at 13
TeV. The former analysis explores the final state with two isolated
photons. For leptonic top quark decays the selection criteria includes
the requirement of one isolated lepton, exactly one b jet, and at least
one non-b jet. In case of hadronic top quark decays the analysis
selects events with no isolated leptons, at least one b jet, and at
least three additional non-b jets. The dominant background processes
are associated with the production of non-resonant $\gamma\gamma$+jets, $t\bar{t}$+jets and
$W$+$\gamma\gamma$ events. 
The resultant limits are $\mathcal{B}(t \to Hu) < 2.4~(1.7) \times
10^{-3}$ and $\mathcal{B}(t \to Hc) < 2.2~(1.6) \times 10^{-3}$.
The search for FCNC in $H \rightarrow WW$ includes the analysis of multilepton final states with either two same-sign or three leptons. The dominant backgrounds arising from the $ttW$, $ttZ$ and non-prompt lepton production are suppressed with a BDT.
The obtained limits are $\mathcal{B}(t \to Hu) < 1.9~(1.5) \times
10^{-3}$ and $\mathcal{B}(t \to Hc) < 1.6~(1.5) \times 10^{-3}$.
The $tHq$ anomalous couplings are probed by CMS in $H \rightarrow
b\bar{b}$ channel in top quark pair events, as well as in single top
associated production with a Higgs boson, at 13
TeV~\cite{Sirunyan:2017uae}. The event selection includes the
requirement of one isolated lepton, at least two b jets, and at least
one additional non-b jet. The dominant $t\bar{t}$ background is
suppressed with a BDT discriminant to set the exclusion limits of
$\mathcal{B}(t \to Hu) < 4.7~(3.4) \times 10^{-3}$ and $\mathcal{B}(t
\to Hc) < 4.7~(4.4) \times 10^{-3}$. Preliminary projections suggest $\mathcal{B}(t \to Hq) < \mathcal{O}(10^{-4})$~\cite{ATL-PHYS-PUB-2016-019, ATL-PHYS-PUB-2013-012}.



\subsection{Anomalous $Wtb$ vertices and \CP-violation effects from $T$-odd kinematic distributions}
{\it \small Authors (TH): Frederic Deliot, Wouter Dekens}

Beyond the SM, contributions to the $Wtb$ vertex have been widely studied in the literature  \cite{Grzadkowski:2008mf,Drobnak:2010ej,AguilarSaavedra:2008zc,GonzalezSprinberg:2011kx,Drobnak:2011aa,Brod:2014hsa,Cao:2015doa,Hioki:2015env,Schulze:2016qas,Kamenik:2011dk,Zhang:2012cd,deBlas:2015aea,Buckley:2015nca,Buckley:2015lku,Bylund:2016phk, Castro:2016jjv,Deliot:2017byp}.
%
Assuming that new physics is much heavier than the electroweak scale, the BSM corrections to the $Wtb$ vertex can be described using an EFT. The first contributions appear at dimension six and, in the conventions of \cite{AguilarSaavedra:2018nen}, take the following form
\begin{equation}\label{eq:Wtb}
\mathcal L_{tb} =-\bar t\left[ \frac{g}{\sqrt{2}}\,\gamma^\mu \big(V_{tb}\big(1+\tfrac{v^2}{2}c_{\varphi q}^{-}\big)P_L+\tfrac{v^2}{2} C_{\varphi tb}P_R\big)W_\mu^+-v\sigma^{\mu\nu}W_{\mu\nu}^+\left(C_{bW}P_R+C_{tW}^*P_L\right)\right]b+{\rm h.c.}\,\,,
\end{equation}
where $C_i = (c_{i}+ic_i^{ [I]})/\Lambda^2$. In writing \eqref{eq:Wtb} we follow Ref.~\cite{Grzadkowski:2008mf} and \Blu{enforce $C_{\varphi q}^{1(33)}+C_{\varphi q}^{3(33)} = 0$ to avoid tree-level FCNC decays of the $Z$ (one also obtains a strong constraint from $Z\to b b$ decays~\cite{Zhang:2012cd})}, such that only the real combination $c_{\varphi q}^-=C_{\varphi q}^{1(33)}-C_{\varphi q}^{3(33)}$ appears.
% in Eq.\ \eqref{eq:Wtb}.

%
The Wilson coefficients in Eq.~\eqref{eq:Wtb} are in terms of the often-used anomalous couplings \cite{AguilarSaavedra:2008zc,AguilarSaavedra:2006fy} given by
\begin{equation}
V_L = V_{tb}^*\big(1+\tfrac{v^2}{2}c_{\varphi q}^{-}\big),\qquad V_R = \tfrac{v^2}{2} C_{\varphi tb}^*,\qquad g_L = -\sqrt{2}v^2 C_{bW}^*,\qquad g_R = -\sqrt{2}v^2 C_{tW}.
\end{equation}
Note that within SMEFT gauge invariance links the above vertices to additional interactions. The full form of the relevant SMEFT operators is given in Sec.~\ref{sec:top:FCNC}.

The above interactions give tree-level contributions to single-top production and to the observables in $t\to W^+b$ decay, in particular, to the $W$-boson helicity fractions. Apart from these processes that are ``directly'' sensitive to the $Wtb$ vertices, there are ``indirect'' observables that receive contributions from the same operators through loop diagrams. Examples are $h\to \bar bb$, $B\to X_s\gamma$, and searches for electric dipole moments (EDMs).
Below we  discuss briefly both the direct and indirect limits on the operators in Eq.\ \eqref{eq:Wtb}, as well as the projections for the HL-LHC (see also Working-Group 1 Section on this topic).

\subsubsection{Direct probes}
%
\begin{table}
\center
\caption{The current  and projected $95\%$ C.L.\ constraints from direct observables on the real and imaginary parts of the Wilson coefficients that contribute to the $Wtb$ vertex, assuming $\Lambda = 1$ TeV and $c_{\varphi q}^-=0$. 
%{\color{red}\bf[Why no sensitivity to $c_{\varphi q}^-$?]}
}
\vspace{.2cm}
\begin{tabular}{llccc}
	\hline\hline
	& Coeff. &$c^{}_{tW}$& $c^{}_{bW}$& $c^{}_{\varphi tb}$ \\\hline
Current&	Re$(\ldots)$ &$[-0.70,\,0.82]$ &$[-2.2,\,2.2]$ &$[-8.9,\,10.9]$  	\\
	& Im$(\ldots)$ &$[-1.5,\,2.2]$ &$[-2.1,\,2.2]$ &$[-9.9,\,9.9]$  
	\\ 
	$3000~{\rm fb}^{-1}$ &	Re$(\ldots)$ &$[-0.23,\,0.58]$&$[-2.2,\,2.0]$&$[-9.3,\,10.6]$\\
 & Im$(\ldots)$ &$[-1.2,\,1.3]$&$[-2.2,\,2.1]$&$[-9.9,\,9.9]$\\
	\hline\hline
\end{tabular}
\label{tabdirectWtb}
\end{table}
%\begin{table}
%\center\small
%\caption{$95\%$ C.L.\ constraints from direct observables on the real and imaginary parts of the Wilson coefficients that contribute to the $Wtb$ vertex. The current  and projected limits are shown in the left and right panel, respectively. We assumed $\Lambda = 1$ TeV and $c_{\varphi q}^-=0$ in both panels. 
%%{\color{red}\bf[Why no sensitivity to $c_{\varphi q}^-$?]}
%}
%\vspace{.2cm}
%\begin{tabular}{lccclccc}
%	\hline\hline
%	Current &$c^{}_{tW}$& $c^{}_{bW}$& $c^{}_{\varphi tb}$ & ~~~~$3000$ fb$^{-1}$ &$c^{}_{tW}$& $c^{}_{bW}$& $c^{}_{\varphi tb}$\\\hline
%	Re$(\ldots)$ &$[-0.70,\,0.82]$ &$[-2.2,\,2.2]$ &$[-8.9,\,10.9]$ & Re$(\ldots)$ &$[-0.23,\,0.58]$&$[-2.2,\,2.0]$&$[-9.3,\,10.6]$
%	\\
%	Im$(\ldots)$ &$[-1.5,\,2.2]$ &$[-2.1,\,2.2]$ &$[-9.9,\,9.9]$ & Im$(\ldots)$ &$[-1.2,\,1.3]$&$[-2.2,\,2.1]$&$[-9.9,\,9.9]$
%	\\
%	\hline\hline
%\end{tabular}
%\label{tabdirectWtb}
%\end{table}
%
The $Wtb$ interactions can be probed directly in single-top production and through the $W$ boson helicity fractions in top decays. In the case of polarized top quarks it is possible to construct additional $T$-odd observables that are sensitive to \CP-violating phases in the $Wtb$ couplings. Here we discuss briefly  these observables and the resulting (projected) constraints.

\smallskip
{\bf Single top production~~}
The single-top production cross section has been measured in the $t$ and $s$ channels at the LHC at $\sqrt{s} = 7$, $8$, and $13$ TeV \cite{Aad:2014fwa,Aaboud:2016ymp,Aaboud:2017pdi,Chatrchyan:2012ep,Khachatryan:2014iya,Sirunyan:2016cdg}. In principle, these cross sections  receive contributions from all $Wtb$ couplings. The measurements can be compared with the SM prediction \cite{Brucherseifer:2014ama} at NNLO in QCD, while BSM contributions have been evaluated at NLO \cite{Zhang:2016omx,Cirigliano:2016nyn,Alioli:2017ce,deBeurs:2018pvs}.
%
In SMEFT the single-top cross sections can receive contributions from other operators, in particular, from several four-quark operators, see, e.g., Refs.~\cite{Buckley:2015lku,Buckley:2015nca}. Thus, in order to perform an exhaustive global analysis one would have to include these effects as well. %For discussions on such operators in the context of single-top production, 

\smallskip
{\bf Top quark decays~~}
The helicity fractions of the $W$ boson in top decays are mostly sensitive to $c_{tW}$, $c_{bW}$, and $c_{\varphi tb}$, and have been measured both at the Tevatron and the LHC \cite{Aaltonen:2012rz,Aad:2012ky,Chatrchyan:2013jna,Aad:2015yem,Khachatryan:2014vma,Aaboud:2016hsq}. In addition, the phase $\delta^-$ between the amplitudes of longitudinally and transversely polarized $W$ bosons, recoiling against a left-handed $b$ quark, carries information on the imaginary part of $C_{tW}$ \cite{Aad:2015yem,Boudreau:2013yna}.
The SM predictions for the helicity fractions are known to NNLO in QCD \cite{Czarnecki:2010gb}, while the BSM contributions have been computed to NLO \cite{Drobnak:2010ej,Zhang:2014rja}.  

As mentioned above, in the case of polarized top-quark decays, it becomes possible to construct additional observables that are sensitive to both the real and imaginary parts of the $Wtb$ couplings. In particular, both the asymmetries constructed in Ref.\ \cite{Aguilar-Saavedra:2015yza,AguilarSaavedra:2010nx} and the triple-differential measurements of Ref.\ \cite{Aaboud:2017yqf} are sensitive to $c_{tW}^{[I]}$ (see also Ref.~\cite{Schulze:2016qas}). As a result, the limits on $c_{tW}^{[I]}$ already improve noticeably when including current experimental measurements of these angular asymmetries \cite{Deliot:2017byp}.

After taking into account the current experimental results on single-top production, the helicity fractions, and angular asymmetries one obtains the current (projected) limits in the left (right) panel of Table \ref{tabdirectWtb} \cite{Deliot:2017byp,Deliot:2018jts}.
These limits were obtained by assuming that $c_{\varphi q}^-=0$.~\footnote{This coupling is harder to constrain as it is degenerate with a shift in $V_{tb}$, see Eq.\ \eqref{eq:Wtb}. This degeneracy can be broken by using CKM unitarity tests, or by considering additional interactions, such as the $Wts$, $Wtd$, and $Ztt$ vertices, that are linked to $c_{\varphi q}^-$ by gauge invariance. A more naive constraint, obtained by setting $V_{tb}=1$, leads to $|c_{\varphi q}^-|\lesssim 1$ \cite{Alioli:2017ce}.}
%
 Comparing the two panels one sees that the current and projected limits on the $c_{bW}$ and $c_{\varphi tb}$ couplings are very similar, while the projected limits on the real and imaginary parts of $c_{tW}$ are roughly a factor of $2$ stronger than the current constraints.

Top decays can also be used to probe physics beyond that of the top sector. For example, Ref. \cite{Harrison:2018bqi} recently suggested that the process $t\to bW\to b\bar b c$ can be used to measure $V_{cb}$ at the $m_W$ scale, instead of $V_{cb}(m_b)$ which is probed in $B$ decays.






\subsubsection{Indirect probes}
\label{sec:8:2:Indirect}
\begin{table}
\center
\caption{Indicative constraints on the SMEFT operators contributing to the $Wtb$ vertex from indirect observables. A single Wilson coefficient is taken to be nonzero at a time, and we set $\Lambda = 1$ TeV.}
\vspace{.2cm}
\begin{tabular}{ccclll}
\hline\hline
Coeff. &$h\to \bar bb$ & EWPT&$B\to X_s\gamma$ &EDMs\\\hline
$c^{-}_{\varphi q}$&$-$&$-$&$[-2.0,2.2]$ & $-$ \\
$c^{}_{\varphi tb}$&$[-0.3,4.3]$ & $-$&$[-4.6,4.9]\cdot 10^{-2}$&$-$\\
$c^{}_{tW}$&$-$&$[-1.1,0.7]$&$[-0.22,0.89]$&  $-$\\
 $c^{}_{bW}$ &$-$&$-$& $[-13,3.5]\cdot 10^{-3}$& $-$\\
$c^{[I]}_{\varphi tb}$&$[-7.3,7.3]$ &$-$&$[-0.20,0.20]$&$[-0.019,0.019]$ \\
$c^{[I]}_{tW}$&$-$&$-$& $[-2.4, 4.5]$ & $[-1.0,1.0]\cdot 10^{-3}$\\
 $c^{[I]}_{bW}$ &$-$&$-$& $[-4.3, 2.3] \cdot 10^{-2}$& $[-5.5,5.5] \cdot 10^{-4}$\\
 \hline\hline
\end{tabular}\label{tabindirectWtb}
\end{table}
%
The anomalous $Wtb$ interactions contribute to other processes through loop diagrams.  Although this can give rise to stringent constraints, their interpretation requires some care. As indirect observables receive contributions from additional operators apart from the $Wtb$ couplings, cancellations between different Wilson coefficients are possible. 
We will assume this is not the case and derive limits for the case that a single dimension-six operator is generated at the BSM scale.

\smallskip
{\bf Electric dipole moments.~}
Electric dipole moments are probes of \CP violation and, therefore, receive contributions from the imaginary parts of the $Wtb$ couplings. 
The most stringent experimental limits have been obtained on the EDMs of the neutron, mercury, and the ThO molecule, the latter of which can be interpreted as a limit on the electron EDM for our purposes.
To obtain the contributions to EDMs one first has to evolve the $Wtb$ operators to low energies, $\mu\sim 2$ GeV. QCD becomes non-perturbative below this scale, and one has to match to Chiral perturbation theory, which describes the \CP-odd interactions in terms of hadrons, photons, and electrons. These interactions can then be used to calculate the EDMs of nucleons, atoms, and molecules.

Among the operators in Eq.\ \eqref{eq:Wtb}, the $c_{\varphi tb}^{[I]}$ and $c_{bW}^{[I]}$ couplings are mainly constrained by the neutron EDM, while $c_{ tW}^{[I]}$ contributes to the electron EDM. The $c_{\varphi tb}^{[I]}$ and $c_{bW}^{[I]}$ first generate the bottom-quark chromo EDM,  $O_{dG}^{(33)}$ (in the notation of \cite{Grzadkowski:2008mf}), through one-loop diagrams \cite{Dekens:2013zca,Alonso:2013hga,Alioli:2017ce}, which subsequently induces the Weinberg operator, $O_{\tilde G}$, after integrating out the bottom quark \cite{Dicus:1989va,Weinberg:1989dx,BraatenPRL,Boyd:1990bx}. The hadronic matrix element of the Weinberg operator contributing to the neutron EDM is poorly known. Combining naive-dimensional-analysis and sum-rule estimates \cite{Pospelov_Weinberg,Weinberg:1989dx,deVries:2010ah}, one has $ |d_n|  = 6|e\,(50\, {\rm MeV})\, C_{\tilde G}(1\, {\rm GeV})|$, with an $\mathcal O(100\%)$ uncertainty.


The contributions to the electron EDM also arise from a two-step process: $c_{ tW}^{[I]}$ first induces \CP-odd operators of the form $X_{\mu\nu }\tilde X^{\mu\nu }\varphi^\dagger\varphi$, with $X_{\mu\nu}$ a $SU(2)_L$ or $U(1)_Y$ field strength, as well as semileptonic interactions of the form, $(\bar e_L \sigma_{\mu\nu}\,e_R) \, (\bar t_L\sigma^{\mu\nu}t_R)$, through the renormalization group equations \cite{Cirigliano:2016nyn,Cirigliano:2016njn,Fuyuto:2017xup}. These additional operators then induce the electron EDM at one loop. 
%
Using the above contributions and the current experimental limits \cite{Afach:2015sja,Baker:2006ts,Andreev:2018ayy}, we obtain the constraints in Table \ref{tabindirectWtb}.

\smallskip
{\bf Rare $B$ decays.~}
Unlike EDMs, measurements of $B\to X_s\gamma$ are sensitive to both the real and imaginary parts of the couplings. Although all four of the $Wtb$ vertices give rise to  flavour-changing $b\to s$ transitions through one-loop diagrams \cite{Grzadkowski:2008mf,Brod:2014hsa,Aebischer:2015fzz}, the largest effects are due to $C_{\varphi tb}$ and $C_{bW}$. Both of these couplings induce contributions proportional to $m_t$ instead of $m_b$ that appears in SM (as well as for $c^-_{\varphi q}$ and $C_{tW}$), leading to a relative enhancement of $m_t/m_b$. Here we consider the constraints from measurements of the $B\to X_s\gamma$ branching ratio and the \CP asymmetry \cite{Amhis:2016xyh}, for which we use the theoretical expressions of \cite{Hurth:2003dk} and \cite{Benzke:2010tq}, respectively. This leads to the constraints  in Table \ref{tabindirectWtb}.

\smallskip
{\bf Electroweak precision tests.~}
The $Wtb$ operators also modify the self energies of the SM gauge bosons through one-loop diagrams, which are often parametrized by the $S$, $T$, and $U$ parameters \cite{Peskin:1990zt,Peskin:1991sw,Barbieri:2004qk}.
Taking into account the RGE contributions, only $c_{tW}$ induces the $S$ parameter by mixing with the $O_{\varphi WB}$ operator ($c_{bW}$ contributions are proportional to $m_b$), which leads to the limits in Table \ref{tabindirectWtb}. Here we assumed only a single dimension-six operator is present at $\mu = \Lambda$, but one can include the couplings of the operators that induce $S$ and $T$ at tree level, $O_{\varphi WB}$ and $O_{HD}$, and marginalize over them. This can be done because electroweak precision observables carry more information than is captured by the $S$, $T$, and $U$ parameters alone. This approach is discussed Ref.\ \cite{Greiner:2011tt,Zhang:2012cd} and leads to weaker limits, $c_{tW}\in [-1.6,0.8]$, $c_{bW}\in [-2,24]$, $c_{\varphi q}^-\in [-0.7,4.7]$ for $\Lambda= 1\,{\rm TeV}$.


\smallskip
{\bf Higgs decays.~}
Finally, the $Wtb$ interactions contribute to the process $h\to \bar bb$ by inducing corrections to the SM bottom-quark Yukawa coupling \cite{Alioli:2017ce}. In particular, the $C_{\varphi tb}$ coupling generates a contribution to the Yukawa coupling that scales as $y_b\sim {y_t} v^2C_{\varphi tb}/{(4\pi)^2}$. Thus, although this contribution only appears at one loop, the suppression is offset by the appearance of the top-quark Yukawa instead of that of the bottom quark. Using the combined ATLAS and CMS analysis \cite{Khachatryan:2016vau} of the Higgs decays to $\gamma\gamma$, $WW$, $ZZ$, $\tau\tau$, $\mu\mu$, and $\bar bb$ signal strengths we obtain the limits in Table \ref{tabindirectWtb}.


\subsection{Determinations of $V_{tx}$} 
{\it\small Authors (TH): Mariel Estevez, Darius Faroughy, Jernej Kamenik }
%
% 
%Fabio Maltoni

%%%%%%%%%%%%%%%%%%%%%%%%%%%%%%%%%
%
\subsubsection{Measuring $|V_{td}|$ at HL-LHC and HE-LHC}
%Jernej Kamenik
%Mariel Estevez,
%
%%%%%%%%%%%%%%%%%%%%%%%%%%%%%%%%%

A possible experimental strategy to probe the $|V_{td}|$ matrix element directly at the HL(HE)-LHC is using single top production associated with a $W$ boson, $pp \rightarrow tW$. The idea is to exploit the production cross-section enhancement, as well as boosts of the top quarks coming from initial state valence $d$-partons. The $d$-quark is a valence constituent of the proton and there is an imbalance with the $\bar{d}$-quark that motivates to explore charge asymmetries as possible $V_{td}$-sensitive observables.
The main backgrounds, contrary to $t$-channel single top production, are charge symmetric or have very small charge asymmetries. 
The $dg \to tW$ associated production process is interesting because of its sizeable charge asymmetry in proton collisions and also because its kinematics predicts a characteristic angular distribution. We expect a relatively large incoming momentum on average from the valence $d$-quark. Consequently, a forward $W^-$ is preferred in the lab frame, which is supposed to produce a forward $\ell ^-$ in signal events. The main two backgrounds to the $\ell ^+ \ell ^- b E_T^{\rm{miss}}$ final state are the dileptonic $t \bar{t}$ production (missing one of the $b$-jets from top decays) and $g b \to tW$ associated production, proportional to $|V_{tb}|^2$. Both backgrounds have very small charge asymmetries. In order to increase the sensitivity and enhance the signal cuts can be imposed that reduce the cross-sections of the backgrounds, see Ref.~\cite{Alvarez:2017ybk} for details. 
%The details of all the considered backgrounds, the events selection, and simulations employed can be found in . 
%We observe that t
The most important difference between signal and background comes from the $\eta (\ell ^-)$ distribution, where the signal clearly prefers forward negatively charged leptons. The asymmetry
\begin{equation}
A(\eta,p_T) = \frac{N^+ - N^-}{N^+ + N^-},
\label{A}
\end{equation}
where
$
N^\pm = N\left( \Delta |\eta(\ell)| \gtrless 0 \,\& \, \Delta p_T(\ell) \gtrless 0 \right),
%\label{i-th contribution}
$
is a $|V_{td}|$ sensitive observable. Each process contributes with
%\begin{equation}
$N^+_i - N^-_i = \sigma_i \cdot {\cal A}_i \cdot \epsilon_i \cdot A_i(\eta,p_T)$,
%\end{equation}
where the factors on the right hand side are the cross-section, acceptance, selection efficiency and asymmetry, respectively. 
%We parameterize eventual departures from SM in $V_{td}$ through the ratio
%\begin{equation}
%r \equiv \left| \frac{V_{td}}{V_{td}^{\rm SM}}\right|\,,
%\end{equation}
%in order to classify processes according to their leading power in $r$. 
To quantify the versatility of the proposed charge asymmetry we study the prospective experimental reach in $r\equiv |V_{td}/V_{td}^{\rm SM}|$ by computing the difference of $A(\eta, p_T)$ to its SM expectation in units of the uncertainty. Based on existing experimental studies of charge asymmetries in top production~\cite{Khachatryan:2016ysn} we include an estimate for the systematic uncertainty of $\Delta _{sys}=0.2 \%$, and 
define the significance as
\begin{equation}
\mbox{significance} = {\left| A(\eta,p_T) - A(\eta,p_T)^{\rm SM}\right| }/{\sqrt{(N^+ + N^-)^{-1} + \Delta_{\rm syst}^2}} .
\end{equation}

\begin{figure}[t]
\centering
\includegraphics[width=0.45\textwidth]{section8/figs/contour_r-L.pdf}
\hspace{7mm}
\includegraphics[width=0.45\textwidth]{section8/figs/contour_r-L_27.pdf}
%\hspace{3mm}
\caption{Contour lines for the projected $2\sigma$, $3\sigma$ and $5\sigma$ upper bounds on $|V_{td}|$ (and $r\equiv|V_{td}/V_{td}^{\rm SM}|$) as functions of the LHC luminosity at $13$ TeV (left)  and $27$ TeV (right). Dashed lines follow if  the dominant $t\bar t$ background is reduced by half, see text for details.} 
\label{figure:Vtd}
\end{figure}

Fig.~\ref{figure:Vtd} shows the contours of expected experimental significance for $A(\eta,p_T)$ (and $r$) as functions of luminosity, for $13$ TeV and $27$ TeV LHC. As a rough guidance we also show with dashed lines the significance for the case that the dominant $t \bar{t}$ background were further reduced by a factor of $2$, e.g., by using multivariate discrimination techniques as already done in existing single top analyses~\cite{Chatrchyan:2014tua, Aad:2015eto, Aaboud:2016lpj}.
% (in dashed lines). 
Values of $r<10$ could be directly accessible at the LHC@13TeV, improving the existing best direct constraint~\cite{Khachatryan:2014nda} by roughly a factor of three. The current direct bound on $|V_{td}|$ can be surpassed with the Run 2 dataset, while with $3000\, {\rm fb}^{-1}$ it could be possible to probe $|V_{td}|\sim |V_{ts}^{\rm SM}| \simeq 0.04$. 

The dominant $t \bar{t}$ background is mostly generated via gluon fusion, which is charge symmetric. There are subleading contributions from processes like $u \bar{u} \rightarrow t \bar{t}$ which are charge asymmetric, but only enter at higher orders in QCD. This is the main reason the dominant $t \bar{t}$ background has a strongly suppressed charge asymmetry. At higher collision energies, the probability of finding energetic enough gluons in the proton increases faster than that of valence quarks. Consequently, the fraction of $t\bar t$ events from the quark-antiquark initial state is reduced~\cite{Khachatryan:2016ysn}.  This leads to a shrinking charge asymmetry with growing collision energy. %A further significant reduction in systematic uncertainties below our current estimate of $0.2 \%$ could allow the (HL)LHC eventually to probe values as low as $|V_{td}| \sim 0.06$ at $13$ TeV. 
Unfortunately, the same happens to the signal, this time because at higher energies the asymmetry between $d$ and $\bar{d}$ partons inside the proton is reduced. 
%Combined with the growing contribution of the (mostly charge symmetric) $t\bar t$ background to the denominator, the end effect is severely diminishing t
The net effect is a severely diminished significance at $27$\,TeV compared to 13\,TeV for comparable luminosities. 
%On the other hand, at $7$ TeV the significance does increase and with $3000 fb^{-1}$, values of $r<10$ could be accesible with $3 \sigma$ and values of $r \sim 10$ with $5 \sigma$.


%%%%%%%%%%%%%%%%%%%%%%%%%%%%%%%%%
%
\subsubsection{Measuring Cabibbo-suppressed decays of the top quark at HL-LHC and HE-LHC}
%Jernej Kamenik
%Darius Faroughy,
%
%%%%%%%%%%%%%%%%%%%%%%%%%%%%%%%%%

	With the large $t\bar t$ statistics at the LHC, one might attempt a direct measurement of Cabibbo-suppressed decays of the top quark, $t\to (s,d)\,W^\pm$, in leptonic $t\bar t$ events. Since there is no practical way to distinguish between strange and down quark jets at the detector level (without dedictated PID systems), one measures 
\begin{equation}
\rho\equiv\sqrt{\big(\BR(t\to sW)+\BR(t\to dW)\big)/\BR(t\to bW)}\,.
\end{equation}
This gives
%the method described here could 
a direct information on $({|V_{ts}|^2+|V_{td}|^2})^{1/2}$, but not on $V_{ts}$ and $V_{td}$ separately. 
%by measuring the ratio of branching fractions
%This quantity measures the hierarchy between diagonal and off-diagonal entries in the third row of the CKM matrix and is supposed to be approximately given by
 In the SM, $\rho\approx |V_{ts}|\approx0.04$. 

%In order to measure top decays into light quarks, 
To perform the measurement it is necessary to discriminate between heavy-flavoured jets, e.g., $b$-jets from the $t\to b W$ background, gluons from ISR/FSR contamination, and the signal -- the light-quark jets ($q$-jets) from the Cabibbo-suppressed top decays. This can be achieved through a $q$-tagger based on existing techniques used for $b$-tagging and quark/gluon jet discrimination~\cite{Gallicchio:2011xq}. There are several known useful observables which can be used in a $q$-tagger. In the following we use: (i) the multiplicity $N_{\mathrm{SV}}$ of secondary vertices (SV) in the jet within a fiducial volume of the tracker, (ii) the fraction of longitudinal momentum of the jet carried by the hardest prompt charged track 
%\begin{equation}\label{zmax}
$ z_{\mathrm{max}}\equiv\mathrm{max}\big[{\vec{p}_{x}\cdot\vec{p}_{\text{jet}}}/{|\vec{p}_\text{jet}|^2}\big]_{x\in\mathrm{jet}}$,
%\end{equation}
%with $x$ the reconstructed prompt track, 
and (iii) the 2-point energy correlation function 
% defined as
%\begin{align}
$U^\beta_1=\sum_{i,j\in \mathrm{jet}} z_T^i z_T^j\,(R_{ij})^\beta$, $z_T^i\equiv p_T^i /p_T^{\mathrm{jet}}$,
%\end{align}
where $R_{ij}^2=(\eta_i-\eta_j)^2+(\phi_i-\phi_j)^2$ and $\beta$ is a free real parameter for which quark/gluon discrimination is optimized at $\beta=0.2$~\cite{Larkoski:2013eya,Moult:2016cvt}, see
%. For more details on these observables in the context of the $q$-tagger we refer the reader to~
Ref.\cite{thePaper} for more details. 
%The resulting performance of the $q$-tagger in simulated samples of $pp\to Zj$ for $j=s,b,c,g$ and $Z\to\mu\bar\mu$ in a realistic LHC environment at $13$\,TeV is presented in Fig.~\ref{figROC}, where we show the ROC curves for the $q$-tagger obtained by sliding the values of the cuts $z_{\mathrm{max}}$ (solid lines) or $U_1$ (dashed lines), independently. 
%\begin{figure}[t]
 % \vspace{-0.3cm}
%\begin{center}{
%\includegraphics[width=8cm]{section8/figs/quark-gluon_ROC.pdf}
%%  \vspace{-0.3cm}
%\caption{{ ROC curves for the two $q$-taggers}} \label{figROC}}
%\end{center}
%\end{figure}
%\paragraph{light-quark content in top decays}
%Equipped with the light-quark tagger we are now in position to study the flavour content of top quark decays from $t\bar t$ production. 

%%%%%%%%%%%%%%%%%%%%%
%\begin{table}[htbp!]
\begin{table}[t!]
\begin{center}
% \small
 {
%\renewcommand{\arraystretch}{1}
\centering
\caption{ Tagging and mis-tagging efficiencies for the $q$-tagging working point used in the $V_{tq}$ analysis.}
\label{tab:effs} 
%\footnotesize
\begin{tabular}{cccccccc}
\hline\hline 
(t) type & Cuts &\  $\epsilon^{\mathrm t}_{q}$\ &\ $\epsilon^{\mathrm t}_{b}$\ &\ $\epsilon^{\mathrm t}_{c}$\ &\ $\epsilon^{\mathrm t}_{g}$
\\ 
\hline
$q$-tagger &$N_{\mathrm SV}=0$ \& $z_{\mathrm max}>0.3$    &  $0.18$ & $0.0031$ & $0.038$ & $ 0.049$ \\
$b$-tagger & $N_{\mathrm SV}>3$     		                          &  $0.0091$ & $0.64$ & $0.09$ & $0.016$ \\ 
\hline\hline
\end{tabular}
}
\end{center}
\end{table}

	%First we generated MC samples of $t\bar t$ and $tW$ plus jets in {\tt Madgraph 5} at LO and decayed the tops into both $t\to bW$ and $t\to (s,d)W$ channels with the W-boson decaying leptonically and SM values for the CKM matrix. Jet matching, parton showering and hadronization were performed in {\tt pythia8} and jet clustering with {\tt FastJet}. 

For the projections we use the reference working point 
%for the utilized taggers is 
shown in Table~\ref{tab:effs}.	
%	A possible extraction of $\rho$ using leptonic $t\bar t$ events then proceeds as follows.
%	After pre-selection requirements, two selected highest \pt jets in an event are subjected to the $b$- and $q$-taggers (for the working points in Table \ref{tab:effs}). The results are b
	We bin preselected events into one of the six tagged dijet categories $\{\jmath\jmath,\jmath b,\jmath q, qb,qq,bb\}$ where $q$, $b$ and $\jmath$ represent $q\,$-jets, $b\,$-jets and non-tagged jets (a jet failing both taggers), respectively. 
%	A fit to event yields in all the categories can then be used to extract $\rho$ together with the ISR/FSR contamination fractions. 
	%Solely for the purpose of illustration we present  in 
	Fig.~\ref{fig:master} shows the resulting upper limits on $\rho$ taking into account only statistical uncertainties in the $qb$ category for the signal significance $S/\sqrt{B}$ at $2\sigma$ (solid boundary) and $5\sigma$ (dashed boundary) as a function of the LHC luminosity, compared to the current best limit by CMS shown by the gray dashed curve. The results suggest that the HL-LHC could find evidence of Cabbibo-suppressed top decays and determine $V_{ts}$ CKM element directly, though the precise precision depends crucially on how well systematic uncertainties can be controlled. %For a more detailed study of these prospects we refer the reader to Ref.~\cite{thePaper}
	The uncertainties in $q$- and $b$-tagging efficiencies could be controlled using $Zj$ production. Then a fit of the categorized dijet data $\{\jmath\jmath,\jmath b,\jmath q, qb,qq,bb\}$ in inclusive dilepton events to a probabilistic model taking into account the tagging efficiencies could be performed, similar to that performed in \cite{Silva:2010qt,Khachatryan:2014nda} for the extraction of $V_{tb}$. Especially relevant for the HE-LHC would be to measure $t\to (s,d)W$ from boosted semi-leptonic $t\bar t$ events. Events with one top-tagged fat jet, one narrow jet and one lepton (both daughter candidates of the leptonic top) can be categorized by the flavour content of the narrow jet. A preliminary analysis~\cite{thePaper} suggests comparable sensitivity to the leptonic $t\bar t$ dataset already at the 13TeV LHC. 


\begin{figure}[t]
%  \vspace{-0.3cm}
\begin{center}{
\includegraphics[width=8cm]{section8/figs/LHC_Resolved_lep_tops.pdf} 
%\vspace{-0.3cm}
\caption{{Illustration of statistical limits on the ratio $\rho \simeq \big(|V_{ts}|^2 + |V_{td}|^2\big)^{1/2}/|V_{tb}|$ from an analysis of Cabbibo-suppressed top decays in leptonic $t\bar t$ events at 13TeV LHC. See text for details.}} \label{fig:master}}
\end{center}
\end{figure}


%\paragraph{Outlook}
%For more reliable (HL/HE-)LHC projections it is necessary to carefully categorize and estimate the main sources of systematic uncertainties. The dominating source of uncertainties is expected to come from the extraction and mapping of The uncertainties in $q$- and $b$-tagging efficiencies could be controlled using $Zj$ production.
%% onto $t\bar t$ events. 
%A complete assessment of these uncertainties is outside the scope of this report. Nevertheless, once these uncertainties are extracted from experimental data the following step is to perform a fit of the categorized dijet data $\{\jmath\jmath,\jmath b,\jmath q, qb,qq,bb\}$ in inclusive dilepton events to a probabilistic model taking into account the tagging efficiencies, similar to that performed in \cite{Silva:2010qt,Khachatryan:2014nda} for the extraction of $V_{tb}$. For further details on this search strategy see Ref.~\cite{thePaper}.


%As a final note, given the small signal-to-background ratio of Cabibbo-suppressed top decays, advanced machine learning approaches to data analysis should be able to further improve upon these simple search strategies.


\end{document}
