%
\documentclass[../report.tex]{subfiles}
\providecommand{\main}{..}
\IfEq{\jobname}{\currfilebase}{\AtEndDocument{\biblio}}{}
%

%
%
%
%
%

\begin{document}

\section{The Higgs boson and flavour physics}
{\it\small Authors (TH): F. Bishara, R. Harnik,  A. Martin, M. Schlafer, Y. Soreq, E. Stamou, F. Yu.}
%
\label{sec:9}
%
%
%
%
%
%
%
%
%
%
%
%
%
%
%
%
%
%
%
%


%
%
%
%
%
%


%

%

%
In the SM the Higgs couplings to the fermions are the origin of the flavour structure. In the SM the  
%
Yukawa couplings, $y_f$, are \CP conserving and proportional to the fermion masses, $m_f$, with a common proportionality factor, 
%
%
\begin{align}
\label{eq:SM:Yukawa:proportionality}
    y^{\rm SM}_f = \sqrt{2} m_f / v \, ,
\end{align}
%
while the tree-level flavour changing couplings are zero. 
Currently, only the third generation Yukawa couplings have been measured and found to be in agreement with the SM predictions, see Refs.~\cite{Aaboud:2017jvq,Aaboud:2018urx,Sirunyan:2018shy,Aad:2015vsa,Aaboud:2018pen,Sirunyan:2018kst,Aaboud:2018zhk}.
%
For the Higgs couplings to the first two generations, only upper bounds exist at present~\cite{Aaboud:2018fhh,Aaboud:2017ojs,Perez:2015aoa,Altmannshofer:2015qra,Kagan:2014ila}.

To parametrize the deviations from the SM, 
%
it is useful to introduce
%
a generalized $\kappa$ framework (the summation is over fermion type $f=u,d,\ell$ and generations $i,j=1,2,3$)
%
%
%
\begin{equation}
\label{eq:L_eff}
\begin{split}
	\mathcal{L}_{\rm eff} &= -\kappa_{f_i} \frac{m_{f_i}}{v} h \bar f_i f_i + i \tilde \kappa_{f_i} \frac{m_{f_i}}{v} h \bar f_i \gamma_5 f_i  
	- \left[\left( \kappa_{f_i f_j} + i \tilde \kappa_{f_i f_j} \right) h \bar f_L^i f_R^j +{\rm h.c.}\right]_{i\neq j}.
\end{split}
\end{equation}
%
%
Experimentally, we want to test a number of SM predictions: {\em (i)} proportionality, $y_f\propto m_f$; {\em (ii)} the factor of proportionality, $\kappa_{f_i}=1$; {\em (iii)} diagonality (no off-diagonal flavour violation), $\kappa_{{f_i}{f_j}}=0$; {\em (iv)} reality (no \CP violation), $\tilde
\kappa_{f_i}=\tilde \kappa_{{f_i}{f_j}}=0$ \cite{Nir:2016zkd}.
%
%
%
%
%
Different Higgs Yukawa couplings are probed both directly and indirectly. 
Direct methods are: for top Yukawa the $t\bar{t}h$ production~\cite{Aaboud:2017jvq,Aaboud:2018urx,Sirunyan:2018shy}); for bottom and charm, $pp\to Vh, h\to b\bar{b}, c\bar{c} $~\cite{Aaboud:2018zhk,Aaboud:2018fhh,Sirunyan:2018kst}; for leptons $pp\to h\to \ell^+ \ell^-$~\cite{Aaboud:2018pen,Aaboud:2017ojs,Aad:2015vsa}; for light quarks the exclusive $h\to V\gamma$ decays~\cite{Aad:2015sda,Khachatryan:2015lga,Aaboud:2018txb,Aaboud:2017xnb}. The 
%
%
%
kinematical distributions~\cite{Soreq:2016rae,Bishara:2016jga} and
global fits to all the Higgs data also provide bounds on different Yukawa couplings.

%

The Higgs production and decay signal strengths from the CMS collaboration~\cite{Sirunyan:2018koj}, and from ATLAS for $h\to c\bar{c}$~\cite{Aaboud:2018fhh} from a global fit, which includes the direct observation of $t\bar t h$ production, 
%
%
gives for the signal strengths
 $\mu_{tth}= 1.18^{+0.30}_{-0.27}, \mu_{bb}=1.12^{+0.29}_{-0.29}, 
    \mu_{cc} < 105,  \mu_{\tau\tau} = 1.20^{+0.26}_{-0.24}, \mu_{\mu\mu}=0.68^{+1.25}_{-1.24}$.
%
In terms of modifications of the flavour-diagonal \CP-conserving Yukawas,  the best fit values are, see also \cite{CMS:2015kwa,CMS:2018lkl},
\begin{equation}
\label{eq:kappalimits}
    \kappa_t = 1.11^{+0.12}_{-0.10}, \qquad \kappa_b = -1.10^{+0.33}_{-0.23},\qquad
    \kappa_\tau = 1.01^{+0.16}_{-0.20}, \qquad \kappa_\mu = 0.79^{+0.58}_{-0.79}.
\end{equation}
The $u$, $d$, $s$, and charm Yukawa couplings can be constrained from a global fit to the Higgs data and the precision electroweak measurements at LEP. Floating all the couplings in the fit gives~ \cite{Perez:2015aoa,Kagan:2014ila},
\begin{align}
\kappa_u&<3.4 \cdot 10^{3},  &\kappa_d&<1.7 \cdot 10^3, &\kappa_s&< 42, & \kappa_c \lesssim 6.2.
\end{align}
The upper bound on $\BR(h\to e^+e^-)$ gives for the electron Yukawa $|\kappa_e|<611$~\cite{Khachatryan:2014aep,Altmannshofer:2015qra}.  
%
The upper bounds on $\kappa_{c,s,d,u}$ roughly
correspond to the size of the SM bottom Yukawa coupling and are thus much bigger than the corresponding SM Yukawa couplings. The upper bounds can be saturated only if one allows for large cancellations between the
contribution to fermion masses from the Higgs vev and an equally large contribution from NP, but
with an opposite sign. 
%
In models
of NP motivated by the hierarchy problem, the effects of NP are
generically well below these bounds.

%

 In the rest of this section we first briefly discuss the expected deviations in a set of NP models, and give the prospects for HL-/HE-LHC (see also~\cite{Perez:2015lra,Brivio:2015fxa,Koenig:2015pha,Aad:2015sda,Bodwin:2014bpa,Bodwin:2013gca}). An expanded version of the discussion is available in write-up of WG2, Sec.~7. 
 %

%






\subsection{New Physics benchmarks for modified Higgs couplings}

The expected sizes of effective Yukawa couplings,
$\kappa_f$, $\tilde \kappa_f$ and $\kappa_{ff'}, \tilde \kappa{ff'}$, in popular models of weak scale NP models, some of them motivated by the hierarchy problem are shown in  
Tables~\ref{tab:upyukawa} and \ref{tab:upFVyukawa}, 
%
%
adapted from
\cite{Bishara:2015cha,Dery:2014kxa,Dery:2013aba,Dery:2013rta,Bauer:2015kzy}.
The predictions are shown for the Standard Model, multi-Higgs-doublet models
(MHDM) with natural flavour conservation (NFC)~\cite{Glashow:1976nt,
  Paschos:1976ay}, a ``flavourful'' two-Higgs-doublet model beyond NFC
(F2HDM)~\cite{Altmannshofer:2015esa, Altmannshofer:2016zrn, 
Altmannshofer:2017uvs, Altmannshofer:2018bch} the MSSM at tree level, 
a single Higgs doublet with
a Froggat-Nielsen mechanism (FN)~\cite{Froggatt:1978nt}, the
Giudice-Lebedev model of quark masses modified to 2HDM
(GL2)~\cite{Giudice:2008uua}, NP models with minimal flavour violation
(MFV)~\cite{D'Ambrosio:2002ex}, Randall-Sundrum models
(RS)~\cite{Randall:1999ee}, and models with a composite Higgs where
Higgs is a pseudo-Nambu-Goldstone boson (pNGB)~\cite{Dugan:1984hq,
  Georgi:1984ef, Kaplan:1983sm, Kaplan:1983fs}. Tables~\ref{tab:upyukawa} and \ref{tab:upFVyukawa} only show predictions for up-quark sector, while the results for down-quark and lepton sectors can be found in (see also Sec.~7 of WG2 write-up). 
  
 
  
%
%
%

%
%

\begin{table}[t]
\begin{center}
\begin{tabular}{l  c  c  c c  }
\hline\hline
%
Model	& $\kappa_t$ & $\kappa_{c (u)}/\kappa_t$  & $\tilde \kappa_t/\kappa_t$ & $\tilde \kappa_{c (u)}/\kappa_t$ \\ %
\hline
SM	& 1	& 1 & 0 & 0 \\
MFV &$1+\frac{\Re(a_uv^2+2b_u m_t^2)}{\Lambda^2}$
&$1-\frac{2\Re(b_u)m_t^2}{\Lambda^2}$
&$\frac{\Im(a_uv^2+2b_u m_t^2)}{\Lambda^2}$ & $\frac{\Im(a_u
  v^2)}{\Lambda^2} $ \\
NFC & $V_{hu}\,v/v_u$	& 1 &  0 &0 \\
F2HDM & $\cos\alpha/\sin\beta$	& $-\tan\alpha/\tan\beta$ & $\mcO\left(\frac{m_c}{m_t}\frac{\cos(\beta-\alpha)}{\cos\alpha\cos\beta}\right)$ & $\mcO\left(\frac{m_{c(u)}^2}{m_t^2} \frac{\cos(\beta-\alpha)}{\cos\alpha\cos\beta}\right)$ \\
MSSM	& $\cos\alpha/\sin\beta$	&1 &0  &0\\
FN & $1+\mcO\left(\frac{v^2}{\Lambda^2}\right)$ &
	$1+\mcO\left(\frac{v^2}{\Lambda^2}\right)$ &
	$\mcO\left(\frac{v^2}{\Lambda^2}\right)$ &
	$\mcO\left(\frac{v^2}{\Lambda^2}\right)$ \\
GL2 	& $\cos\alpha/\sin\beta$& $\simeq 3(7)$ & 0 & 0 \\
RS &$1-{\mathcal O}\Big(\frac{ v^2}{m_{KK}^2}\bar Y^2\Big)$&$1+{\mathcal O}\Big(\frac{ v^2}{m_{KK}^2}\bar Y^2\Big)$ &${\mathcal O}\Big(\frac{ v^2}{m_{KK}^2}\bar Y^2\Big)$ &${\mathcal O}\Big(\frac{ v^2}{m_{KK}^2}\bar Y^2\Big)$ \\
pNGB & $1+{\mathcal O}\Big(\frac{ v^2}{f^2}\Big)+{\mathcal O}\Big(y_*^2 \lambda^2 \frac{ v^2}{M_*^2}\Big)$ & $1+{\mathcal O}\Big(y_*^2 \lambda^2 \frac{ v^2}{M_*^2}\Big)$ & ${\mathcal O}\Big(y_*^2 \lambda^2 \frac{ v^2}{M_*^2}\Big)$ & ${\mathcal O}\Big(y_*^2 \lambda^2 \frac{ v^2}{M_*^2}\Big)$ \\
\hline\hline
%
\end{tabular}
\caption{Predictions for the flavour-diagonal up-type Yukawa couplings
  in a sample of NP models (see text for details).
}
\label{tab:upyukawa}
\end{center}
\end{table}

%
%
%
%
%
%
%
%
%
%
%
%
%
%
%
%
%
%
%
%
%
%
%
%
%
%
%
%
%

%
%
%
%
%
%
%
%
%
%
%
%
%
%
%
%
%
%
%
%
%
%
%
%
%
%
%
%
%


\begin{table}[t]
\begin{center}
\begin{tabular}{l  c  c  c }
%
\hline\hline
Model	& $\kappa_{ct (tc)}/\kappa_t$ & $\kappa_{ut (tu)}/\kappa_t$  & $\kappa_{uc (cu)}/\kappa_t$ \\ 
%
\hline
MFV &$ \frac{\Re\big( c_u m_b^2 V_{cb}^{(*)}\big)}{\Lambda^2}\frac{\sqrt2 m_{t(c)}}{v} $~&~ $ \frac{\Re\big( c_u m_b^2 V_{ub}^{(*)}\big)}{\Lambda^2} \frac{\sqrt2 m_{t(u)}}{v}$~&~ $ \frac{\Re\big( c_u m_b^2 V_{ub(cb)}V_{cb(ub)}^{*}\big)}{\Lambda^2} \frac{\sqrt2 m_{c(u)}}{v}$\\\vspace{0.15cm}
F2HDM & $\mcO\left(\frac{m_c}{m_t}\frac{\cos(\beta-\alpha)}{\cos\alpha\cos\beta}\right)$ & $\mcO\left(\frac{m_u}{m_t}\frac{\cos(\beta-\alpha)}{\cos\alpha\cos\beta}\right)$ & $\mcO\left(\frac{m_c m_u}{m_t^2}\frac{\cos(\beta-\alpha)}{\cos\alpha\cos\beta}\right)$ \\\vspace{0.15cm}
FN &  $\mathcal{O}\left(\frac{v m_{t(c)}}{\Lambda^2} |V_{cb}|^{\pm 1}\right)$ &
	$\mathcal{O}\left(\frac{v m_{t(u)}}{\Lambda^2} |V_{ub}|^{\pm 1}\right)$ &
	$\mathcal{O}\left(\frac{v m_{c(u)}}{\Lambda^2} |V_{us}|^{\pm 1}\right)$\\\vspace{0.15cm}
GL2	& $\epsilon (\epsilon^2)$ & $\epsilon (\epsilon^2)$ & $\epsilon^3$ \\\vspace{0.15cm}
RS & $\sim \lambda^{(-)2} \frac{m_{t(c)}}{v} \bar Y^2\frac{v^2}{m_{KK}^2} $&$\sim \lambda^{(-)3} \frac{m_{t(u)}}{v} \bar Y^2\frac{v^2}{m_{KK}^2} $&$\sim \lambda^{(-)1} \frac{m_{c(u)}}{v} \bar Y^2\frac{v^2}{m_{KK}^2} $ \\\vspace{0.15cm}
pNGB & ${\mathcal O}(y_*^2 \frac{m_t}{v}\frac{\lambda_{L (R),2} \lambda_{L(R),3}m_W^2}{M_*^2})$ & ${\mathcal O}(y_*^2 \frac{m_t}{v}\frac{\lambda_{L (R),1} \lambda_{L(R),3}m_W^2}{M_*^2})$  & ${\mathcal O}(y_*^2 \frac{m_c}{v}\frac{\lambda_{L (R),1} \lambda_{L(R),2}m_W^2}{M_*^2})$ \\
\hline\hline
%
\end{tabular}
\caption{Same as Table \ref{tab:upyukawa} but for flavour-violating up-type Yukawa couplings. In the SM,
  NFC and the tree-level MSSM the Higgs Yukawa couplings are flavour
  diagonal. The CP-violating $\tilde \kappa_{ff'}$ are obtained by replacing the real part, ${\Re}$, with the imaginary part, ${\Im}$. All the other models predict a zero contribution to these flavour changing couplings.
}
\label{tab:upFVyukawa}
\end{center}
\end{table}

%
%
%
%
%
%
%
%
%
%
%
%
%
%
%
%
%
%
%
%
%
%
%

%
%
%
%
%
%
%
%
%
%
%
%
%
%
%
%
%
%
%
%
%
%
%


 

%
%
%
%
%
%
%
%
%
%
%
%
%
%
%
%
%
%
%
%
%
%
%
%
%
%
%
%
%
%
%
        



%
%
%
%
%
%
%
%
%
%
%

%
%
%
%
%
%
%
%
%
%
%
%
%
%
%
%
%
%
%
%


%
%
%
%
%
%
%
%
%
%





%
%
%
%
%
%
%
%
%
%
%
%
%
%
%
%
%
%
%

%
%
%
%
%
%
%
%
%
%
%
%
%
%
%
%
%
%
%
%
%
%
%
%
%
%
%
%
%
%
%
%
%
%
%
%
%
%



%
%
%
%
%
%
%
%
%
%
%
%
%
%
%
%
%
%
%
%
%
%
%
%
%
%
%
%
%
%
%
%
%
%
%
%
%
%
%
%
%
%
%
%
%
%
%
%

%
%
%
%
%
%
%
%
%
%
%
%
%
%
%
%
%
%
%
%
%
%
%
%
%
%
%
%
%
%
%
%
%
%
%
%
%
%
%
%
%
%
%
%
%
%
%
%
%
%
%
%
%
%
%
%
%
%
%
%
%
%
%
%
%
%
%
%
%
%
%
%
%
%
%
%
%
%
%
%
%
%
%
%
%
%
%
%
%

In Tables~\ref{tab:upyukawa} and \ref{tab:upFVyukawa}, $v=246$ GeV is the electroweak vev, while $\Lambda$ is the typical NP scale. For instance, if SM is corrected by dimension six operators with Minimal Flavour Violation (MFV), then the up-quark couplings receive a contribution $ Y_u^\prime\bar{Q}_L H^c u_R/{\Lambda^2}$, so that the Yukawa coupling is $y_u=Y_u +3 Y_u' {v^2}/({2 \Lambda^2})$, with $Y_u^{\prime}= a_uY_u +
        b^{\phantom\dagger}_uY^{\phantom\dagger}_uY_u^\dagger
        Y^{\phantom\dagger}_u + c^{\phantom\dagger}_u
        Y^{\phantom\dagger}_dY_d^\dagger
        Y^{\phantom\dagger}_u+\cdots$, where $Y_{u,d}$ are the SM Yukawas. The $v^2/\Lambda^2$ contributions correct both the diagonal Yukawa couplings, and lead to off-diagonal, flavour violating, couplings. 
        
 Not all NP models lead to flavour-violating Yukawa couplings. For instance, in multi-Higgs-doublet models with natural flavour conservation by assumption only one doublet, $H_u$, couples to the up-type quarks, only one Higgs doublet, $H_d$, couples to the down-type quarks, and only one doublet, $H_\ell$ couples to leptons (it is possible that any of these coincide, as in the SM where $H=H_u=H_d=H_\ell$)~\cite{Glashow:1976nt, Paschos:1976ay}. This only modifies diagonal Yukawa couplings, while off-diagonal remain to be zero, as in the SM. Similar result applies to the MSSM tree-level Higgs potential  where $h_{u,d}$ mix into the Higgs mass-eigenstates $h$ and
$H$ as $h_u=\cos \alpha h+\sin\alpha H$, $h_d = -\sin\alpha h +\cos\alpha H$,
where $h$ is the observed SM-like Higgs, and the vevs are $v_u=\sin\beta\, v$, $v_d=\cos\beta$. Flavourful two-Higgs-doublet model~\cite{Altmannshofer:2015esa}, on the other hand,  introduces mass suppressed off-diagonal and \CP violating contributions, a direct consequence of the fact that one Higgs doublet couples only to top, bottom and tau, and a second Higgs doublet couples to the remaining fermions (see also~\cite{Botella:2016krk,Ghosh:2015gpa,Das:1995df,Blechman:2010cs}). Off-diagonal and \CP violating Yukawa couplings are typical of any model of flavour with new degrees of freedom that are light enough, such as if the Higgs mixes with the flavon from the Froggatt-Nielsen (FN) mechanism~\cite{Froggatt:1978nt}, or if FN mechanism gives the structure of both dimension 4 and dimension 6 operators in SMEFT. Another example is the model of quark masses introduced by Giudice and Lebedev~\cite{Giudice:2008uua}, where the quark masses, apart from the top mass, are small, because they arise from higher dimensional operators. 

In Randall-Sundrum warped extra-dimensional models, that address simultaneously the hierarchy problem and the hierarchy of the SM fermion masses ~\cite{Randall:1999ee, Gherghetta:2000qt, Grossman:1999ra, Huber:2000ie,
	Huber:2003tu}, the corrections to the Yukawa couplings are suppressed by the masses of Kaluza-Klein (KK) modes,  $m_{KK}$. If Higgs is a
pseudo-Goldstone boson arising from the spontaneous breaking of a
global symmetry in a strongly coupled sector, coupling to the composite
sector with a typical coupling $y_*$ \cite{Dugan:1984hq,
  Georgi:1984ef, Kaplan:1983sm, Kaplan:1983fs} (for a review, see~\cite{Panico:2015jxa}), the corrections to the Yukawa couplings are suppressed by the mass of composite resonance with a typical mass $M_*\sim \Lambda$.
  
  In conclusion, we see that the NP effects in Higgs couplings to the SM fermions are either
suppressed by $1/\Lambda^2$, where $\Lambda$ is the NP scale, or are
proportional to the mixing angles with the extra scalars. This means that in the decoupling limit, $\Lambda \to \infty$ and/or small mixing angles, all the NP effects vanish. In the decoupling limit the Higgs couplings coincide with  the SM predictions, $\kappa_f=1$, while $\tilde \kappa_f=0, \kappa_{ff'}=0, \tilde \kappa_{f f'}=0$. 




%
%
%
%
%
%
%
%
%
%





\subsection{Probing charm and light quark Yukawa couplings}
%




The inclusive method of probing the charm-quark Yukawa is in many ways complementary 
to searches for exclusive decays (see discussion of Sec.~\ref{sec:exclusiveHiggs}) or 
searches for deviations in Higgs distributions (see Sec.~\ref{sec:HiggsDist}).
For example, in the inclusive approach an underlying assumption is 
that the Higgs coupling to $WW$ and $ZZ$ ---entering Higgs production--- is SM-like,
while the interpretation of Higgs distributions assumes no additional new physics contribution that affects them in a significant way.
An important difference between the inclusive and the exclusive approach is that the latter relies on interference with the SM $H\to \gamma\gamma$ amplitude while the former does not.
Therefore, in principle the exclusive approach may be sensitive to the sign and 
 \CP  properties of the coupling to which the inclusive approach is insensitive to.
At the same time, measurements of exclusive decays of the Higgs are challenging due 
to the small probability of fragmenting into the specific final state and 
large QCD backgrounds, which is why the inclusive approach appears to be the most promising one to probe deviations in the magnitude of the Higgs to charm coupling.

\begin{figure}[t]
	\centering
	\includegraphics[width=0.49\textwidth]{\main/section9/plots/mass}
	\includegraphics[width=0.49\textwidth]{\main/section9/plots/flavor}
	\caption{Summary of the HL-LHC projections discussed in the main text:  from exclusive decays (blue), from Higgs kinematic distributions (purple) and from inclusive constraints (yellow), compared to the present 8 TeV constraints from the  $h\to \gamma \gamma$ and  $ZZ$ line shapes (green), and from global fits (red).	
	\label{fig:higgs:forecast}}
\end{figure}

\Blu{
The summary of the projections that are discussed in the following sections, is given in Fig. \ref{fig:higgs:forecast}. Shown are the expected HL-LHC constraints from exclusive decays (blue), from Higgs kinematic distributions (purple) and from inclusive $c$-tagging measurements (yellow), as well as the present 8 TeV constraints from the combined $h\to \gamma \gamma$ and  $h\to ZZ$ line shapes (green), and from the global fits to Higgs data (red).  
}

%
%
%
%
%



\subsubsection{Inclusive charm quark tagging}
%

The most straight-forward way of inclusively probing charm quark Yukawa is to 
expand the $h\to b\bar b$ search with a search for 
$pp \to (Z/W\to\ell\ell/\nu) (h\to c\bar c)$ \cite{Perez:2015aoa}. Another possibility is to search for $pp\to h c$ production~\cite{Brivio:2015fxa}. The search for $h\to c \bar c$ from $pp\to Z h$ at $\sqrt{s}=13$~TeV was recently performed on a $36.1$~fb$^{-1}$ sample by ATLAS \cite{Aaboud:2018fhh}. The 
%
$c$-tagging algorithms are similar to $b$-tagging ones, with the most relevant quantities the 
%
%
displaced vertices due to a finite lifetime of $c$-hadrons.
Prior to its use in Higgs physics, $c$-tagging was used early on in Run I of the LHC by ATLAS and CMS in searches for supersymmetry~\cite{Aad:2014nra,Aad:2015gna}.
Its usefulness in relations to Higgs physics was first discussed in Ref.~\cite{Delaunay:2013pja} and subsequently
used in Ref.~\cite{Perez:2015aoa} to recast ATLAS and CMS Run~I analyses for $h\to b\bar b$ to provide the first direct LHC constraint on charm Yukawa. 


%
%



%

%
%
%



%
%
%
%
%
%
%
%
%
%
%


%
%
%
%
%
%
%
%
%
%
%
%
%
%

%
%
%
%
%
%
%

%
%
%
The efficiency of jet flavour tagging algorithms in associating a jet to a specific quark is 
correlated with the confidence to reject other hypotheses, e.g., production from light-quarks.
%
%
%
%
%
%
%
%
%
%
%
%
%
%
%
%
%
%
%
Given the rather similar lifetimes of $b$ and $c$ hadrons, there is always
a non-negligible contamination of the $c$-jet sample with
%
$b$ jets~\cite{Perez:2015aoa}. The ATLAS analysis \cite{Aaboud:2018fhh} used a working point  with an  efficiency of approximately $41\%$ to tag $c$-jets and rejection factors of roughly $4$ and $20$ for $b$- and light-quark-jets, respectively.
An inclusive $h\to c\bar c$ analysis
%
must thus either assume a SM value 
for the bottom Yukawa (as in Ref.~\cite{Aaboud:2018fhh}), or 
%
%
break the degeneracy between $y_b$ and $y_c$, e.g., by using more than one tagging working point with different ratios of $c$-tagging to $b$-tagging efficiencies~\cite{Perez:2015aoa,Perez:2015lra}.  
%
%
%
%
%
%

\begin{figure}[t]
	\centering
	\includegraphics[width=0.45\textwidth]{\main/section9/plots/plot_chi2_14TeV_kappabkappac_A_ctagII}
	\includegraphics[width=0.45\textwidth]{\main/section9/plots/plot_chi2_14TeV_kappabkappac_A_ctagIII}
	\caption{Projections for measuring charm Yukawa modifications from an inclusive 
		$H\to c\bar c$ search at $\sqrt{s}=14$\,TeV using two different 
		$c$-taggers (left and right panel) \cite{Perez:2015lra}.
		In red the $95\%$ CL region employing an integrated luminosity 
		of $2\times 300$~fb$^{-1}$ and in blue the region 
		employing $2\times 3000$~fb$^{-1}$.
	\label{fig:inclusiveforecast}}
\end{figure}

The prospects for measuring $pp\to Zh(\to c\bar c)$ at the HL-LHC were obtained in Ref. \cite{ATL-PHYS-PUB-2018-016}, by rescaling the 
%
Run 2 analysis \cite{Aaboud:2018fhh} 
%
to an integrated luminosity of $3000$~fb$^{-1}$.
Possibilities to reduce the systematic uncertainties were discussed as well.
%
%
Assuming $\kappa_b=1$,
an upper bound  on the signal strength 
 of $\mu_{Zh(c\bar c)} <6.3$ at $95\%$ CL can be set.
%
%
%
%
In Ref.~\cite{Perez:2015lra} instead, 
%
the prospects for measuring $h\to b\bar b$ at $\sqrt{s}=14$\,TeV 
\cite{ATLAS-collaboration:2012iza} were recast 
 to obtain the projection for an inclusive $h\to c \bar c$ measurement. Both $\kappa_c$ and $\kappa_b$ 
 %
 are treated as free variables. 
Fig.~\ref{fig:inclusiveforecast} shows results for two flavour tagging working points: a $c$-tagging efficiency of $30\%$ ($c$-tag II, left panel) or $50\%$ ($c$-tag III, right panel), while 
%
in both cases the $b$(light)-jet rejection is $5$($200$). 
%
 The %
$\kappa_b$ direction is profiled away, giving the HL-LHC sensitivity of 
$\kappa_c \simeq 21 (6)$  at $95\%$ CL with
$c$-tag II ($c$-tag III) and $2\times 3000$\,fb$^{-1}$ at $\sqrt{s}=14$\,TeV,
%
%
indicated by the blue regions in Fig.~\ref{fig:inclusiveforecast}.
%
%
%
%
%
%
%
%
%
%
%
%
%
%


%


Even though the LHCb experiment operates at lower luminosity  compared to ATLAS and CMS, it has unique capabilities for discrimination between $b$- and $c$-jets thanks to its excellent vertex reconstruction system~\cite{LHCb-PAPER-2015-016}. With the secondary vertex tagging (SV-tagging) LHCb achieved an identification efficiency of 60$\%$ on $b$-jets, of 25$\%$ on $c$-jets and a light jets (light quarks or gluons) mis-identification probability of less than 0.2$\%$. Further discrimination between light and heavy jets and between $b$- and $c$-jets is achieved by exploiting the secondary vertex kinematic properties, using Boosted Decision Tree techniques (BDTs). For instance, an additional cut on the BDT that separates $b$- from $c$-jets removes 90$\%$ of $h\rightarrow b \bar{b}$ while retaining 62$\%$ of $h\rightarrow c \bar{c}$ events~\cite{LHCb:2016yxg}. 
%

The LHCb acceptance covers $\sim 5$\% of the associated production of $W/Z+h$ at 13 TeV.  Fig.~\ref{fig:hbbetas} (left) shows the coverage of  LHCb for the $b\bar{b}$ pair produced in the decay of the  Higgs boson in association with a vector boson. When the two $b$-jets are within the acceptance, the lepton from $W/Z$ tends to be in acceptance as well ($\sim 60$\% of the time). Due to the forward geometry, Lorentz-boosted Higgs bosons are  likely to be properly reconstructed.

\begin{figure}[t]
	\centering
	\includegraphics[width=0.3\linewidth]{\main/section9/plots/hbb_etas}~~~~~~~~~
		\includegraphics[width=0.65\linewidth]{\main/section9/plots/c-tag-eff}
	\caption{Left: 2D histogram showing the coverage of the LHCb acceptance for the $b\bar{b}$ pair produced by the Higgs decay in associated production with a $W$ or a $Z$ boson. Right: LHCb $c$-jet SV-tagging efficiency for different scenarios in the HL-LHC conditions.}
	\label{fig:hbbetas}
\end{figure}

%
%
%
%
%
%

LHCb set  upper limits on the $V+h(\rightarrow b \bar{b})$ and $V+h(\rightarrow c \bar{c})$ production~\cite{LHCb:2016yxg} with data from LHC Run 1.
Without any improvements in the analysis or detector, the extrapolation of this  to 300fb$^{-1}$ at 14\,TeV leads to a sensitivity of $\mu_{Zh(cc)}\lesssim 50\,$.  Detector improvements are expected in future upgrades, in particular in impact parameter resolution which directly affects the $c$-tagging performance. 
If the detector improvement is taken into account, the $c$-jet tagging efficiency with the SV-tagging is expected to  improve as shown in Fig.~\ref{fig:hbbetas} (right). 
Further improvement is expected from the electron reconstruction due to upgraded versions of the electromagnetic calorimeter. Electrons are used in the identification of the vector bosons associated with the Higgs.
With these improvements, the expected limit can be pushed down to $\mu_{Zh(cc)}\lesssim 5-10$ which corresponds to a limit of 2-3 times the Standard Model prediction on the charm Yukawa coupling.
This extrapolation does not include improvements in analysis techniques: for instance  Deep Learning methods can be applied to exploit correlations in  jets substructure properties to reduce the backgrounds.
%




%

\subsubsection{Strange quark tagging}
\label{sec:strange-tagging}

%
%
The main idea behind the strange tagger described in~\Bref{Duarte-Campderros:2018ouv} is that
strange quarks---more than other partons---hadronize to prompt kaons which carry a large fraction of the jet momentum. %
%
Although the current focus at LHC is
mainly on charm and bottom tagging, recognizing strange jets has been attempted before at
DELPHI~\cite{Boudinov:1998fao} and SLD~\cite{Kalelkar:2000ig}, albeit in $Z$ decays.

Fig.~\ref{fig:stagger-efficiencies} shows results for a strange tagger
%
based on an analysis of event samples of Higgs and $W$ events generated with
\texttt{PYTHIA} 8.219~\cite{Sjostrand:2006za, Sjostrand:2014zea}. In each of the two hemispheres of the resonance decay, the charged pions and kaons stemming from the resonance are selected with an
assumed efficiency of 95\%. Similarly, $K_S$ are identified with an efficiency of 85\% if they decay via $K_S\to \pi^+\pi^-$
within \Unit{80}{cm} of the interaction point, which 
%
allows one to reconstruct
the decaying neutral kaon. Among the two lists of kaon candidates---one per hemisphere---one kaon in
each list is chosen for further analysis such that the scalar sum of their momenta is maximized
while rejecting charged same-sign pairs. The events are separated into the categories
charged-charged~(CC), charged-neutral~(CN) and neutral-neutral~(NN) with relative abundances of about CC:CN:NN$\approx 9:6:1$.
%
%

All selected candidates are required to carry a large momentum $p_{||}$ along the hemisphere
axis. This cut reduces the background from gluon jets as gluons radiate more than quarks
and therefore tend to spread their energy among more final state particles. In addition, charged kaons need to be produced promptly, in order to reject heavy flavour jets. The latter requirement is implemented by a cut on the impact parameter $d_0$ after the truth value has been smeared by the detector resolution.
\begin{figure}[t]
  \centering
  \includegraphics[width=0.5\textwidth]{\main/section9/plots/fs_legend}\\
  \includegraphics[width=0.45\textwidth]{\main/section9/plots/eff_CC_2s_1_014}
  \includegraphics[width=0.45\textwidth]{\main/section9/plots/eff_NC_2s_1_014}
  \caption{Efficiencies as function of the cut on $p_{||}$ and for $d_0<$\Unit{14}{$\mu$m} to
    reconstruct the different Higgs decay channels and $W$ decays as $s\bar s$ event by the
    described tagger. The left plot shows the CC channel, the right the CN channel.}
  \label{fig:stagger-efficiencies}
\end{figure}

The efficiencies obtained in the CC and CN channel for a cut of $d_0<$\Unit{14}{$\mu$m} are shown in
Fig.~\ref{fig:stagger-efficiencies}. While there is clearly still ample room for improvement, this
simple tagger  already shows  a good suppression of the bottom, charm and gluon
background  by orders of magnitude. Due to missing particle identification at ATLAS and CMS, the efficiencies for first-generation jets and
strange jets are degenerate in the CC channel. However, in the CN channel, due to the required
$K_S$, a suppression of pions is achieved that breaks this degeneracy. This is particularly
interesting in light of the HL-LHC, where a large background from first generation jets is expected.


\subsubsection{Exclusive Higgs decays}
\label{sec:exclusiveHiggs}

Exclusive Higgs decays to a vector meson, $V$, and a photon, $h\to V\gamma$, can directly probe  bottom, charm~\cite{Bodwin:2013gca,Bodwin:2014bpa}, strange, down and up~\cite{Kagan:2014ila} quark Yukawas.
%
%
%
On the experimental side, both ATLAS and CMS reported first upper bounds on $h\to \Upsilon\gamma, J/\psi\gamma$~\cite{Aad:2015sda,Khachatryan:2015lga}, $h\to\phi\gamma$ and $h\to\rho\gamma$~\cite{Aaboud:2016rug,Aaboud:2017xnb}, sensitive to diagonal Yukawa couplings. 
%
%
The $h\to V\gamma$ decays
%
%
receive two contributions, from $h\to \gamma\gamma^*$ decay followed by a $\gamma^*\to V$ fragmentation, and from a numerically smaller amplitude that involves the direct coupling of quarks to the Higgs~\cite{Keung:1983ac,Bodwin:2013gca,Bodwin:2014bpa,Kagan:2014ila}. The sensitivity to the quark Yukawa couplings thus comes mostly from the interference of the two amplitudes.
%
%
%
%
%
%
%
%
%
%
%
%
%

Normalizing to the $ h\to ZZ^*\to4\ell$ channel the total Higgs width cancels~\cite{Perez:2015aoa,Koenig:2015pha}, 
%
%
\begin{equation}
	\cR_{V\gamma,ZZ^*} 
\equiv	\frac{\mu_{V\gamma} }{\mu_{ZZ^*}}  \frac{ \BR^{\rm SM}_{h\to V\gamma}}{ \BR^{\rm SM}_{h\to ZZ^*}}  
	\simeq  
	\frac{\Gamma_{h\to V\gamma}}{ \Gamma_{h\to ZZ^*}},
\end{equation}
%
where $ \BR^{\rm SM}_{h\to ZZ^*}$ is the SM branching ratio for $h\to ZZ^*\to 4\ell$.
In the last equality we also assumed perfect cancellation of the production mechanisms (this is entirely correct, if $h\to \gamma\gamma$ is used as normalization channel, but at present this leads to slightly worse bounds on light quark Yukawas). Using predictions from Ref.~\cite{Koenig:2015pha} gives the currently allowed ranges for light quark Yukawa couplings, collected in Table \ref{tab:hexclusive} (here $\kappa_Z$ and $\kappa_{\gamma\gamma}^{\rm eff}$ parametrize deviations of $h\to ZZ, \gamma\gamma$ amplitudes relative to their SM values).

%
%
%
%
%
%
%
%
%
%
%
%
%
%
%
%
%
%
%
%
%
%
%
%
%
%
%
%
%
%
%
%
%
%
%
%
%
%
%
%
%
%
%
%
%
%
\begin{table}[t]
\begin{center}
\begin{tabular}{llcc}
\hline\hline
mode & ~~~~~~~~~~~~$\BR_{h\to V\gamma}<$ & $\cR_{V\gamma,ZZ^*}<$ &Yukawa range  \\
\hline
$J/\psi\,\gamma$  &
$1.5\cdot 10^{-3}$~\cite{Aad:2015sda,Khachatryan:2015lga} & 
$9.3$ &
$-295\kappa_Z + 16\kappa^{\rm eff}_{\gamma\gamma} < \kappa_c < 295\kappa_Z + 16\kappa^{\rm eff}_{\gamma\gamma} $
\\
%
$\phi\,\gamma$ &
$4.8 \cdot 10^{-4}$~\cite{Aaboud:2016rug,Aaboud:2017xnb} & 
$3.2$ &
$-140\kappa_Z + 10\kappa^{\rm eff}_{\gamma\gamma} < \bar{\kappa}_s < 140\kappa_Z + 10\kappa^{\rm eff}_{\gamma\gamma} $
\\
%
$\rho\,\gamma$ &
$8.8 \cdot 10^{-4}$~\cite{Aaboud:2017xnb} & 
$5.8$ &
$-285\kappa_Z + 42\kappa^{\rm eff}_{\gamma\gamma} < 2\bar{\kappa}_u + \bar{\kappa}_d < 285\kappa_Z + 42\kappa^{\rm eff}_{\gamma\gamma} $\\
\hline\hline
\end{tabular}
\end{center}
\caption{The current 95\%\,C.L. upper bounds, assuming SM Higgs production, on different exclusive $h\to V\gamma$ decays, and the interpretation in terms of the Higgs Yukawa couplings. Note that $\bar{\kappa}_q = y_q/y^{\rm SM}_b\,$. %
}
\label{tab:hexclusive}
\end{table}%
%


For prospects to probe light quark Yukawa at HL-/HE-LHC 
%
we follow 
%
Ref.~\cite{Perez:2015lra}.
%
Rescaling with luminosity and the increased production cross sections both the signal and backgrounds, while ignoring any changes to the analysis that may change the ratios of the two, gives the projected sensitivities 
%
listed in Table \ref{tab:exclusiveproj}.
The estimates in Table~\ref{tab:exclusiveproj} are in agreement with the ATLAS projection for $h\to J/\psi\gamma$~\cite{ATL-PHYS-PUB-2015-043}, which quotes $\cR_{J/\psi\gamma,ZZ^*}<0.34^{+0.14}_{-0.1}$.
We see that only large enhancements, with Yukawa coupling of light quarks well above the values of bottom Yukawa will be probed. To be phenomenologically viable, these would require large cancellations in the contributions to the light quark masses (and a mechanism to avoid indirect bounds from global fits to the Higgs data).
%

%
%
%
%
%
%
%
%
%
%
%
%
%
%
%
%
\begin{table}[t]
\begin{center}
\begin{tabular}{cccc}
\hline\hline
mode & collider energy & $\cR_{V\gamma,ZZ^*}<$ &Yukawa range ($\kappa_V=\kappa^{\rm eff}_{\gamma\gamma}=1$)   \\
\hline
$J/\psi\,\gamma$ &
14\,TeV &  $0.47\sqrt{L_3}$ &
$16 - 67 L_3^{1/4}  < \kappa_c < 16 + 67 L_3^{1/4} $
\\
&27\,TeV &  $0.28\sqrt{L_3}$ &
$16 - 52 L_3^{1/4}  < \kappa_c < 16 + 52 L_3^{1/4} $\\
%
%
%
%
\hline
$\phi\,\gamma$ &
14\,TeV & 
$0.33\sqrt{L_3}$ &
$11 - 46 L_3^{1/4} < \bar{\kappa}_s < 11 + 46 L_3^{1/4} $\\
&
27\,TeV & 
$0.20\sqrt{L_3}$ &
$11 - 35 L_3^{1/4} < \bar{\kappa}_s < 11 + 35 L_3^{1/4} $\\
%
%
%
%
\hline
$\rho\,\gamma$ &
14\,TeV & 
$0.60\sqrt{L_3}$ &
$44 - 93 L_3^{1/4} < 2\bar{\kappa}_u + \bar{\kappa}_d < 44 + 93 L_3^{1/4} $\\
&
27\,TeV & 
$0.36\sqrt{L_3}$ &
$44 - 72 L_3^{1/4} < 2\bar{\kappa}_u + \bar{\kappa}_d < 44 + 72 L_3^{1/4} $\\
%
%
%
%
\hline\hline
\end{tabular}
\end{center}
\caption{The projections of bounds on Yukawa couplings for HL-/HE-LHC as functions of integrated luminosity, $L_3\equiv (3/{\rm ab})/\cL\,$. Note that $\bar{\kappa}_q = y_q/y^{\rm SM}_b\,$. }
\label{tab:exclusiveproj}
\end{table}%
%

%
Higgs exclusive decays can in principle also probe off-diagonal couplings by measuring modes such as $h\to B^*_s \gamma$~\cite{Kagan:2014ila}. 
%
%
However, the Higgs flavour violating couplings are strongly constrained by meson mixing~\cite{Blankenburg:2012ex,Harnik:2012pb}, so that the expected rates are too small to be observed. 
For a detailed discussion of $h \to MZ,MW$ channels see~\cite{Alte:2016yuw}.





\subsubsection{Yukawa constraints from Higgs distributions}
\label{sec:HiggsDist}

\subsubsubsection{Higgs \pt and rapidity distributions}


In general, the Higgs \pt distribution probes
%
whether NP particles are running in the $gg\to h +(g)$ loops 
\cite{Arnesen:2008fb,Biekotter:2016ecg,Brehmer:2015rna,Dawson:2015gka,Schlaffer:2014osa,Grojean:2013nya,Langenegger:2015lra,Bramante:2014hua,Buschmann:2014twa,Azatov:2013xha,Banfi:2013yoa,Buschmann:2014sia}. 
However,  the soft part of the \pt spectrum is also an indirect probe of light quark Yukawas~\cite{Soreq:2016rae,Bishara:2016jga}. If 
Higgs, unlike in the SM, is produced from the $u \bar u$ or $d \bar d$ fusion,
%
%
%
then {\em (i)} the Sudakov peak is shifted to $p_T\sim 5$\,GeV from $\sim10$\,GeV in the SM~\cite{Collins:1984kg,Soreq:2016rae}, and  {\em (ii)} the rapidity distributions are more forward, i.e., shifted toward larger $\eta$~\cite{Soreq:2016rae}, see Fig. \ref{fig:HiggsDist}.
%
%
%
%
%
Enhanced $s$ or $c$ Yukawa couplings also lead to softer Higgs \pt spectrum. The dominant effect for charm quark is due to one loop contributions  to $gg\to hj$ process that are enhanced by double logarithms~\cite{Baur:1989cm}, while for strange quark it is due to $s\bar s$ fusion production of the Higgs~\cite{Soreq:2016rae}. 
%
%
%
%
Fig.~\ref{fig:HiggsDist} shows the corresponding normalized distributions, for which many theoretical and experimental uncertainties cancel.
%
%
%
\begin{figure}[t]
\begin{center}
\includegraphics[width=0.28\textwidth]{\main/section9/plots/normalizedcharm8TeV.pdf}~~~~
\includegraphics[width=0.34\textwidth]{\main/section9/plots/DistpT.pdf}~~~~
\includegraphics[width=0.34\textwidth]{\main/section9/plots/Disteta.pdf}
\caption{Normalized Higgs \pt and rapidity distributions for modified charm (left) and $u$, $d$ and $s$ Yukawas (middle and right). Taken from~\cite{Bishara:2016jga} and~\cite{Soreq:2016rae}. 
}
\label{fig:HiggsDist}
\end{center}
\end{figure}
%

The 8\,TeV ATLAS results on Higgs \pt distributions~\cite{Aad:2015lha} were converted to  the following $95\,\%$ CL bounds in Ref.~\cite{Soreq:2016rae}
%
%
\begin{align}
	\bar{\kappa}_u = y_u/y^{\rm SM}_b < 0.46 \, , \qquad\qquad
	\bar{\kappa}_d = y_d/y^{\rm SM}_b< 0.54 \, ,
\end{align}
%
stronger than the corresponding bounds from fits to the inclusive Higgs production cross sections. 
%
%
The sensitivities expected at HL-LHC are shown in Fig.~\ref{fig:HiggsDistFuture} (right), assuming a $5\%$  total uncertainty in each \pt bin. CMS interpreted the $35.9\,$fb$^{-1}$ measurement of 13\,TeV Higgs \pt spectrum in terms of  95\,\% CL  bounds on $c$ and $b$ Yukawa couplings~\cite{CMS:2018hhg}
%
%
\begin{align}
    -8.7 (-18.0) < \kappa_c < 10.6 (22.9),  \qquad
    -1.9(-2.8) < \kappa_b < 2.9(9.9),
\end{align}
assuming the branching fractions  depend only on $\kappa_c$ or $\kappa_b$ (or are floated freely). 
%
%
%
%
%
%
%
These bounds on the $c$ and $b$ Yukawa are  weaker than the bounds from the global fit of the 8\,TeV Higgs data along with the electroweak precision data allowing all Higgs coupling to float~\cite{Perez:2015aoa} and from the direct measurement of $h\to b\bar{b}$ by using $b$-tagging. The projected HL-LHC sensitivity is shown in Fig. \ref{fig:HiggsDistFuture} (left).

%
%
\begin{figure}[t]
\begin{center}
%
\includegraphics[width=0.35\textwidth]{\main/section9/plots/future12.pdf}~~~~~~~~~~~~~~~
\includegraphics[width=0.37\textwidth]{\main/section9/plots/Naive13TeVpT.pdf}
\caption{The sensitivity of  Higgs \pt distributions to the light quark Yukawa couplings at 13\,TeV LHC Run2 and HL-LHC: to charm and bottom~\cite{Bishara:2016jga} (left) and to up and down~\cite{Soreq:2016rae} (right, assuming total $5\%$ systematic+statistical uncertainty).  
%
%
%
}
\label{fig:HiggsDistFuture}
\end{center}
\end{figure}
%


\Blu{ Fig.~\ref{fig:kbkc_couplingdependentBRs} shows expected one sigma $\kappac, \kappab$ contours from $3000$fb$^{-1}$ global fits,
obtained by extrapolating the measured constraints in Ref.~\cite{CMS:2018hhg}. The fit uses expected differential distributions, obtained by extrapolating the $\sqrt{s}=13\,$TeV \pt distributions in the $\hgg$~\cite{Sirunyan:2018kta} and 
$\hzz^{(*)}\to 4\ell$~\cite{Sirunyan:2017exp}
($\ell = e$ or $\muon$) decay channels, as well as 
 a search for $\hbb$ at large \pt~\cite{Sirunyan:2017dgc},
 which enhances the sensitivity  to $\kappat$.
 The simultaneous extended maximum likelihood fit to the diphoton mass, four-lepton mass, and soft-drop mass, $\msd$,~\cite{Dasgupta:2013ihk,Larkoski:2014wba} spectra in all the analysis categories of the $\hgg$, $\hzz$, and $\hbb$ channels, gives the results in Fig.~\ref{fig:kbkc_couplingdependentBRs}  (left) when only $\kappa_{c}$ and $\kappa_b$ are varied, while all the other couplings are set to the SM value, and in Fig.~\ref{fig:kbkc_couplingdependentBRs} (right), if in addition the Higgs total decay width is allowed to float.  In the fit for Fig.~\ref{fig:kbkc_couplingdependentBRs}  (left) the largest sensitivity to $\kappa_c, \kappa_b$ comes from the total cross sections times branching ratios, while for Fig.~\ref{fig:kbkc_couplingdependentBRs}  (right) it is due to normalized differential distributions $(1/\sigma)(d\sigma/dp_T)$. 
In Fig.~\ref{fig:kbkc_couplingdependentBRs} the systematic uncertainties are conservatively kept at current level (dubbed Scenario 1).
}
%
There is only a minor change in the projected constraints on $\kappa_c, \kappa_b$ in case of reduced systematic uncertainties (Scenario 2) compared to Scenario 1. 


%
%
%
%
%
%
%
%
%
%
%
%
%
%
%
%
%
%
%
%
%
%
%
%
%
%
%
%
%
%
%
%
%
%
%
%
%
%
%

%

\begin{figure}[t]
  \begin{center}
  \includegraphics[width=0.4\linewidth]{\main/section9/plots/projection_kbkc_plot_couplingdependentBRs_sm.pdf}~~~~~~~~~~
   \includegraphics[width=0.4\linewidth]{\main/section9/plots/projection_kbkc_plot_floatingBRs.pdf} 
\caption{
 Left: projected simultaneous fit of $\kappab$ and $\kappac$ 
 from $3{\rm ab}^{-1}$ 13 TeV data, assuming current systematics (Scenario 1).
The one standard deviation contours are drawn for the $\hgg$ channel (the $\hzz$, the combination of $\hgg$ and $\hzz$) with solid red (blue, black).
For the combination the two standard deviation contour is drawn as a black dashed line. The negative log-likelihood value are given on the coloured axis.   Right: same as left, but with the branching fractions implemented as nuisance parameters with no prior constraint. (Taken from \cite{CMS-PAS-FTR-18-011}.).
        }
    \label{fig:kbkc_couplingdependentBRs}
  \end{center}
\end{figure}


%

%
%
%
%
%
%
%
%
%
%
%
%
%
%
%
%
%


%
\subsubsubsection{$W^\pm h$ charge asymmetry}
%

The $W^\pm h$ charge asymmetry, 
\begin{equation}
	A 
= 	\frac{ \sigma (W^+ h) - \sigma (W^- h)}
	{\sigma (W^+ h) + \sigma (W^- h) },
\end{equation}
  is a  production-based probe of light quark Yukawa
couplings~\cite{Yu:2016rvv}.  
%
%
%
%
%
%
%
%
%
%
In the SM, the inclusive HE-LHC charge asymmetry is expected to be $17.3\%$, while the HL-LHC charge asymmetry is expected to be $21.6\%$.  
%
%
%
The dominant $W^\pm h$ production mode is due to Higgs boson
radiating from $W^\pm$ intermediate lines, if the Yukawa-mediated
diagrams are negligible.  If the quark Yukawas are not SM-like,
however, the charge asymmetry can either increase or decrease,
depending on the overall weight of the relevant PDFs.  In particular,
the charge asymmetry will increase if the down or up quark Yukawa
couplings are large, reflecting the increased asymmetry of $u \bar{d}$
vs.~$\bar{u} d$ PDFs; the charge asymmetry will decrease if the
strange or charm Yukawa couplings are large, reflecting the symmetric
nature of $c \bar{s}$ vs.~$\bar{c} s$ PDFs.  The subleading correction
from the Cabibbo angle-suppressed PDF contributions determines the
asymptotic behavior for extremely large Yukawa enhancements.

\begin{figure}[t]
  \begin{center}
 %
 \includegraphics[width=0.6\textwidth, angle=0]{\main/section9/plots/AsymmetryNLO_HELHC.pdf}
 \caption{Inclusive charge asymmetry for $W^\pm h$ production at the
   27 TeV HE-LHC (solid colored bands), and 14 TeV HL-LHC (dotted
   colored bands), calculated at NLO QCD from MadGraph\_aMC@NLO using
   NNPDF 2.3 as a function of individual Yukawa rescaling factors
   $\bar{\kappa}_f$ for $f = u$ (red), $d$ (green), $s$ (blue), and
   $c$ (purple).  Shaded bands correspond to scale uncertainties at
   $1\sigma$ from individual $\sigma(W^+ h)$ and $\sigma(W^- h)$
   production, which are conservatively taken to be fully
   uncorrelated.  The expected statistical errors from this
   measurement using 10 ab$^{-1}$ of HE-LHC data and 3 ab$^{-1}$ of
   HL-LHC data are also shown.}
  \label{fig:asymmetry}
  \end{center}
\end{figure}

The effect of individual $d$, $u$, $s$, or $c$ quark Yukawa
enhancements on the inclusive charge asymmetry is shown in
Fig.~\ref{fig:asymmetry}, in units of $\bar{\kappa}_f = y_f /
y_{b,\text{SM}}$, evaluated at the Higgs mass scale.  Since $W^\pm h$
production probes lower Bjorken-$x$ at the HE-LHC compared to the
HL-LHC, the expected SM charge asymmetry is lower at the higher energy
collider.  The bands denote the change in the charge
asymmetry from varying the renormalization and factorization
scales within a factor of 2. The error bars denote 
%
the expected
$0.45\%$ ($0.25\%$) statistical sensitivity to the charge asymmetry at HL-LHC (HE-LHC) in the $W^\pm h \to
\ell^\pm \ell^\pm jj \nu \nu$ final state~\cite{Yu:2016rvv}.  The HE-LHC sensitivity was estimated
by simply rescaling with the appropriate luminosity ratio, since we expect the increases in both signal and
background electroweak rates to largely cancel.  The
constraint from the CMS Run I direct Higgs width upper bound is also shown~\cite{Yu:2016rvv}.  
%
%
%
%
%
%
%
%
%
%
If the signal strengths are fixed to the
SM expectation and the central prediction is used, the HE-LHC charge
asymmetry measurement could constrain $\bar{\kappa}_f \lesssim 2-3$
for up and charm quarks, and $\bar{\kappa}_f \lesssim 7$ for down or
strange quarks.

%


\subsection{LFV decays of the Higgs}

The flavour violating Yukawa couplings are well constrained by the low-energy
flavour-changing neutral current measurements
\cite{Harnik:2012pb,Blankenburg:2012ex,Gorbahn:2014sha}. 
\Blu{
For instance, CMS bounds $\kappa_{\mu e}^2+\kappa_{e\mu}^2<(5.4\times 10^{-4})^2$  from the $h\to e\mu$ search \cite{Khachatryan:2016rke}, compared to indirect bound from $\mu\to e\gamma$, which is $\kappa_{\mu e}^2+\kappa_{e\mu}^2<(3.6\times 10^{-6})^2$ \cite{Harnik:2012pb}.
}
A notable
exception are the flavour-violating couplings involving a tau lepton, where the
strongest constraints on $\kappa_{\tau\mu}, \kappa_{\mu\tau},
\kappa_{\tau e}, \kappa_{e \tau}$ are from direct searches for flavour-violating Higgs decays at
the LHC \cite{Sirunyan:2017xzt,Aad:2016blu}.
Currently, the CMS 13\,TeV search with 35.9\,fb$^{-1}$~\cite{Sirunyan:2017xzt} gives the strongest constraint 
%
\Blu{
\begin{align}
    \sqrt{\kappa_{\mu\tau}^2 + \kappa^2_{\tau\mu}} < 1.43 \times 10^{-3}\, , \qquad
    \sqrt{\kappa_{e\tau}^2 + \kappa^2_{\tau e}} < 2.26 \times 10^{-3}\, ,
\end{align}
}
%
obtained from  95\,\%~CL upper limits $\BR(h\to \mu\tau)<0.25\%$ and $\BR(h\to e\tau)<0.61\%$, respectively. One can also directly measure the difference between the branching ratios of $h\to\tau e$ and $h\to\tau\mu$, as proposed in~\cite{Bressler:2014jta}.
Assuming naively that both systematics and statistical error scale with square root of the luminosity, one can expect that the sensitivity of $3000\,$fb$^{-1}$ HL-LHC will be  around the half per-mil level for both the  $h\to e\tau$ and $h \to \mu\tau$ branching ratios. 


The LHC can also set bounds on rare FCNC top decays involving a Higgs~\cite{Aaboud:2017mfd,Khachatryan:2016atv,Aad:2015pja,Aad:2014dya}. The strongest current bounds are ${|\kappa_{ct}|^2+|\kappa_{tc}|^2}<(0.06)^2$ 
\Blu{and ${|\kappa_{ut}|^2+|\kappa_{tu}|^2}<(0.07)^2$ at 95\,\%~CL}.



%
%
%
%
%
%
%
%
%
%
 %
 %
%
%
%
%


%

%
%
%
%
%
%
%
%
%
%
%

\subsection{CP violating Yukawa couplings}
The CP-violating flavour-diagonal Yukawa couplings, $\tilde \kappa_{f_i}$,
are well constrained from bounds on the electric dipole moments~(EDMs)
\cite{Brod:2013cka,Chien:2015xha,Altmannshofer:2015qra,Brod:2018pli} under the assumption of no 
cancellation with other contributions to EDMs.
%
For the electron Yukawa, the latest ACME measurement~\cite{Baron:2013eja,Andreev:2018ayy} results in an upper bound of $\tilde\kappa_e<1.9\times 10^{-3}$~\cite{Altmannshofer:2015qra}. For the bottom and charm Yukawas the strongest limits come from the neutron EDM~\cite{Brod:2018pli}. Using the NLO QCD theoretical prediction, this translates into the upper bounds $\tilde\kappa_b<5$ and $\tilde\kappa_c<21$ when theory errors are taken into account.
For the light quark CPV Yukawas, measurements of the Mercury EDM place strong bounds on the up and down Yukawas of $\tilde\kappa_u<0.06$ and $\tilde\kappa_d<0.03$~\cite{Brod:2018xyz} (no theory errors, $90\%$ CL), while the neutron EDM measurement gives a weaker constraint on the strange quark Yukawa of $\tilde\kappa_s<2.2$~\cite{Brod:2018xyz} (no theory errors, $90\%$ CL).


The top and $\tau$ Yukawa phases can be directly probed at HL-LHC, as we discuss below. For constraints from EDMs and other phases see~\cite{Brod:2013cka,Chien:2015xha,Altmannshofer:2015qra,Brod:2018pli,Baron:2013eja,Andreev:2018ayy,Altmannshofer:2015qra}. 

\subsubsection{$t\bar{t} h$}



CP violation in the top quark-Higgs coupling is strongly constrained by EDM measurements~\cite{Brod:2013cka}, if the light quark Yukawa couplings and $hWW$ couplings have their SM values. If this is not the case, the indirect constraints on the phase of the top Yukawa coupling can be substantially relaxed.
    Assuming the EDM constraints can be avoided, the \CP structure of the top quark Yukawa can be probed directly in $pp \to t\bar t h$. Many simple observables, such as $m_{t\bar t h}$ and $p_{T,h}$ are sensitive to the \CP structure, but require reconstructing the top quarks and Higgs.

Recently, several $t\bar t h$ observables have been proposed  that access the \CP structure without requiring full event reconstruction. These include the azimuthal angle between the two leptons in a fully leptonic $t/\bar{t}$ decay with the additional requirement that the $p_{T,h} > 200\, \text{GeV}$~\cite{Buckley:2015vsa}, and the angle between the leptons, in a fully leptonic $t/\bar t$ system, projected onto the plane perpendicular to the $h$ momentum~\cite{Boudjema:2015nda}. These observables only require that the Higgs is reconstructed and are inspired by the sensitivity of $\Delta \phi_{\ell^+\ell^-}$ to top/anti-top spin correlations in $pp \to t\bar t$~\cite{Mahlon:1995zn}. The sensitivity of both of these observables improves at higher Higgs boost, and therefore higher energy, making them promising targets for the HE-LHC, though no dedicated studies have been carried out to date.

\begin{figure}[t]
\centering
 \includegraphics[width=0.32 \textwidth]{\main/section9/plots/dilep.pdf}
~
 \includegraphics[width=0.32\textwidth]{\main/section9/plots/semilep.pdf}
~
 \includegraphics[width=0.32\textwidth]{\main/section9/plots/combination.pdf}
 \caption{Expected CL, assuming the SM, for exclusion of the pure \CP-odd scenario, $\kappa_t=0, \tilde \kappa_t=1$, as a function of the integrated luminosity. Left: using the $t\bar{t}h$ ($h\rightarrow b\bar{b}$) dileptonic analysis only, middle: using the $t\bar{t}h$ ($h\rightarrow b\bar{b}$)  semileptonic analysis only, right: combining observables in each individual channel and combining both channels, treating  the observables as uncorrelated.  A likelihood ratio computed from the binned distribution of the corresponding discriminant observable was used as test statistic.}
 \label{combination}
\end{figure}

%
%
%
%
%
%
%
%
%
%
%
%
%
%
%
%
%
%
%
%
%
%
%
%
%
%
%

%
%

Fig.~\ref{combination} shows the expected CL, assuming the SM, for exclusion of the pure \CP-odd scenario, $\kappa_t=0, \tilde \kappa_t=1$, as a function of the integrated luminosity. Samples of $t\bar t h$($h\rightarrow b\bar{b}$) events were generated at the LHC for $\sqrt{s}=13$~TeV, with {\tt MadGraph5\_aMC@NLO}~\cite{Alwall:2014hca}
%
using the \texttt{HC\_NLO\_X0} model~\cite{Artoisenet:2013puc}, as were  all relevant SM background processes.
%
The analyses of the $t\bar t h$ ($h\rightarrow b\bar b$) events were carried out in the dileptonic and semileptonic decay channels of the $t\bar t$ system. {\tt Delphes}~\cite{deFavereau:2013fsa} was used for a parametrised detector simulation and both analyses used kinematic fits to fully reconstruct the $t\bar t h$ system. The results were extrapolated, as a function of luminosity, up to the 3000~fb$^{-1}$.
%

Fig.~\ref{combination} left (middle) shows results using the dileptonic (semileptonic) analysis only. The CL were obtained from a signal-enriched region (with at least 3 $b$-tagged jets) in which a likelihood ratio was computed from binned distributions of various discriminant observables~\cite{AmorDosSantos:2017ayi, Demartin:2014fia}. Only statistical uncertainties were considered. Fig.~\ref{combination} right shows CL obtained from the combination of different observables in each channel i.e., $\Delta\eta(\ell^+,\ell^-)$, $\Delta\phi(t,\bar t)$ and $\sin(\theta^{t\bar tH}_{t})\sin(\theta^{H}_{W^+})$ in the dileptonic channel and, $b_4$ and $\sin(\theta^{t\bar tH}_{\bar t})\sin(\theta^{H}_{b_H})$ in the semileptonic channel. The combination of the two channels is also shown for comparison. The observables were treated as uncorrelated. 
%

The main conclusions of these studies can be summarized in what follows: {\it i)} many angular observables 
%
are available with the potential of discriminating between values of $\kappa_t, \tilde \kappa_t$ 
%
 in the top quark Yukawa coupling; {\it ii)} the sensitivity of the semileptonic final state of $t\bar{t}h$($h\rightarrow b\bar{b}$) is roughly a factor 3 better than that of the dileptonic channel alone (Fig.~\ref{combination} left); {\it iii)} the combination of the two channels (semi- and dileptonic) is roughly a factor 5 more sensitive than the dileptonic channel, providing a powerful test of the top quark-Higgs interactions in the fermionic sector.


\subsubsection{$\tau\bar{\tau} h$}

The most promising direct probe of \CP violation in fermionic Higgs
decays is the $\tau^+ \tau^-$ decay channel, which benefits from a
relatively large $\tau$ Yukawa, resulting in a SM branching fraction of
$6.3\%$. Measuring the \CP violating phase in the tau Yukawa requires a measurement of the linear polarizations of both $\tau$ leptons and the azimuthal angle between them. This can be done by analyzing tau substructure, namely the angular distribution of the various components of the tau decay products.

The main $\tau$ decay modes studied include $\tau^\pm \to
\rho^\pm (770) \nu$, $\rho^\pm \to \pi^\pm \pi^0$~\cite{Bower:2002zx,
  Desch:2003mw, Desch:2003rw, Harnik:2013aja, Askew:2015mda,
  Jozefowicz:2016kvz} and $\tau^\pm \to \pi^\pm
\nu$~\cite{Berge:2008wi, Berge:2008dr, Berge:2011ij}.  Assuming CPT
symmetry, collider observables for \CP violation must be built from
differential distributions based on triple products of three-vectors.
In the first case, $h \to \pi^\pm \pi^0 \pi^\mp \pi^0 \nu \nu$,
angular distributions built only from the outgoing charged and neutral
pions are used to determine the \CP properties of the initial $\tau$
Yukawa coupling.  In the second case, $h \to \pi^\pm \pi^\mp \nu \nu$,
there are not enough reconstructible independent momenta to construct an observable sensitive to \CP violation, requiring additional kinematic information such as the $\tau$ decay impact parameter.

In the kinematic limit when each outgoing neutrino is taken to be
collinear with its corresponding reconstructed $\rho^\pm$ meson, the
acoplanarity angle, denoted $\Phi$, between the two decay planes
spanned by the $\rho^\pm \to \pi^\pm \pi^0$ decay products is exactly
analogous to the familiar acoplanarity angle from $h \to 4 \ell$
\CP-property studies.  Hence, by measuring the $\tau$ decay products in
the single-prong final state, suppressing the irreducible $Z \to
\tau^+ \tau^-$ and reducible QCD backgrounds, and reconstructing the
acoplanarity angle of $\rho^+$ vs.~$\rho^-$, the differential
distribution in $\Phi$ gives a sinusoidal shape whose maxima and
minima correspond to the \CP-phase in the $\tau$ Yukawa coupling, $\varphi_\tau=\arctan(\tilde \kappa_\tau/\kappa_\tau)$.  

An optimal observable using the colinear approximation was derived in~\cite{Harnik:2013aja}. Assuming 70\% efficiency for tagging hadronic $\tau$ final states, and
neglecting detector effects, the estimated sensitivity for the
CP-violating phase $\varphi_\tau$ using 3 ab$^{-1}$ at
the HL-LHC is 8.0$^\circ$.  A more sophisticated
analysis~\cite{Askew:2015mda} found that detector resolution effects
on the missing transverse energy distribution degrade the expected
sensitivity considerably, and as such, about 1 ab$^{-1}$ is required
to distinguish a pure scalar coupling ($\kappa_\tau=1, \tilde \kappa_\tau=0$) from a pure
pseudoscalar coupling ($\kappa_\tau=0, \tilde \kappa_\tau=1$).

At the HE-LHC, the increased signal cross section for Higgs production
is counterbalanced by the increased background rates, and so the main
expectation is that improvements in sensitivity will be driven by the
increased luminosity and more optimized experimental methodology.
Rescaling with the appropriate luminosity factors, the optimistic
sensitivity to the $\tau$ Yukawa phase $\varphi_\tau$ from acoplanarity studies is
4-5$^\circ$, while the more conservative estimate is roughly an order
of magnitude worse.

\end{document}
