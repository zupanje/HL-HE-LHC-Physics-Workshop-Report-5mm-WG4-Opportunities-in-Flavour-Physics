% don't remove the folling lines, and edit the defintion of \main if needed
\documentclass[../report.tex]{subfiles}
\providecommand{\main}{..}
\IfEq{\jobname}{\currfilebase}{\AtEndDocument{\biblio}}{}
% until here
\def\Qbar    {{\kern 0.2em\overline{\kern -0.2em \PQ}{}}\xspace}

%%%%%%%%%%%%%%%%%%%%%%%%%%%%%%%%%%%%%%%%%%%%%%%%%%%%%%%%%%%%%
\newcommand{\fk}[1]{{\color{blue}#1}}
%\newcommand{\fkcor}[1]{{\color{blue}\st{#1}}}
\newcommand{\sasa}[1]{{\color{brown}#1}}
\newcommand{\others}[1]{{\color{red}#1}}
\newcommand{\juan}[1]{{\color{green}#1}}
\definecolor{gold}{rgb}{1.0, 0.65, 0.0}
\newcommand{\ale}[1]{{\color{gold}#1}} 
%\newcommand{\alecor}[1]{{\color{gold}\st{#1}}}
%%%%%%%%%%%%%%%%%%%%%%%%%%%%%%%%%%%%%%%%%%%%%%%%%%%%%%%%%%%


\begin{document}

\section{Hadron spectroscopy and QCD exotica}
{\bf Theory Authors: A. Esposito, Feng-Kun Guo, J. Nieves, A. Pilloni, A. Polosa, Sasa Prelovsek} 

\subsection{Open questions in spectroscopy}

A large number of  ``exotic'' experimental discoveries, which did not fit the expectations of the, until then, very successful quark model, as well as the unprecedented statistical precision obtained by LHCb, BESIII and other experiments, have led to a recent renaissance of hadron spectroscopy.
Understanding the strong interactions at low energies requires  an explanation on how the emergent hadron spectrum is  organized,  which is a nontrivial task due to the confinement property of QCD.  

To answer this question, a joint effort by experimental and theoretical communities is needed,  the latter making a combined use of lattice QCD results and effective field theory (EFT) methods.     The spectra of weakly decaying 
%stable 
hadrons as post-dicted or predicted by Lattice QCD, generally agree with the observed ones. Valuable conclusions 
can be drawn from lattice  also  for several   strongly decaying resonances,  where the  challenge increases with the number of open strong decay channels.  From the phenomenological side, potential quark models, inspired by QCD, describe mesons as bound states of a quark-antiquark pair, and baryons as bound states of three quarks. %, and were generally quite successful until 2003.  
One would expect for these models to work  particularly well in describing the heavy quarkonium sector. However,  other color configurations are  allowed,  such as mesons with two quarks and two anti-quarks, baryons with four quarks and one anti-quark, dubbed tetraquarks and pentaquarks, respectively.\footnote{In fact, such configurations were already suggested in Gell-Mann's seminal paper on quark model~\cite{GellMann:1964nj}.} These are the so-called exotic hadrons.  As already mentioned, their search has been an important theme in high energy experiments, although plagued by several ambiguous claims. The key questions are: Are exotic hadrons really allowed by QCD? If yes, what kinds are allowed? Why are they so scarce? Can lattice QCD predict their spectrum? Is it possible to construct QCD-based phenomenological models for (at least) some of them?


The situation dramatically changed in 2003,  when the charm-strange mesons $D_{s0}^*(2317)$, $D_{s1}(2460)$ and the charmonium-like $X(3872)$ were discovered by the BaBar, CLEO and Belle Collaborations, respectively.  None of them is in line with the predictions from potential quark models. For instance, their masses are lower than the predictions by Godfrey and Isgur by about 160~MeV, 70~MeV and 80~MeV, respectively~\cite{Godfrey:1985xj}.  As time passed, more and more unknown resonant structures were reported from various high-energy experiments, mostly BaBar, Belle, BESIII and LHCb.  Many new structures, not compatible with predictions, have been found in the charmonium mass region. These are traditionally called $XYZ$ states.\footnote{Notice, however, that the naming scheme was changed in the 2018 edition of the Reviews of Particle Physics (RPP) by the Particle Data Group~\cite{PDG2018}. For charmonium(-like) states, the isoscalar states with quantum numbers $J^{--}$, $J^{++}$ are called $\psi_{J}$, $\chi_{cJ}$, respectively, and the isovector states with $J^{--}$, $J^{+-}$  are called $R_{cJ}$, $Z_{cJ}$, respectively.} 
In particular, the isovector $Z$ states are explicitly exotic, because they have nonzero isospin and at the same time contain a heavy $Q\bar Q$ pair. Similarly, LHCb~\cite{Aaij:2015tga} reported two structures in the $J/\psi\, p$ spectrum.   If these were produced  by real QCD baryon resonances ($P_c$),  these states would contain at least five valence quarks/antiquarks.  So far there has been no consensus about the nature of these resonances, and  some of the related experimental peaks might even receive large contributions from kinematical effects, making their interpretation as real states ambiguous. To resolve these issues, high statistics data and the search for signatures of these states in several different processes are highly desirable.

Next, we introduce briefly two main types of models for these intriguing exotic hadron candidates.

\subsubsection{Tetraquarks} \label{tetra}


Two quarks and two antiquarks can lead to color singlets in different ways, which are difficult to choose from first principles. One possibility is for the constituents to be bound in a compact tetraquark (see, e.g.,~\cite{Maiani:2004vq,Maiani:2014aja}). 
% 
%
For experimental reasons, 
%connected to the experiments, 
the most studied systems are of the form $Q\bar Q q\bar q^{\,\prime}$, where $Q=c,b$ and $q^{(\prime)}$ are light quarks. Their spectrum is well described in terms of diquarks, with a spin-spin interaction between the constituents given by $H_I=2\kappa_{Qq}\big(\bm{S}_Q\cdot \bm S_q+\bm S_{\bar Q}\cdot \bm S_{\bar q^{\,\prime}}\big)$, with no coupling between the constituents of different diquarks~\cite{Maiani:2014aja}. The chromomagnetic couplings $\kappa$ are extracted from observed masses. It is found that two quasi-degenerate $1^{++}$ and $1^{+-}$ states are expected (identified with the $X(3872)$ and $Z_c(3900)$), together with a higher $1^{+-}$ state (the  $Z_c^\prime (4020)$). The same pattern should be replicated in beauty system where the two $Z_b(1^{+-})$'s have been discovered but the $X_b(1^{++})$ is still missing. 
The $Z_c$'s and $Z_b$'s are expected to fill in triplets of states, as confirmed experimentally.  The $X$ is an exception: only the neutral state is observed. This could be due to its accidental vicinity to threshold~\cite{Maiani:2017kyi}. The $X^{\pm}$ components are forced to decay only into charmonia and the computed rates are below the exclusion limits set experimentally, as of today. Decay modes such as, for example, $X^\pm\to J/\psi\rho^\pm$ should be eagerly sought in data.

Radial excitations are also possible: the charged $Z(4430)$ being the most remarkable example~\cite{Maiani:2008zz,Maiani:2007wz,Maiani:2004vq}. Its mass and decay modes fit perfectly a picture where it is the first radial excitation of the $Z_c$.
Lastly, one also has orbital excitations, allowing for a $1^{-}$ spectroscopy.  Prominent candidates are the observed  $Y(1^-)$ states. An example is the $Y(4630)-Y(4660)$ system (identified as a single particle in~\cite{Cotugno:2009ys}). Its mass fits the diquarkonium spectroscopy and its strong preference to decay into two baryons is easily explained by the breaking of the QCD string between the diquarks. For a review see~\cite{Esposito:2014rxa,Chen:2016qju,Lebed:2016hpi,Esposito:2016noz,Ali:2017jda}.
Remarkably, the $X$, $Z^{(\prime)}_c$ and $Z_b^{(\prime)}$ resonances are found experimentally $\delta\lesssim 10$~MeV {\it above} their related meson-meson thresholds ($D^0\bar D^{*0}$ for the $X$, etc.). 
%with a distance . 
Their measured widths follow a $\Gamma=A\sqrt{\delta}$ law, with the same coefficient $A$ for both charm and beauty states, as first observed in~\cite{Esposito:2016itg}.

The above feature, the suppression of the interactions across different diquarks and the preference for decaying into open-flavour mesons, can be qualitatively explained with a separation in space of the two diquarks due to a (tiny) potential barrier between the two~\cite{Selem:2006nd, Maiani:2017kyi,Esposito:2018cwh}. This would also make these exotic hadrons slightly larger than the standard ones ($\approx 1.3$~fm)~\cite{Esposito:2018cwh}. 
The idea of the diquark--antidiquark pair being slightly separated in space has also been presented in~\cite{Brodsky:2014xia}, where it was shown that considering diquarks as dynamical and moving with respect to each other leads to an interesting way of computing tetraquarks decays. The same picture has also been employed to study the spectrum of tetraquarks and possible selection rules for their decays~\cite{Lebed:2017min}, as well as to study pentaquarks~\cite{Lebed:2015tna}.
Lastly, it was also proposed that the scaling of the tetraquark exclusive production cross section with the center of mass energy might reveal important information on the composite nature of these states~\cite{Brodsky:2015wza,Lebed:2018jcr}. It follows that the study of production of exotic states is crucial.





\subsubsection{Hadronic molecules}

These are complex systems for which the dominant components are two or more interacting hadrons. For a review of theoretical and phenomenological aspects of hadronic molecules with a focus on the new heavy-flavour hadrons, we refer to~\cite{Guo:2017jvc}.

The simplest  hadronic molecule  would be a loosely bound state of two hadrons ($H_1, H_2$) with a sizable extension.  Such an object  would be the analog to a neutron-proton pair bound to form a deuteron.  If the physical state is located slightly below an $S$-wave threshold of two hadrons $H_1$ and $H_2$, the scattering length ($a$) and effective range ($r$)  of the continuum $H_1+H_2\to H_1+H_2$  scattering amplitude would be approximately given by
%
\begin{equation}
a=- 2 \frac{1-\lambda^2}{2-\lambda^2}\,\frac1\gamma + {\cal O} (\gamma/\beta), \qquad
r =- \frac{\lambda^2}{1-\lambda^2}\,\frac1\gamma + {\cal O} (\gamma/\beta), 
\end{equation}
%
where $\gamma = \sqrt{2\mu E_B}$ ,  with $\mu$ the $H_1, H_2$ reduced mass and $E_B>0 $  the binding energy of the physical state (i.e., the difference between the $H_1, H_2$ threshold and its mass),  $\beta$ the inverse range of forces  and $\lambda^2$ the wave function renormalization constant, which for a pure molecule vanishes. 
There is also a relation for the effective coupling of the physical state to $H_1$ and $H_2$, given by (using the non-relativistic normalization)
%
\begin{equation}
  g_\text{NR,eff}^2 = \frac{2\pi\gamma}{\mu} (1-\lambda^2) + {\cal O}(\gamma/\beta) .
\end{equation}
%
The $(1-\lambda^2)$ factor gives the molecular probability, i.e., the probability of finding the $H_1H_2$ component in the physical state (the Weinberg compositeness criterion~\cite{Weinberg:1965zz}).  Thus, for a pure molecule ($\lambda^2=0$),  one finds that the scattering length takes its maximum value,  $a=-1/\gamma$,  and in addition $r={\cal O} (1/\beta)$, while for a compact state $(\lambda^2=1)$ one gets $ a={\cal O} (1/\beta)$  and $r\to -\infty$. 
In addition, the effective coupling takes the maximal value for a pure molecule  and it vanishes when there is no $H_1H_2$ component in the physical state.
These differences produce distinctive  signatures on the line shapes of near-threshold states (see, e.g., the discussion in \cite{Guo:2017jvc,  Sekihara:2014kya}).	In the case of coupled-channel dynamics, one might also find states  that can decay into some of the open channels. That  will also qualify as hadronic molecules, if they strongly couple to any of the channels considered in the dynamical space. 



The hadron-hadron interactions are usually taken from EFTs  based on exact or 
%in general 
approximate QCD symmetries, which are employed to construct amplitudes satisfying unitarity in coupled channels.  The %appear 
undetermined low energy constants (LECs) are fitted to data or lattice QCD results, both  in the perturbative expansion, implicit in  the EFT, and  in the unitarization procedure.    



\subsection{Hadrons with a single heavy quark}

\noindent
\subsubsection{Mesons}

  The lowest positive-parity states include the very narrow charm-strange mesons $D_{s0}^*(2317)$, $D_{s1}(2460)$, and the very broad charm-nonstrange mesons $D_0^*(2400)$ and $D_1(2430)$~\cite{PDG2018}. Their bottom partners are still waiting for the discovery,  likely by LHC experiments.

\noindent
%\subsubsubsection{Phenomenology}
\paragraph{Phenomenology}\label{sec:spectro:mesons:pheno}
There were attempts trying to interpret the $D_{s0}^*(2317)$ and $D_{s1}(2460)$ as $c\bar s$ mesons, see, e.g.,~\cite{Godfrey:2003kg,Colangelo:2003vg,Mehen:2005hc,Lakhina:2006fy}, or as the positive-parity chiral partners of the ground state $D_s$ and $D_s^*$~\cite{Bardeen:2003kt,Nowak:2003ra}. In these cases, the isospin breaking hadronic decay widths are predicted to be of the order of 10~keV. Predictions for the radiative decays can be found in~\cite{Godfrey:2003kg,Bardeen:2003kt}. An updated calculation of the excited $c\bar q$ ($q=u,d,s$) meson spectrum and the decay properties in the relativized quark model can be found in~\cite{Godfrey:2015dva}.

The $D_{s0}^*(2317)$ and $D_{s1}(2460)$ states could be compact tetraquarks, namely $[cq][\bar s\bar q^{\,\prime}]$ states,
%~\cite{Maiani:2004vq}, 
where the  two quarks and  the two antiquarks are coupled to form a color singlet, see, e.g.,~\cite{Maiani:2004vq}.   Using the mass of the $X(3872)$  to fix the chromomagnetic coupling in the tetraquark model,  one can indeed accommodate both of them. Of course, this predicts other similar states with quantum numbers $(0,1,2)^+$, all expected to decay into $D_s^{(*)+} \pi^0$. The $D^{(*)} K$ mode should be open for the heavier members in the multiplet.  %  \sasa{\it I propose to list relevant decay channels for searching these other states and their present status. }


% Right after the discovery of the $D_{s0}^*(2317)$ and $D_{s1}(2460)$, they were suggested to be
Another possibility is to describe   the $D_{s0}^*(2317)$ and  $D_{s1}(2460)$ as $DK$ and $D^*K$ molecules~\cite{Barnes:2003dj,vanBeveren:2003kd}.
Dynamically they can be generated from charmed-meson--light-meson scattering in unitarized heavy meson chiral effective theory (UHMChPT)~\cite{Kolomeitsev:2003ac,Guo:2006fu,Guo:2006rp,Gamermann:2006nm},  using as input the scattering lengths calculated on the lattice~\cite{Liu:2012zya}.  Such scheme predicts  a (strangeness, isospin) $(S,I)=(1,0)$ state  with a  mass $2315^{+18}_{-28}$~MeV, in a agreement with that of the  $D_{s0}^*(2317)$ resonance.  Applying the Weinberg's compositeness condition allows to estimate the size of the molecular component to be about 70\%~\cite{Liu:2012zya}.  Such a picture is supported by the lattice energy levels reported in~\cite{Bali:2017pdv, Lang:2014yfa}, as discussed in \cite{Torres:2014vna, Albaladejo:2018mhb}. 

A decisive measurement is that of the width of the $D_{s0}^*(2317)$, which is expected to be around 100~keV in the molecular picture and smaller in other models~\cite{Colangelo:2003vg, Godfrey:2003kg}. So far, only an upper limit has been provided. It would be difficult to measure such a tiny width at LHC experiments for two reasons: because it requires a very high mass resolution, and because there is a neutral pion in the dominant decay mode $D_s^+\pi^0$. However, it could be possible to measure the width of its spin partner $D_{s1}(2460)$, which carries similar information, through decay mode $D_s\pi^+\pi^-$ and the Dalitz decay $D_{s1}(2460)\to D_s \gamma(\to\mu^+\mu^-)$. Dalitz decays have already been probed successfully for the $\chi_c$ states~\cite{Aaij:2017vck}. Both decays modes benefit from a very good mass resolution ($\sim1$~MeV) given the small $Q$ values of the reactions. Though a measurement of the width with a precision of 100~keV might be 
%a bit 
optimistic, 
%for sure 
LHCb should at least be able to improve on the current upper limit ($<3.5$~MeV). The proposed decay modes suffer from tiny branching fractions but the large integrated luminosity in the HL-HE era would help cope with that. 
% \others{M.Pappagallo: Ds pi0 is the only known decay mode for the Ds0*(2317). As the current text states, LHC(b) can't measure the natural width with the desired precision. However let me point out that the other state, Ds1(2460), goes through the dipion decay mode Ds pi pi as well. Another "new" decay mode would be Ds1(2460) --> Ds gamma (-> mu mu). Dalitz decays have been already probed successfully for the $\chi_c$ states [arXiv.1709.04247]. Both decays modes benefit of a very good mass resolution (~1 MeV) given the small Q values of the reactions. Though a measurement of the width with a precision of 100-keV might be a bit optimistic, for sure LHCb can pin down the current upper limit (3.5 MeV). The proposed decay modes suffer from tiny branching fractions but the large integrated luminosity in the HL-HE era would cope with that.} 

 In the $(S,I)=(0,1/2)$ sector,  UHMChPT suggests the presence of two broad $D_0^*$ states at about $2.10-i\,0.10$~GeV and $2.45-i\,0.13$~GeV~\cite{Albaladejo:2016lbb}, which would have masses different from the $D_0^*(2400)$ given in RPP~\cite{PDG2018},  determined from fitting to the $D\pi$ mass distributions using a single Breit--Wigner function. 
%It has been shown
Ref.~\cite{Du:2017zvv} showed that these {amplitudes} can reproduce well the $B^-\to D^+\pi^-\pi^-$ process measured by the LHCb Collaboration. The combination of angular moments, $\langle P_1\rangle- \frac{14}{9}\langle P_3\rangle$, is particularly sensitive to the $D\pi$ $S$-wave, and a {higher statistics} measurement in the energy range between 2.4 and 2.5~GeV will provide invaluable information on the $D\pi$--$D\eta$--$D_s\bar K$ coupled-channel dynamics. 
%Similarly, 
{Ref.~\cite{Du:2017zvv} also finds} two broad $D_1$ states at $2.25-i\,0.11$~GeV and $2.56-i\,0.20$~GeV, in addition to the relatively well-understood narrow $D_1(2420)$. A precise study of $\langle P_1\rangle- \frac{14}{9}\langle P_3\rangle$  for $D^*\pi$ $S$-wave is needed in the $\bar B\to D^{*}\pi\pi$ channel,
as well as in $\bar B\to D^{(*)}\eta\pi$, $\bar B\to D^{(*)}_s\bar K\pi$ $\bar B\to D^{(*)}_s\bar K\bar K$, $\bar B_s\to D^{(*)}\bar K\pi$, $\bar B_s\to D^{(*)}_s\eta\pi$, etc. In particular, signals of the predicted higher $D_0^*$ and $D_1$ states, coupled dominantly to $D_s^{(*)}\bar K$ could be near-threshold enhancements in the $D_s^{(*)}\bar K$ spectrum. Such enhancements exist in low-statistic data by BaBar~\cite{Aubert:2007xma} and Belle~\cite{Wiechczynski:2009rg,Wiechczynski:2014kxh}, and need to be investigated at LHCb using data sets with much higher statistics. 

Heavy quark flavour symmetry allows to predict the bottom partners of charmed mesons. The lowest $B_{s0}^*$ and $B_{s1}$ are predicted to be at about 5.72~GeV and 5.77~GeV~\cite{Albaladejo:2016lbb,Du:2017zvv}, consistent with lattice results~\cite{Lang:2015hza}. A good channel to search for both of them is $B_s^*\gamma$~\cite{Bardeen:2003kt,Cleven:2014oka}.
They can also decay into the isospin breaking hadronic channels $B_s^{(*)}\pi^0$, whose widths are expected to be smaller than their charm partners because the isospin splitting between $B^0$ and $B^\pm$ is one-order-of-magnitude smaller than that between $D^0$ and $D^\pm$. The axial state $B_{s1}$ can decay into $B_s\gamma$ as well. Both the $B_s\gamma$ and $B_s^*\gamma$ decay modes should appear in the $B_s\gamma$ spectrum with similar efficiencies (similarly to the $B_{s2}^*(5840) \to B^*K$ decay which peaks in the $BK$ spectrum as well~\cite{Aaij:2012uva}).
% \others{M. Pappagallo: I would suggest to add Bs pi0 for Bs0*. If it behaves as the Ds0*, the branching ratios of the radiative decays would be highly suppressed. Another remark is that Bs1 can decay to Bs g and Bs* g as well. Both decays modes should appear in the Bs g spectrum with similar efficiencies. (Similarly to Bs2* -> B*K decay which peaks in the BK spectrum as well [arXiv: 1211.5994])}. 
{Large integrated luminosity will cope with the low efficiency for detecting soft photons and LHCb experiment will have sensitivity to observe such states for the first time.}% (internal communication with M. Pappagallo, LHCb).
The predicted poles for the $B_0^*$ mesons are at about $5.54-i\,0.11$~GeV and $5.85-i\,0.04$~GeV, while for the $B_1$ mesons are at about $5.58-i\,0.12$~GeV and $5.91-i\,0.04$~GeV~\cite{Albaladejo:2016lbb,Du:2017zvv}. The bottom meson spectrum in quark models is different, for an updated calculation of the excited $b\bar q$ ($q=u,d,s$) meson spectrum and the decay properties in the relativized quark model can be found in~\cite{Godfrey:2016nwn}. And the predictions of bottom mesons in the parity doubling model can be found in~\cite{Bardeen:2003kt}. These resonances can be searched for in $B\pi$ and, for higher excited states, in $B_s\bar K$ final states.


Another puzzling charmed meson is the $D^*_{s1}(2860)$, whose decay pattern $\mathcal{B}(D^*K)/\mathcal{B}(DK)\simeq1$~\cite{Aubert:2009ah} is much larger than the expectation for the $D$-wave $c\bar s$ meson, the only available quark model option in that mass region. The above ratio can be understood if the $D^*_{s1}(2860)$ is mainly a $D_1(2420)K$ bound state. Its $2^-$ spin partner is predicted to be at about 2.91~GeV, decaying into $D^*K$ and $D_s^*\eta$. Their bottom analogues are instead at about 6.15~GeV and 6.17~GeV, decaying into $B^{(*)}\bar K$ and $B_s^{(*)}\eta$~\cite{Guo:2011dd}.


\noindent
\paragraph{Lattice QCD}

 
The most extensive spectrum  of higher-lying $D$ and $D_s$ mesons  on the lattice \cite{Cheung:2016bym} was obtained in the single-hadron approximation, where the decays of resonances and the effects of thresholds are not taken into account.  This lattice study predicts a large number  of states with $J\leq 4$~\cite{Cheung:2016bym}, most of which have not been discovered yet. Some of them might contain substantial gluonic components (hybrid mesons).   

A proper   treatment of   strongly-decaying resonance $R\to H_1H_2$,
with one open channel, 
% above one   threshold   
 requires simulations of single-channel  $H_1H_2$ scattering. This has been  accomplished  recently by several lattice collaborations for a series of resonances. Mostly resonances composed of $u,d,s$ were considered.    The infinite-volume scattering matrix $T(E)$ is extracted from the energies of $H_1H_2$ eigenstates on the finite lattice via the rigorous L\"uscher's formalism \cite{Luscher:1990ux}. The resulting scattering matrix $T(E)$ renders resonance masses, $M$, and decay widths, $\Gamma$, via Breit-Wigner type fits. A related strategy is to analytically continue   $T(E_c)$ to the complex energy plane, where the pole positions, $E_c\simeq M-\tfrac{i}{2}\Gamma$, are related to the resonance parameters.   The resonances  that strongly decay   to several final states require simulation of coupled-channel scattering, which is much more challenging. The Hadron Spectrum Collaboration managed to extract the coupled-channel scattering matrix   for few selected channels, while most of channels  are awaiting future simulations. 


Strong decays of resonances containing a single heavy quark were considered only for  the   low-lying charmed resonances  with $J^P=0^+$, $1^\pm $,  $2^+$ \cite{Mohler:2012na,Moir:2016srx}.
The $N_f=2$   simulation of $D\pi$ scattering \cite{Mohler:2012na} finds a broad $D_0^*$ in rough agreement with experiment.    As commented above,  a reanalysis of  $D\pi$--$D\eta$--$D_s\bar K$  coupled channels  \cite{Moir:2016srx} for $N_f=2+1$ suggests two $D_0^*$ states, with masses located around $2.10$ and $2.45$ GeV~\cite{Albaladejo:2016lbb}.  


The strongly-stable states that lie closely below $H_1H_2$ threshold might be sensitive to threshold effects. A proper way to treat these is to simulate $H_1H_2$ scattering. The mass of a shallow bound state corresponds to the  energy $E<m_1+m_2$, where the scattering matrix, $T(E)$, has a pole.  In this way, the effects of   $D^{*}K$ thresholds were found to push the masses  of the   $D_{s0}^*$ and $D_{s1}^*$ down, bringing them close to the experimental values \cite{Bali:2017pdv,Lang:2014yfa}.  Analogously, the yet-undiscovered strongly stable $B_{s0}^*$ and $B_{s1}^0$ were predicted at  $5.71$ and  $5.75$ GeV,  respectively~\cite{Lang:2015hza}. For the decay modes that can be used in searching, we refer to the discussion in \ref{sec:spectro:mesons:pheno}.
%in the ``Phenomenology" part of this subsection.
%  \\
%
%\noindent
%
%
%\noindent
\subsubsection{Baryons}


The most extensive spectrum of yet undiscovered singly charmed baryons was predicted on the lattice in 2013 \cite{Padmanath:2013bla}. It provided five \others{M.Pappagallo: 5 are the states observed so far. The paper predicts many more states.}   $\Omega_c$ baryons, in agreement with the 2017 LHCb discovery.  This work  predicted also up to ten $\Lambda_c,~\Sigma_c,~\Omega_c,~\Xi_c$ states in each  channel  with $J\leq 7/2$,  where resonances are treated in a simplified single-hadron approach.   A follow-up precision lattice study of the five discovered $\Omega_c$ resonances  confirms their most likely quantum numbers \cite{Padmanath:2017lng}.  

The experimental discovery of the five $\Omega_c$ states has triggered large theoretical activity,
% in
%the field, and thus 
with some of the quark models 
%have been 
revisited in view of the new result~\cite{Karliner:2017kfm, Wang:2017hej, Chen:2017gnu, Cheng:2017ove, Wang:2017vnc, Yang:2017rpg}. 
%On the other hand, t
The role of diquarks in the $\Omega_c$ spectrum was discussed in~\cite{Ali:2017wsf, Kim:2017jpx},  while the odd-parity molecular interpretation for two or three of the states seen by LHCb was proposed in Refs.~\cite{Nieves:2017jjx, Montana:2017kjw,  Debastiani:2017ewu}.  The
meson-baryon interactions used in the  molecular schemes, derived in \cite{Nieves:2017jjx,  Debastiani:2017ewu},  are consistent with both chiral and heavy-quark spin symmetries, and lead to  successful descriptions of the observed lowest-lying odd parity $\Lambda_c(2595)$, $\Lambda_c(2625)$~\cite{GarciaRecio:2008dp, Romanets:2012hm, Liang:2014kra, Liang:2014eba} and $\Lambda_b(5912)$,  $\Lambda_b(5920)$~\cite{GarciaRecio:2012db} resonances.  Some of the $\Omega_c$ states observed by LHCb could thus be  spin-flavour symmetry partners of these  $\Lambda^*_{Q=c,b}$ baryons.  However,  the masses and decay widths of $\Lambda^*_{Q=c,b}$,  at least the charmed ones,  can also be accommodated  within  usual constituent quark models (see for instance \cite{Yoshida:2015tia, Nagahiro:2016nsx }), and thus the importance of the molecular components in their structure has not been settled yet.  

Molecular schemes also predict partners of the $\Omega_c$ baryons in the bottom sector (see for instance, Ref.~\cite{Liang:2017ejq}). 
%Noticeably, 
LHCb recently reported~\cite{Aaij:2018yqz}  a peak in both the  $\Lambda_b^0 K^-$ and $\Xi^0_b\pi^-$ invariant mass spectra, that  might  correspond to either radially or orbitally excited quark model  $\Xi_b^-$(6627) resonance, with quark content $bds$, or to  a hadron molecule. In the latter, it would be  dynamically generated from the coupled-channels interaction between Goldstone bosons and the lowest even parity bottom baryons.  For a review of recent observations and phenomenological models of open-flavour heavy hadrons, we refer to~\cite{Chen:2016spr}.

\subsection{Hadrons containing $\bar cc$, $\bar bb$ or $\bar cb$}
 
The current experimental status for the hadrons containing a heavy quark-antiquark pair can be found in  several recent reviews~\cite{Esposito:2016noz, Guo:2017jvc, Olsen:2017bmm, PDG2018},  and will not be repeat here. In the following, we discuss  isospin-scalar quarkonium(-like) states and, separately,  
%explicitly 
exotic charged states, and focus on a few selected important issues.
\noindent
\subsubsection{Quarkonia}

\paragraph{Phenomenology}
The $J^{PC}=1^{++}$ $X(3872)$ state has several salient features: the central value of its mass coincides 
%exactly 
with the $D^0\bar D^*$ threshold within a small uncertainty of 180~keV~\cite{PDG2018}; despite the tiny phase space, its decay branching fraction into $D^0\bar D^0\pi^0$ is larger than 40\%~\cite{PDG2018}; it decays into  $\omega J/\psi$ ($I=0$) and $\rho J/\psi$ ($I=1$) with similar partial decay widths. 

This state could be interpreted in terms of a $[cq][\bar c\bar q]$ compact tetraquark. In this case, the isospin violation is explained by the smallness of $\alpha_s(m_c)$~\cite{Rossi:2004yr,Maiani:2004vq}, which suppresses the mixing between the almost degenerate mass eigenstates $X_u=[cu][\bar c\bar u]$ and $X_d=[cd][\bar c\bar d]$.
Isospin symmetry predicts both a degenerate charged partner $X^+$, as well as $0^{++}$, $0^{++\prime}$ and $2^{++}$ partners with masses around $M(0^{++})\simeq3.8$~GeV and $M(0^{++\prime})\simeq M(2^{++})\simeq4$~GeV~\cite{Maiani:2014aja}.  Were the $Z_c(4050)$ a scalar or a tensor, it would have been a suitable candidate for one of the above states (see e.g.~\cite{Olsen:2015zcy}). If confirmed, another possible candidate  for the heavier scalar state could be the resonance observed by LHCb in $\eta_c\pi$~\cite{Aaij:2018bla}. \fk{FK: here I am confused: what is the isospin of the $0,2^{++}$ states? I thought they were isoscalar. $Z_c(4050)$ and $\eta_c\pi$ are isospin vectors. }



%As for t
Given the expected mass,  the charged partner predicted in the tetraquark picture, $X^+$, might only decay into a charmonium and a light meson. If one assumes a potential barrier between the diquarks~\cite{Maiani:2017kyi, Esposito:2018cwh}, such a decay would require the tunneling of a heavy quark. This would make the rate of strong decay comparable to the electromagnetic ones, hence making the above channel particularly elusive. Nevertheless, it should be visible with sufficient statistics, and its search is of great interest. As already stated in Section~\ref{tetra}, the charged partners of the $X$ could be searched in the $J/\psi\rho^\pm$ channel.


The properties of the $X(3872)$, on the other hand,  indicate that it couples strongly to $D\bar D^*$. This lead to the proposal that it could be treated as dominantly a $D\bar D^*$ molecule. The line shapes of the $X(3872)$ in both $J/\psi\pi^+\pi^-$ and $D^0\bar D^{*0}$ modes are crucial to reveal its nature and binding mechanism~\cite{Hanhart:2007yq,Kalashnikova:2009gt,Artoisenet:2010va} (for a review, see~\cite{Kalashnikova:2018vkv}). Improved measurements at the LHC experiments are foreseen. In the hadronic molecular picture,  the $2^{++}$ heavy-quark spin partner of the $X$, dubbed $X_2$, would decay into $D\bar D^{(*)}$ and $D\bar D$ in $D$ wave,  and it is expected to be narrow~\cite{Albaladejo:2015dsa,Baru:2016iwj}. However,
it has  been suggested~\cite{Cincioglu:2016fkm} that the mixing of the $D^*\bar D^{*}$ molecule with bare charmonium  $\chi_{c2}(2P)$ might destabilize the $X_2$~\cite{Ortega:2017qmg}, making it hardly visible. 
Better experimental information in the $2^{++}$ spectrum around 4~GeV is hence of great importance, and it can be searched for in $D\bar D$ and $J/\psi\omega$. So far, there is only one observed state $\chi_{c2}(3930)$  compatible with a standard charmonium assignment, with no hint for an $X_2$ candidate.  \td{JZ: not a very clear sentence, please improve:} Yet, the existence of the charmonium $\chi_{c1}(2P)$, and its eventual relation with the $X(3872)$,  is actually an open problem of the utmost importance from the theoretical perspective~\cite{Cincioglu:2016fkm}.


The $0^{++}$ spectrum is also still unclear: the spin of the narrow $X(3915)$ is not fixed and the broad $X(3860)$~\cite{Guo:2012tv,Chilikin:2017evr} needs confirmation. Since the $(0,1,2)^{++}$ and $1^{+-}$ heavy quarkonia differ from each other only by the quark polarization, it is necessary to consider systematically physical states with these quantum numbers ~\cite{Cincioglu:2016fkm,Zhou:2017dwj}.  Thus, searching for an isoscalar $1^{+-}$ state around 3.9~GeV is also of high interest~\cite{Lebed:2017yme}. Its important decay modes include $D\bar D^*$, $J/\psi\eta$ and $J/\psi\pi\pi$. Hints for such a state have been seen by COMPASS~\cite{Aghasyan:2017utv}, albeit with low statistics.

It is also crucial to look for the analogue of the $X(3872)$ in the bottom sector. Such a state could decay, for example, into $B\bar B\gamma$, 
 $\chi_{bJ}\pi\pi$, $\Upsilon\pi\pi\pi$ and $\Upsilon\gamma$~\cite{Guo:2013sya,Guo:2014sca,Karliner:2014lta}. 
 %In particular, t
 The $X_b\to\Upsilon\rho$ channel has already been investigated, with no observation~\cite{Chatrchyan:2013mea,Aad:2014ama}. Note that in the tetraquark model one would expect isospin violation, while the opposite is true in a molecular scheme. The isospin conserving channel $X_b \to \Upsilon \omega$ has found no evidence for the state either~\cite{He:2014sqj}. The spin-2 bottomonium-like $X_{b2}$ and $B_c$-like state $X_{bc2}$~\cite{Guo:2013sya} can instead be searched in $D$-wave $B\bar B$ and $DB$ final states, respectively.
 
  
Vector $1^{--}$ states have also been observed at $e^+e^-$ colliders (see e.g.~\cite{Aubert:2005rm,Liu:2013dau,Ablikim:2013dyn,Wang:2014hta,Lees:2012pv}). The number of observed vector charmonium-like states far exceeds the predictions in the $c\bar c$ potential quark models, and thus some of them must have an exotic origin. In the tetraquark model these are diquarkonia with orbital angular momentum $L=1$, and their spectrum easily matches the  experimental observations~\cite{Maiani:2014aja}, which also leads to a distinctive spectrum 
%of predicting 
with a low-mass $3^{--}$ state~\cite{Cleven:2015era}. Complexity in understanding these structures also comes from the many $S$-wave thresholds of hadron pairs, such as the $\bar D D_1(2420)$ for the $Y(4260)$~\cite{Wang:2013cya} and $\psi'f_0(980)$ for the $Y(4660)$~\cite{Guo:2008zg}. The search for these states in, for example, prompt production and/or $B$ decays can help to understand their nature.
 

%\noindent
\paragraph{Lattice QCD}

The lattice spectrum of bottomonia presented in~\cite{Wurtz:2015mqa} contains almost all the observed $\bar bb$ states up to the $\bar BB$ threshold. It also predicts a plethora of undiscovered states where the $\bar bb$ pair carries an orbital angular momentum $L=2,3,4$, and total angular momentum $J\leq 4$.   Fourteen $B_c$ mesons with $J\leq 3$   are predicted up to the $BD$ threshold~\cite{Wurtz:2015mqa}. Only two of them have been discovered so far.  

All the charmonia below the $D\bar D$ threshold have been experimentally discovered, while the treatment of strongly decaying resonances is much more challenging. The most extensive spectrum obtained in the simplified  single-hadron approach predicts several excited $\bar cc$ states, as well as $\bar cc g$ hybrids up to $4.7$ GeV, carrying   $J\leq 4$~\cite{Cheung:2016bym}, also including exotic $1^{-+},~0^{+-}, ~2^{+-}$ quantum numbers. Most of them have not been observed yet.  Only one exploratory study considered the resonant nature of charmonia above the open-charm threshold \cite{Lang:2015sba} and underlines the need to  experimentally explore further $\bar DD$ in $S$ and $D$ wave.    A neutral $X(3872)$ is found  as a state slightly below $D\bar D^*$~\cite{Prelovsek:2013cra}.  No other   $1^{++}$ state is found between threshold and  $4.1$ GeV, in agreement with the experiment. Similarly, no indication for the isospin-1 partner of $X(3872)$  is found   \cite{Padmanath:2015era}. Note that these simulations were performed  in the $m_u = m_d$  limit.    


\noindent
\subsubsection{Charged exotic states}




%\noindent
\paragraph{Phenomenology}

The $Z_c^{(\prime)}$ and $Z_b^{(\prime)}$ are 
%also 
predicted by the diquarkonium model as part of the $[cq][\bar c\bar q]$ spectrum~\cite{Ali:2011ug,Maiani:2014aja}. A possible hint on their internal structure could be provided by the $Z_c^{(\prime)}\to\eta_c\rho$ decay, for which the tetraquark and molecular models predict statistically different results~\cite{Esposito:2014hsa,Agaev:2016dev,Agaev:2017tzv,Xiao:2018kfx}. Similar analyses could also be performed for $Z_b^{(\prime)}\to\eta_b\rho$. Searching for the decay $Z_c^\prime\to J/\psi\pi$ with higher statistics is also important to confirm/disprove a possible tension between the molecular model and the experimental data~\cite{Esposito:2014hsa}. Analyses as in~\cite{Pilloni:2016obd}, can give a more robust extraction of the resonant parameters, thus helping in distinguishing between the possible interpretations.
The $Z(4430)$ is also easily accommodated in the tetraquark model, as the first radial excitation of the $Z_c$~\cite{Maiani:2014aja}. Such a state has been discovered in a $\psi(2S)\pi$ final state, but the search in other channels is also important (e.g., $J/\psi\pi$).

From a molecular perspective, once again the $Z_c^{(\prime)}$ and $Z_b^{(\prime)}$ are close to the $D^{(*)}\bar D^*$ and $B^{(*)}\bar B^*$ thresholds, respectively,  and thus their pole locations are not precisely determined. It was shown in~\cite{Albaladejo:2015lob,Pilloni:2016obd} that the BESIII data on the $J/\psi\pi$ and $D\bar D^*$ distributions are consistent with either a virtual state or a resonance, which is also consistent~\cite{Albaladejo:2016jsg} with the energy levels calculated on the lattice~\cite{Prelovsek:2014swa}. The data for the lighter $Z_b$ are also consistent with a virtual state~\cite{Guo:2016bjq,Wang:2018jlv},  although the state lies so close to threshold, that is hard to draw final conclusions based on  the present data. Line shapes near the open-flavour thresholds in both the open- and hidden-flavour channels need to be extracted precisely, in order to pin down the positions of the poles.

It is worth noticing that a threshold cusp with a nearby pole is not enough to produce peaks as narrow as the ones observed  for the $Z_c$~\cite{Guo:2014iya}.
There are suggestions that nearby triangle singularities play an important role in producing the $Z_{c(b)}$ structures associated with a pion in $e^+e^-$ collisions~\cite{Wang:2013cya,Wang:2013hga,Szczepaniak:2015eza,Pilloni:2016obd,Gong:2016jzb,Bondar:2016pox}. In order to disentangle such effects, the $Z_{c(b)}$ states need to be searched in other processes. Recently, the D0 Collaboration reported an evidence for the $Z_c$ in semi-inclusive $b$-flavoured hadron decays~\cite{Abazov:2018cyu}, which is waiting for a confirmation from the LHC experiments. 

In the molecular picture the $Z_b$ states probably have more isospin-vector partners with quantum numbers $(0,1,2)^{++}$, denoted as $W_{bJ}$ states~\cite{Voloshin:2011qa}. The searches for  isospin vector $X_b$ in $J/\psi\pi^+\pi^-$, by CMS~\cite{Chatrchyan:2013mea} and ATLAS~\cite{Aad:2014ama}, should be reinterpreted as negative results of searching for the $W_{bJ}$ states. Improved experimental information would be of great importance for understanding the charged bottomonium-like structures.

%\noindent
\paragraph{Lattice QCD}

Lattice QCD does not find candidates for $\bar QQ\bar du$ ($Q=c,b$) states below the strong decay thresholds $m_{\bar QQ}+m_\pi$. The study of the $Z_c$ and $Z_b$ resonance above threshold is much more challenging as it requires the simulation of coupled-channel scattering.  For this reason, lattice  results for the charged $Z_c$ and $Z_b$  states using the L\"uscher's approach are not (yet) available. 

  The  indication for the lowest-lying $Z_c(3900)^+$   was found in the lattice study based on the somewhat simpler HALQCD method~\cite{Ikeda:2016zwx}, where the peak arises due to a threshold cusp and not to a resonance pole.   This seems to be in accordance with the absence of additional energy eigenstates in \cite{Prelovsek:2014swa,Cheung:2017tnt}, but more work based on the more rigorous L\"uscher's method is needed. An indication for the $Z_b$ below the $B\bar B^*$ threshold was found using static $b$ quarks, but $\Upsilon \pi$ states  with non-zero pion momentum must be incorporated before drawing any firm conclusion~\cite{Peters:2017hon}.   
 

\subsubsection{Pentaquarks}


%\noindent
\paragraph{Phenomenology}

The observation of  two peaks in the $J/\psi p$  spectrum  in $\Lambda^0_b \to J/\psi K^-p$ decays by LHCb in 2015 ~\cite{Aaij:2015tga} triggered a large activity in the field.  If they are interpreted as  actual  QCD resonances,  they will be  consistent with pentaquark states with opposite parities. This 
%latter feature  
 seems to indicate that diquarks play a role in their structure~\cite{Maiani:2015vwa}; it is 
 %indeed 
 hard to accommodate orbitally excited states in the framework of a bound state of color neutral hadrons. Moreover, the diquark-diquark-antiquark model correctly reproduces the observed mass splitting between the two pentaquarks~\cite{Maiani:2015vwa}. The SU(3) flavour symmetry then opens up a vast spectroscopy~\cite{Maiani:2015vwa,Maiani:2015iaa,Ali:2016dkf,Ali:2017ebb}, for which experimental study is of great interested. Such pentaquarks could be found in reactions as, for example, $\Xi_b(5749)\to K\,P_c\to K(J/\psi\Sigma(1385))$, $\Omega_b(6049)\to \phi \,P_c\to\phi(J/\psi\Omega(1672))$ or $\Omega_b(6049)\to \phi \,P_c\to K(J/\psi\Xi(1387))$~\cite{Maiani:2015vwa}. More pentaquarks could also be observed in the $J/\psi p$ or $\eta_c p$ final states~\cite{Ali:2017ebb}. Predictions of hidden-charm pentaquarks prior to the discovery can be found in~\cite{Wu:2010jy,Yang:2011wz,Wu:2012md}.
There exists also molecular  (${\bar D}^{(*)}\Sigma_c^{(*)}$) interpretations for the narrowest of the peaks (see for instance~\cite{Roca:2015dva, Roca:2016tdh}) which imply existence of additional  pentaquark states, not yet observed~\cite{Xiao:2013yca}.   

One complexity in interpreting the experimental observation comes from triangle singularities,  which are indeed present in this mass region~\cite{Guo:2015umn,Liu:2015fea}. In particular, a triangle singularity can produce a narrow peak around  $\simeq4.45$~GeV, if coupled to the $\chi_{c1}p$ in $S$ wave~\cite{Bayar:2016ftu}.  There are ways to distinguish triangle singularities from real resonances: $i)$ quantum numbers, since triangle singularities produce narrow peaks only for $S$-wave couplings; $ii)$ the near-threshold $\chi_{c1}p$ distribution for the process $\Lambda_b\to \chi_{c1}p K$,  since a triangle singularity would not lead to a near-threshold peak in the $\chi_{c1}p$ invariant mass distribution~\cite{Guo:2015umn} if the inelasticity between the channels is not large; $iii)$ searching for the $P_c$ in processes having different kinematics, as in photoproduction~\cite{Kubarovsky:2015aaa,Blin:2016dlf}.
Hidden-charm pentaquarks also need to be searched for in $\Lambda_c\bar D^{(*)}$ since their branching fractions could be much larger than that of $J/\psi p$~\cite{Shen:2016tzq,Lin:2017mtz,Eides:2018lqg}. 
There exist model predictions of hidden-bottom pentaquarks~\cite{Wu:2010rv,Wu:2017weo,Yamaguchi:2017zmn,Shen:2017ayv,Lin:2018kcc,Yang:2018oqd}, which can only be searched for at LHC currently. Important modes include $\Upsilon p$ and $\Lambda_b B^{(*)}$.


 \subsection{Doubly-heavy hadrons}
 
   
%  \noindent
\subsubsection{Baryons} 
 

 
%\noindent
\paragraph{Phenomenology}

    The quark model predictions for the spectrum of the states can be found in~\cite{Gershtein:2000nx,Kiselev:2001fw,Ebert:2002ig,Kiselev:2002iy, Albertus:2006ya}.  Decay properties of the doubly-charm baryons are discussed in~\cite{Yu:2017zst,Wang:2017mqp}. It is worth noticing that the recent LHCb determination~\cite{Aaij:2017ueg}  of the $\Xi_{cc}^{++}$ (ground state) mass ($3621.40 \pm  0.72 \pm  0.27 \pm  0.14$) MeV solves a longstanding discrepancy between quark model and lattice QCD predictions, and the previous value ($3519 \pm 1$ MeV) reported by the SELEX Collaboration~\cite{Mattson:2002vu}.
    
    The next doubly-charm baryons above the $1/2^+$ and $3/2^+$ states are the $1/2^-$ ones.  There are two possible origins: either the $[cc]$ diquark is in a $P$-wave state, or the diquark and the remaining quark have an $L=1$ orbital angular momentum relatively to each other. These two different possibilities mix to form the physical $1/2^-$ states. As a result, there will likely be two $1/2^-$ $\Omega_{cc}$ below the $\Xi_{cc} \bar K$ threshold, and three  $1/2^-$ $\Xi_{cc}$ isospin-doublets below 4.2~GeV. Among the latter, the lowest one (about 200~MeV heavier than the observed ground state $\Xi_{cc}$) is narrow and can be searched for in the $\Xi_{cc}^{++}\pi^-$ channel~\cite{Yan:2018zdt}. 

 
% \noindent
\paragraph{Lattice QCD}

 The mass of the only discovered doubly-heavy baryon  $\Xi_{cc}^{++}=ccu$ is in a very good agreement with the lattice predictions for the $1/2^+$ state.     Other low-lying baryons, containing at least two heavy quarks, have been predicted by a number of simulations, which broadly agree on their masses. The most extensive spectra of $\Xi_{cc}=ccq$ and $\Omega_{cc}=ccs$ baryons were predicted  using the single-hadron approximation in \cite{Padmanath:2015jea}, where up to ten of them were found in each channel with $J\leq 7/2$. Finding the lowest lying $3/2^+$ state $\Xi_{cc}$ may be challenging for LHCb, as it is predicted to be only  80-100 MeV above the  discovered $\Xi_{cc}^{++}$. The next candidate for discovery may well be $\Omega_{cc}$~\cite{Karliner:2018hos} predicted at  $3712\pm 10 \pm 12$ MeV \cite{Mathur:2018rwu},  with decay modes such as $\Xi_c^+\bar K^0$, $\Xi_c^+K^-\pi^+$, $\Xi_c^0 K^- \pi^+\pi^+$   and $\Omega_c \pi^+$. 
%  \others{M. Pappagallo: why not Xic+ K- pi+ as well? If the new narrow Omegac** are there, it would help to kill the combinatorial background.}
 The recent predictions for the masses of hadrons containing both charm and bottom quarks are discussed, for example, in~\cite{Mathur:2018epb}. Their production and decays are discussed, for example, in~\cite{Karliner:2018hos,Karliner:2014gca}. 

 
   \noindent
\subsubsection{Tetraquarks}

%\noindent
\paragraph{Phenomenology}

The discovery of the $\Xi_{cc}^{++}$ revived the study of doubly-heavy tetraquarks. Using heavy quark symmetry arguments, it is generally expected~\cite{Karliner:2017qjm,Eichten:2017ffp,Czarnecki:2017vco} that the doubly-bottom ground state $\bar b\bar b ud$ is stable so that it can only decay weakly, in line with the lattice results. 
In fact, the stability of  ground state $\bar b\bar b ud$ has been pointed out long ago~\cite{Ader:1981db,Manohar:1992nd}. 
Excited states are discussed in Ref.~\cite{Mehen:2017nrh}.  The doubly-charm states are likely above the open-charm thresholds, and may form resonances~\cite{Esposito:2013fma,Karliner:2017qjm,Eichten:2017ffp}. The results of various models are collected in Ref.~\cite{Luo:2017eub}. 
Being explicitly exotic, the search of doubly-heavy tetraquarks is of high interest. The production of doubly-heavy tetraquarks at LHC has been discussed in Refs.~\cite{Yuqi:2011gm,Ali:2018xfq}, and the results show that it is promising to discover them at the LHC experiments CMS, ATLAS and LHCb. The weakly decaying ground state doubly-beauty hadrons could be searched for by displaced $B_c^-$ mesons as an inclusive signature~\cite{Gershon:2018gda}.

One of the most appealing features of these states is the presence of a doubly-charged particle in their spectrum~\cite{Esposito:2013fma}. If it were to be discovered, this resonance would be undeniably a compact tetraquark: Coulomb repulsion prevents molecules of this kind from binding.
 


 %\noindent
\paragraph{Lattice QCD}
 
A strongly stable exotic state $\bar b\bar b du$ with $I(J^P)=0(1^+)$   is found up to $0.2$ GeV below the $B\bar B^*$ threshold in several independent lattice approaches, for example \cite{Bicudo:2012qt,Francis:2016hui}. The search for this exotic tetraquark  should be of prime interest at LHC.  Examples of fully reconstructible modes for its weak
decays are   $B^+ \bar D^0$ and $J/\psi B^+ K^0$,  with $\bar D^0\to K^+\pi^-$, $B^+\to \bar D^0\pi^+$ and $K^0\to \pi^+\pi^- K_s$ \cite{Francis:2016hui}. 
   The   $1^+$ state $\bar b\bar bqs$  is also found below the $BB_s$ threshold~\cite{Francis:2016hui} and  it could be   searched for in the weak decays to $J/\psi B_sK^+$, $J/\psi B^+\phi$ for $q=u$, and $B^+D_s^-$, $B_s\bar D^0$, $J/\psi B^0\phi$, $J/\psi B_s K^0$ for $q=d$.   The  systems $\bar c\bar cud$, $\bar c\bar cus$, $\bar b\bar bss$, $\bar b\bar bcc$, $\bar b\bar bbb$, on the other hand, were not found to have compelling signals of bound states below strong decay thresholds  in the channels  explored (see for example \cite{Bicudo:2015vta,Hughes:2017xie}). 


\subsection{All-heavy states}

%\noindent
%\subsubsection{Phenomenology}

%There have been predictions for the existence of s
States with all heavy constituents were predicted in Refs.~\cite{Heller:1985cb,Berezhnoy:2011xn,Wu:2016vtq,Chen:2016jxd,Karliner:2016zzc,Bai:2016int,Wang:2017jtz,Richard:2017vry,Anwar:2017toa,Vega-Morales:2017pmm,Eichten:2017ual,Esposito:2018cwh}. The widths of the ground state $cc\bar c\bar c$ and $bb\bar b\bar b$ are expected to be of the order of a few tens of MeV due to the decays through annihilation of a quark-antiquark pair~\cite{Chao:1980dv,Anwar:2017toa}.
The experimental search of a $bb\bar b\bar b$ state has been of particular interest recently. Indeed, the presence/absence of such a state would be very informative about the nature of the exotic mesons, and different models predict that, with sufficient luminosity, it should be observed~\cite{Eichten:2017ual,Esposito:2018cwh}.

%\noindent
%\subsubsection{Lattice QCD}

The only lattice simulation of the  $\bar b\bar bbb$ system \cite{Hughes:2017xie} found no evidence for  bound states with a mass below the lowest   bottomonium-pair thresholds, i.e., no $0^{++}$, $1^{+-}$ and $2^{++}$ states    below $\eta_b\eta_b$,   $\Upsilon\Upsilon$ and $\eta_b\Upsilon$ thresholds, respectively. The $b$ quarks were treated using improved NRQCD and all four   Wick contractions were taken into account (omitting bottom annihilation). If only part of the Wick contractions (for example "direct") are taken into account, the system seems to show false binding  \cite{Hughes:2017xie}, which  could be reason why some model approaches (effectively treating only part of the Wick contractions) find false bound states below thresholds.     No indication was found also of  $\bar b\bar b cc$ bound states with $J^P=0^+,~1^+,~2^+$ below the   thresholds \cite{Bicudo:2015vta}, where  $cc$ carry spin $1$; this was explored using static $b$ quarks using Born-Oppenheimer approach.        The existence of resonances with two heavy quarks and two heavy antiquarks ($Q=b,c$)    above strong decay thresholds has not been explored on the lattice.   



% \subsection{Vector states}

% \noindent
% {\bf Phenomenology}



\subsection{Probes from prompt production in $pp$}
The sizable prompt production cross section of the $X(3872)$ in high energy hadron collisions has triggered lots of debates in the literature on whether it is compatible or not with a pure molecular nature.
The argument, first formulated  in~\cite{Bignamini:2009sk}, relies on the observation that in ${\mathcal O}(\text{TeV})$ collisions, it is unlikely to produce a pair of $D^0$ and $\bar D^{*0}$ --- the alleged constituents of the $X(3872)$ --- with a relative momentum $k_\text{rel}$ small enough for the binding to occur. The identification of the support of $k_\text{rel}$ is crucial, since by phase space arguments the cross section scales with the cube of the maximum $k_\text{rel}$ allowed. Using the indetermination relations and the upper bound on the binding energy of the $X$, Ref.~\cite{Bignamini:2009sk} estimated $k_\text{rel} \lesssim 30~\text{MeV}$. Using Monte Carlo generators, this turns into a production cross section that is two orders of magnitude smaller then the one measured by CDF. This challenges the molecular interpretation. The choice $k_\text{rel}\sim 30~\text{MeV}$ was criticized in~\cite{Artoisenet:2009wk}: the inclusion of Final State Interaction (FSI) generating the bound state would require a relative momentum of the scale of the mediator, i.e. ,the pion mass. Integrating up to $k_\text{rel}\sim 300~\text{MeV}$ permits to reach the experimental cross section. However,
%The 
implementation of FSI in an environment polluted by hundreds of other particles is controversial~\cite{Bignamini:2009fn,Artoisenet:2010uu,Esposito:2013ada,Guerrieri:2014gfa}. The choice of the cutoff has  been object of a new debate recently~\cite{Albaladejo:2017blx,Esposito:2017qef}. 
A somewhat pragmatic approach was adopted in~\cite{Meng:2013gga}: the $X(3872)$ is considered to be a superposition of a molecular state, and of the ordinary $\chi_{c1}(2P)$. The former does not contribute to the production cross section, whereas the latter is estimated in NRQCD. An alternative estimate using NRQCD, mostly agnostic about the nature of the $X$, was anticipated in~\cite{Artoisenet:2009wk}. Note that, if the $X$ were to be a compact tetraquark, the high production cross section would not be an issue.

In~\cite{Esposito:2015fsa} the production cross section of the $X(3872)$ was compared, after high $p_T$ extrapolation, with the one of deuteron and other light nuclei. The cross section of the $X(3872)$ overshoots the others by a few orders of magnitude, thus undermining the idea that they have the same nature. Refs.~\cite{Albaladejo:2017blx,Wang:2017gay} proposed that the production of exotic states depends on their short-range nature, namely on the number of quarks in the minimal Fock component of the state, regardless of their molecular or compact nature, so that prompt production could give little information about the nature of these states. This would imply that some form of mixing with more compact states is required to reproduce the cross section. Prompt production, together with other processes sensitive to the short-range structure of the exotics, would thus probe the details of the mixing with compact charmonia and tetraquarks.  
Finally, under some  assumptions for the FSI, also predictions for production of the $Z_{c,b}$~\cite{Guo:2013ufa}, $X_b$~\cite{Guo:2014sca}, and $D_{sJ}$~\cite{Guo:2014ppa} have been made.



\subsection{Summary of interesting processes and states}

\fk{FK: Table to be added.}
\subsection{Experimental prospects}

The LHCb \upgradetwo\ detector will have a large impact on sensitivity in searches for heavy states. Aside from the much larger integrated luminosity, many of the 
detector improvements planned for LHCb \upgradetwo may have significant benefits for spectroscopy studies. For example, the potential removal of the VELO RF foils, together with the improved particle identification provide by the TORCH, will enhance the reconstruction efficiency for multibody \B decays, such as \decay{\Bcp}{\Dsp \Dz \Dzb}; the selection of short-lived particles ($\eg$ \Bc, \Xicc, \Omegacc, $\Xires_{\bquark\cquark}$, \etc) will also benefit from an improved vertex resolution; the Magnet Side Stations will help in studying dipion transitions such as \decay{\PX(3872)}{\chicone \pip \pim} or \decay{\B_\cquark^{**+}}{\Bc \pip \pim}; improved \piz and \etaz mass resolutions will increase the sensitivity in searching for the $C$-odd and charged partners of the $\PX(3872)$ meson by \decay{\PX(3872)^{C\rm{-odd}}}{\jpsi \etaz} and \decay{\PX(3872)^\pm}{\jpsi \piz \pipm}. A summary of the expected yields in certain important modes, and a comparison with \belletwo, is given in Table~\ref{tab:sec9_summary}.
In the following sections we will give some more details on the prospects for specific studies and analyses.

\begin{table}[tb]
\centering
\caption{\label{tab:yields}Expected data samples at \lhcb \upgradetwo and \belletwo for key decay modes for the spectroscopy of heavy flavoured hadrons. The expected yields at \belletwo are estimated by assuming similar efficiencies as at \belle.}
\begin{tabular}{l|ccc|c}
\hline
                      & \multicolumn{3}{c|}{\lhcb} & \belletwo\\
%\hline
Decay mode & 23\invfb & 50\invfb & 300\invfb & 50\invab\\
\hline
% See UpgradeII_vs_BelleII_yields.txt for details on these numbers
\decay{\Bp}{\PX(3872) (\to \jpsi \pip \pim)\Kp} &14k &30k&180k&11k\\
\decay{\Bp}{\PX(3872) (\to \psitwos \g)\Kp}     &500 &1k& 7k & 4k \\
\decay{\Bz}{\psitwos \Km \pip}                  &340k &700k&4M& 140k\\
\decay{\Bcp}{\Dsp \Dz \Dzb}                     & 10 &20& 100 & ---\\
\decay{\Lb}{\jpsi \proton \Km}                  &340k &700k&4M& ---\\
\decay{\Xibm}{\jpsi \Lz \Km}                    &4k &10k&55k& ---\\
\decay{\Xiccpp}{\Lc\Km\pip\pip}                 &7k &15k&90k & $<$6k\\
\decay{\Xires_{\bquark\cquark}^+}{\jpsi \Xires_\cquark^+} & 50 &100& 600 &---\\
\hline
\end{tabular}
\label{tab:sec9_summary}
\end{table}

\subsubsection{Taxonomy of tetraquarks and pentaquarks}

To advance our understanding of the $\PX(3872)$ state, it will be very important to learn even more about its decay pattern.
In particular, if it really has a strong $\Pchi_{\cquark 1}(2P)$ component, it should have \pip\pim transitions to the $\Pchi_{\cquark1}(1P)$ state.
Unfortunately at \lhcb, the reconstruction efficiency for the dominant $\Pchi_{\cquark 1}(1P)$ decay  to \g\jpsi decay is low, making this prediction hard to test.
 The very large data set of \upgradetwo  will allow this problem to be overcome. 
The very large data set of the \upgradetwo will be essential
in detecting or refuting such transitions.
Studies of the $\PX(3872)$ lineshape by a simultaneous fit to all detected channels are important for pinning down the location
of its resonant pole and determining its natural width; both are very important inputs in helping with the understanding of the state.
Therefore a very large data set will be essential for the statistical precision of such studies and reconstruction of decays to \Dz\Dstarzb, which are relevant given the proximity of the $\PX(3872)$ mass to the \Dz\Dstarzb threshold.

Searching for prompt production of any known exotic hadron candidates at \lhc remains an important task,
since its detection would signify compact component, either conventional quarkonium,  or a tightly bound tetraquark or pentaquark.
To date, the $\PX(3872)$ is the only exotic hadron candidate with $\PQ\Qbar$ content that has been confirmed to be produced promptly. This anecdotally speaks against compact interpretations for the other states.
However, it will be important to quantify the upper limits in negative searches to allow more rigorous phenomenological analysis. 

The prediction of magnetic moments for the light baryon states was a key success of the quark model.
As such these observables provide an additional handle on the structure of baryons and can potentially be used to
distinguish between several of the models that attempt to explain the structure of the observed pentaquarks~\cite{Wang:2016dzu}.

Experimentally the measurement of the magnetic moments is challenging.
Access can be gained
through radiative decays of pentaquark states. The observation of radiative decays
involving exotic baryons would thus provide a new window into the structure of the
pentaquark candidates. Interestingly, since the parity of the two observed states $\PP_\cquark(4380)^+$ and $\PP_\cquark(4450)^+$  are favoured to be 
opposite and their spins to be different by one unit, the radiative transition
between the two states should be allowed and might be observable by performing an amplitude analysis of the
\jpsi\proton{}\g\kaon decay. The proposed improved ECAL is crucial to the
feasibility of such a measurement in a high-luminosity environment.

To get an idea of the typical suppression
of these decays with respect to hadronic decays we can look at the measured branching fractions for
ordinary baryons and for exotic mesons. The $\decay{\PX(3872)}{\jpsi  \g}$ radiative decays are suppressed
by about a factor 50 with respect to decays into open charm and a factor of 5 with respect to the \jpsi \pip\pim decay.
Branching fractions of the order of $1\%$ (\eg $\BF(\decay{\Lz(1520)}{\Lz \g})=0.85 \pm 0.15\% $)
are common for radiative decays of baryons, which again are around a factor $50$ lower than the largest hadronic branching fraction.

\begin{figure}[th]
 \centering
\includegraphics[width=0.6\textwidth]{section6/figures/argand-upgrade-bw-fluc_v2.pdf}
\caption {Argand diagram of the $\PZ(4430)^-$ amplitude (${\rm A}_{\PZ(4430)^-}$) in bins of $m^2_{\psitwos \pim}$ from a fit to the $\decay{\Bz}{\psitwos \Kp \pim}$ decays. The black points are the results based on Run 1 data~\cite{LHCb-PAPER-2014-014} while the blue points correspond to an extrapolation to an integrated luminosity of 300\invfb expected at the \lhcb \upgradetwo. The red curve is the prediction from the Breit-Wigner formula with a resonance mass (width) of 4475 (172)\mev. Units are arbitrary.}
 \label{fig:argand}
\end{figure}

Many puzzling charged exotic meson candidates (\eg $\PZ(4430)^+$) decaying to \jpsi, \psitwos or $\Pchi_{\cquark 1}$ plus a charged pion have been observed in \B decays.
Some of them are broad, and none can be satisfactorily explained by any of the available phenomenological models.
The hidden-charmed mesons, observed in the $\jpsi\Pphi$ decay~\cite{LHCb-PAPER-2016-018,Chatrchyan:2013dma,Aaltonen:2009tz}, also belong to this category.
The determination of their properties, or even claim for their existence, relies on advanced amplitude analyses,
which allow the exotic contributions to be separated from the (typically) dominant non-exotic components.
Further investigation of these $\PQ\Qbar\quark\quarkbar$ structures
will require much larger data samples and refinement of theoretical approaches to parametrisations of hadronic amplitudes.
 Similar comments apply to improvements in the determination of the properties of the pentaquark candidates
$\PP_\cquark(4380)^+$ and $\PP_\cquark(4450)^+$ and to the spectroscopy of excited \Lz baryons in $\decay{\Lb}{\jpsi \proton \kaon}$ decays. The large data set collected during the \lhcb \upgradetwo would allow to test further the resonant character of the $\PP_\cquark(4380)^+$, $\PP_\cquark(4450)^+$ and $\PZ(4430)^+$ states (Fig.~\ref{fig:argand}), while improvements in calorimetry would help in searching for new decay modes (\eg $\decay{\PP_\cquark^+}{\Pchi_{\cquark 1,2} (\to \jpsi \g) \proton}$) by amplitude analyses of $\decay{\Lb}{\Pchi_{\cquark 1,2}\proton\Km}$ decays~\cite{LHCb-PAPER-2017-011,Guo:2015umn}.

\subsubsection{Searches for further tetra- and pentaquarks}

Though the true nature of the $\PX(3872)$ meson is still unclear, both the molecular~\cite{Nieves:2012tt}
and tetraquark~\cite{Maiani:2004vq} models predict that a $C$-odd partner ($\PX(3872)^{C\rm{-odd}}$)
and charged partners ($\PX(3872)^\pm$) may exist and decay to $\jpsi \etaz / \g \Pchi_{\cquark J}$ and \jpsi \piz \pipm respectively.

Similarly the existence of the $P_\cquark(4380)^+$ and $P_\cquark(4450)^+$ pentaquark states raises the question
of whether there is a large pentaquark multiplet. The observed states have an isospin
3-component of $I_3=+\frac{1}{2}$. They could be part of an isospin doublet with $I=\frac{1}{2}$ or a quadruplet with $I=\frac{3}{2}$.
In both cases there should be a neutral $I_3=-\frac{1}{2}$ state decaying into \jpsi \neutron. However this final state does
not lend itself well to observation. Instead the search for the neutral pentaquark candidate
 can be carried out using decays into pairs of open charm in particular in the process
\decay{\Lb}{\Lc\Dm\Kstarzb}, where the neutral pentaquark states would appear as resonances
in the \Lc\Dm subsystem (Fig.~\ref{fig:feynman}a). Such decays can be very well reconstructed but the total reconstruction
efficiency suffers from the large number of tracks and the small branching fractions of $\Lc$ and $\Dm$ reconstructable final states; the total reconstruction
efficiency is about a factor 50
smaller than the efficiency for the \decay{\Lb}{ \jpsi \proton \Km} channel.
In the case of the existence of an isospin quadruplet, there is the interesting possibility to find doubly charged
pentaquarks decaying into $\Sigmares_\cquark^{++} \Dzb$. Channels such as these
require very large data sets to offset the low efficiency. The magnet side stations will also improve the reconstruction efficiency of such decay modes with several tracks in the final states. 

The relative coupling of the pentaquark states to their decays into the double open-charm channels will
depend on their internal structure and the spin structure of the respective decay. For that reason
it is important to study decays involving \Dstarp resonances as well (\eg \decay{P_\cquark^+}{\Dstarm \Sigmares_\cquark^{++}})
 to investigate the internal structure of pentaquarks~\cite{Lin:2017mtz}. Since
these decays require the reconstruction of slow pions from the \Dstarp decays, the proposed tracking stations inside the magnet, enhancing the acceptance for low-momentum particles, will be highly beneficial for this study.

\begin{figure}[th]
 \centering
\includegraphics[width=0.32\textwidth]{section6/figures/Lb2PcKst_Pc2LcDm.pdf}
\includegraphics[width=0.32\textwidth]{section6/figures/Bc2D0D0bDs.pdf}
\includegraphics[width=0.32\textwidth]{section6/figures/Xbbud2DBpi.pdf}
\caption {Feynman diagrams.}
 \label{fig:feynman}
\end{figure}

Invoking $\rm{SU}_3$ flavour symmetry,
 one would expect the existence of pentaquarks with strangeness, which would
decay into channels like \jpsi \Lz or \Lc\Dsm. To explore the potential of
the former case the decay \decay{\Xires_\bquark^-}{ \jpsi \Lz\Km} has been studied using Run~1 data.
About 300 signal decays have been observed~\cite{LHCB-PAPER-2016-053}.
Complementary information can be also achieved by a study of the \decay{\Lb}{ \jpsi \Lz \Pphi} decays.
An increase of the available integrated luminosity by a factor of 100 would allow detailed amplitude analyses to be performed for 
 these final states, with a similar sensitivity as was the case for the pentaquark discovery channel.

The history of $\PX(3872)$ studies illustrates well the difficulty of distinguishing between exotic and conventional explanations for a hidden-charm state.
 Therefore it is appealing to search for  states with  an uncontroversial exotic signature.
A good candidate in this category would be a $\mathcal T_{\cquark\cquark}$ doubly charmed tetraquark~\cite{Moinester:1995fk,DelFabbro:2004ta,Carames:2011zz,Yuqi:2011gm,Hyodo:2012pm,Esposito:2013fma,Ikeda:2013vwa,Guerrieri:2014nxa,Maciula:2016wci,Richard:2016eis,Esposito:2016noz,Luo:2017eub,Eichten:2017ffp,Hyodo:2017hue,Cheung:2017tnt,Wang:2017dtg,Yan:2018gik}, being a meson with constituent quark content $\cquark\cquark\quarkbar\quarkbar^{\prime}$, where the light quarks \quark and $\quark^{\prime}$ could be \uquark, \dquark or \squark.

If the masses of the doubly charmed tetraquarks are below their corresponding open-charm thresholds, they would manifest as 
weakly decaying hadrons with properties including masses, lifetimes and decay modes not too different from the recently observed \Xiccpp baryons~\cite{LHCb-PAPER-2017-018},
and as for the $\Xires_{\cquark\cquark}$ baryons, the most promising searches are in prompt production. 

If the masses of $\mathcal T_{\cquark\cquark}$ states are instead above the open-charm threshold and their widths are broad, it will be very challenging to observe these states via prompt production. 
Instead, \Bcp decays to open-charm mesons can offer unique opportunity to test for their existence. 
In Run 5 the \Bc mesons will be copiously produced at the \lhc, because of the large production cross-sections of \bbbar and \ccbar pairs and of the enormous data sample.
Similarly to the amplitude analysis of the $\decay{\Lb}{\jpsi\proton\Km}$ decay, which led to the observation of the $P_\cquark^+$ pentaquark candidates~\cite{LHCb-PAPER-2015-029}, 
studying the angular distributions of the multi-body final states of the \Bc meson
has the potential of indicating new states, \eg $\mathcal T_{\cquark\cquark}$, inaccessible through decays of lighter hadrons.
It also allows for the determination of the spin-parity quantum numbers of any state that is observed. 
A good example is to study the $\mathcal T_{\cquark\cquark}^+$ state in the decay mode $\decay{\Bcp}{\Dsp \Dz \Dzb}$ (Fig.~\ref{fig:feynman}b)
through the decay chain $\decay{\Bc}{\mathcal T_{\cquark\cquark}^+ \Dzb}$ and 
$\decay{\mathcal T_{\cquark\cquark}^+}{\Dsp \Dz}$ as discussed in Ref.~\cite{Esposito:2013fma}.

The decay $\decay{\Bc}{\Dsp \Dz \Dzb}$ has not been observed with the Run 1 data,
and predictions on the branching fractions of \Bcp decays are subject to very large 
uncertainties. Estimates of the integrated luminosity needed to perform a full amplitude analysis
are therefore imprecise, and can only be formulated through considerations of other decay modes such as $\decay{\Bc}{\jpsi \Dsp}$.
The signal yield of $\decay{\Bc}{\jpsi \Dsp (\to \Pphi \pip)}$ decays observed in Run 1 data is $30 \pm 6$~\cite{LHCb-PAPER-2013-010}.
Considering the branching fraction of the decay of the additional charm hadron and the lower efficiency due to the higher track multiplicity,
the estimated number of signal of $\decay{\Bc}{\Dsp \Dz \Dzb}$ decays is $\mathcal O(10^2)$ 
in a future dataset corresponding to an integrated luminosity of $300$\invfb collected with $\mathcal O(100\%)$ trigger efficiency~\cite{LHCb-PAPER-2013-010}.
Since the \Dz and \Dsp mesons are pseudoscalars, the amplitude analysis simplifies, 
 and can provide conclusive results already with few hundreds decays.

Finally, strongly decaying doubly charmed tetraquarks with a narrow decay width as predicted by pure tetraquark models with 
spin-parity quantum numbers of $0^+$, $1^+$ and $2^+$, can also be searched for in prompt production.
The expected yields can be estimated by the associated production of open charm mesons measured with a fraction of the Run 1 data~\cite{LHCb-PAPER-2012-003}. 
With a data sample of $300$\invfb, the yield of \Dp\Dp (\Dp\Dsp) associated production is around $750$k ($150$k), which is a very promising sample in which to search for narrow $\mathcal T_{\cquark\cquark}$ states.

If the coincidence of the $\Pchi_{\cquark 1}(2P)$ charmonium state with
the \Dz \Dstarzb threshold is responsible for the $\PX(3872)$ state,
 there is likely no bottomonium analogue of it, since  the $\Pchi_{\bquark 1}(3P)$
 state was detected well below the $\B\Bbar{}^{*}$ threshold,
and the $\Pchi_{\bquark 1}(4P)$ state is predicted to be too far above it.
However, if molecular forces dominate its dynamics, there could be an isosinglet state just below this threshold
decaying to \omegaz\OneS, where \omegaz could be reconstructed via the decay to \pip \pim \piz.
Unfortunately its prompt production would likely be very small unless
driven by tightly bound tetraquark dynamics.
The improved $\piz$ reconstruction in the \upgradetwo will help for these searches.

The prompt production at \lhc remains the best hope for unambiguously establishing the existence   
of stable, weakly decaying $\bquark\bquark\uquarkbar\dquarkbar$ tetraquark predicted by both lattice QCD and phenomenological models,                                                                             
However the inclusive reconstruction efficiencies for such states are tiny due to the small branching fractions of \B and \D mesons decays to low multiplicity final states (Fig.~\ref{fig:feynman}c).

Recently there have  been several predictions for an exotic state with quark composition $\bquark\bquarkbar\bquark\bquarkbar$~\cite{Heller:1985cb,Berezhnoy:2011xn,Wu:2016vtq,Chen:2016jxd,Karliner:2016zzc,Bai:2016int,Wang:2017jtz,Richard:2017vry,Anwar:2017toa,Vega-Morales:2017pmm,Eichten:2017ual} with a mass below, the $2m_{\Peta_\bquark}$ threshold,
which implies that it can decay to  $\PUpsilon \mumu$. However lattice QCD calculations do not find evidence for such a state in the hadron spectrum~\cite{Hughes:2017xie}. Given the presence of four muons in the final state, \lhcb will have good sensitivity for observing the first exotic state composed of more than two heavy quarks~\cite{LHCb-PAPER-2018-027}.

Motivated by the discovery of the hidden-charm pentaquarks theorists have extended
the respective models for multiplet systems to include beauty quarks. In Ref.~\cite{Wu:2017weo} 
$\PQ\Qbar\quark\quark\quark$ ground states are investigated in an effective Hamiltonian
framework assuming a colour-magnetic interaction
between colour-octet \quark\quark\quark and $\PQ\Qbar$ subsystems. Several resonant states are predicted.
Such beautiful pentaquarks could be searched for in the
$\PUpsilon\proton$, $\PUpsilon\Lz$,  $\Bcpm \proton$ and $\Bcpm \Lz$ mass spectra.
In analogy with the popular $\Sigmares_\cquark \Lcbar$ molecular model,
Refs.~\cite{Yamaguchi:2017zmn} and~\cite{Shen:2017ayv} investigate similar dynamics in the hidden-bottom sector and predict
a large number of exotic resonances. Indeed in the hidden-beauty sector the theory calculations are
found to be even more stable than for the hidden charm, motivating searches for resonances close to the
$\B^*\Sigmares_\bquark, \B\Sigmares_\bquark^*, \B^*\Sigmares_\bquark^*$ and $\B \Lz^{*0}_\bquark, \B^*\Lb$ thresholds.

Another possibility is the existence of pentaquarks with open beauty and quark contents such as
\bquarkbar\dquark\uquark\uquark\dquark, \bquark\uquarkbar\uquark\dquark\dquark,
\bquark\dquarkbar\uquark\uquark\dquark and \bquarkbar\squark\uquark\uquark\dquark~\cite{Stewart:2004pd, Oh:1994np}.
If those states lie below the respective baryon-meson threshold containing beauty, then
they could be stable against strong decay and would predominantly decay through a weak
\decay{\bquark}{\cquark\cquarkbar\squark} transition. A search using a data set corresponding to $3\invfb$
in four decay channels $\jpsi\proton\hadron^+\hadron^-$ (\hadron = \kaon, \pion)  has been performed~\cite{LHCb-PAPER-2017-043}.
No signals have been found and $90\%$ confidence limits have been put on the production cross section times branching fraction
relative to the \Lb in the \jpsi\proton\Km mode. The obtained limits are of the order of $10^{-3}$, which
does not yet rule out the estimates for the production of such an object provided in Ref.~\cite{Stewart:2004pd}.
Similar searches in channels with open charm hadrons in the final state again lead
to large multiplicities and the respective small reconstruction efficiencies but could
profit from favoured branching fractions. Investigations of a large number of channels
will maximise sensitivity for weakly decaying exotic hadrons.

It has  also been proposed to search for  exotic \Omegab states~\cite{Liang:2017ejq}
in analogy to the recently discovered excited \Omegac states~\cite{LHCb-PAPER-2017-002}.
Such open-beauty exotic states could be searched for in decays to \Xib\kaon final state.

%\subsubsection{Exotic spectroscopy with $B_c$ decays}
%CHECK IF COVERED IN OTHER SUBSECTIONS!

\subsubsection{Study of doubly-heavy baryons}

  The discovery of the \Xiccpp baryon has opened an exciting new line of research that
  \lhcb is avidly pursuing.
  Measurements of the lifetime and relative production cross-section of \Xiccpp,
  searches for additional decay modes, and searches for its isospin partner
  \Xiccp and their strange counterpart \Omegacc are underway.

  A signal yield of  $313 \pm 33$ \decay{\Xiccpp}{\Lc\Km\pip\pip} decays
  was observed in $1.7\invfb$ of Run 2 data~\cite{LHCb-PAPER-2017-018}.
  Improvements in the trigger for the upgraded \lhcb detector are projected to
  increase selection efficiencies by a factor two for most charm decays, with
  decays to high-multiplicity final states, such as those from the cascade
  decays of doubly charmed baryons, potentially benefiting much
  more~\cite{LHCb-TDR-012,LHCb-PAPER-2012-031}.

  Thus the  Run 5 sample  will contain more than 90\,000 decays of this mode.
  The branching fractions for \decay{\Xiccpp}{\Lc\Km\pip\pip} has been
  theoretically estimated to be up to 10\% making it one of the most frequent
  nonleptonic decay modes, but several other lower multiplicity modes with
  predicted branching fractions of $\mathcal{O}(1\%)$ will yield samples of 
  comparable size~\cite{LHCb-PAPER-2018-026, Wang:2017mqp,Gutsche:2017hux,Sharma:2017txj}.

  The efficiency with which \lhcb can disentangle weak decays of doubly charmed
  baryons from prompt backgrounds depends on the lifetime of the
  baryon~\cite{LHCb-PAPER-2013-049}.
  Although the predicted lifetimes for the \Xiccp, \Xiccpp, and \Omegacc baryons span
  almost an order of magnitude, the relative lifetimes of \Xiccp and \Omegacc
  are expected to be approximately $1/3$ that of the
  \Xiccpp baryon~\cite{Kiselev:2001fw,Karliner:2014gca,Fleck:1989mb,Guberina:1999mx,Kiselev:1998sy,Chang:2007xa,Berezhnoy:2016wix,LHCb-PAPER-2018-019}.
  Assuming a relative efficiency of 0.25 with respect to \Xiccpp due to the
  shorter lifetimes and an additional production suppression of
  $\sigma(\Omegacc)/\sigma(\Xiccpp) \sim 0.2$ for
  \Omegacc~\cite{Kiselev:2001fw}, \lhcb will have Run 5 yields of around 25\,000 for \Xiccp and 4\,500 for \Omegacc in each of several decay modes.

  \lhcb will be the primary experiment for studies of the physics of doubly
  charmed baryons for the foreseeable future, and its potential will not be
  exhausted by the end of Run 5.
  With the data collected in Run 2, \lhcb should observe all three weakly decaying
  doubly charmed baryons and characterise their physical properties.
  Run 5 will supply precision measurements of doubly differential
  cross-sections that will that will provide insight into production
  mechanisms of doubly heavy baryons. In addition 
  Run 5 will allow the spectroscopy of excited states and bring studies of the
  rich decay structure of doubly charmed hadrons into the domain of precision
  physics.

The production cross section of the $\Xires_{\bquark\cquark}$ baryons within the \lhcb acceptance is expected to be about 77\nb~\cite{Zhang:2011hi}.
This value is about 1/6 of the expected production cross-section of a \Bcp meson~\cite{Chang:2003cr,Gao:2010zzc}. It should be noted that the relative \Lb production rate is \pt-dependent. In the typical \pt range in the 
\lhcb acceptance, a ratio of production rates, 
$\sigma(\decay{\proton\proton}{\Lb \PX})/\sigma(\decay{\proton\proton}{\Bzb X})\sim0.5$~\cite{LHCb-PAPER-2011-018,LHCb-PAPER-2014-004} is measured. It is therefore
conceivable that the $\Xires_{bc}$ production rates are also larger than predicted by the above calculations.

To observe and study $\Xires_{\bquark\cquark}$ and $\Omegares_{\bquark\cquark}^0$ baryons is quite challenging due to the low production rate, the small product of branching fractions
and the selection efficiency for reconstructing all of the final-state particles. To collect a large sample of $\Xires_{\bquark\cquark}$
baryons will require the higher integrated luminosity and detector enhancements planned for in the \upgradetwo.  
  Using the notation that $\PX_\cquark$ is a charmed baryon
containing a single charm quark, some of the most promising decay modes to detect $\Xires_{\bquark\cquark}$ and $\Omegares_{\bquark\cquark}^0$ baryons are:
\begin{itemize}
  \item $\jpsi \PX_\cquark$ modes: \jpsi\Xicp, \jpsi\Xicz, \jpsi\Lc, \jpsi\Lc\Km; 
  \item $\Xires_{\cquark\cquark}$ modes: $\Xires_{\cquark\cquark}\pim$; 
  \item Doubly charmed modes: \Dz \Lc, \Dz \Lc \pim, \Dz\Dz \proton; 
  \item Penguin topologies: \Lc\Km, \Xicp\pim;
  \item $\Xires_\bquark$ , \Bd or \Lb modes: \Xib\pip, \Lb\pip, \Bz \proton, using fully reconstructed or semileptonic \Bz, \Lb or \Xib decays~\cite{LHCb-PAPER-2018-032};
  \item $W$-exchange between \bquark-\cquark quarks, which is not helicity suppressed. This can give rise to final state with just one charmed particle, $\eg$ \Lc\Km.
\end{itemize}

To put this in context, the \lhcb collaboration observed $30\pm6$ $\decay{\Bcp}{\jpsi \Dsp (\to\Pphi\pip)}$ decays with a 3\invfb data at 7 and 8\tev~\cite{LHCb-PAPER-2013-010}. 
With looser selections about $100$ signal decays can be obtained with reasonably good signal-to-background ratio.
The $\decay{\Xires^+_{\bquark\cquark}}{\jpsi\Xicp}$ decay is kinematically very similar.
Assuming
$\frac{f_{\Xires_{\bquark\cquark}^+}}{f_{\Bcp}}\sim0.2$,
$\frac{\BR(\decay{\Xires_{\bquark\cquark}^+}{\jpsi\Xicp})}{\BR(\decay{\Bcp}{\jpsi\Dsp})}\sim~1$, 
$\frac{\BR(\decay{\Xicp}{p\Km\pip})}{\BR(\decay{\Dsp}{\Kp\Km\pip})}\sim0.1$, and
$\epsilon_{\Xires_{\bquark\cquark}^+}/\epsilon_{\Bcp}\sim~0.5$,
a signal yield of about 600 $\decay{\Xires_{\bquark\cquark}^+}{\jpsi\Xicp}$ signal decays
in expected in Run 5, albeit with sizeable uncertainties.
Other modes could also provide sizeable signal samples.
It is likely \lhcb will observe the $\Xires_{\bquark\cquark}$ baryons in Run 3/4, and further probe the spectrum of other doubly heavy baryons with the large samples accessible in the proposed \upgradetwo. 

\subsubsection{Precision measurements of quarkonia}

The correct interpretation of the experimental polarisation results
for $S$-wave quarkonia requires a rigorous analysis of the feed-down contributions from
higher excited states ~\cite{LHCb-PAPER-2011-030,LHCb-PAPER-2014-031}.
The direct measurement of the polarisation for $\Pchi_\cquark$ and $\Pchi_\bquark$ states is necessary to decrease this 
model dependence. Since $P$-wave states are practically free from the feed-down from higher excited states,
any $\Pchi_\cquark$ and $\Pchi_{\bquark}$ polarisation measurements could be interpreted in a robust manner
 without additional model assumptions.

Recent discovery of the $\decay{\Pchi_{\cquark1,2}}{\jpsi\mumu}$ decays~\cite{LHCb-PAPER-2017-036} opens the possibility
to perform a detailed study of $\Pchi_{\cquark}$ production, allowing almost background-free measurements even for
very low transverse momentum of $\Pchi_\cquark$~candidates. 
Due to the excellent mass resolution, the vector state $\Pchi_{\cquark1}$  and the tensor state $\Pchi_{\cquark2}$ are well separated,
eliminating the possible systematic uncertainty  caused by the large overlap of
these states in the $\decay{\Pchi_{\cquark}(\Pchi_\bquark)}{\jpsi(\PUpsilon)\g}$ decay\cite{
  LHCb-PAPER-2011-019,
  LHCb-PAPER-2011-030,
  LHCb-PAPER-2012-015,
  LHCb-PAPER-2013-028,
  LHCb-PAPER-2014-031,
  LHCb-PAPER-2014-040}.

An integrated luminosity of 300\invfb will allow the high-multipole contributions 
to the  $\decay{\Pchi_{\cquark}}{\jpsi\mumu}$ amplitude to be probed, namely
the magnet-dipole contribution for $\Pchi_{\cquark1}$~decays and
the magnet-dipole and electrical-octupole contributions for $\Pchi_{\cquark2}$ decays. 

Use of Run 5 dataset is also necessary  to measure the $(\pt,y)$ dependence of $\Pchi_\cquark$ polarisation parameters. 
 In addition the effect of the form factor in
the~decays $\decay{\Pchi_{\cquark}}{\jpsi\mumu}$~\cite{Faessler:1999de,Luchinsky:2017pby}
could be probed with the precision of several percent from the shape of the $m(\mumu)$ spectra.

Studies of the double quarkonia production allows independent tests for the  quarkonia production mechanism,
and in particular for the role of the colour-octet. So far the \lhcb collaboration has  analysed double \jpsi production at 7\tev and 13\tev data with relatively small
datasets~\cite{LHCb-PAPER-2011-013,LHCb-PAPER-2016-057}.
Using $280$\invpb of data collected at $\sqs=13\tev$,
$(1.05\pm0.05)\times10^3$ signal \jpsi\jpsi events are observed.
However, even with the larger sample of \jpsi\jpsi events now available, 
it is not possible to distinguish between 
the different theory descriptions of the single-parton scattering (SPS) mechanism~\cite{Sun:2014gca,Likhoded:2016zmk, Lansberg:2013qka, Lansberg:2014swa,Lansberg:2015lva,Shao:2012iz,Shao:2015vga, Baranov:2011zz,Baranov:1993qv} nor to separate the
  contributions from the SPS and double parton  scattering (DPS) mechanisms~\cite{Bansal:2014paa,Belyaev:2017sws}. 
Though the larger samples collected during  \upgradeone will allow some progress on these questions, the measurement of the correlation of \jpsi polarisation parameters will only be possible with the \upgradetwo data set.

In addition, while it is likely that  $\PUpsilon\PUpsilon$ and $\jpsi\PUpsilon$ production will be observed in the near future (assuming the dominance of the DPS mechanism),
the determination of the relative SPS and DPS contributions, as well as the discrimination between different SPS theory models, will require precision measurements only possible with  Run 5 data.


\subsection{Combined theory/experiment perspective}
%Including how far ATLAS/CMS can contribute to finding new states, confirming the %pentaquark etc. observations, and studying their properties (?)
Hadronic spectroscopy has been a very challenging field historically, from both the
experimental and theoretical perspectives. While \belletwo will make a major impact 
in the study of excited mesons and lighter exotic hadrons, only the HL-LHC experiments 
have the potential to comprehensively study the full range of possible excited
hadronic states. Such a comprehensive understanding and characterisation of quark
structures would not only in some sense complete the taxonomy of SM particles, but
would also sharpen our understanding of QCD in a particularly difficult energy
regime. For this reason, while not directly probing BSM physics, spectroscopy
measurements will continue to make essential contributions to the interpretation
of any observed BSM effects in the flavour sector. In addition spectroscopy can provide tools to probe BSM effects by using a new observed state as a tagger\cite{Aaij:2017inn}, \cite{ Aaij:2016rdg}.

While the \upgradetwo of LHCb will
by design offer unparalleled capabilities in this area, the legacy of HL-LHC for
spectroscopy will be much stronger if ATLAS and CMS are also able to continue
pursuing this work in the HL-LHC period. The planned hardware tracking
triggers and much higher data rates sent to their software triggers are promising
in this respect.

\end{document}
