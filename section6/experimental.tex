\subsection{Experimental prospects}

The LHCb \upgradetwo\ detector will have a large impact on sensitivity in searches for heavy states. Aside from the much larger integrated luminosity, many of the 
detector improvements planned for LHCb \upgradetwo may have significant benefits for spectroscopy studies. For example, the potential removal of the VELO RF foils, together with the improved particle identification provide by the TORCH, will enhance the reconstruction efficiency for multibody \B decays, such as \decay{\Bcp}{\Dsp \Dz \Dzb}; the selection of short-lived particles ($\eg$ \Bc, \Xicc, \Omegacc, $\Xires_{\bquark\cquark}$, \etc) will also benefit from an improved vertex resolution; the Magnet Side Stations will help in studying dipion transitions such as \decay{\PX(3872)}{\chicone \pip \pim} or \decay{\B_\cquark^{**+}}{\Bc \pip \pim}; improved \piz and \etaz mass resolutions will increase the sensitivity in searching for the $C$-odd and charged partners of the $\PX(3872)$ meson by \decay{\PX(3872)^{C\rm{-odd}}}{\jpsi \etaz} and \decay{\PX(3872)^\pm}{\jpsi \piz \pipm}. A summary of the expected yields in certain important modes, and a comparison with \belletwo, is given in Table~\ref{tab:sec9_summary}.
In the following sections we will give some more details on the prospects for specific studies and analyses.

\begin{table}[tb]
\centering
\caption{\label{tab:yields}Expected data samples at \lhcb \upgradetwo and \belletwo for key decay modes for the spectroscopy of heavy flavoured hadrons. The expected yields at \belletwo are estimated by assuming similar efficiencies as at \belle.}
\begin{tabular}{l|ccc|c}
\hline
                      & \multicolumn{3}{c|}{\lhcb} & \belletwo\\
%\hline
Decay mode & 23\invfb & 50\invfb & 300\invfb & 50\invab\\
\hline
% See UpgradeII_vs_BelleII_yields.txt for details on these numbers
\decay{\Bp}{\PX(3872) (\to \jpsi \pip \pim)\Kp} &14k &30k&180k&11k\\
\decay{\Bp}{\PX(3872) (\to \psitwos \g)\Kp}     &500 &1k& 7k & 4k \\
\decay{\Bz}{\psitwos \Km \pip}                  &340k &700k&4M& 140k\\
\decay{\Bcp}{\Dsp \Dz \Dzb}                     & 10 &20& 100 & ---\\
\decay{\Lb}{\jpsi \proton \Km}                  &340k &700k&4M& ---\\
\decay{\Xibm}{\jpsi \Lz \Km}                    &4k &10k&55k& ---\\
\decay{\Xiccpp}{\Lc\Km\pip\pip}                 &7k &15k&90k & $<$6k\\
\decay{\Xires_{\bquark\cquark}^+}{\jpsi \Xires_\cquark^+} & 50 &100& 600 &---\\
\hline
\end{tabular}
\label{tab:sec9_summary}
\end{table}

\subsubsection{Taxonomy of tetraquarks and pentaquarks}

To advance our understanding of the $\PX(3872)$ state, it will be very important to learn even more about its decay pattern.
In particular, if it really has a strong $\Pchi_{\cquark 1}(2P)$ component, it should have \pip\pim transitions to the $\Pchi_{\cquark1}(1P)$ state.
Unfortunately at \lhcb, the reconstruction efficiency for the dominant $\Pchi_{\cquark 1}(1P)$ decay  to \g\jpsi decay is low, making this prediction hard to test.
 The very large data set of \upgradetwo  will allow this problem to be overcome. 
The very large data set of the \upgradetwo will be essential
in detecting or refuting such transitions.
Studies of the $\PX(3872)$ lineshape by a simultaneous fit to all detected channels are important for pinning down the location
of its resonant pole and determining its natural width; both are very important inputs in helping with the understanding of the state.
Therefore a very large data set will be essential for the statistical precision of such studies and reconstruction of decays to \Dz\Dstarzb, which are relevant given the proximity of the $\PX(3872)$ mass to the \Dz\Dstarzb threshold.

Searching for prompt production of any known exotic hadron candidates at \lhc remains an important task,
since its detection would signify compact component, either conventional quarkonium,  or a tightly bound tetraquark or pentaquark.
To date, the $\PX(3872)$ is the only exotic hadron candidate with $\PQ\Qbar$ content that has been confirmed to be produced promptly. This anecdotally speaks against compact interpretations for the other states.
However, it will be important to quantify the upper limits in negative searches to allow more rigorous phenomenological analysis. 

The prediction of magnetic moments for the light baryon states was a key success of the quark model.
As such these observables provide an additional handle on the structure of baryons and can potentially be used to
distinguish between several of the models that attempt to explain the structure of the observed pentaquarks~\cite{Wang:2016dzu}.

Experimentally the measurement of the magnetic moments is challenging.
Access can be gained
through radiative decays of pentaquark states. The observation of radiative decays
involving exotic baryons would thus provide a new window into the structure of the
pentaquark candidates. Interestingly, since the parity of the two observed states $\PP_\cquark(4380)^+$ and $\PP_\cquark(4450)^+$  are favoured to be 
opposite and their spins to be different by one unit, the radiative transition
between the two states should be allowed and might be observable by performing an amplitude analysis of the
\jpsi\proton{}\g\kaon decay. The proposed improved ECAL is crucial to the
feasibility of such a measurement in a high-luminosity environment.

To get an idea of the typical suppression
of these decays with respect to hadronic decays we can look at the measured branching fractions for
ordinary baryons and for exotic mesons. The $\decay{\PX(3872)}{\jpsi  \g}$ radiative decays are suppressed
by about a factor 50 with respect to decays into open charm and a factor of 5 with respect to the \jpsi \pip\pim decay.
Branching fractions of the order of $1\%$ (\eg $\BF(\decay{\Lz(1520)}{\Lz \g})=0.85 \pm 0.15\% $)
are common for radiative decays of baryons, which again are around a factor $50$ lower than the largest hadronic branching fraction.

\begin{figure}[th]
 \centering
\includegraphics[width=0.6\textwidth]{section6/figures/argand-upgrade-bw-fluc_v2.pdf}
\caption {Argand diagram of the $\PZ(4430)^-$ amplitude (${\rm A}_{\PZ(4430)^-}$) in bins of $m^2_{\psitwos \pim}$ from a fit to the $\decay{\Bz}{\psitwos \Kp \pim}$ decays. The black points are the results based on Run 1 data~\cite{LHCb-PAPER-2014-014} while the blue points correspond to an extrapolation to an integrated luminosity of 300\invfb expected at the \lhcb \upgradetwo. The red curve is the prediction from the Breit-Wigner formula with a resonance mass (width) of 4475 (172)\mev. Units are arbitrary.}
 \label{fig:argand}
\end{figure}

Many puzzling charged exotic meson candidates (\eg $\PZ(4430)^+$) decaying to \jpsi, \psitwos or $\Pchi_{\cquark 1}$ plus a charged pion have been observed in \B decays.
Some of them are broad, and none can be satisfactorily explained by any of the available phenomenological models.
The hidden-charmed mesons, observed in the $\jpsi\Pphi$ decay~\cite{LHCb-PAPER-2016-018,Chatrchyan:2013dma,Aaltonen:2009tz}, also belong to this category.
The determination of their properties, or even claim for their existence, relies on advanced amplitude analyses,
which allow the exotic contributions to be separated from the (typically) dominant non-exotic components.
Further investigation of these $\PQ\Qbar\quark\quarkbar$ structures
will require much larger data samples and refinement of theoretical approaches to parametrisations of hadronic amplitudes.
 Similar comments apply to improvements in the determination of the properties of the pentaquark candidates
$\PP_\cquark(4380)^+$ and $\PP_\cquark(4450)^+$ and to the spectroscopy of excited \Lz baryons in $\decay{\Lb}{\jpsi \proton \kaon}$ decays. The large data set collected during the \lhcb \upgradetwo would allow to test further the resonant character of the $\PP_\cquark(4380)^+$, $\PP_\cquark(4450)^+$ and $\PZ(4430)^+$ states (Fig.~\ref{fig:argand}), while improvements in calorimetry would help in searching for new decay modes (\eg $\decay{\PP_\cquark^+}{\Pchi_{\cquark 1,2} (\to \jpsi \g) \proton}$) by amplitude analyses of $\decay{\Lb}{\Pchi_{\cquark 1,2}\proton\Km}$ decays~\cite{LHCb-PAPER-2017-011,Guo:2015umn}.

\subsubsection{Searches for further tetra- and pentaquarks}

Though the true nature of the $\PX(3872)$ meson is still unclear, both the molecular~\cite{Nieves:2012tt}
and tetraquark~\cite{Maiani:2004vq} models predict that a $C$-odd partner ($\PX(3872)^{C\rm{-odd}}$)
and charged partners ($\PX(3872)^\pm$) may exist and decay to $\jpsi \etaz / \g \Pchi_{\cquark J}$ and \jpsi \piz \pipm respectively.

Similarly the existence of the $P_\cquark(4380)^+$ and $P_\cquark(4450)^+$ pentaquark states raises the question
of whether there is a large pentaquark multiplet. The observed states have an isospin
3-component of $I_3=+\frac{1}{2}$. They could be part of an isospin doublet with $I=\frac{1}{2}$ or a quadruplet with $I=\frac{3}{2}$.
In both cases there should be a neutral $I_3=-\frac{1}{2}$ state decaying into \jpsi \neutron. However this final state does
not lend itself well to observation. Instead the search for the neutral pentaquark candidate
 can be carried out using decays into pairs of open charm in particular in the process
\decay{\Lb}{\Lc\Dm\Kstarzb}, where the neutral pentaquark states would appear as resonances
in the \Lc\Dm subsystem (Fig.~\ref{fig:feynman}a). Such decays can be very well reconstructed but the total reconstruction
efficiency suffers from the large number of tracks and the small branching fractions of $\Lc$ and $\Dm$ reconstructable final states; the total reconstruction
efficiency is about a factor 50
smaller than the efficiency for the \decay{\Lb}{ \jpsi \proton \Km} channel.
In the case of the existence of an isospin quadruplet, there is the interesting possibility to find doubly charged
pentaquarks decaying into $\Sigmares_\cquark^{++} \Dzb$. Channels such as these
require very large data sets to offset the low efficiency. The magnet side stations will also improve the reconstruction efficiency of such decay modes with several tracks in the final states. 

The relative coupling of the pentaquark states to their decays into the double open-charm channels will
depend on their internal structure and the spin structure of the respective decay. For that reason
it is important to study decays involving \Dstarp resonances as well (\eg \decay{P_\cquark^+}{\Dstarm \Sigmares_\cquark^{++}})
 to investigate the internal structure of pentaquarks~\cite{Lin:2017mtz}. Since
these decays require the reconstruction of slow pions from the \Dstarp decays, the proposed tracking stations inside the magnet, enhancing the acceptance for low-momentum particles, will be highly beneficial for this study.

\begin{figure}[th]
 \centering
\includegraphics[width=0.32\textwidth]{section6/figures/Lb2PcKst_Pc2LcDm.pdf}
\includegraphics[width=0.32\textwidth]{section6/figures/Bc2D0D0bDs.pdf}
\includegraphics[width=0.32\textwidth]{section6/figures/Xbbud2DBpi.pdf}
\caption {Feynman diagrams.}
 \label{fig:feynman}
\end{figure}

Invoking $\rm{SU}_3$ flavour symmetry,
 one would expect the existence of pentaquarks with strangeness, which would
decay into channels like \jpsi \Lz or \Lc\Dsm. To explore the potential of
the former case the decay \decay{\Xires_\bquark^-}{ \jpsi \Lz\Km} has been studied using Run~1 data.
About 300 signal decays have been observed~\cite{LHCB-PAPER-2016-053}.
Complementary information can be also achieved by a study of the \decay{\Lb}{ \jpsi \Lz \Pphi} decays.
An increase of the available integrated luminosity by a factor of 100 would allow detailed amplitude analyses to be performed for 
 these final states, with a similar sensitivity as was the case for the pentaquark discovery channel.

The history of $\PX(3872)$ studies illustrates well the difficulty of distinguishing between exotic and conventional explanations for a hidden-charm state.
 Therefore it is appealing to search for  states with  an uncontroversial exotic signature.
A good candidate in this category would be a $\mathcal T_{\cquark\cquark}$ doubly charmed tetraquark~\cite{Moinester:1995fk,DelFabbro:2004ta,Carames:2011zz,Yuqi:2011gm,Hyodo:2012pm,Esposito:2013fma,Ikeda:2013vwa,Guerrieri:2014nxa,Maciula:2016wci,Richard:2016eis,Esposito:2016noz,Luo:2017eub,Eichten:2017ffp,Hyodo:2017hue,Cheung:2017tnt,Wang:2017dtg,Yan:2018gik}, being a meson with constituent quark content $\cquark\cquark\quarkbar\quarkbar^{\prime}$, where the light quarks \quark and $\quark^{\prime}$ could be \uquark, \dquark or \squark.

If the masses of the doubly charmed tetraquarks are below their corresponding open-charm thresholds, they would manifest as 
weakly decaying hadrons with properties including masses, lifetimes and decay modes not too different from the recently observed \Xiccpp baryons~\cite{LHCb-PAPER-2017-018},
and as for the $\Xires_{\cquark\cquark}$ baryons, the most promising searches are in prompt production. 

If the masses of $\mathcal T_{\cquark\cquark}$ states are instead above the open-charm threshold and their widths are broad, it will be very challenging to observe these states via prompt production. 
Instead, \Bcp decays to open-charm mesons can offer unique opportunity to test for their existence. 
In Run 5 the \Bc mesons will be copiously produced at the \lhc, because of the large production cross-sections of \bbbar and \ccbar pairs and of the enormous data sample.
Similarly to the amplitude analysis of the $\decay{\Lb}{\jpsi\proton\Km}$ decay, which led to the observation of the $P_\cquark^+$ pentaquark candidates~\cite{LHCb-PAPER-2015-029}, 
studying the angular distributions of the multi-body final states of the \Bc meson
has the potential of indicating new states, \eg $\mathcal T_{\cquark\cquark}$, inaccessible through decays of lighter hadrons.
It also allows for the determination of the spin-parity quantum numbers of any state that is observed. 
A good example is to study the $\mathcal T_{\cquark\cquark}^+$ state in the decay mode $\decay{\Bcp}{\Dsp \Dz \Dzb}$ (Fig.~\ref{fig:feynman}b)
through the decay chain $\decay{\Bc}{\mathcal T_{\cquark\cquark}^+ \Dzb}$ and 
$\decay{\mathcal T_{\cquark\cquark}^+}{\Dsp \Dz}$ as discussed in Ref.~\cite{Esposito:2013fma}.

The decay $\decay{\Bc}{\Dsp \Dz \Dzb}$ has not been observed with the Run 1 data,
and predictions on the branching fractions of \Bcp decays are subject to very large 
uncertainties. Estimates of the integrated luminosity needed to perform a full amplitude analysis
are therefore imprecise, and can only be formulated through considerations of other decay modes such as $\decay{\Bc}{\jpsi \Dsp}$.
The signal yield of $\decay{\Bc}{\jpsi \Dsp (\to \Pphi \pip)}$ decays observed in Run 1 data is $30 \pm 6$~\cite{LHCb-PAPER-2013-010}.
Considering the branching fraction of the decay of the additional charm hadron and the lower efficiency due to the higher track multiplicity,
the estimated number of signal of $\decay{\Bc}{\Dsp \Dz \Dzb}$ decays is $\mathcal O(10^2)$ 
in a future dataset corresponding to an integrated luminosity of $300$\invfb collected with $\mathcal O(100\%)$ trigger efficiency~\cite{LHCb-PAPER-2013-010}.
Since the \Dz and \Dsp mesons are pseudoscalars, the amplitude analysis simplifies, 
 and can provide conclusive results already with few hundreds decays.

Finally, strongly decaying doubly charmed tetraquarks with a narrow decay width as predicted by pure tetraquark models with 
spin-parity quantum numbers of $0^+$, $1^+$ and $2^+$, can also be searched for in prompt production.
The expected yields can be estimated by the associated production of open charm mesons measured with a fraction of the Run 1 data~\cite{LHCb-PAPER-2012-003}. 
With a data sample of $300$\invfb, the yield of \Dp\Dp (\Dp\Dsp) associated production is around $750$k ($150$k), which is a very promising sample in which to search for narrow $\mathcal T_{\cquark\cquark}$ states.

If the coincidence of the $\Pchi_{\cquark 1}(2P)$ charmonium state with
the \Dz \Dstarzb threshold is responsible for the $\PX(3872)$ state,
 there is likely no bottomonium analogue of it, since  the $\Pchi_{\bquark 1}(3P)$
 state was detected well below the $\B\Bbar{}^{*}$ threshold,
and the $\Pchi_{\bquark 1}(4P)$ state is predicted to be too far above it.
However, if molecular forces dominate its dynamics, there could be an isosinglet state just below this threshold
decaying to \omegaz\OneS, where \omegaz could be reconstructed via the decay to \pip \pim \piz.
Unfortunately its prompt production would likely be very small unless
driven by tightly bound tetraquark dynamics.
The improved $\piz$ reconstruction in the \upgradetwo will help for these searches.

The prompt production at \lhc remains the best hope for unambiguously establishing the existence   
of stable, weakly decaying $\bquark\bquark\uquarkbar\dquarkbar$ tetraquark predicted by both lattice QCD and phenomenological models,                                                                             
However the inclusive reconstruction efficiencies for such states are tiny due to the small branching fractions of \B and \D mesons decays to low multiplicity final states (Fig.~\ref{fig:feynman}c).

Recently there have  been several predictions for an exotic state with quark composition $\bquark\bquarkbar\bquark\bquarkbar$~\cite{Heller:1985cb,Berezhnoy:2011xn,Wu:2016vtq,Chen:2016jxd,Karliner:2016zzc,Bai:2016int,Wang:2017jtz,Richard:2017vry,Anwar:2017toa,Vega-Morales:2017pmm,Eichten:2017ual} with a mass below, the $2m_{\Peta_\bquark}$ threshold,
which implies that it can decay to  $\PUpsilon \mumu$. However lattice QCD calculations do not find evidence for such a state in the hadron spectrum~\cite{Hughes:2017xie}. Given the presence of four muons in the final state, \lhcb will have good sensitivity for observing the first exotic state composed of more than two heavy quarks~\cite{LHCb-PAPER-2018-027}.

Motivated by the discovery of the hidden-charm pentaquarks theorists have extended
the respective models for multiplet systems to include beauty quarks. In Ref.~\cite{Wu:2017weo} 
$\PQ\Qbar\quark\quark\quark$ ground states are investigated in an effective Hamiltonian
framework assuming a colour-magnetic interaction
between colour-octet \quark\quark\quark and $\PQ\Qbar$ subsystems. Several resonant states are predicted.
Such beautiful pentaquarks could be searched for in the
$\PUpsilon\proton$, $\PUpsilon\Lz$,  $\Bcpm \proton$ and $\Bcpm \Lz$ mass spectra.
In analogy with the popular $\Sigmares_\cquark \Lcbar$ molecular model,
Refs.~\cite{Yamaguchi:2017zmn} and~\cite{Shen:2017ayv} investigate similar dynamics in the hidden-bottom sector and predict
a large number of exotic resonances. Indeed in the hidden-beauty sector the theory calculations are
found to be even more stable than for the hidden charm, motivating searches for resonances close to the
$\B^*\Sigmares_\bquark, \B\Sigmares_\bquark^*, \B^*\Sigmares_\bquark^*$ and $\B \Lz^{*0}_\bquark, \B^*\Lb$ thresholds.

Another possibility is the existence of pentaquarks with open beauty and quark contents such as
\bquarkbar\dquark\uquark\uquark\dquark, \bquark\uquarkbar\uquark\dquark\dquark,
\bquark\dquarkbar\uquark\uquark\dquark and \bquarkbar\squark\uquark\uquark\dquark~\cite{Stewart:2004pd, Oh:1994np}.
If those states lie below the respective baryon-meson threshold containing beauty, then
they could be stable against strong decay and would predominantly decay through a weak
\decay{\bquark}{\cquark\cquarkbar\squark} transition. A search using a data set corresponding to $3\invfb$
in four decay channels $\jpsi\proton\hadron^+\hadron^-$ (\hadron = \kaon, \pion)  has been performed~\cite{LHCb-PAPER-2017-043}.
No signals have been found and $90\%$ confidence limits have been put on the production cross section times branching fraction
relative to the \Lb in the \jpsi\proton\Km mode. The obtained limits are of the order of $10^{-3}$, which
does not yet rule out the estimates for the production of such an object provided in Ref.~\cite{Stewart:2004pd}.
Similar searches in channels with open charm hadrons in the final state again lead
to large multiplicities and the respective small reconstruction efficiencies but could
profit from favoured branching fractions. Investigations of a large number of channels
will maximise sensitivity for weakly decaying exotic hadrons.

It has  also been proposed to search for  exotic \Omegab states~\cite{Liang:2017ejq}
in analogy to the recently discovered excited \Omegac states~\cite{LHCb-PAPER-2017-002}.
Such open-beauty exotic states could be searched for in decays to \Xib\kaon final state.

%\subsubsection{Exotic spectroscopy with $B_c$ decays}
%CHECK IF COVERED IN OTHER SUBSECTIONS!

\subsubsection{Study of doubly-heavy baryons}

  The discovery of the \Xiccpp baryon has opened an exciting new line of research that
  \lhcb is avidly pursuing.
  Measurements of the lifetime and relative production cross-section of \Xiccpp,
  searches for additional decay modes, and searches for its isospin partner
  \Xiccp and their strange counterpart \Omegacc are underway.

  A signal yield of  $313 \pm 33$ \decay{\Xiccpp}{\Lc\Km\pip\pip} decays
  was observed in $1.7\invfb$ of Run 2 data~\cite{LHCb-PAPER-2017-018}.
  Improvements in the trigger for the upgraded \lhcb detector are projected to
  increase selection efficiencies by a factor two for most charm decays, with
  decays to high-multiplicity final states, such as those from the cascade
  decays of doubly charmed baryons, potentially benefiting much
  more~\cite{LHCb-TDR-012,LHCb-PAPER-2012-031}.

  Thus the  Run 5 sample  will contain more than 90\,000 decays of this mode.
  The branching fractions for \decay{\Xiccpp}{\Lc\Km\pip\pip} has been
  theoretically estimated to be up to 10\% making it one of the most frequent
  nonleptonic decay modes, but several other lower multiplicity modes with
  predicted branching fractions of $\mathcal{O}(1\%)$ will yield samples of 
  comparable size~\cite{LHCb-PAPER-2018-026, Wang:2017mqp,Gutsche:2017hux,Sharma:2017txj}.

  The efficiency with which \lhcb can disentangle weak decays of doubly charmed
  baryons from prompt backgrounds depends on the lifetime of the
  baryon~\cite{LHCb-PAPER-2013-049}.
  Although the predicted lifetimes for the \Xiccp, \Xiccpp, and \Omegacc baryons span
  almost an order of magnitude, the relative lifetimes of \Xiccp and \Omegacc
  are expected to be approximately $1/3$ that of the
  \Xiccpp baryon~\cite{Kiselev:2001fw,Karliner:2014gca,Fleck:1989mb,Guberina:1999mx,Kiselev:1998sy,Chang:2007xa,Berezhnoy:2016wix,LHCb-PAPER-2018-019}.
  Assuming a relative efficiency of 0.25 with respect to \Xiccpp due to the
  shorter lifetimes and an additional production suppression of
  $\sigma(\Omegacc)/\sigma(\Xiccpp) \sim 0.2$ for
  \Omegacc~\cite{Kiselev:2001fw}, \lhcb will have Run 5 yields of around 25\,000 for \Xiccp and 4\,500 for \Omegacc in each of several decay modes.

  \lhcb will be the primary experiment for studies of the physics of doubly
  charmed baryons for the foreseeable future, and its potential will not be
  exhausted by the end of Run 5.
  With the data collected in Run 2, \lhcb should observe all three weakly decaying
  doubly charmed baryons and characterise their physical properties.
  Run 5 will supply precision measurements of doubly differential
  cross-sections that will that will provide insight into production
  mechanisms of doubly heavy baryons. In addition 
  Run 5 will allow the spectroscopy of excited states and bring studies of the
  rich decay structure of doubly charmed hadrons into the domain of precision
  physics.

The production cross section of the $\Xires_{\bquark\cquark}$ baryons within the \lhcb acceptance is expected to be about 77\nb~\cite{Zhang:2011hi}.
This value is about 1/6 of the expected production cross-section of a \Bcp meson~\cite{Chang:2003cr,Gao:2010zzc}. It should be noted that the relative \Lb production rate is \pt-dependent. In the typical \pt range in the 
\lhcb acceptance, a ratio of production rates, 
$\sigma(\decay{\proton\proton}{\Lb \PX})/\sigma(\decay{\proton\proton}{\Bzb X})\sim0.5$~\cite{LHCb-PAPER-2011-018,LHCb-PAPER-2014-004} is measured. It is therefore
conceivable that the $\Xires_{bc}$ production rates are also larger than predicted by the above calculations.

To observe and study $\Xires_{\bquark\cquark}$ and $\Omegares_{\bquark\cquark}^0$ baryons is quite challenging due to the low production rate, the small product of branching fractions
and the selection efficiency for reconstructing all of the final-state particles. To collect a large sample of $\Xires_{\bquark\cquark}$
baryons will require the higher integrated luminosity and detector enhancements planned for in the \upgradetwo.  
  Using the notation that $\PX_\cquark$ is a charmed baryon
containing a single charm quark, some of the most promising decay modes to detect $\Xires_{\bquark\cquark}$ and $\Omegares_{\bquark\cquark}^0$ baryons are:
\begin{itemize}
  \item $\jpsi \PX_\cquark$ modes: \jpsi\Xicp, \jpsi\Xicz, \jpsi\Lc, \jpsi\Lc\Km; 
  \item $\Xires_{\cquark\cquark}$ modes: $\Xires_{\cquark\cquark}\pim$; 
  \item Doubly charmed modes: \Dz \Lc, \Dz \Lc \pim, \Dz\Dz \proton; 
  \item Penguin topologies: \Lc\Km, \Xicp\pim;
  \item $\Xires_\bquark$ , \Bd or \Lb modes: \Xib\pip, \Lb\pip, \Bz \proton, using fully reconstructed or semileptonic \Bz, \Lb or \Xib decays~\cite{LHCb-PAPER-2018-032};
  \item $W$-exchange between \bquark-\cquark quarks, which is not helicity suppressed. This can give rise to final state with just one charmed particle, $\eg$ \Lc\Km.
\end{itemize}

To put this in context, the \lhcb collaboration observed $30\pm6$ $\decay{\Bcp}{\jpsi \Dsp (\to\Pphi\pip)}$ decays with a 3\invfb data at 7 and 8\tev~\cite{LHCb-PAPER-2013-010}. 
With looser selections about $100$ signal decays can be obtained with reasonably good signal-to-background ratio.
The $\decay{\Xires^+_{\bquark\cquark}}{\jpsi\Xicp}$ decay is kinematically very similar.
Assuming
$\frac{f_{\Xires_{\bquark\cquark}^+}}{f_{\Bcp}}\sim0.2$,
$\frac{\BR(\decay{\Xires_{\bquark\cquark}^+}{\jpsi\Xicp})}{\BR(\decay{\Bcp}{\jpsi\Dsp})}\sim~1$, 
$\frac{\BR(\decay{\Xicp}{p\Km\pip})}{\BR(\decay{\Dsp}{\Kp\Km\pip})}\sim0.1$, and
$\epsilon_{\Xires_{\bquark\cquark}^+}/\epsilon_{\Bcp}\sim~0.5$,
a signal yield of about 600 $\decay{\Xires_{\bquark\cquark}^+}{\jpsi\Xicp}$ signal decays
in expected in Run 5, albeit with sizeable uncertainties.
Other modes could also provide sizeable signal samples.
It is likely \lhcb will observe the $\Xires_{\bquark\cquark}$ baryons in Run 3/4, and further probe the spectrum of other doubly heavy baryons with the large samples accessible in the proposed \upgradetwo. 

\subsubsection{Precision measurements of quarkonia}

The correct interpretation of the experimental polarisation results
for $S$-wave quarkonia requires a rigorous analysis of the feed-down contributions from
higher excited states ~\cite{LHCb-PAPER-2011-030,LHCb-PAPER-2014-031}.
The direct measurement of the polarisation for $\Pchi_\cquark$ and $\Pchi_\bquark$ states is necessary to decrease this 
model dependence. Since $P$-wave states are practically free from the feed-down from higher excited states,
any $\Pchi_\cquark$ and $\Pchi_{\bquark}$ polarisation measurements could be interpreted in a robust manner
 without additional model assumptions.

Recent discovery of the $\decay{\Pchi_{\cquark1,2}}{\jpsi\mumu}$ decays~\cite{LHCb-PAPER-2017-036} opens the possibility
to perform a detailed study of $\Pchi_{\cquark}$ production, allowing almost background-free measurements even for
very low transverse momentum of $\Pchi_\cquark$~candidates. 
Due to the excellent mass resolution, the vector state $\Pchi_{\cquark1}$  and the tensor state $\Pchi_{\cquark2}$ are well separated,
eliminating the possible systematic uncertainty  caused by the large overlap of
these states in the $\decay{\Pchi_{\cquark}(\Pchi_\bquark)}{\jpsi(\PUpsilon)\g}$ decay\cite{
  LHCb-PAPER-2011-019,
  LHCb-PAPER-2011-030,
  LHCb-PAPER-2012-015,
  LHCb-PAPER-2013-028,
  LHCb-PAPER-2014-031,
  LHCb-PAPER-2014-040}.

An integrated luminosity of 300\invfb will allow the high-multipole contributions 
to the  $\decay{\Pchi_{\cquark}}{\jpsi\mumu}$ amplitude to be probed, namely
the magnet-dipole contribution for $\Pchi_{\cquark1}$~decays and
the magnet-dipole and electrical-octupole contributions for $\Pchi_{\cquark2}$ decays. 

Use of Run 5 dataset is also necessary  to measure the $(\pt,y)$ dependence of $\Pchi_\cquark$ polarisation parameters. 
 In addition the effect of the form factor in
the~decays $\decay{\Pchi_{\cquark}}{\jpsi\mumu}$~\cite{Faessler:1999de,Luchinsky:2017pby}
could be probed with the precision of several percent from the shape of the $m(\mumu)$ spectra.

Studies of the double quarkonia production allows independent tests for the  quarkonia production mechanism,
and in particular for the role of the colour-octet. So far the \lhcb collaboration has  analysed double \jpsi production at 7\tev and 13\tev data with relatively small
datasets~\cite{LHCb-PAPER-2011-013,LHCb-PAPER-2016-057}.
Using $280$\invpb of data collected at $\sqs=13\tev$,
$(1.05\pm0.05)\times10^3$ signal \jpsi\jpsi events are observed.
However, even with the larger sample of \jpsi\jpsi events now available, 
it is not possible to distinguish between 
the different theory descriptions of the single-parton scattering (SPS) mechanism~\cite{Sun:2014gca,Likhoded:2016zmk, Lansberg:2013qka, Lansberg:2014swa,Lansberg:2015lva,Shao:2012iz,Shao:2015vga, Baranov:2011zz,Baranov:1993qv} nor to separate the
  contributions from the SPS and double parton  scattering (DPS) mechanisms~\cite{Bansal:2014paa,Belyaev:2017sws}. 
Though the larger samples collected during  \upgradeone will allow some progress on these questions, the measurement of the correlation of \jpsi polarisation parameters will only be possible with the \upgradetwo data set.

In addition, while it is likely that  $\PUpsilon\PUpsilon$ and $\jpsi\PUpsilon$ production will be observed in the near future (assuming the dominance of the DPS mechanism),
the determination of the relative SPS and DPS contributions, as well as the discrimination between different SPS theory models, will require precision measurements only possible with  Run 5 data.
