% don't remove the folling lines, and edit the defintion of \main if needed
\documentclass[../report.tex]{subfiles}
\providecommand{\main}{..}
\IfEq{\jobname}{\currfilebase}{\AtEndDocument{\biblio}}{}
% until here

\begin{document}

\section{Lattice QCD in the HL/HE-LHC era}\label{sec:lattice}
{\it\small Authors (TH): Michele Della Morte, Elvira G\'amiz, Enrico Lunghi} 

We discuss the ten-year projections described in 
Sect.~\ref{sec:metro-CKM}
for the lattice inputs on hadronic parameters, by presenting the current status
and reviewing the main sources of uncertainty for different quantities.
A naive application of Moore's law (computing power doubling 
each two years) would give a reduction in the errors
by a factor around 3 by 2025. However, such an extrapolation is not always appropriate since the dominant 
uncertainties are often systematic.
For this reason attempting to extrapolate even farther in the future is subject to
very large uncertainties. The lattice approach is systematically improvable by
construction, however in order to almost completely remove the main systematic
affecting current computations one would have to extrapolate the performance of
present algorithms to unexplored regions of parameters. Any such extrapolation
would be quite unreliable. It is anyway reasonable to expect that by 2030 the
uncertainties related to the inclusion of electromagnetic effects, where relevant,
will be removed.

The last FLAG review~\cite{Aoki:2016frl}, or
 its online version~\footnote{{\tt{http://flag.unibe.ch/}}}, still provides
an almost up-to-date picture of the precision reached so far.
For example, the target accuracy on
$f_{B_s}$ and the ratio $f_{B_s}/f_{B_d}$ of about 0.4\% has already been achieved.
The latest $N_f=2+1+1$ results by the FNAL/MILC Collaboration 
in~\cite{Bazavov:2017lyh} quote
very similar errors for those two quantities.
Electromagnetic corrections (together with the known long- and short-distance 
electroweak contributions) in that case are directly subtracted from
the experimental decay rate quoted by PDG, such that the decay constant, which
is purely a QCD quantity, is the only hadronic parameter needed for
the extraction of $|V_{ub}|$ from leptonic $B$-decays.
Alternatively, one could consider the full transition rate on the lattice,
as proposed in~\cite{Carrasco:2015xwa}, and further discussed in~\cite{Patella:2017fgk}
concerning several subtleties arising in a straightforward application of the
method to heavy-meson decays.

In addition to the inclusion of isospin breaking corrections, the most important
limiting factor in the achievable precision of future lattice computations of decay constants
is probably going to be the scale setting, i.e., the conversion from
lattice units to MeV. 
This is quite obviously important for dimensionful
quantities, but it is also critical for some dimensionless ones, e.g., 
the hadronic contribution to the anomalous magnetic moment of the muon,
where one needs to convert the muon mass to lattice units (see~\cite{DellaMorte:2017dyu} for a detailed discussion of this point).
The current precision in the knowledge of the lattice spacing
is at the half-percent level. Going beyond that is challenging, as one needs
a quantity which is both very well known experimentally and precisely computable
on the lattice, with both small statistical as well as systematic uncertainties.
Future strategies may involve combining/averaging several quantities for the scale 
setting.~\footnote{We acknowledge S. Gottlieb and R. Van de Water for a discussion on this issue.}
On the other hand, ratios of decay constants can be obtained very precisely  from lattice simulations,
especially for light mesons, in which case discretization effects are not as severe as
for heavy-light mesons.
For example, the last calculations of the ratio of decay constants
$f_{K^\pm}/f_{\pi^\pm}$ has reached a 0.15\% error~\cite{Bazavov:2017lyh}.

We turn next to the bag parameter $B_K$, relevant for the theoretical prediction
of  $\varepsilon_K$, which encodes indirect CP-violation in the neutral kaon system.
The global estimate from FLAG for the $N_f=2+1$ theory has a 1.3\% error.  
Again, the error is systematic-dominated. In particular, one of the largest uncertainties
comes from the conversion between the lattice renormalization scheme (e.g., Schr\"odinger Functional (SF) 
or MOM) and $\overline{\rm MS}$. This is typically done at one-loop and to improve on that
one needs either a higher order computation~\footnote{See Ref.~\cite{Giusti:2018guw}
for recent work in this direction.}, or to follow the non-perturbative running up to
as large a scale as possible. In the SF scheme, but only in the two-flavor case, the matching scale 
has been pushed to the elctroweak scale, where truncation errors can be safely neglected~\cite{Dimopoulos:2007ht}.
The FLAG averages of the $B_K$ parameter have been very stable throughout different reviews. The main reason is,
at least for the 2+1 flavor setup, that the final value is dominated by a single computation from 2011~\cite{Durr:2011ap}.
It should therefore be possible to significantly reduce the error in the next few years.
In fact, with current methods and existing configurations several collaborations can probably reach that precision.

The main hadronic
uncertainty in the theoretical prediction of $\varepsilon_K$ is currently due to long-distance contributions 
not captured by the short-distance parameter $B_K$.
An approach for computing those non-perturbatively has been put forward in~\cite{Christ:2012se}
and preliminary results have been reported at the last Lattice conferences, with the most recent update
in~\cite{Bai:2016gzv}. Although the parameters in the lattice simulations are not physical,
producing rather heavy pions (329 MeV), and only one lattice resolution with $a>$ 0.1 fm has
been considered, the long distance correction to $\varepsilon_K$
is found to be rather large, amounting to about 8(5)\% of the experimental value.
For a more accurate estimate, the calculation is being repeated with physical kinematics
and finer lattice spacings~\footnote{See {\tt{https://indico.fnal.gov/event/15949/session/3/contribution/151/material/slides/0.pdf}}.}. 
It is worth pointing out that other approaches (e.g., in~\cite{Buras:2010pza}) produce values corresponding to a 4\% contribution at most.
Finally, it has been emphasized in~\cite{Bailey:2018feb} that the theoretical 
estimate of $\varepsilon_K$
based on lattice inputs strongly depends on the value of $V_{cb}$. Its current uncertainty, due
to the tension between inclusive and exclusive determinations, represents then the largest source of systematic error. 

In the $B$-sector, theoretical predictions of $B_q-\bar B_q$ mixing observables in the SM and beyond
depend on hadronic matrix elements of local four-fermion operators. 
Historically, these matrix elements have been parametrized in terms of the
so-called bag parameters. In the SM, the mass difference for a
$B_q^0$ meson depends on a single matrix element, proportional to
$f_{B_q}^2\hat B_{B_q}$, or, equivalently, on a single bag parameter, $\hat B_{B_q}$. 

There has been steady progress  in the last decade in lattice determinations of these matrix elements, 
with errors $\sim 4-5\%$ for $\hat B_{B_q}$~\cite{Aoki:2016frl}. However,
there is still a lot of room for improvement. First of all, current calculations
are not done on the last generation ensembles with $N_f=2+1+1$,   
physical light-quark masses, the smallest lattice spacings, and/or the most
improved lattice actions. Performing simulations on those ensembles will
reduce, and in some cases eliminate, the dominant errors in $B$-mixing: statistics,
continuum and chiral extrapolations, no inclusion of charm quarks on the sea,
and heavy-quark discretization. In addition, current calculations with $N_f=2+1$
flavors of sea quarks rely on effective field theories for the description of
the $b$ quark. Using a fully relativistic description instead, will further
reduce the heavy-quark discretization uncertainty and will also allow more precise
renormalization techniques, the other large source of uncertainty. 
For a particular lattice calculation, either the decay constants $f_B$
and $f_{B_s}$ are obtained within the same analysis in a correlated way,
or external inputs are used to get the bag parameters from the extracted matrix
elements. The recent and projected progress on the
determination of these decay constants will thus partly contribute to the reduction in the bag parameters uncertainty. 

All the improvements described above could considerably reduce the error in
$\hat B_{B_s}$ to a 0.8\% level. 
For the ratio of bag parameters, $B_{B_s}/B_{B_d}$, the error could be reduced
from the current 2\% to 0.5\% or even less. A similar reduction
can be achieved for the bag parameters describing the BSM contributions to the mixing. 
On-going calculations of hadronic matrix elements of NLO operators~\cite{Davies:2017jbi}
will also contribute to a substantial reduction in the uncertainty of $\Delta\Gamma_q$
in the next years.

Matrix elements describing the short-distance contribution to $D$-meson mixing in
the SM and beyond could benefit from a similar error reduction. However, in contrast to $B$-mixing, $D$-mixing is dominated by long-distance contributions and thus
reducing the error of the short-distance contribution is not so pressing.

Next we review the status and the prospects for lattice computations
of form factors for a number of semileptonic decays.
 The vector form factor at zero momentum transfer, $f_+^{K^0\pi^-}(q^2=0)$, needed to extract
$|V_{us}|$ from experimental measurements of $K\to\pi\ell\nu$ decay widths, 
is among the most accurate quantities obtained on the lattice. 
The most recent calculation of this form factor, not included in the last
FLAG review, has reduced the error to 0.20\%~\cite{Bazavov:2018kjg}, the 
same level as the experimental error~\cite{Moulson:2017ive}. 
The main source of uncertainty on the lattice side is
now statistics, which could be reduced by several lattice collaborations, both with
$N_f=2+1$ and $N_f=2+1+1$, extending their simulations to existing and planned
ensembles. Those ensembles include set of configurations with smaller lattice spacings
than those that are currently being used for calculations of $f_+^{K^0\pi^-}(q^2=0)$,
which will further reduce the total uncertainty. Other important sources of error
on current calculations arise from the scale setting and the 
uncertainty in the input parameters: quark masses 
and ChPT low energy constants. Better determinations of those parameters, 
which are needed in the analyses of many observables, are underway or expected. 
With those improvements the error in $f_+^{K^0\pi^-}(q^2=0)$, including
strong isospin-breaking corrections, will be soon enterely dominated by statistics
and could achieve the 0.12\% level in ten years. 

The uncertainty of the experimental average $|V_{us}|f_+^{K^0\pi^-}$, $0.19\%$, includes
errors of the estimated long-distance electromagnetic effects and
the difference between strong isospin-breaking effects in charged and neutral modes.
Those estimates, which use phenomenology and ChPT techniques, will become dominant
sources of error in the future. There are proposals to extend the study of full
leptonic transition rates to semileptonic decays~\footnote{See talk by Chris Sachrajda at
  Lattice 2018, https://indico.fnal.gov/event/15949/session/3/contribution/163}.  
Future calculations on the lattice thus should be able to not only reduce the error
on the QCD form factor, but also the electromagnetic and strong isospin-breaking
corrections contributing to the experimental uncertainty.

Lattice calculations are not limited to $q^2=0$ determination of $f_+^{K^0\pi^-}$. There exist results
for the momentum depedence of both the vector and scalar form
factors~\cite{Carrasco:2016kpy}. The $q^2$ dependence of $f_+^{K\pi}(q^2)$ obtained in
Ref.~\cite{Carrasco:2016kpy} agrees very well with experimental data, although the
determination of the constants entering in the dispersive parametrization of the
energy dependence, usually adopted in experiment, is still less precise when using lattice data.
However, in the future, with the improvements discussed above, lattice could provide
the most precise evaluation of the phase-space integral for kaon semileptonic decays. 

We discuss next three groups of $b$-hadron form factors of crucial phenomenological importance: heavy meson to stable pseudoscalar meson transitions, $B\to (\pi,K,D)$ and $B_s \to (K, D_s)$; heavy meson to unstable vector meson transitions, $B\to (D^*, K^*))$; and heavy baryon to baryon transitions, $\Lambda_b \to (p, \Lambda_c)$.
Form factors in the first group have been calculated by several collaborations, since the presence of a stable final state particle simplifies the analysis. Form factors in the second group are complicated by the presence of an unstable final state meson. While there is a solid theoretical foundation to treat such situation on the lattice~\cite{Briceno:2014uqa}, it should be noted that the huge hierarchy $\Gamma_{D^*}/M_{D^*} \sim 4 \times 10^{-5} \ll \Gamma_{K^*}/M_{K^*} \sim 6\times  10^{-2}$ reduces the impact of the final state meson decay for the $D^*$ case. 
In all cases, lattice calculations have higher accuracy for kinematical configurations in which the final state hadron has low recoil( large $q^2$) while experimental measurements tend to have better efficiency at large recoil (small $q^2$). 

Moreover, it is important to stress the role of form factors parametrizations which must be used to extrapolate the lattice results at low-$q^2$, for example the $B\to D^{(*)}$ form factors, or when combining lattice and experimental results in order to extract CKM parameters, e.g., $|V_{cb}|$ and $|V_{ub}|$. 
As mentioned above, lattice and experiments present differential (binned) distributions which tend to have higher accuracy
at low and high recoil, respectively. Their combination covers the whole kinematic spectrum, thus reducing drastically the
sensitivity to any given form factor parametrization. 
In situations in which experiments and/or lattice collaborations perform one sided extrapolations, the impact of the chosen parametrization can be very large (see, for instance, the extraction of $|V_{cb}|$ from $B\to D^*\ell\nu$ discussed in Refs.~\cite{Bigi:2017njr, Grinstein:2017nlq}).

The state of the art for the form factor determinations is 
represented by the averages in the latest FLAG review~\cite{Aoki:2016frl}. In the following we estimate the theoretical accuracy on the form factors that can be achieved over the next eight years and their impact on the extraction of $|V_{ub}|$ and $|V_{cb}|$. As a general rule, we estimate the overall drop in uncertainty to be a factor of 3 both in its statistical and systematic components. 
Thus, for many calculations the total extrapolated lattice uncertainty drops below the 1\% level, at which point QED effects need to be included. This can be done on the lattice either in the quenched QED limit or by explicitly generating mixed QCD-QED configurations. The inclusion of QED corrections to form factors is complicated  by IR safety issues which can, nonetheless,  be addressed in lattice calculations~\cite{Carrasco:2015xwa}. It is likely, but not guaranteed, that QED corrections to the form factors we discuss below will be calculated in the time frame we consider. We consider two scenarios according to whether QED corrections have or have not been calculated; in the latter case we add in quadrature an extra 1\% uncertainty. 
\vspace{-0.35cm}
\begin{itemize}
\item
{ $B\to\pi$ and $|V_{ub}|^{}$.}
The overall point-by-point uncertainty should reduce to the sub-percent level due the adoption of Highly Improved Staggered Quark (HISQ) for heavy quarks on the finer lattices and physical light quark masses. Currently the uncertainty on the extraction of $|V_{ub}|$ from a simultaneous fit of lattice form factors~\cite{Lattice:2015tia, Flynn:2015mha} and binned measurements of the semileptonic branching ratio is 3.7\%. The theoretical and experimental contributions to this error can be estimated at the 2.9\% and 2.3\% level, respectively. A reduction of the lattice uncertainties along the lines mentioned above would naively reduce the former uncertainty to about 1\%. As a cautionary note we point out that these estimates are sensitive to the information pertaining to the shape of the form factor, which is controlled by the correlation in the synthetic data and by the accuracy of future experimental results. In conclusion, the theory uncertainty on $|V_{ub}|$ is expected to decrease from 2.9\% to 1\% (1.4\% if QED corrections are not calculated).
%
\item
{ $B\to K$.}
These form factors have been calculated in Refs.~\cite{Bouchard:2013pna, Bailey:2015dka}. The FLAG average of the various form factors has uncertainties on the first coefficient in the BCL $z$-expansion of about 2\%. Future improvements can push this uncertainty to the 0.7\% (1.2\%) level according to the assumption on the inclusion of QED corrections.
%
\item
{ $B\to D$ and $|V_{cb}|^{}$.}
A simultaneous fit of the lattice $B\to D$ form factor~\cite{Na:2015kha, Lattice:2015rga}, and experimental data on the semileptonic $B\to D\ell\nu$ branching ratio yields an uncertainty on $|V_{cb}|$ at the 2.5\% level. The theoretical and experimental contributions to this error can be estimated as 1.4\% and 2.0\%, respectively. The lattice contribution to the uncertainty is essentially controlled by the 1.5\% error on the coefficients $a_0^+$ of the BCL $z$-expansion. Note that in the FLAG fit the zero-recoil value has an uncertainty of only 0.7\%: ${\cal G}^{B\to D}(1) = 1.059 (7)$. Leaving aside the issue that this uncertainty is lower than the conservative ballpark of the missing QED corrections, in order to use this very precise result one needs the corresponding zero-recoil experimental branching ratio which has an uncertainty of about 3\%. Clearly the zero-recoil extraction of $|V_{cb}|$ is inferior to the simultaneous fit method.  The uncertainty on the form factors is expected to drop by about a factor of 3 and to reach the 0.5\% (1.1\%) level. Correspondingly, the theory uncertainty on $|V_{cb}|$ is expected to decrease from 1.4\% to 0.3\% (1\%).
%
\item
{ $B_s\to K$,  $B_s \to D_s$ and $|V_{ub}|/|V_{cb}|$.}
These form factors have been calculated in Refs.~\cite{Bouchard:2014ypa, Flynn:2015mha}. 
The FLAG average of the various form factors has uncertainties on the first coefficient in the BCL $z$-expansion of 
about 4\%. Future improvements can push this uncertainty to the 1.3\% (1.7\%) level according to the assumption on the 
inclusion of QED corrections. Very recently the authors of Ref.~\cite{Monahan:2018lzv}
presented a calculation of the ratio $f_+^{(B\to K)} (0) / f_+^{(B_s\to D_s)} (0)$ with a total uncertainty of 13\%,
which is expected to reduce to about 4\% in the time frame we consider. This would also be the expected theory
uncertainty on the extraction of the CKM ratio $|V_{ub}|/|V_{cb}|$ from future measurements. Note that Ref.~\cite{Monahan:2018lzv}
presents full information on the $q^2$ distribution of the two form factors and that once differential experimental results become available the
above mentioned theory uncertainty will further decrease.
%
\item
{ $B\to D^*$ and $|V_{cb}|^{}$.}
There are two calculations of the form factor at zero-recoil~\cite{Bailey:2014tva,Harrison:2017fmw}. 
The most precise one in Ref.~\cite{Bailey:2014tva}
quotes an uncertainty of 1.4\% on ${\cal F}^{B\to D^*}(1)$. This uncertainty includes  an estimate of the size of missing QED effects at the 0.5\% level. Assuming a factor 3 reduction on lattice uncertainties one projects uncertainties on this form factor at the 0.4\% level (the estimate increases to 0.7\%, (1.1\%) in case QED corrections are not calculated, and are estimated to be 0.5\% (1\%) in size). We should point out that the Fermilab/MILC collaboration is about to publish a calculation of the form factor at non-zero recoil. This will allow simultaneous theory/experiment fits which are expected to further reduce the uncertainty on $|V_{cb}|$. 
%
\item
{ $B\to K^*$.}
The $B\to K^*$ form factors have been calculated in Refs.~\cite{Horgan:2013hoa, Horgan:2013pva}. These calculations are performed for a stable $K^*$ and, as we mentioned, receive potentially large corrections from the relatively large $K^*$ width. Currently an effort towards implementing the proper decay chain $B\to K^* \to K \pi$ along the lines described in Refs.~\cite{Luscher:1986pf, Luscher:1990ux, Luscher:1991cf, Lage:2009zv, Bernard:2010fp, Doring:2011vk, Hansen:2012tf, Briceno:2012yi, Dudek:2014qha, Briceno:2014uqa} is undergoing. This type of calculation will remove the uncontrolled uncertainty due to the assumption of a stable final state vector meson. Unfortunately, it is too soon to present an estimate on what the expected form factor uncertainty is going to be. Once the first complete result will become available, it will be possible to estimate the expected progression in error reduction as for all the other quantities we have considered.
%
\item 
{ $\Lambda_b\to p$, $\Lambda_b\to \Lambda_c$ and $|V_{ub}|/|V_{cb}|$.}
Currently there is only one calculation of these form factors~\cite{Detmold:2015aaa} which allows an extraction of the ratio $|V_{ub}|/|V_{cb}|$ with a theory uncertainty of 4.9\%. Over the next few years the total lattice uncertainty on these form factors is expected to reduce by a factor of 2~\cite{Meinel:2018MITP}. We estimate the improvement over the following decade to be another factor of 2~\footnote{We thank S. Meinel for a discussion on this point.}. This implies that in the time frame we are considering we expect a reduction in the theory uncertainty on $|V_{ub}|/|V_{cb}|$ to the 1.2\% (1.6\%) level.
%
\end{itemize}
An important development from the phenomenological point of view is related to
the computation of long-distance effects for rare decays as $K \to \pi \ell^+ \ell^-$.
A lattice approach has been put forward in~\cite{Christ:2015aha} and exploratory
results have appeared in~\cite{Christ:2016mmq}. The extension to $b \to s$
transitions however is by no means straightforward.

\vspace{-0.2cm}
In order to make it easily accessible, we collect all the information presented in this Section, including our projections for 2025, in Table~\ref{table:Proj}: 
\vspace{-0.3cm}
\begin{itemize}  
\item In the second and third columns we quote the current published best averages of lattice results with references.
\item In the fourth column we quote the error corresponding to the published value in the second column and,
  in parenthesis, the error coming from a couple of recent calculations, significantly reducing current errors, but that either are not published yet ($f_+(0)^{K\to\pi}$) or are not included in the FLAG-2016 averages (decay constants). The latest will be included in the next release of the FLAG averages in 2019.
\item Significant reduction of current errors in the decay constants are very
  unlikely, so we did not quote any numbers for these quantities. 
\item For semileptonic decays:
  \vspace{-0.13cm}
  \begin{itemize}
  \item $X\to Y$ for $|V_{ab}|$ means the theory error in the extraction of $|V_{ab}|$ from  that exclusive mode.
  \item Fifth column: The two errors correspond to assuming that isospin breaking corrections are calculated by that
    time (first number) or that they are still estimated phenomenologically (second number in parenthesis).
\end{itemize}
\end{itemize}

\begin{center}
\begin{sidewaystable}
\begin{center}
\begin{tabular}{cccccc}
\hline\hline
Quantity & Published averages & Reference & error (to be published/not in FLAG-2016)
& Phase I & Phase II\\
\hline
$f_{K^\pm}$ & $155.7(7)$~MeV & $N_f=2+1$~\cite{Aoki:2016frl} & 0.4\%  & 0.4\% & 0.4\%\\
$f_{K^\pm}/f_{\pi^{\pm}}$ & 1.193(3) &  $N_f=2+1+1$~\cite{Aoki:2016frl} & 0.25\%(0.15\%, symmet.~\cite{Bazavov:2017lyh})& 0.15\% & 0.15\% \\
$f_+^{K\to\pi}(0)$ & $0.9706(27)$ & $N_f=2+1+1$~\cite{Aoki:2016frl} & 0.28\% (0.20\%~\cite{Bazavov:2018kjg}) & 0.12\% & 0.12\%\\
$B_K$  & $0.7625(97)$ &  $N_f=2+1$~\cite{Aoki:2016frl} & 1.3\% & 0.7\% & 0.5\% \\
\hline
$f_{D_s}$ & 248.83(1.27) &  $N_f=2+1+1$~\cite{Aoki:2016frl} & 0.5\%(0.16\%~\cite{Bazavov:2017lyh}) & 0.16\% & 0.16\%\\
$f_{D_s}/f_{D^+}$ & 1.1716(32) &  $N_f=2+1+1$~\cite{Aoki:2016frl} & 0.27\%(0.14\%~\cite{Bazavov:2017lyh}) & 0.14\% & 0.14\%\\
$f_{B_s}$ & 228.4(3.7) &  $N_f=2+1$~\cite{Aoki:2016frl} & 1.6\%(0.56\%~\cite{Bazavov:2017lyh}) & 0.5\% & 0.5\% \\
$f_{B_s}/f_{B^+}$ & 1.205(7)& $N_f=2+1+1$~\cite{Aoki:2016frl} & 0.6\%(0.4\%~\cite{Bazavov:2017lyh}) & 0.4\% & 0.4\%\\
$B_{B_s}$ & 1.32(5)/1.35(6) & $N_f=2/N_f=2+1$~\cite{Aoki:2016frl} & $\sim 4$\% & 0.8\% & 0.5\%\\
$B_{B_s}/B_{B_d}$ & 1.007(21)/1.032(28) & $N_f=2/N_f=2+1$~\cite{Aoki:2016frl} & 2.1\%/2.7\% & 0.5\% & 0.3\%\\
$\xi$ & 1.206(17) & $N_f=2+1$~\cite{Aoki:2016frl} & 1.4\% & 0.3\% & 0.3\%\\
$\overline m_c (\overline m_c)$ & 1.275(8)~GeV &  $N_f=2+1$ ~\cite{Aoki:2016frl} & 0.6\% & 0.4\% & 0.4\%\\
\hline
$B\to \pi$ for $|V_{ub}|_{{\rm theor}}$ &  & $N_f=2+1$~\cite{Aoki:2016frl} & 2.9\% & 1\%(1.4\%) & 1\%\\
$B\to D$ for $|V_{cb}|_{{\rm theor}}$ &  & $N_f=2+1$~\cite{Aoki:2016frl} & 1.4\% & 0.3\%(1\%) & 0.3\%\\
(first param. BCL $z$-exp.) &  & $N_f=2+1$~\cite{Aoki:2016frl} & 1.5\% & 0.5\%(1.1\%) & 0.5\%\\
\begin{tabular}{c}$B\to D^*$ for $|V_{cb}|_{{\rm theor}}$\\$h_{A_1}^{B\to D^*}(\omega=1)$\\\end{tabular} &  & $N_f=2+1$~\cite{Aoki:2016frl} & 1.4\% & 0.4\%(0.7\%) & 0.4\% \\  
$P_1^{B\to D^*}(\omega=1)$ & & No LQCD available & & 1-1.5\% & 1\%\\
\begin{tabular}{c}$\Lambda_b\to p(\Lambda_c)$\\for $|V_{ub}/V_{cb}|_{{\rm theor}}$\\\end{tabular} & &
\cite{Detmold:2015aaa}  & 4.9\% & 1.2\%(1.6\%) & 1.2\%\\
\hline
$B\to K$ &  & $N_f=2+1$~\cite{Aoki:2016frl} & 2\% & 0.7\%(1.2\%) & 0.7\%\\
(first param. BCL $z$-exp.) & & & & & \\ 
$B_s\to K$ &  & $N_f=2+1$~\cite{Aoki:2016frl} & 4\% & 1.3\%(1.7\%) & 1.3\%\\
(first param. BCL $z$-exp.) & & & & & \\ 
\hline
\hline
\end{tabular}
\end{center}
\caption{Current estimates and projections for lattice QCD determinations of hadronic inputs. In the fourth column we quote the error corresponding to the published value in the second column and,
  in parenthesis, the error for forthcoming calculations that either are not published yet or are not included in the FLAG-2016 averages. See text for further explanations.}
\label{table:Proj}
\end{sidewaystable}
\end{center}

\end{document}
