% Title --------------------------------------------------
\title{{\normalfont\bfseries\boldmath\huge
\begin{center}
%  HL/HE-LHC Physics Workshop Report\\
%  WG3 : Beyond the Standard Model Physics
Beyond the Standard Model Physics\\
at the HL-LHC and HE-LHC\\
\begin{normalsize} 
\href{http://lpcc.web.cern.ch/hlhe-lhc-physics-workshop}{Report from Working Group 1 on the Physics of the HL-LHC, and Perspectives at the HE-LHC} 
\end{normalsize}
\end{center}\vspace*{0.2cm}
}}


\begin{titlepage}

% Header ---------------------------------------------------
\vspace*{-1.5cm}

\noindent
\begin{tabular*}{\linewidth}{lc@{\extracolsep{\fill}}r@{\extracolsep{0pt}}}
\vspace*{-1.2cm}\mbox{\!\!\!\includegraphics[width=.14\textwidth]{scripts/cern.jpg}} & & \\
 & & CERN-PH-2018-YYY \\  % ID 
 & & \today \\ % Date - Can also hardwire e.g.: 23 March 2010
 & & \\
\hline
\end{tabular*}

\vspace*{4.0cm}

% Title --------------------------------------------------
{\normalfont\bfseries\boldmath\huge
\begin{center}
% DO NOT EDIT HERE. Instead edit macro in main.tex to keep metadata correct
  HL/HE-LHC Physics Workshop Report\\
  WG4 : Opportunities in Flavour Physics
\end{center}
}

\vspace*{2.0cm}

% Authors -------------------------------------------------
\begin{center}
% If changing to list here, make pdfauthors in main.tex a comma
% separated list with the same names. Otherwise metadata in file will be wrong.
%Big Brain Collective Ltd.$^1$.
\bigskip
%{\normalfont\itshape\footnotesize
%$ ^1$Institute of futurology\\
%}
\author{Convenors: \\ 
\href{http://inspirehep.net/record/1021757}{A.~Cerri}$^{1}$,\, 
\href{http://inspirehep.net/record/1057204}{V.~V.~Gligorov}$^{2}$,\, 
\href{http://inspirehep.net/record/999046}{S.~Malvezzi}$^{3}$,\, 
\href{http://inspirehep.net/record/1056267}{J.~Martin Camalich}$^{4, 5}$,\, 
\href{http://inspirehep.net/record/1020959}{J.~Zupan}$^{6}$,\, 
\href{http://inspirehep.net/record/1073335}{S.~Akar}$^{6}$,\, 
\\ \vspace*{4mm} 
 Contributors: 
 \\ 
\href{http://inspirehep.net/record/1068305}{J.~Alimena}$^{7}$,\, 
\href{http://inspirehep.net/record/1018633}{B.~C.~Allanach}$^{8}$,\, 
\href{http://inspirehep.net/record/1048366}{W.~Altmannshofer}$^{9}$,\, 
\href{http://inspirehep.net/record/1189488}{L.~Anderlini}$^{10}$,\, 
\href{http://inspirehep.net/record/1057775}{F.~Archilli}$^{11}$,\, 
\href{http://inspirehep.net/record/1017778}{P.~Azzi}$^{12}$,\, 
\href{http://inspirehep.net/record/1277169}{S.~Banerjee}$^{13}$,\, 
\href{http://inspirehep.net/record/1079058}{W.~Barter}$^{14}$,\, 
\href{http://inspirehep.net/record/1071810}{A.~E.~Barton}$^{15}$,\, 
\href{http://inspirehep.net/record/1070828}{I.~Belyaev}$^{16}$,\, 
\href{http://inspirehep.net/record/1096983}{S.~Benson}$^{11}$,\, 
\href{http://inspirehep.net/record/1059727}{M.~Bettler}$^{17}$,\, 
\href{http://inspirehep.net/record/1311547}{R.~Bhattacharya}$^{18}$,\, 
\href{http://inspirehep.net/record/1060779}{S.~Bifani}$^{19}$,\, 
\href{http://inspirehep.net/record/1378053}{A.~Birnkraut}$^{20}$,\, 
\href{http://inspirehep.net/record/1279838}{F.~Bishara}$^{21}$,\, 
\href{http://inspirehep.net/record/1070826}{T.~Blake}$^{22}$,\, 
\href{http://inspirehep.net/record/1016078}{S.~Blusk}$^{23}$,\, 
\href{http://inspirehep.net/record/1015816}{E.~Boos}$^{24}$,\, 
\href{http://inspirehep.net/record/1262186}{M.~Borsato}$^{25}$,\, 
\href{http://inspirehep.net/record/1015512}{C.~Bozzi}$^{26, 27}$,\, 
\href{http://inspirehep.net/record/1609167}{ABragagnolo}$^{12, 28}$,\, 
\href{http://inspirehep.net/record/1056126}{J.~Brod}$^{6}$,\, 
\href{http://inspirehep.net/record/1031499}{J.~Brodzicka}$^{29}$,\, 
\href{http://inspirehep.net/record/1014978}{A.~J.~Buras}$^{30}$,\, 
\href{http://inspirehep.net/record/1321289}{L.~Cadamuro}$^{31}$,\, 
\href{http://inspirehep.net/record/1061899}{A.~Carbone}$^{32, 33}$,\, 
\href{http://inspirehep.net/record/984325}{M.~Carena}$^{34, 35}$,\, 
\href{http://inspirehep.net/record/1052521}{A.~Carmona}$^{36}$,\, 
\href{http://inspirehep.net/record/1014241}{F.~R.~Cavallo}$^{33}$,\, 
\href{http://inspirehep.net/record/1189874}{A.~Celis}$^{37}$,\, 
\href{http://inspirehep.net/record/1064384}{M.~Cepeda}$^{26}$,\, 
\href{http://inspirehep.net/record/1071755}{G.~S.~Chahal}$^{13, 38}$,\, 
\href{http://inspirehep.net/record/1272128}{M.~Chala}$^{13}$,\, 
\href{http://inspirehep.net/record/1013990}{J.~Charles}$^{39}$,\, 
\href{http://inspirehep.net/record/1020539}{M.~Charles}$^{2}$,\, 
\href{http://inspirehep.net/record/1031500}{K.~F.~Chen}$^{40}$,\, 
\href{http://inspirehep.net/record/1223006}{V.~Chobanova}$^{41}$,\, 
\href{http://inspirehep.net/record/1096980}{M.~Chrzaszcz}$^{26}$,\, 
\href{http://inspirehep.net/record/1079012}{G.~Ciezarek}$^{26}$,\, 
\href{http://inspirehep.net/record/1013451}{V.~Cirigliano}$^{42}$,\, 
\href{http://inspirehep.net/record/1013445}{M.~Ciuchini}$^{43}$,\, 
\href{http://inspirehep.net/record/1070807}{H.~Cliff}$^{17}$,\, 
\href{http://inspirehep.net/record/1013314}{J.~Cogan}$^{44}$,\, 
\href{http://inspirehep.net/record/1013274}{G.~Colangelo}$^{45}$,\, 
\href{http://inspirehep.net/record/1069963}{A.~Contu}$^{46}$,\, 
\href{http://inspirehep.net/record/1028393}{R.~Covarelli}$^{47}$,\, 
\href{http://inspirehep.net/record/1023478}{G.~Cowan}$^{48}$,\, 
\href{http://inspirehep.net/record/1060042}{A.~Crivellin}$^{49}$,\, 
\href{http://inspirehep.net/record/1012622}{G.~D'Ambrosio}$^{50}$,\, 
\href{http://inspirehep.net/record/1293755}{N.~P.~Dang}$^{51}$,\, 
\href{http://inspirehep.net/record/1262261}{A.~Davis}$^{14}$,\, 
\href{http://inspirehep.net/record/1096978}{K.~De Bruyn}$^{26}$,\, 
\href{http://inspirehep.net/record/1224351}{W.~Dekens}$^{52, 53}$,\, 
\href{http://inspirehep.net/record/1072922}{H.~De la Torre}$^{54}$,\, 
\href{http://inspirehep.net/record/1012113}{F.~Deliot}$^{55}$,\, 
\href{http://inspirehep.net/record/996726}{M.~Della Morte}$^{56}$,\, 
\href{http://inspirehep.net/record/1021058}{S.~Demers}$^{57}$,\, 
\href{http://inspirehep.net/record/1061025}{D.~Derkach}$^{58}$,\, 
\href{http://inspirehep.net/record/1028949}{U.~De Sanctis}$^{59, 60}$,\, 
\href{http://inspirehep.net/record/1066064}{O.~Deschamps}$^{61}$,\, 
\href{http://inspirehep.net/record/1011958}{S.~Descotes-Genon}$^{62}$,\, 
\href{http://inspirehep.net/record/1070796}{F.~Dettori}$^{63}$,\, 
\href{http://inspirehep.net/record/1054949}{A.~Di Canto}$^{26}$,\, 
\href{http://inspirehep.net/record/1023527}{M.~Dinardo}$^{3, 64}$,\, 
\href{http://inspirehep.net/record/1021779}{P.~Dini}$^{3}$,\, 
\href{http://inspirehep.net/record/1028663}{M.~D'Onofrio}$^{63}$,\, 
\href{http://inspirehep.net/record/1096976}{F.~Dordei}$^{46}$,\, 
\href{http://inspirehep.net/record/1060820}{M.~Dorigo}$^{26, 65}$,\, 
\href{http://inspirehep.net/record/1011495}{A.~dos Reis}$^{66}$,\, 
\href{http://inspirehep.net/record/1011294}{L.~Dudko}$^{24}$,\, 
\href{http://inspirehep.net/record/1401131}{L.~Dufour}$^{11}$,\, 
\href{http://inspirehep.net/record/1273362}{G.~Durieux}$^{21, 67}$,\, 
\href{http://inspirehep.net/record/1011136}{S.~Dutta}$^{18}$,\, 
\href{http://inspirehep.net/record/1096974}{A.~Dziurda}$^{29}$,\, 
\href{http://inspirehep.net/record/1233187}{A.~Esposito}$^{68}$,\, 
M.~Estevez$^{69}$,\, 
\href{http://inspirehep.net/record/1498216}{D.~A.~Faroughy}$^{70}$,\, 
\href{http://inspirehep.net/record/1073229}{G.~Fedi}$^{71}$,\, 
\href{http://inspirehep.net/record/1080244}{S.~Fiorendi}$^{3, 64}$,\, 
\href{http://inspirehep.net/record/1064681}{F.~Fiori}$^{71}$,\, 
\href{http://inspirehep.net/record/1071746}{C.~Fitzpatrick}$^{26}$,\, 
\href{http://inspirehep.net/record/1009833}{R.~Fleischer}$^{11}$,\, 
\href{http://inspirehep.net/record/1078979}{M.~Fontana}$^{26}$,\, 
\href{http://inspirehep.net/record/1009609}{P.~J.~Fox}$^{34}$,\, 
\href{http://inspirehep.net/record/1074607}{M.~Freytsis}$^{72}$,\, 
E.~Gabriel$^{48}$,\, 
\href{http://inspirehep.net/record/1008976}{P.~Gambino}$^{47}$,\, 
\href{http://inspirehep.net/record/1008971}{E.~G\'amiz}$^{73}$,\, 
\href{http://inspirehep.net/record/1341934}{J.~Garc{\'\i}a Pardi{\~n}as}$^{74}$,\, 
\href{http://inspirehep.net/record/1279168}{L.~S.~Geng}$^{75}$,\, 
\href{http://inspirehep.net/record/1064668}{E.~Gersabeck}$^{14}$,\, 
\href{http://inspirehep.net/record/1058420}{M.~Gersabeck}$^{14}$,\, 
\href{http://inspirehep.net/record/1027525}{T.~Gershon}$^{22}$,\, 
\href{http://inspirehep.net/record/1008386}{A.~Gilbert}$^{26}$,\, 
\href{http://inspirehep.net/record/1058698}{M.~Gonzalez-Alonso}$^{26}$,\, 
\href{http://inspirehep.net/record/1062192}{P.~Govoni}$^{3}$,\, 
\href{http://inspirehep.net/record/1007670}{G.~Graziani}$^{10}$,\, 
\href{http://inspirehep.net/record/1198373}{A.~Greljo}$^{26}$,\, 
\href{http://inspirehep.net/record/1074923}{L.~Grillo}$^{14}$,\, 
\href{http://inspirehep.net/record/1007511}{B.~Grinstein}$^{53}$,\, 
\href{http://inspirehep.net/record/1050762}{A.~Grohsjean}$^{21}$,\, 
\href{http://inspirehep.net/record/1007437}{Y.~Grossman}$^{76}$,\, 
\href{http://inspirehep.net/record/1032647}{D.~Guadagnoli}$^{77}$,\, 
\href{http://inspirehep.net/record/1027513}{F.~-K.~Guo}$^{78, 79}$,\, 
\href{http://inspirehep.net/record/1700541}{L.~Guzzi}$^{3, 64}$,\, 
\href{http://inspirehep.net/record/1019892}{J.~Haller}$^{80}$,\, 
\href{http://inspirehep.net/record/1073314}{B.~Hamilton}$^{81}$,\, 
\href{http://inspirehep.net/record/1019568}{R.~Harnik}$^{34}$,\, 
\href{http://inspirehep.net/record/1189481}{D.~Hill}$^{82}$,\, 
\href{http://inspirehep.net/record/1005991}{G.~Hiller}$^{20}$,\, 
\href{http://inspirehep.net/record/1005812}{K.~Hoepfner}$^{83}$,\, 
\href{http://inspirehep.net/record/1068037}{J.~M.~Hogan}$^{84, 85}$,\, 
\href{http://inspirehep.net/record/1005239}{T.~Hurth}$^{36}$,\, 
\href{http://inspirehep.net/record/1019801}{O.~Igonkina}$^{11, 86}$,\, 
\href{http://inspirehep.net/record/1070755}{P.~Ilten}$^{19}$,\, 
\href{http://inspirehep.net/record/1004578}{S.~J{\"a}ger}$^{1}$,\, 
\href{http://inspirehep.net/record/1064540}{S.~Jain}$^{87}$,\, 
\href{http://inspirehep.net/record/1028390}{M.~John}$^{82}$,\, 
\href{http://inspirehep.net/record/1071815}{D.~Johnson}$^{26}$,\, 
\href{http://inspirehep.net/record/1258549}{M.~Jung}$^{88}$,\, 
\href{http://inspirehep.net/record/1280642}{N.~Jurik}$^{82}$,\, 
\href{http://inspirehep.net/record/1003971}{M.~Kado}$^{89}$,\, 
\href{http://inspirehep.net/record/1003952}{A.~L.~Kagan}$^{6}$,\, 
\href{http://inspirehep.net/record/1033909}{J.~F.~Kamenik}$^{70, 90}$,\, 
\href{http://inspirehep.net/record/1003603}{M.~Karliner}$^{91}$,\, 
\href{http://inspirehep.net/record/1073117}{M.~Kenzie}$^{17}$,\, 
\href{http://inspirehep.net/record/1070746}{B.~Khanji}$^{26}$,\, 
\href{http://inspirehep.net/record/1123712}{J.~Kieseler}$^{26}$,\, 
\href{http://inspirehep.net/record/1274068}{T.~Kitahara}$^{92}$,\, 
\href{http://inspirehep.net/record/1396923}{T.~Klijnsma}$^{93}$,\, 
\href{http://inspirehep.net/record/1002577}{M.~Knecht}$^{39}$,\, 
\href{http://inspirehep.net/record/1057492}{R.~Kogler}$^{80}$,\, 
\href{http://inspirehep.net/record/1019544}{P.~Koppenburg}$^{11}$,\, 
\href{http://inspirehep.net/record/1002130}{A.~Korytov}$^{31}$,\, 
\href{http://inspirehep.net/record/1033910}{N.~Ko\v snik}$^{70, 90}$,\, 
\href{http://inspirehep.net/record/1028692}{M.~Kreps}$^{22}$,\, 
\href{http://inspirehep.net/record/1058264}{C.~Langenbruch}$^{83}$,\, 
\href{http://inspirehep.net/record/1001114}{U.~Langenegger}$^{49}$,\, 
\href{http://inspirehep.net/record/1028434}{T.~Latham}$^{22}$,\, 
\href{http://inspirehep.net/record/1000869}{R.~F.~Lebed}$^{94}$,\, 
\href{http://inspirehep.net/record/1000579}{A.~J.~Lenz}$^{13}$,\, 
\href{http://inspirehep.net/record/1000515}{O.~Leroy}$^{44}$,\, 
\href{http://inspirehep.net/record/1074984}{Q.~Li}$^{95}$,\, 
\href{http://inspirehep.net/record/1000247}{F.~Ligabue}$^{71, 96}$,\, 
\href{http://inspirehep.net/record/999554}{E.~Lunghi}$^{97}$,\, 
\href{http://inspirehep.net/record/1028960}{F.~Mahmoudi}$^{98}$,\, 
\href{http://inspirehep.net/record/1028307}{G.~Mancinelli}$^{44}$,\, 
\href{http://inspirehep.net/record/1395721}{P.~Mandrik}$^{99}$,\, 
\href{http://inspirehep.net/record/998950}{T.~Mannel}$^{100}$,\, 
\href{http://inspirehep.net/record/1066755}{J.~F.~Marchand}$^{101}$,\, 
\href{http://inspirehep.net/record/1061000}{M.~Martinelli}$^{26}$,\, 
\href{http://inspirehep.net/record/1070724}{D.~Mart\'inez Santos}$^{41}$,\, 
\href{http://inspirehep.net/record/1070724}{D.~Martinez Santos}$^{41}$,\, 
\href{http://inspirehep.net/record/998598}{F.~Martinez Vidal}$^{102}$,\, 
\href{http://inspirehep.net/record/1078065}{D.~Marzocca}$^{65}$,\, 
\href{http://inspirehep.net/record/998391}{J.~Matias}$^{103}$,\, 
\href{http://inspirehep.net/record/1642114}{P.~Matorras Cuevas}$^{74}$,\, 
\href{http://inspirehep.net/record/1274309}{O.~Matsedonskyi}$^{104}$,\, 
\href{http://inspirehep.net/record/1589821}{A.~Mauri}$^{74}$,\, 
\href{http://inspirehep.net/record/998164}{K.~Mazumdar}$^{87}$,\, 
\href{http://inspirehep.net/record/997762}{M.~Merk}$^{11}$,\, 
\href{http://inspirehep.net/record/997651}{A.~B.~Meyer}$^{21}$,\, 
\href{http://inspirehep.net/record/1401143}{E.~Michielin}$^{12}$,\, 
\href{http://inspirehep.net/record/997245}{G.~Mitselmakher}$^{31}$,\, 
\href{http://inspirehep.net/record/1705262}{L.~Mittnacht}$^{36}$,\, 
\href{http://inspirehep.net/record/1035221}{S.~Monteil}$^{61}$,\, 
\href{http://inspirehep.net/record/1054017}{M.~J.~Morello}$^{71, 96}$,\, 
\href{http://inspirehep.net/record/1074067}{M.~Morgenstern}$^{11, 86}$,\, 
\href{http://inspirehep.net/record/996070}{M.~Narain}$^{84}$,\, 
\href{http://inspirehep.net/record/1069385}{M.~Nardecchia}$^{26}$,\, 
\href{http://inspirehep.net/record/995939}{M.~Needham}$^{48}$,\, 
\href{http://inspirehep.net/record/1024278}{N.~Neri}$^{105, 106}$,\, 
\href{http://inspirehep.net/record/995832}{M.~Neubert}$^{107}$,\, 
\href{http://inspirehep.net/record/1057768}{S.~Neubert}$^{25}$,\, 
\href{http://inspirehep.net/record/995659}{U.~Nierste}$^{108}$,\, 
\href{http://inspirehep.net/record/995644}{J.~Nieves}$^{102}$,\, 
\href{http://inspirehep.net/record/995581}{Y.~Nir}$^{104}$,\, 
\href{http://inspirehep.net/record/995578}{A.~Nisati}$^{109, 110}$,\, 
\href{http://inspirehep.net/record/1280647}{D.~P.~O'Hanlon}$^{33}$,\, 
\href{http://inspirehep.net/record/994733}{E.~Oset}$^{102}$,\, 
\href{http://inspirehep.net/record/1096954}{P.~Owen}$^{74}$,\, 
\href{http://inspirehep.net/record/1091422}{O.~Ozcelik}$^{111, 112}$,\, 
\href{http://inspirehep.net/record/1048820}{S.~Pagan Griso}$^{113, 114}$,\, 
\href{http://inspirehep.net/record/1028707}{E.~Palencia Cortezon}$^{115}$,\, 
\href{http://inspirehep.net/record/994458}{F.~Palla}$^{71, 96}$,\, 
\href{http://inspirehep.net/record/1031278}{M.~Palutan}$^{116}$,\, 
\href{http://inspirehep.net/record/1028266}{M.~Pappagallo}$^{48}$,\, 
\href{http://inspirehep.net/record/994204}{C.~Parkes}$^{14, 26}$,\, 
\href{http://inspirehep.net/record/994113}{G.~Passaleva}$^{10, 26}$,\, 
\href{http://inspirehep.net/record/1026502}{E.~Passemar}$^{97, 117, 118}$,\, 
\href{http://inspirehep.net/record/1071750}{M.~Patel}$^{38}$,\, 
\href{http://inspirehep.net/record/1262632}{A.~Pearce}$^{26}$,\, 
\href{http://inspirehep.net/record/1067964}{K.~Pedro}$^{34}$,\, 
\href{http://inspirehep.net/record/1070699}{S.~Perazzini}$^{26}$,\, 
\href{http://inspirehep.net/record/1033012}{M.~Perfilov}$^{24}$,\, 
\href{http://inspirehep.net/record/1064078}{L.~Perrozzi}$^{93}$,\, 
\href{http://inspirehep.net/record/1262633}{L.~Pescatore}$^{119}$,\, 
\href{http://inspirehep.net/record/993634}{B.~A.~Petersen}$^{26}$,\, 
\href{http://inspirehep.net/record/993563}{A.~A.~Petrov}$^{120}$,\, 
\href{http://inspirehep.net/record/993429}{A.~Pich}$^{102}$,\, 
\href{http://inspirehep.net/record/1203356}{A.~Pilloni}$^{117, 121}$,\, 
\href{http://inspirehep.net/record/1028406}{F.~Polci}$^{2}$,\, 
\href{http://inspirehep.net/record/992800}{S.~Prelovsek}$^{70, 90, 122}$,\, 
\href{http://inspirehep.net/record/1070686}{A.~Puig Navarro}$^{74}$,\, 
\href{http://inspirehep.net/record/992614}{G.~Punzi}$^{71, 123}$,\, 
\href{http://inspirehep.net/record/1022060}{J.~Rademacker}$^{124}$,\, 
\href{http://inspirehep.net/record/1028400}{M.~Rama}$^{71}$,\, 
M.~Reboud$^{101}$,\, 
\href{http://inspirehep.net/record/1474585}{A.~Reimers}$^{80}$,\, 
\href{http://inspirehep.net/record/1057205}{P.~Reznicek}$^{125}$,\, 
\href{http://inspirehep.net/record/1064922}{D.~J.~Robinson}$^{9}$,\, 
\href{http://inspirehep.net/record/991224}{J.~L.~Rosner}$^{126}$,\, 
\href{http://inspirehep.net/record/1054727}{R.~Ruiz}$^{13, 127}$,\, 
\href{http://inspirehep.net/record/1632393}{S.~Saito}$^{87}$,\, 
\href{http://inspirehep.net/record/990298}{S.~Sarkar}$^{18}$,\, 
\href{http://inspirehep.net/record/1029853}{A.~Savin}$^{128}$,\, 
\href{http://inspirehep.net/record/1632393}{S.~Sawant}$^{87}$,\, 
\href{http://inspirehep.net/record/1270412}{S.~Schacht}$^{129}$,\, 
\href{http://inspirehep.net/record/1259433}{M.~Schlaffer}$^{104}$,\, 
\href{http://inspirehep.net/record/1064622}{A.~Schmidt}$^{83}$,\, 
\href{http://inspirehep.net/record/1074089}{B.~Schneider}$^{34}$,\, 
\href{http://inspirehep.net/record/989719}{A.~Schopper}$^{26}$,\, 
\href{http://inspirehep.net/record/989620}{M.~H.~Schune}$^{89}$,\, 
\href{http://inspirehep.net/record/1062861}{J.~Segovia}$^{130}$,\, 
\href{http://inspirehep.net/record/1039590}{M.~Selvaggi}$^{26}$,\, 
\href{http://inspirehep.net/record/1043163}{N.~Serra}$^{74}$,\, 
\href{http://inspirehep.net/record/1342183}{L.~Sestini}$^{12}$,\, 
\href{http://inspirehep.net/record/1028824}{D.~Shih}$^{131}$,\, 
\href{http://inspirehep.net/record/1078754}{R.~Silva Coutinho}$^{74}$,\, 
\href{http://inspirehep.net/record/988661}{L.~Silvestrini}$^{26, 109}$,\, 
\href{http://inspirehep.net/record/1066490}{K.~Skovpen}$^{132}$,\, 
\href{http://inspirehep.net/record/988428}{T.~Skwarnicki}$^{23}$,\, 
\href{http://inspirehep.net/record/988275}{M.~Smizanska}$^{15}$,\, 
\href{http://inspirehep.net/record/988068}{A.~Soni}$^{133}$,\, 
\href{http://inspirehep.net/record/1073494}{Y.~Soreq}$^{26, 67}$,\, 
\href{http://inspirehep.net/record/1028303}{P.~Spradlin}$^{134}$,\, 
\href{http://inspirehep.net/record/987425}{S.~Stone}$^{23}$,\, 
\href{http://inspirehep.net/record/1058096}{S.~Stracka}$^{71}$,\, 
\href{http://inspirehep.net/record/1048607}{D.~M.~Straub}$^{88}$,\, 
\href{http://inspirehep.net/record/986950}{A.~PSzczepaniak}$^{97}$,\, 
\href{http://inspirehep.net/record/986950}{A.~P.~Szczepaniak}$^{117, 118}$,\, 
\href{http://inspirehep.net/record/1071090}{Y.~Takahashi}$^{74}$,\, 
\href{http://inspirehep.net/record/986279}{F.~Teubert}$^{26}$,\, 
\href{http://inspirehep.net/record/986203}{E.~Thomas}$^{26}$,\, 
\href{http://inspirehep.net/record/986052}{V.~Tisserand}$^{61}$,\, 
\href{http://inspirehep.net/record/1032902}{S.~T'Jampens}$^{101}$,\, 
\href{http://inspirehep.net/record/1064514}{R.~Torre}$^{26, 135}$,\, 
\href{http://inspirehep.net/record/1408142}{F.~Tresoldi}$^{1}$,\, 
\href{http://inspirehep.net/record/1063955}{D.~Tsiakkouri}$^{136}$,\, 
\href{http://inspirehep.net/record/1114384}{S.~Turchikhin}$^{137}$,\, 
\href{http://inspirehep.net/record/1028311}{K.~A.~Ulmer}$^{138}$,\, 
\href{http://inspirehep.net/record/1030879}{V.~Vagnoni}$^{33}$,\, 
\href{http://inspirehep.net/record/1074150}{D.~van Dyk}$^{139}$,\, 
\href{http://inspirehep.net/record/1070629}{J.~van Tilburg}$^{11}$,\, 
\href{http://inspirehep.net/record/1070643}{S.~Vecchi}$^{27}$,\, 
\href{http://inspirehep.net/record/1071756}{R.~Venditti}$^{140}$,\, 
\href{http://inspirehep.net/record/1054622}{M.~Vesterinen}$^{22}$,\, 
\href{http://inspirehep.net/record/1032745}{J.~Virto}$^{139, 141}$,\, 
\href{http://inspirehep.net/record/1312897}{P.~Volkov}$^{24}$,\, 
\href{http://inspirehep.net/record/1367258}{G.~Vorotnikov}$^{24}$,\, 
\href{http://inspirehep.net/record/1077733}{E.~Vryonidou}$^{26}$,\, 
\href{http://inspirehep.net/record/1055327}{J.~Walder}$^{15}$,\, 
\href{http://inspirehep.net/record/984240}{W.~Walkowiak}$^{100}$,\, 
\href{http://inspirehep.net/record/1060811}{J.~Wang}$^{31}$,\, 
\href{http://inspirehep.net/record/1058694}{W.~Wang}$^{142}$,\, 
\href{http://inspirehep.net/record/1069158}{C.~Weiland}$^{13, 143}$,\, 
\href{http://inspirehep.net/record/1070636}{M.~Whitehead}$^{83}$,\, 
\href{http://inspirehep.net/record/983530}{G.~Wilkinson}$^{82}$,\, 
\href{http://inspirehep.net/record/1096938}{J.~M.~Williams}$^{144}$,\, 
\href{http://inspirehep.net/record/983483}{M.~R.~J.~Williams}$^{14}$,\, 
\href{http://inspirehep.net/record/983449}{F.~Wilson}$^{145}$,\, 
\href{http://inspirehep.net/record/1046258}{Y.~Xie}$^{146}$,\, 
\href{http://inspirehep.net/record/982785}{Z.~Yang}$^{147}$,\, 
\href{http://inspirehep.net/record/1087415}{T.~You}$^{17}$,\, 
\href{http://inspirehep.net/record/1056696}{F.~Yu}$^{36}$,\, 
C.~Zhang$^{148}$,\, 
\href{http://inspirehep.net/record/1027553}{L.~Zhang}$^{147}$,\, 
\href{http://inspirehep.net/record/1666186}{W.~Zhang}$^{84}$ 
\vspace*{1cm} 
 }
\institute{\small 
$^{1}$ \href{http://inspirehep.net/record/903239}{U. Sussex, Brighton, Dept. Phys. Astron.}, $^{2}$ \href{http://inspirehep.net/record/909603}{Paris U., IV}, $^{3}$ \href{http://inspirehep.net/record/909939}{INFN, Milan Bicocca}, $^{4}$ \href{http://inspirehep.net/record/905888}{U. Laguna, Tenerife, Dept. Phys.}, $^{5}$ \href{http://inspirehep.net/record/910482}{IAC, La Laguna}, $^{6}$ \href{http://inspirehep.net/record/902734}{U. Cincinnati, Dept. Phys.}, $^{7}$ \href{http://inspirehep.net/record/1118764}{Ohio State U., Columbus}, $^{8}$ \href{http://inspirehep.net/record/907623}{U. Cambridge, DAMTP}, $^{9}$ \href{http://inspirehep.net/record/1218068}{UC, Santa Cruz, SCIPP}, $^{10}$ \href{http://inspirehep.net/record/902880}{INFN, Florence}, $^{11}$ \href{http://inspirehep.net/record/903832}{Nikhef, Amsterdam}, $^{12}$ \href{http://inspirehep.net/record/902884}{INFN, Padua}, $^{13}$ \href{http://inspirehep.net/record/908399}{Durham U., IPPP}, $^{14}$ \href{http://inspirehep.net/record/902984}{U. Manchester, Sch. Phys. Astron.}, $^{15}$ \href{http://inspirehep.net/record/902948}{Lancaster U., Dept. Phys.}, $^{16}$ \href{http://inspirehep.net/record/903031}{ITEP, Moscow}, $^{17}$ \href{http://inspirehep.net/record/902712}{U. Cambridge, Cavendish Lab.}, $^{18}$ \href{http://inspirehep.net/record/903178}{Saha Inst. Nucl. Phys., Kolkata}, $^{19}$ \href{http://inspirehep.net/record/902671}{U. Birmingham, Sch. Phys. Astron.}, $^{20}$ \href{http://inspirehep.net/record/902774}{Tech. U., Dortmund, Dept. Phys.}, $^{21}$ \href{http://inspirehep.net/record/902770}{DESY, Hamburg}, $^{22}$ \href{http://inspirehep.net/record/903734}{U. Warwick, Dept. Phys.}, $^{23}$ \href{http://inspirehep.net/record/903242}{Syracuse U., Dept. Phys.}, $^{24}$ \href{http://inspirehep.net/record/908795}{Moscow State U., SINP}, $^{25}$ \href{http://inspirehep.net/record/902842}{U. Heidelberg, Phys.Inst.}, $^{26}$ \href{http://inspirehep.net/record/902725}{CERN, Geneva}, $^{27}$ \href{http://inspirehep.net/record/905268}{INFN, Ferrara}, $^{28}$ \href{http://inspirehep.net/record/903113}{U. Padua, Dept. Phys.}, $^{29}$ \href{http://inspirehep.net/record/902756}{IFJ, Krakow}, $^{30}$ \href{http://inspirehep.net/record/911544}{Tech. U., Munich, IAS}, $^{31}$ \href{http://inspirehep.net/record/902804}{U. Florida, Gainesville, Dept. Phys.}, $^{32}$ \href{http://inspirehep.net/record/902674}{U. Bologna, Dept. Phys.}, $^{33}$ \href{http://inspirehep.net/record/902878}{INFN, Bologna}, $^{34}$ \href{http://inspirehep.net/record/902796}{Fermilab}, $^{35}$ \href{http://inspirehep.net/record/946080}{U. Chicago}, $^{36}$ \href{http://inspirehep.net/record/1280366}{U. Mainz, PRISMA}, $^{37}$ \href{http://inspirehep.net/record/903038}{LMU Munich, Dept. Phys.}, $^{38}$ \href{http://inspirehep.net/record/902868}{Imperial Coll., London, Dept. Phys.}, $^{39}$ \href{http://inspirehep.net/record/902988}{CPT, Marseille}, $^{40}$ \href{http://inspirehep.net/record/904661}{Natl. Taiwan U., Taipei, Phys. Dept.}, $^{41}$ \href{http://inspirehep.net/record/910711}{U. Santiago de Compostela, IGFAE}, $^{42}$ \href{http://inspirehep.net/record/1268260}{Los Alamos Natl. Lab., Theor. Div.}, $^{43}$ \href{http://inspirehep.net/record/907692}{INFN, Rome 3}, $^{44}$ \href{http://inspirehep.net/record/902989}{CPPM, Marseille}, $^{45}$ \href{http://inspirehep.net/record/1124249}{U. Bern, AEC}, $^{46}$ \href{http://inspirehep.net/record/904735}{INFN, Cagliari}, $^{47}$ \href{http://inspirehep.net/record/902889}{INFN, Turin}, $^{48}$ \href{http://inspirehep.net/record/902787}{U. Edinburgh, Sch. Phys. Astron.}, $^{49}$ \href{http://inspirehep.net/record/905405}{PSI, Villigen}, $^{50}$ \href{http://inspirehep.net/record/902883}{INFN, Naples}, $^{51}$ \href{http://inspirehep.net/record/903537}{U. Louisville, Dept. Phys.}, $^{52}$ \href{http://inspirehep.net/record/902968}{Los Alamos Natl. Lab.}, $^{53}$ \href{http://inspirehep.net/record/903305}{UC, San Diego, Dept. Phys.}, $^{54}$ \href{http://inspirehep.net/record/903006}{Michigan State U., East Lansing, Dept. Phys. Astron.}, $^{55}$ \href{http://inspirehep.net/record/1625414}{IRFU, Saclay, DPP}, $^{56}$ \href{http://inspirehep.net/record/911713}{U. Southern Denmark, CP3-Origins}, $^{57}$ \href{http://inspirehep.net/record/903357}{Yale U., Dept. Phys.}, $^{58}$ \href{http://inspirehep.net/record/912370}{Natl. Res. U. Higher Sch. Econ., Moscow}, $^{59}$ \href{http://inspirehep.net/record/904475}{U. Rome 2, Tor Vergata, Dept. Phys.}, $^{60}$ \href{http://inspirehep.net/record/907691}{INFN Rome Tor Vergata}, $^{61}$ \href{http://inspirehep.net/record/902740}{LPC, Clermont-Ferrand}, $^{62}$ \href{http://inspirehep.net/record/903101}{LPT, Orsay}, $^{63}$ \href{http://inspirehep.net/record/902964}{U. Liverpool, Dept. Phys.}, $^{64}$ \href{http://inspirehep.net/record/907960}{U. Milan Bicocca, Dept. Phys.}, $^{65}$ \href{http://inspirehep.net/record/902888}{INFN, Trieste}, $^{66}$ \href{http://inspirehep.net/record/903158}{CBPF, Rio de Janeiro}, $^{67}$ \href{http://inspirehep.net/record/903257}{Technion, IIT, Dept. Phys.}, $^{68}$ Theoretical Particle Physics Laboratory (LPTP), Institute of Physics, EPFL, Lausanne, Switzerland, $^{69}$ \href{http://inspirehep.net/record/1677475}{ICAS, UNSAM, Buenos Aires}, $^{70}$ \href{http://inspirehep.net/record/903408}{J. Stefan Inst., Ljubljana}, $^{71}$ \href{http://inspirehep.net/record/902886}{INFN, Pisa}, $^{72}$ \href{http://inspirehep.net/record/903097}{U. Oregon, Eugene, Dept. Phys.}, $^{73}$ \href{http://inspirehep.net/record/909078}{U. Granada}, $^{74}$ \href{http://inspirehep.net/record/903370}{U. Zurich, Phys. Inst.}, $^{75}$ \href{http://inspirehep.net/record/908524}{Beihang U.}, $^{76}$ \href{http://inspirehep.net/record/909080}{Cornell U., LEPP}, $^{77}$ \href{http://inspirehep.net/record/908074}{LAPTH, Annecy}, $^{78}$ \href{http://inspirehep.net/record/903895}{CAS, ITP, Beijing}, $^{79}$ \href{http://inspirehep.net/record/904611}{UCAS, Beijing}, $^{80}$ \href{http://inspirehep.net/record/902832}{U. Hamburg, Inst. Exp. Phys.}, $^{81}$ \href{http://inspirehep.net/record/902990}{U. Maryland, College Park, Dept. Phys.}, $^{82}$ \href{http://inspirehep.net/record/903112}{U. Oxford, Part. Phys. Dept.}, $^{83}$ \href{http://inspirehep.net/record/910724}{RWTH Aachen}, $^{84}$ \href{http://inspirehep.net/record/902692}{Brown U., Dept. Phys.}, $^{85}$ \href{http://inspirehep.net/record/1241132}{Bethel Coll.}, $^{86}$ \href{http://inspirehep.net/record/902640}{Amsterdam U. (main)}, $^{87}$ \href{http://inspirehep.net/record/1120892}{TIFR, Mumbai, DHEP}, $^{88}$ \href{http://inspirehep.net/record/911565}{Excel. Cluster Universe, Munich}, $^{89}$ \href{http://inspirehep.net/record/903100}{LAL, Orsay}, $^{90}$ \href{http://inspirehep.net/record/903532}{U. Ljubljana, Fac. Math. Phys.}, $^{91}$ \href{http://inspirehep.net/record/903259}{Tel-Aviv U., Dept. Part. Phys.}, $^{92}$ \href{http://inspirehep.net/record/1225433}{Nagoya U.}, $^{93}$ \href{http://inspirehep.net/record/903369}{ETH, Zurich, Dept. Phys.}, $^{94}$ \href{http://inspirehep.net/record/902646}{Arizona State U., Tempe, Dept. Phys. Astron.}, $^{95}$ \href{http://inspirehep.net/record/1210798}{Peking U., Beijing, SKLNPT}, $^{96}$ \href{http://inspirehep.net/record/903128}{Scuola Normale Superiore, Pisa}, $^{97}$ \href{http://inspirehep.net/record/902874}{Indiana U., Bloomington, Dept. Phys.}, $^{98}$ \href{http://inspirehep.net/record/902974}{IPNL, Lyon}, $^{99}$ \href{http://inspirehep.net/record/903194}{IHEP, Serpukhov}, $^{100}$ \href{http://inspirehep.net/record/903203}{U. Siegen, Dept. Phys.}, $^{101}$ \href{http://inspirehep.net/record/903421}{LAPP, Annecy}, $^{102}$ \href{http://inspirehep.net/record/907907}{IFIC, Valencia}, $^{103}$ \href{http://inspirehep.net/record/907904}{U. Barcelona, IFAE}, $^{104}$ \href{http://inspirehep.net/record/903342}{Weizmann Inst. Sci., Rehovot, Fac. Phys.}, $^{105}$ \href{http://inspirehep.net/record/902882}{INFN, Milan}, $^{106}$ \href{http://inspirehep.net/record/903009}{U. Milan, Dept. Phys.}, $^{107}$ \href{http://inspirehep.net/record/911853}{U. Mainz}, $^{108}$ \href{http://inspirehep.net/record/911469}{KIT, Karlsruhe}, $^{109}$ \href{http://inspirehep.net/record/902887}{INFN, Rome 1}, $^{110}$ \href{http://inspirehep.net/record/903168}{U. Rome 1, La Sapienza, Dept. Phys.}, $^{111}$ \href{http://inspirehep.net/record/903448}{Bogazici U., Istanbul, Dept. Phys.}, $^{112}$ \href{http://inspirehep.net/record/905303}{LIP, Lisbon}, $^{113}$ \href{http://inspirehep.net/record/902953}{LBNL, Berkeley, Div. Phys.}, $^{114}$ \href{http://inspirehep.net/record/903299}{UC, Berkeley, Dept. Phys.}, $^{115}$ \href{http://inspirehep.net/record/906106}{U. Oviedo, Dept. Phys.}, $^{116}$ \href{http://inspirehep.net/record/902807}{INFN, LNF, Frascati}, $^{117}$ \href{http://inspirehep.net/record/904961}{JLab, Newport News}, $^{118}$ \href{http://inspirehep.net/record/912374}{Indiana U., Bloomington, CEEM}, $^{119}$ \href{http://inspirehep.net/record/903523}{EPFL, Lausanne, LPHE}, $^{120}$ \href{http://inspirehep.net/record/903341}{Wayne State U., Detroit, Dept. Phys. Astron.}, $^{121}$ \href{http://inspirehep.net/record/906718}{ECT, Trento}, $^{122}$ \href{http://inspirehep.net/record/903753}{U. Regensburg, Dept. Phys.}, $^{123}$ \href{http://inspirehep.net/record/903129}{U. Pisa, Dept. Phys.}, $^{124}$ \href{http://inspirehep.net/record/902687}{U. Bristol, Wills Phys. Lab.}, $^{125}$ \href{http://inspirehep.net/record/902727}{Charles U., Prague, Inst. Part. Nucl. Phys.}, $^{126}$ \href{http://inspirehep.net/record/902730}{Chicago U., EFI}, $^{127}$ \href{http://inspirehep.net/record/910783}{Cathol. U. Louvain, CP3}, $^{128}$ \href{http://inspirehep.net/record/903349}{U. Wisconsin, Madison,  Dept. Phys.}, $^{129}$ \href{http://inspirehep.net/record/907520}{Cornell U., Dept. Phys.}, $^{130}$ \href{http://inspirehep.net/record/910794}{Pablo de Olavide U., Seville}, $^{131}$ \href{http://inspirehep.net/record/903404}{Rutgers U., Piscataway, Dept. Phys. Astron.}, $^{132}$ \href{http://inspirehep.net/record/907933}{Vrije U. Brussels, Dept. Phys. Astrophys.}, $^{133}$ \href{http://inspirehep.net/record/902689}{Brookhaven Natl. Lab., Dept. Phys.}, $^{134}$ \href{http://inspirehep.net/record/902823}{U. Glasgow, Sch. Phys. Astron.}, $^{135}$ \href{http://inspirehep.net/record/902881}{INFN, Genoa}, $^{136}$ \href{http://inspirehep.net/record/906319}{U. Cyprus, Nicosia, Dept. Phys.}, $^{137}$ \href{http://inspirehep.net/record/902780}{JINR, Dubna}, $^{138}$ \href{http://inspirehep.net/record/902748}{U. Colorado, Boulder, Dept. Phys.}, $^{139}$ \href{http://inspirehep.net/record/903037}{Tech. U., Munich, Dept. Phys.}, $^{140}$ \href{http://inspirehep.net/record/902877}{INFN, Bari}, $^{141}$ \href{http://inspirehep.net/record/1237813}{MIT, Cambridge, CTP}, $^{142}$ \href{http://inspirehep.net/record/1083225}{Jiao Tong U., Shanghai, INPAC}, $^{143}$ \href{http://inspirehep.net/record/903130}{U. Pittsburgh, Dept. Phys. Astron.}, $^{144}$ \href{http://inspirehep.net/record/907455}{MIT, Cambridge}, $^{145}$ \href{http://inspirehep.net/record/903174}{RAL, Didcot}, $^{146}$ \href{http://inspirehep.net/record/1279835}{CCNU, Wuhan, Inst. Part. Phys.}, $^{147}$ \href{http://inspirehep.net/record/1205356}{Tsinghua U., Beijing, CHEP}, $^{148}$ \href{http://inspirehep.net/record/903123}{CAS, IHEP, Beijing},  }\normalsize
\end{center}

\vspace{\fill}

% Abstract ------
    \begin{abstract}
  %\noindent
  Motivated by the success of the flavour physics programme carried out over the last decade at the Large Hadron Collider (LHC), we characterize in detail the physics potential of its High-Luminosity and High-Energy upgrades in this domain of physics. We document the extraordinary breadth of the HL/HE-LHC programme enabled by a putative Upgrade II of the dedicated flavour physics experiment LHCb and the evolution of the established flavour physics role of the ATLAS and CMS general purpose experiments. We connect the dedicated flavour physics program to studies of the top quark, Higgs boson, and direct high-$p_T$ searches for new particles and force carriers. We discuss the complementarity of their discovery potential for physics beyond the Standard Model, affirming the necessity to fully exploit the LHC's flavour physics potential throughout its upgrade eras.
  %Render unto Caesar what is Caesar's. Argh, et tu Brute?
\end{abstract}

\vspace*{2.0cm}
\vspace{\fill}
\author{Convenors: \\ 
\href{http://inspirehep.net/record/1021757}{A.~Cerri}$^{1}$,\, 
\href{http://inspirehep.net/record/1057204}{V.~V.~Gligorov}$^{2}$,\, 
\href{http://inspirehep.net/record/999046}{S.~Malvezzi}$^{3}$,\, 
\href{http://inspirehep.net/record/1056267}{J.~Martin Camalich}$^{4, 5}$,\, 
\href{http://inspirehep.net/record/1020959}{J.~Zupan}$^{6}$,\, 
\href{http://inspirehep.net/record/1073335}{S.~Akar}$^{6}$,\, 
\\ \vspace*{4mm} 
 Contributors: 
 \\ 
\href{http://inspirehep.net/record/1068305}{J.~Alimena}$^{7}$,\, 
\href{http://inspirehep.net/record/1018633}{B.~C.~Allanach}$^{8}$,\, 
\href{http://inspirehep.net/record/1048366}{W.~Altmannshofer}$^{9}$,\, 
\href{http://inspirehep.net/record/1189488}{L.~Anderlini}$^{10}$,\, 
\href{http://inspirehep.net/record/1057775}{F.~Archilli}$^{11}$,\, 
\href{http://inspirehep.net/record/1017778}{P.~Azzi}$^{12}$,\, 
\href{http://inspirehep.net/record/1277169}{S.~Banerjee}$^{13}$,\, 
\href{http://inspirehep.net/record/1079058}{W.~Barter}$^{14}$,\, 
\href{http://inspirehep.net/record/1071810}{A.~E.~Barton}$^{15}$,\, 
\href{http://inspirehep.net/record/1070828}{I.~Belyaev}$^{16}$,\, 
\href{http://inspirehep.net/record/1096983}{S.~Benson}$^{11}$,\, 
\href{http://inspirehep.net/record/1059727}{M.~Bettler}$^{17}$,\, 
\href{http://inspirehep.net/record/1311547}{R.~Bhattacharya}$^{18}$,\, 
\href{http://inspirehep.net/record/1060779}{S.~Bifani}$^{19}$,\, 
\href{http://inspirehep.net/record/1378053}{A.~Birnkraut}$^{20}$,\, 
\href{http://inspirehep.net/record/1279838}{F.~Bishara}$^{21}$,\, 
\href{http://inspirehep.net/record/1070826}{T.~Blake}$^{22}$,\, 
\href{http://inspirehep.net/record/1016078}{S.~Blusk}$^{23}$,\, 
\href{http://inspirehep.net/record/1015816}{E.~Boos}$^{24}$,\, 
\href{http://inspirehep.net/record/1262186}{M.~Borsato}$^{25}$,\, 
\href{http://inspirehep.net/record/1015512}{C.~Bozzi}$^{26, 27}$,\, 
\href{http://inspirehep.net/record/1609167}{ABragagnolo}$^{12, 28}$,\, 
\href{http://inspirehep.net/record/1056126}{J.~Brod}$^{6}$,\, 
\href{http://inspirehep.net/record/1031499}{J.~Brodzicka}$^{29}$,\, 
\href{http://inspirehep.net/record/1014978}{A.~J.~Buras}$^{30}$,\, 
\href{http://inspirehep.net/record/1321289}{L.~Cadamuro}$^{31}$,\, 
\href{http://inspirehep.net/record/1061899}{A.~Carbone}$^{32, 33}$,\, 
\href{http://inspirehep.net/record/984325}{M.~Carena}$^{34, 35}$,\, 
\href{http://inspirehep.net/record/1052521}{A.~Carmona}$^{36}$,\, 
\href{http://inspirehep.net/record/1014241}{F.~R.~Cavallo}$^{33}$,\, 
\href{http://inspirehep.net/record/1189874}{A.~Celis}$^{37}$,\, 
\href{http://inspirehep.net/record/1064384}{M.~Cepeda}$^{26}$,\, 
\href{http://inspirehep.net/record/1071755}{G.~S.~Chahal}$^{13, 38}$,\, 
\href{http://inspirehep.net/record/1272128}{M.~Chala}$^{13}$,\, 
\href{http://inspirehep.net/record/1013990}{J.~Charles}$^{39}$,\, 
\href{http://inspirehep.net/record/1020539}{M.~Charles}$^{2}$,\, 
\href{http://inspirehep.net/record/1031500}{K.~F.~Chen}$^{40}$,\, 
\href{http://inspirehep.net/record/1223006}{V.~Chobanova}$^{41}$,\, 
\href{http://inspirehep.net/record/1096980}{M.~Chrzaszcz}$^{26}$,\, 
\href{http://inspirehep.net/record/1079012}{G.~Ciezarek}$^{26}$,\, 
\href{http://inspirehep.net/record/1013451}{V.~Cirigliano}$^{42}$,\, 
\href{http://inspirehep.net/record/1013445}{M.~Ciuchini}$^{43}$,\, 
\href{http://inspirehep.net/record/1070807}{H.~Cliff}$^{17}$,\, 
\href{http://inspirehep.net/record/1013314}{J.~Cogan}$^{44}$,\, 
\href{http://inspirehep.net/record/1013274}{G.~Colangelo}$^{45}$,\, 
\href{http://inspirehep.net/record/1069963}{A.~Contu}$^{46}$,\, 
\href{http://inspirehep.net/record/1028393}{R.~Covarelli}$^{47}$,\, 
\href{http://inspirehep.net/record/1023478}{G.~Cowan}$^{48}$,\, 
\href{http://inspirehep.net/record/1060042}{A.~Crivellin}$^{49}$,\, 
\href{http://inspirehep.net/record/1012622}{G.~D'Ambrosio}$^{50}$,\, 
\href{http://inspirehep.net/record/1293755}{N.~P.~Dang}$^{51}$,\, 
\href{http://inspirehep.net/record/1262261}{A.~Davis}$^{14}$,\, 
\href{http://inspirehep.net/record/1096978}{K.~De Bruyn}$^{26}$,\, 
\href{http://inspirehep.net/record/1224351}{W.~Dekens}$^{52, 53}$,\, 
\href{http://inspirehep.net/record/1072922}{H.~De la Torre}$^{54}$,\, 
\href{http://inspirehep.net/record/1012113}{F.~Deliot}$^{55}$,\, 
\href{http://inspirehep.net/record/996726}{M.~Della Morte}$^{56}$,\, 
\href{http://inspirehep.net/record/1021058}{S.~Demers}$^{57}$,\, 
\href{http://inspirehep.net/record/1061025}{D.~Derkach}$^{58}$,\, 
\href{http://inspirehep.net/record/1028949}{U.~De Sanctis}$^{59, 60}$,\, 
\href{http://inspirehep.net/record/1066064}{O.~Deschamps}$^{61}$,\, 
\href{http://inspirehep.net/record/1011958}{S.~Descotes-Genon}$^{62}$,\, 
\href{http://inspirehep.net/record/1070796}{F.~Dettori}$^{63}$,\, 
\href{http://inspirehep.net/record/1054949}{A.~Di Canto}$^{26}$,\, 
\href{http://inspirehep.net/record/1023527}{M.~Dinardo}$^{3, 64}$,\, 
\href{http://inspirehep.net/record/1021779}{P.~Dini}$^{3}$,\, 
\href{http://inspirehep.net/record/1028663}{M.~D'Onofrio}$^{63}$,\, 
\href{http://inspirehep.net/record/1096976}{F.~Dordei}$^{46}$,\, 
\href{http://inspirehep.net/record/1060820}{M.~Dorigo}$^{26, 65}$,\, 
\href{http://inspirehep.net/record/1011495}{A.~dos Reis}$^{66}$,\, 
\href{http://inspirehep.net/record/1011294}{L.~Dudko}$^{24}$,\, 
\href{http://inspirehep.net/record/1401131}{L.~Dufour}$^{11}$,\, 
\href{http://inspirehep.net/record/1273362}{G.~Durieux}$^{21, 67}$,\, 
\href{http://inspirehep.net/record/1011136}{S.~Dutta}$^{18}$,\, 
\href{http://inspirehep.net/record/1096974}{A.~Dziurda}$^{29}$,\, 
\href{http://inspirehep.net/record/1233187}{A.~Esposito}$^{68}$,\, 
M.~Estevez$^{69}$,\, 
\href{http://inspirehep.net/record/1498216}{D.~A.~Faroughy}$^{70}$,\, 
\href{http://inspirehep.net/record/1073229}{G.~Fedi}$^{71}$,\, 
\href{http://inspirehep.net/record/1080244}{S.~Fiorendi}$^{3, 64}$,\, 
\href{http://inspirehep.net/record/1064681}{F.~Fiori}$^{71}$,\, 
\href{http://inspirehep.net/record/1071746}{C.~Fitzpatrick}$^{26}$,\, 
\href{http://inspirehep.net/record/1009833}{R.~Fleischer}$^{11}$,\, 
\href{http://inspirehep.net/record/1078979}{M.~Fontana}$^{26}$,\, 
\href{http://inspirehep.net/record/1009609}{P.~J.~Fox}$^{34}$,\, 
\href{http://inspirehep.net/record/1074607}{M.~Freytsis}$^{72}$,\, 
E.~Gabriel$^{48}$,\, 
\href{http://inspirehep.net/record/1008976}{P.~Gambino}$^{47}$,\, 
\href{http://inspirehep.net/record/1008971}{E.~G\'amiz}$^{73}$,\, 
\href{http://inspirehep.net/record/1341934}{J.~Garc{\'\i}a Pardi{\~n}as}$^{74}$,\, 
\href{http://inspirehep.net/record/1279168}{L.~S.~Geng}$^{75}$,\, 
\href{http://inspirehep.net/record/1064668}{E.~Gersabeck}$^{14}$,\, 
\href{http://inspirehep.net/record/1058420}{M.~Gersabeck}$^{14}$,\, 
\href{http://inspirehep.net/record/1027525}{T.~Gershon}$^{22}$,\, 
\href{http://inspirehep.net/record/1008386}{A.~Gilbert}$^{26}$,\, 
\href{http://inspirehep.net/record/1058698}{M.~Gonzalez-Alonso}$^{26}$,\, 
\href{http://inspirehep.net/record/1062192}{P.~Govoni}$^{3}$,\, 
\href{http://inspirehep.net/record/1007670}{G.~Graziani}$^{10}$,\, 
\href{http://inspirehep.net/record/1198373}{A.~Greljo}$^{26}$,\, 
\href{http://inspirehep.net/record/1074923}{L.~Grillo}$^{14}$,\, 
\href{http://inspirehep.net/record/1007511}{B.~Grinstein}$^{53}$,\, 
\href{http://inspirehep.net/record/1050762}{A.~Grohsjean}$^{21}$,\, 
\href{http://inspirehep.net/record/1007437}{Y.~Grossman}$^{76}$,\, 
\href{http://inspirehep.net/record/1032647}{D.~Guadagnoli}$^{77}$,\, 
\href{http://inspirehep.net/record/1027513}{F.~-K.~Guo}$^{78, 79}$,\, 
\href{http://inspirehep.net/record/1700541}{L.~Guzzi}$^{3, 64}$,\, 
\href{http://inspirehep.net/record/1019892}{J.~Haller}$^{80}$,\, 
\href{http://inspirehep.net/record/1073314}{B.~Hamilton}$^{81}$,\, 
\href{http://inspirehep.net/record/1019568}{R.~Harnik}$^{34}$,\, 
\href{http://inspirehep.net/record/1189481}{D.~Hill}$^{82}$,\, 
\href{http://inspirehep.net/record/1005991}{G.~Hiller}$^{20}$,\, 
\href{http://inspirehep.net/record/1005812}{K.~Hoepfner}$^{83}$,\, 
\href{http://inspirehep.net/record/1068037}{J.~M.~Hogan}$^{84, 85}$,\, 
\href{http://inspirehep.net/record/1005239}{T.~Hurth}$^{36}$,\, 
\href{http://inspirehep.net/record/1019801}{O.~Igonkina}$^{11, 86}$,\, 
\href{http://inspirehep.net/record/1070755}{P.~Ilten}$^{19}$,\, 
\href{http://inspirehep.net/record/1004578}{S.~J{\"a}ger}$^{1}$,\, 
\href{http://inspirehep.net/record/1064540}{S.~Jain}$^{87}$,\, 
\href{http://inspirehep.net/record/1028390}{M.~John}$^{82}$,\, 
\href{http://inspirehep.net/record/1071815}{D.~Johnson}$^{26}$,\, 
\href{http://inspirehep.net/record/1258549}{M.~Jung}$^{88}$,\, 
\href{http://inspirehep.net/record/1280642}{N.~Jurik}$^{82}$,\, 
\href{http://inspirehep.net/record/1003971}{M.~Kado}$^{89}$,\, 
\href{http://inspirehep.net/record/1003952}{A.~L.~Kagan}$^{6}$,\, 
\href{http://inspirehep.net/record/1033909}{J.~F.~Kamenik}$^{70, 90}$,\, 
\href{http://inspirehep.net/record/1003603}{M.~Karliner}$^{91}$,\, 
\href{http://inspirehep.net/record/1073117}{M.~Kenzie}$^{17}$,\, 
\href{http://inspirehep.net/record/1070746}{B.~Khanji}$^{26}$,\, 
\href{http://inspirehep.net/record/1123712}{J.~Kieseler}$^{26}$,\, 
\href{http://inspirehep.net/record/1274068}{T.~Kitahara}$^{92}$,\, 
\href{http://inspirehep.net/record/1396923}{T.~Klijnsma}$^{93}$,\, 
\href{http://inspirehep.net/record/1002577}{M.~Knecht}$^{39}$,\, 
\href{http://inspirehep.net/record/1057492}{R.~Kogler}$^{80}$,\, 
\href{http://inspirehep.net/record/1019544}{P.~Koppenburg}$^{11}$,\, 
\href{http://inspirehep.net/record/1002130}{A.~Korytov}$^{31}$,\, 
\href{http://inspirehep.net/record/1033910}{N.~Ko\v snik}$^{70, 90}$,\, 
\href{http://inspirehep.net/record/1028692}{M.~Kreps}$^{22}$,\, 
\href{http://inspirehep.net/record/1058264}{C.~Langenbruch}$^{83}$,\, 
\href{http://inspirehep.net/record/1001114}{U.~Langenegger}$^{49}$,\, 
\href{http://inspirehep.net/record/1028434}{T.~Latham}$^{22}$,\, 
\href{http://inspirehep.net/record/1000869}{R.~F.~Lebed}$^{94}$,\, 
\href{http://inspirehep.net/record/1000579}{A.~J.~Lenz}$^{13}$,\, 
\href{http://inspirehep.net/record/1000515}{O.~Leroy}$^{44}$,\, 
\href{http://inspirehep.net/record/1074984}{Q.~Li}$^{95}$,\, 
\href{http://inspirehep.net/record/1000247}{F.~Ligabue}$^{71, 96}$,\, 
\href{http://inspirehep.net/record/999554}{E.~Lunghi}$^{97}$,\, 
\href{http://inspirehep.net/record/1028960}{F.~Mahmoudi}$^{98}$,\, 
\href{http://inspirehep.net/record/1028307}{G.~Mancinelli}$^{44}$,\, 
\href{http://inspirehep.net/record/1395721}{P.~Mandrik}$^{99}$,\, 
\href{http://inspirehep.net/record/998950}{T.~Mannel}$^{100}$,\, 
\href{http://inspirehep.net/record/1066755}{J.~F.~Marchand}$^{101}$,\, 
\href{http://inspirehep.net/record/1061000}{M.~Martinelli}$^{26}$,\, 
\href{http://inspirehep.net/record/1070724}{D.~Mart\'inez Santos}$^{41}$,\, 
\href{http://inspirehep.net/record/1070724}{D.~Martinez Santos}$^{41}$,\, 
\href{http://inspirehep.net/record/998598}{F.~Martinez Vidal}$^{102}$,\, 
\href{http://inspirehep.net/record/1078065}{D.~Marzocca}$^{65}$,\, 
\href{http://inspirehep.net/record/998391}{J.~Matias}$^{103}$,\, 
\href{http://inspirehep.net/record/1642114}{P.~Matorras Cuevas}$^{74}$,\, 
\href{http://inspirehep.net/record/1274309}{O.~Matsedonskyi}$^{104}$,\, 
\href{http://inspirehep.net/record/1589821}{A.~Mauri}$^{74}$,\, 
\href{http://inspirehep.net/record/998164}{K.~Mazumdar}$^{87}$,\, 
\href{http://inspirehep.net/record/997762}{M.~Merk}$^{11}$,\, 
\href{http://inspirehep.net/record/997651}{A.~B.~Meyer}$^{21}$,\, 
\href{http://inspirehep.net/record/1401143}{E.~Michielin}$^{12}$,\, 
\href{http://inspirehep.net/record/997245}{G.~Mitselmakher}$^{31}$,\, 
\href{http://inspirehep.net/record/1705262}{L.~Mittnacht}$^{36}$,\, 
\href{http://inspirehep.net/record/1035221}{S.~Monteil}$^{61}$,\, 
\href{http://inspirehep.net/record/1054017}{M.~J.~Morello}$^{71, 96}$,\, 
\href{http://inspirehep.net/record/1074067}{M.~Morgenstern}$^{11, 86}$,\, 
\href{http://inspirehep.net/record/996070}{M.~Narain}$^{84}$,\, 
\href{http://inspirehep.net/record/1069385}{M.~Nardecchia}$^{26}$,\, 
\href{http://inspirehep.net/record/995939}{M.~Needham}$^{48}$,\, 
\href{http://inspirehep.net/record/1024278}{N.~Neri}$^{105, 106}$,\, 
\href{http://inspirehep.net/record/995832}{M.~Neubert}$^{107}$,\, 
\href{http://inspirehep.net/record/1057768}{S.~Neubert}$^{25}$,\, 
\href{http://inspirehep.net/record/995659}{U.~Nierste}$^{108}$,\, 
\href{http://inspirehep.net/record/995644}{J.~Nieves}$^{102}$,\, 
\href{http://inspirehep.net/record/995581}{Y.~Nir}$^{104}$,\, 
\href{http://inspirehep.net/record/995578}{A.~Nisati}$^{109, 110}$,\, 
\href{http://inspirehep.net/record/1280647}{D.~P.~O'Hanlon}$^{33}$,\, 
\href{http://inspirehep.net/record/994733}{E.~Oset}$^{102}$,\, 
\href{http://inspirehep.net/record/1096954}{P.~Owen}$^{74}$,\, 
\href{http://inspirehep.net/record/1091422}{O.~Ozcelik}$^{111, 112}$,\, 
\href{http://inspirehep.net/record/1048820}{S.~Pagan Griso}$^{113, 114}$,\, 
\href{http://inspirehep.net/record/1028707}{E.~Palencia Cortezon}$^{115}$,\, 
\href{http://inspirehep.net/record/994458}{F.~Palla}$^{71, 96}$,\, 
\href{http://inspirehep.net/record/1031278}{M.~Palutan}$^{116}$,\, 
\href{http://inspirehep.net/record/1028266}{M.~Pappagallo}$^{48}$,\, 
\href{http://inspirehep.net/record/994204}{C.~Parkes}$^{14, 26}$,\, 
\href{http://inspirehep.net/record/994113}{G.~Passaleva}$^{10, 26}$,\, 
\href{http://inspirehep.net/record/1026502}{E.~Passemar}$^{97, 117, 118}$,\, 
\href{http://inspirehep.net/record/1071750}{M.~Patel}$^{38}$,\, 
\href{http://inspirehep.net/record/1262632}{A.~Pearce}$^{26}$,\, 
\href{http://inspirehep.net/record/1067964}{K.~Pedro}$^{34}$,\, 
\href{http://inspirehep.net/record/1070699}{S.~Perazzini}$^{26}$,\, 
\href{http://inspirehep.net/record/1033012}{M.~Perfilov}$^{24}$,\, 
\href{http://inspirehep.net/record/1064078}{L.~Perrozzi}$^{93}$,\, 
\href{http://inspirehep.net/record/1262633}{L.~Pescatore}$^{119}$,\, 
\href{http://inspirehep.net/record/993634}{B.~A.~Petersen}$^{26}$,\, 
\href{http://inspirehep.net/record/993563}{A.~A.~Petrov}$^{120}$,\, 
\href{http://inspirehep.net/record/993429}{A.~Pich}$^{102}$,\, 
\href{http://inspirehep.net/record/1203356}{A.~Pilloni}$^{117, 121}$,\, 
\href{http://inspirehep.net/record/1028406}{F.~Polci}$^{2}$,\, 
\href{http://inspirehep.net/record/992800}{S.~Prelovsek}$^{70, 90, 122}$,\, 
\href{http://inspirehep.net/record/1070686}{A.~Puig Navarro}$^{74}$,\, 
\href{http://inspirehep.net/record/992614}{G.~Punzi}$^{71, 123}$,\, 
\href{http://inspirehep.net/record/1022060}{J.~Rademacker}$^{124}$,\, 
\href{http://inspirehep.net/record/1028400}{M.~Rama}$^{71}$,\, 
M.~Reboud$^{101}$,\, 
\href{http://inspirehep.net/record/1474585}{A.~Reimers}$^{80}$,\, 
\href{http://inspirehep.net/record/1057205}{P.~Reznicek}$^{125}$,\, 
\href{http://inspirehep.net/record/1064922}{D.~J.~Robinson}$^{9}$,\, 
\href{http://inspirehep.net/record/991224}{J.~L.~Rosner}$^{126}$,\, 
\href{http://inspirehep.net/record/1054727}{R.~Ruiz}$^{13, 127}$,\, 
\href{http://inspirehep.net/record/1632393}{S.~Saito}$^{87}$,\, 
\href{http://inspirehep.net/record/990298}{S.~Sarkar}$^{18}$,\, 
\href{http://inspirehep.net/record/1029853}{A.~Savin}$^{128}$,\, 
\href{http://inspirehep.net/record/1632393}{S.~Sawant}$^{87}$,\, 
\href{http://inspirehep.net/record/1270412}{S.~Schacht}$^{129}$,\, 
\href{http://inspirehep.net/record/1259433}{M.~Schlaffer}$^{104}$,\, 
\href{http://inspirehep.net/record/1064622}{A.~Schmidt}$^{83}$,\, 
\href{http://inspirehep.net/record/1074089}{B.~Schneider}$^{34}$,\, 
\href{http://inspirehep.net/record/989719}{A.~Schopper}$^{26}$,\, 
\href{http://inspirehep.net/record/989620}{M.~H.~Schune}$^{89}$,\, 
\href{http://inspirehep.net/record/1062861}{J.~Segovia}$^{130}$,\, 
\href{http://inspirehep.net/record/1039590}{M.~Selvaggi}$^{26}$,\, 
\href{http://inspirehep.net/record/1043163}{N.~Serra}$^{74}$,\, 
\href{http://inspirehep.net/record/1342183}{L.~Sestini}$^{12}$,\, 
\href{http://inspirehep.net/record/1028824}{D.~Shih}$^{131}$,\, 
\href{http://inspirehep.net/record/1078754}{R.~Silva Coutinho}$^{74}$,\, 
\href{http://inspirehep.net/record/988661}{L.~Silvestrini}$^{26, 109}$,\, 
\href{http://inspirehep.net/record/1066490}{K.~Skovpen}$^{132}$,\, 
\href{http://inspirehep.net/record/988428}{T.~Skwarnicki}$^{23}$,\, 
\href{http://inspirehep.net/record/988275}{M.~Smizanska}$^{15}$,\, 
\href{http://inspirehep.net/record/988068}{A.~Soni}$^{133}$,\, 
\href{http://inspirehep.net/record/1073494}{Y.~Soreq}$^{26, 67}$,\, 
\href{http://inspirehep.net/record/1028303}{P.~Spradlin}$^{134}$,\, 
\href{http://inspirehep.net/record/987425}{S.~Stone}$^{23}$,\, 
\href{http://inspirehep.net/record/1058096}{S.~Stracka}$^{71}$,\, 
\href{http://inspirehep.net/record/1048607}{D.~M.~Straub}$^{88}$,\, 
\href{http://inspirehep.net/record/986950}{A.~PSzczepaniak}$^{97}$,\, 
\href{http://inspirehep.net/record/986950}{A.~P.~Szczepaniak}$^{117, 118}$,\, 
\href{http://inspirehep.net/record/1071090}{Y.~Takahashi}$^{74}$,\, 
\href{http://inspirehep.net/record/986279}{F.~Teubert}$^{26}$,\, 
\href{http://inspirehep.net/record/986203}{E.~Thomas}$^{26}$,\, 
\href{http://inspirehep.net/record/986052}{V.~Tisserand}$^{61}$,\, 
\href{http://inspirehep.net/record/1032902}{S.~T'Jampens}$^{101}$,\, 
\href{http://inspirehep.net/record/1064514}{R.~Torre}$^{26, 135}$,\, 
\href{http://inspirehep.net/record/1408142}{F.~Tresoldi}$^{1}$,\, 
\href{http://inspirehep.net/record/1063955}{D.~Tsiakkouri}$^{136}$,\, 
\href{http://inspirehep.net/record/1114384}{S.~Turchikhin}$^{137}$,\, 
\href{http://inspirehep.net/record/1028311}{K.~A.~Ulmer}$^{138}$,\, 
\href{http://inspirehep.net/record/1030879}{V.~Vagnoni}$^{33}$,\, 
\href{http://inspirehep.net/record/1074150}{D.~van Dyk}$^{139}$,\, 
\href{http://inspirehep.net/record/1070629}{J.~van Tilburg}$^{11}$,\, 
\href{http://inspirehep.net/record/1070643}{S.~Vecchi}$^{27}$,\, 
\href{http://inspirehep.net/record/1071756}{R.~Venditti}$^{140}$,\, 
\href{http://inspirehep.net/record/1054622}{M.~Vesterinen}$^{22}$,\, 
\href{http://inspirehep.net/record/1032745}{J.~Virto}$^{139, 141}$,\, 
\href{http://inspirehep.net/record/1312897}{P.~Volkov}$^{24}$,\, 
\href{http://inspirehep.net/record/1367258}{G.~Vorotnikov}$^{24}$,\, 
\href{http://inspirehep.net/record/1077733}{E.~Vryonidou}$^{26}$,\, 
\href{http://inspirehep.net/record/1055327}{J.~Walder}$^{15}$,\, 
\href{http://inspirehep.net/record/984240}{W.~Walkowiak}$^{100}$,\, 
\href{http://inspirehep.net/record/1060811}{J.~Wang}$^{31}$,\, 
\href{http://inspirehep.net/record/1058694}{W.~Wang}$^{142}$,\, 
\href{http://inspirehep.net/record/1069158}{C.~Weiland}$^{13, 143}$,\, 
\href{http://inspirehep.net/record/1070636}{M.~Whitehead}$^{83}$,\, 
\href{http://inspirehep.net/record/983530}{G.~Wilkinson}$^{82}$,\, 
\href{http://inspirehep.net/record/1096938}{J.~M.~Williams}$^{144}$,\, 
\href{http://inspirehep.net/record/983483}{M.~R.~J.~Williams}$^{14}$,\, 
\href{http://inspirehep.net/record/983449}{F.~Wilson}$^{145}$,\, 
\href{http://inspirehep.net/record/1046258}{Y.~Xie}$^{146}$,\, 
\href{http://inspirehep.net/record/982785}{Z.~Yang}$^{147}$,\, 
\href{http://inspirehep.net/record/1087415}{T.~You}$^{17}$,\, 
\href{http://inspirehep.net/record/1056696}{F.~Yu}$^{36}$,\, 
C.~Zhang$^{148}$,\, 
\href{http://inspirehep.net/record/1027553}{L.~Zhang}$^{147}$,\, 
\href{http://inspirehep.net/record/1666186}{W.~Zhang}$^{84}$ 
\vspace*{1cm} 
 }
\institute{\small 
$^{1}$ \href{http://inspirehep.net/record/903239}{U. Sussex, Brighton, Dept. Phys. Astron.}, $^{2}$ \href{http://inspirehep.net/record/909603}{Paris U., IV}, $^{3}$ \href{http://inspirehep.net/record/909939}{INFN, Milan Bicocca}, $^{4}$ \href{http://inspirehep.net/record/905888}{U. Laguna, Tenerife, Dept. Phys.}, $^{5}$ \href{http://inspirehep.net/record/910482}{IAC, La Laguna}, $^{6}$ \href{http://inspirehep.net/record/902734}{U. Cincinnati, Dept. Phys.}, $^{7}$ \href{http://inspirehep.net/record/1118764}{Ohio State U., Columbus}, $^{8}$ \href{http://inspirehep.net/record/907623}{U. Cambridge, DAMTP}, $^{9}$ \href{http://inspirehep.net/record/1218068}{UC, Santa Cruz, SCIPP}, $^{10}$ \href{http://inspirehep.net/record/902880}{INFN, Florence}, $^{11}$ \href{http://inspirehep.net/record/903832}{Nikhef, Amsterdam}, $^{12}$ \href{http://inspirehep.net/record/902884}{INFN, Padua}, $^{13}$ \href{http://inspirehep.net/record/908399}{Durham U., IPPP}, $^{14}$ \href{http://inspirehep.net/record/902984}{U. Manchester, Sch. Phys. Astron.}, $^{15}$ \href{http://inspirehep.net/record/902948}{Lancaster U., Dept. Phys.}, $^{16}$ \href{http://inspirehep.net/record/903031}{ITEP, Moscow}, $^{17}$ \href{http://inspirehep.net/record/902712}{U. Cambridge, Cavendish Lab.}, $^{18}$ \href{http://inspirehep.net/record/903178}{Saha Inst. Nucl. Phys., Kolkata}, $^{19}$ \href{http://inspirehep.net/record/902671}{U. Birmingham, Sch. Phys. Astron.}, $^{20}$ \href{http://inspirehep.net/record/902774}{Tech. U., Dortmund, Dept. Phys.}, $^{21}$ \href{http://inspirehep.net/record/902770}{DESY, Hamburg}, $^{22}$ \href{http://inspirehep.net/record/903734}{U. Warwick, Dept. Phys.}, $^{23}$ \href{http://inspirehep.net/record/903242}{Syracuse U., Dept. Phys.}, $^{24}$ \href{http://inspirehep.net/record/908795}{Moscow State U., SINP}, $^{25}$ \href{http://inspirehep.net/record/902842}{U. Heidelberg, Phys.Inst.}, $^{26}$ \href{http://inspirehep.net/record/902725}{CERN, Geneva}, $^{27}$ \href{http://inspirehep.net/record/905268}{INFN, Ferrara}, $^{28}$ \href{http://inspirehep.net/record/903113}{U. Padua, Dept. Phys.}, $^{29}$ \href{http://inspirehep.net/record/902756}{IFJ, Krakow}, $^{30}$ \href{http://inspirehep.net/record/911544}{Tech. U., Munich, IAS}, $^{31}$ \href{http://inspirehep.net/record/902804}{U. Florida, Gainesville, Dept. Phys.}, $^{32}$ \href{http://inspirehep.net/record/902674}{U. Bologna, Dept. Phys.}, $^{33}$ \href{http://inspirehep.net/record/902878}{INFN, Bologna}, $^{34}$ \href{http://inspirehep.net/record/902796}{Fermilab}, $^{35}$ \href{http://inspirehep.net/record/946080}{U. Chicago}, $^{36}$ \href{http://inspirehep.net/record/1280366}{U. Mainz, PRISMA}, $^{37}$ \href{http://inspirehep.net/record/903038}{LMU Munich, Dept. Phys.}, $^{38}$ \href{http://inspirehep.net/record/902868}{Imperial Coll., London, Dept. Phys.}, $^{39}$ \href{http://inspirehep.net/record/902988}{CPT, Marseille}, $^{40}$ \href{http://inspirehep.net/record/904661}{Natl. Taiwan U., Taipei, Phys. Dept.}, $^{41}$ \href{http://inspirehep.net/record/910711}{U. Santiago de Compostela, IGFAE}, $^{42}$ \href{http://inspirehep.net/record/1268260}{Los Alamos Natl. Lab., Theor. Div.}, $^{43}$ \href{http://inspirehep.net/record/907692}{INFN, Rome 3}, $^{44}$ \href{http://inspirehep.net/record/902989}{CPPM, Marseille}, $^{45}$ \href{http://inspirehep.net/record/1124249}{U. Bern, AEC}, $^{46}$ \href{http://inspirehep.net/record/904735}{INFN, Cagliari}, $^{47}$ \href{http://inspirehep.net/record/902889}{INFN, Turin}, $^{48}$ \href{http://inspirehep.net/record/902787}{U. Edinburgh, Sch. Phys. Astron.}, $^{49}$ \href{http://inspirehep.net/record/905405}{PSI, Villigen}, $^{50}$ \href{http://inspirehep.net/record/902883}{INFN, Naples}, $^{51}$ \href{http://inspirehep.net/record/903537}{U. Louisville, Dept. Phys.}, $^{52}$ \href{http://inspirehep.net/record/902968}{Los Alamos Natl. Lab.}, $^{53}$ \href{http://inspirehep.net/record/903305}{UC, San Diego, Dept. Phys.}, $^{54}$ \href{http://inspirehep.net/record/903006}{Michigan State U., East Lansing, Dept. Phys. Astron.}, $^{55}$ \href{http://inspirehep.net/record/1625414}{IRFU, Saclay, DPP}, $^{56}$ \href{http://inspirehep.net/record/911713}{U. Southern Denmark, CP3-Origins}, $^{57}$ \href{http://inspirehep.net/record/903357}{Yale U., Dept. Phys.}, $^{58}$ \href{http://inspirehep.net/record/912370}{Natl. Res. U. Higher Sch. Econ., Moscow}, $^{59}$ \href{http://inspirehep.net/record/904475}{U. Rome 2, Tor Vergata, Dept. Phys.}, $^{60}$ \href{http://inspirehep.net/record/907691}{INFN Rome Tor Vergata}, $^{61}$ \href{http://inspirehep.net/record/902740}{LPC, Clermont-Ferrand}, $^{62}$ \href{http://inspirehep.net/record/903101}{LPT, Orsay}, $^{63}$ \href{http://inspirehep.net/record/902964}{U. Liverpool, Dept. Phys.}, $^{64}$ \href{http://inspirehep.net/record/907960}{U. Milan Bicocca, Dept. Phys.}, $^{65}$ \href{http://inspirehep.net/record/902888}{INFN, Trieste}, $^{66}$ \href{http://inspirehep.net/record/903158}{CBPF, Rio de Janeiro}, $^{67}$ \href{http://inspirehep.net/record/903257}{Technion, IIT, Dept. Phys.}, $^{68}$ Theoretical Particle Physics Laboratory (LPTP), Institute of Physics, EPFL, Lausanne, Switzerland, $^{69}$ \href{http://inspirehep.net/record/1677475}{ICAS, UNSAM, Buenos Aires}, $^{70}$ \href{http://inspirehep.net/record/903408}{J. Stefan Inst., Ljubljana}, $^{71}$ \href{http://inspirehep.net/record/902886}{INFN, Pisa}, $^{72}$ \href{http://inspirehep.net/record/903097}{U. Oregon, Eugene, Dept. Phys.}, $^{73}$ \href{http://inspirehep.net/record/909078}{U. Granada}, $^{74}$ \href{http://inspirehep.net/record/903370}{U. Zurich, Phys. Inst.}, $^{75}$ \href{http://inspirehep.net/record/908524}{Beihang U.}, $^{76}$ \href{http://inspirehep.net/record/909080}{Cornell U., LEPP}, $^{77}$ \href{http://inspirehep.net/record/908074}{LAPTH, Annecy}, $^{78}$ \href{http://inspirehep.net/record/903895}{CAS, ITP, Beijing}, $^{79}$ \href{http://inspirehep.net/record/904611}{UCAS, Beijing}, $^{80}$ \href{http://inspirehep.net/record/902832}{U. Hamburg, Inst. Exp. Phys.}, $^{81}$ \href{http://inspirehep.net/record/902990}{U. Maryland, College Park, Dept. Phys.}, $^{82}$ \href{http://inspirehep.net/record/903112}{U. Oxford, Part. Phys. Dept.}, $^{83}$ \href{http://inspirehep.net/record/910724}{RWTH Aachen}, $^{84}$ \href{http://inspirehep.net/record/902692}{Brown U., Dept. Phys.}, $^{85}$ \href{http://inspirehep.net/record/1241132}{Bethel Coll.}, $^{86}$ \href{http://inspirehep.net/record/902640}{Amsterdam U. (main)}, $^{87}$ \href{http://inspirehep.net/record/1120892}{TIFR, Mumbai, DHEP}, $^{88}$ \href{http://inspirehep.net/record/911565}{Excel. Cluster Universe, Munich}, $^{89}$ \href{http://inspirehep.net/record/903100}{LAL, Orsay}, $^{90}$ \href{http://inspirehep.net/record/903532}{U. Ljubljana, Fac. Math. Phys.}, $^{91}$ \href{http://inspirehep.net/record/903259}{Tel-Aviv U., Dept. Part. Phys.}, $^{92}$ \href{http://inspirehep.net/record/1225433}{Nagoya U.}, $^{93}$ \href{http://inspirehep.net/record/903369}{ETH, Zurich, Dept. Phys.}, $^{94}$ \href{http://inspirehep.net/record/902646}{Arizona State U., Tempe, Dept. Phys. Astron.}, $^{95}$ \href{http://inspirehep.net/record/1210798}{Peking U., Beijing, SKLNPT}, $^{96}$ \href{http://inspirehep.net/record/903128}{Scuola Normale Superiore, Pisa}, $^{97}$ \href{http://inspirehep.net/record/902874}{Indiana U., Bloomington, Dept. Phys.}, $^{98}$ \href{http://inspirehep.net/record/902974}{IPNL, Lyon}, $^{99}$ \href{http://inspirehep.net/record/903194}{IHEP, Serpukhov}, $^{100}$ \href{http://inspirehep.net/record/903203}{U. Siegen, Dept. Phys.}, $^{101}$ \href{http://inspirehep.net/record/903421}{LAPP, Annecy}, $^{102}$ \href{http://inspirehep.net/record/907907}{IFIC, Valencia}, $^{103}$ \href{http://inspirehep.net/record/907904}{U. Barcelona, IFAE}, $^{104}$ \href{http://inspirehep.net/record/903342}{Weizmann Inst. Sci., Rehovot, Fac. Phys.}, $^{105}$ \href{http://inspirehep.net/record/902882}{INFN, Milan}, $^{106}$ \href{http://inspirehep.net/record/903009}{U. Milan, Dept. Phys.}, $^{107}$ \href{http://inspirehep.net/record/911853}{U. Mainz}, $^{108}$ \href{http://inspirehep.net/record/911469}{KIT, Karlsruhe}, $^{109}$ \href{http://inspirehep.net/record/902887}{INFN, Rome 1}, $^{110}$ \href{http://inspirehep.net/record/903168}{U. Rome 1, La Sapienza, Dept. Phys.}, $^{111}$ \href{http://inspirehep.net/record/903448}{Bogazici U., Istanbul, Dept. Phys.}, $^{112}$ \href{http://inspirehep.net/record/905303}{LIP, Lisbon}, $^{113}$ \href{http://inspirehep.net/record/902953}{LBNL, Berkeley, Div. Phys.}, $^{114}$ \href{http://inspirehep.net/record/903299}{UC, Berkeley, Dept. Phys.}, $^{115}$ \href{http://inspirehep.net/record/906106}{U. Oviedo, Dept. Phys.}, $^{116}$ \href{http://inspirehep.net/record/902807}{INFN, LNF, Frascati}, $^{117}$ \href{http://inspirehep.net/record/904961}{JLab, Newport News}, $^{118}$ \href{http://inspirehep.net/record/912374}{Indiana U., Bloomington, CEEM}, $^{119}$ \href{http://inspirehep.net/record/903523}{EPFL, Lausanne, LPHE}, $^{120}$ \href{http://inspirehep.net/record/903341}{Wayne State U., Detroit, Dept. Phys. Astron.}, $^{121}$ \href{http://inspirehep.net/record/906718}{ECT, Trento}, $^{122}$ \href{http://inspirehep.net/record/903753}{U. Regensburg, Dept. Phys.}, $^{123}$ \href{http://inspirehep.net/record/903129}{U. Pisa, Dept. Phys.}, $^{124}$ \href{http://inspirehep.net/record/902687}{U. Bristol, Wills Phys. Lab.}, $^{125}$ \href{http://inspirehep.net/record/902727}{Charles U., Prague, Inst. Part. Nucl. Phys.}, $^{126}$ \href{http://inspirehep.net/record/902730}{Chicago U., EFI}, $^{127}$ \href{http://inspirehep.net/record/910783}{Cathol. U. Louvain, CP3}, $^{128}$ \href{http://inspirehep.net/record/903349}{U. Wisconsin, Madison,  Dept. Phys.}, $^{129}$ \href{http://inspirehep.net/record/907520}{Cornell U., Dept. Phys.}, $^{130}$ \href{http://inspirehep.net/record/910794}{Pablo de Olavide U., Seville}, $^{131}$ \href{http://inspirehep.net/record/903404}{Rutgers U., Piscataway, Dept. Phys. Astron.}, $^{132}$ \href{http://inspirehep.net/record/907933}{Vrije U. Brussels, Dept. Phys. Astrophys.}, $^{133}$ \href{http://inspirehep.net/record/902689}{Brookhaven Natl. Lab., Dept. Phys.}, $^{134}$ \href{http://inspirehep.net/record/902823}{U. Glasgow, Sch. Phys. Astron.}, $^{135}$ \href{http://inspirehep.net/record/902881}{INFN, Genoa}, $^{136}$ \href{http://inspirehep.net/record/906319}{U. Cyprus, Nicosia, Dept. Phys.}, $^{137}$ \href{http://inspirehep.net/record/902780}{JINR, Dubna}, $^{138}$ \href{http://inspirehep.net/record/902748}{U. Colorado, Boulder, Dept. Phys.}, $^{139}$ \href{http://inspirehep.net/record/903037}{Tech. U., Munich, Dept. Phys.}, $^{140}$ \href{http://inspirehep.net/record/902877}{INFN, Bari}, $^{141}$ \href{http://inspirehep.net/record/1237813}{MIT, Cambridge, CTP}, $^{142}$ \href{http://inspirehep.net/record/1083225}{Jiao Tong U., Shanghai, INPAC}, $^{143}$ \href{http://inspirehep.net/record/903130}{U. Pittsburgh, Dept. Phys. Astron.}, $^{144}$ \href{http://inspirehep.net/record/907455}{MIT, Cambridge}, $^{145}$ \href{http://inspirehep.net/record/903174}{RAL, Didcot}, $^{146}$ \href{http://inspirehep.net/record/1279835}{CCNU, Wuhan, Inst. Part. Phys.}, $^{147}$ \href{http://inspirehep.net/record/1205356}{Tsinghua U., Beijing, CHEP}, $^{148}$ \href{http://inspirehep.net/record/903123}{CAS, IHEP, Beijing},  }\normalsize
\clearpage
\end{titlepage}