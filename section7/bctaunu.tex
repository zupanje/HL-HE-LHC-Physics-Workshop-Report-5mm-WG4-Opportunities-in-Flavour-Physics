\subsubsection{$B\rightarrow D^{(*)}l\nu$ form factors and anatomy of $R(D^{(*)})$}
Signs of LFUV have not only been seen in loop-suppressed flavor-changing 
neutral currents discussed above, but also in tree-level decays, namely, in tensions with the SM predictions for the ratios
\begin{equation}
  R(D^{(*)}) = \frac{\mathcal{B}(B \to D^{(*)}\tau\nu)}{\mathcal{B}(B \to D^{(*)}l\nu)},\,\hspace{0.5cm}(l=\mu,~e).
\end{equation}
Assuming for the moment SM particle content only, the $b\to c\ell\nu_{\ell^\prime}$ transitions at scales near $m_b$ can be described by a $SU(3) \times U(1)$-invariant effective Hamiltonian,
\begin{equation}\label{eq:Heffbcellnu}
  \mathcal H_\text{eff}^{b\to c\tau\nu} = \frac{4 G_F}{\sqrt{2}} V_{cb}
    \sum_{\ell\ell^\prime}\Big(\Blu{(\delta_{\ell\ell^\prime}+C_L^{\ell\ell^\prime})}O_{V_L}^{\ell\ell^\prime} + \sum_{i} C_i^{\ell\ell'} O_i^{\ell\ell'} + \text{h.c.}\Big),
\end{equation}
where $\ell,\ell'=e,\mu,\tau$ denotes the charged-lepton and neutrino flavor, respectively, and the sum over $i$ runs over the following operators,
\begin{equation}
\label{eq:ops}
\begin{split}
  O_{V_{L,R}}^{\ell\ell'} &= (\bar c_{L,R} \gamma^\mu b_{L,R})(\bar \ell_L \gamma_\mu \nu_{\ell' L}), 
  %\\
  \qquad O_{S_{L,R}}^{\ell\ell'} = (\bar c_{R,L} b_{L,R})(\bar \ell_R \nu_{\ell' L}), 
  \\
  O_T^{\ell\ell'} &= (\bar c_R \sigma^{\mu\nu} b_L)(\bar \ell_R \sigma_{\mu\nu}\nu_{\ell' L}).
\end{split}
\end{equation}
The NP coefficients $C_i^{\ell\ell'}$ depend, in general, on both charged-lepton and neutrino flavor. These operators arise from more fundamental interactions at a higher scale $\Lambda$ which, in accordance with available LHC data, can be taken to be larger than the electroweak scale $v˜$. The operators of Eq.~(\ref{eq:ops}) should then be embedded in ones consisting of fields with full SM gauge quantum numbers. For $C_{V_R}$ this is impossible at dimension-6, leading to a parametric suppression of at least $v^2/\Lambda^2$ for tree-level UV completions. Linearly realized electroweak symmetry breaking, captured by the Standard Model effective field theory (SMEFT)~\cite{Buchmuller:1985jz,Grzadkowski:2010es}, also yields the prediction $C_{V_R}^{\ell\ell'} \equiv C_{V_R}\delta_{\ell\ell'}$~\cite{Cirigliano:2009wk,Cata:2015lta}, in which case sizable contributions to $C_{V_R}$ are excluded by $b\to c(e,\mu)\nu$~\cite{Jung:2018lfu}. While deviations from this prediction are possible in a non-linear realization of electroweak symmetry breaking~\cite{Cata:2015lta}, at least one of the above-named sources of suppression always remains and right-handed currents do not play a role.

The relevant hadronic matrix elements of the SM operator $O_{V_L}^{\ell\ell}$, i.e., $ \bra{D} \bar{c} \gamma^\mu b \ket{\bar{B}}$, $\bra{D^*} \bar{c} \gamma^\mu b \ket{\bar{B}}$, and 
$\bra{D^*} \bar{c} \gamma^\mu \gamma_5 b \ket{\bar{B}}$, are parametrized using one, two, and three form factors, respectively~\cite{Caprini:1997mu}.
%
%The meson form factors stemming from the hadronic matrix elements of the SM operator $O_{V_L}^{\ell\ell}$ can be parametrized as~\cite{Caprini:1997mu}
%\begin{align}
%  \bra{D(v_D)} \bar{c} \gamma^\mu b \ket{\bar{B}(v_B)} &= h_+(w) (v_B + v_D)^\mu + h_-(w)(v_B - v_D)^\mu \,, \\
%%
%  \bra{D^*(v_{D^*}, \varepsilon_{D^*})} \bar{c} \gamma^\mu b \ket{\bar{B}(v_B)} &= 
%	i h_V \varepsilon^{\mu\nu\alpha\beta} \varepsilon^*_{D^* \nu} v_{D^* \alpha} v_{B\, \beta} \,, \\
%%
%  \bra{D^*(v_{D^*}, \varepsilon_{D^*})} \bar{c} \gamma^\mu \gamma_5 b \ket{\bar{B}(v_B)} &= 
%	h_{A_1}(w) (w+1) \varepsilon^{*\, \mu}_{D^*} - 
%	\left(h_{A_2}(w) v_B^\mu + h_{A_3}(w) v_{D^*}^\mu\right) \varepsilon_{D^*}^* \cdot v_B\,,
%\end{align}
%with form factors $h_i$, polarization vector $\varepsilon_{D^*}$, velocities $v_P$, and the dimensionless variable
%$w = (m_B^2 + m_{D^{(*)}}^2-q^2)/(2 m_B m_{D^{(*)}})$. 
%One can reformulate the above parametrization in terms of helicity amplitudes with definite parity and spin in the 
%convention of Caprini, Lellouch, and Neubert (CLN) as~\cite{Caprini:1997mu}
%\begin{multline}
%  V_1 = h_+ - \frac{1-r}{1+r} h_-\,, \qquad
%  S_1 = h_+ - \frac{1+r}{1-r} \frac{w-1}{w+1} h_-\,, \qquad
%  V_4 = h_V, \qquad 
%  A_1 = h_{A_1}\,, \\  
%  A_5 = \frac{(w-r) h_{A_1} - (w-1) (r h_{A_2} + h_{A_3} )}{1-r}\,, \,
%  P_1 = \frac{(w+1) h_{A_1} - (1 - w r) h_{A_2} - (w - r) h_{A_3} }{1+r} \,,
%\end{multline}
%where $r=m_{D^{(*)}}/m_B$.
%The translation table to the form factor convention of Boyd, Grinstein, and Lebed (BGL) \cite{Boyd:1997kz} can be found in Ref.~\cite{Bigi:2017jbd}.
%Using the helicity basis has the advantage that the unitarity bounds can be written in an elegant way. 
%Employing the BGL parametrization \cite{Boyd:1997kz}
%\begin{equation}
%  F(z) = \frac{1}{P_F(z) \phi_F(z)} \sum_{n=0}^{\infty} a_n^F z^n, \qquad
%  z = \frac{\sqrt{1+w}-\sqrt{2}}{\sqrt{1+w} + \sqrt{2}},
%\end{equation} 
%with Blaschke factor $P_F(z)$ and outer function $\phi_F(z)$, the unitarity bounds have the simple form
%\begin{equation}
%\label{eq:unitarity-constr}
%  \sum_{n=0}^\infty \left( a_n^F \right)^2 \leq 1\,,
%\end{equation}
%where \emph{e.g.}, the coefficients of $A_1$ and $A_5$ contribute to the same sum.
The helicity form factors obey unitarity bounds which can be written in an elegant way using the parametrization of Boyd-Grinstein-Lebed (BGL) \cite{Boyd:1997kz}. The contribution of the helicity form factors $S_1$ and $P_1$ \Blu{(as defined in~\cite{Caprini:1997mu})} to the branching ratios of $B\rightarrow Dl\nu$ and $B\rightarrow D^*l\nu$ are
suppressed by the mass of the final-state lepton. In view of the lack of experimental information on these form factors we need  input from theory. 
In the case of $B \to D$ we are in the fortunate position that lattice results
for both $V_1$ and $S_1$ at $w\geq 1$ exist~\cite{Lattice:2015rga,Na:2015kha,Aoki:2016frl}, where $w = (m_B^2 + m_{D^{(*)}}^2-q^2)/(2 m_B m_{D^{(*)}})$.
This is the reason for the very good agreement of all SM predictions in this case, see Table~\ref{tab:RD-and-RDstar-anomaly}. \Blu{Recently, also soft photon corrections to $R(D)$ have been discussed~\cite{deBoer:2018ipi}.}
For $B \to D^*$ we have only one data point from lattice, namely $A_1(w=1)$~\cite{Bailey:2014tva,Harrison:2017fmw}, so that it is necessary to use 
Heavy Quark Effective Theory (HQET)~\cite{Bernlochner:2017jka,Caprini:1997mu,Luke:1990eg,Neubert:1991xw,Neubert:1993mb,Neubert:1992wq,Neubert:1992pn,Ligeti:1993hw}
to relate the $B \to D$ and $B \to D^*$ form factors order-by-order in the heavy quark expansion in terms of Isgur--Wise functions,
in order to obtain a prediction for $P_1$ and hence $R(D^*)$. 

At NLO in the heavy-quark expansion, {i.e.}, expanding to linear order in $\alpha_s/\pi$ and $1/m_{c,b}$,
there is sufficient differential information in the $B \to D^{(*)}l\nu$ decays to fit to the four Isgur-Wise functions
that arise at this order \cite{Bernlochner:2017jka,Jaiswal:2017rve}. 
In the literature the theoretical error from using NLO HQET results is under discussion, leading to different 
results for the error of the SM prediction. 
HQET can also be used to obtain a stronger version of the unitarity bounds
% than Eq.~(\ref{eq:unitarity-constr}); 
(see Refs.~\cite{Boyd:1997kz,Bigi:2017jbd} for details).
Additional model-dependent theoretical input at maximal $w$ can be provided by Light Cone Sum Rules (LCSR)~\cite{Faller:2008tr,Gubernari:2018wyi},
or at zero recoil by QCD sum rules~\cite{Neubert:1992wq,Neubert:1992pn,Neubert:1993mb,Ligeti:1993hw}.
We give a summary of theoretical predictions for $R(D^{(*)})$ in Table~\ref{tab:RD-and-RDstar-anomaly}.

Because lattice data is presently available only at zero recoil for $B \to D^*$,
and since kinematic suppression requires $d\Gamma[B \to D^*l\nu]/dw$ to vanish at $w=1$,  
$|V_{cb}|$ can be obtained from $B \to D^*l\nu$ only by extrapolating $d\Gamma[B \to D^*l\nu]/dw$,
fitted to $w > 1$ data, back to zero recoil. 
This procedure can be highly sensitive to the chosen form-factor parameterization and other theoretical inputs;
for recent analyses see 
Refs.~\cite{Abdesselam:2017kjf,Bigi:2017njr,Bigi:2017jbd,Grinstein:2017nlq,Bernlochner:2017jka,Bernlochner:2017xyx,Jaiswal:2017rve,Harrison:2017fmw,Schacht:2017vfd}. 
Lattice data beyond zero recoil is expected soon (see \cite{Aviles-Casco:2017nge} for preliminary results),
from both domain wall and AsqTad ensemble approaches. 
Combined with abundant future data for $B \to D^*l\nu$ from HL-LHC, extractions of $|V_{cb}|$ that are less sensitive to theoretical inputs
will become possible, thereby either resolving or more concretely establishing tensions between exclusive and inclusive measurements of $|V_{cb}|$.

With future experimental data and lattice QCD results (see Sec.~\ref{sec:lattice} and Table~\ref{table:Proj} therein) the SM 
prediction of $R(D^*)$ will improve considerably. For an estimate, one may note that the dependence of $d\Gamma/dw$ on the $P_1$ form factor arises 
only incoherently, via a contribution of the form $m_l^2|P_1(w)|^2$. This $P_1$ term contributes approximately 10\% of the total 
integrated $B \to D^*\tau\nu$ rate, which suggests that the dependence of $R(D^*)$ on this form factor should be limited. 
Assuming a 1\% future precision for $P_1$, using the 50 ab$^{-1}$ which Belle~II will presumably gather by 2025, and taking into 
account the expected improvement of the form factors $A_{1,5}$ and $V_4$ from the lattice, it is hence reasonable to assume that
an error of 0.001 for the SM value of $R(D^*)$ might be achieved.

\begin{table*}[t]
\caption{
Current SM $R(D^{(*)})$ theory predictions and their deviation from experiment 
$R(D)^{\mathrm{exp}} = 0.407(39)(24)$ \cite{Amhis:2016xyh,Lees:2012xj,Lees:2013uzd,Huschle:2015rga}
and
$R(D^*) = 0.306(13)(7)$ \cite{Amhis:2016xyh,Lees:2012xj,Lees:2013uzd,Huschle:2015rga,Sato:2016svk,LHCb-PAPER-2015-025,Hirose:2016wfn,Hirose:2017dxl,LHCb-PAPER-2017-017,LHCb-PAPER-2017-027}
(HFLAV 2018 summer update). 
For older SM predictions see Refs.~\cite{Fajfer:2012vx,Celis:2012dk,Tanaka:2012nw}.
Table adapted and extended from Ref.~\cite{Schacht:2017vfd}.
\label{tab:RD-and-RDstar-anomaly}} 
\begin{center}
\begin{tabular}{ccc}
  \hline\hline
  Ref.                       & $R(D)$      &  Exp. deviation \\
  \hline
  \cite{Bigi:2016mdz}        & $0.299(3)$  & $2.4\sigma$ \\ 
  \cite{Bernlochner:2017jka} & $0.299(3)$  & $2.4\sigma$ \\
  \cite{Jaiswal:2017rve}     & $0.302(3)$  & $2.3\sigma$ \\
  \hline\hline 
\end{tabular}
\qquad
\begin{tabular}{ccc}
  \hline\hline
  Ref.                       & $R(D^*)$   &  Exp. deviation \\
  \hline 
  \cite{Bernlochner:2017jka} & $0.257(3)$ & $3.3\sigma$ \\
  \cite{Bigi:2017jbd}        & $0.260(8)$ & $2.7\sigma$ \\ 
  \cite{Jaiswal:2017rve}     & $0.257(5)$ & $3.1\sigma$ \\
  \hline\hline
\end{tabular}
\end{center}
\end{table*}

\subsubsection{Excited states and other $b$ hadrons: $R(D^{**})$, $R(J/\psi)$ and $R(\Lambda_c)$}

Measurements of $|V_{cb}|$ and lepton universality can also be probed via $B$ decays to the $D^{**}$ excited states,
as well as decays of strange or charmed $b$ hadrons, including $B_s \to D^{(*)}_s$, $B_c \to J/\psi$,
and the baryonic $\Lambda_b \to \Lambda_c$ transitions. In comparison to the $B \to D^{(*)}\ell \nu$ decay modes, these modes may 
variously exhibit higher sensitivities to specific NP operators, or in some cases,
may be theoretically cleaner than the decays to the $D^{(*)}$ ground states. 
These modes can also be important downfeed or crossfeed backgrounds to the $B \to D^{(*)}\ell \nu$ decays and,
to the extent they are affected by the same NP operators, must also be understood and measured carefully.
The HL-LHC is the only planned experiment that will yield significant samples of these heavier $b$-hadrons, 
with precision analyses anticipated from the LHCb experiment.
In this subsection we present the motivations and theoretical prospects for the measurement of each of these decay modes.

The $D^{**}$ excited states comprise four different charmed hadrons: The $D_0^*$, $D_1^*$, $D_1$, $D_2^*$.
In the language of HQET, these furnish two doublets, $\{D_0^*\,,D_1^*\}$ and $\{D_1\,,D_2^*\}$,
with spin-parity $s_{\ell}^{\pi_\ell} = \frac{1}{2}^+$ and $\frac{3}{2}^+$, respectively.
(In the heavy-quark limit, spin-parity is a conserved quantum number.)
The $\frac{3}{2}^+$ states are narrow, with $\Gamma \sim 30$--$50$\,MeV,
because their hadronic decays to $D^{(*)}\pi$ either proceed via a $D$-wave or violate heavy quark-symmetry,
while the $\frac{1}{2}^+$ states are quite broad.  Although isolating these excited-state decays will likely be simpler at $e^+e^-$ $B$ factories, 
which can more easily reconstruct $\pi^0$'s and photons, analyses of $B \to D^{**}\ell\nu$ decays are also feasible at LHCb \upgradetwo, 
especially for the narrow $\frac{3}{2}^+$ states subsequently decaying to charged hadrons.

The crucial attractive feature of the $B \to D^{**}$ transitions is that various leading-order contributions to the form factors
vanish in the heavy-quark limit at zero recoil ($w=1$), so that $\mathcal{O}(\alpha_s)$ and $\mathcal{O}(1/m_{c,b})$ corrections
become important~\cite{Leibovich:1997tu,Leibovich:1997em,Bernlochner:2012bc,Bernlochner:2016bci}. 
The richer structure of the subleading form-factor contributions has the consequence that sensitivity to various NP currents
can be much larger than in the ground state decays~\cite{Bernlochner:2016bci, Bernlochner:2017jxt}. 
For example, including only a NP tensor operator, $O^{\ell\ell^\prime}_T$, one finds the ratios
$R(X)/R(X)_{\text{SM}} \simeq \{1.5, 1.3\}$ for $X = \{D, D^*\}$ at the best fit to the $R(D^{(*)})$ data. 
However, the same Wilson coefficients result in $R(X)/R(X)_{\text{SM}}$ greater than $4.0$ or less than $0.5$ for $X=\{D_0^*, D_1^*, D_1, D_2^*\}$.
The current SM predictions for all four modes, from fits to Belle data including NLO HQET contributions, are~\cite{Bernlochner:2017jxt}
\begin{equation}
\label{eq:RDDs}
  R(D_0^*) = 0.08 \pm 0.03,\quad  R(D_1^*) = 0.05 \pm 0.02, \quad
  %\nonumber \\*
  R(D_1) =  0.10 \pm 0.02,\quad  R(D_2^*) = 0.07 \pm 0.01\,.
\end{equation}

Decays of $B$ mesons to these excited states have total SM branching ratios comparable to the $B \to D^{(*)}\ell \nu$ decays themselves.
Combined with the possible large enhancement of the semitauonic modes by NP contributions,
this means that the subsequent $D^{**} \to D^{(*)}X$ decays can then induce important downfeed  backgrounds to the $D^{(*)}$ measurements.
Analyses of $B \to D^{(*)}\ell \nu$ will therefore typically have to fold in contributions from these excited states.
Moreover, anticipated analyses for the inclusive $B \to D\pi\ell\nu$ decays provide an opportunity to probe these excited-state decays,
and their associated larger NP sensitivities, \emph{collectively} with the ground-state decays.
In this context, rather than being thought of as a background, these contributions should be more properly thought of as additional sources of (NP) signal.
The large data sets from HL-LHC will then provide a very sensitive set of channels for probing NP contributions to $b \to c \ell \nu$.

The $B_s$ and $B_c$ mesons have production ratios $\sigma(B_s)/\sigma(b\bar{b}) \sim 10\%$~\cite{LHCb-PAPER-2013-004}
and $\sigma(B_c)/\sigma(b\bar{b}) \sim 0.2\%$~\cite{LHCb-PAPER-2013-044} in Run 1 LHC.
A much smaller sample of $B_s$ mesons may also be produced at $B$ factories running on the $\Upsilon(5S)$ resonance.
However, for the $B_c$ the only significant sample of mesons will be produced at HL-LHC.
At first glance, the theoretical structure of $B_s \to D_s^{(*,**)}\ell\nu$ can be mapped directly from $B \to D^{(*,**)}\ell\nu$
via the approximate $SU(3)$ flavor symmetry.
Some crucial differences are that $\mathcal{B}(D_s^{*} \to D_s\gamma) \simeq 94\%$, which will be difficult to see at LHCb \upgradetwo,
while the four $D_s^{**}$ excited states are all narrow, and therefore may be easier to resolve.

The leptonic $B_c \to (J/\psi \to \mu\mu) \ell \nu$ decay mode is reasonably clean experimentally, with measurements for $R(J/\psi)$ already available from LHCb Run 1 data,
%\begin{equation}
$  R(J/\psi) = 0.71 \pm 0.17\,\text{(stat)}\, \pm 0.18\,\text{(sys)}$~\cite{LHCb-PAPER-2017-035},
%\end{equation}
albeit with large uncertainties at present. A central difficulty in probing this mode lies in the large theoretical uncertainties for the $B_c \to J/\psi$ form-factor parameterizations. Predictions for the form factors are typically hadronic-model-dependent, making use of either perturbative QCD, the constituent-quark model, the (non)relativitistic quark model, or QCD sum rules~\cite{Anisimov:1998uk,Kiselev:1999sc,Ivanov:2000aj,Kiselev:2002vz,Hernandez:2006gt,Ivanov:2006ni,Wen-Fei:2013uea,Qiao:2012vt,Rui:2016opu}. The LHCb results have motivated several studies of the form factors~\cite{Dutta:2017xmj,Tran:2018kuv,Issadykov:2018myx,Watanabe:2017mip}. A recent, more model-independent result combines preliminary lattice QCD results with dispersive bounds and zero-recoil heavy-quark relations, leading to the prediction
%\begin{equation}
$  0.20 \le R(J/\psi) \le 0.39$~\cite{Cohen:2018dgz}, 
%\end{equation}
at $95\%$ CL, implying a mild tension at the $1.3\sigma$ level with the data.

Abundant samples of $\Lambda_b$'s will be produced only at (HL-)LHC, with a production cross-section $\sigma(\Lambda_b)/\sigma(b\bar{b}) \sim 10\%$~\cite{LHCb-PAPER-2011-018}. 
From an HQET point of view, the $\Lambda_b \to \Lambda_c$ transitions are theoretically cleaner than the $B \to D^{(*)}$ decays,
because the ``brown muck'' dressing the heavy quark lies in the $s_{\ell}^{\pi_\ell}= 0^+$ ground state.
A consequence of this is a relatively simpler form-factor structure, where not only the $\mathcal{O}(\alpha_s)$
but also the $\mathcal{O}(1/m_{c,b})$ and $\mathcal{O}(\alpha_s/m_{c,b})$ subleading contributions
are fully fixed by the leading-order HQET structure, reducing the number of free parameters in the form-factor fits.
These modes are therefore promising, clean candidates for testing the behavior of the HQET expansion itself,
by e.g., assessing the impact of $\mathcal{O}(1/m_c^2)$ contributions.
Fitting the subsubleading $\mathcal{O}(\alpha_s, \alpha_s/m_{c,b}, 1/m_c^2)$ HQET structure to existing LHCb data~\cite{LHCb-PAPER-2017-016}
and lattice form factor results~\cite{Detmold:2015aaa} implies such terms are of the expected size~\cite{Bernlochner:2018kxh}. 
More $\Lambda_b \to \Lambda_c \ell \nu$ data from LHCb will improve the precision of these results, allowing access to other subleading terms.
Moreover, with precision lattice calculations of the form factors (see, e.g., Ref.~\cite{Detmold:2015aaa}), 
additional data for these modes may permit precision measurement of $|V_{cb}|$ in an environment with reduced theoretical uncertainties.

\Blu{The heavy quark expansion is also applicable to $\Lambda_b$ transitions into excited
    $\Lambda_c^* = \Lambda_c(2595)$, $\Lambda_c(2625)$ states \cite{Leibovich:1997az}.
    The complexity of the HQET description lies between that of $\Lambda_b\to\Lambda_c$
    and $B\to D^{(*)}$ transitions, with two unknown functions up to $\mathcal{O}(1/m_{c,b})$.
   It was recently demonstrated that simultaneous binned likelihood fits to the rich angular distributions
   of $\Lambda_b\to \Lambda_c^*\mu\bar\nu$ decays can determine these two functions
   and produce data-driven predictions of $R(\Lambda_c^*)$~\cite{Boer:2018vpx}. The projected
   precision due to parametric effects reaches $\sim 2\%$ for the LHCb Upgrade 1 data set.
   Due to the spin structure of the $\Lambda_c^*$ states, these decays provide a
   complementary LFU probe with a different set of systematic uncertainties
   when compared to $R(D^{(*)})$ and $R(\Lambda_c)$. More data from LHCb will similarly improve the precision of these results.}

\subsubsection{Models of NP for $b\to c\tau\nu$}
\label{sec7:ModelsNP:bctaunu}
%{\bf\color{red}[Move this to the end of the section?]}\\

%The first thing to note in a discussion of a
A possible NP origin of the deviations in $R(D^{(*)})$ requires a large 
%is that such a 
contribution with respect to the SM. 
%has to be large: 
Defining $\hat{R}(X)=R(X)/R(X)|_{\rm SM}$, we have with present data $\hat{R}(D)=1.36 \pm 0.15$ and $\hat{R}(D^*)=1.19 \pm 0.06$. This points to $\gtrsim 10\%$ NP contribution to the amplitude  when the SM and NP contributions interfere, and 
%for deviations driven by NP and SM interfering and 
$\gtrsim 40\%$ if they do not. 
%As a consequence, 
\Blu{The scale of NP for the $R(D^{(*)})$ anomaly is then
%of the anomalies points to a new physics scale of
% 
\begin{equation}
 \Lambda_\text{NP} = \frac{1}{\sqrt{|V_{cb}|}} \frac{1}{\sqrt{|\Delta C_i|}} \frac{v}{\sqrt{2}} \simeq \frac{1~\text{TeV}}{\sqrt{|\Delta C_i|}}~,
\end{equation}
%
and the contributions are expected to enter at} tree level. Possible mediators, assuming they couple only to SM fields, were classified in~\cite{Freytsis:2015qca}. If the present central values $R(D^{(*)})$ are sustained, 
%another implication is that 
this anomaly will be established by the time of the measurements at HL/HE-LHC, cf. Fig.~\ref{fig:projections} left. The focus will hence shift to model differentiation in $b\to c\tau\nu$ and the analysis of the lepton- and quark-flavor structure of the NP contributions.
% In any case, precision measurements in $b\to c\tau\nu$ discussed here will provide the strictest direct limits on flavor-non-universality in these transitions and hence remain valuable constraints on NP models. 
A completely general NP analysis will require theoretical care. For instance, form-factor determinations from $b\to c \ell \nu$ decays could also be sensitive to NP contributions, subject to the constraint that extractions of $|V_{cb}|$ from these  decays can be made consistent with all other global data on CKM unitarity. Consequently, experimentally determined form-factor parameters may require a simultaneous fit to the deviations or additional determinations of form-factor ratios, which, however, are expected to be available at the required precision by the start of HL/HE-LHC. See Sec.~\ref{sec:lattice} for corresponding prospects from lattice QCD.

\begin{figure}
\includegraphics[width=0.5\textwidth]{section7/figures/RDRDstar_projection.png}
\qquad\parbox[b]{0.5\textwidth}{\includegraphics[width=7cm,height=4cm]{section7/figures/plotRDstarAlambdastar18.png}\\
\includegraphics[width=7cm,height=4cm]{section7/figures/plotRDstarFLDstar.png}}
\caption{\label{fig:projections}
Left: Present status of the $R(D)$-$R(D^*)$ anomaly, showing the individual measurements (68\% CL contours), the world average (68\% and 95\% CL filled ellipses), the SM prediction (95\% CL filled ellipse) as well as the projections for LHCb measurements by 2025 (dashed contour), and \upgradetwo (dotted contour, both at 68\% CL, assuming the present central value).
Right: Correlations between $R(D^*)$ and the $\tau$-polarization asymmetry $A_\lambda(D^*)[=-P_\tau(D^*)]$ (upper panel) and the longitudinal fraction $F_L(D^*)$ (lower panel) for NP scenarios with only a left-handed vector coupling (green) or only scalar couplings (blue), together with the first measurements~\cite{Hirose:2016wfn,Hirose:2017dxl} (light blue) and the experimental average for $R(D^*)$ (yellow bands), excluding \cite{Hirose:2016wfn} in the upper panel. Updated and adapted from Ref.~\cite{Celis:2016azn}.}
\end{figure}

%Classifying models accommodating the anomalies according to 
The tree-level mediators that can explain the $R(D)$-$R(D^*)$ anomaly
%and to the contribution they give in terms of the effective operators in Eq.~(\ref{eq:Heffbcellnu}), the options 
are a $W'$~\cite{Greljo:2015mma,Boucenna:2016wpr,Boucenna:2016qad,Megias:2017ove} generating $C_{V_L}$, a charged color-neutral scalar~\cite{Crivellin:2012ye,Celis:2012dk,Crivellin:2015hha,Celis:2016azn,Chen:2017eby,Iguro:2017ysu,Chen:2018hqy,Li:2018rax} generating $C_{S_{L,R}}$ in Eq.~(\ref{eq:Heffbcellnu}), and leptoquarks~\cite{Fajfer:2012jt,Deshpande:2012rr,Sakaki:2013bfa,Duraisamy:2014sna,Calibbi:2015kma,Fajfer:2015ycq,Barbieri:2015yvd,Alonso:2015sja,Bauer:2015knc,Das:2016vkr,Deshpand:2016cpw,Sahoo:2016pet,Dumont:2016xpj,Li:2016vvp,Becirevic:2016yqi,Barbieri:2016las,DiLuzio:2017vat,Chen:2017hir,Bordone:2017bld,Altmannshofer:2017poe,Becirevic:2018afm,Iguro:2018vqb} generating various couplings, mostly $C_{V_L}$ or $C_{S_L} \sim C_T$. For comparisons between these models,  see for instance~\cite{Tanaka:2012nw,Freytsis:2015qca,Bhattacharya:2016zcw,Ivanov:2017mrj,Alok:2017qsi,Bifani:2018zmi,Blanke:2018yud}.
Allowing for additional light particles opens up the possibility to address the anomalies with contributions involving right-handed neutrinos~\cite{He:2012zp,He:2017bft,Becirevic:2016yqi,Greljo:2018ogz,Asadi:2018wea,Robinson:2018gza,Azatov:2018kzb,Heeck:2018ntp}, since the neutrino is not detected. 

%The large number of possibilities reflects the fact that models typically introduce at least 2 degrees of freedom (one complex coupling), which mostly allows to address $R(D^{(*)})$.\footnote{Models generating only $C_{V_L}$ effectively introduce only one degree of freedom, since they amount to rescaling $V_{cb}$ in the modes involving $\tau$; hence this class of models implies universal $\hat R(X)$ ratios, specifically $\hat R(D)=\hat R(D^*)$.} 
\Blu{In order to differentiate between the different combinations of NP operators that these models produce at low energies,} information beyond $R(D^{(*)})$ is needed. Presently, the available additional observables in $b\to c\tau\nu$ transitions are: {\em (i)} differential distributions in $q^2$~\cite{Lees:2013uzd,Huschle:2015rga}, already excluding some fine-tuned scenarios despite their large uncertainties; {\em (ii)} the first measurements of the $\tau$ polarization asymmetry and the longitudinal fraction in $B\to D^*\tau\nu$~\cite{Hirose:2016wfn,Hirose:2017dxl}; {\em (iii)} the measurement of $R(J/\psi)$~\cite{LHCb-PAPER-2017-035}, which is up to $2\sigma$ above the SM prediction, as well as that of any NP model (see the previous subsection); {\em (iv)} the measurement of the inclusive rate $b\to X\tau\nu$ at LEP~\cite{Abbaneo:2001bv}, yielding $\hat R_b(X_c)=0.99\pm0.10$, in slight tension with the measurements for $R(D^{(*)})$~\cite{Ligeti:2014kia,Freytsis:2015qca,Celis:2016azn,Mannel:2017jfk}; {\it (v)} the (indirect) bound on $B_c \to \tau\nu$ from the $B_c$ lifetime~\cite{Li:2016vvp,Alonso:2016oyd,Akeroyd:2017mhr}, providing a strong constraint on models with only scalar couplings and disfavoring them as a solution for $R(D^*)$ for values close to the present central value.

Additional indirect constraints apply within UV-complete NP models:
% trying to formulate actual models, as opposed to only considering effective couplings at the scale of $m_b$: 
high-$p_T$ searches for signatures related to potential mediators of these transitions often provide strong constraints via, e.g., $b\bar{b} \to \tau\tau$~\cite{Faroughy:2016osc},   $b\bar{c} \to \tau\nu$~\cite{Greljo:2018tzh} or $h\to\tau\tau$~\cite{Feruglio:2018fxo}, while radiative corrections
%the renormalization-group evolution of the relevant operators from the mediator scale down to the $b$ scale 
can result in constraints from lepton universality in $\tau$ decays~\cite{Feruglio:2017rjo}, lepton-flavor violating decays~\cite{Feruglio:2017rjo}, charged-lepton magnetic moments~\cite{Feruglio:2018fxo} and electric dipole moments~\cite{Dekens:2018bci}. Current data on $\Upsilon(1S) \to \tau\tau$ decays also constrain most mediator models, and a future program of measuring both $\Upsilon$ and $\psi$ decays can have sensitivity to all NP UV completions~\cite{Aloni:2017eny}. The current bounds on $b \to s \nu\bar{\nu}$, now only $\mathcal{O}(1)$ above the SM~\cite{Lees:2013kla,Grygier:2017tzo}, can also put severe constraints on particular NP models~\cite{Blake:2016olu}. \Blu{Finally, it is interesting to note that contributions to $b\to s\ell\ell$ and $b\to s\gamma$ are also generically produced at loop level~\cite{Crivellin:2018yvo}.} 
%, although their interpretation is more model-dependent, being sensitive to assumptions on both mediator and specific flavor structure of couplings.

At the HL/HE-LHC qualitative progress in identifying potential NP in $b\to c\tau\nu$ can be understood chiefly in terms of two classes of observables: 
%arising from analyses of the data:
%
\begin{itemize}
\item[$\bullet$] Precision results for $R(X)$: $R(D^{(*)})$ can establish NP and give basic model differentiation. $R(\Lambda_c)$ is a measurement with independent systematics that improves model discrimination since it is sensitive to a different combination of NP parameters. The same applies to inclusive $R(X_c)$, to be 
%which, however, is expected to be 
measured
by Belle~II. Other modes, like $R(D_s^{(*)})$ or $R(J/\psi)$, will add to these and provide cross-checks with independent systematic uncertainties.
% however, $R(D_s)$ is expected to have very similar NP dependence as $R(D)$, while $R(D_s^*)$ and (less so) $R(J/\psi)$ constrain similar combinations as $R(D^*)$, hence their model-differentiating power is limited. 
\item[$\bullet$] Differential-rate measurements in $B\to X\tau\nu$: Differential measurements in $q^2$ as well as in the helicity angles and the $\tau$-polarization are powerful discriminators between the SM and NP, as well as between different NP models \cite{Korner:1987kd,Hagiwara:1989gza,Tanaka:1994ay,Chen:2005gr,Chen:2006nua,Nierste:2008qe,Tanaka:2010se,Fajfer:2012vx,Datta:2012qk,Sakaki:2012ft,Celis:2012dk,Duraisamy:2013kcw,Alonso:2016gym,Ligeti:2016npd,Celis:2016azn,Ivanov:2017mrj,Alonso:2017ktd,Jung:2018lfu}.
%The different dependence on kinematic variables allows for optimized searches for specific NP contributions. 
For instance, the low-recoil region in $B\to D\tau\nu$ is very sensitive to scalar contributions, tensor contributions change the polarization in the high-recoil region in $B\to D^*$ in a unique manner, and left-handed vector contributions leave normalized quantities unchanged while affecting the total rates sizably. Examples for the discriminating power of such measurements are given in Fig.~\ref{fig:projections} (right), where correlations in NP models are shown together with the first measurements of two proposed quantities by Belle~\cite{Hirose:2016wfn,Hirose:2017dxl}. First studies regarding the reach of LHCb for such observables are presented in the Section \ref{sec:bclnu:exp}. 
%These measurements are highly nontrivial, given the missing neutrinos and the fact that so far for instance the expected $q^2$ distribution has been used in the measurements of $R(D^*)$ for background suppression. However, 
Already semi-integrated quantities like the forward-backward asymmetry or the observables shown in Fig.~\ref{fig:projections} can be very powerful in distinguishing different NP scenarios.
\end{itemize}

At high-$p_T$, $b c \tau \nu$ operators can also be directly probed by the $pp \to \tau X$+MET signature at the LHC, inclusively~\cite{Greljo:2018tzh}, or with a $b$-tag in the final state, with model discrimination possible at the beyond the $3\sigma$ level in at the HL-LHC~\cite{Altmannshofer:2017yso}. This channel would allow one to probe all the NP scenarios addressing the $R(D^{(*)})$ anomalies in the HL-LHC phase (see Sec.~\ref{sec:highpT}).  Analysis of NP effects in $b \to c \tau \nu$ will be made more powerful and self-consistent by the development of dedicated NP reweighting tools such as \texttt{Hammer}~\cite{Duell:2016maj}. These tools will permit experimental collaborations to efficiently reweight their very large simulated datasets to arbitrary NP models and thus fit for WCs as part of experimental analyses. 
% NP analyses can then be performed directly on their $b \to c \tau \nu$ data, fitting the full differential space of the decay cascades to arbitrary NP models and simultaneously floating against backgrounds to obtain self-consistent results for best-fit NP Wilson coefficients, including possible NP contributions in the light lepton modes.
Observation of NP in $b\to c\tau\nu$ would warrant precise measurements at HL-/HE-LHC of related modes, 
%analysis of its flavor structure, \emph{e.g.}, in 
$b\to c (e,\mu)\nu$, $b\to u\tau\nu$ and $t\to b\tau\nu$ transitions. 
%While these are not a focus here, it is worth emphasizing that the HL/HE-LHC will be a powerful instrument to analyze these modes. 

\FloatBarrier