\subsection{Experimental perspectives}%, including interplay with Belle II}
\label{sec:7_exp}
\subsubsection{Measurements of $B_q\to ll$ from LHCb/ATLAS/CMS}
Following the observation of $\decay{\Bs}{\mumu}$ by the CMS and LHCb collaborations~\cite{CMS:2014xfa}, the most stringent constraints on $\decay{\Bs}{\mumu}$ and $\decay{\Bd}{\mumu}$ have been set by the LHCb~\cite{LHCb-PAPER-2017-001} and ATLAS~\cite{ATLAS-CONF-2018-046} collaborations, cf., Fig.~\ref{fig:btomumu}.
%: in 2017 LHCb measured~\cite{LHCb-PAPER-2017-001} ${\cal B}(\decay{\Bs}{\mumu})=\left(3.0\pm 0.6^{+0.3}_{-0.2}\right)\times 10^{-9}$ and ${\cal B}(\decay{\Bd}{\mumu})=\left(1.5^{+1.2 +0.2}_{-1.0 -0.1}\right) \times 10^{-10}$ %using data collected in $pp$ collisions corresponding to a total integrated luminosity of 4.4\invfb 
%(Fig.~\ref{fig:btomumu}, left and right) and in 2018 ATLAS measured~\cite{ATLAS-CONF-2018-046} ${\cal B}(\decay{\Bs}{\mumu})=\left(2.8^{+0.8}_{-0.7}\right)\times 10^{-9}$ and ${\cal B}(\decay{\Bd}{\mumu})<2.1 \times 10^{-10}$ at 95\% CL (Fig.~\ref{fig:btomumu}, middle) .
Both sets of measurements are compatible with the SM predictions.% but statistically limited. 

\begin{figure}[!tb]
\centering
\hbox{\hspace{-0.2cm}}
\includegraphics[width=0.35\textwidth]{section7/figures/btomumu_BF.pdf}
\hbox{\hspace{-0.7cm}}
\includegraphics[width=0.315\textwidth]{section7/figures/ATLAS_likelihoodContoursPlotCombination.pdf}
\hbox{\hspace{-0.35cm}}
\includegraphics[width=0.375\textwidth]{section7/figures/btomumu_decaytime.pdf}
\caption{A two-dimensional representation of the LHCb (left) and ATLAS (middle) branching fraction measurements for \Bdmm and \Bsmm. The central values are indicated with the marker. The profile likelihood contours for 1,2,3\dots Gaussian $\sigma$ are shown as blue contours (left) and grey shaded areas (middle). The red cross labelled SM reports the Standard Model predictions. Right: background-subtracted \Bsmm decay-time distribution with the LHCb fit result superimposed.}
\label{fig:btomumu}
\end{figure}

HL-LHC will offer a compelling opportunity to extend the ATLAS and CMS  $B^0_s \to \mu^+\mu^-$ and  $B^0\to \mu^+\mu^-$ studies to the expected integrated high luminosity.
Flexible trigger systems and inner tracker improvements will allow both experiments to maintain efficient low-\pt dimuon triggers and achieve good mass resolution, which are the key ingredients for the $B_{d,s} \to\mu^+\mu^-$ analysis.
Fig.~\ref{fig:GPDMass} demonstrates the ATLAS and CMS reconstructed $B \to \mu^+\mu^-$ invariant mass capabilities in the HL-LHC era.
 The pseudorapidity |$\eta_f$| of the most forward muon (of the candidate) is used to visualize the CMS results and compare the performances of  Phase~2 and Run~2. 
 In Fig.~\ref{fig:GPDMass} the signal mass distribution for |$\eta_f$| < 1.4 are overlaid. The CMS improved separation between $B^0 \to \mu^+\mu^-$ and $B^0_s \to \mu^+\mu^-$ in Phase~2 is evident.
 
The LHCb detector is already well optimised for this decay, and planned improvements to the tracking and muon detector shielding in LHCb's upgrades will ensure that the muon reconstruction and identification performance does not degrade with increasing pileup.   %\td{something about LHCb detector enhancements relevant for Bmm?}
%Fig.~\ref{fig:ATLASMass} compares the ATLAS $B^0_s \to \mu^+\mu^-$ reconstructed invariant mass for $|\eta| < 2.5$ with the current (Run~2) and upgraded tracking detector (ITk).
 \begin{figure}[t]
   \begin{center}
    \includegraphics[width=0.4\textwidth]{\main/section7/figures/ATLASmass.pdf}\hbox{\hspace{-0.5cm}}
     \includegraphics[width=0.3\textwidth]{\main/section7/figures/CMS_B2mu_Run2.pdf}\hbox{\hspace{-0.2cm}}
     \includegraphics[width=0.3\textwidth]{\main/section7/figures/CMS_B2mu_Phase2.pdf}
      \caption{Left: comparison of the ATLAS invariant mass spectra for simulated $B^0_s \to \mu^+\mu^-$ events with the current (Run~2) and upgraded (ITk) detectors. A vertical line at the $B_d$ mass value is drawn to visualize the $B_s$-$B_d$ signals separation. Middle and right plots:  CMS $B^0_s$ and $B^0$ invariant mass distributions in the Run~2 and Phase~2 scenario respectively. The $B^0_s$ distribution is normalized to unity and the $B^0$  distribution is normalized according to the SM expectation.
 (CMS plots are taken from \cite{CMS-PAS-FTR-18-013}).}
     \label{fig:GPDMass}\end{center}\end{figure}\setlength{\extrarowheight}{4pt}    
Both ATLAS and CMS make HL-LHC extrapolations with a PU scenario of 200 interactions per bunch-crossing, which was found not to have strong impact on the analysis performance. 
While the uncertainty on ${\cal B}(\decay{\Bd}{\mumu})$  remains statistically limited for HL-LHC projections (300\invfb/3\invab), the projected uncertainty on ${\cal B}(\decay{\Bs}{\mumu})$ depends on the assumptions made for the systematic uncertainties. 
% Systematics
%         LHCb
% fs/fd 5.8\% 
% BR   3\%
% PID 2\%
% track reco 2\%
The current systematic uncertainty is dominated by sources external to the analysis, such as the relative uncertainty associated with the $b$-quark fragmentation probability ratio, $f_s/f_d$~\cite{fsfd}, followed by the branching fractions of the normalisation modes and less significant systematics arising from internal analysis effects (e.g., $2\%$ each from particle identification and track reconstruction in the case of LHCb, individual signal and background yields, efficiencies, etc., in the case of ATLAS and CMS). 

Systematic uncertainties are treated slightly differently in the projections of the three experiments.
ATLAS conservatively assumes in the HL-LHC projections that the $f_s/f_d$ and the normalization modes branching fractions uncertainties will be at the same level as previously used, i.e., $5.8\%$ and $3\%$, while 
 CMS and LHCb project them to be $3.5\%$ and $1.4\%$, respectively, based on reasonable assumptions about additional Belle~II inputs and improvements in our knowledge of form-factor ratios and branching fraction measurements.
 A more complete discussion of systematic uncertainty projections from ATLAS and CMS 
 can be found in 
 \cite{ATL-PHYS-PUB-2018-005} and \cite{CMS-PAS-FTR-18-013}, respectively. 
For LHCb, the remaining experimental systematic uncertainties are already at the $\sim 3$~percent level, and because they rely on data-driven corrections from calibration samples they can be expected to be reduced to the $\sim 1.4$~percent level in \upgradetwo.
%The remaining analysis-dependent systematic uncertainties are combined together by ATLAS and parameterised as a function of the signal yield, while e.g. CMS provides the detailed breakdown shown in table~\ref{tab:CMSb2musys} (a more complete discussion of the CMS systematic uncertainty treatment an be found in \cite{CMS-PAS-FTR-18-013}). 
The resulting HL-LHC projected statistical and systematic uncertainties for the three experiments are summarized in table~\ref{LHC_Bmumu_Reach}; for the omparisons of the ATLAS, CMS and LHCb reaches (such as the one carried in Fig.~\ref{fig:HLCombinedReachFromVava}) , the reference scenarios considered are respectively the "3$\ab^{-1}$ Intermediate", "3 $\ab^{-1}$" and "300 $\fb^{-1}$".


%\begin{table}
%\centering
%\caption{CMS input sources of systematic uncertainties and the propagated uncertainties on the $B \to \mu^+\mu^-$ for the HL-LHC scenario.}
%\label{tab:CMSb2musys}
%\begin{tabular}{cccc}\hline
%Source & Input uncertainties & $\delta {\cal B} (B^0_s \to \mu^+\mu^-)$ & $\delta {\cal B }(B^0 \to \mu^+\mu^-)$ \\ \hline 
%Muon ID efficiency ratio & 1\% & 1\% & 1 \%\\ 
%$B^+$ normalization yield & 1.4\% & 1.4 \% & 1.4 \%\\
%$f_s/f_u$ ratio & 3.5\% & 3.5 \% & - \\
%Effective lifetime & 2\%  & 2\% & - \\
%Trigger efficiency & 1.5\% & 1.5 \% & 1.5 \% \\
%Other sources & 3\% & 3 \% & 3\% \\
%\hline
%Peaking background yield & 10\% & \multirow{2}{*}{0.5\%}& %\multirow{2}{*}{2.7\%}\\ 
%Semileptonic background yield & 7.5\% & & \\
%\end{tabular}\end{table}


\begin{table}
  \centering
  \caption{Projected ATLAS, CMS and LHCb uncertainty on $\cal{B}$$(B^0_s \to \mu^+ \mu^-)$ and $\cal{B}$$(B^0 \to \mu^+ \mu^-)$. 
  %The results are centered on the SM theoretical prediction~\cite{Bobeth}.
    For each extrapolation the  total (statistical+systematic) uncertainties are reported. 
    %\td{The uncertainty on ${\cal B}(\decay{\Bd}{\mumu})/{\cal B}(\decay{\Bs}{\mumu})$ is reported
    %for the LHCb reach on $\decay{\Bd}{\mumu}$, assuming the uncertainty to be statistics-dominated for that channel.}
  }
  \label{LHC_Bmumu_Reach}
  \renewcommand{\arraystretch}{1.3}
  %\begin{tabular}{ l c c c c c }
  %  \hline\hline
  %  & & \multicolumn{2}{c}{$\cal{B}$$(B^0_s \to \mu^+ \mu^-)$ } & %\multicolumn{2}{c}{$\cal{B}$$(B^0 \to \mu^+ \mu^-)$} \\
  % Experiment & Scenario & stat $\%$ & stat + syst $\%$ & stat $\%$ & stat + syst %$\%$ \\
  %  \hline
  %  LHCb   & 23\invfb          &       &   8.2  &       &  33       \\
  %  LHCb   & 300\invfb         &    1.8   &  4.4  &   9.2    &  9.4      \\
  %   CMS   & 300\invfb         &       &   12 &       & 46  \\
  %   CMS   & 3 \invab          &       &   7  &       & 16  \\
  %  ATLAS & Run 2              & 19.1 & 22.7 & 130 & 135\\
  %  ATLAS & 3 \invab Conservative  & 8.8 & 15.1 & 50 & 51\\
  %  ATLAS & 3 \invab Intermediate   & 5.2 & 12.9 & 28 & 29\\
  %  ATLAS & 3 \invab High-yield       & 4.9 & 12.6 & 25 & 26\\
  %  \hline\hline
  %\end{tabular}
  \begin{tabular}{ l c c c }
    \hline\hline
    & & $\BR(B^0_s \to \mu^+ \mu^-)$ & $\BR(B^0 \to \mu^+ \mu^-)$ \\
   Experiment & Scenario & stat + syst $\%$ & stat + syst $\%$ \\
    \hline
    LHCb   & 23\invfb          &   8.2  &  33       \\
    LHCb   & 300\invfb         &   4.4  &  9.4      \\
     CMS   & 300\invfb         &   12   &  46  \\
     CMS   & 3 \invab          &   7    &  16  \\
    ATLAS & Run 2              & 22.7 & 135\\
    ATLAS & 3 \invab Conservative  & 15.1 & 51\\
    ATLAS & 3 \invab Intermediate   &  12.9 & 29\\
    ATLAS & 3 \invab High-yield       & 12.6 & 26\\
    \hline\hline
  \end{tabular}
\end{table}

%The HL-LHC projected relative statistical uncertainty of the current LHCb analysis is 1.8\%.  
At the end of the Upgrade II data taking period, LHCb assumes to achieve an overall uncertainty on $\BR(\decay{\Bs}{\mumu})$ of about $4.4\%$, which
would imply an uncertainty on $\BR(\decay{\Bs}{\mumu})$ to be approximately $0.30\times 10^{-9}$ with 23\invfb and $0.16\times 10^{-9}$ with $300\invfb$.  The LHCb reach on the ratio of branching fractions ${\cal B}(\decay{\Bd}{\mumu})/{\cal B}(\decay{\Bs}{\mumu})$ is expected to remain limited by statistics and decrease from $90\%$ for the current measurement to $\sim 34\%$ with $23\invfb$ and $\sim 10\%$ with $300\invfb$. 

%The CMS study is performed under the assumption of maintaining the current trigger selection for the signal and normalization channel. The major sources of systematic uncertainties for the CMS extrapolation are listed in table~\ref{tab:CMSb2musys}. The uncertainty on the normalization channel branching ratio is assumed to be improved by the Belle collaboration measurements to an uncertainty of 1.4\% . CMS determines the uncertainty on the muon identification(ID) efficiency ratio
%from $\mathrm{B^+ \to  J}/\psi K^+$ and $\mathrm{B^0_s \to J}/\psi \phi$, extrapolating its value at the HL-LHC to be 1\%, and the relative uncertainties on the yield of the semileptonic and peaking backgrounds - arising from a hadron mis-identification uncertainty of 10\% - to be 20\% and 15\% in Run 2. These same uncertainties are extrapolated to reduce by an additional factor 2 for the HL-LHC era. The trigger efficiency is estimated by the difference between the dimuon triggers for signal and for j/$\psi$ which is assumed to diminish to 1.5 \%. Other sources of systematics (acceptance, selection, etc) have been studied and the related uncertainties halved: the sensitivity for  the significance of the $\mathrm{B^0}$ observation and its branching fraction is not significantly affected whereas there is a 1 \% improvement on the sensitivity for the $\mathrm{B^0_s}$ branching fraction. The quadratic sum of these uncertainties is 3\%.
The CMS projections are obtained by
 extrapolating the Run~2 analysis performance to the HL-LHC scenario. Trigger efficiencies comparable to those of Run~2, with manageable rates, are expected to be attained \cite{CMS-PAS-FTR-14-015}
 and are here assumed. The effect of the increased pileup on the signal selection efficiency was also found to be manageable. The inner tracker of the CMS HL-LHC detector is estimated to provide a 40-50\% improvement relative to Run~2 on the dimuon mass resolution. This results in an improved separation of the $B^0_s$ and $B^0$ signals, lowering the signal cross-feed contamination that is specially crucial for the $B^0$ observation, and a reduction of the level of the semileptonic background in the signal region (see Fig.~\ref{fig:GPDMass}). With the full HL-LHC expected integrated luminosity of 3000  fb$^{−1}$, CMS expects to measure the 
 $B^0_s \to \mu^+ \mu^-$  branching fraction at the level of 7\%.
  precision, and observe 
  the $B^0 \to \mu^+\mu^-$ decay with a significance in excess of  $5\sigma$. 
Fig.~\ref{fig:CMSFitMass} shows the invariant mass fit projections
for the $B_{s,d} \to \mu^+\mu^-$ analyses for an integrated luminosity of 3000 fb$^{-1}$. 
 \begin{figure}[t]
   \begin{center}
     \includegraphics[width=0.45\textwidth]{\main/section7/figures/CMS_mass_barrel.pdf}\hbox{\hspace{-0.2cm}}
     \includegraphics[width=0.45\textwidth]{\main/section7/figures/CMS_mass_forward.pdf}
      \caption{
      Projected dimuon invariant mass distributions with overlaid fit results for the $B \to \mu^+\mu^-$ analyses by CMS for an integrated luminosity of  3000 fb$^{-1}$.  Events where the most forward muon lies in the  barrel (left) and forward (right)  regions of the detector display different mass resolutions and are categorized accordingly in the analysis. The SM relative $B^0_s$ and $B^0$ contributions are here assumed. (Plots taken from \cite{CMS-PAS-FTR-18-013}).}
     \label{fig:CMSFitMass}\end{center}\end{figure}\setlength{\extrarowheight}{4pt}     

The ATLAS projections are extrapolated from the ATLAS Run~1 analysis \cite{Aaboud:2016ire}. Assumptions include the training of a multivariate classifier capable of similar background rejection and signal purities and an analysis selection with comparable pile-up immunity as Run~1. The study takes into account the scaling of $B$ production cross-section and integrated luminosity relative to Run~1, and explores different triggering scenarios corresponding to different dimuon transverse momentum thresholds, ($p_T^{\mu_1}$,$p_T^{\mu_2}$): $(6 \,\text{GeV},6 \,\text{GeV})$, $(6 \, \text{GeV},10 \,\text{GeV})$ and $(10 \,\text{GeV},10 \,\text{GeV})$. For each of these scenarios the sensitivity is categorized on the basis of the signal statistics expected relative
to the Run~1 analysis (x15, x60 and x75 respectively, in $3\ab^{-1}$ of HL-LHC ATLAS data), yielding the projected $68.3\%$, $95.5\%$ and $99.7\%$ likelihood contours in Fig.~\ref{fig:ATLAS_HLLHCextrapolation}.

\begin{figure}[t]
  \centering  
  \includegraphics[width=0.33\textwidth]{\main/section7/figures/plot_preliminary_x15.pdf}\hbox{\hspace{-0.2cm}}
  \includegraphics[width=0.33\textwidth]{\main/section7/figures/plot_preliminary_x60.pdf}\hbox{\hspace{-0.2cm}}
  \includegraphics[width=0.33\textwidth]{\main/section7/figures/plot_preliminary_x75.pdf}
  \caption{ATLAS projected 68.3\% (solid), 95.5\% (dashed) and 99.7\% (dotted) confidence level profiled likelihood ratio contours for the ``conservative'' (top), ``intermediate'' (middle) and ``high-yield'' HL-LHC extrapolations.
    Red contours do not include the systematic uncertainties, which are then included
    in the blue ellipsoids. The black point shows the SM theoretical prediction and its uncertainty \cite{Bobeth}.}
  \label{fig:ATLAS_HLLHCextrapolation}
\end{figure}


%\td{JZ: why this statement? It is quite unclear what is meant here, and why MSSM, it comes out of the blue} 

\begin{figure}[htb]
  \centering  
  \includegraphics[width=0.5\textwidth]{\main/section7/figures/B2MuMu_HL-LHC.pdf}\hbox{\hspace{-0.2cm}}
  \caption{$B^0_s \to \mu^+ \mu^-$ and $B^0 \to \mu^+ \mu^-$ branching ratios as computed using new sources of flavour-changing neutral currents, as discussed in Ref.~\cite{Dutta:2015dla}. The green points are the subset consistent with other measurements. The black cross point is the SM prediction, while the coloured contours show the expected 1-sigma HL-LHC sensitivites of ATLAS, CMS, and LHCb.}
  \label{fig:HLCombinedReachFromVava}
\end{figure}

Fig.~\ref{fig:HLCombinedReachFromVava} compares the projected experimental sensitivities of ATLAS, CMS, and LHCb with the BR predictions from a particular class of BSM models~\cite{Dutta:2015dla}.   
All estimates use the quoted SM predictions as central values for the branching ratios. The estimated experimental sensitivity at HL-LHC is close to the uncertainty of the current SM prediction from theory, which is dominated by the uncertainty on the $\Bs$ decay constant, $f_{B_s}$, determined from lattice QCD calculations, and the CKM matrix elements. Both are expected to improve in precision in the future. The power of the HL-LHC dataset to discriminate not only between the SM and BSM models, but also within the parameter space of those BSM models, is clear. %{\bf\color{red}[Discussion on systematics above from ATLAS and CMS unclear]}

With the HL-LHC datasets, precise measurements of additional observables are possible, namely the effective lifetime, $\tau_{\mu\mu}^{\rm eff}$, and the time-dependent \CP\ asymmetry of \Bsmm decays. Both quantities are sensitive to possible new contributions from the scalar and pseudo-scalar sector in a way complementary to the branching ratio measurement\cite{DeBruyn:2012wk}. The effective lifetime is related to the mean \Bs lifetime $\tau_{B_s}$ through the relation 
\begin{equation}
\tau_{\mu\mu}^{\rm eff}=\frac{\tau_{B_s}}{1-y^2_s}\frac{1+2A^{\mu\mu}_{\Delta\Gamma}y_s+y^2_s}{1+A^{\mu\mu}_{\Delta\Gamma}y_s}\,, 
\end{equation}
where $y_s=\tau_{B_s}\Delta\Gamma_s/2$, and $\Delta\Gamma_s=\Gamma_{B^0_{sL}}-\Gamma_{B^0_{sH}}$. In the SM,  $A^{\mu\mu}_{\Delta\Gamma}=1$, with only the heavy mass eigenstate decaying to \mumu. In BSM scenarios it
 can take any value between $-1$ and 1. LHCb has performed the first measurement of the \Bsmm effective lifetime using a dataset of 4.4\invfb, resulting in $\tau_{\mu\mu}^{\rm eff}=2.04\pm 0.44\pm 0.05 \ps$~\cite{LHCb-PAPER-2017-001} (Fig.~\ref{fig:btomumu}, right). The relative uncertainty on $\tau_{\mu\mu}^{\rm eff}$ is expected to decrease to approximately $8\%$  with 23\invfb and $2\%$ with 300\invfb, being statistically limited. 
%CMS effective lifetime

The CMS sensitivity for a measurement of the
 $B^0_s \to \mu ^+ \mu^-$ effective 
lifetime is estimated using an ensemble of  pseudo-experiments generated with parameters reflecting the projected Phase~2 conditions. 
The signal lifetime distribution for each pseudo-experiment is obtained using the sPlot technique\cite{splot} to separate out the background, and then fitted with a model consisting of an exponential function, convolved with a Gaussian function that describes the expected decay time resolution, and multiplied by an efficiency function that accounts for reconstruction effects. The outcome of such a pseudo-experiment is shown in Fig.~\ref{fig:CMS_lifetime}.
The effective lifetime is expected to be measured with a statistical precision of 3\% at 3000 fb$^{-1}$.
%A projection at 300 fb$^{-1}$ results in a statistical precision of 9\%.
Fig.~\ref{fig:CMS_lifetime} shows the $B^0_s$ decay time distribution with the fit projection superimposed, for a simulated statistics comparable
to the HL-LHC expectations. 

\begin{figure}[t]
\begin{center}
\includegraphics[width=0.4\textwidth]{\main/section7/figures/CMS_tau_splot_toy_bs.pdf}
\caption{The CMS binned maximum likelihood fit to the background subtracted $B^0_s \to \mu ^+ \mu^-$ decay time distribution for the HL-LHC scenario (taken from \cite{CMS-PAS-FTR-18-013}). The effective lifetime from the fit is 1.61 $\pm$ 0.05 ps.}
 \label{fig:CMS_lifetime}\end{center}\end{figure}\setlength{\extrarowheight}{4pt}     

While the current experimental uncertainty is larger than for $\tau_{B^0_{sH}}-\tau_{B^0_{sL}}$,
a $2-3\%$ uncertainty on $\tau_{\mu\mu}^{\rm eff}$ would allow to set stringent constraints on $A^{\mu\mu}_{\Delta\Gamma}$ and in particular would allow to break the degeneracy between any possible contribution from new scalar and pseudoscalar mediators.

Assuming a tagging power of about $3.7\%$~\cite{LHCb-PAPER-2014-059}, a dataset of 300\invfb allows LHCb to reconstruct a pure sample of more than 100 flavour-tagged \Bsmm decays (effective yield) and measure their time-dependent \CP asymmetry. From the relation  
\begin{equation}
\frac{\Gamma(B^0_s(t)\to\mu^+\mu^-)-\Gamma(\bar{B}^0_s\to\mu^+\mu^-)}{\Gamma(B^0_s(t)\to\mu^+\mu^-)+\Gamma(\bar{B}^0_s\to\mu^+\mu^-)}=\frac{S_{\mu\mu}\sin(\Delta m_st)}{\cosh(y_st/\tau_{B_s})+A^{\mu\mu}_{\Delta\Gamma}\sinh(y_st/\tau_{B_s})}\,, 
\end{equation}
where $t$ is the signal proper time and $\Delta m_s$ is the mass difference of the heavy and light \Bs mass eigenstates, $S_{\mu\mu}$ can be measured with an uncertainty of about 0.2. The signal yield expected in a 23\invfb dataset, on the other hand,  is too low to allow a meaningful constraint to be set on $S_{\mu\mu}$. A nonzero value for $S_{\mu\mu}$ would automatically indicate evidence of \CP-violating phases beyond the SM.  

%The CMS experiment ability of measuring the %$\mathrm{B^0_s} \to \mu^+\mu^-$ effective lifetime will be %limited by the knowledge 
%of the trigger efficiency as a function of the %to be well measured 
 %with the $\mathrm{B^+} \to J/\psi K^+$ decay using %similar triggers. 

Being sensitive to a wider set of effective operators ($O_7$, $O_9$ and $O_{10}$)~\cite{Kruger:2002gf}, the \bsmumugamma decay offers 
an interesting counterpart to \bsmumu. 
The theoretical branching fraction is expected to be one order of magnitude 
larger than the \bsmumu one~\cite{Melikhov:2004mk}, owing to the removal of the helicity suppression, 
when integrated over the full $q^2$ spectrum. 
However, the presence of the photon makes the direct reconstruction 
challenging at LHCb.
No limit exist today on the \bsmumugamma channel, 
while the \bdmumugamma is limited at $1\times 10^{-7}$ at 90\% CL by the 
\babar experiment~\cite{Aubert:2007up}.

Given the experimental difficulty, two complementary techniques are employed  
for the study of the \bsmumugamma decay at LHCb. The first is a full reconstruction
which is more sensitive at low and mid $q^2$, where the photon energy is higher, 
and the second, recently proposed in Ref.~\cite{Dettori:2016zff}, without photon reconstruction
but only sensitive at high $q^2$. 

The only non-negligible partially reconstructed background is the not yet 
measured $B_d \to \mu^{+} \mu^{-} \pi^{0} \xspace$, whose branching fraction is 
theoretically estimated to be of the same order of magnitude as the signal.
The main difficulty of the measurement is therefore the combinatorial 
background, because the uncertainty on the photon momentum enlarges the signal 
width and blurs its kinematics.
Based on current reconstruction efficiencies, the expected sensitivity at the 
end of Run~3 (Upgrade II) is $\sim 9 \sigma$ ($\sim 22 \sigma$).
The use of $B_s \to J/\psi \eta \xspace$ and $B_d \to K\pi\gamma \xspace$ as 
normalisation channels reduces the systematic uncertainties due to the selection.

The partially reconstructed method consists of studying the \bsmumugamma decay 
as a shoulder on the left of the \bsmumu peak in the dimuon mass distribution. 
The SM contribution as background has been considered negligible so far.
Conversely, large branching fractions could be easily excluded~\cite{Dettori:2016zff}
when considering this as an additional component. 
The SM branching fraction for this region would be around $2 \times 10^{-10}$, 
implying a first observation would be possible with Run~3 and certainly with Run~4 data, 
while extremely tight limits could already be determined with Run~2. 

\subsubsection{Measurements of $b\to s\ell\ell$ from LHCb/ATLAS/CMS}

\subsubsubsection{Yield and systematics evolution}

With the large data set that will be collected at the end of Run~5,   it will
be possible to make a precise determination of the angular observables in narrow
bins of \qsq or using a \qsq-unbinned approach~\cite{Hurth:2017sqw,Chrzaszcz:2018yza}; LHCb foresees to achieve around 440\,000 fully reconstructed \decay{\Bz}{\Kstarz\mumu} decays, CMS around 700\,000 excluding the $q^2$ range 
 overlapping with the resonant decays \cite{CMS-PAS-FTR-18-033}. 
 
 The current CMS and LHCb measurements are statistically limited.
 For LHCb 
 %the measurement is currently heavily statistically limited, and
 the major systematics are those which affect the calculation of the angular acceptance. Most of the systematic uncertainties are expected to reduce significantly with more integrated luminosity due to larger control samples. Both the LHCb statistical and systematic uncertainties are therefore scaled with integrated luminosity when obtaining the projections. In this context it is interesting to note that \upgradetwo will provide signal yields that are of the order of the current tree-level control modes $B^0\to \jpsi \Kstarz$ and $B^0_s\to \jpsi\phi$. %The most recent determination of $\phis$ is done via a time-dependent angular analysis of $B^0_s\to \jpsi\phi$. {\bf\color{red}[Check]}
 In the most recent measurement of $\phis$ from $B^0_s\to \jpsi\phi$, the systematic uncertainties on the decay amplitudes due to modeling of the angular acceptance were already at the 0.001-0.002 level. Therefore, even without considering further improvements from the larger control samples available in \upgradetwo, these uncertainties will not systematically limit the \upgradetwo analysis.
 
 The CMS sensitivity for the measurement of the $P_5'$ parameter at HL-LHC is extrapolated \cite{CMS-PAS-FTR-18-033} 
  from the CMS Run~1 results \cite{Sirunyan:2017dhj} under some assumptions: effects of improvements in the analysis strategy (e.g. different selection criteria or fits) are not considered and the trigger thresholds and efficiency are assumed to remain the same.  This latter is likely to be a conservative assumption since the availability of tracking information at the first level trigger may result in a higher efficiency than in Run~1. The extrapolation method also assumes that the signal-to-background is the same; indeed
  the main 
  source of background is from other b hadron decays, whose cross-section scales the same as the signal.
 Samples of simulated events are used to evaluate three relevant aspects of the analysis: mass resolution, CP mistagged rate, and the effect of pileup to justify the extrapolation method; no degradation in the projected analysis performance was found.
  For each $q^2$ bin, the expected $B^0 \to K^{*0}\mu^+\mu^-$ signal yields are obtained from a sample of simulated signal events generated with the Phase~2 conditions, including an average of 200 pileup, and scaled to the integrated luminosity of 3000 fb $^{-1}$.
%An extended unbinned maximum likelihood fit is performed to the $K^+\pi^-\mu^+\mu^-$ invariant mass in each bin, parametrizing the signal with 
%the sum of two Gaussian distributions and the background with an exponential distribution.
The estimated  statistical uncertainty on the ${P'_5}$  parameter at 3000 fb$^{-1}$ is obtained by scaling the statistical uncertainty measured in Run~1 
by the square root of the ratio between the yields observed in Run~1 and the Phase~2 simulation.
 The evolution of the systematic uncertainties is also extrapolated from the Run~1 analysis.
 Improved understanding of theory and the experimental apparatus is expected to reflect in a factor of 2 reduction in many uncertainties in the Phase~2 scenario. 
 %These uncertainties are those related to contamination from resonant decays, signal mass shape, CP mistagging rate, efficiency, angular resolution, and other simulation modeling.
The uncertainties which depend on the available amount of data
%, i.e., the description of the background mass distribution, those associated with the propagation of the uncertainty on $F_L, F_S$ abd $A_S$ () and the fit bias introduced by the fitting procedure %
are scaled the same as the statistical uncertainties. 
The uncertainty related to the limited number of simulation events is neglected, under the assumption that sufficiently large simulation samples will be available by the time the HL-LHC becomes operational.
%A complete discussion can be found in %\cite{CMS-PAS-FTR-18-033}. 
%Uncertainties concerning the simulation mismodeling,  efficiency determination,
 %$K\pi$ mistagging rate, choice of the signal mass PDF, effect of the contamination from feed-through events from the resonant decays,
 %and angular resolution are assumed to be reduced by a factor of 2 in higher statistics scenarios; 
 % uncertainties associated to the fit bias introduced by the fitting procedure, description of the background mass distribution and propagation of the uncertainty 
 %on $F_L$, $F_S$ and $A_S$ depend on the available amount of data and are scaled by the ratio of the square root 
 %of the number of events. The uncertainty related to the limited number of simulation events is neglected, under the assumption that sufficiently large simulation samples  will be available by the time of future analyses.
  

 
 For both the Run~3 and HL-LHC datasets, the precision of the  $b\to s\ell\ell$ branching fraction measurements will be limited by the knowledge of the \decay{\B}{\jpsi X} decay modes that are used to normalise the observed signals. 
The knowledge of these branching fractions will be improved by the \belletwo collaboration but will inevitably limit the precision of the absolute branching fractions of rare $\bquark\to\squark \ellell$ processes. 
The comparison between the predicted and measured branching fractions will in any case be limited by the theoretical knowledge of the form factors.
 
 \subsubsubsection{Sensitivity projections}

In order to estimate the sensitivity to BSM effects in $b\to s\ell\ell$ decays, a number of benchmark NP scenarios are considered, see Table~\ref{tab:penguins:NPscenarios}. 
Scenarios~I and II are inspired by the current discrepancies. The first scenario is the one that best explains the present $b\to s\mu\mu$ decay data. The second is the best explaining the rare semileptonic measurements within a purely left-handed scenario. This requirement is theoretically well motivated and arises in models designed to simultaneously explain the discrepancies seen in both tree-level semitauonic and loop-level semileptonic decays. The third and fourth scenarios assume that the current discrepancies are not confirmed but there is instead a small contribution from right-handed currents that would not be visible with the current level of experimental precision. These scenarios will serve to illustrate the power of the large LHCb \upgradetwo data set to distinguish between different NP models. This power relies critically on the ability to exploit multiple related decay channels. 

\begin{table}[!tb]
   \centering
\caption{
    Wilson coefficients in benchmark NP scenarios. The first scenario is inspired by the present discrepancies in the
    rare decays, including the angular distributions of the decay \decay{\Bz}{\Kstarz\mumu} and the measurements of the branching fraction ratios $R_K$ and $R_{K^*}$. 
    The second
    scenario is inspired by the possibility of explaining the
    rare decays discrepancies and those measured in the observables
    $R(D^{(*)})$. The third and fourth scenarios assume small nonzero
    right-handed couplings.} 
            \label{tab:penguins:NPscenarios}
   \begin{tabular}{crrrr}
       \hline\hline
            scenario & $C_9^{\rm NP}$ & $C_{10}^{\rm NP}$ & $C_{9}^{\prime}$ & $C_{10}^{\prime}$ \\
       \hline
       I & $-1.4$ & 0 & 0 & 0 \\
       II & $-0.7$ & 0.7 & 0 & 0\\
       III & 0 & 0& 0.3 & 0.3\\
       IV & 0 & 0& 0.3 & $-0.3$\\
       \hline\hline
   \end{tabular}
\end{table}

%The branching fractions of a number of exclusive $\bquark\to\squark \ellell$ processes have been studied using the LHCb Run~1 data set. The experimental measurements are systematically lower than their corresponding SM predictions. The largest discrepancy appears for $\BF(\decay{\Bs}{\phi\mumu})$ which, in the  $1<q^2<6\gevgevcccc$ region, is more than $3\,\sigma$ from the SM predictions~\cite{LHCb-PAPER-2015-023}.
Unlike the systematics limited branching fractions, a more precise comparison between theory and experiment can be achieved by studying isospin and \CP asymmetries, which will be experimentally probed at percent-level precision with the LHCb \upgradetwo data set. This will also enable new decay modes to be studied, for example higher spin \Kstar states and modes with larger numbers of decay products. It is also possible to reduce theoretical and experimental uncertainties by comparing regions in angular phase-space of $\bquark\to\squark \ell\ell$ decays. The angular distribution of \decay{\B}{V \ellell} decays, where $V$ is a vector meson, can be expressed in terms of eight \qsq-dependent angular coefficients that depend on the Wilson coefficients and the form factors. 
Measurements of angular observables in \decay{\Bz}{\Kstarz\mumu} decays show a
discrepancy with respect to the SM predictions~\cite{LHCb-PAPER-2013-037,LHCb-PAPER-2015-051,LHCb-PAPER-2014-006,LHCb-PAPER-2013-019,Aubert:2006vb,Lees:2015ymt,Wei:2009zv,Aaltonen:2011ja,Chatrchyan:2013cda,Khachatryan:2015isa,Aaboud:2018krd,Sirunyan:2017dhj,Beneke:2004dp,Egede:2015kha,Kruger:2005ep,Lyon:2014hpa,Hurth:2013ssa,Beaujean:2013soa,Altmannshofer:2013foa,Descotes-Genon:2013wba}. 
This discrepancy is largest in the so-called optimized observable $P'_{5}$~\cite{LHCb-PAPER-2015-051}. 
The decay \decay{\Bs}{\phi\mumu} can also be described by the same angular formalism as the \decay{\Bz}{\Kstarz\mumu} decay.
However, in this case the \Bs and \Bsb mesons decay to a common final state and it is not possible to determine the full set of observables   without tagging the initial flavour of the \Bs. 
%This is discussed further in Sec.~\ref{sec:penguin:timedependent}. 
 
 %\subsubsubsection{CMS P5' sensitivity}
%\par\noindent
 
The expected precision of an unbinned LHCb-only determination of $P_5^{\prime}$ in the SM and in Scenarios~I and II is illustrated in Fig.~\ref{fig:penguin:P5primeUnbinned}, where we have followed the theoretical approaches in Refs.~\cite{Hurth:2017sqw,Chrzaszcz:2018yza} for the predictions. 
\upgradetwo will enable these scenarios to be clearly separated from the SM and from each other.
By combining information from all of the angular observables in the decay, it will also be possible for LHCb to distinguish models with much smaller NP contributions. 
Fig.~\ref{fig:penguin:WCKstmm} shows the expected $3\,\sigma$ sensitivity to NP in the Wilson
coefficients $C_{9,10}^{\prime}$ assuming the central values for the SM, Scenario~III and Scenario~IV.  
These scenarios are also clearly distinguishable with the precision that will be available with the \upgradetwo data set.

\begin{figure}[!tb]
\centering
\includegraphics[width=0.45\linewidth]{section7/figures/P5p_Run3.pdf}
\includegraphics[width=0.45\linewidth]{section7/figures/P5p_HL.pdf}
\caption{Total experimental sensitivity, including systematics, at LHCb to the $P_5^{\prime}$ angular observable in the SM,
  Scenarios~I and II for  the Run~3 (left) and  the \upgradetwo (right) data sets. 
  The sensitivity is computed assuming that the charm-loop contribution is determined from the data. %{\bf\color{red}[What do these errorrs contain? Experimental only?]}
}
\label{fig:penguin:P5primeUnbinned}
\end{figure}

\begin{figure}[!tb]
\centering
\includegraphics[width=0.4\linewidth]{section7/figures/ellipses_Cp_Run3.pdf}
\includegraphics[width=0.4\linewidth]{section7/figures/ellipses_Cp_HL.pdf}
\caption{Expected sensitivity to the Wilson coefficients $C_9^{\prime}$ and
  $C_{10}^{\prime}$ from the future LHCb analysis of the  \decay{\Bz}{\Kstarz\mumu} decay. 
  The ellipses correspond to 3\,$\sigma$ contours for
  the SM, Scenario~III and Scenario~IV for  the Run~3 (left) and  the \upgradetwo (right) data sets. 
  }
  \label{fig:penguin:WCKstmm}
\end{figure}

\begin{figure}[t]
  \begin{center}
  \includegraphics[width=0.6\textwidth]{section7/figures/R1_and_errorP2_rebinningSystButStatOnly.pdf}
 \caption{Projected statistical (hatched regions) and total (open box) uncertainties on the CMS $P'_5$ parameter versus $q^2$ in the Phase~2 scenario at 3000 fb$^{-1}$. The CMS Run~1 measurement of $P'_5$ is also shown by circles with inner vertical bars representing the statistical uncertainties and outer vertical bars representing the total uncertainties.
The vertical shaded regions correspond to the $J/\Psi$ and $\psi'$ resonances.
The two lower pads represent the  statistical (upper pad) and total (lower pad)  uncertainties with the finer $q^2$ binning. (The plot is taken from \cite{CMS-PAS-FTR-18-033}).} 
 \label{fig:CMSP5presults}
 \end{center}
 \end{figure}
 
The CMS projected statistical uncertainties and total uncertainties for the measurement of $P'_5$ versus $q^2$ for an integrated luminosity of 3000 fb$^{-1}$ are shown in Fig.~\ref{fig:CMSP5presults} along with the Run~1 results. 
%CMS Improvements

The $P_5'$ total uncertainties in the $q^2$ bins are estimated to improve up to a factor of 15 in the  3000 fb$^{-1}$ scenario \cite{CMS-PAS-FTR-18-033}, compared to those quoted in the Run~1 analysis.
The extrapolation to 300 fb$^{-1}$ integrated luminosity results in an improvement of up to a factor of 7 with respect to the Run~1 analysis.
The foreseen Phase~2 total integrated luminosity offers the opportunity to perform the angular analysis in narrower $q^2$ bins, in order to measure the $P'_5$  shape as a function of $q^2$ with finer granularity. The $q^2$ region below the $J/\psi$ mass(-squared), which is more sensitive to possible new physics effects, is considered. Each Run~1 $q^2$ bin is split into smaller and equal-size bins to achieve a statistical uncertainty of the order of the total systematic uncertainty in the same bin with the additional constraint of having a bin width at least 5 times larger than the dimuon mass resolution $\sigma_r$, as measured 
on the MC simulation as a function of $q^2$.
If both conditions cannot be satisfied, then only the looser requirement on the 5 $\sigma_r$ bin width is imposed. 
%The dimuon mass resolution is measured on the MC simulation as a function of $q^2$.
With respect to the Phase~2 systematic uncertainties with wider bins, the systematic uncertainties that were scaled the same as the statistical uncertainties are adjusted to account for less data in each finer bin while the other uncertainties are unchanged.
The corresponding  statistical and total uncertainties on $P'_5$  are shown in the lower two pads of Fig.~\ref{fig:CMSP5presults}.

The analysis in narrow $q^2$ bins provides a better determination of the $P_5'$ parameter shape which will allow us to test theoretical predictions. 
The CMS projection is  for the single $P_5'$ angular parameter; however, it is worth noting that, with the foreseen HL-LHC statistics, CMS will have the capability to perform a full angular analysis of the \decay{\Bz}{\Kstarz\mumu} decay mode.

%The major challenge for \decay{\B}{V \ellell} decays is to disentangle NP effects from SM contributions. 
Furthermore,  with a large data set it will  be possible to probe \decay{\B}{V \ellell} SM contributions, under the premise that a genuine NP contribution is expected to have no $q^2$ dependence, while, \eg a 
charm loop contribution is expected to grow when approaching the pole of
the charmonia resonances.  
A measurement using Breit-Wigner functions to parametrise the
resonances, and their interference with the
short-distance contributions to the decay, was proposed in Ref.~\cite{Blake:2017fyh}. 
A similar technique has already been applied to the Run~1 data for the \decay{\Bp}{\Kp\mumu} decay~\cite{LHCb-PAPER-2016-045}. 
An alternative approach using additional phenomenological inputs has also been proposed~\cite{Bobeth:2017vxj}.
Such a combination of phenomenological and experimental
methods may improve our knowledge of the charm-loop contribution and form factors, which would allow $C_9^\mu$ and $C_{10}^\mu$ to be determined with great precision in $\bquark\to\squark\mu\mu$ transitions.


\subsubsection{Measurements of $b\to dll$}

Thanks to LHCb's particle identification capabilities, the \upgradetwo data set will provide a unique opportunity to make precise measurements of $\bquark\to\dquark \ell\ell$ processes.
Using the Run~1 and 2 data sets, LHCb data have been used to observe the decays \decay{\Bp}{\pip\mumu}~\cite{LHCb-PAPER-2012-020,LHCb-PAPER-2015-035} and \decay{\Lb}{p\pim\mumu}~\cite{LHCb-PAPER-2016-049}, and to find evidence for the decays \decay{\Bz}{\pip\pim\mumu} (in a $\pip\pim$ mass region that  is expected to be dominated by \decay{\Bz}{\rhoz\mumu}) and \decay{\Bs}{\Kstarzb\mumu}~\cite{LHCb-PAPER-2018-004} with branching fractions at the $\mathcal{O}(10^{-8})$ level.
The existing data samples comprise $\mathcal{O}(10)$ decays in these decay modes. 
The upgrade will provide samples of thousands, or tens of thousands of such decays.
The ability to measure the properties of these processes depends heavily on the PID performance of the LHCb subdetectors. 
In the case of the \decay{\Bs}{\Kstarzb\mumu} decay, excellent mass resolution is also critical to separate \Bs and \Bz decays. 

The ratio of branching fractions between the CKM-suppressed $b \to d\ell\ell$ transitions and their CKM-favoured  $\bquark \to \squark \ellell$ counterparts, together with theoretical input on the ratio of the relevant form factors, enables the ratio of CKM elements  $|V_{td}|/|V_{ts}|$ to be determined. The precision on $|V_{td}|/|V_{ts}|$ from such decays is dominated at present by the statistical uncertainty on the experimental measurements of \decay{\Bp}{\pip\mumu}, and is much less precise than the determination from mixing measurements. The theoretical uncertainty at high-$q^2$ is at the level of $4\%$ %{\bf\color{red}[Where does this number come from? Long-distance is even more serious than in $b\to s\mu\mu$]} 
and is expected to improve with further progress on the form factors from lattice QCD~\cite{Du:2015tda}. Around 17\,000 \decay{\Bp}{\pip\mumu} decays are expected in the full 300\invfb dataset, allowing an experimental precision better than $2\%$.

The current set of measurements of $\bquark\to\squark \ell\ell$ processes have demonstrated the importance of angular measurements in the precision determination of Wilson coefficients. With the LHCb \upgradetwo dataset, where a sample of 4300  \decay{\Bs}{\Kstarzb\mumu} decays is expected,  it will be possible to make a full angular analysis of a $\bquark\to\dquark \ellell$ transition. The \decay{\Bs}{\Kstarzb\mumu} decay is both self-tagging and has a final state involving only charged particles. The LHCb \upgradetwo data set will allow the angular observables in this decay to be measured with better precision than the existing measurements of the \decay{\Bz}{\Kstarz\mumu} angular distribution. 

The LHCb \upgradetwo dataset will also give substantial numbers of \decay{\B^{0,+}}{\rho^{0,+}\mumu}  and \decay{\Lb}{N\mumu} decays. Although the  \decay{\Bz}{\rhoz\mumu} decay does not give the flavour of the initial \B meson, untagged measurements will give sensitivity to a subset of the interesting angular observables. 
Analysis of the \decay{\Lb}{N^{*}\mumu} decay will require statistical separation of  overlapping $p\pim$ resonances with different $J^P$ by performing an amplitude analysis of the final-state particles. 

The combination of information from $\BF(\decay{\Bz}{\mumu})$, the differential branching fraction of the \decay{\Bp}{\pip\mumu} decay, and angular measurements, notably of \decay{\Bs}{\Kstarzb\mumu}, will indicate whether NP effects are present in $\bquark\to\dquark$ transitions at the level of $20\%$ of the SM amplitude with more than $5\sigma$ significance. 

\subsubsection{LFU tests in $b\to (s,d)\ell\ell$}

The Run~1 \lhcb data have been used to perform the most precise measurements of $R_{K}$ and $R_{\Kstar}$ to-date~\cite{LHCb-PAPER-2017-013,LHCb-PAPER-2014-024} (see Fig.~\ref{fig:penguin:RXscenarios}).
These measurements are compatible with the SM at the level of 2.1--2.6 standard deviations. Assuming the current detector performance, approximately 46\,000 \decay{\Bp}{\Kp\epem} and 20\,000 \decay{\Bz}{\Kstarz\epem} candidates are expected in the range $1.1 < \qsq < 6.0\gevgevcccc$ in the \upgradetwo data set.
The ultimate precision on $R_{K}$ and $R_{\Kstar}$ will be better than 1\%.
The importance of the \upgradetwo data set in distinguishing between different NP scenarios is highlighted in Fig.~\ref{fig:penguin:RXscenarios}. 
With this data set all four NP scenarios could be distinguished at more than 5\,$\sigma$ significance.

The \upgradetwo data set will also enable the measurement of other $R_{X}$ ratios \eg $R_{\phi}, R_{p\kaon}$ and the ratios in CKM suppressed decays. 
For example, with 300\invfb, it will be possible to determine $R_{\pi}=\BR(\decay{\Bp}{\pip\mumu})/\BR(\decay{\Bp}{\pip\epem})$ with a few percent statistical precision.
A summary of the expected performance for a number of different $R_X$ ratios is indicated in Table~\ref{tab:penguin:LU_extrapolations}.

\begin{figure}[!t] 
\centering
\includegraphics[width=0.75\textwidth]{section7/figures/RX_scenario.pdf}
\caption{
Projected sensitivity for the $R_{K}$, $R_{K^*}$ and $R_{\phi}$ measurements in different NP scenarios with the Upgrade~II data set. 
The existing Run~1 measurements of $R_{K}$ and $R_{K^*}$ are shown for comparison. 
}
\label{fig:penguin:RXscenarios}
\end{figure}

\begin{table}[!tb]
\caption{
Estimated yields of $\bquark\to\squark\ep\en$ and $\bquark\to\dquark\ep\en$ processes and the statistical uncertainty on $R_{X}$ in the range $1.1<\qsq<6.0\gevgevcccc$ extrapolated from the Run~1 data. 
A linear dependence of the \bbbar production cross section on the $pp$ centre-of-mass energy and unchanged Run~1 detector performance are assumed. 
Where modes have yet to be observed, a scaled estimate from the corresponding muon mode is used.
}
\label{tab:penguin:LU_extrapolations}
\centering
\begin{tabular}{lrrrrr}
\hline\hline
Yield  & Run~1 result & 9\invfb & 23\invfb & 50\invfb & 300\invfb \\
\hline
 \decay{\Bu}{\Kp\epem} 		& $254 \pm 29$\cite{LHCb-PAPER-2014-024} & 1\,120 & 3\,300 & 7\,500 & 46\,000 \\
 \decay{\Bd}{\Kstarz\epem}	& $111 \pm 14$\cite{LHCb-PAPER-2017-013} & 490 & 1\,400 & 3\,300 & 20\,000 \\
 \decay{\Bs}{\phi\epem}		& -- & 80 & 230 & 530 & 3\,300 \\
 \decay{\Lb}{\proton\kaon\epem} & -- & 120 & 360 & 820 & 5\,000 \\
\decay{\Bp}{\pip\epem}		& -- & 20 & 70 & 150 & 900 \\ 
\hline
$R_X$ precision &  Run~1 result & 9\invfb & 23\invfb & 50\invfb & 300\invfb \\
\hline
$R_{\kaon}$					& $0.745 \pm 0.090\pm 0.036$\cite{LHCb-PAPER-2014-024} & 0.043 & 0.025 & 0.017 & 0.007 \\
$R_{\Kstarz}$					& $0.69 \pm 0.11\pm 0.05$\cite{LHCb-PAPER-2017-013} & 0.052 & 0.031 & 0.020 & 0.008 \\
$R_{\phi}$					& -- & 0.130 & 0.076 & 0.050 & 0.020 \\
$R_{\proton\kaon}$  & -- & 0.105 & 0.061 & 0.041 & 0.016 \\
$R_{\pi}$						& -- & 0.302 & 0.176 & 0.117 & 0.047 \\ 
\hline\hline
\end{tabular}
\end{table}

In addition to improvements in the $R_{X}$ measurements, the enlarged \upgradetwo data set will give access to new observables. For example, the data will allow precise comparisons of the angular distribution of dielectron and dimuon final-states. 
Differences between angular observables in \decay{\B}{X\mumu} and \decay{\B}{X\ep\en} decays are theoretically pristine~\cite{Capdevila:2016ivx,Serra:2016ivr}
and are sensitive to different combinations of Wilson coefficients compared to the $R_X$ measurements. 
Fig.~\ref{fig:penguin:DeltaC} shows that an upgraded \lhcb detector will enable such decays to be used to discriminate between different NP models, for example separating between Scenarios~I and II~\cite{Mauri:2018vbg}. 
Excellent NP sensitivity can be achieved irrespective of the assumptions made about the hadronic contributions to the decays.

\begin{figure}[t!]
\centering
\includegraphics[width=0.49\textwidth]{section7/figures/ellipses_DeltaC.pdf}
\caption{
  Constraints on the difference in the $C_9$ and $C_{10}$ Wilson coefficients from angular analyses of the electron and muon modes with the Run~3 and \upgradetwo data sets.
  The $3\sigma$ regions for the Run~3 data sample are shown for the SM (solid blue), a vector-axial-vector new physics contribution (red dotted) and for a purely vector new physics contribution (green dashed). 
  The shaded regions denote the corresponding constraints for the \upgradetwo data set.
}
\label{fig:penguin:DeltaC}
\end{figure}

In the existing \lhcb detector, electron modes have an approximately factor five lower efficiency than the corresponding muon modes, owing to the tendency for the electrons to lose a significant fraction of their energy through bremsstrahlung in the detector. 
This loss impacts on the ability to reconstruct, trigger and select the electron modes.
The precision with which observables can be extracted therefore depends primarily on the electron modes and not the muon modes.
In order for $R_{X}$ measurements to benefit from the large \upgradetwo data samples, it will be necessary to reduce systematic uncertainties to the percent level.
These uncertainties are controlled by taking a double ratio between $R_{X}$ and the decays \decay{\B}{\jpsi X}, where the \jpsi decays to \mumu and $\ep\en$.
This approach is expected to work well, even with very large data sets.  

Other sources of systematic uncertainty can be mitigated through design choices for the upgraded detector.
The recovery of bremsstrahlung photons is inhibited by the ability to find the relevant photons in the ECAL (over significant backgrounds) and by the energy resolution.
A reduced amount of material before the magnet would reduce the amount of bremsstrahlung and hence would increase the electron reconstruction efficiency and improve the electron momentum resolution. 
Higher transverse granularity would aid signal selection and help reduce the backgrounds. 
With a large number of primary $pp$ collisions, the combinatorial background will increase and will need to be controlled with the use of timing information.
However, the Run~1 data set indicates that it may be possible to tolerate a significant (\ie\ larger than a factor two) increase in combinatorial backgrounds without destroying the signal selection ability.

\subsubsection{Time dependent angular analyses in $b\to (s,d)ll$}

Time dependent analyses of rare decays into \CP-eigenstates can deliver orthogonal experimental information to time-integrated observables. 
So far, no time-dependent measurement of the $\decay{\Bs}{\phi\mumu}$ decay has been performed due to the limited signal yield of $432\pm 24$ in the Run~1 data sample~\cite{LHCb-PAPER-2015-023}.
However, the larger data samples available in \upgradetwo will enable time-dependent studies. 
The framework describing $\Bbar$ and $\B\to V\ellell$ transitions to a common final-state is discussed in  Ref.~\cite{Descotes-Genon:2015hea}, 
where several observables are discussed that can be accessed with and without flavour tagging.  
Two observables called $s_8$ and $s_9$, which are only accessible through a time-dependent flavour-tagged analysis, are of particular interest. 
These observables  are proportional to the mixing term $\sin\left(\Delta m_s t\right)$ and provide information that is not available through flavour specific decays.
Assuming a time resolution of around $45\fs$ and an effective tagging power of $5\%$ 
results in an effective signal yield of  $2000$ decays for the \upgradetwo data set. 

As a first step towards a full time-dependent analysis, the effective lifetime of the decay $\decay{\Bs}{\phi\mumu}$ can be studied. 
The untagged time-dependent decay rate is given by
\begin{align}
  \frac{\deriv\Gamma}{\deriv t} &\propto  e^{-\Gamma_s}\left[\cosh\left(\frac{\Delta\Gamma_st}{2}\right)+A^{\Delta\Gamma}\sinh\left(\frac{\Delta\Gamma_st}{2}\right)\right].
\end{align}
The observable $A^{\Delta\Gamma}$ can be related to the angular observables $F_{\rm L}$ and $S_3$ via $A^{\Delta\Gamma}=2S_3-F_{\rm L}$. 
Due to the significant lifetime difference $\Delta\Gamma_s$ in the \Bs\ system, even an untagged analysis can probe right-handed currents. 
For the combined low- and high-\qsq regions, preliminary studies suggest a statistical sensitivity to $A^{\Delta\Gamma}$ of $0.05$ can be achieved with a $300\invfb$ data set. 

With the \upgradetwo data set it will also be possible to perform a time-dependent angular analysis of the $\bquark \to \dquark$ process \decay{\Bd}{\rho^0\mumu}. 
This process differs from \decay{\Bs}{\phi\mumu} in two important regards: it is CKM suppressed and therefore has a smaller SM branching fraction; and $\Delta \Gamma_d \approx 0$, removing sensitivity to $A^{\Delta\Gamma}$. 
The uncertainties on the angular observables are expected to be of the order of $0.1$ for this case. 

The time-dependent angular analyses will still be statistically limited even with $300\invfb$. 
It will be important to maintain good decay-time resolution and the performance of the particle identification will be crucial to control backgrounds, as well as to improve flavour tagging performance. 

\subsubsection{Measurements of $b\to s\gamma$}

The time dependent \CP asymmetry of \decay{\B}{f_{\CP}\gamma} arises from the interference between decay amplitudes with and without $\Bds-\Bdsb$ mixing and is predicted to be small in the SM\ \cite{Atwood:1997zr,Ball:2006cva,Matsumori:2005ax}.
As a consequence, a large asymmetry due to interference between the \B mixing and decay diagrams can only be present if the two photon helicities contribute to both \B and \Bb decays.
From the time dependent decay rate
\begin{align}
    \Gamma(\decay{\Bds(\Bdsb)}{f_{\CP}\gamma})(t) \sim e^{-\Gamma_s t}
    &\big[
        \cosh\left(\frac{\Delta\Gamma_{(s)}}{2}\right)
        - \mathcal{A}^{\Delta}\sinh\left(\frac{\Delta\Gamma_{(s)}}{2}\right) \pm \nonumber\\
        & \pm \mathcal{C}_\CP\cos\left(\Delta m_{(s)} t\right)
        \mp \mathcal{S}_\CP\sin\left(\Delta m_{(s)} t\right)
    \big],
\end{align}
where $A^\Delta$, $\mathcal{C}_\CP$ and $\mathcal{S}_\CP$ depend on the photon polarisation~\cite{Muheim:2008vu}.
Two strategies can be devised:
one studying the decay rate independently of the flavour of the \B meson, which allows $A^\Delta$ to be accessed, and one tagging the flavour of the \B meson, which accesses $\mathcal{S}_\CP$ and $\mathcal{C}_\CP$.
The first strategy has been exploited at \lhcb to study the $4000$ \decay{\Bs}{\phi\gamma} candidates collected in Run 1 to obtain $A^\Delta=-0.98^{+0.46}_{-0.52}\stat ^{+0.23}_{-0.20}\syst$~\cite{LHCb-PAPER-2016-034}, compatible at two standard deviations from the prediction of $A^\Delta_{\text{SM}}=0.047^{+0.029}_{-0.025}$.

With $\sim60$k signal candidates expected with $50\,\invfb$, the full analysis, including flavour tagging information, will improve the statistical uncertainty on $A^\Delta$ to $\sim0.07$, and will need a careful control of the systematic 
uncertainties.
The analysis performed with $\sim800$k signal decays expected with $300\,\invfb$, with a statistical uncertainty to $\sim0.02$, requires some of the possible improvements in \piz reconstruction of the \upgradetwo detector 
to be able to use the full statistical power of the data.

In addition to studying the \Bs system, \lhcb can study the time-dependent decay rate of $\Bz\to\KS\pip\pim\gamma$ decays, which permits access of the photon polarisation through the $\mathcal{S}_\CP$ term.
With $\mathcal{O}(1000)$ signal events in Run 1, around $35$k and $200$k are expected at the end of Run~4 and \upgradetwo, respectively
($1.75$k and $10$k when considering the flavour tagging efficiency), opening the doors to a very competitive measurement of $\mathcal{S}_\CP$ in the \Bz system.

Another way to study the photon polarisation is through the angular correlations among the three-body decay products of a kaonic resonance in \decay{\B}{K_{\mathrm{res}}(\to K\pi\pi)\gamma},
which allows the direct measurement of the photon polarisation parameter in the effective radiative weak Hamiltonian~\cite{Gronau:2002rz}. 
As a first step towards the photon polarisation measurement, \lhcb observed nonzero photon polarisation for the first time by studying the photon angular distribution in bins of $\Kp\pim\pip$ invariant mass\ \cite{LHCb-PAPER-2014-001}, but the determination of the value of this polarisation could not be performed due to the lack of knowledge of the hadronic system.
To overcome this problem, a method to measure the photon polarisation using a full amplitude analysis of \decay{\B}{K\pi\pi\gamma} decays is currently under development~\cite{Bellee:2018}, with an expected statistical sensitivity on the photon polarisation parameter of $\sim5\%$ in the charged mode with the Run 1 dataset.
The extrapolation of the precision to $300\,\invfb$  results in a statistical precision better than $1\%$, and hence control of the systematic uncertainties will be crucial.

The polarisation of the photon emitted in $b\to s\gamma$ transitions can also be accessed via semileptonic $b\to s\ell\ell$ transitions, for example in the decay $B^0\to\Kstarz\ell^+\ell^-$. Indeed, as mentioned in 
previous sections, at very low $q^2$ these decays are dominated by the electromagnetic dipole operator ${\cal O}_7^{(\prime)}$. Namely, the longitudinal polarisation fraction ($\FL$) is expected to be below $20\%$ for $q^2<0.2\gevgevcccc$.
In this $q^2$ region, the angle $\phi$ between the planes defined by the dilepton system and the $\Kstarz\to K^+\pi^-$ decay is sensitive to the $b\to s\gamma$ photon helicity.

While the $\Kstarz\mumu$ final state is experimentally easier to select and measure at LHCb, the $\Kstarz\epem$ final state allows $q^2$ values below $4m_\mu^2$ to be probed, 
where the sensitivity to the photon helicity is maximal.
Compared to the radiative channels used for polarisation measurements, the $B^0\to\Kstarz\epem$ final state is fully charged and gives better mass resolution and therefore better separation from partially reconstructed backgrounds.

The sensitivity of this decay channel at LHCb was demonstrated by an angular analysis performed with Run 1 data~\cite{LHCb-PAPER-2014-066}. The angular observables most sensitive to the photon polarisation at low $q^2$ are $A_{\rm T}^{(2)}$ and $A_{\rm T}^{\rm Im}$, as defined in Ref.~\cite{Becirevic:2011bp,LHCb-PAPER-2014-066}. 
Indeed, in the limit $q^2\to 0$, these observables can be expressed by the following functions of $C_{7\gamma}^{(\prime)}$ (assuming NP contributions to be much smaller than $|C_{7\gamma}^{\rm SM}|$):
\begin{equation}
  \label{eq:qSqToZero} 
  A_{\rm T}^{(2)} (\qsq \to 0) \simeq 2 \frac{\Real (C_7^{' \ast}) } {| C_7 |}  \quad {\rm and} \quad  A_{\rm T}^{\rm Im} (\qsq \to 0) \simeq 2 \frac{\Imag (C_7^{' \ast}) } {| C_7 |}.
\end{equation}
In order to maximise the sensitivity to the photon polarisation, 
the angular analysis should be performed as close as possible 
to the low $q^2$ endpoint. However, the events at extremely low $q^2$ have worse $\phi$ resolution (because the two electrons are almost collinear) and are polluted by $B^0\to\Kstarz\gamma$ decays with the $\gamma$ converting in the VELO material. In the Run 1 analysis~\cite{LHCb-PAPER-2014-066} the minimum required $m(\epem)$ was set at 20\mevcc, but this should
be reduced as the \upgradetwo VELO detector will have a significantly lower material budget (multiple scattering is the main effect worsening the $\phi$ resolution). 
Similarly, the background from $\gamma$ conversions will be reduced with a lighter RF-foil or with the complete removal of it in \upgradetwo~\cite{Aaij:2244311}.

Using the signal yield as given in Table~\ref{tab:penguin:LU_extrapolations} leads to the following statistical sensitivities to $A_{\rm T}^{(2)}$ and $A_{\rm T}^{\rm Im}$: 12\% with 8\invfb, 7\% with 23\invfb and 2\% with 300\invfb.
The theoretical uncertainty induced when this observable is translated into a photon polarisation measurement is currently at the level of $2\%$ but should improve by the time of the \upgradetwo analyses.
The current measurements performed with Run 1 data have a systematic uncertainty of order $5\%$ coming mainly from the modelling of the angular acceptance and from the uncertainty on the angular shape of the combinatorial background. The acceptance is independent of $\phi$ at low $q^2$ and its
modelling can be improved with larger simulation samples and using the proxy channel $\Bd\to\Kstarz\jpsi(\to\epem)$. 

Weak radiative decays of \bquark baryons are largely unexplored, with the best limits coming from CDF: $\mathcal{B}(\decay{\Lb}{\Lz\gamma}) < 1.3\times10^{-3}$ at $90\%\,$CL~\cite{Acosta:2002fh}.
They offer a unique sensitivity to the photon polarisation through the study of their angular distributions, and will constitute one of the main topics in the radiative decays programme in the \lhcb \upgradetwo.

With predicted branching fractions of $O(10^{-5}-10^{-6})$, the first challenge for \lhcb will be their observation, as the production of long-lived particles in their decay, in addition to the photon, means in most cases that the $b$-baryon secondary vertex cannot be reconstructed. This makes their separation from background considerably more difficult than in the case of regular radiative \bquark decays.

The most abundant of these decays is \decay{\Lb}{\Lz(\to p\pim)\gamma}, which is sensitive to the photon polarisation mainly\footnote{In the following, we assume that the \Lb (and any other beauty baryon) polarisation is zero~\cite{LHCb-PAPER-2012-057}, removing part of the photon polarisation dependence.} through the distribution of the angle between the proton and the \Lz momentum in the rest frame of the \Lz ($\theta_p$), 
\begin{equation}
    \frac{\operatorname{d}\Gamma}{\operatorname{d}\cos\theta_p} \propto 1 - \alpha_\gamma \alpha_{p,1/2}\cos\theta_p, 
\end{equation}
where $\alpha_\gamma$ is the asymmetry between left- and right-handed amplitudes and $\alpha_{p,1/2}=0.642\pm0.013$~\cite{PDG2018} is the \decay{\Lz}{p\pim} decay parameter.
Using specialised trigger lines for this mode $15-150$ signal events are expected using the Run 2 dataset. Preliminary studies show that a statistical sensitivity to $\alpha_\gamma$ of $(20-25)\%$ is expected with these data, which would be reduced to $\sim15\%$ with $23\,\invfb$ and below $4\%$ with $300\,\invfb$.
In the \lhcb \upgradetwo, the addition of timing information in the calorimeter will be important to be able to study this combinatorial-background dominated decay;
additionally, improved downstream reconstruction would allow the use of downstream \Lz decays, which make up more than $2/3$ of the total signal.

The \decay{\Xib}{\Xi^-(\to\Lz(\to p\pim)\pim)\gamma} decay presents a richer angular distribution, with dependence to the photon polarisation in both the \Lz angle ($\theta_\Lz$) and proton angle ($\theta_p$),
\begin{equation}
    \frac{\operatorname{d}\Gamma}{\operatorname{d}\cos\theta_\Lz\cos\theta_p} \propto 1 - \alpha_\gamma \alpha_\Xi \cos\theta_\Lz + \alpha_{p,1/2}\cos\theta_p\left(\alpha_\Xi-\alpha_\gamma\cos\theta_\Lz\right),
\end{equation}
but the lower $\sigma(\decay{pp}{\Xib})$, combined with a lower reconstruction efficiency due to the presence of one extra track, results in an order of magnitude fewer events than in the \Lb case, making the increase of 
statistics from the \upgradetwo even more relevant. With a similar sensitivity to the photon polarisation to that of \decay{\Lb}{\Lz(\to p\pim)\gamma}, \decay{\Xib}{\Xi^-\gamma} decays will allow this parameter to be probed with a precision of $40\%$ and $10\%$ with $23$ and $300\,\invfb$, respectively.

\subsubsection{Measurements of $b\to c\ell\nu$ including $B_c$ and $b$-baryon prospects}
\label{sec:bclnu:exp}

\begin{figure}[!tb]
\centering
\includegraphics[width=0.7\linewidth]{section7/figures/RX_projection_stat_syst.pdf}
\caption{
The projected absolute uncertainties on $\mathcal{R}(D^{\ast})$ and $\mathcal{R}(J/\psi)$ from the current sensitivities (at 3\invfb) to 23\invfb, 50\invfb, and 300\invfb.
}
\label{fig:RXproj}
\end{figure} 

LHCb has made measurements of \RDb using both muonic ($\tau^{+} \to \mup \nu \nu$) and hadronic ($\tau^{+} \to \pip \pim \pip \nu$) decays of the tau lepton~\cite{LHCb-PAPER-2015-025,LHCb-PAPER-2017-017,LHCb-PAPER-2017-027}.
Due to the presence of multiple neutrinos these decays are extremely challenging to measure.
The measurements rely on isolation techniques to suppress partially reconstructed backgrounds, 
\B meson flight information to constrain the kinematics of the unreconstructed neutrinos, and a multidimensional template fit to determine the signal yield. 
Fig.~\ref{fig:RXproj} shows how the absolute uncertainties on the LHCb muonic and hadronic $\mathcal{R}(D^{\ast})$ measurements are projected to evolve with respect to the current status.
The major uncertainties are the statistical uncertainty from the fit, the uncertainties on the background modelling and the limited size of simulated samples.
A major effort is already underway to commission fast simulation tools.
The background modelling is driven by a strategy of dedicated control samples in the data, and so this uncertainty will continue to improve with larger data samples.
From Run 3 onward it is assumed that, taking advantage of the full software HLT, the hadronic analysis can normalise directly to the $B^0 \to D^{\ast -}\mu^+\nu_{\mu}$  decay,
thus eliminating the uncertainty from external measurements of $\mathcal{B}(B^0 \to D^{\ast -} \pi^+\pi^-\pi^+)$.
It is assumed that all other sources of systematic uncertainty will scale as $\sqrt{\mathcal{L}}$.
With these assumptions, an absolute uncertainty on $\mathcal{R}(D^{\ast})$ of $0.003$ will be achievable for the muonic and hadronic modes with the 300\invfb Upgrade II dataset.

On the timescale of \upgradetwo, interest will shift toward new observables  beyond the branching fraction ratio~\cite{Becirevic:2016hea}.
The kinematics of the \dsttaunu decays is fully described by the dilepton mass, and three angles which are denoted $\chi$, $\theta_L$ and $\theta_D$. 
LHCb is capable of resolving these three angles, as can be seen in~\figref{fig:RDAngles}.
However, the broad resolutions demand very large samples to extract the underlying physics. The decay distributions within this kinematic space are governed by the underlying spin structure, and precise measurements of these distributions will allow the different helicity amplitudes to be disentangled.
This can be used both to constrain the Lorentz structure of any potential NP contribution, and to measure the hadronic parameters
governing the \dsttaunu decay, serving as an essential baseline for SM and non-SM studies.
The helicity-suppressed amplitude which presently dominates the theoretical uncertainty on \RDb is too strongly suppressed in the 
\dbmunu decays to be measurable, however this can be accessed in the \dbtaunu decay directly.
If any potential NP contributions are assumed not to contribute via the helicity-suppressed amplitude then the combined measurements of \dbmunu and \dbtaunu decays will allow for a fully data-driven prediction for
\RDb under the assumption of lepton universality, eliminating the need for any theory input relating to hadronic form factors. 
However, these measurements have yet to be demonstrated with existing data.
This exciting programme of differential measurements needs to be developed on Run~1 and 2 data before any statement is made about the precise sensitivity, 
but it offers unparalleled potential to fully characterise both the SM and non-SM contributions to the $b \to c \tau \nu$ transition.

\begin{figure}[!tb]
\centering
\includegraphics[width=0.45\linewidth]{section7/figures/thetaD.pdf}
\includegraphics[width=0.45\linewidth]{section7/figures/thetaL.pdf}
\includegraphics[width=0.45\linewidth]{section7/figures/thetaC.pdf}
\caption{
Angular resolution for simulated \dstmunu (black) and \dsttaunu  (red) decays, with $\tau^{+} \to \mup \nu \nu$. 
This demonstrates our ability to resolve the full angular distribution, with some level of statistical dilution.
}
\label{fig:RDAngles}
\end{figure} 

As measurements in \RDst become more statistically precise, it will become increasingly more 
urgent to provide supplementary measurements in other \bquark-hadron species with different background 
structure and different sources of systematic uncertainties. For example, the \decay{\Bsb}{\Dsp \taum\neub} and $\Bsb\to\Dssp\taum\neub$ decays
will allow supplementary 
measurements at high yields, and do not suffer as badly from cross-feed backgrounds from other 
mesons, unlike, for example, \decay{\Bzb }{ \Dstarp \taum\neub}, where the \Bp and \Bs both contribute to 
the $\Dstarp \mu X$ or $\Dstarp \threepim X$ final states. 
Furthermore, the comparison of decays with different spins of the $b$ and $c$ hadrons can enhance our sensitivity 
to different NP scenarios~\cite{Azatov:2018knx,LHCb-PAPER-2017-016}.
No published measurements exist for the $\Bs$ case yet, but based on known relative efficiencies and assuming 
the statistical power of this mode tracks \RDorDst, we expect less than $6\%$ relative uncertainty 
after Run~3, and $2.5\%$ with the \upgradetwo data, where limiting systematic uncertainties 
are currently expected to arise from corrections to 
simulated pointing and vertex resolutions, from knowledge of particle identification efficiencies, 
and from knowledge of the backgrounds from random combinations of charm and muons. It is conceivable 
that new techniques and control samples could further increase the precision of these measurements. 

Methods 
are currently under development for separating the \decay{\Bsb}{\Dssp\ell^{-}\neub} and \decay{\Bsb}{\Dsp\ell^{-}\neub} modes, and given the
relative slow pion (\decay{\Dstarp }{ \Dz\pip}) and soft photon (\decay{\Dssp }{ \Dsp\gamma}) efficiencies, the precision in \decay{\Bsb}{ \Dsp \tau \nu} decays can be expected to exceed that in
\decay{\Bsb }{ \Dssp \tau \nu}, the reverse of the situation for $\RDorDst$.
An upgraded ECAL would extend the breadth and sensitivity of $\mathcal{R}(D^{*(*)+}_{s})$ measurements possible
in the \upgradetwo\ scenario above and beyond the possible benefits of improved neutral isolation in 
$\RD$ or $\mathcal{R}(\Dsp)$ measurements.

Of particular interest are the semitauonic decays of
\bquark baryons and of \Bcp mesons.
The former provides probes of entirely new Lorentz structures of NP operators
which pseudoscalar to pseudoscalar or vector transitions simply do not access.
The value of probing this supplementary space of couplings has already been demonstrated by
LHCb with its Run~1
measurement of $|\Vub|$ via the decay \decay{\Lb}{\proton\mun\neub}, which places strong constraints on 
right-handed currents sometimes invoked to explain the inclusive-exclusive tensions in that quantity. By the end of Run~3, it is expected that the relative uncertainty for $\mathcal{R}(\Lc)$ will reach below $4\%$, and $2.5\%$ by the end of \upgradetwo. 
A further exciting prospect is the study of $b \to u\mu \nu$ decays, which have been beyond experimental reach thus far.
For example the decay $B^+ \to p\bar{p}\mu\nu$ offers a clean experimental signature. 
Our capabilities with this decay could benefit from the enhanced low momentum proton identification with the TORCH subdetector.

Meanwhile, the \decay{\Bcp }{ \jpsi \taum \neub} decay
is an entirely unique state among the flavoured mesons as the bound state of two distinct 
flavors of heavy quark, and, through its abundant decays to charmonium final states, provides a 
highly efficient signature for triggering and reconstruction at high instantaneous luminosities. 
Measurements of \decay{\Bcp }{ \jpsi \ell \neub} decays involve a trade-off between the approximately 100 times 
smaller production cross-section for \Bcp verses the extremely efficient \decay{\jpsi}{\mup\mun} signature in the LHCb 
trigger. For illustration, in Run~1, LHCb reconstructed 
and selected 19\,000 \decay{\Bcp }{ \jpsi \mun \neub} decays, compared with 360\,000 \decay{\Bzb}{\Dstarp\mun\neub}. This resulted in a measurement of $\mathcal{R}(\jpsi)=0.71\pm 0.17\pm 0.18$ \cite{LHCb-PAPER-2017-035}. As a result of the smaller production cross-section, the muonic measurements have large backgrounds from \decay{h}{ \mu} misidentification from the relatively abundant \decay{\B }{ \jpsi X_h} decays, where $X_h$ is any collection of hadrons, and so they
are very sensitive to the performance of the muon system and PID algorithms in the future. Here it is assumed that 
it will be possible to achieve similar performance to Run 1 in the upgraded system. 

To project the sensitivity for \decay{\Bcp }{ \jpsi \taum \neub} based on Ref.~\cite{LHCb-PAPER-2017-035}, it is assumed that 
all the systematic uncertainties can be reduced with the size of the input data except for those that were assumed not to
scale with data for the previous predictions. For these, we assume that they can be reduced down until they reach 
the same absolute size as the corresponding systematic uncertainties in the Run~1 muonic $\RDst$ analysis. In addition, it is
assumed that sometime in the 2020s lattice QCD calculations of the form factors for this process will allow 
the systematic uncertainty due to signal form factors to be reduced by an additional factor of two. This results in
a projected absolute uncertainty for the muonic mode of $0.07$ at the end of Run~3 
and $0.02$ by the end of \upgradetwo,  as can be seen in Fig.~\ref{fig:RXproj}.
Measurements in the hadronic mode can be expected to reach similar sensitivities.

\subsubsection{Searches for LFV, LNV, BNV and interplay with tests of LU}

The LHCb collaboration has recently published~\cite{LHCb-PAPER-2017-031} the world's best limits on the branching fractions of the \BsToEMu
and \BdToEMu decays using the first 3\invfb collected in 2011 and 2012 at 7 and 8\tev respectively. The acceptance of the \BsToEMu decays can be affected by the relative contribution of the two \Bs mass eigenstates to the total decay amplitude, due to their large lifetime difference. Therefore, the upper limit on the branching fraction of \BsToEMu decays is evaluated in two extreme hypotheses: where the amplitude is completely dominated by the heavy eigenstate or by the light eigenstate. The results are  $\BF(\BsToEMu) < 6.3\,(5.4) \times 10^{-9}$ and $\BF(\BsToEMu) < 7.2\,(6.0) \times 10^{-9}$ at $95\%\,(90\%)$ CL, respectively. The limit for the branching fraction of the \Bd mode is  $\BF(\BdToEMu) < 1.3\,(1.0) \times 10^{-9}$ at $95\%\,(90\%)$ CL.

Assuming similar performances in background rejection and signal retention as in the current analysis, at the end of Run~4 the LHCb
experiment will be able to probe branching fractions of \BsToEMu and \BdToEMu decays down to $8\times 10^{-10}$
and $2\times 10^{-10}$, respectively. The additional statistics accumulated during the \upgradetwo data taking period will push down
these limits to $3\times 10^{-10}$ and $9\times 10^{-11}$ respectively, close to the interesting region where NP effects may appear. The \upgradetwo improvement in electron  reconstruction will be very important in attaining, or exceeding, this goal. 

An upper limit on the \BdToTauMu channel has been already set by $\babar$: $\BR\left( \BdToTauMu\right)< 2.2\times 10^{-5} $ at $90\%$~CL~\cite{Aubert:2008cu}. The first search on the \BsToTauMu channel is in progress in LHCb and the results are expected soon on data recorded in 2011 and 2012 using the \tauppp and \tauppppi0 decay modes.
Given the presence of a neutrino that escapes detection this kind of analysis is much more complicated than those investigating electron or muon final states. A specific reconstruction technique is used in order to infer the energy of the $\nu$, taking advantage of the known $\tau$ vertex position given by the $3\pi$ reconstructed vertex. 
This way, the complete kinematics of the process can be solved up to a two-fold ambiguity. LHCb expects to reach sensitivities of a few times $10^{-5}$ with the Run~1 and 2 data sets. Extrapolating the current measurements to the \upgradetwo LHCb could reach  $\BR\left( \BdToTauMu\right)< 3\times 10^{-6} $ at $90\%$~CL. The mass reconstruction technique depends heavily on the uncertainty on the primary and the $\tau$ decay vertices, hence improvement in the tracking system in \upgradetwo, including a removal or reduction in material of the VELO RF foil, will be very valuable.

In many generic NP models with LFUV, CLFV decays of $b$-hadrons can be linked with the anomalies recently measured in $b\rightarrow s\ell\ell$ decays~\cite{LHCb-PAPER-2014-024,LHCb-PAPER-2017-013,LHCb-PAPER-2015-051}. If NP indeed allows for CLFV then the branching fractions of $B\to K \ell \ell^{'}$  or $\Lb \to \Lz \ell \ell^{'}$ will be enhanced with respect to their purely leptonic counterparts, since the helicity suppression is lower. Furthermore, if observed, they would allow the measurement of more observables with respect to the lepton flavour violating decays discussed in the previous sections, thanks to their multi-body final states and, in the case of $\Lb$, to the non-zero initial spin. 

The current  limits set by the \bfactories on the branching fractions of $B \to Ke \mu$ and $B \to K\tau \mu$ decays are $< 13\times 10^{-8}$~\cite{Aubert:2006vb} and $< 4.8\cdot 10^{-5}$~\cite{Lees:2012zz}  at 90\% confidence level, respectively.  

At LHCb, searches for $\decay{\Bu}{\Kp e^\pm\mu^\mp}$, $\decay{\Bd}{\Kstarz\tau^\pm\mu^\mp}$, $\decay{\Bu}{\Kp\tau^\pm\mu^\mp}$ and $\decay{\Lb}{\Lz e^\pm\mu^\mp}$ are ongoing. These searches are complementary, as charged lepton flavour violation couplings among different families are expected to be different.
The analyses involving $\tau$ leptons reconstruct candidates via the $\decay{\taum}{\pim\pim\pip\nu_{\tau}}$ channel,
which allows the reconstruction of the $\tau$ decay vertex.\footnote{It should be noted that searches for $\decay{\Bu}{\Kp\tau^\pm\mu^\mp}$ from $B_{s2}^*$ without $\tau$ reconstruction can give complementary information.} 
All these decays contain at least one muon, which is used to efficiently trigger on the event. Usually, since these decays involve combinations of leptons that are not allowed in the SM, the backgrounds can be kept well under control,  leaving very clean samples only polluted by candidates formed by the random combinations of tracks. This combinatorial effect is higher for the channel with a $\tau$ in the final state decaying into three charged pions.  The other relevant background  comes from chains of semileptonic decays, where two or more neutrinos are emitted and therefore combinations of leptons of different flavours are possible. These decays have typically a low reconstructed invariant mass, due to the energy carried away by the neutrinos, and so they do not significantly pollute the signal region. 

The expected upper limits at LHCb using the first $9 \invfb$ of data taken are
$\mathcal{O}(10^{-9})$ and $\mathcal{O}(10^{-6})$ for the $\decay{\Bu}{\Kp e^\pm\mu^\mp}$ and $\decay{\Bd}{\Kstarz\tau^\pm\mu^\mp}$ decays respectively, at 90\% confidence level.
The limit for $\decay{\Bu}{\Kp\tau^\pm\mu^\mp}$ is expected to be similar to $\decay{\Bd}{\Kstarz\tau^\pm\mu^\mp}$.
The sensitivity of these analyses scales almost linearly with luminosity  for $\decay{\Bu}{\Kp e^\pm\mu^\mp}$, 
and with the square root of the luminosity for $\decay{\Bd}{\Kstarz\tau^\pm\mu^\mp}$.
In both cases, the expected limits using the \upgradetwo data are in the region of interest of the models currently developed for explaining the $B$ anomalies, so they will provide strong constraints on the NP scenarios with CLFV 

Experimentally, Lepton-Number Violating (LNV) and Baryon-Number Violating (BNV) measurements are null searches, so sensitivity is assumed
to scale linearly with luminosity $\mathcal L$ when the background
is negligible and as $\sqrt \mathcal L$ if the background is significant. LHCb has already published searches in certain channels, and others are
in progress:
\begin{itemize}
  \item[$\bullet$] Searches for LNV in various $B$-meson decays of the form
    $B \to X \mu^+ \mu^+$, where $X$ is a system of one or more hadrons.
    The principal motivation is the sensitivity to contributions from Majorana neutrinos~\cite{Atre:2009rg},
    which may be on-shell or off-shell, depending on the decay mode.
    The published results consist of
      searches for $\decay{\Bp}{\Km \mu^+ \mu^+}$, $\decay{\Bp}{\pim \mu^+ \mu^+}$ and $\decay{\Bu}{D^+_{(s)}\mun\mun}$~\cite{LHCb-PAPER-2011-009, LHCb-PAPER-2011-038, LHCb-PAPER-2013-064}. A limit of $\mathcal{B}(B^+ \to \pi^- \mu^+ \mu^+) < 4 \times 10^{-9}$ is set at the 95\% confidence level, along with more detailed limits as a function of the Majorana neutrino mass. Since the combinatorial background was found to be low but not negligible with the Run~1 data, we estimate that the limit can be improved by a factor of ten with the full \upgradetwo dataset.
  \item[$\bullet$] Search for BNV in \Xibz oscillations~\cite{LHCb-PAPER-2017-023}.
    Six-fermion, flavour-diagonal operators, involving two fermions from each
    generation, could give rise to BNV without violating the nucleon stability
    limit~\cite{Smith:2011rp,Durieux:2012gj}.
    Since the \Xibz~($bsd$) has one valence quark from each generation, it could
    couple directly to such an operator and oscillate to \Xibzbar.
    The published search used the Run1 data and set a lower limit on the oscillation
    period of 80\ps. Since events are tagged by decays of
    the \XibPrimeMinus and \XibStarMinus resonances, with the former being particularly
    clean, and since the analysis also uses the decay-time distribution of events,
    the sensitivity is expected to scale linearly. 
    Although the decay mode used in the published analysis is hadronic
    ($\Xibz \to \Xicp \pim$), future work could also benefit from the
    lower-purity but higher-yield semileptonic mode $\Xibz \to \Xicp \mun \neumb$.
  \item[$\bullet$] $\Lc \to \antiproton \mu^+ \mu^+$.
    This channel has previously been investigated at the $e^+ e^-$ \bfactories.
    The current upper limit, obtained by \babar~\cite{Lees:2011hb}, is
    $\mathcal{B}(\Lc \to \antiproton \mup \mup) < 9.4 \times 10^{-6}$
    at the 90\% confidence level.
    With Run~1 and~2 data alone, it should be possible to reduce this
    to $1 \times 10^{-6}$. Further progress depends on the background
    level, but an additional factor of 5--10 with the full \upgradetwo statistics
    is likely.
  \item[$\bullet$] $\Lc \to \mu^+ \mu^- \mu^+$.
    Experimentally, this is a particularly promising decay mode:
    the final state with three muons is very clean, and there are
    no known sources of peaking background.
    This search could be added for little extra effort to the
    $\taum \to \mup \mun \mun$ search described in the
    preceding section.
\end{itemize}
