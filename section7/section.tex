% don't remove the folling lines, and edit the defintion of \main if needed
\documentclass[../report.tex]{subfiles}
\providecommand{\main}{..}
\IfEq{\jobname}{\currfilebase}{\AtEndDocument{\biblio}}{}
% until here
\def\bsmumu{\ensuremath{B^0_s \to \mu^{+} \mu^{-}}\xspace}
\def\bsmumugamma{\ensuremath{B^0_s \to \mu^{+} \mu^{-} \gamma}\xspace}
\def\bdmumugamma{\ensuremath{B^0 \to \mu^{+} \mu^{-} \gamma}\xspace}
\def\epm    {{\ensuremath{e^\pm}}\xspace}
\def\emp    {{\ensuremath{e^\mp}}\xspace}
\def\mupm   {{\ensuremath{\mu^\pm}}\xspace}
\def\mump   {{\ensuremath{\mu^\mp}}\xspace}
\def\BToEMu    {\decay{\B}{\epm\mump}}
\def\BdToEMu   {\decay{\Bd}{\epm\mump}}
\def\BsToEMu   {\decay{\Bs}{\epm\mump}}
\newcommand{\subdecay}[2]{\ensuremath{#1(\!\to #2)}\xspace}
% Particles
\def\mupm   {{\ensuremath{\mu^\pm}}\xspace}
\def\mump   {{\ensuremath{\mu^\mp}}\xspace}
\def\taupm   {{\ensuremath{\tau^\pm}}\xspace}
\def\taump   {{\ensuremath{\tau^\mp}}\xspace}
%% Decays
\def\tauppp{\decay{\taupm}{\pi^\pm\pi^\mp\pi^\pm\nu}}
\def\tauppppi0{\decay{\taupm}{\pi^\pm\pi^\mp\pi^\pm\piz\nu}}
\def\BToTauMu    {\decay{\B}{\taupm\mump}}
\def\BdToTauMu   {\decay{\Bd}{\taupm\mump}}
\def\BsToTauMu   {\decay{\Bs}{\taupm\mump}}
\def\BdToTauMupm   {\decay{\Bd}{\taupm\mump}}
\def\BsToTauMupm   {\decay{\Bs}{\taupm\mump}}
\def\BdsToTauMu   {\decay{\B^0_{\left( s\right) }}{\taupm\mump}}
\def\BdsToTauMuSS {\decay{\B^0_{\left( s\right) }}{\taupm\mupm}}
\def\Btaumu{\decay{\B}{\subdecay{\taupm}{\pipm\pimp\pipm\nu}\mump}}
\def\Bstotaumu{\decay{\Bs}{\subdecay{\taupm}{\pipm\pimp\pipm\nu}\mump}}
\def\Bdtotaumu{\decay{\Bd}{\subdecay{\taupm}{\pipm\pimp\pipm\nu}\mump}}
\def\Bdstaumu{\decay{\B^0_{\left( s\right) }}{\subdecay{\taupm}{\pipm\pimp\pipm\nu}\mump}}
\def\Bdtaumu{\decay{\Bd}{\subdecay{\taupm}{\pipm\pimp\pipm\nu}\mump}}
\def\Bstaumu{\decay{\Bs}{\subdecay{\taupm}{\pipm\pimp\pipm\nu}\mump}}
\def\Bstaumupi0{\decay{\Bs}{\subdecay{\taupm}{\pipm\pimp\pipm\piz\nu}\mump}}
\def\Bdtaumupi0{\decay{\Bd}{\subdecay{\taupm}{\pipm\pimp\pipm\piz\nu}\mump}}
\def\BToDPi    {\decay{\Bd}{\D^\pm\pi^\mp}}
\def\dbtaunu {\decay{\B}{D^{(*)} \tau \nu}}
\def\dbmunu {\decay{\B}{D^{(*)} \mu \nu}}
\def\dbenu {\decay{\B}{D^{(*)} e \nu}}
\def\dmunu {\decay{\B}{D \mu \nu}}
\def\denu {\decay{\B}{D e \nu}}
\def\dsttaunu {\decay{\B}{D^{*} \tau \nu}}
\def\dstztaunu {\decay{\B}{D^{*0} \tau \nu}}
\def\dstptaunu {\decay{\B_{d}}{D^{*-} \tau^{+} \nu}}
\def\dstmunu {\decay{\B}{D^{*} \mu \nu}}
\def\dstenu {\decay{\B}{D^{*} e \nu}}
\def\dstpmunu {\decay{\B}{D^{*+} \mu \nu}}
\def\dstzmunu {\decay{\B}{D^{*0} \mu \nu}}
\def\dstlnu {\decay{\B}{D^{*} \ell \nu}}
\def\dblnu {\decay{\B}{D^{(*)} \ell \nu}}
\def\RDst {\ensuremath{\mathcal{R}(\Dstar)}\xspace}
\def\RDstp {\ensuremath{\mathcal{R}(\Dstarp)}\xspace}
\def\RD {\ensuremath{\mathcal{R}(D)}\xspace}
\def\RDb {\ensuremath{\mathcal{R}(D^{(*)})}\xspace}
\def\RDst {\ensuremath{\mathcal{R}(\Dstar)}\xspace}
\def\RDorDst {\ensuremath{\mathcal{R}(D^{(*)})}\xspace}
\def\RDsteq {\ensuremath{\mathcal{R}(\Dstar)\equiv\frac{\mathcal{B}\left(\decay{\B}{\Dstarb\taup\neu}\right)}{\mathcal{B}\left(\decay{\B}{\Dstarb\ellp\neu}\right)}}\xspace}
\def\Req {\ensuremath{\mathcal{R}(D^{*})\equiv\frac{\mathcal{B}\left(\decay{\B}{\Db^{(*)}\taup\neu}\right)}
         {\mathcal{B}\left(\decay{\B}{\Db^{(*)}\ellp\neu}\right)}}\xspace}
\def\RD {\mathcal{R}(D)}
\def\threepim {\ensuremath{\pim \pip \pim}\xspace}
\def\threepip {\ensuremath{\pip \pim \pip}\xspace}
\def\mupm   {{\ensuremath{\mu^\pm}}\xspace}
\def\mump   {{\ensuremath{\mu^\mp}}\xspace}
\def\taupm   {{\ensuremath{\tau^\pm}}\xspace}
\def\taump   {{\ensuremath{\tau^\mp}}\xspace}
%% Decays
\def\taummm{\decay{\taupm}{\mupm \mu^+ \mu^-}}
\def\Dsetamunu{\decay{\Ds}{\subdecay{\eta}{\mu^+\mu^-\gamma}\mu^+\nu_{\mu}}}

\begin{document}

\section{Bottom quark probes of new physics and prospects for $B$-anomalies \label{secBanom}}
{\bf Authors: J. Virto, D. Straub, W. Altmannshofer,  D. Robinson, S. Schacht, Martin Jung} (10 pp,TH:EXP= 50\%:50\%)

A key contribution of the HL-LHC datasets will be a comprehensive global picture of rare, suppressed, and (semi)-leptonic quark transitions. This global picture has value whether or not the current ``flavour anomalies'' are confirmed with Run~2 or Run~3 LHC data. If the most statistically sensitive decay modes ($R_{K}$ and \RDb) confirm the anomalies, a  global analysis of all analogous modes will be the only way to discriminate between the full range of possible BSM explanations, and to decisively rule out  long-distance SM effects as the cause. Such a global analysis will also help to pin down the mass scale of the BSM particles and force-carriers, which together with the direct HL-LHC searches will further restrict the range of possible explanations for the anomalies. On the other hand, if the anomalies decrease in significance with Run~2 and Run~3 data, we may return to searching for small BSM
effects on top of the dominant SM amplitudes. In this case, a global analysis of
deviations from the SM predictions may be the only way to see such a small BSM
signal, and understand its energy scale. The statistical power of the HL-LHC datasets will also enable a unique reach for other BSM signatures such as LFV and BNV searches, particularly in the baryon and heavy meson sectors.

We would like to stress that in the HL-LHC era, many rare and FCNC beauty decays will actually become abundant and enter the precision regime. 
Together with Belle~II, the HL-LHC collaborations must therefore
ensure a coherent treatment of systematic uncertainties, and enable systematic correlations between different experiments to be understood. 
A good example of this is the treatment of the ratio of
hadronization fractions $f_s/f_d$ in any LHC $b\to \ell\ell$ combination, as discussed further in Sec.~\ref{sec:b2llexp}. Another example would be ensuring
the coherent use of common packages e.g. PHOTOS between LHCb and Belle~II in future
lepton universality measurements.

While much of the power of the HL-LHC and Belle~II datasets will lie in the breadth of precisely measured rare and (semi)-leptonic decays, sociological and practical considerations mean that experiments will continue to publish analyses of individual decay modes as each is completed. It will therefore be particularly important to provide these results in a way which facilitates both the combination of results between experimental collaborations, and the inclusion of these results in global fits and tests of phenomenological models~\cite{Hurth:2017sqw,Blake:2017fyh,Chrzaszcz:2018yza}. 
Systematically publishing experimental likelihoods, efficiency maps, and resolution unfoldings, as already done for some headine analyses, should therefore be strongly encouraged.

\subsection{Phenomenology of $b\to s \ell\ell$ decays}
{\bf Authors: J. Virto, D. Straub, W. Altmannshofer}

In this section we discuss the status of the interpretation of $b\to s$ transitions within the SM and BSM, the prospects from future sensitivities at the LHC, and the complementarity with the corresponding measurements at Belle~II. All current and planned measurements of processes involving $b\to s$ transitions are related to weak decays of $b$-hadrons. At present, most of the data is on $B$ decays, with a small fraction of $B_s$ decays. LHCb has already provided a few measurements on $\Lambda_b$ decays, with data sets that will increase dramatically in future runs, including a significant output on decays of other $b$-hadrons such as $\Omega_b$
or $\Xi_b$~\cite{EoI} (see Sec.~\ref{}). 

Theoretically, the decays of $b$-hadrons are best described within the Weak Effective Theory (WET), where flavor-changing transitions are mediated by ``effective" dimension-six
operators with Wilson coefficients (WCs) that encapsulate all SM weak and heavy NP effects. Thus all observables can be calculated in full generality (model-independently) in terms of these WCs and of QCD hadronic matrix elements. The relevant effective Lagrangian for $b\to s$ transitions at low-energies in the SM is
\begin{equation}
\label{eq:Heff}
  {\cal{H}}_{\rm eff}^{\rm SM}=
  \frac{4 G_F}{\sqrt{2}}\sum_{p=u,c}\lambda_{ps}\left(C_1 \mathcal{O}_1^p+C_2 O_2^p +\sum_{i=3}^{10} C_i  O_i\right),
\end{equation}
with $\lambda_{ps}=V_{pb}^{} V_{p s}^\ast$. The $O_{1,2}^p$,  $O_{3,\ldots,6}$, and $O_{8}$ are the so-called ``current-current'', ``QCD-penguins'' and ``chromo-magnetic'' operators, respectively, and they contribute to the $b\to s\gamma$ and $b\to s\ell\ell$ transitions via an electromagnetic interaction. The $O_{7}$ and $O_{9,10}$ are the electromagnetic penguin and the semileptonic operators, respectively.
The BSM can enter through the WCs $C_1,\ldots,C_{10}$, the chirally-flipped ($b_{L(R)}\to b_{R(L)}$) versions of them, $O_{7,\ldots, 10}'$, or through scalar and tensor semileptonic operators~\cite{Bobeth:2007dw}. Furthermore, in BSM, semileptonic operators can induce lepton-flavor universality violation (LFUV) or charged-lepton flavor violation (CLFV).
For the purpose of this Section, the relevant operators for the interpretation of $b\to s\gamma$ and $b\to s\ell\ell^\prime$ data in terms of short-distances and BSM are
%
\begin{align}
&O_{7\gamma}^{(\prime)}=\frac{e}{16\pi^2}(\bar s\sigma_{\mu\nu}P_{R(L)} b)F^{\mu\nu},\nonumber\\
&O_9^{\ell\ell^\prime(\prime)} =\frac{\alpha_{\rm em}}{4\pi}
(\bar{s} \gamma_{\mu} P_{L(R)} b)(\bar{\ell} \gamma^\mu \ell^\prime)\,,
\,\,
O_{10}^{\ell\ell^\prime(\prime)} =\frac{\alpha_{\rm em}}{4\pi}
(\bar{s} \gamma_{\mu} P_{L(R)} b)( \bar{\ell} \gamma^\mu \gamma_5 \ell^\prime),\label{eq:O10}
\end{align}
where we have added the leptonic flavor labels in the semileptonic operators.

%The semi-leptonic operators with scalar or tensor structure cannot explain the observed deviations, nor can the 4-quark operators if there is lepton-flavor non-universality (however, see~\cite{Jager:2017gal}).

By calculating and measuring a large set of independent observables, one can perform global fits to all the relevant WCs, and comparing with their SM values,
extract information on NP model-independently. After that, the resulting information can be contrasted to NP models to learn about the details of the underlying dynamics. This program has been carried out since the start of the LHC, and has led to the current $b\to s$ anomalies~\cite{1304.6325,1307.5683,1308.1501}.
The future runs at the LHC, including the HL
phase will allow to establish or refine these discoveries. Here we discuss the {\bf status} and {\bf prospects} in this program, from the point of view of 1) the calculation of the observables, 2) the global fits to the data, and 3) the interpretation in terms of models.


\subsubsection{Observables and Hadronic Matrix Elements}

When measuring the WCs of the electromagnetic dipole and semileptonic operators $C^{(\prime)}_{7\gamma}$ and $C^{(\prime)\ell}_{i}$ one looks at
radiative, leptonic and semileptonic inclusive and exclusive decays of $b$-hadrons. 



\subsubsubsection{$B_{q}\to\ell\ell^\prime$}

From the theoretical perspective, the purely leptonic channels, $B_{s}\to \ell^+ \ell'^-$, are the cleanest exclusive decays. Up to QED corrections~\cite{1708.09152}, all QCD effects are contained in a decay constant  which is precisely and reliably computed in the lattice, $f_{B_s}=228.4(3.7)$~MeV. Going beyond this accuracy in $f_{B_s}$ is difficult (see Sec.~11). The current theoretical uncertainty on the muonic $B_s$ branching ratio is at the level of 5\%~\cite{1708.09152} and is dominated by those of the relevant CKM parameters. The latest experimental LHCb measurement has an error of~$\sim 25\%$~\cite{Aaij:2017vad}, while future measurements at the LHC will bring this down to the percent level (see Sec.~\ref{}). 

Beyond the SM, the decay $B_{s}\to \mu^+ \mu^-$ give very strong constraints to the scalar and pseudoscalar operators, and also to $C^\mu_{10}$, which has an impact on the fits and the $b\to s\mu\mu$ anomalies.


\subsubsubsection{$B_q\to M \ell^+\ell^-$}
The current $b\to s$ anomalies are all observed in exclusive $B_q\to M \mu^+\mu^-$ modes, specifically in $B\to K^{(*)}\mu^+\mu^-$
and $B_s\to \phi \mu^+\mu^-$. These are also the modes that dominate the global fits, because of the level of precision that has been achieved in their measurements
(mostly by LHCb, but also Babar, Belle, ATLAS and CMS), and by the huge number of independent observables they lead to, which constrain many independent
combinations of WCs as well as many hadronic parameters. Keeping in mind there is no experimental information related to the polarization of the final-state leptons,
measurable observables arise from the kinematic differential distributions of the final state momenta. These differential distributions are customarily written in the form
of an angular distribution with coefficients that depend on the dilepton invariant mass squared $q^2$. These coefficients, integrated in specific bins of $q^2$,
are the observables that are predicted and measured, and which enter the global fits. In the case of the three-body mode $B\to K\mu^+\mu^-$, there are three
angular coefficients: the differential branching fraction $d{\cal B}/dq^2$, the forward-backward asymmetry $A_{\rm FB}(q^2)$
and the ``flat term" $F_H(q^2)$~\cite{0709.4174}. The kinematic distribution of the four-body decays $B\to V(\to M_1M_2) \ell^+\ell^-$ contains many more
independent angular coefficients, up to 12 in the most general case, called $I_i(q^2)$ or $J_i(q^2)$~\cite{0811.1214,1202.4266}.
Normalizing these by the total differential rate $d\Gamma/dq^2$ and symmetrizing or antisymmetrizing on the two charge-conjugated modes leads to the
observables $S_i$, $A_i$~\cite{0811.1214}.
The angular observables $J_i$ are not always independent~\cite{1005.0571,1202.4266}. In addition, it is convenient to define certain combinations of angular
observables where form factor uncertainties cancel in certain limits, called ``optimized observables". An independent set of observables optimized at low-$q^2$
is given by the $P_i^{(\prime)}$ basis~\cite{1202.4266,1207.2753,1303.5794}. Optimized observables at large-$q^2$ also exist~\cite{1006.5013,1303.5794}.
The whole set of angular observables $P_i^{(\prime)}$ and $S_i,A_i$ in $B^0\to K^{*0}\mu^+\mu^-$ in bins across the whole $q^2$ range have been measured by LHCb~\cite{1304.6325,1308.1707,1512.04442}, and some subset of these observables have also been measured by Babar~\cite{1508.07960},
Belle~\cite{1604.04042}, CMS~\cite{1710.02846} and ATLAS~\cite{ATLAS-CONF-2017-023}. A subset of these observables is also available from LHCb for
$B_s\to \phi \mu^+\mu^-$~\cite{1506.08777}, and also for $B^0\to K^{*0} e^+ e^-$ but only at very low $q^2$~\cite{1501.03038}.


The exclusive semileptonic decays of the form $B\to M \ell^+ \ell^-$ discussed in this section and all the observables that can be defined from them are specified
by transversity amplitudes which depend on two types of hadronic matrix elements: ``local" (form factors) and ``non-local" (see e.g.~\cite{1707.07305}).
Local form factors for $B\to K$, $B\to K^*$ and $B_s\to \phi$ transitions are known at low-$q^2$ from two different versions of light-cone sum rules (LCSRs)~\cite{hep-ph/0611193,1503.05534}, and at large-$q^2$ from Lattice QCD~\cite{1509.06235,1310.3722}. A comparison between both determinations is done by parametrizing
the $q^2$ dependence via the $z$-expansion~\cite{0807.2722}, which is based on the analytic structure of the matrix elements. Future prospects on the theoretical
precision on the calculation of the form factors rely mostly on the improvements from the Lattice QCD side. In the case of vector mesons, however, both determinations
assume the narrow-width limit. A calculation beyond the narrow-width limit is possible within the LCSRs~\cite{1701.01633,1709.00173}, and points towards a correction
of up to $10\%$. There are some prospects on the treatment of hadronic resonances on the lattice, too~\cite{1704.05439}.

Non-local effects are significantly more difficult to approach theoretically~\cite{hep-ph/0106067,1006.4945,1101.5118}, but data-driven methods might be able to eliminate most of the uncertainties.
Some of these methods will be possible and improve significantly with the high statistics collected at LHCb after the upgrade.
They are all based on precise measurements of the $q^2$ spectra, together with a theoretically motivated parametrization of the $q^2$ dependence
of the amplitudes and a theory benchmark that allows to separate short- from long-distance contributions.
The situation is different at low and high $q^2$:\\[2mm]
%
$\blacktriangleright$ At low $q^2$, theory input is based on a light-cone Operator Product Expansion (OPE)
at very low (or negative) $q^2$~\cite{hep-ph/0106067,1006.4945},
an expansion that breaks down at the perturbative $c\bar c$ threshold $q^2\simeq 4m_c^2$.
Parametrizations of the $q^2$ dependence are based on a naive expansion in powers of $q^2$~\cite{1512.07157}, on a dispersion relation~\cite{1006.4945} or on a
$z$-expansion~\cite{1707.07305} very similar to the one for local form factors. The latter two parametrizations are consistent with analyticity and allow to use
extra information, such as data on the $B\to \psi K^*$ with $\psi=\{J/\psi, \psi(2S)\}$. In~\cite{1707.07305} it is shown quantitatively how this information can be used
\emph{a priori} to produce theory predictions for the non-local effect independent of NP, or \emph{a posteriori} to fit all the $B\to \psi K^*$ and $B\to K^*\mu\mu$ spectra up to $q^2=m_{\psi(2S)}^2$ simultaneously to the hadronic parameters and NP. In this last approach, short- and long-distance effects are disentangled by the experimental input from $B\to \psi K^*$, the fixed $q^2$ dependence of the NP contribution, and by the theory constraints at negative $q^2$.
A notable byproduct is the fact that experimental data between the two narrow charmonia are not only used, but very useful.
The prospects for this data-driven approach with the future data from LHCb, including the prospects of doing without theory input altogether,
have been studied in~\cite{1805.06378} with very positive outcome. Thus, high statistics studies of $b\to s\mu\mu$ exclusive transitions at the LHC with fine
$q^2$ binning (or unbinned) will not only probe for NP, but also improve the reliability of non-local effects. While current global fits to different $q^2$ bins show
consistency with the treatment of non-local effects~\cite{1510.04239}, future LHC data will require, and provide, a higher level of control over them.\\[2mm]
%
$\blacktriangleright$ At high $q^2$, the theory input is based on the low-recoil OPE~\cite{hep-ph/0404250,1006.5013,1101.5118}. This method in practice relies
on the fact that resonant effects from ``above-threshold" charmonia will average out to some extent within sufficiently broad $q^2$ bins. Thus only a single bin
in the whole low-$q^2$ region is typically considered in the global fits, but the $q^2$ spectrum can be used to test models for the ``duality violating" contributions
(those that do not average out), and give reliable estimates of these intrinsically non-perturbative effects. Currently these analyses are carried out within the
``Kruger-Sehgal" (naive factorization) approach~\cite{hep-ph/9603237}, which allows to use data on the $R(s)$ ratio in $e^+e^-$ annihilation.
These analyses have been performed in~\cite{1101.5118,1406.0566,1606.00775}. Ref.~\cite{1606.00775} uses all currently available data on $B\to K^*\mu\mu$
at low recoil and finds agreement with the OPE within $\sim 20\%$ in all the bins. Notably, future precision data from the LHC with the
expected fine binning will be essential in refining these data-driven methods and disentangle NP contributions, with the prospects of confirming the
$b\to s\mu\mu$ anomalies in the large $q^2$ region. As in the low-$q^2$ case, combined fits to hadronic parameters and NP are also beneficial~\cite{1606.00775}.\\

Tests of lepton-flavor universality (LFU) are also intrinsically a data-driven approach to hadronic effects. Ratios such as $R_{K^{(*)}}$~\cite{hep-ph/0310219}
have been measured by Babar~\cite{1204.3933}, Belle~\cite{0904.0770}, and LHCb~\cite{1406.6482,1705.05802}, the latter showing a $\sim 2.5\,\sigma$
tension with respect to the precise SM (very close to universal) prediction $R_{K^{(*)}}\simeq 1$.
Much higher precision will come from future data at LHCb~\cite{EoI},
and also from Belle-II which has access to the low- and high-$q^2$ region and can confirm $R_K$ with $5\,\sigma$ with $20\,\text{ab}^{-1}$~\cite{Belle2}. 
But in the presence of LFU violation, predictions are not so precise.
Some ``optimized" observables in this case have also been defined, such as $Q_{4,5}$~\cite{1605.03156}, which have also been measured by Belle~\cite{1612.05014}. 
As these tensions are consistent with the $b\to s\mu\mu$ anomalies if one assumes that NP is present only in the muon modes, one may adopt this assumptions and
use the future precise and fine-binned measurements in $b\to se^+e^-$ exclusive modes to fit to the hadronic parameters directly, as discussed above, providing
model-independent SM predictions for $b\to s\mu^+\mu^-$. These
precise measurements will be performed at the LHC in its HL phase, and constitute again a physics case for NP searches with data-driven determinations of
the SM background. A study of the prospects at LHCb, in the $z$-expansion framework of Ref.~\cite{1707.07305} has been performed in~\cite{1805.06401}
for the $B\to K^*\ell^+\ell^-$ case, with promising results.

Other measurements will be possible at the HL phase with a significant impact on the $b\to s\ell\ell$ anomalies. First, the excellent sensitivity of the
measurements on $B_s$ and $\Lambda_b$ decays will make possible to implement the program outlined above to these modes, too.
In particular, the $\Lambda_b\to \Lambda \mu^+\mu^-$ measurements~\cite{1503.07138} have only been used at low recoil~\cite{1603.02974} and do not seem
to confirm the $C^\mu_{9}$ anomaly. 
For $B_s\to\phi \mu^+\mu^-$, a flavor-tagged time-dependent analysis allows to access independent observables~\cite{1502.05509}, with sensitivity to NP and
different properties. In addition, the study of exclusive $b\to d\mu^+\mu^-$ transitions will reach the current level of precision we have in $b\to s\mu^+\mu^-$.
This on its own will allow to perform analogous global analyses to test $b\to d\mu^+\mu^-$ transitions, but can also be used to study hadronic effects
in $b\to s\mu^+\mu^-$ in the U-spin limit, which constitutes an independent test, albeit with some limitations given by unknown U-spin-breaking effects.



\subsubsubsection{$B_q\to M \nu \bar \nu$}

The second cleanest exclusive $b\to s$ modes are $B\to K^{(*)} \nu \bar \nu$ and relatives~\cite{1409.4557}, which depend only on local form factors (see below).
These modes, however, are very challenging at the LHC. Belle-II, on the other hand, is expected to provide measurements with an uncertainty of about $\sim 10\%$,
(assuming the rates are SM-like)~\cite{Belle2}. These modes do not probe $C^\mu_{9}$ directly either, but a NP contribution to  $C_{9\mu}$ is likely correlated with a NP
contribution to  $C^\nu_{9}$ by $SU(2)_L$ gauge invariance, if the NP is above the EW scale.

\subsubsubsection{Inclusive $B$ decays}

The inclusive observables used in current fits are the branching fractions
${\cal B}(B\to X_s \gamma)$ and ${\cal B}(B\to X_s \ell^+\ell^-)$. These inclusive observables are accessible at $B$-factories and very difficult for the LHC experiments.
Belle-II will improve these measurements, including good measurements of the Forward-Backward asymmetry in $B\to X_s \ell^+\ell^-$, which will also have
an interesting effect in the fits~\cite{Belle2}. In particular, Belle-II measurements of $B\to X_s \mu^+\mu^-$ will be able
to test the LHCb anomalies independently by 2024~\cite{Belle2}. Theoretically, these inclusive rate are dominated by the perturbative
partonic decay of the $b$ quark up to small non-perturbative effects~\cite{1003.5012,1705.10366}, given an appropriate cut on the invariant mass of the hadronic final state. Calculations of both rates have been done to high accuracy~\cite{1503.01789,1503.04849}.

\subsubsection{Model-Independent Fits}

Existing measurements have shown deviations from SM expectations in three
different classes of measurements:
in $B\to K^*\mu^+\mu^-$ angular observables \cite{Aaij:2015oid},
in branching fractions of exclusive $b\to s\mu^+\mu^-$ decays (in particular
$B_s\to\phi\mu^+\mu^-$ \cite{Aaij:2015esa}),
and in $\mu$-$e$ universality (LFU) tests \cite{Aaij:2014ora,Aaij:2017vbb}.
None of the individual measurements is in tension with the SM by more than
$4\sigma$. However, a global significance can be defined in a specific
framework of new physics, such as the model-independent framework provided by the weak effective Hamiltonian.

%The most model-independent framework is the general weak effective Hamiltonian for $b\to s\ell\ell$ transitions. Splitting this effective Hamiltonian in a SM and a new physics part, $\mathcal H_\text{eff} = \mathcal H_\text{eff}^\text{SM} +\mathcal H_\text{eff}^\text{NP}$, the latter contains semi-leptonic operators that could generate a deviation from lepton flavour universality,


Several groups have performed global fits of the Wilson coefficients to existing $b\to s\ell\ell$ data
(see \cite{Altmannshofer:2017fio,Capdevila:2017bsm,Altmannshofer:2017yso,Hurth:2017hxg,Ciuchini:2017mik,Geng:2017svp}
for recent fits).
Three classes of fits can be distinguished: fits to $b\to s\mu\mu$ data only,
fits to $\mu$-$e$ LFU ratios only, and combined fits assuming no new physics
in $b\to see$ transitions. All these fits agree -- up to differences that can be
attributed to different theoretical inputs or to incomplete inclusion
of experimental data -- on two basic conclusions.
First, that there is a discrepancy with $b\to s\mu\mu$ data alone, that could be
explained by a shift in the Wilson coefficient $C_9^\mu$~\cite{1307.5683}, or by an unexpectedly large
hadronic effect that is moreover hard to bring into agreement with non-leptonic
measurements \cite{Bobeth:2017vxj}.
Second, that the new physics explanation of the $b\to s\mu\mu$ anomaly, assuming
it does not affect $b\to see$ transitions, is in perfect agreement with the
deviations from LFU observed in the $\mu/e$ ratios $R_K$ and $R_{K^*}$~\cite{Alonso:2014csa}.
Singling out the Wilson coefficient $C_9^\mu$ in the muonic transition and performing
a global fit to all the data, the log-likelihood ratio between the best-fit
point and the SM hypothesis corresponds to more than six Gaussian standard deviations.

Clearly, future experimental efforts to settle the question whether
these deviations are due to physics beyond the SM are of the utmost
importance. If LFU is indeed violated, LFU ratios like $R_K$ and
$R_{K^*}$ are theoretically extremely clean smoking guns to
establish a deviation from the SM. However, global fits to
all relevant observables will
remain an indispensible tool for several reasons,
\begin{itemize}
 \item if the hints for LFU violation disappear with more statistics, a LFU new physics effect might still hide in the less clean observables,
 \item to identify the {\em nature} of the new physics (and not just its presence), the values of the BSM Wilson coefficients
 have to be determined (and even ratios such as $R_K$ and
$R_{K^*}$ have non-negligible theory uncertainties {\em away}
from the SM limit),
 \item they allow in principle to simultaneously determine poorly known hadronic effects from the data.
\end{itemize}


\begin{figure}
    \centering
    \includegraphics[width=0.46\textwidth]{section7/C9mumu_C10mumu_pp_Run4.pdf}
    \includegraphics[width=0.46\textwidth]{section7/C9mumu_C9ee_pp_Run4.pdf}
    \caption{Current (not filled) and projected global fit results for
    LHCb and Belle-II (inclusive and exclusive decays) in the planes
    of the Wilson coefficients $C_9^\mu$, $C_{10}^\mu$, and $C_9^e$, 
    allowing two of them to vary at a time.
    The central values of the extrapolations correspond to NP scenarios
    with different central values. The contours correspond to $1\sigma$ uncertainties.
    The light gray contours around 
    the Standard Model point (black dot)
    show the hypothetical exclusion power attainable with a combined sensitivity
    of LHCb's 50~fb$^{-1}$ and Belle-II's 50~ab$^{-1}$ datasets.
    Figure taken from~\cite{Albrecht:2017odf}.}
    \label{fig:WC}
\end{figure}

Extrapolations of the sensitivity of global fits
after future LHCb runs and Belle-II,
assuming the current theory technologies,
have been performed in ref.~\cite{Albrecht:2017odf}.
The projections in the planes of the Wilson coefficients $C_9^\mu$, $C_{10}^\mu$
and $C_9^\mu$, $C_{9}^e$, are shown in Figure~\ref{fig:WC}.
These projections can however be affected significantly by
additional qualitative improvements on the theoretical and
the experimental side. For instance,
\begin{itemize}
\item the above projections have not yet taken into account the angular analysis
of $B\to K^*e^+e^-$, which will however play a crucial role in disentangling
NP models;
\item in the presence of LFU violation, 
a simultaneous amplitude analysis of $B\to K^*\mu^+\mu^-$ and 
$B\to K^*e^+e^-$ decays is even more powerful than separate 
analyses~\cite{1805.06401};
\item combined fits to semi-leptonic and non-leptonic decays as well as the
precise dilepton invariant mass spectra will allow to better disentangle
long- and short-distance effects in $B\to K^*\mu^+\mu^-$~\cite{1805.06378};
\item using the full event information would allow an unbinned determination
of the Wilson coefficients~\cite{Hurth:2017sqw}, fully exploiting the experimental
potential. In this case, an important problem to solve is communicating this
full even information to theorists to allow extracting the Wilson coefficients
with different theoretical models for form factors and hadronic contributions.
\end{itemize}


\subsubsection{Models of new physics}

The model independent new physics analysis of the anomalies points to a new physics scale of
% 
\begin{equation}
 \Lambda_\text{NP} = \frac{4\pi}{e} \frac{1}{\sqrt{|V_{tb} V_{ts}^*|}} \frac{1}{\sqrt{|\Delta C_9^\mu|}} \frac{v}{\sqrt{2}} \simeq \frac{35~\text{TeV}}{\sqrt{|\Delta C_9^\mu|}}~,
\end{equation}
%
where $\Delta C_i^\ell$ denotes the NP contribution to $C_i^\ell$.
The actual mass of the new physics degrees of freedom that are responsible for an explanation of the anomalies can be much smaller, if the new physics couplings are smaller than 1 and/or the new physics contributions arise not at the tree level, but at the loop level.

There are many new physics models that can address the observed anomalies. At tree level there are two types of new physics particles that can contribute to $b \to s \ell \ell$ transitions: $Z^\prime$ and leptoquarks (see left and center diagrams in Fig.~\ref{fig:bsll_diagrams}). Also loop level contributions of leptoquarks have been studied extensively in the literature (see right diagram in Fig.~\ref{fig:bsll_diagrams}). Other loop level models have been considered for example in~\cite{Belanger:2015nma,Gripaios:2015gra,Arnan:2016cpy} but will not be discussed here.\\

%%%%%%%%%%%%%%%%%%%%%%%%%%%%%%%%%%%%%%%%%%%%%%%%%%%
\begin{figure}[tb]
\begin{center} 
\includegraphics[width=0.25\textwidth]{section7/diagram_Zprime.pdf} ~~~~~~~~
\includegraphics[width=0.25\textwidth]{section7/diagram_LQ.pdf} ~~~~~~~~
\includegraphics[width=0.29\textwidth]{section7/diagram_loop.pdf}
\caption{Examples of new physics contributions to $b \to s \mu \mu$ transitions. Left: tree level $Z^\prime$ contribution. Center: tree level leptoquark contribution. Right: 1-loop leptoquark contribution.}
\label{fig:bsll_diagrams}
\end{center}
\end{figure}
%%%%%%%%%%%%%%%%%%%%%%%%%%%%%%%%%%%%%%%%%%%%%%%%%%%

\noindent $\blacktriangleright$ \underline{\bf \boldmath $Z^\prime$ models}:
Models with a $Z^\prime$ that has flavour violating couplings to quarks and that couples non-universally to leptons can explain the observed anomalies in $b \to s \ell \ell$. In a ``simplified model'' approach, treating the $Z^\prime$ mass and its coupling to SM fermions as completely free parameters, irreducible constraints arise from $B_s$ mixing, neutrino trident production and LEP bounds on four-lepton contact interactions.

Combining these constraints one finds the following maximal values for the $Z^\prime$ contributions to the relevant Wilson coefficients $\Delta C_{9,10}^{e,\mu}$ in the case of LH lepton couplings~(\ref{eq:C910_LH}) and of vectorial lepton couplings~(\ref{eq:C910_vec})~\cite{Altmannshofer:2014rta}
%
\begin{eqnarray} \label{eq:C910_LH}
 && |\Delta C_9^\mu| = |\Delta C_{10}^\mu| \lesssim 5.4 ~,\quad |\Delta C_9^e| = |\Delta C_{10}^e| \lesssim 0.64 ~, \\  \label{eq:C910_vec}
 && |\Delta C_9^\mu| \lesssim 9.3 ~,\quad |\Delta C_9^e| \lesssim 0.72 ~,\quad \Delta C_{10}^\mu = \Delta C_{10}^e = 0 ~.
\end{eqnarray}
%
This shows that $Z^\prime$ models can comfortably explain the anomalies in $R_K$ and $R_{K^*}$ as well as the other anomalies in $b \to s \mu \mu$ transitions by affecting the muonic modes. Adressing $R_K$ and $R_{K^*}$ through a $Z^\prime$ that affects the electronic modes is only possible in parameter regions close to the current bounds from $B_s$ meson mixing and bounds on four-lepton contact interactions. Constraints from $B_s$ mixing will likely become stronger at the time scale of the HE/HL LHC due to improved lattice predictions of hadronic matrix elements.
The $Z^\prime$ mass can be at most several TeV, otherwise an explanation of the anomalies requires couplings to leptons that are non-perturbatively large. If the $Z^\prime$ has very week couplings to light quarks and electrons, its production cross section at colliders is tiny. Therefore the $Z^\prime$ could in principle be light, for example at the electro-weak scale, or in certain models even much lighter~\cite{Datta:2017pfz,Sala:2017ihs,Altmannshofer:2017bsz}. 

Many explicit $Z^\prime$ models have been put forward as possible explanations of the anomalies.
One popular class of models is based on gauging the difference of muon-number and tau-number, $L_\mu - L_\tau$. Once physics is introduced that generates flavor violating couplings of the $Z'$ to quarks, the observed tensions in $b \to s \ell \ell$ decays can be explained~\cite{Altmannshofer:2014cfa,Crivellin:2015mga,Altmannshofer:2015mqa,Fuyuto:2015gmk,Altmannshofer:2016jzy,Baek:2017sew,Chen:2017usq}. $L_\mu - L_\tau$ models predict
%
\begin{equation}
 \Delta C_9^e = 0 ~,\quad \Delta C_9^\mu = - \Delta C_9^\tau ~,\quad \Delta C_{10}^e = \Delta C_{10}^\mu = \Delta C_{10}^\tau = 0~.
\end{equation}
%
Besides $L_\mu - L_\tau$, various other combinations of gauged flavor symmetries have been used to construct $Z'$ models that can address the $b \to s \ell \ell$ anomalies~\cite{Crivellin:2015lwa,Celis:2015ara,Falkowski:2015zwa,Boucenna:2016wpr,Boucenna:2016qad,Celis:2016ayl,Crivellin:2016ejn,Alonso:2017bff,Ellis:2017nrp,Alonso:2017uky,Bonilla:2017lsq,Babu:2017olk,Bian:2017rpg,Tang:2017gkz,Cline:2017ihf}.
Also scenarios where the $Z'$ couples to both quarks and leptons indirectly through the mixing with heavy vectorlike fermions have been considered~\cite{Sierra:2015fma,King:2017anf,Fuyuto:2017sys}.

In models with partial compositeness the $Z'$ can be identified with a heavy neutral spin-1 resonance of the composite sector, typically denoted as $\rho$. Such a resonance generically features flavor violating couplings to quarks. A large degree of compositeness of the left-handed muons is required to explain the $B$ decay anomalies~\cite{Niehoff:2015bfa,Carmona:2015ena,Megias:2016bde,Carmona:2017fsn,Megias:2017ove,Sannino:2017utc}.
The generic expectation in models with partial compositeness is that the $\rho$ couplings are strongest to SM fermions of the third generation, reflecting the mass hierarchy of the SM fermions that is related to their degree of compositeness.
If one assumes the dominance of couplings to left-handed leptons, these models therefore suggest the pattern 
%
\begin{equation}
 \Delta C_9^e \simeq -\Delta C_{10}^e \ll \Delta C_9^\mu \simeq -\Delta C_{10}^\mu \ll \Delta C_9^\tau \simeq -\Delta C_{10}^\tau ~.
\end{equation}
%
Models in which the $Z^\prime$ is part of a $SU(2)_L$ triplet have been suggested as an simultaneous explanation of the $b \to s \ell \ell$ anomalies and hints for lepton flavor universality violation in semileptonic charged current decays $R_D$ and $R_{D^*}$~\cite{Bhattacharya:2014wla,Greljo:2015mma,Bhattacharya:2016mcc,Boucenna:2016wpr,Boucenna:2016qad,Buttazzo:2017ixm,Kumar:2018kmr}. 

The different $Z'$ models predict different patterns of new physics effects in $b \to s \ell \ell$ and the related $b \to s \nu \nu$ transitions. Future measurements of these transitions at LHCb and Belle II will therefore allow us to narrow down viable $Z^\prime$ models.
For example, the $Z^\prime$ models based on the gauged $L_\mu - L_\tau$ symmetry predict effects in the semileptonic $b \to s \mu^+ \mu^-$ and $b \to s \tau^+ \tau^-$ transitions of opposite sign, while $b \to s e^+ e^-$ transitions remain SM-like.
Also the purely leptonic $B_s \to \mu^+\mu^-$ and $B_s \to \tau^+\tau^-$ decays as well as the neutrino modes $B \to K^{(*)} \nu \bar \nu$ are predicted to be SM-like in these models~\cite{Altmannshofer:2014cfa}.

A markedly different pattern arises in $Z^\prime$ scenarios that are based on dominant couplings to left-handed fermions of the third generation. In those models the $b \to s \tau^+ \tau^-$ and $B \to K^{(*)} \nu \bar \nu$ rates are typically enhanced compared to SM predictions by a factor of few. The $B_s \to \mu^+\mu^-$ rate is predicted to be suppressed by approximately $25\%$ compared to the SM prediction. Finally, in contrast to the $L_\mu - L_\tau$ models, rare lepton flavor violating decays like $B \to K^{(*)} \tau \mu$ are predicted at levels of $O(10^{-8})$~\cite{Glashow:2014iga} which might be in reach at the HE/HL LHC.\\


\noindent $\blacktriangleright$ \underline{\bf Leptoquark models:}
There are seven quantum number assignments for leptoquarks that allow tree-level couplings to down-type quarks and charged leptons of the SM.
These are~\cite{Dorsner:2016wpm} $S_3 = (\bar 3,3,1/3)$, $R_2 = (3,2,7/6)$, $\tilde R_2 = (3,2,1/6)$, $\tilde S_1 = (\bar 3,1,4/3)$, $U_3 = (3,3,2/3)$, $V_2 = (\bar 3,2,5/3)$ and $U_1 = (3,1,-1/3)$.
Among those, the couplings of $R_2$, $\tilde R_2$, $\tilde S_1$, and $V_2$ necessarily involve right-handed currents and therefore cannot explain the anomalies.
Thus one is left with the triplet scalar $S_3$, the singlet vector $U_1$, and the triplet vector $U_3$. They all contribute at tree level to the operator $(\bar s \gamma_\alpha P_L b)(\bar \mu \gamma^\alpha P_L \mu)$ and can explain the $b \to s \ell \ell$ anomalies.

In a simplified model approach the constraints on the leptoquark models are very weak.
While the leptoquarks $S_3$, $U_1$, and $U_3$ contribute to $B_s$ mixing, they only do so at the 1-loop level. 
Correspondingly, the bounds on the leptoquark couplings from $B_s$ mixing allow leptoquark masses as high as several 10's of TeV.
Lower bounds on the leptoquark masses come from direct searches at hadron colliders. As leptoquarks necessarily have color charge, they can be pair produced through the strong interactions. Depending on details of the models, current bounds from direct searches at the LHC are at the level of several 100~GeV to 1~TeV.

There is an additional leptoquark that has been identified as possible explanation of the rare $B$ decay anomalies. With the appropriate couplings, the scalar singlet leptoquark $S_1 = (3,1,-1/3)$ can contribute to $b \to s \ell \ell$ transitions through 1-loop box diagrams~\cite{Bauer:2015knc}. At the tree level, $S_1$ contributes to the neutrino modes $B \to K^{(*)} \nu \bar \nu$, and at the loop level to $B_s$ mixing as well as $Z \to \mu\mu$. If $S_1$ is responsible for the anomalies, new physics effects in these processes are expected close to the current experimental sensitivities. 

Most studies treat leptoquarks in the simplified model approach~\cite{Hiller:2014yaa,Bauer:2015knc,Fajfer:2015ycq,Alonso:2015sja,Becirevic:2016oho,Bhattacharya:2016mcc,Hiller:2017bzc,Chen:2017hir,Crivellin:2017zlb,Kumar:2018kmr}.
Going beyond the simplified model approach one finds that the leptoquark couplings that are required to explain the anomalies go likely beyond the principle of Minimal Flavor Violation~\cite{Aloni:2017ixa} and constitute new sources of flavor violation beyond the SM Yukawa couplings.
Both scalar and vector leptoquarks could arise for example in composite models~\cite{Gripaios:2014tna,Barbieri:2016las}. The vector leptoquarks could also be the gauge bosons of an enlarged gauge group that is broken not far above the TeV scale~\cite{Das:2016vkr,Bordone:2017bld,DiLuzio:2017vat,Bordone:2018nbg,Blanke:2018sro}.
The scalar singlet leptoquark that contributes to $b \to s \ell \ell$ transitions at the loop level can be identified with the right-handed sbottom in the Minimal Supersymmetric Standard Model with R-parity violation~\cite{Das:2017kfo,Earl:2018snx}.

Some leptoquark scenarios are also able to simultaneously address the $b \to s \ell \ell$ anomalies as well as the hints for lepton flavor universality violation in $R_{D^{(*)}}$~\cite{Bauer:2015knc,Bhattacharya:2016mcc,Barbieri:2016las,Das:2016vkr,Chen:2017hir,Buttazzo:2017ixm,Bordone:2017bld,DiLuzio:2017vat,Crivellin:2017zlb,Bordone:2018nbg,Blanke:2018sro,Kumar:2018kmr}.
However, many of the models that attempt a simultaneous explanation are strongly constrained by measurements of the $\tau^+\tau^-$ invariant mass spectrum at the LHC~\cite{Faroughy:2016osc}, by existing bounds on $B \to K \nu\nu$ and $B \to K^* \nu\nu$ from BaBar and Belle, existing bounds on lepton flavor violating tau decays like $\tau \to 3\mu$, from precision measurements of the leptonic couplings of the $Z$ at LEP~\cite{Feruglio:2016gvd,Feruglio:2017rjo,Cornella:2018tfd}, and by lepton universality tests in leptonic tau decays $\tau\to\ell \nu_\tau\bar\nu_\ell$~\cite{Feruglio:2016gvd,Feruglio:2017rjo,Cornella:2018tfd}.





\subsection{Phenomenology of $b\to c \ell \nu$ decays}
{\bf Authors: D. Robinson, S. Schacht, M. Freytsis, M. Jung}


\subsubsection{$B\rightarrow D^{(*)}l\nu$ Form factors and Anatomy of $R(D^{(*)})$}

Quite unexpectedly, signs of lepton flavor universality violation have not only been seen in loop-suppressed flavor-changing 
neutral currents discussed above, but also in tree-level decays, namely in tensions with the SM predictions for the ratios
\begin{equation}
  R(D^{(*)}) = \frac{\mathcal{B}(B \to D^{(*)}\tau\nu)}{\mathcal{B}(B \to D^{(*)}l\nu)},\,\hspace{0.5cm}(l=\mu,~e).
\end{equation}

Assuming for the moment only SM particle content, the $b\to c\ell\nu_{\ell^\prime}$ transitions at scales near $m_b$ can be described by a $SU(3) \times U(1)$-invariant effective Hamiltonian,
\begin{equation}\label{eq:Heffbcellnu}
  \mathcal H_\text{eff}^{b\to c\tau\nu_{\ell}} = \frac{4 G_F}{\sqrt{2}} V_{cb}
    \sum_{\ell}\left(O_{V_L}^{\ell\ell} + \sum_{i,\ell'} C_i^{\ell\ell'} O_i^{\ell\ell'} + \text{h.c.}\right),
\end{equation}
where $\ell,\ell'=e,\mu,\tau$ denotes the charged lepton and neutrino flavor, respectively, and the sum over $i$ runs over the following operators:
\begin{equation}
\label{eq:ops}
\begin{split}
  O_{V_{L,R}}^{\ell\ell'} &= (\bar c_{L,R} \gamma^\mu b_{L,R})(\bar \ell_L \gamma_\mu \nu_{\ell' L}) \,, \\
  O_{S_{L,R}}^{\ell\ell'} &= (\bar c_{R,L} b_{L,R})(\bar \ell_R \nu_{\ell' L}) \,, \\
  O_T^{\ell\ell'} &= (\bar c_R \sigma^{\mu\nu} b_L)(\bar \ell_R \sigma_{\mu\nu}\nu_{\ell' L}) \,,
\end{split}
\end{equation}
with NP coefficients $C_i^{\ell\ell'}$ dependent on both charged-lepton and neutrino flavor in general. These operators arise from more fundamental interactions at a higher scale $\Lambda$ which, in accordance with available LHC data, can be taken to be larger than the electroweak scale $v˜$. The operators of Eq.~(\ref{eq:ops}) should then be embedded in ones consisting of fields with full SM gauge quantum numbers. For $C_{V_R}$ this is impossible at dimension-6, leading to a parametric suppression of at least $v^2/\Lambda^2$ for tree-level UV completions. Linearly realized electroweak symmetry breaking, captured by the \emph{Standard Model effective field theory} (SMEFT)~\cite{Buchmuller:1985jz,Grzadkowski:2010es}, also yields the prediction $C_{V_R}^{\ell\ell'} \equiv C_{V_R}\delta_{\ell\ell'}$~\cite{Cirigliano:2009wk,Cata:2015lta}, in which case sizable contributions to $C_{V_R}$ are excluded by $b\to c(e,\mu)\nu$~\cite{Jung:2018lfu}. While deviations from this prediction are possible in a non-linear realization of electroweak symmetry breaking~\cite{Cata:2015lta}, at least one of the above-named sources of suppression always remains and right-handed currents do not play a role.

The meson form factors stemming from the hadronic matrix elements of the SM operator $O_{V_L}^{\ell\ell}$ can be parametrized as~\cite{Caprini:1997mu}
\begin{align}
  \bra{D(v_D)} \bar{c} \gamma^\mu b \ket{\bar{B}(v_B)} &= h_+(w) (v_B + v_D)^\mu + h_-(w)(v_B - v_D)^\mu \,, \\
%
  \bra{D^*(v_{D^*}, \varepsilon_{D^*})} \bar{c} \gamma^\mu b \ket{\bar{B}(v_B)} &= 
	i h_V \varepsilon^{\mu\nu\alpha\beta} \varepsilon^*_{D^* \nu} v_{D^* \alpha} v_{B\, \beta} \,, \\
%
  \bra{D^*(v_{D^*}, \varepsilon_{D^*})} \bar{c} \gamma^\mu \gamma_5 b \ket{\bar{B}(v_B)} &= 
	h_{A_1}(w) (w+1) \varepsilon^{*\, \mu}_{D^*} - 
	\left(h_{A_2}(w) v_B^\mu + h_{A_3}(w) v_{D^*}^\mu\right) \varepsilon_{D^*}^* \cdot v_B\,,
\end{align}
with form factors $h_i$, polarization vector $\varepsilon_{D^*}$, velocities $v_P$, and the dimensionless variable
$w = (m_B^2 + m_{D^{(*)}}^2-q^2)/(2 m_B m_{D^{(*)}})$. 
One can reformulate the above parametrization in terms of helicity amplitudes with definite parity and spin in the 
convention of Caprini, Lellouch, and Neubert (CLN) as~\cite{Caprini:1997mu}
\begin{multline}
  V_1 = h_+ - \frac{1-r}{1+r} h_-\,, \qquad
  S_1 = h_+ - \frac{1+r}{1-r} \frac{w-1}{w+1} h_-\,, \qquad
  V_4 = h_V, \qquad 
  A_1 = h_{A_1}\,, \\  
  A_5 = \frac{(w-r) h_{A_1} - (w-1) (r h_{A_2} + h_{A_3} )}{1-r}\,, \,
  P_1 = \frac{(w+1) h_{A_1} - (1 - w r) h_{A_2} - (w - r) h_{A_3} }{1+r} \,,
\end{multline}
where $r=m_{D^{(*)}}/m_B$.
The translation table to the form factor convention of Boyd, Grinstein, and Lebed (BGL) \cite{Boyd:1997kz} can be found in Ref.~\cite{Bigi:2017jbd}.
Using the helicity basis has the advantage that the unitarity bounds can be written in an elegant way. 
Employing the BGL parametrization \cite{Boyd:1997kz}
\begin{equation}
  F(z) = \frac{1}{P_F(z) \phi_F(z)} \sum_{n=0}^{\infty} a_n^F z^n, \qquad
  z = \frac{\sqrt{1+w}-\sqrt{2}}{\sqrt{1+w} + \sqrt{2}},
\end{equation} 
with Blaschke factor $P_F(z)$ and outer function $\phi_F(z)$, the unitarity bounds have the simple form
\begin{equation}
\label{eq:unitarity-constr}
  \sum_{n=0}^\infty \left( a_n^F \right)^2 \leq 1\,,
\end{equation}
where \emph{e.g.}, the coefficients of $A_1$ and $A_5$ contribute to the same sum.
The contribution of the helicity form factors $S_1$ and $P_1$ to the branching ratios of $B\rightarrow Dl\nu$ and $B\rightarrow D^*l\nu$ are
suppressed by the mass of the final state lepton. In view of the lack of experimental information on these form factors we need  input from theory. 
In case of $B \to D$ we are in the fortunate position that lattice results
for both $V_1$ and $S_1$ at $w\geq 1$ exist~\cite{Lattice:2015rga,Na:2015kha,Aoki:2016frl}.
This is the reason for the very good agreement of all SM predictions in this case, see Table~\ref{tab:RD-and-RDstar-anomaly}.
For $B \to D^*$ we have only one data point from lattice, namely $A_1(w=1)$~\cite{Bailey:2014tva,Harrison:2017fmw}, so that it is necessary to use 
Heavy Quark Effective Theory (HQET)~\cite{Bernlochner:2017jka,Caprini:1997mu,Luke:1990eg,Neubert:1991xw,Neubert:1993mb,Neubert:1992wq,Neubert:1992pn,Ligeti:1993hw}
to relate the $B \to D$ and $B \to D^*$ form factors order-by-order in the heavy quark expansion in terms of Isgur--Wise functions,
in order to obtain a prediction for $P_1$ and hence $R(D^*)$. 

At NLO in the heavy quark expansion, \emph{i.e.}, expanding to linear order in $\alpha_s/\pi$ and $1/m_{c,b}$,
there is sufficient differential information in the $B \to D^{(*)}l\nu$ decays to fit to the four Isgur-Wise functions
that arise at this order \cite{Bernlochner:2017jka,Jaiswal:2017rve}. 
In the literature the theoretical error from using NLO HQET results is under discussion, leading to different 
results for the error of the SM prediction. 
HQET can also be used to obtain a stronger version of the unitarity bounds than Eq.~(\ref{eq:unitarity-constr}); 
see Refs.~\cite{Boyd:1997kz,Bigi:2017jbd} for details.
Additional model dependent theoretical input at maximal $w$ can be provided by Light Cone Sum Rules (LCSR)~\cite{Faller:2008tr},
or at zero recoil by QCD sum rules~\cite{Neubert:1992wq,Neubert:1992pn,Neubert:1993mb,Ligeti:1993hw}.
We give a summary of theoretical predictions for $R(D^{(*)})$ in Table~\ref{tab:RD-and-RDstar-anomaly}.

Because lattice data is presently available only at zero recoil for $B \to D^*$
and since kinematic suppression requires $d\Gamma[B \to D^*l\nu]/dw$ to vanish at $w=1$,  
$|V_{cb}|$ can be obtained from $B \to D^*l\nu$ only by extrapolating the slope of $d\Gamma[B \to D^*l\nu]/dw$,
fitted to $w > 1$ data, back to zero recoil. 
This procedure can be highly sensitive to the chosen form factor parameterization and other theoretical inputs;
for recent analyses see 
Refs.~\cite{Abdesselam:2017kjf,Bigi:2017njr,Bigi:2017jbd,Grinstein:2017nlq,Bernlochner:2017jka,Bernlochner:2017xyx,Jaiswal:2017rve,Harrison:2017fmw,Schacht:2017vfd}. 
Lattice data beyond zero recoil is expected soon (see \cite{Aviles-Casco:2017nge} for preliminary results),
from both domain wall and AsqTad ensemble approaches. 
Combined with abundant future data for $B \to D^*l\nu$ from HL-LHC, extractions of $|V_{cb}|$ that are less sensitive to theoretical inputs
will become possible, thereby either resolving or more concretely establishing tensions between exclusive and inclusive measurements of $|V_{cb}|$.

With future experimental data and lattice QCD results (see Sec.~\ref{sec:lattice} and Table~\ref{table:Proj} therein) the SM 
prediction of $R(D^*)$ will improve considerably. For an estimate, one may note that the dependence of $d\Gamma/dw$ on $P_1$ arises 
only incoherently, via a contribution of the form $m_l^2|P_1(w)|^2$. This $P_1$ term contributes approximately 10\% of the total 
integrated $B \to D^*\tau\nu$ rate, which suggests that the dependence of $R(D^*)$ on this form factor should be limited. 
Assuming a 1\% future precision for $P_1$, using the 50 ab$^{-1}$ which Belle~II will presumably gather by 2025, and taking into 
account the expected improvement of the form factors $A_{1,5}$ and $V_4$ from the lattice, it is hence reasonable to assume
an error of 0.001 for the SM value of $R(D^*)$ might be achieved.

\begin{table*}[t]
\begin{center}
\begin{tabular}{|ccc|}
  \hline\hline
  Ref.                       & $R(D)$      &  Exp. deviation \\
  \hline
  \cite{Bigi:2016mdz}        & $0.299(3)$  & $2.4\sigma$ \\ 
  \cite{Bernlochner:2017jka} & $0.299(3)$  & $2.4\sigma$ \\
  \cite{Jaiswal:2017rve}     & $0.302(3)$  & $2.3\sigma$ \\
  \hline\hline 
\end{tabular}
\qquad
\begin{tabular}{|ccc|}
  \hline\hline
  Ref.                       & $R(D^*)$   &  Exp. deviation \\
  \hline 
  \cite{Bernlochner:2017jka} & $0.257(3)$ & $3.3\sigma$ \\
  \cite{Bigi:2017jbd}        & $0.260(8)$ & $2.7\sigma$ \\ 
  \cite{Jaiswal:2017rve}     & $0.257(5)$ & $3.1\sigma$ \\
  \hline\hline
\end{tabular}
\caption{
Current SM $R(D^{(*)})$ theory predictions and their deviation from experiment 
$R(D)^{\mathrm{exp}} = 0.407(39)(24)$ \cite{Amhis:2016xyh,Lees:2012xj,Lees:2013uzd,Huschle:2015rga}
and
$R(D^*) = 0.306(13)(7)$ \cite{Amhis:2016xyh,Lees:2012xj,Lees:2013uzd,Huschle:2015rga,Sato:2016svk,Aaij:2015yra,Hirose:2016wfn,Hirose:2017dxl,Aaij:2017uff,Aaij:2017deq}
(HFLAV 2018 summer update). 
See for older SM predictions Refs.~\cite{Fajfer:2012vx,Celis:2012dk,Tanaka:2012nw}.
Table adapted and extended from Ref.~\cite{Schacht:2017vfd}.
\label{tab:RD-and-RDstar-anomaly}} 
\end{center}
\end{table*}

\subsubsection{Excited states and other $b$ hadrons: $R(D^{**})$, $R(J/\psi)$ and $R(\Lambda_c)$}

Measurement of $|V_{cb}|$ and lepton universality can also be probed via $B$ decays to the $D^{**}$ excited states,
as well as decays of strange or charmed $b$ hadrons, including $B_s \to D^{(*)}_s$, $B_c \to J/\psi$,
and the baryonic $\Lambda_b \to \Lambda_c$ transitions. In comparison to the $B \to D^{(*)}\ell \nu$ decay modes, these modes may 
variously exhibit higher sensitivities to various operators of NP, or in some cases,
may be theoretically cleaner than the decays to the $D^{(*)}$ ground states. 
These modes can also be important downfeed or crossfeed backgrounds to the $B \to D^{(*)}\ell \nu$ decays and,
to the extent they are affected by the same NP operators, must also be understood and measured carefully.
The HL-LHC is the only planned experiment that will yield significant samples of these heavier $b$-hadrons, 
with precision analyses anticipated from the LHCb experiment.
In this subsection we present the motivations and theoretical prospects for the measurement of each of these decay modes.

The $D^{**}$ excited states comprise four different charmed hadrons: The $D_0^*$, $D_1^*$, $D_1$, $D_2^*$.
In the language of HQET, these furnish two doublets, $\{D_0^*\,,D_1^*\}$ and $\{D_1\,,D_2^*\}$,
with spin-parity $s_{\ell}^{\pi_\ell} = \frac{1}{2}^+$ and $\frac{3}{2}^+$, respectively.
(In the heavy quark limit, spin-parity is a conserved quantum number.)
The $\frac{3}{2}^+$ states are narrow, with $\Gamma \sim 30$--$50$\,MeV,
because their hadronic decays to $D^{(*)}\pi$ either proceed via a $d$-wave or violate heavy quark symmetry,
while the $\frac{1}{2}^+$ states are quite broad.  Although isolating these excited state decays will likely be simpler at $e^+e^-$ $B$ factories, 
which can more easily reconstruct $\pi^0$'s and photons, analyses of $B \to D^{**}\ell\nu$ decays are also feasible at HL-LHCb, 
especially for the narrow $\frac{3}{2}^+$ states subsequently decaying to charged hadrons.

The crucial attractive feature of the $B \to D^{**}$ transitions is that various leading order contributions to the form factors
vanish in the heavy quark limit at zero recoil ($w=1$), so that $\mathcal{O}(\alpha_s)$ and $\mathcal{O}(1/m_{c,b})$ corrections
become important~\cite{Leibovich:1997tu,Leibovich:1997em,Bernlochner:2012bc,Bernlochner:2016bci}. 
The richer structure of the subleading form factor contributions has the consequence that sensitivity to various NP currents
can be much larger than in the ground state decays~\cite{Bernlochner:2016bci, Bernlochner:2017jxt}. 
For example, including only a NP tensor current, $C_T$, one finds the ratios
$R(X)/R(X)_{\text{SM}} \simeq \{1.5, 1.3\}$ for $X = \{D, D^*\}$ at the best fit to the $R(D^{(*)})$ data. 
However, the same Wilson coefficients result in $R(X)/R(X)_{\text{SM}}$ greater than $4.0$ or less than $0.5$ for $X=\{D_0^*, D_1^*, D_1, D_2^*\}$.
The current SM predictions for all four modes, from fits to Belle data including NLO HQET contributions, are~\cite{Bernlochner:2017jxt}
\begin{align}
\label{eq:RDDs}
  R(D_0^*) &= 0.08 \pm 0.03\,, & R(D_1^*) &= 0.05 \pm 0.02\,, \nonumber \\*
  R(D_1) &=  0.10 \pm 0.02\,, & R(D_2^*) &= 0.07 \pm 0.01\,.
\end{align}

Decays of $B$ mesons to these excited states have total SM branching ratios comparable to the $B \to D^{(*)}\ell \nu$ decays themselves.
Combined with the possible large enhancement of the semitauonic modes by NP contributions,
this means that the subsequent $D^{**} \to D^{(*)}X$ decays can then induce important downfeed  backgrounds to the $D^{(*)}$ measurements.
Analyses of $B \to D^{(*)}\ell \nu$ typically will therefore have to fold in contributions from these excited states.
Moreover, anticipated analyses for the inclusive $B \to D\pi\ell\nu$ decays provide an opportunity to probe these excited state decays,
and their associated larger NP sensitivities, \emph{collectively} with the ground state decays:
in this context, rather than being thought of as a background, these contributions should be more properly thought of as additional sources of (NP) signal.
The large data sets from HL-LHC will then provide a very sensitive set of channels for probing NP contributions to $b \to c \ell \nu$. {\bf \color{red}[Check!]}

The $B_s$ and $B_c$ mesons have production ratios $\sigma(B_s)/\sigma(b\bar{b}) \sim 10\%$~\cite{Aaij:2013noa}
and $\sigma(B_c)/\sigma(b\bar{b}) \sim 0.2\%$~\cite{Aaij:2013cda} from Run 1 LHC data.
A much smaller sample of $B_s$ mesons may also be produced at $B$ factories running on the $\Upsilon(5S)$ resonance.
However, for the $B_c$ the only significant sample of mesons will be produced at HL-LHC.

At first glance, the theoretical structure of $B_s \to D_s^{(*,**)}\ell\nu$ can be mapped directly from $B \to D^{(*,**)}\ell\nu$
via the approximate $SU(3)$ flavor symmetry.
Some crucial differences are that $\mathcal{B}(D_s^{*} \to D_s\gamma) \simeq 94\%$, which will be difficult to see at HL-LHCb,
while the four $D_s^{**}$ excited states are all narrow, and therefore may be easier to resolve.

The leptonic $B_c \to (J/\psi \to \mu\mu) \ell \nu$ decay mode is reasonably experimentally clean with measurements for $R(J/\psi)$ already available from LHCb Run 1 data~\cite{Aaij:2017tyk},
\begin{equation}
  R(J/\psi) = 0.71 \pm 0.17\,\text{(stat)}\, \pm 0.18\,\text{(sys)} \,,
\end{equation}
albeit with large uncertainties at present. A central difficulty in probing this mode lies in the large theoretical uncertainties for the $B_c \to J/\psi$ form factor parameterizations. Predictions for the form factors are typically hadronic-model-dependent, making use of either perturbative QCD, the constituent quark model, the (non)relativitistic quark model, or QCD sum rules~\cite{Anisimov:1998uk,Kiselev:1999sc,Ivanov:2000aj,Kiselev:2002vz,Hernandez:2006gt,Ivanov:2006ni,Wen-Fei:2013uea,Qiao:2012vt,Rui:2016opu}. The recent LHCb results have motivated several more recent studies of the form factors~\cite{Dutta:2017xmj,Tran:2018kuv,Issadykov:2018myx,Watanabe:2017mip}. A recent, more model-independent result combines preliminary lattice QCD results with dispersive bounds and zero-recoil heavy-quark relations, leading to the prediction~\cite{Cohen:2018dgz}
\begin{equation}
  0.20 \le R(J/\psi) \le 0.39
\end{equation}
at $95\%$ CL, implying a mild tension at the $1.3\sigma$ level with the data.

Abundant samples of $\Lambda_b$'s will be produced only at (HL-)LHC, with a production cross-section $\sigma(\Lambda_b)/\sigma(b\bar{b}) \sim 10\%$~\cite{Aaij:2011jp}. 
From an HQET point of view, the $\Lambda_b \to \Lambda_c$ transitions are theoretically cleaner than the $B \to D^{(*)}$ decays,
because the ``brown muck'' dressing the heavy quark lies in the $s_{\ell}^{\pi_\ell}= 0^+$ ground state.
A consequence of this is a relatively simpler form factor structure where not only the $\mathcal{O}(\alpha_s)$
but also the $\mathcal{O}(1/m_{c,b})$ and $\mathcal{O}(\alpha_s/m_{c,b})$ subleading contributions
are fully fixed by the leading order HQET structure, reducing the number of free parameters in the form factor fits.
These modes are therefore promising, clean candidates for testing the behavior of the HQET expansion itself,
by \emph{e.g.}, assessing the impact of $\mathcal{O}(1/m_c^2)$ contributions.
Fitting the subsubleading $\mathcal{O}(\alpha_s, \alpha_s/m_{c,b}, 1/m_c^2)$ HQET structure to existing LHCb data~\cite{Aaij:2017svr}
and lattice form factor results~\cite{Detmold:2015aaa} implies such terms are of the expected size~\cite{Bernlochner:2018kxh}.
More $\Lambda_b \to \Lambda_c \ell \nu$ data from LHCb will improve the precision of these results, allowing access to other subleading terms.
Moreover, with precision lattice calculations of the form factors (see \emph{e.g.}, Ref.~\cite{Detmold:2015aaa}), additional data for these modes may permit precision measurement of $|V_{cb}|$ in an environment with reduced theoretical uncertainties.

\subsubsection{Models of NP for $b\to c\tau\nu$}
{\bf\color{red}[Move this to the end of the section?]}\\

The first thing to note in a discussion of a possible NP origin of the deviations in $R(D^{(*)})$ is that such a NP contribution has to be large: Defining $\hat{R}(X)=R(X)/R(X)|_{\rm SM}$, we have with present data $\hat{R}(D)=1.36 \pm 0.15$ and $\hat{R}(D^*)=1.19 \pm 0.06$. This points to a NP contribution to the amplitude $\gtrsim 10\%$ of the SM for deviations driven by NP and SM interfering and $\gtrsim 40\%$ without interference. As a consequence, NP is expected to contribute at tree level, for which possible mediators involving only SM fields have been classified in~\cite{Freytsis:2015qca}. If the present central value of the deviation is sustained, another implication is that this anomaly will be established by the time of measurements at HL/HE-LHC, as illustrated in Fig.~\ref{fig:projections} on the left. The focus will hence shift to model differentiation in $b\to c\tau\nu$ and the analysis of the lepton- and quark-flavor structure of the NP contributions. In any case, precision measurements in $b\to c\tau\nu$ discussed here will provide the strictest direct limits on flavor-non-universality in these transitions and hence remain valuable constraints on NP models. A completely general NP analysis will require theoretical care. For instance, form factor determinations from $b\to c$ semileptonic decays with light leptons in the final state could also be sensitive to NP contributions, subject to the constraint that extractions of $|V_{cb}|$ from the selfsame decays can be made consistent with all other global data on CKM unitarity. Consequently, form factor parameters determined experimentally in these modes may require a simultaneous fit to the deviations themselves or additional determinations of form factor ratios, which, however, are expected to be available at the required precision by the start of HL/HE-LHC. See Sec.~\ref{sec:lattice} for corresponding prospects from lattice QCD.

\begin{figure}
\includegraphics[width=0.5\textwidth]{section7/figures/RDRDstar_projection.png}
\qquad\parbox[b]{0.5\textwidth}{\includegraphics[width=7cm,height=4cm]{section7/figures/plotRDstarAlambdastar18.png}\\
\includegraphics[width=7cm,height=4cm]{section7/figures/plotRDstarFLDstar.png}}
\caption{\label{fig:projections}
Left: Present status of the $R(D)$-$R(D^*)$ anomaly, showing the individual measurements (68\% CL contours), the world average (68\% and 95\% CL filled ellipses), the SM prediction (95\% CL filled ellipse) and the projections for LHCb measurements by 2025 (dashed contour) and after the second upgrade (dotted contour, both at 68\% CL, assuming the present central value).
Right: Correlations between $R(D^*)$ and the $\tau$-polarization asymmetry $A_\lambda(D^*)[=-P_\tau(D^*)]$ (upper panel) and the the longitudinal fraction $F_L(D^*)$ (lower panel) for NP scenarios with only left-handed vector coupling (green) or only scalar couplings (blue), together with the recent first measurements~\cite{Hirose:2016wfn,Hirose:2017dxl} (light blue) and the experimental average for $R(D^*)$ (excluding the measurement \cite{Hirose:2016wfn} in the upper panel). Updated and adapted from Ref.~\cite{Celis:2016azn}.}
\end{figure}

Classifying models accommodating the anomalies according to tree-level mediator and to the contribution they give in terms of the effective operators in eq.~(\ref{eq:Heffbcellnu}), the options are a $W'$~\cite{Greljo:2015mma,Boucenna:2016wpr,Boucenna:2016qad,Megias:2017ove}, generating $C_{V_L}$, a charged color-neutral scalar~\cite{Crivellin:2012ye,Celis:2012dk,Crivellin:2015hha,Celis:2016azn,Chen:2017eby,Iguro:2017ysu,Chen:2018hqy,Li:2018rax}, generating $C_{S_{L,R}}$, and leptoquarks~\cite{Fajfer:2012jt,Deshpande:2012rr,Sakaki:2013bfa,Duraisamy:2014sna,Calibbi:2015kma,Fajfer:2015ycq,Barbieri:2015yvd,Alonso:2015sja,Bauer:2015knc,Das:2016vkr,Deshpand:2016cpw,Sahoo:2016pet,Dumont:2016xpj,Li:2016vvp,Becirevic:2016yqi,Barbieri:2016las,DiLuzio:2017vat,Chen:2017hir,Bordone:2017bld,Altmannshofer:2017poe,Becirevic:2018afm}, generating various couplings, mostly $C_{V_L}$ or $C_{S_L} \sim C_T$. For comparisons between these models,  see for instance~\cite{Tanaka:2012nw,Freytsis:2015qca,Bhattacharya:2016zcw,Ivanov:2017mrj,Alok:2017qsi,Bifani:2018zmi}.
Allowing for additional light particles opens up the possibility to address the anomalies with contributions involving right-handed neutrinos~\cite{He:2012zp,He:2017bft,Greljo:2018ogz,Asadi:2018wea,Robinson:2018gza,Azatov:2018kzb,Heeck:2018ntp}, since the neutrino is not detected. 

The large number of possibilities reflects the fact that models typically introduce at least 2 degrees of freedom (one complex coupling), which mostly allows to address $R(D^{(*)})$.\footnote{Models generating only $C_{V_L}$ effectively introduce only one degree of freedom, since they amount to rescaling $V_{cb}$ in the modes involving $\tau$; hence this class of models implies universal $\hat R(X)$ ratios, specifically $\hat R(D)=\hat R(D^*)$.} In order to differentiate between these models, a precise determination of $R(D^{(*)})$ can therefore only be a first step. Presently, available additional observables constraining directly $b\to c\tau\nu$ transitions are the following: (a) Differential distributions in $q^2$~\cite{Lees:2013uzd,Huschle:2015rga}, already excluding some fine-tuned scenarios despite their large uncertainties. (b) First measurements of the $\tau$ polarization asymmetry and the longitudinal fraction in $B\to D^*\tau\nu$~\cite{Hirose:2016wfn,Hirose:2017dxl}. (c) The measurement of $R(J/\psi)$~\cite{Aaij:2017tyk}, which is up to $2\sigma$ above the SM prediction, as well as that of any NP model, for which see the previous subsection. (d) The measurement of the inclusive rate $b\to X\tau\nu$ at LEP~\cite{Abbaneo:2001bv}, yielding $\hat R_b(X_c)=0.99\pm0.10$, in slight tension with the measurements for $R(D^{(*)})$~\cite{Ligeti:2014kia,Freytsis:2015qca,Celis:2016azn,Mannel:2017jfk}. (e) The (indirect) bound on $B_c \to \tau\nu$ from the $B_c$ lifetime~\cite{Li:2016vvp,Alonso:2016oyd,Akeroyd:2017mhr}, providing a strong constraint on models with only scalar couplings and disfavoring them as a solution for $R(D^*)$ for values close to the present central value.

Additional indirect constraints apply when trying to formulate actual models, as opposed to only considering effective couplings at the scale of $m_b$: high-$p_T$ searches for signatures related to potential mediators of these transitions often provide strong constraints via, \emph{e.g.}, $b\bar{b} \to \tau\tau$ or $h\to\tau\tau$~\cite{Faroughy:2016osc,Feruglio:2018fxo}, and the renormalization-group evolution of the relevant operators from the mediator scale down to the $b$ scale can induce constraints from lepton-universality in $\tau$ decays~\cite{Feruglio:2017rjo}, lepton-flavor violating decays~\cite{Feruglio:2017rjo}, charged-lepton magnetic moments~\cite{Feruglio:2018fxo} and electric dipole moments in models with non-vanishing imaginary parts~\cite{Dekens:2018bci}. Current data on $\Upsilon(1S) \to \tau\tau$ decays also constrains most mediator models, and a future program of measuring both $\Upsilon$ and $\psi$ decays can have sensitivity to all completions~\cite{Aloni:2017eny}. The current bounds on $b \to s \nu\bar{\nu}$, now only $\mathcal{O}(1)$ above the SM~\cite{Lees:2013kla,Grygier:2017tzo}, can also be severe constraints on particular models~\cite{Blake:2016olu}, although their interpretation is more model-dependent, being sensitive to assumptions on both mediator and specific flavor structure of couplings.

At the HL/HE-LHC qualitative progress in identifying potential NP in $b\to c\tau\nu$ can be understood chiefly in terms of two classes of observables arising from analyses of the data:
%
\begin{itemize}
\item[\bullet] Precision results for $R(X)$: $R(D^{(*)})$ are used for establishing NP in these modes, as well as basic model differentiation. $R(\Lambda_c)$, as discussed in the last subsection, is sensitive to a different combination of NP parameters and hence provides a measurement with independent systematics as well as improved model discrimination. The same is true for the inclusive measurement $R(X_c)$, which, however, is expected to be measured at Belle~(II) only. Other modes, like $R(D_s^{(*)})$ or $R(J/\psi)$ will add to these and provide cross-checks with independent systematic uncertainties; however, $R(D_s)$ is expected to have very similar NP dependence as $R(D)$, while $R(D_s^*)$ and (less so) $R(J/\psi)$ constrain similar combinations as $R(D^*)$, hence their model-differentiating power is limited. 
\item[\bullet] Differential-rate measurements in $B\to X\tau\nu$: Differential measurements in $q^2$ as well as in the helicity angles and the $\tau$-polarization are powerful discriminators between the SM and NP, as well as between different NP models \cite{Korner:1987kd,Hagiwara:1989gza,Tanaka:1994ay,Chen:2005gr,Chen:2006nua,Nierste:2008qe,Tanaka:2010se,Fajfer:2012vx,Datta:2012qk,Sakaki:2012ft,Celis:2012dk,Duraisamy:2013kcw,Alonso:2016gym,Ligeti:2016npd,Celis:2016azn,Ivanov:2017mrj,Alonso:2017ktd,Jung:2018lfu}.
The different dependence on kinematic variables allows for optimized searches for specific NP contributions. For instance, the low-recoil region in $B\to D\tau\nu$ is very sensitive to scalar contributions, tensor contributions change the polarization in the high-recoil region in $B\to D^*$ in a unique manner and left-handed vector contributions leave normalized quantities unchanged while affecting the total rates sizably. Examples for the discriminating power of such measurements are given in Fig.~\ref{fig:projections} on the right, where correlations in NP models are shown together with the first measurements of two proposed quantities by Belle~\cite{Hirose:2016wfn,Hirose:2017dxl}. First studies regarding the reach of LHCb for such observables are presented in the following subsection. These measurements are highly nontrivial, given the missing neutrinos and the fact that so far for instance the expected $q^2$ distribution has been used in the measurements of $R(D^*)$ for background suppression. However, already semi-integrated quantities like the forward-backward asymmetry or the observables shown in Fig.~\ref{fig:projections} can be very powerful in distinguishing different NP scenarios.
\end{itemize}

At high-$p_T$, the $b c \tau \nu$ operator can also be directly probed in the $pp \to b\tau + \slashed{E}_T$ final state, with model discrimination possible at the beyond the $3\sigma$ level in at the HL-LHC~\cite{Altmannshofer:2017yso} (see Sec.~\ref{sec:highpT}). {\bf\color{red}[Update! (Jorge)]} Analysis of NP in $b \to c \tau \nu$ will be made more powerful and self-consistent by the development of dedicated NP reweighting tools such as \texttt{Hammer}~\cite{Duell:2016maj}. These tools will permit experimental collaborations to efficiently reweight their very large simulated datasets to arbitrary NP models. NP analyses can then be performed directly on their $b \to c \tau \nu$ data, fitting the full differential space of the decay cascades to arbitrary NP models and simultaneously floating against backgrounds to obtain self-consistent results for best-fit NP Wilson coefficients, including possible NP contributions in the light lepton modes.


Observation of NP in $b\to c\tau\nu$ would warrant an analysis of its flavor structure, \emph{e.g.}, in $b\to c (e,\mu)\nu$, $b\to u\tau\nu$ and $t\to b\tau\nu$ transitions. While these are not a focus here, it is worth emphasizing that the HL/HE-LHC will be a powerful instrument to analyze these modes. 


\subsection{Experimental perspective (including interplay with Belle II)}
\subsubsection{Measurements of $b\to ll$ from LHCb/ATLAS/CMS}

In 2017 LHCb measured~\cite{LHCb-PAPER-2017-001} ${\cal B}(\decay{\Bs}{\mumu})=\left(3.0\pm 0.6^{+0.3}_{-0.2}\right)\times 10^{-9}$ and ${\cal B}(\decay{\Bd}{\mumu})=\left(1.5^{+1.2 +0.2}_{-1.0 -0.1}\right) \times 10^{-10}$ using data collected in $pp$ collisions corresponding to a total integrated luminosity of 4.4\invfb (Fig.~\ref{fig:btomumu}, left). Both measurements are compatible with the SM predictions but statistically limited. 

\begin{figure}[!tb]
\centering
\includegraphics[scale=0.37]{section7/figures/btomumu_BF.pdf}
\includegraphics[scale=0.4]{section7/figures/btomumu_decaytime.pdf}
\caption{(Left) A two-dimensional representation of the LHCb branching fraction measurements for \Bdmm and \Bsmm. The central values are indicated with the black plus sign. The profile likelihood contours for 1,2,3\dots $\sigma$ are shown as blue contours. The Standard Model value is shown as the red cross labelled SM. (Right) Background-subtracted \Bsmm decay-time distribution with the fit result superimposed.}
\label{fig:btomumu}
\end{figure}

While the uncertainty of ${\cal B}(\decay{\Bd}{\mumu})$  remains statistically limited on a 300\invfb dataset, the projected uncertainty of ${\cal B}(\decay{\Bs}{\mumu})$ depends on the assumptions made for the systematic uncertainties. The current systematic uncertainty is dominated by the 5.8\% relative uncertainty associated to the $b$-quark fragmentation probability ratio ($f_s/f_d$)~\cite{fsfd} followed by approximately 3\% from the branching fractions of the normalisation modes, 2\% from particle identification and 2\% from track reconstruction.
The projected relative statistical uncertainty of the current analysis on a dataset of 300\invfb is 1.8\%. 
At the end of the Upgrade II data taking period, it is reasonable to assume a systematic uncertainty on ${\cal B}(\decay{\Bs}{\mumu})$ of about 4\%, which
would imply an
uncertainty of ${\cal B}(\decay{\Bs}{\mumu})$ to be approximately $0.30\times 10^{-9}$ with 23\invfb and $0.16\times 10^{-9}$ with $300\invfb$. The increased precision will be able to cover larger part of the unconstrained parameter space of MSSM models. 
The ratio of branching fractions, ${\cal B}(\decay{\Bd}{\mumu})/{\cal B}(\decay{\Bs}{\mumu})$, is a powerful observable to test minimal flavour violation. The relative uncertainty of this ratio is expected to remain limited only by statistics and decrease from $90\%$ for the current measurement to about $34\%$ with $23\invfb$ and $10\%$ with $300\invfb$. 
All estimates use the quoted SM predictions as central values for the branching ratios and similar detector and analysis performance as in Ref.~\cite{LHCb-PAPER-2017-001}. 

The estimated experimental uncertainties at 300\invfb are close to the uncertainty of the current SM prediction from theory, which is dominated by the uncertainty of the $\Bs$ decay constant, determined from lattice QCD calculations, and the CKM matrix elements. Both are expected to improve in precision in the future. 

With a 300\invfb dataset, precise measurements of additional observables are possible, namely the effective lifetime ($\tau_{\mu\mu}^{\rm eff}$) and the time-dependent \CP\ asymmetry of \Bsmm decays. Both quantities are sensitive to possible new contributions from the scalar and pseudo-scalar sector in a way complementary to the branching ratio measurement\cite{DeBruyn:2012wk}. 

The effective lifetime is related to the mean \Bs lifetime $\tau_{B_s}$ through the relation 
\begin{equation}
\tau_{\mu\mu}^{\rm eff}=\frac{\tau_{B_s}}{1-y^2_s}\frac{1+2A^{\mu\mu}_{\Delta\Gamma}y_s+y^2_s}{1+A^{\mu\mu}_{\Delta\Gamma}y_s}\,, 
\end{equation}
where $y_s=\tau_{B_s}\Delta\Gamma_s/2$ and $\Delta\Gamma_s=\Gamma_{B^0_{sL}}-\Gamma_{B^0_{sH}}$. The parameter $A^{\mu\mu}_{\Delta\Gamma}$ is equal to 1 in the SM, where only the heavy mass eigenstate decays to \mumu, but can take any value between $-1$ and 1 in scenarios beyond the SM. LHCb has performed the first measurement of the \Bsmm effective lifetime using a dataset of 4.4\invfb, resulting in $\tau_{\mu\mu}^{\rm eff}=2.04\pm 0.44\pm 0.05 \ps$~\cite{LHCb-PAPER-2017-001} (Fig.~\ref{fig:btomumu}, right). The relative uncertainty on $\tau_{\mu\mu}^{\rm eff}$ is expected to decrease to approximately 8\%  with 23\invfb and 2\% with 300\invfb, being statistically limited. While the current experimental uncertainty is larger than $\tau_{B^0_{sH}}-\tau_{B^0_{sL}}$,
a 2\% uncertainty on $\tau_{\mu\mu}^{\rm eff}$ would allow to set stringent constraints on $A^{\mu\mu}_{\Delta\Gamma}$ and in particular would allow to break the degeneracy between any possible contribution from new scalar and pseudoscalar mediators.

Assuming a tagging power of about 3.7\%\cite{LHCb-PAPER-2014-059}, a dataset of 300\invfb allows to reconstruct a pure sample of more than 100 flavour-tagged \Bsmm decays (effective yield) and measure their time-dependent \CP asymmetry. From the relation  
\begin{equation}
\frac{\Gamma(B^0_s(t)\to\mu^+\mu^-)-\Gamma(\bar{B^0_s}\to\mu^+\mu^-)}{\Gamma(B^0_s(t)\to\mu^+\mu^-)+\Gamma(\bar{B^0_s}\to\mu^+\mu^-)}=\frac{S_{\mu\mu}\sin(\Delta M_st)}{\cosh(y_st/\tau_{B_s})+A^{\mu\mu}_{\Delta\Gamma}\sinh(y_st/\tau_{B_s})}\,, 
\end{equation}
where $t$ is the signal proper time and $\Delta M_s$ is the mass difference of the heavy and light \Bs mass eigenstates, $S_{\mu\mu}$ can be measured with an uncertainty of about 0.2. On the other hand, the signal yield expected in a 23\invfb dataset is too low to allow a meaningful constraint to be set on $S_{\mu\mu}$. A nonzero value for $S_{\mu\mu}$ would automatically indicate evidence of \CP-violating phases beyond the SM.  

Being sensitive to a wider set of effective operators ($\mathcal O_7$, $\mathcal 
O_9$ and $\mathcal O_{10}$)~\cite{Kruger:2002gf}, the \bsmumugamma decay offers 
an interesting counterpart to \bsmumu. 
The theoretical branching fraction is expected to be one order of magnitude 
larger than the \bsmumu one~\cite{Melikhov:2004mk}, owing to the removal of the helicity suppression, 
when integrated over the full $q^2$ spectrum. 
However the presence of the photon makes the direct reconstruction 
challenging at LHCb.
No limit exist today on the \bsmumugamma channel, 
while the \bdmumugamma is limited at $1\times 10^{-7}$ at 90\% CL by the 
\babar experiment~\cite{Aubert:2007up}.

Given the experimental difficulty, two complementary techniques are employed  
for the study of the \bsmumugamma decay at LHCb. The first is a full reconstruction
which is more sensitive at low and mid $q^2$, where the photon energy is higher, 
and the second, recently proposed in Ref.~\cite{Dettori:2016zff}, without photon reconstruction
but only sensitive at high $q^2$. 

The only non-negligible partially reconstructed background is the (not yet 
measured) $B_d \to \mu^{+} \mu^{-} \pi^{0} \xspace$, whose branching fraction is 
theoretically estimated to be of the same order of magnitude as the signal.
The main difficulty of the measurement is therefore the combinatorial 
background, because the uncertainty on the photon momentum enlarges the signal 
width and blurs its kinematics.
Based on current reconstruction efficiencies, the expected sensitivity at the 
end of Upgrade I (Upgrade II) data taking period is $\sim 9 \sigma$ ($\sim 22 \sigma$).
The use of $B_s \to J/\psi \eta \xspace$ and $B_d \to K\pi\gamma \xspace$ as 
normalisation channels reduces the systematic uncertainties due to the selection.

The partially reconstructed method consists of studying the \bsmumugamma decay 
as a shoulder on the left of the \bsmumu peak in the dimuon mass distribution. 
The SM contribution as background has been considered negligible so far.
Conversely large branching fractions could be easily excluded~\cite{Dettori:2016zff}
when considering this as additional component. 
The SM branching fraction for this region would be around $2 \times 10^{-10}$, 
implying a first observation would be possible with Run 3 and certainly with Run 4 data, 
while extremely tight limits could already be determined with Run 2. 

\subsubsubsection{ATLAS/CMS}

HL-LHC will offer a compelling opportunity to extend the ATLAS and CMS  $B^0_s \to \mu^+\mu^-$ and  $B^0\to \mu^+\mu^-$ studies to the expected integrated high luminosity.
\par\noindent
Flexible trigger systems and inner tracker improvements will allow both experiments to maintain efficient low pT dimuon triggers and achieve good mass resolution, which are the key ingredients for the $B_{d,s} \to 2 \mu$ analysis
\par\noindent
Fig.~\ref{fig:CMSMass} shows the CMS reconstructed $B^0_s \to \mu^+\mu^-$ and  $B^0\to \mu^+\mu^-$ invariant in the barrel (left) and forward region. 
Fig.~\ref{fig:ATLASMAss} shows the $B^0_s \to \mu^+\mu^-$ reconstructed invariant mass for $|\eta| < 2.5$

The present ATLAS projections are extrapolated from the ATLAS Run I analysis \cite{Aaboud:2016ire}. Assumptions include the training of a multivariate classifier capable of similar background rejection and signal purities and an analysis selection with comparable pile-up immunity as Run 1.
The study takes into account the scaling of B production cross-section and integrated luminosity relative to Run 1, and explores different triggering scenarios corresponding to different dimuon transverse momentum thresholds:
($p_T^{\mu_1}$,$p_T^{\mu_2}$): $(6 \,\text{GeV},6 \,\text{GeV})$, $(6 \, \text{GeV},10 \,\text{GeV})$ and $(10 \,\text{GeV},10 \,\text{GeV})$.

Depending on these dimuon thresholds, three working points are obtained from detector simulations. These are expressed in terms of signal and background statistics relative to Run 1: x15 (conservative), x60 (intermediate) and x75 (high-yield).
An unbinned maximum-likelihood fit is applied to datasets generated in the assumptions discussed above, in order to produce 68.3\%, 95.5\% and 99.7\% likelihood contours.
For each level, contours are calculated for statistical-only and statistical+systematic uncertainties. The Run 1 analysis parametrizes two classes of systematic uncertainties: the
ones coming from external inputs (e.g. the $f_s/f_d$ ratio and
the $B^+\to J/\psi [\to\mu\mu]K^\pm$ branching ratio) and those depending on internal analysis effects (invariant mass fit shapes, efficiencies, etc.).
The latter category is parameterized as a function of the signal yields where dependencies are found to be significant.\\
This study extrapolates the same systematic uncertainties including the same signal yield parameterization found in the Run 1 studies. As for the external sources of systematic uncertainties, it is plausible to expect that these will be reduced with other measurements and could optimistically scale for the most part like
statistical uncertainties. This study however conservatively assumes their values to be those used in the Run 1 analysis.

Figures~\ref{fig:HLLHCextrapolation_x15}, \ref{fig:HLLHCextrapolation_x60} and \ref{fig:HLLHCextrapolation_x75} summarise the contours derived in these assumptions,
in the $BR\left[B^0 \to \mu\mu\right]$-$BR\left[B_{s}^0 \to \mu\mu\right]$ plane. Table~\ref{BMuMUTable} provides the estimated statistical and statistical+systematic uncertainties on the two branching ratios expected for the extrapolations discussed.

\begin{table}
  \centering
  \caption{Uncertainty on $\cal{B}$$(B^0_s \to \mu^+ \mu^-)$ and $\cal{B}$$(B^0 \to \mu^+ \mu^-)$ as reported
    by the fitting procedure applied to  the toy simulations. The results are centered on the SM theoretical prediction~\cite{Bobeth}.
    For each extrapolation performed, statistical and statistical+systematic uncertainties are reported in units of $10^{-10}$.
    The table reports a sufficient number of significant digits to highlight the difference between statistical+systematics and
    systematics-only uncertainties.
  }
  \label{BMuMuTable}
  \renewcommand{\arraystretch}{1.3}
  \begin{tabular}{ l c c c c }
    \hline
    & \multicolumn{2}{c}{$\cal{B}$$(B^0_s \to \mu^+ \mu^-)$ } & \multicolumn{2}{c}{$\cal{B}$$(B^0 \to \mu^+ \mu^-)$} \\
    & stat $[10^{-10}]$ & stat + syst $[10^{-10}]$ & stat $[10^{-10}]$ & stat + syst $[10^{-10}]$ \\
    \hline
    Run 2 & 7.0 & 8.3 & 1.42 & 1.43\\
    HL-LHC: Conservative  & 3.2 & 5.5 & 0.53 & 0.54\\
    HL-LHC: Intermediate  & 1.9 & 4.7 & 0.30 & 0.31\\
    HL-LHC: High-yield  & 1.8 & 4.6 & 0.27 & 0.28\\
    \hline
  \end{tabular}
\end{table}


\begin{figure}[ht]
  \centering  
  \includegraphics[width=1\textwidth]{\main/section7/figures/plot_preliminary_x15.pdf}
  \caption{Comparison of 68.3\% (solid), 95.5\% (dashed) and 99.7\% (dotted) confidence level profiled likelihood ratio contours for the
    working point at $\times 15$ Run 1 statistics. This corresponds to the 'conservative' HL-LHC extrapolation, based on yield
    projections for the $(10 \text{GeV},10 \text{GeV})$ dimuon trigger.
    Red contours do not include the systematic uncertainties, which are then included
    in the blue ellipsoids. The black point shows the SM theoretical prediction and its uncertainty \cite{Bobeth}.}
  \label{fig:HLLHCextrapolation_x15}
\end{figure}


\begin{figure}[ht]
  \centering
\includegraphics[width=1\textwidth]{\main/section7/figures/plot_preliminary_x60.pdf}
  \caption{Comparison of 68.3\% (solid), 95.5\% (dashed) and 99.7\% (dotted) confidence level profiled likelihood ratio contours for the
    working point at $\times 60$ Run 1 statistics. This corresponds to the 'intermediate' HL-LHC extrapolation, based on yield
    projections for the $(6 \text{GeV},10 \text{GeV})$ dimuon trigger.
    Red contours do not include the systematic uncertainties, which are then included
    in the blue ellipsoids. The black point shows the SM theoretical prediction and its uncertainty \cite{Bobeth}.}
  \label{fig:HLLHCextrapolation_x60}
\end{figure}

The current CMS projection, compared to the previous Run-I analysis \cite{CMSRun1},  includes a more advanced UML fit and an improved muon identification algorithm. 
The impact of a PU scenario of 200 interactions was found not to have strong impact on the isolation variable distributions. The signal-to-background ratio (S/B), which depends on momentum resolution, is best in the barrel region and degrades  if one or both muons are detected in the forward region. Therefore, the analysis is performed in two different regions defined by the pseudorapidity of the most forward muon   $|\eta_f|$, defined  as $|\eta_f|  < 0.7$ and  0.7 < $|\eta_f|  <1.4$.
Fig.~\ref{fig:CMSMass}  shows the invariant mass distributions with the fit projection overlayed, corresponding to an
integrated luminosity of 3000 fb$^{-1}$ for the two $\eta_f|$ regions.

 \begin{figure}[htbp]
   \begin{center}
    \includegraphics[width=0.5\textwidth]{\main/section7/figures/CMS_mass_barrel.pdf}
     \includegraphics[width=0.5\textwidth]{\main/section7/figures/CMS_mass_forward.pdf}
       \caption{Invariant mass distributions with the fit projection overlayed, corresponding to an
integrated luminosity of 3000 fb$^{-1}$. 
    The left plot shows the central barrel region, $\eta| $< 0.7 and the right plot is for to 0.7< $\eta|$ < 1.4.}
     \label{fig:CMSMass}\end{center}\end{figure}\setlength{\extrarowheight}{4pt}     


An unbinned maximum likelihood fit to the dimuon invariant mass distribution is performed to determine the $\mathrm{B^0_s} \to \mu ^+ \mu^-$ effective 
lifetime. Based on the fitting results, a projection along the proper decay time distribution for the $B^0_s$ signal events is built with the 
sPlot technique [ref for sPlot and FTR-18-013]. The model used in the lifetime fit is based on an exponential function, convoluted with a Gaussian
function which describes the expected decay time resolution.

Fig~\ref{fig:lifetime} shows the $\mathrm{B^0_s}$ decay time distribution with the fit projection superimposed.
The effective lifetime from the fit is 1.61 $\pm$ 0.05 ps.

   \begin{figure}[htbp]
   \begin{center}
    \includegraphics
    [width=0.5\textwidth]{\main/section7/figures/CMS_tau_splot_toy_bs.pdf}
       \caption{The binned maximum likelihood fit to the decay time distribution for the Phase-II scenario}
     \label{fig:lifetime}\end{center}\end{figure}\setlength{\extrarowheight}{4pt}     

The major sources of systematic uncertainties, listed in Table~\ref{tab:b2musys}, are from external physics parameters 
(e.g. fu/ fs ratio and SM branching fractions of  $\mathrm{B}\to  \mu^+ \mu^-$  and B$^+ \to   \mathrm{J}/\psi K^+$ ) and those depending on internal 
analysis effects (e.g. individual signal and background yields, efficiencies, etc.) 

The uncertainty on the muon identification(ID) efficiency ratio is determined by the difference of data and MC efficiency ratio 
from $\mathrm{B^+ \to  J}/\psi K^+$ and $\mathrm{B^0_s \to J}/\psi \phi$ is assumed to diminish to 1 \% 
for the Phase-II case.  In this study, it is expected that the dimuon trigger for the signal and normalization channels will 
remain the same 
The major systematics shown in Table~\ref{tab:b2msys} are implemented in the PDFs as nuisance parameters.The external input to UML, the value of 
$f_u/f_s$ relative uncertainty is currently 5.8\%: This uncertainty is assumed to be 3.5 \%, which is dominated by systematic uncertainties 
from form-factor ratios and branching fraction measurements.

The systematic uncertainty on the normalization yield enters the branching fraction directly. We determine this uncertainty from the yield 
difference between the results of the unconstrained $\mathrm{B+}$ and fit the J/$\psi$ mass-constrained invariant mass distribution.
The Belle II collaboration is expected to be able to improve this measurement to 1,4 \%.
The uncertainty due to the peaking and semileptonic backgrounds is currently dominated by the uncertainty on the hadron misidentification probability.
To determine the systematic uncertainty on the muon misidentification probability for pion, kaons and protons, we calculate the bin-averaged 
misidentification probability for data and MC simulation. From these values and their uncertainties we calculate the error-weighted average and
its uncertainty. The relative uncertainty of this average is used as systematic uncertainty.
We assume the uncertainty on hadron misID to be 10\%.
As a result of this, the relative uncertainties on the yield of peaking and semileptonic background are 20 \% and 15 \% respectively, during Run 2.
As a reasonable assumption, these two uncertainties are expected to reduce by a factor of 2 and to be 10 \% and 7.5 \% respectively for the Phase II
scenario.

The selection efficiency depends on the $\mathrm{B^0_s}$ effective lifetime.and its uncertainty assumed to be 2\% during Phase II era.
The systematic uncertainty for the determination of the $\mathrm{B^0_s} \to \mu^+\mu^-$ effective lifetime will be limited by the knowledge 
of the trigger efficiency as a funztion of the $\mathrm{B^0_s}$ decay time.
We expect that this uncertainty can be well measured with the $\mathrm{B^+} \to J/\psi K^+$ decay using simila triggers ( except for the dimuon mass
range).
The trigger efficiency is estimated by the difference between the dimuon triggers for signal and for j/$\psi$ which is assumed to dimish to 1.5 \% 
Other sources of systematics (acceptance, selection, etc) have been studied and the related uncertainties halved: the sensitoivity for 
the significance of the $\mathrm{B^0}$ observation ans its branching fraction is not significantly affected whereas there is a 1 \% improvement 
on the sensitivity for the $\mathrm{B^0_s}$ branching fraction. 
The quadratic sum of these uncertainties is 3\% 


\begin{table}
\centering
\caption{Systematic uncertainties for Phase II  scenario.}
\label{tab:b2musys}
\begin{tabular}{cc}\hline
Source & Uncertainty \\ \hline Muon ID efficiency ratio & 1\% \\ 
$\mathrm{B^+}$ normalization yield & 1,4\% \\
Peaking background yield & 10\% \\ 
Semileptonic background yield & 7,5\% \\
fs/fu ratio & 3.5\% \\Effective lifetime & 2\% \\
Trigger efficiency & 1,5\%\\\hline
\end{tabular}\end{table}

The sensitivities of the measurement for the $\mathrm{B^0_s}$ effective lifetime and branching fractions of the rare decays 
$\mathrm{B^0_s}$ and $\mathrm{B^0}$ to dimuons are reported in Table~\ref{tab:b2musensitivity} 
The total relative uncertainties on the branching fractions of the $\mathrm{B^0_s }\to \mu^+\mu^-$ and $\mathrm{B^0} \to \mu^+\mu^-$ include 
both systematic and statistical uncertainties, while the unceratinty on the $\mathrm{B^0_s}$ effective lifetime is statistical only.


\begin{table}[!htb]
\begin{center}
\caption{Estimated analysis sensitivity for different integrated luminosities. 
Columns in the table, from left to right: the total integrated luminosity, the number of reconstructed $\mathrm{B^0_s}$ and $\mathrm{B^0}$  
mesons, the total uncertainties on the bsmm and bdmm branching fractions, the range of the significance of $\mathrm{B^0}$ 
observation(the range indicates the $\pm 1 \sigma$ of the distribution of significance) and the statistical uncertainty on the 
bsm effective lifetime. All results are based on the Phase II  detector configuration except for the first row.} 
\label{tab:b2musensitivity}
\begin{tabular}{l|l|l|l|l|l|l} $\mathcal{L}$ (fb$^{-1}$) & $N(B_s)$ & $N(B^0)$ & $\delta\mathcal{B}(B_s\to\mu\mu)$ & 
$\delta\mathcal{B}(B^0\to\mu\mu)$ & $\sigma(B^0\to\mu\mu)$ & $\delta[\tau(B_s)]$\\ \hline
300 & 205 & 21 & 12\% & 46\% & $1.2-3.3\sigma$ &
 0.15 \\ 3000 & 2048 & 215 & 7\% & 16\% & $6.0-8.3\sigma$ & 0.05 \\
\end{tabular}\end{center}\end{table}


\subsubsection{Measurements of $b\to sll$ from LHCb/ATLAS/CMS}

In order to estimate our sensitivity to BSM effects in $b\to sll$ decays, a number of benchmark NP scenarios are considered (see Table~\ref{tab:penguins:NPscenarios}). 
Scenarios~I and II are inspired by the current discrepancies. The first scenario is the one that best explains the present rare semileptonic decay data. The second scenario best explains the rare semileptonic measurements if a purely left-handed coupling to
quarks and leptons is required for NP. This requirement is theoretically well motivated and arises in models designed to simultaneously explain the discrepancies seen in both tree-level semitauonic and loop-level semileptonic decays. The third and fourth scenarios assume that the current discrepancies are not confirmed but there is instead a small contribution from right-handed currents that would not be visible with the current level of experimental precision. These scenarios will serve to illustrate the power of the large \upgradetwo data set to distinguish between different NP models.
This power relies critically on the ability to exploit multiple related decay channels. 

\begin{table}[!tb]
   \centering
\caption{
    Benchmark NP scenarios. The first scenario is inspired by the present discrepancies in the
    rare decays, including the angular distributions of the decay \decay{\Bz}{\Kstarz\mumu} and the measurements of the branching fraction ratios $R_K$ and $R_{K^*}$. 
    The second
    scenario is inspired by the possibility of explaining the
    rare decays discrepancies and those measured in the observables
    $R(D^{(*)})$. The third and fourth scenarios assume a small 
    right-handed chirality coupling. The Wilson coefficients ($C_i$) are discussed in Sec.~\ref{sec:penguins:framework}} 
            \label{tab:penguins:NPscenarios}
   \begin{tabular}{crrrr}
       \hline
            scenario & $C_9^{\rm NP}$ & $C_{10}^{\rm NP}$ & $C_{9}^{\prime}$ & $C_{10}^{\prime}$ \\
       \hline
       I & $-1.4$ & 0 & 0 & 0 \\
       II & $-0.7$ & 0.7 & 0 & 0\\
       III & 0 & 0& 0.3 & 0.3\\
       IV & 0 & 0& 0.3 & $-0.3$\\
       \hline
   \end{tabular}
\end{table}

The branching fractions of a number of exclusive $\bquark\to\squark \ellell$ processes have been studied using the LHCb Run~1 data set. The experimental measurements are systematically lower than their corresponding SM predictions. The largest discrepancy appears for $\BF(\decay{\Bs}{\phi\mumu})$ which, in the  $1<q^2<6\gevgevcccc$ region, is more than $3\,\sigma$ from the SM predictions~\cite{LHCb-PAPER-2015-023}.
For both the \upgradeone and \upgradetwo datasets,the precision of the measurement of the branching fractions will be limited by the knowledge of the \decay{\B}{\jpsi X} decay modes that are used to normalise the observed signals. 
The knowledge of these branching fractions will be improved by the \belletwo collaboration but will inevitably limit the precision of the absolute branching fractions of rare $\bquark\to\squark \ellell$ processes. 
The comparison between the predicted and measured branching fractions will in any case be limited by the theoretical knowledge of the form factors, even if in the future these are determined parameterically using the data.
A better comparison between theory and experiment can be achieved by studying isospin and \CP asymmetries, which with the \upgradetwo data set will be experimentally probed with percent level precision. The \upgradetwo data set will also enable new decay modes to be studied, for example higher spin-\Kstar states and modes with larger numbers of decay products.  

It is also possible to reduce theoretical and experimental uncertainties by comparing regions of angular phase-space of $\bquark\to\squark \ellell$ decays. 
The angular distribution of \decay{\B}{V \ellell} decays, where $V$ is a vector meson, can be expressed in terms of eight \qsq-dependent angular coefficients that depend on the Wilson coefficients and the form-factors. 
Measurements of angular observables in \decay{\Bz}{\Kstarz\mumu} decays show a
discrepancy with respect to SM predictions~\cite{LHCb-PAPER-2013-037,LHCb-PAPER-2015-051,LHCb-PAPER-2014-006,LHCb-PAPER-2013-019,Aubert:2006vb,Lees:2015ymt,Wei:2009zv,Aaltonen:2011ja,Chatrchyan:2013cda,Khachatryan:2015isa,Aaboud:2018krd,Sirunyan:2017dhj,Beneke:2004dp,Egede:2015kha,Kruger:2005ep,Lyon:2014hpa,Hurth:2013ssa,Beaujean:2013soa,Altmannshofer:2013foa,Descotes-Genon:2013wba}. 
This discrepancy is largest in the so-called form-factor independent observable $P'_{5}$~\cite{LHCb-PAPER-2015-051}. 
The decay \decay{\Bs}{\phi\mumu} can also be described by the same angular formalism as the \decay{\Bz}{\Kstarz\mumu} decay.
However, in this case the \Bs and the \Bsb mesons decay to a common final state and it is not possible to determine the full set of observables   without tagging the initial flavour of the \Bs. 
This is discussed further in Sec.~\ref{sec:penguin:timedependent}. 

With the large data set that will be collected with \upgradetwo,  
corresponding to around 440\,000 fully reconstructed \decay{\Bz}{\Kstarz\mumu} decays, it will
be possible to make a precise determination of the angular observables in narrow
bins of \qsq or using a \qsq-unbinned approach~\cite{Hurth:2017sqw}. 
The expected precision of an unbinned determination of $P_5^{\prime}$ in the SM and in Scenarios~I and II is illustrated in Fig.~\ref{fig:penguin:P5primeUnbinned}. 
\upgradetwo will enable these scenarios to be clearly separated from the SM and from each other.
By combining information from all of the angular observables in the decay, it will also be possible to distinguish models with much smaller NP contributions. 
Figure~\ref{fig:penguin:WCKstmm} shows the expected $3\,\sigma$ sensitivity for the Wilson
coefficients for $C_{9,10}^{\prime}$ for the SM, Scenario~III and Scenario~IV.  
These scenarios are also clearly distinguishable with the precision that will be available with the \upgradetwo data set.

\begin{figure}[!tb]
\centering
\includegraphics[width=0.45\linewidth]{section7/figures/P5p_Run3.pdf}
\includegraphics[width=0.45\linewidth]{section7/figures/P5p_HL.pdf}
\caption{Experimental sensitivity to the $P_5^{\prime}$ angular observable in the SM,
  Scenarios~I and II for (left) the Run~3 and (right) the \upgradetwo data sets. 
  The sensitivity is computed assuming that the charm-loop contribution is determined from the data.
}
\label{fig:penguin:P5primeUnbinned}
\end{figure}

\begin{figure}[!tb]
\centering
\includegraphics[width=0.4\linewidth]{section7/figures/ellipses_Cp_Run3.pdf}
\includegraphics[width=0.4\linewidth]{section7/figures/ellipses_Cp_HL.pdf}
\caption{Expected sensitivity for the Wilson coefficients $C_9^{\prime}$ and
  $C_{10}^{\prime}$ from the analysis of the decay \decay{\Bz}{\Kstarz\mumu}. 
  The ellipses correspond to 3\,$\sigma$ contours for
  the SM, Scenario~III and Scenario~IV for (left) the Run~3 and (right) the \upgradetwo data sets. 
  }
  \label{fig:penguin:WCKstmm}
\end{figure}

The major challenge for \decay{\B}{V \ellell} decays is to disentangle NP effects from
SM contributions. 
With a large data set it will be possible to probe the SM contributions, under the premise that a genuine NP contribution is expected to have no $q^2$ dependence, while \eg a 
charm loop contribution is expected to grow approaching the pole of
the charmonia resonances.  
A measurement using Breit-Wigner functions to parametrise the
resonances, and their interference with the
short-distance contributions to the decay, is proposed in Ref.~\cite{Blake:2017fyh}. 
A similar technique has already been applied to the Run~1 data for the \decay{\Bp}{\Kp\mumu} decay~\cite{LHCb-PAPER-2016-045}. 
An alternative approach using additional phenomenological inputs has also been proposed~\cite{Bobeth:2017vxj}.
A precise knowledge of the charm loop contribution and a parametric determination of the form factors,
 will come from a combination of phenomenological and experimental
methods and will allow $C_9$ and
$C_{10}$ to be determined with great precision in $\bquark\to\squark\mumu$ transitions.

%Sandra: including CMS P5'
\subsubsubsection{CMS P5' sensitivity}
The expected $\mathrm{B^0}$ signal yield per each $q^2$ bin is measured on the Phase II MC sample.
An extended unbinned maximum likelihood fit is performed to the $K^+\pi^-\mu^+\mu^-$ invariant mass in each bin, parametrizing the signal with 
the sum of two Gaussian distributions and the background with an exponential distribution. 
 The expected yield of fully reconstructed $\mathrm{B^0}\to \mathrm{K^{*0}} \mu^+\mu^-$ signal events, excluding the $q^2$ range 
 overlapping with the resonant decays, is around 700~k.
 The projected statistical uncertainty on the $\mathrm{P'_5}$  parameter measurement is obtained scaling the statistical uncertainty measured in Run I 
 according to the square root of the ratio between the corresponding signal yields.
 The evolution of the systematic uncertainties is estimated by rescaling 
 the values evaluated in the Run I analysis 
 \cite{Sirunyan:2017dhj}.
Uncertainties concerning the simulation mismodeling, the fit bias introduced by the fitting procedure, the efficiency determination,
 the $K\pi$ mistagging rate, the choice of the signal mass pdf, the effect of the contamination from feed-through events from the resonant decays,
 and the angular resolution are assumed to be reduced by a factor of 2 in the Phase II scenario; 
 the uncertainty on the description of the background mass distribution and the one associated with the propagation of the uncertainty 
 on $F_L$, $F_S$ and $A_S$ depend on the available amount of data. These uncertainties are therefore scaled by the ratio of the square root 
 of the number of events.The uncertainty due to the finite size of the simulation is considered to be zero in the Phase II scenario.
\par\noindent
The projections for the measurement of $\mathrm{P'_5}$ versus $q^2$ are shown in Fig.~\ref{fig:CMSP5presults}, together with the measurements from the Run I analysis.
 \begin{figure}[htbp]
  \begin{center}
  \includegraphics[width=0.6\textwidth]{section7/figures/R1_and_errorP2_rebinningSystButStatOnly.pdf}
 \caption{Projection of the CMS sensitivity on the $\mathrm{P'_5}$ parameter versus $q^2$ in the Phase II scenario at 3000 fb$^{-1}$, shown with the black shaded region. The red shaded region correspond to the projected statistical uncertainty. The measurement of $\mathrm{P'_5}$ from the CMS Run I analysis is reported in pale-blue: the statistical uncertainties are shown by the  vertical bars, while the outer vertical bars represent the total uncertainties. 
 The vertical shaded regions correspond to the J/$\psi$ and $\psi'$ resonances.The lower pads represent the total (black) and statistical (red) 
uncertainties for an optimized finer $q^2$ binning.} 
 \label{fig:CMSP5presults}
 \end{center}
 \end{figure}
The projections for the Statistical-only scenario are also included.
\par\noindent    
The increased amount of collected data foreseen for the Phase II integrated luminosity offers the opportunity to perform the angular analysis in
 narrower $q^2$ bins, in order to measure the $\mathrm{P'_5}$  shape as a function of $q^2$ with finer granularity. The $q^2$ region below the J/$\psi$ 
  mass(squared), which is more sensitive to possible new physics effects, is considered. Each Run I $q^2$ bin is split into smaller 
 and equal-size bins trying to achieve a statistical uncertainty of the order of the total systematic uncertainty in the same bin with the
 additional constraint of having a bin width at least 5 times larger than the dimuon mass resolution $\sigma_r$.
 If both conditions cannot be satisfied, then only the looser requirement on the 5 $\sigma_r$ bin width is imposed.
 The dimuon mass resolution is measured on the MC simulation as a function of $q^2$.Each systematic uncertainty in the Run I analysis bin is 
 rescaled to the finer bins according to its classification and correlation between $q^2$ bins.
\par\noindent
The corresponding total and statistical only uncertainties 
 on $\mathrm{P'_5}$  are shown in the two lower pads of Fig.~\ref{fig:CMSP5presults}.

\subsubsection{Measurements of $b\to dll$}

The \upgradetwo data set will provide a unique opportunity to make precise measurements of $\bquark\to\dquark \ellell$ processes. 
Using the Run~1 and 2 data sets, LHCb data have been used to observe the decays \decay{\Bp}{\pip\mumu}~\cite{LHCb-PAPER-2012-020,LHCb-PAPER-2015-035} and \decay{\Lb}{p\pim\mumu}~\cite{LHCb-PAPER-2016-049} and find evidence for the decays \decay{\Bz}{\pip\pim\mumu} (in a $\pip\pim$ mass region that  is expected to be dominated by \decay{\Bz}{\rhoz\mumu}) and \decay{\Bs}{\Kstarzb\mumu}~\cite{LHCb-PAPER-2018-004} with branching fractions at the $\mathcal{O}(10^{-8})$ level.
The existing data samples comprise $\mathcal{O}(10)$ decays in these decay modes. 
The upgrade will provide samples of thousands, or tens of thousands of such decays.
The ability to measure the properties of these processes depends heavily on the PID performance of the LHCb subdetectors. 
In the case of the \decay{\Bs}{\Kstarzb\mumu} decay, excellent mass resolution is also critical to separate \Bs and \Bz decays. 

The ratio of branching fractions between the CKM-suppressed $b \to d\ellell$ transitions and their CKM-favoured  $\bquark \to \squark \ellell$ counterparts, together with theoretical input on the ratio of the relevant form factors, enables the ratio of CKM elements  $|V_{td}|/|V_{ts}|$ to be determined. The precision on $|V_{td}|/|V_{ts}|$ from such decays is presently dominated by the statistical uncertainty on the experimental measurements of \decay{\Bp}{\pip\mumu}, and is much less precise than the determination from mixing measurements. The theoretical uncertainty is at the level of 4\% and is expected to improve with further progress on the form-factors from lattice QCD. Around 17\,000 \decay{\Bp}{\pip\mumu} decays are expected in the full 300\invfb dataset, allowing an experimental precision better than 2\%.

The current set of measurements of $\bquark\to\squark \ellell$ processes have demonstrated the importance of angular measurements in the precision determination of Wilson coefficients. With the \upgradetwo dataset, where a sample of 4300  \decay{\Bs}{\Kstarzb\mumu} decays is expected,  it will be possible to make a full angular analysis of a $\bquark\to\dquark \ellell$ transition. The \decay{\Bs}{\Kstarzb\mumu} decay is both self-tagging and has a final state involving only charged particles. The \upgradetwo data set will allow the angular observables in this decay to be measured with better precision than the existing measurements of the \decay{\Bz}{\Kstarz\mumu} angular distribution. 

The \upgradetwo dataset will also give substantial numbers of \decay{\B^{0,+}}{\rho^{0,+}\mumu}  and \decay{\Lb}{N\mumu} decays. Although the  \decay{\Bz}{\rhoz\mumu} decay does not give the flavour of the initial \B meson, untagged measurements will give sensitivity to a subset of the interesting angular observables. 
Analysis of the \decay{\Lb}{N^{*}\mumu} decay will require statistical separation of  overlapping $p\pim$ resonances with different $J^P$ by performing an amplitude analysis of the final-state particles. 

The combination of information from $\BF(\decay{\Bz}{\mumu})$, the differential branching fraction of the \decay{\Bp}{\pip\mumu} decay, and angular measurements, notably of \decay{\Bs}{\Kstarzb\mumu}, will indicate whether NP effects are present in $\bquark\to\dquark$ transitions at the level of 20\% of the SM amplitude with more than $5\sigma$ significance. 

\subsubsection{LFU tests in $b\to (s,d)ll$}

The Run~1 \lhcb data have been used to perform the most precise measurements of $R_{K}$ and $R_{\Kstar}$ to-date~\cite{LHCb-PAPER-2017-013,LHCb-PAPER-2014-024} (see Fig.~\ref{fig:penguin:RXscenarios}).
These measurements are compatible with the SM at the level of 2.1--2.6 standard deviations. Assuming the current detector performance, approximately 46\,000 \decay{\Bp}{\Kp\epem} and 20\,000 \decay{\Bz}{\Kstarz\epem} candidates are expected in the range $1.1 < \qsq < 6.0\gevgevcccc$ in the \upgradetwo data set.
The ultimate precision on $R_{K}$ and $R_{\Kstar}$ will be better than 1\%.
The importance of the \upgradetwo data set in distinguishing between different NP scenarios is highlighted in Fig.~\ref{fig:penguin:RXscenarios}. 
With this data set all four NP scenarios could be distinguished at more than 5\,$\sigma$ significance.

The \upgradetwo data set will also enable the measurement of other $R_{X}$ ratios \eg $R_{\phi}, R_{p\kaon}$ and the ratios in CKM suppressed decays. 
For example, with 300\invfb, it will be possible to determine $R_{\pi}=\BR(\decay{\Bp}{\pip\mumu})/\BR(\decay{\Bp}{\pip\epem})$ with a few percent statistical precision.
A summary of the expected performance for a number of different $R_X$ ratios is indicated in Table~\ref{tab:penguin:LU_extrapolations}.

\begin{figure}[!t] 
\centering
\includegraphics[width=0.75\textwidth]{section7/figures/RX_scenario.pdf}
\caption{
Projected sensitivity for the $R_{K}$, $R_{K^*}$ and $R_{\phi}$ measurements in different NP scenarios with the Upgrade~II data set. 
The existing Run~1 measurements of $R_{K}$ and $R_{K^*}$ are shown for comparison. 
}
\label{fig:penguin:RXscenarios}
\end{figure}

\begin{table}[!tb]
\caption{
Estimated yields of $\bquark\to\squark\ep\en$ and $\bquark\to\dquark\ep\en$ processes and the statistical uncertainty on $R_{X}$ in the range $1.1<\qsq<6.0\gevgevcccc$ extrapolated from the Run~1 data. 
A linear dependence of the \bbbar production cross section on the $pp$ centre-of-mass energy and unchanged Run~1 detector performance are assumed. 
Where modes have yet to be observed, a scaled estimate from the corresponding muon mode is used.
}
\label{tab:penguin:LU_extrapolations}
\centering
\begin{tabular}{lrrrrr}
\hline
Yield  & Run~1 result & 9\invfb & 23\invfb & 50\invfb & 300\invfb \\
\hline
 \decay{\Bu}{\Kp\epem} 		& $254 \pm 29$\cite{LHCb-PAPER-2014-024} & 1\,120 & 3\,300 & 7\,500 & 46\,000 \\
 \decay{\Bd}{\Kstarz\epem}	& $111 \pm 14$\cite{LHCb-PAPER-2017-013} & 490 & 1\,400 & 3\,300 & 20\,000 \\
 \decay{\Bs}{\phi\epem}		& -- & 80 & 230 & 530 & 3\,300 \\
 \decay{\Lb}{\proton\kaon\epem} & -- & 120 & 360 & 820 & 5\,000 \\
\decay{\Bp}{\pip\epem}		& -- & 20 & 70 & 150 & 900 \\ 
\hline
$R_X$ precision &  Run~1 result & 9\invfb & 23\invfb & 50\invfb & 300\invfb \\
\hline
$R_{\kaon}$					& $0.745 \pm 0.090\pm 0.036$\cite{LHCb-PAPER-2014-024} & 0.043 & 0.025 & 0.017 & 0.007 \\
$R_{\Kstarz}$					& $0.69 \pm 0.11\pm 0.05$\cite{LHCb-PAPER-2017-013} & 0.052 & 0.031 & 0.020 & 0.008 \\
$R_{\phi}$					& -- & 0.130 & 0.076 & 0.050 & 0.020 \\
$R_{\proton\kaon}$  & -- & 0.105 & 0.061 & 0.041 & 0.016 \\
$R_{\pi}$						& -- & 0.302 & 0.176 & 0.117 & 0.047 \\ 
\hline
\end{tabular}
\end{table}

In addition to improvements in the $R_{X}$ measurements, the enlarged \upgradetwo data set will give access to new observables. For example, the data will allow precise comparisons of the angular distribution of dielectron and dimuon final-states. 
Differences between angular observables in \decay{\B}{X\mumu} and \decay{\B}{X\ep\en} decays are theoretically pristine~\cite{Capdevila:2016ivx}
and are sensitive to different combinations of Wilson coefficients compared to the $R_X$ measurements. 
Figure~\ref{fig:penguin:DeltaC} shows that an upgraded \lhcb detector will enable such decays to be used to discriminate between different NP models, for example separating between Scenarios~I and II. 
Excellent NP sensitivity can be achieved irrespective of the assumptions made about the hadronic contributions to the decays.

\begin{figure}[t!]
\centering
\includegraphics[width=0.49\textwidth]{section7/figures/ellipses_DeltaC_Run3.pdf}
\includegraphics[width=0.49\textwidth]{section7/figures/ellipses_DeltaC_HL.pdf}
\caption{
Constraints on the difference in the $C_9$ and $C_{10}$ Wilson coefficients from electron and muon modes with (left) Run~3 and (right) \upgradetwo data sets. The $3\sigma$ regions are shown for the SM (blue), for a vector-axial-vector new physics contribution (red) and for a purely vector new physics contribution (green).}
\label{fig:penguin:DeltaC}
\end{figure}

In the existing \lhcb detector, electron modes have an approximately factor five lower efficiency than the corresponding muon modes, owing to the tendency for the electrons to lose a significant fraction of their energy through bremsstrahlung in the detector. 
This loss impacts on the ability to reconstruct, trigger and select the electron modes.
The precision with which observables can be extracted therefore depends primarily on the electron modes and not the muon modes.
In order for $R_{X}$ measurements to benefit from the large \upgradetwo data samples, it will be necessary to reduce systematic uncertainties to the percent level.
These uncertainties can be controlled by taking a double ratio between $R_{X}$ and the decays \decay{\B}{\jpsi X}, where the \jpsi decays to \mumu and $\ep\en$.
This approach is expected to work well, even with very large data sets.  

Other sources of systematic uncertainty can be mitigated through design choices for the upgraded detector.
The recovery of bremsstrahlung photons is inhibited by the ability to find the relevant photons in the ECAL (over significant backgrounds) and by the energy resolution.
A reduced amount of material before the magnet would reduce the amount of bremsstrahlung and hence would increase the electron reconstruction efficiency and improve the electron momentum resolution. 
Higher transverse granularity would aid signal selection and help reduce the backgrounds. 
With a large number of primary $pp$ collisions, the combinatorial background will increase and will need to be controlled with the use of timing information.
However, the Run~1 data set indicates that it may be possible to tolerate a significant (\ie\ larger than a factor two) increase in combinatorial backgrounds without destroying the signal selection ability.

\subsubsection{Time dependent angular analyses in $b\to (s,d)ll$}

Time dependent analyses of rare decays into \CP-eigenstates can deliver orthogonal experimental information to time-integrated observables. 
So far, no time-dependent measurement of the $\decay{\Bs}{\phi\mumu}$ decay has been performed due to the limited signal yield of $432\pm 24$ in the Run~1 data sample~\cite{LHCb-PAPER-2015-023}.
However, the larger data samples available in \upgradetwo will enable time-dependent studies. 
The framework describing $\Bbar$ and $\B\to V\ellell$ transitions to a common final-state is discussed in  Ref.~\cite{Descotes-Genon:2015hea}, 
where several observables are discussed that can be accessed with and without flavour tagging.  
Two observables called $s_8$ and $s_9$, which are only accessible through a time-dependent flavour-tagged analysis, are of particular interest. 
These observables  are proportional to the mixing term $\sin\left(\Delta m_s t\right)$ and provide information that is not available through flavour specific decays.
Assuming a time resolution of around $45\fs$ and an effective tagging power of $5\%$ 
results in an effective signal yield of  $2000$ decays for the \upgradetwo data set. 

As a first step towards a full time-dependent analysis, the effective lifetime of the decay $\decay{\Bs}{\phi\mumu}$ can be studied. 
The untagged time-dependent decay rate is given by
\begin{align}
  \frac{\deriv\Gamma}{\deriv t} &\propto  e^{-\Gamma_s}\left[\cosh\left(\frac{\Delta\Gamma_st}{2}\right)+A^{\Delta\Gamma}\sinh\left(\frac{\Delta\Gamma_st}{2}\right)\right].
\end{align}
The observable $A^{\Delta\Gamma}$ can be related to the angular observables $F_{\rm L}$ and $S_3$ via $A^{\Delta\Gamma}=2S_3-F_{\rm L}$. 
Due to the significant lifetime difference $\Delta\Gamma_s$ in the \Bs\ system, even an untagged analysis can probe right-handed currents. 
For the combined low- and high-\qsq regions, preliminary studies suggest a statistical sensitivity to $A^{\Delta\Gamma}$ of $0.05$ can be achieved with a $300\invfb$ data set. 

With the \upgradetwo data set it will also be possible to perform a time-dependent angular analysis of the $\bquark \to \dquark$ process \decay{\Bd}{\rho^0\mumu}. 
This process differs from \decay{\Bs}{\phi\mumu} in two important regards: it is CKM suppressed and therefore has a smaller SM branching fraction; and $\Delta \Gamma_d \approx 0$, removing sensitivity to $A^{\Delta\Gamma}$. 
The uncertainties on the angular observables are expected to be on the order of $0.1$ for this case. 

The time-dependent angular analyses will still be statistically limited even with $300\invfb$. 
It will be important to maintain good decay-time resolution and the performance of the particle identification will be crucial to control backgrounds, as well as to improve flavour tagging performance. 

\subsubsection{Measurements of $b\to s\gamma$}

The time dependent \CP asymmetry of \decay{\B}{f_{\CP}\gamma} arises from the interference between decay amplitudes with and without $\Bds-\Bdsb$ mixing and is predicted to be small in the SM\ \cite{Atwood:1997zr,Ball:2006cva,Matsumori:2005ax}.
As a consequence, a large asymmetry due to interference between the \B mixing and decay diagrams can only be present if the two photon helicities contribute to both \B and \Bb decays.
From the time dependent decay rate
\begin{align}
    \Gamma(\decay{\Bds(\Bdsb)}{f_{\CP}\gamma})(t) \sim e^{-\Gamma_s t}
    &\big[
        \cosh\left(\frac{\Delta\Gamma_{(s)}}{2}\right)
        - \mathcal{A}^{\Delta}\sinh\left(\frac{\Delta\Gamma_{(s)}}{2}\right) \pm \nonumber\\
        & \pm \mathcal{C}_\CP\cos\left(\Delta m_{(s)} t\right)
        \mp \mathcal{S}_\CP\sin\left(\Delta m_{(s)} t\right)
    \big],
\end{align}
where $A^\Delta$, $\mathcal{C}_\CP$ and $\mathcal{S}_\CP$ depend on the photon polarisation~\cite{Muheim:2008vu}.
Two strategies can be devised:
one studying the decay rate independently of the flavour of the \B meson, which allows $A^\Delta$ to be accessed, and one tagging the flavour of the \B meson, which accesses $\mathcal{S}_\CP$ and $\mathcal{C}_\CP$.
The first strategy has been exploited at \lhcb to study the $4000$ \decay{\Bs}{\phi\gamma} candidates collected in Run 1 to obtain $A^\Delta=-0.98^{+0.46}_{-0.52}\stat ^{+0.23}_{-0.20}\syst$~\cite{LHCb-PAPER-2016-034}, compatible at two standard deviations from the prediction of $A^\Delta_{\text{SM}}=0.047^{+0.029}_{-0.025}$.

With $\sim60$k signal candidates expected with $50\,\invfb$, the full analysis, including flavour tagging information, will improve the statistical uncertainty on $A^\Delta$ to $\sim0.07$, and will need a careful control of the systematic 
uncertainties.
The analysis performed with $\sim800$k signal decays expected with $300\,\invfb$, with a statistical uncertainty to $\sim0.02$, requires some of the possible improvements in \piz reconstruction of the \upgradetwo detector 
to be able to use the full statistical power of the data.

In addition to studying the \Bs system, \lhcb can study the time-dependent decay rate of \decay{\Bz}{\KS\pip\pim\gamma} decays, which permits access of the photon polarisation through the $\mathcal{S}_\CP$ term.
With $\mathcal{O}(1000)$ signal events in Run 1, around $35$k and $200$k are expected at the end of \upgradeone and \upgradetwo, respectively
($1.75$k and $10$k when considering the flavour tagging efficiency), opening the doors to a very competitive measurement of $\mathcal{S}_\CP$ in the \Bz system.

Another way to study the photon polarisation is through the angular correlations among the three-body decay products of a kaonic resonance in \decay{\B}{K_{\mathrm{res}}(\to K\pi\pi)\gamma},
which allows the direct measurement of the photon polarisation parameter in the effective radiative weak Hamiltonian~\cite{Gronau:2002rz}. 
As a first step towards the photon polarisation measurement, \lhcb observed nonzero photon polarisation for the first time by studying the photon angular distribution in bins of $\Kp\pim\pip$ invariant mass\ \cite{LHCb-PAPER-2014-001}, but the determination of the value of this polarisation could not be performed due to the lack of knowledge of the hadronic system.
To overcome this problem, a method to measure the photon polarisation using a full amplitude analysis of \decay{\B}{K\pi\pi\gamma} decays is currently under development~\cite{Bellee:2018}, with an expected statistical sensitivity on the photon polarisation parameter of $\sim5\%$ in the charged mode with the Run 1 dataset.
The extrapolation of the precision to $300\,\invfb$  results in a statistical precision better than $1\%$, and hence control of the systematic uncertainties will be crucial.

The polarisation of the photon emitted in $b\to s\gamma$ transitions can also be accessed via semileptonic $b\to s\ell\ell$ transitions, for example in the decay $B^0\to\Kstarz\ell^+\ell^-$. Indeed, as mentioned in 
previous sections, at very low $q^2$ these decays are dominated by the electromagnetic dipole operator ${\cal O}_7^{(\prime)}$. Namely, the longitudinal polarisation fraction ($\FL$) is expected to be below $20\%$ for $q^2<0.2\gevgevcccc$.
In this $q^2$ region, the angle $\phi$ between the planes defined by the dilepton system and the $\Kstarz\to K^+\pi^-$ decay is sensitive to the $b\to s\gamma$ photon helicity.

While the $\Kstarz\mumu$ final state is experimentally easier to select and measure at LHCb, the $\Kstarz\epem$ final state allows $q^2$ values below $4m_\mu^2$ to be probed, 
where the sensitivity to the photon helicity is maximal.
Compared to the radiative channels used for polarisation measurements, the $B^0\to\Kstarz\epem$ final state is fully charged and gives better mass resolution and therefore better separation from partially reconstructed backgrounds.

The sensitivity of this decay channel at LHCb was demonstrated by an angular analysis performed with Run 1 data~\cite{LHCb-PAPER-2014-066}. The angular observables most sensitive to the photon polarisation at low $q^2$ are $A_{\rm T}^{(2)}$ and $A_{\rm T}^{\rm Im}$, as defined in Ref.~\cite{LHCb-PAPER-2014-066}. 
Indeed, in the limit $q^2\to 0$, these observables can be expressed by the following functions of $C_{7,7^{\prime}}$ (assuming NP contributions to be much smaller than $|C_7^{\rm SM}|$):
\begin{equation}
  \label{eq:qSqToZero} 
  A_{\rm T}^{(2)} (\qsq \to 0) \simeq 2 \frac{\Real (C_7^{' \ast}) } {| C_7 |}  \quad {\rm and} \quad  A_{\rm T}^{\rm Im} (\qsq \to 0) \simeq 2 \frac{\Imag (C_7^{' \ast}) } {| C_7 |}.
\end{equation}
In order to maximise the sensitivity to the photon polarisation, 
the angular analysis should be performed as close as possible 
to the low $q^2$ endpoint. However, the events at extremely low $q^2$ have worse $\phi$ resolution (because the two electrons are almost collinear) and are polluted by $B^0\to\Kstarz\gamma$ decays with the $\gamma$ converting in the VELO material. In the Run 1 analysis~\cite{LHCb-PAPER-2014-066} the minimum required $m(\epem)$ was set at 20\mevcc, but this should
be reduced as the \upgradetwo VELO detector will have a significantly lower material budget (multiple scattering is the main effect worsening the $\phi$ resolution). 
Similarly, the background from $\gamma$ conversions will be reduced with a lighter RF-foil or with the complete removal of it in \upgradetwo~\cite{LHCb-PII-EoI}.

Using the signal yield as given in Table~\ref{tab:penguin:LU_extrapolations} leads to the following statistical sensitivities to $A_{\rm T}^{(2)}$ and $A_{\rm T}^{\rm Im}$: 12\% with 8\invfb, 7\% with 23\invfb and 2\% with 300\invfb.
The theoretical uncertainty induced when this observable is translated into a photon polarisation measurement is currently at the level of $2\%$ but should improve by the time of the \upgradetwo analyses.
The current measurements performed with Run 1 data have a systematic uncertainty of order $5\%$ coming mainly from the modelling of the angular acceptance and from the uncertainty on the angular shape of the combinatorial background. The acceptance is independent of $\phi$ at low $q^2$ and its
modelling can be improved with larger simulation samples and using the proxy channel $\Bd\to\Kstarz\jpsi(\to\epem)$. 

Weak radiative decays of \bquark baryons are largely unexplored, with the best limits coming from CDF: $\mathcal{B}(\decay{\Lb}{\Lz\gamma}) < 1.3\times10^{-3}$ at $90\%\,$CL~\cite{Acosta:2002fh}.
They offer a unique sensitivity to the photon polarisation through the study of their angular distributions, and will constitute one of the main topics in the radiative decays programme in the \lhcb \upgradetwo.

With predicted branching fractions of $O(10^{-5}-10^{-6})$, the first challenge for \lhcb will be their observation, as the production of long-lived particles in their decay, in addition to the photon, means in most cases that the $b$-baryon secondary vertex cannot be reconstructed. This makes their separation from background considerably more difficult than in the case of regular radiative \bquark decays.

The most abundant of these decays is \decay{\Lb}{\Lz(\to p\pim)\gamma}, which is sensitive to the photon polarisation mainly\footnote{In the following, we assume that the \Lb (and any other beauty baryon) polarisation is zero~\cite{LHCb-PAPER-2012-057}, removing part of the photon polarisation dependence.} through the distribution of the angle between the proton and the \Lz momentum in the rest frame of the \Lz ($\theta_p$), 
\begin{equation}
    \frac{\operatorname{d}\Gamma}{\operatorname{d}\cos\theta_p} \propto 1 - \alpha_\gamma \alpha_{p,1/2}\cos\theta_p, 
\end{equation}
where $\alpha_\gamma$ is the asymmetry between left- and right-handed amplitudes and $\alpha_{p,1/2}=0.642\pm0.013$~\cite{PDG2018} is the \decay{\Lz}{p\pim} decay parameter.
Using specialised trigger lines for this mode $15-150$ signal events are expected using the Run 2 dataset. Preliminary studies show that a statistical sensitivity to $\alpha_\gamma$ of $(20-25)\%$ is expected with these data, which would be reduced to $\sim15\%$ with $23\,\invfb$ and below $4\%$ with $300\,\invfb$.
In the \lhcb \upgradetwo, the addition of timing information in the calorimeter will be important to be able to study this combinatorial-background dominated decay;
additionally, improved downstream reconstruction would allow the use of downstream \Lz decays, which make up more than $2/3$ of the total signal.

The \decay{\Xib}{\Xi^-(\to\Lz(\to p\pim)\pim)\gamma} decay presents a richer angular distribution, with dependence to the photon polarisation in both the \Lz angle ($\theta_\Lz$) and proton angle ($\theta_p$),
\begin{equation}
    \frac{\operatorname{d}\Gamma}{\operatorname{d}\cos\theta_\Lz\cos\theta_p} \propto 1 - \alpha_\gamma \alpha_\Xi \cos\theta_\Lz + \alpha_{p,1/2}\cos\theta_p\left(\alpha_\Xi-\alpha_\gamma\cos\theta_\Lz\right),
\end{equation}
but the lower $\sigma(\decay{pp}{\Xib})$, combined with a lower reconstruction efficiency due to the presence of one extra track, results in an order of magnitude fewer events than in the \Lb case, making the increase of 
statistics from the \upgradetwo even more relevant. With a similar sensitivity to the photon polarisation to that of \decay{\Lb}{\Lz(\to p\pim)\gamma}, \decay{\Xib}{\Xi^-\gamma} decays will allow this parameter to be probed with a precision of $40\%$ and $10\%$ with $23$ and $300\,\invfb$, respectively.

\subsubsection{Measurements of $b\to cl\nu$ including $B_c$ and $b$-baryon prospects}

\begin{figure}[!tb]
\centering
\includegraphics[width=0.7\linewidth]{section7/figures/RX_projection_stat_syst.pdf}
\caption{
The projected absolute uncertainties on $\mathcal{R}(D^{\ast}$ and $\mathcal{R}(J/\psi)$ (see Sect.~\ref{subsec:OtherBhadronRX}) from the current sensitivities (at 3\invfb) to 23\invfb, 50\invfb, and 300\invfb.
}
\label{fig:RXproj}
\end{figure} 

LHCb has made measurements of \RDb using both muonic ($\tau^{+} \to \mup \nu \nu$) and hadronic ($\tau^{+} \to \pip \pim \pip \nu$) decays of the tau lepton~\cite{LHCb-PAPER-2015-025,LHCb-PAPER-2017-017,LHCb-PAPER-2017-027}.
Due to the presence of multiple neutrinos these decays are extremely challenging to measure.
The measurements rely on isolation techniques to suppress partially reconstructed backgrounds, 
\B meson flight information to constrain the kinematics of the unreconstructed neutrinos, and a multidimensional template fit to determine the signal yield. 
Figure~\ref{fig:RXproj} shows how the absolute uncertainties on the LHCb muonic and hadronic $\mathcal{R}(D^{\ast})$ measurements are projected to evolve with respect to the current status.
The major uncertainties are the statistical uncertainty from the fit, the uncertainties on the background modelling and the limited size of simulated samples.
A major effort is already underway to commission fast simulation tools.
The background modelling is driven by a strategy of dedicated control samples in the data, and so this uncertainty will continue to improve with larger data samples.
From Run 3 onward it is assumed that, taking advantage of the full software HLT, the hadronic analysis can normalise directly to the $B^0 \to D^{\ast -}\mu^+\nu_{\mu}$  decay,
thus eliminating the uncertainty from external measurements of $\mathcal{BR}(B^0 \to D^{\ast -} \pi^+\pi^-\pi^+)$.
It is assumed that all other sources of systematic uncertainty will scale as $\sqrt{\mathcal{L}}$.
With these assumptions, an absolute uncertainty on $\mathcal{R}(D^{\ast}$ of $0.003$ will be achievable for the muonic and hadronic modes with the 300\invfb Upgrade II dataset.

On the timescale of \upgradetwo, interest will shift toward new observables  beyond the branching fraction ratio~\cite{Becirevic:2016hea}.
The kinematics of the \dsttaunu decays is fully described by the dilepton mass, and three angles which are denoted $\chi$, $\theta_L$ and $\theta_D$. 
LHCb is capable of resolving these three angles, as can be seen in~\figref{fig:RDAngles}.
However, the broad resolutions demand very large samples to extract the underlying physics. The decay distributions within this kinematic space are governed by the underlying spin structure, and precise measurements of these distributions will allow the different helicity amplitudes to be disentangled.
This can be used both to constrain the spin structure of any potential new physics contribution, and to measure the hadronic parameters
governing the \dsttaunu decay, serving as an essential baseline for SM and non-SM studies.
The helicity-suppressed amplitude which presently dominates the theoretical uncertainty on \RDb is too strongly suppressed in the 
\dbmunu decays to be measurable, however this can be accessed in the \dbtaunu decay directly.
If any potential new physics contributions are assumed not to contribute via the helicity-suppressed amplitude then the combined measurements of \dbmunu and \dbtaunu decays will allow for a fully data-driven prediction for
\RDb under the assumption of lepton universality, eliminating the need for any theory input relating to hadronic form factors.
However, these measurements have yet to be demonstrated with existing data.
This exciting programme of differential measurements needs to be developed on Run~1 and 2 data before any statement is made about the precise sensitivity, 
but it offers unparalleled potential to fully characterise both the SM and non-SM contributions to the $b \to c \tau \nu$ transition.

\begin{figure}[!tb]
\centering
\includegraphics[width=0.45\linewidth]{section7/figures/thetaD.pdf}
\includegraphics[width=0.45\linewidth]{section7/figures/thetaL.pdf}
\includegraphics[width=0.45\linewidth]{section7/figures/thetaC.pdf}
\caption{
Angular resolution for simulated \dstmunu (black) and \dsttaunu  (red) decays, with $\tau^{+} \to \mup \nu \nu$. 
This demonstrates our ability to resolve the full angular distribution, with some level of statistical dilution.
}
\label{fig:RDAngles}
\end{figure} 

As measurements in \RDst become more statistically precise, it will become increasingly more 
urgent to provide supplementary measurements in other \bquark-hadron species with different background 
structure and different sources of systematic uncertainties. For example, the \decay{\Bsb}{\Dsp \taum\neub} and \decay{\Bsb}{\Dssp\taum\neub} decays
will allow supplementary 
measurements at high yields, and do not suffer as badly from cross-feed backgrounds from other 
mesons, unlike, for example, \decay{\Bzb }{ \Dstarp \taum\neub}, where the \Bp and \Bs both contribute to 
the $\Dstarp \mu X$ or $\Dstarp \threepim X$ final states. 
Furthermore, the comparison of decays with different spins of the $b$ and $c$ hadrons can enhance our sensitivity 
to different NP scenarios~\cite{Azatov:2018knx,LHCb-PAPER-2017-016}.
No published measurements exist for the $\Bs$ case yet, but based on known relative efficiencies and assuming 
the statistical power of this mode tracks \RDorDst, we expect less than $6\%$ relative uncertainty 
after Run~3, and $2.5\%$ with the \upgradetwo data, where limiting systematic uncertainties 
are currently expected to arise from corrections to 
simulated pointing and vertex resolutions, from knowledge of particle identification efficiencies, 
and from knowledge of the backgrounds from random combinations of charm and muons. It is conceivable 
that new techniques and control samples could further increase the precision of these measurements. 

Methods 
are currently under development for separating the \decay{\Bsb}{\Dssp\ell^{-}\neub} and \decay{\Bsb}{\Dsp\ell^{-}\neub} modes, and given the
relative slow pion (\decay{\Dstarp }{ \Dz\pip}) and soft photon (\decay{\Dssp }{ \Dsp\gamma}) efficiencies, the precision in \decay{\Bsb}{ \Dsp \tau \nu} decays can be expected to exceed that in
\decay{\Bsb }{ \Dssp \tau \nu}, the reverse of the situation for $\RDorDst$.
An upgraded ECAL would extend the breadth and sensitivity of $\mathcal{R}(D^{*(*)+}_{s})$ measurements possible
in the \upgradetwo\ scenario above and beyond the possible benefits of improved neutral isolation in 
$\RD$ or $\mathcal{R}(\Dsp)$ measurements.

Of particular interest are the semitauonic decays of
\bquark baryons and of \Bcp mesons.
The former provides probes of entirely new Lorentz structures of NP operators
which pseudoscalar to pseudoscalar or vector transitions simply do not access.
The value of probing this supplementary space of couplings has already been demonstrated by
LHCb with its Run~1
measurement of $|\Vub|$ via the decay \decay{\Lb}{\proton\mun\neub}, which places strong constraints on 
right-handed currents sometimes invoked to explain the inclusive-exclusive tensions in that quantity. By the end of Run~3, it is expected that the relative uncertainty for $\mathcal{R}(\Lc)$ will reach below $4\%$, and $2.5\%$ by the end of \upgradetwo. 
A further exciting prospect is the study of $b \to u\mu \nu$ decays, which have been beyond experimental reach thus far.
For example the decay $B^+ \to p\bar{p}\mu\nu$ offers a clean experimental signature. 
Our capabilities with this decay could benefit from the enhanced low momentum proton identification with the TORCH subdetector.

Meanwhile, the \decay{\Bcp }{ \jpsi \taum \neub} decay
is an entirely unique state among the flavoured mesons as the bound state of two distinct 
flavors of heavy quark, and, through its abundant decays to charmonium final states, provides a 
highly efficient signature for triggering and reconstruction at high instantaneous luminosities. 
Measurements of \decay{\Bcp }{ \jpsi \ell \neub} decays involve a trade-off between the approximately 100 times 
smaller production cross-section for \Bcp verses the extremely efficient \decay{\jpsi}{\mup\mun} signature in the LHCb 
trigger. For illustration, in Run~1, LHCb reconstructed 
and selected 19\,000 \decay{\Bcp }{ \jpsi \mun \neub} decays, compared with 360\,000 \decay{\Bzb}{\Dstarp\mun\neub}. This resulted in a measurement of $\mathcal{R}(\jpsi)=0.71\pm 0.17\pm 0.18$ \cite{LHCb-PAPER-2017-035}. As a result of the smaller production cross-section, the muonic measurements have large backgrounds from \decay{h}{ \mu} misidentification from the relatively abundant \decay{\B }{ \jpsi X_h} decays, where $X_h$ is any collection of hadrons, and so they
are very sensitive to the performance of the muon system and PID algorithms in the future. Here it is assumed that 
it will be possible to achieve similar performance to Run 1 in the upgraded system. 

To project the sensitivity for \decay{\Bcp }{ \jpsi \taum \neub} based on Ref.~\cite{LHCb-PAPER-2017-035}, it is assumed that 
all the systematic uncertainties can be reduced with the size of the input data except for those that were assumed not to
scale with data for the previous predictions. For these, we assume that they can be reduced up until they reach 
the same absolute size as the corresponding systematic uncertainties in the Run~1 muonic $\RDst$ analysis. In addition, it is
assumed that sometime in the 2020s lattice QCD calculations of the form factors for this process will allow 
the systematic uncertainty due to signal form factors to be reduced by an additional factor of two. This results in
a projected absolute uncertainty for the muonic mode of $0.07$ at the end of Run~3 
and $0.02$ by the end of \upgradetwo,  as can be seen in Fig.~\ref{fig:RXproj}.
Measurements in the hadronic mode can be expected to reach similar sensitivities.

\subsubsection{Searches for LFV, LNV, BNV and interplay with tests of LU}

The LHCb collaboration has recently published~\cite{LHCb-PAPER-2017-031} the world's best limits on the branching fractions of the \BsToEMu
and \BdToEMu decays using the first 3\invfb collected in 2011 and 2012 at 7 and 8\tev respectively. The acceptance of the \BsToEMu decays can be affected by the relative contribution of the two \Bs mass eigenstates to the total decay amplitude, due to their large lifetime difference. Therefore, the upper limit on the branching fraction of \BsToEMu decays is evaluated in two extreme hypotheses: where the amplitude is completely dominated by the heavy eigenstate or by the light eigenstate. The results are  $\BF(\BsToEMu) < 6.3\,(5.4) \times 10^{-9}$ and $\BF(\BsToEMu) < 7.2\,(6.0) \times 10^{-9}$ at $95\%\,(90\%)$ CL, respectively. The limit for the branching fraction of the \Bd mode is  $\BF(\BdToEMu) < 1.3\,(1.0) \times 10^{-9}$ at $95\%\,(90\%)$ CL.

Assuming similar performances in background rejection and signal retention as in the current analysis, at the end of the \upgradeone data taking period the LHCb
experiment will be able to probe branching fractions of \BsToEMu and \BdToEMu decays down to $8\times 10^{-10}$
and $2\times 10^{-10}$, respectively. The additional statistics accumulated during the \upgradetwo data taking period will push down
these limits to $3\times 10^{-10}$ and $9\times 10^{-11}$ respectively, close to the interesting region where NP effects may appear. The improvement at \upgradetwo\ over
\upgradeone\ in electron  reconstruction will be very important in attaining, or exceeding, this goal. 

An upper limit on the \BdToTauMu channel has been already set by $\babar$: $\BR\left( \BdToTauMu\right)< 2.2\times 10^{-5} $ at $90\%$~CL~\cite{Aubert:2008cu}. The first search on the \BsToTauMu channel is in progress in LHCb and the results are expected soon on data recorded in 2011 and 2012 using the \tauppp and \tauppppi0 decay modes.
Given the presence of a neutrino that escapes detection this kind of analysis is much more complicated than those investigating electron or muon final states. A specific reconstruction technique is used in order to infer the energy of the $\nu$, taking advantage of the known $\tau$ vertex position given by the $3\pi$ reconstructed vertex. 
This way, the complete kinematics of the process can be solved up to a two-fold ambiguity. LHCb expects to reach sensitivities of a few times $10^{-5}$ with the Run~1 and 2 data sets. Extrapolating the current measurements to the \upgradetwo LHCb could reach  $\BR\left( \BdToTauMu\right)< 3\times 10^{-6} $ at $90\%$~CL. The mass reconstruction technique depends heavily on the uncertainty on the primary and the $\tau$ decay vertices, hence improvement in the tracking system in \upgradetwo, including a removal or reduction in material of the VELO RF foil, will be very valuable.

If New Physics allows for charged lepton flavour violation then the branching fractions of $B\to K \ell \ell^{'}$  or $\Lb \to \Lz \ell \ell^{'}$ will be enhanced with respect to their purely leptonic counterparts, since the helicity suppression is lower. 
Furthermore, if observed, they would allow the measurement of more observables with respect to the lepton flavour violating decays discussed in the previous sections, thanks to their multi-body final states and, in the case of $\Lb$, to the non-zero initial spin. 

In many generic new physics models with lepton flavour universality violation, charged lepton flavour violating decays of $b$-hadrons can be linked with the anomalies recently measured in $b\rightarrow s\ell\ell$ decays~\cite{LHCb-PAPER-2014-024,LHCb-PAPER-2017-013,LHCb-PAPER-2015-051}. 
The current  limits set by the \bfactories on the branching fractions of $B \to Ke \mu$ and $B \to K\tau \mu$ decays are $< 13\times 10^{-8}$~\cite{Aubert:2006vb} and $< 4.8\cdot 10^{-5}$~\cite{Lees:2012zz}  at 90\% confidence level, respectively.  

At LHCb, searches for $\decay{\Bu}{\Kp e^\pm\mu^\mp}$, $\decay{\Bd}{\Kstarz\tau^\pm\mu^\mp}$, $\decay{\Bu}{\Kp\tau^\pm\mu^\mp}$ and $\decay{\Lb}{\Lz e^\pm\mu^\mp}$ are ongoing. These searches are complementary, as charged lepton flavour violation couplings among different families are expected to be different.
The analyses involving $\tau$ leptons reconstruct candidates via the $\decay{\taum}{\pim\pim\pip\nu_{\tau}}$ channel,
which allows the reconstruction of the $\tau$ decay vertex.\footnote{It should be noted that searches for $\decay{\Bu}{\Kp\tau^\pm\mu^\mp}$ from $B_{s2}^*$ without $\tau$ reconstruction can give complementary information.} 
All these decays contain at least one muon, which is used to efficiently trigger on the event. Usually, since these decays involve combinations of leptons that are not allowed in the Standard Model, the backgrounds can be kept well under control,  leaving very clean samples only polluted by candidates formed by the random combinations of tracks. This combinatorial effect is higher for the channel with a $\tau$ in the final state decaying into three charged pions.  The other relevant background  comes from chains of semileptonic decays, where two or more neutrinos are emitted and therefore combinations of leptons of different flavours are possible. These decays have typically a low reconstructed invariant mass, due to the energy carried away by the neutrinos, and so they do not significantly pollute the signal region. 

The expected upper limits at LHCb using the first $9 \invfb$ of data taken are
$\mathcal{O}(10^{-9})$ and $\mathcal{O}(10^{-6})$ for the $\decay{\Bu}{\Kp e^\pm\mu^\mp}$ and $\decay{\Bd}{\Kstarz\tau^\pm\mu^\mp}$ decays respectively, at 90\% confidence level.
The limit for $\decay{\Bu}{\Kp\tau^\pm\mu^\mp}$ is expected to be similar to $\decay{\Bd}{\Kstarz\tau^\pm\mu^\mp}$.
The sensitivity of these analyses scales almost linearly with luminosity  for $\decay{\Bu}{\Kp e^\pm\mu^\mp}$, 
and with the square root of the luminosity for $\decay{\Bd}{\Kstarz\tau^\pm\mu^\mp}$.
In both cases, the expected limits using the \upgradetwo data are in the region of interest of the models currently developed for explaining the $B$ anomalies, so they will provide strong constraints on the New Physics scenarios with charged lepton flavour violation. 

Experimentally, LNV and BNV measurements are null searches, so sensitivity is assumed
to scale linearly with luminosity $L$ when the background
is negligible and as $\sqrt L$ if the background is significant. LHCb has already published searches in certain channels, and others are
in progress:
\begin{itemize}
  \item Searches for LNV in various $B$-meson decays of the form
    $B \to X \mu^+ \mu^+$, where $X$ is a system of one or more hadrons.
    The principal motivation is the sensitivity to contributions from Majorana neutrinos~\cite{Atre:2009rg},
    which may be on-shell or off-shell, depending on the decay mode.
    The published results consist of
      searches for $\decay{\Bp}{\Km \mu^+ \mu^+}$, $\decay{\Bp}{\pim \mu^+ \mu^+}$ and $\decay{\Bu}{D^+_{(s)}\mun\mun}$~\cite{LHCb-PAPER-2011-009, LHCb-PAPER-2011-038, LHCb-PAPER-2013-064}. A limit of $\mathcal{B}(B^+ \to \pi^- \mu^+ \mu^+) < 4 \times 10^{-9}$ is set at the 95\% confidence level, along with more detailed limits as a function of the Majorana neutrino mass. Since the combinatorial background was found to be low but not negligible with the Run~1 data, we estimate that the limit can be improved by a factor of ten with the full \upgradetwo dataset.
  \item Search for BNV in \Xibz oscillations~\cite{LHCb-PAPER-2017-023}.
    Six-fermion, flavour-diagonal operators, involving two fermions from each
    generation, could give rise to BNV/LNV without violating the nucleon stability
    limit~\cite{Smith:2011rp,Durieux:2012gj}.
    Since the \Xibz~($bsd$) has one valence quark from each generation, it could
    couple directly to such an operator and oscillate to \Xibzbar.
    The published search used the Run1 data and set a lower limit on the oscillation
    period of 80\ps. Since events are tagged by decays of
    the \XibPrimeMinus and \XibStarMinus resonances, with the former being particularly
    clean, and since the analysis also uses the decay-time distribution of events,
    the sensitivity is expected to scale linearly. 
    Although the decay mode used in the published analysis is hadronic
    ($\Xibz \to \Xicp \pim$), future work could also benefit from the
    lower-purity but higher-yield semileptonic mode $\Xibz \to \Xicp \mun \neumb$.
  \item $\Lc \to \antiproton \mu^+ \mu^+$.
    This channel has previously been investigated at the $e^+ e^-$ \bfactories.
    The current upper limit, obtained by \babar~\cite{Lees:2011hb}, is
    $\mathcal{B}(\Lc \to \antiproton \mup \mup) < 9.4 \times 10^{-6}$
    at the 90\% confidence level.
    With Run~1 and~2 data alone, it should be possible to reduce this
    to $1 \times 10^{-6}$. Further progress depends on the background
    level, but an additional factor of 5--10 with the full \upgradetwo statistics
    is likely.
  \item $\Lc \to \mu^+ \mu^- \mu^+$.
    Experimentally, this is a particularly promising decay mode:
    the final state with three muons is very clean, and there are
    no known sources of peaking background.
    This search could be added for little extra effort to the
    $\taum \to \mup \mun \mun$ search described in the
    preceding section.
\end{itemize}


%\subsection{Combined theory/experiment perspective on global interpretations (EFT/non-EFT)}
%a particular emphasis on impact of LHC-only combined analysis and complementarity of this with Belle II

%Needs to be expanded, Belle II brought into it...
%Bring in the topic of non-EFT explanations? Keep looking for new light degrees of freedom?
\end{document}
