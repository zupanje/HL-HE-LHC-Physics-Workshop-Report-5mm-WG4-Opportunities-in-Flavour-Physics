In this section we discuss the status of the interpretation of $b\to s$ transitions in the SM and BSM, the prospects for future sensitivities at the LHC, and the complementarity with  Belle-II. All current and planned measurements of processes involving $b\to s$ transitions are related to weak decays of $b$-hadrons. At present, most of the data is on $B$ decays, with a smaller fraction on $B_s$ decays. LHCb has already provided a few measurements on $\Lambda_b$ decays, with data sets that will increase dramatically in future runs, including a significant output on decays of other $b$-hadrons such as $\Omega_b$
or $\Xi_b$~\cite{EoI} (see Sec.~\ref{sec:7_exp}). 

Theoretically, the decays of $b$-hadrons are best described within the Weak Effective Theory (WET), where flavor-changing transitions are mediated by ``effective" dimension-six
operators with Wilson coefficients (WCs) that encapsulate all SM weak and heavy NP effects. Thus all observables can be calculated in full generality model-independently in terms of the WCs and QCD hadronic matrix elements. The relevant effective Lagrangian for $b\to s$ transitions at low-energies in the SM is~\cite{Buchalla:1995vs}
\begin{equation}
\label{eq:Heff}
  {\cal{H}}_{\rm eff}^{\rm SM}=
  \frac{4 G_F}{\sqrt{2}}\sum_{p=u,c}\lambda_{ps}\left(C_1 O_1^p+C_2 O_2^p +\sum_{i=3}^{10} C_i  O_i\right),
\end{equation}
with $\lambda_{ps}=V_{pb}^{} V_{p s}^\ast$. The $O_{1,2}^p$,  $O_{3,\ldots,6}$, and $O_{8}$ are the so-called ``current-current'', ``QCD-penguin'' and ``chromo-magnetic'' operators, respectively, and they contribute to the $b\to s\gamma$ and $b\to s\ell\ell$ transitions via an electromagnetic interaction. The $O_{7\gamma}$ and $O_{9,10}$ are the ``electromagnetic dipole'' and the ``semileptonic'' operators, respectively.
The BSM can enter through the WCs $C_1,\ldots,C_{10}$, or their chirally-flipped versions, $b_{L(R)}\to b_{R(L)}$, giving $O_{7\gamma,\ldots, 10}'$, or through scalar and tensor semileptonic operators~\cite{Bobeth:2007dw}. Furthermore, in BSM, semileptonic operators can induce lepton-flavor universality violation (LFUV) or charged-lepton flavor violation (CLFV).
For the purpose of this Section, the relevant operators for the interpretation of $b\to s\gamma$ and $b\to s\ell\ell^\prime$ data in terms of short-distance BSM are
%
\begin{equation}
\begin{split}
&O_{7\gamma}^{(\prime)}=\frac{e}{16\pi^2}(\bar s\sigma_{\mu\nu}P_{R(L)} b)F^{\mu\nu},
\\
&O_9^{\ell\ell^\prime(\prime)} =\frac{\alpha_{\rm em}}{4\pi}
(\bar{s} \gamma_{\mu} P_{L(R)} b)(\bar{\ell} \gamma^\mu \ell^\prime),\quad
\,\,
O_{10}^{\ell\ell^\prime(\prime)} =\frac{\alpha_{\rm em}}{4\pi}
(\bar{s} \gamma_{\mu} P_{L(R)} b)( \bar{\ell} \gamma^\mu \gamma_5 \ell^\prime),\label{eq:O10}
\end{split}
\end{equation}
where we have added the leptonic flavor labels (denoting $C_i^\ell\equiv C_i^{\ell\ell}$). 


By calculating and measuring a large set of independent observables, one can perform global fits to all the relevant WCs, and comparing with their SM values,
extract information on NP model-independently. 
%After that, the resulting information can be contrasted to NP models to learn about the details of the underlying dynamics. 
This program has been carried out since the start of the LHC, and culminated in the current $b\to s$ anomalies~\cite{1304.6325,1307.5683,1308.1501}.
The future runs at the LHC
%, including the HL
%phase
 will allow to establish the anomalies or refine our understanding of these transitions,
 %these discoveries, 
 as we discuss below. 
 %Here we discuss the {\bf status} and {\bf prospects} in this program, from the point of view of 1) the calculation of the observables, 2) the global fits to the data, and 3) the interpretation in terms of models.


\subsubsection{Observables and Hadronic Matrix Elements}

\subsubsubsection{$B_q\to\ell^+\ell^-$}

From a theoretical perspective, the purely leptonic channel, $B_{s}\to \ell^+ \ell^-$, is the cleanest exclusive $b\to s\ell\ell$ decay. Up to QED corrections~\cite{1708.09152}, all QCD effects are contained in a decay constant  which is precisely and reliably computed on the lattice, $f_{B_s}=228.4(3.7)$~MeV. Going beyond this accuracy in $f_{B_s}$ is difficult (see Section~\ref{sec:lattice}) \td{JZ: but the projected error reduction is a factor of 3} and the current theoretical error is at the level of 5\%~\cite{1708.09152},
\begin{align}\label{eq:Bsmumu_th}
\mathcal B(B_s\to\mu^+\mu^-)_{\rm SM}= (3.57\pm0.17)\cdot10^{-9},    
\end{align}
which is dominated by the uncertainties on the relevant CKM parameters. The latest experimental measurement by LHCb has an error of~$\sim 25\%$~\cite{LHCb-PAPER-2017-001}, whereas future measurements at the LHC plan to bring this down to the percent level (see Sec.~\ref{sec:7_exp}). Beyond the SM, the decay $B_{s}\to \mu^+ \mu^-$ gives very strong constraints to the scalar and pseudoscalar operators~\cite{Alonso:2014csa}, and also to $C^\mu_{10}$, which has an impact on the fits and the $b\to s\mu\mu$ anomalies. Searches for the CLFV channels $B_s\to \tau\mu$ and $B_s\to\mu e$ are important, because they would be an unambiguous signal of NP, which can be connected to the LFUV signals in $R_{K^{(*)}}$~\cite{Glashow:2014iga}. Besides the branching fractions, an effective lifetime observable is accessible by exploiting the nonvanishing width difference in the $B_s$ system~\cite{DeBruyn:2012wk}. This provides complementary constraints to the WCs~\cite{Altmannshofer:2017wqy} and a precise measurement will be accessible to the LHCb. 

These measurements are simultaneously sensitive to the decay channel $B_d\to\mu^+\mu^-$, although its branching fraction is further suppressed by a CKM factor~\cite{Bobeth:2013uxa},{\bf\color{red}[Update parameters?]}
\begin{align}\label{eq:Bdmumu_th}
\mathcal B(B_d\to\mu^+\mu^-)_{\rm SM}= (1.06\pm 0.09)\cdot10^{-10}. 
\end{align}
The ratio $\mathcal B(B_s\to\mu^+\mu^-)/\mathcal B(B_d\to\mu^+\mu^-)$ can be predicted more accurately in the SM, testing the flavor structure of the short-distance dynamics. Finally, the lower end of the $m_{\mu^+\mu^-}$ spectrum is sensitive to the decay $B_s\to\mu^+\mu^-\gamma$ (where the photon is not measured)~\cite{Dettori:2016zff}. Theoretically, this mode is interesting because the extra photon lifts the chiral suppression of the leptonic mode and gives access to the WC $C_9^\mu$. However, it is also challenging to predict because of long-distance hadronic contributions~\cite{Guadagnoli:2017quo}.

\subsubsubsection{$B_q\to M \ell^+\ell^-$}

The most important semileptonic decay modes are $B\to K^{(*)}\mu^+\mu^-$ and $B_s\to \phi \mu^+\mu^-$. Currently they dominate the global fits to $b\to s\ell\ell$ data because of the experimental precision achieved
and the large number of observables they give access to. This allows one to constrain several independent combinations of WCs and hadronic parameters. Since there is no experimental information related to the polarization of the final-state leptons,
the measurable observables arise from the kinematic differential distributions of the final state momenta. These are customarily written as  angular distributions with coefficients that depend on the dilepton invariant mass squared $q^2$, which, integrated in specific bins of $q^2$,
lead to the observables that are predicted and measured. 

In the case of the three-body mode $B\to K\mu^+\mu^-$, there are three
angular observables: the differential branching fraction, $d{\cal B}/dq^2$, the forward-backward asymmetry, $A_{\rm FB}(q^2)$,
and the ``flat term", $F_H(q^2)$~\cite{0709.4174}.
The kinematic distribution of the four-body decay, $B\to V(\to M_1M_2) \ell^+\ell^-$, contains many more independent angular coefficients, up to 12 in the most general case, called $I_i(q^2)$ or $J_i(q^2)$~\cite{0811.1214,1202.4266}.
Normalizing these by the total differential rate $d\Gamma/dq^2$ and symmetrizing or antisymmetrizing in the two charge-conjugate modes leads to the
observables $S_i$, $A_i$~\cite{0811.1214}.
A subset of these observables is also available for
$B_s\to \phi \mu^+\mu^-$. It is convenient to define certain combinations of angular
observables where form factor uncertainties largely cancel in the heavy-quark limit, called ``optimized observables''. An independent set of these, optimized at low-$q^2$,
is given by the $P_i^{(\prime)}$ basis~\cite{1202.4266,1207.2753,1303.5794}. Optimized observables at large-$q^2$ also exist~\cite{1006.5013,1303.5794}. 


The  observables in exclusive semileptonic decays, $B\to M \ell^+ \ell^-$,  
%of the form  discussed in this section and all the observables that can be defined from them 
are specified
by transversity (or helicity) amplitudes. They depend on two types of hadronic matrix elements: ``local'' (form factors) and ``non-local'' (see, e.g.,~\cite{Jager:2012uw,1707.07305}).
Local form factors for $B\to K$, $B\to K^*$ and $B_s\to \phi$ transitions are known at low-$q^2$ from two different versions of light-cone sum rules (LCSRs)~\cite{hep-ph/0611193,1503.05534}, and at large-$q^2$ from Lattice QCD~\cite{1509.06235,1310.3722}. A comparison between both determinations is done by parametrizing
the $q^2$ dependence via the $z$-expansion~\cite{0807.2722}, which is based on the analytic structure of the matrix elements. Future prospects for the theoretical
precision on 
%the calculation of 
the form factors rely mostly on the improvements in Lattice QCD calculations. Note that both lattice QCD and LCSRs work in the narrow-width limit for  
%the case of 
vector mesons.
%, however, both determinations
%assume the narrow-width limit. 
A calculation beyond this approximation is possible within the LCSRs~\cite{1701.01633,1709.00173}, and points to
%wards 
a correction
of up to $10\%$. There are some prospects to treat hadronic resonances on the lattice, as well~\cite{1704.05439}.

Non-local effects are significantly more difficult to estimate~\cite{hep-ph/0106067,1006.4945,1101.5118}. Data-driven methods might be able to reduce the uncertainties on these hadronic contributions and will benefit significantly from the high statistics collected at LHCb in the HL phase.
All of these methods are based on precise measurements of the $q^2$ spectra, in conjunction with a theoretically motivated parametrization of the $q^2$ dependence
of the amplitudes, and a theory benchmark that allows to separate short-distance contributions from long-distance contributions.

At low $q^2$, the theory input is based on light-cone Operator Product Expansion (OPE)
at very low (or negative) $q^2$~\cite{hep-ph/0106067,1006.4945},
an expansion that breaks down at the perturbative $c\bar c$ threshold $q^2\simeq 4m_c^2$.
Parametrizations of the $q^2$ dependence are based on Taylor expansion in powers of $q^2$~\cite{1512.07157}, on  dispersion relation~\cite{1006.4945} or on 
$z$-expansion~\cite{1707.07305}. The latter two parametrizations implement analyticity constraints and use
extra information, such as data on (or in between) the $B\to \psi K^*$ decays, with $\psi=\{J/\psi, \psi(2S)\}$. Short- and long-distance effects are disentangled by the experimental input from $B\to \psi K^*$, the fixed $q^2$ dependence of the NP contribution, and by the theory constraints at negative $q^2$. The experimental prospects for this data-driven approach were studied in~\cite{1805.06378}, showing that future LHC data could provide a higher level of control over the long-distance contribution at low $q^2$.

At high $q^2$, the theory input is based on the low-recoil OPE~\cite{hep-ph/0404250,1006.5013,1101.5118}. This method relies
on the theoretical assumption that resonant effects from ``above-threshold'' charmonia will average out within sufficiently broad $q^2$ bins (``quark-hadron duality''). Thus, only a single bin
in the whole high-$q^2$ region is typically considered in the global fits. The $q^2$ spectrum can be used to give estimates and test models for the intrinsically nonperturbative ``duality violating'' effects. Currently, these analyses are carried out within the
``Kr\"{u}ger-Sehgal'' (naive factorization) approach~\cite{hep-ph/9603237}, which allows to use data on the $R(s)$ ratio in $e^+e^-$ annihilation~\cite{1101.5118,1406.0566,1606.00775}. Ref.~\cite{1606.00775} uses all currently available data on $B\to K^*\mu\mu$
at low recoil and finds agreement with the OPE within $\sim 20\%$ in all the bins. Notably, future precision data from the LHC with the
expected fine binning will be essential in refining these data-driven methods and disentangle NP contributions. As in the low-$q^2$ case, combined fits to hadronic parameters and NP are also beneficial~\cite{1606.00775}.

Hadronic uncertainties largely cancel in lepton-universality ratios such as  $R_{K^{(*)}}$~\cite{hep-ph/0310219}. The SM predictions are thus limited only by the electromagnetic corrections~\cite{Bordone:2016gaq}. Current LHCb measurements show tensions with the SM in $R_{K}$~\cite{1406.6482} and two bins of $R_{K^*}$ ~\cite{1705.05802} at approximately $\sim 2.5\,\sigma$ each. Much higher precision will come from future data at LHCb~\cite{EoI},
and, independently, from Belle-II, which can confirm $R_K$ at $5\,\sigma$ with $20\,\text{ab}^{-1}$~\cite{Belle2}. In presence of LFUV contributions, the predictions are less precise and ``optimized'' observables based on angular analyses of muonic and electronic modes can improve the sensitivity to different BSM scenarios~\cite{1605.03156,1612.05014}. In fact, in many BSM, LFUV and CLFV are connected, and decays such as $B\to K^{(*)}\tau\mu$ and $B\to K^{(*)}\mu e$ become clear targets for the HL-LHC.
% in the context of confirming the $R_{K^{(*)}}$ anomalies.   
Assuming that the LFUV anomaly persists and the BSM contributions are in the muonic WCs, as the global fits to $b\to s \mu\mu$ currently suggest, one can use future precise and fine-binned measurements in $b\to se^+e^-$ exclusive modes to fit for the hadronic parameters directly~\cite{1805.06401}. 

Other measurements with potential impact on the $b\to s\ell\ell$ fits will be possible at the HL-LHC.  The LHC experiments have 
%offers a quite 
a unique opportunity to  measure the $\Lambda_b\to\Lambda\mu^+\mu^-$ decays, probing the $b\to s\mu^+\mu^-$ transition in a baryonic system. 
For $B_s\to\phi \mu^+\mu^-$, a flavor-tagged time-dependent analysis allows to access independent observables~\cite{1502.05509}
%, with 
sensitive to BSM.
% and
%different properties. 
The excellent expected sensitivity of the $B_s$ and $\Lambda_b$ decay measurements will make it possible to extend the 
%above 
global fit program, outlined above,
 to these modes.
 % too. 
  Finally, the study of exclusive $b\to d\mu^+\mu^-$ transitions will reach the 
  %current 
  level of precision we now have in $b\to s\mu^+\mu^-$.
This will allow to extend the program of global fits  to the $b\to d\mu^+\mu^-$ transitions, 
%the WCs of these processes, to 
setting constraints in a different flavor sector or, if  $b\to s\ell\ell$ anomalies remain, to give further insights on the BSM flavour structure). The $b\to d\mu^+\mu^-$ modes are theoretically more challenging than their $b\to s\ell\ell$ counterparts, since new large long-distance contributions appear at low $q^2$, e.g., in the form of light-resonances~\cite{Hambrock:2015wka}. 

\subsubsubsection{Radiative decays: $B_s\to \phi\gamma$ and related modes}

The radiative decays are obvious probes of the electromagnetic dipole operators.  Strategies to determine with these decays the helicity of the photon (and therefore the presence of BSM WC $C_{7\gamma}^{\prime}$)  
have been investigated intensively~\cite{Atwood:1997zr,
Gronau:2001ng,Gronau:2002rz,Ball:2006eu,
Kou:2010kn,Becirevic:2012dx}. Approaches based on
the $\CP$-averaged exclusive decay rates are prone to hadronic uncertainties
and are sensitive quadratically to $C_{7\gamma}^{\prime}$~\cite{Kou:2010kn,Becirevic:2012dx}.
In contrast, interference between helicity amplitudes of the $B_{(s)}$ and $\bar B_{(s)}$ decays is directly sensitive to the photon polarization. The interference can be measured by using time-dependent $\CP$-asymmetries in $B\to V\gamma$ (the $S_{V\gamma}$ parameter), {\color{red}\bf[Check]} or those induced by the width-differences $\Delta\Gamma-q$, 
%which have been 
shown to be clean null tests of the SM~\cite{Atwood:1997zr,Muheim:2008vu}. Experimentally, $S_{K^*\gamma}$ has been measured in the $B$-factories~\cite{Aubert:2005bu,Ushiroda:2006fi}, while a measurement of the width-difference effects have been achieved at the LHCb~\cite{LHCb-PAPER-2016-034}. Prospects in the HL-LHC include reaching a few percent precision in this observable, which will provide a strong constraint on the chirality of the electromagnetic dipole operators.  

The $B\to K^{(*)}\ell^+\ell^-$ decays are also sensitive to the interference of the two helicities close to the photon pole $q^2\simeq0$ through the angular observables $P_1$ and $P_3^{CP}$ (also called $A_{\rm T}^{(2)}$ and $A_{\rm T}^{\rm Im}$). The electronic mode is especially suited for these measurements given the lower dilepton threshold~\cite{Jager:2014rwa}. This has been  measured by the LHCb at a $\sim20\%$ precision with data of Run 1~\cite{1501.03038}, and will be improved to a few percent precision in the HL-LHC. 



\subsubsubsection{Interplay with Belle-II: Inclusive $B$ decays and $B_q\to M \nu \bar \nu$}

Besides a parallel program of measurements in many of the channels and observables mentioned above~\cite{Belle2}, Belle-II will also measure decays which are very challenging in the LHC environment. This includes inclusive decays and the $b\to s\nu\bar\nu$ transitions. The inclusive observables used in current fits are the branching fractions
${\cal B}(B\to X_s \gamma)$ and ${\cal B}(B\to X_s \ell^+\ell^-)$. 
Belle-II will improve these, including good measurements of the forward-backward asymmetry in $B\to X_s \ell^+\ell^-$, which will also have
an interesting effect in the fits~\cite{Belle2}. In particular, Belle-II measurements of $B\to X_s \mu^+\mu^-$ will be able
to test the LHCb anomalies independently by 2024~\cite{Belle2}. Theoretically, these inclusive rates are dominated by the perturbative
partonic decay of the $b$ quark up to small non-perturbative effects~\cite{1003.5012,1705.10366}, given an appropriate cut on the invariant mass of the hadronic final state. Calculations of both rates have been done to high accuracy~\cite{1503.01789,1503.04849}. 
%On the other hand, t
The exclusive decays $B\to K^{(*)} \nu \bar \nu$ and related modes~\cite{1409.4557}, on the other hand, depend only on local form factors. Belle-II is expected to provide measurements with an uncertainty of about $\sim 10\%$,
assuming the rates are SM-like~\cite{Belle2}. These modes do not probe directly the WCs of the $b\to s \ell\ell$ transitions . However, they are likely correlated through $SU(2)_L$ gauge invariance, if the BSM is above the EW scale.

\subsubsection{Model-Independent Fits}

Existing measurements show hints of deviations from the SM expectations in three
different classes of measurements:
in $B\to K^*\mu^+\mu^-$ angular observables \cite{LHCb-PAPER-2015-051},
in branching fractions of exclusive $b\to s\mu^+\mu^-$ decays (in particular
$B_s\to\phi\mu^+\mu^-$ \cite{LHCb-PAPER-2015-023}),
and in $\mu$-$e$ universality (LFU) tests \cite{LHCb-PAPER-2014-024,LHCb-PAPER-2017-013}.
None of the individual measurements is in tension with the SM by more than
$4\sigma$. However, a global significance of the tensions can be defined in a specific
framework of NP, such as the model-independent framework provided by the weak effective Hamiltonian.


Several groups have performed global fits of the WCs to existing $b\to s\ell\ell$ data
(see \cite{Altmannshofer:2017fio,Capdevila:2017bsm,Altmannshofer:2017yso,Hurth:2017hxg,Ciuchini:2017mik,Geng:2017svp}
for recent fits).
Three classes of fits can be distinguished: fits to $b\to s\mu\mu$ data only,
fits to $\mu$-$e$ LFU ratios only, and combined fits assuming no NP
in $b\to see$ transitions. All these fits agree -- up to differences that can be
attributed to different theoretical inputs or  different selection
of observables -- and arrive at two basic conclusions.
First, that there is a tension in $b\to s\mu\mu$ data alone, and could be
explained by a NP shift in the WC $C_9^\mu$~\cite{1307.5683} (maybe combined with $C_{10}^\mu$), or by a not well understood
hadronic effect.
Second, the LHCb measurements of $R_{K^{(*)}}$ alone are already in tension with the SM and lepton-flavor universality at 4$\sigma$. This cannot be explained by hadronic effects. Assuming that it is due to NP coupling only to muons, one finds it is consistent with the NP needed to accommodate the $b\to s\mu\mu$ anomaly~\cite{Alonso:2014csa}.
Singling out the WC $C_9^\mu$ in the muonic transition and performing
a global fit to all the data, the log-likelihood ratio between the best-fit
point and the SM hypothesis corresponds to a deviation ranging from $4\sigma$ to more than $6\sigma$, depending on the theoretical assumptions~\cite{Altmannshofer:2017fio,Capdevila:2017bsm,Altmannshofer:2017yso,Hurth:2017hxg,Ciuchini:2017mik,Geng:2017svp}. 

Clearly, future experimental efforts that can clarify the origin of the above tensions 
%o settle the question whether these deviations are due to physics beyond the SM 
are of 
%the 
utmost
importance. If 
%effects of 
LFU is indeed violated in 
% are indeed surfacing in the 
$b\to s\ell\ell$ transitions, then 
%lepton-universality ratios like 
$R_K$ and
$R_{K^*}$ are theoretically 
%extremely 
clean smoking guns that allow to establish a deviation from the SM. 
%However, g
At the same time, global fits to all the relevant observables will remain 
%highly 
relevant 
%an indispensible tool 
for several reasons: {\em (i)}
%\begin{itemize}
% \item[$\bullet$] 
 if the hints for LFUV disappear with more statistics, LFU new physics effects might still hide in  observables that are  less clean theoretically, {\em (ii)}
% \item[$\bullet$] 
 to identify the nature of NP
 %new physics 
 (and not just its presence), the values of the BSM Wilson coefficients
 have to be determined, and {\em (iii)},
 %(and even ratios such as $R_K$ and
%$R_{K^*}$ may have non-negligible theory uncertainties {\em away}
%from the SM limit),
 %\item[$\bullet$] 
 they allow to simultaneously determine poorly known hadronic effects from the data.
%\end{itemize}


\begin{figure}
    \centering
    \includegraphics[width=0.46\textwidth]{section7/C9mumu_C10mumu_pp_Run4.pdf}
    \includegraphics[width=0.46\textwidth]{section7/C9mumu_C9ee_pp_Run4.pdf}
    \caption{Current (not filled) and projected global fit results for
    LHCb and Belle-II (inclusive and exclusive decays) in the planes
    of the Wilson coefficients $C_9^\mu$, $C_{10}^\mu$, and $C_9^e$, 
    allowing two of them to vary at a time.
    The central values of the extrapolations correspond to NP scenarios
    with different central values. The contours correspond to $1\sigma$ uncertainties.
    The light gray contours around 
    the Standard Model point (black dot)
    show the hypothetical exclusion power attainable with a combined sensitivity
    of LHCb's 50~fb$^{-1}$ and Belle-II's 50~ab$^{-1}$ datasets.
    Figure taken from~\cite{Albrecht:2017odf}.}
    \label{fig:WC}
\end{figure}

Sensitivity extrapolations 
%of the  
of global fits
after future LHCb runs and Belle-II,
assuming the current theory approaches,
were performed in Ref.~\cite{Albrecht:2017odf}.
The projections in the  $(C_9^\mu$, $C_{10}^\mu)$
and $(C_9^\mu$, $C_{9}^e) $ planes
%Wilson coefficients 
are shown in Figure~\ref{fig:WC}. {\bf\color{red}[Update!]}
Further improvements beyond those taken into account in  projections in Figure~\ref{fig:WC} are:
{\em (i)}  $B\to K^*e^+e^-$ angular analysis can be included, where for LFUV NP a simultaneous amplitude analysis of $B\to K^*\mu^+\mu^-$ and  $B\to K^*e^+e^-$ decays is more powerful than separate analyses~\cite{1805.06401};
{\em (ii)} combined fits to semi-leptonic and non-leptonic decays as well as the
precise dilepton invariant mass spectra will allow to better disentangle
long- and short-distance effects in $B\to K^*\mu^+\mu^-$~\cite{1805.06378}; {\em (iii)} 
unbinned determination of Wilson coefficients~\cite{Hurth:2017sqw} will exploit fully the experimental
potential.

%can however be affected significantly by
%additional qualitative improvements on the theoretical and
%the experimental side. For instance,
%\begin{itemize}
%\item[$\bullet$] the above projections have not yet taken into account the angular analysis
%of $B\to K^*e^+e^-$, which will however play a crucial role in disentangling
%NP models;
%\item[$\bullet$] in the presence of LFU violation, 
%a simultaneous amplitude analysis of $B\to K^*\mu^+\mu^-$ and 
%$B\to K^*e^+e^-$ decays is even more powerful than separate 
%analyses~\cite{1805.06401};
%\item[$\bullet$] combined fits to semi-leptonic and non-leptonic decays as well as the
%precise dilepton invariant mass spectra will allow to better disentangle
%long- and short-distance effects in $B\to K^*\mu^+\mu^-$~\cite{1805.06378};
%\item[$\bullet$] using the full event information would allow an unbinned determination
%of the Wilson coefficients~\cite{Hurth:2017sqw}, fully exploiting the experimental
%potential. In this case, an important problem to solve is communicating this information to theorists to allow extracting the Wilson coefficients
%with different theoretical models for form factors and hadronic contributions.
%\end{itemize}


\subsubsection{Models of NP for $b\to s\ell\ell$}
%{\bf\color{red}[Move this to the end of the section?]}\\
Finally, we briefly review the NP models that can explain the $b\to s\ell \ell$ anomalies, assuming these become statistically significant. The model independent analysis gives for the NP scale
%of the anomalies points to a new physics scale of
% 
\begin{equation}
 \Lambda_\text{NP} = \frac{4\pi}{e} \frac{1}{\sqrt{|V_{tb} V_{ts}^*|}} \frac{1}{\sqrt{|\Delta C_9^\mu|}} \frac{v}{\sqrt{2}} \simeq \frac{35~\text{TeV}}{\sqrt{|\Delta C_9^\mu|}}~,
\end{equation}
%
where $\Delta C_i^\ell$ denotes the NP contribution to $C_i^\ell$.
The actual mass of the NP degrees of freedom responsible for the anomalies can be much smaller, if the NP couplings are small, $\ll 1$, and/or if the NP contributions arise at loop level, instead of at tree level.

%There are many new physics models that can address the observed anomalies. 
At tree level there are two types of NP particles that can contribute to $b \to s \ell \ell$ transitions: $Z^\prime$ and leptoquarks (see left and center diagrams in Fig.~\ref{fig:bsll_diagrams}). Loop level contributions of leptoquarks have been studied extensively in the literature (see right diagram in Fig.~\ref{fig:bsll_diagrams}). Other loop level models have been put forward, see, 
%considered, 
e.g., 
%for example 
Refs.~\cite{Belanger:2015nma,Gripaios:2015gra,Arnan:2016cpy,Kamenik:2017tnu}, but we will not discuss them here in any detail.

%%%%%%%%%%%%%%%%%%%%%%%%%%%%%%%%%%%%%%%%%%%%%%%%%%%
\begin{figure}[t]
\begin{center} 
\includegraphics[width=0.23\textwidth]{section7/diagram_Zprime.pdf} ~~~~~~~~
\includegraphics[width=0.23\textwidth]{section7/diagram_LQ.pdf} ~~~~~~~~
\includegraphics[width=0.26\textwidth]{section7/diagram_loop.pdf}
\caption{Examples of new physics contributions to $b \to s \mu \mu$ transitions. Left: tree level $Z^\prime$ contribution. Center: tree level leptoquark contribution. Right: 1-loop leptoquark contribution.}
\label{fig:bsll_diagrams}
\end{center}
\end{figure}
%%%%%%%%%%%%%%%%%%%%%%%%%%%%%%%%%%%%%%%%%%%%%%%%%%%

\subsubsubsection{$Z^\prime$ models}
Models with a $Z^\prime$ that has flavour violating couplings to quarks and that couples non-universally to leptons can explain the observed anomalies in $b \to s \ell \ell$. In a ``simplified model'' approach, treating the $Z^\prime$ mass and its couplings to the SM fermions as free parameters, irreducible constraints arise from $B_s$ mixing, neutrino trident production and LEP bounds on four-lepton contact interactions.

Combining these constraints gives the following maximal values for the $Z^\prime$ contributions to the relevant Wilson coefficients $\Delta C_{9,10}^{e,\mu}$ in the case of {\em (i)} LH lepton couplings~(\ref{eq:C910_LH}), or {\em (ii)} in the case of vectorial lepton couplings~(\ref{eq:C910_vec})~\cite{Altmannshofer:2014rta},
%
\begin{eqnarray} \label{eq:C910_LH}
 && |\Delta C_9^\mu| = |\Delta C_{10}^\mu| \lesssim 5.4 ~,\quad |\Delta C_9^e| = |\Delta C_{10}^e| \lesssim 0.64 ~, \\  \label{eq:C910_vec}
 && |\Delta C_9^\mu| \lesssim 9.3 ~,\quad |\Delta C_9^e| \lesssim 0.72 ~,\quad \Delta C_{10}^\mu = \Delta C_{10}^e = 0 ~.
\end{eqnarray}
%
%This shows that 
The $Z^\prime$ coupling to muons 
%models 
can comfortably explain the anomalies in $R_K$ and $R_{K^*}$ and the 
%as well as the 
other anomalies in $b \to s \mu \mu$ transitions.
% by affecting the muonic modes. 
Adressing $R_K$ and $R_{K^*}$ through a $Z^\prime$ coupling to electrons
%that affects the electronic modes 
is  possible only in parameter region close to the current upper bounds from $B_s$ meson mixing and
% bounds on 
four-lepton contact interactions. Constraints from $B_s$ mixing will likely become stronger at the time scale of the HL/HE-LHC due to improved lattice predictions of hadronic matrix elements.
The $Z^\prime$ mass can be at most several TeV, otherwise an explanation of the anomalies requires couplings to leptons that are non-perturbatively large. If the $Z^\prime$ has very weak couplings to light quarks and electrons, its production cross section at colliders is tiny. Therefore the $Z^\prime$ could in principle be light, for example at the electroweak scale, or in certain models even much lighter~\cite{Datta:2017pfz,Sala:2017ihs,Altmannshofer:2017bsz}. 

%Many explicit $Z^\prime$ models have been put forward as possible explanations of the anomalies.
A popular class of UV complete $Z'$ models is based on gauging the difference of muon-number and tau-number, $L_\mu - L_\tau$. Once physics is introduced that generates flavor violating couplings of the $Z'$ to quarks, the observed tensions in $b \to s \ell \ell$ decays can be explained~\cite{Altmannshofer:2014cfa,Crivellin:2015mga,Altmannshofer:2015mqa,Fuyuto:2015gmk,Altmannshofer:2016jzy,Baek:2017sew,Chen:2017usq}. $L_\mu - L_\tau$ models predict
%
\begin{equation}
 \Delta C_9^e = 0 ~,\quad \Delta C_9^\mu = - \Delta C_9^\tau ~,\quad \Delta C_{10}^e = \Delta C_{10}^\mu = \Delta C_{10}^\tau = 0~.
\end{equation}
%
Besides $L_\mu - L_\tau$, various other combinations of gauged flavor symmetries have been used to construct $Z'$ models that can address the $b \to s \ell \ell$ anomalies~\cite{Crivellin:2015lwa,Celis:2015ara,Falkowski:2015zwa,Boucenna:2016wpr,Boucenna:2016qad,Celis:2016ayl,Crivellin:2016ejn,Alonso:2017bff,Ellis:2017nrp,Alonso:2017uky,Bonilla:2017lsq,Babu:2017olk,Bian:2017rpg,Tang:2017gkz,Cline:2017ihf}.
Also scenarios where the $Z'$ couples to both quarks and leptons indirectly through the mixing with heavy vectorlike fermions have been considered~\cite{Sierra:2015fma,King:2017anf,Fuyuto:2017sys}.

In models with partial compositeness the $Z'$ can be identified with a heavy neutral spin-1 resonance of the composite sector, typically denoted as $\rho$. Such a resonance generically features flavor violating couplings to quarks. A large degree of compositeness of the left-handed muons is required to explain the $B$ decay anomalies~\cite{Niehoff:2015bfa,Carmona:2015ena,Megias:2016bde,Carmona:2017fsn,Megias:2017ove,Sannino:2017utc}.
The generic expectation in models with partial compositeness is that the $\rho$ couplings are strongest to SM fermions of the third generation, reflecting the mass hierarchy of the SM fermions that is related to their degree of compositeness.
Assuming the dominance of couplings to left-handed leptons, these models 
%therefore 
suggest the pattern 
%
\begin{equation}
 \Delta C_9^e \simeq -\Delta C_{10}^e \ll \Delta C_9^\mu \simeq -\Delta C_{10}^\mu \ll \Delta C_9^\tau \simeq -\Delta C_{10}^\tau ~.
\end{equation}
%
Models in which the $Z^\prime$ is part of a $SU(2)_L$ triplet have been suggested as a simultaneous explanation of the $b \to s \ell \ell$ anomalies and hints for LFUV in semileptonic charged current decays, $R(D)$ and $R({D^*})$~\cite{Bhattacharya:2014wla,Greljo:2015mma,Bhattacharya:2016mcc,Boucenna:2016wpr,Boucenna:2016qad,Buttazzo:2017ixm,Kumar:2018kmr}. 

Different $Z'$ models predict different patterns of NP effects in $b \to s \ell \ell$ and the related $b \to s \nu \nu$ transitions. Future measurements of these transitions at LHCb and Belle II will therefore allow to narrow down viable $Z^\prime$ models.
For example, the $Z^\prime$ models based on the gauged $L_\mu - L_\tau$ symmetry predict effects in the semileptonic $b \to s \mu^+ \mu^-$ and $b \to s \tau^+ \tau^-$ transitions of opposite sign, while $b \to s e^+ e^-$ transitions remain SM-like.
In these models, the purely leptonic $B_s \to \mu^+\mu^-$ and $B_s \to \tau^+\tau^-$ decays, as well as the neutrino modes $B \to K^{(*)} \nu \bar \nu$, are predicted to be SM-like~\cite{Altmannshofer:2014cfa}.

A markedly different pattern arises in the $Z^\prime$ scenarios based on dominant couplings to left-handed fermions of the third generation. In those models the $b \to s \tau^+ \tau^-$ and $B \to K^{(*)} \nu \bar \nu$ rates are typically enhanced compared to the SM predictions by a factor of few. The $B_s \to \mu^+\mu^-$ rate is predicted to be suppressed by approximately $25\%$ compared to the SM prediction. Finally, in contrast to the $L_\mu - L_\tau$ models, rare lepton flavor violating decays like $B \to K^{(*)} \tau \mu$ are predicted at levels of $O(10^{-8})$~\cite{Glashow:2014iga} which might be in reach at the HL-/HE-LHC.\\


\subsubsubsection{Leptoquark models}
There are seven quantum number assignments for leptoquarks that allow tree-level couplings to down-type quarks and charged leptons of the SM.
These are~\cite{Dorsner:2016wpm} $S_3 = (\bar 3,3,1/3)$, $R_2 = (3,2,7/6)$, $\tilde R_2 = (3,2,1/6)$, $\tilde S_1 = (\bar 3,1,4/3)$, $U_3 = (3,3,2/3)$, $V_2 = (\bar 3,2,5/3)$ and $U_1 = (3,1,-1/3)$.
Among these, the couplings of $R_2$, $\tilde R_2$, $\tilde S_1$, and $V_2$ necessarily involve right-handed currents and therefore cannot explain the anomalies.
Thus one is left with the triplet scalar $S_3$, the singlet vector $U_1$, and the triplet vector $U_3$. They all contribute at tree level to the operator $(\bar s \gamma_\alpha P_L b)(\bar \mu \gamma^\alpha P_L \mu)$ and can explain the $b \to s \ell \ell$ anomalies.

In a simplified model approach the constraints on the leptoquark models are very weak.
While the leptoquarks $S_3$, $U_1$, and $U_3$ contribute to $B_s$ mixing, they only do so at the 1-loop level. 
Correspondingly, the bounds on the leptoquark couplings from $B_s$ mixing allow leptoquark masses as high as several 10's of TeV.
Lower bounds on the leptoquark masses come from direct searches at hadron colliders. All leptoquarks are charged under color 
%necessarily have color charge, 
and 
%they 
can be pair produced through strong interactions in $pp$ collisions. Depending on details of the models, current bounds from direct searches at the LHC are at the level of several 100~GeV to 1~TeV.

There is an additional leptoquark that has been identified as possible explanation of the rare $B$ decay anomalies. With the appropriate couplings, the scalar singlet leptoquark $S_1 = (3,1,-1/3)$ can contribute to $b \to s \ell \ell$ transitions through 1-loop box diagrams~\cite{Bauer:2015knc}. At the tree level, $S_1$ contributes to the neutrino modes $B \to K^{(*)} \nu \bar \nu$, and at the loop level to $B_s$ mixing as well as to $Z \to \mu\mu$. If $S_1$ is responsible for the anomalies, NP effects in these processes are expected close to the current experimental sensitivities. 

Most studies treat leptoquarks in the simplified model approach~\cite{Hiller:2014yaa,Bauer:2015knc,Fajfer:2015ycq,Alonso:2015sja,Becirevic:2016oho,Bhattacharya:2016mcc,Hiller:2017bzc,Chen:2017hir,Crivellin:2017zlb,Kumar:2018kmr}.
Going beyond simplified models, one finds that the leptoquark couplings, 
%that are 
required to explain the anomalies, 
are likely not of Minimal Flavor Violation type~\cite{Aloni:2017ixa}, and constitute new sources of flavor violation beyond the SM Yukawa couplings.
Both scalar and vector leptoquarks could arise for example in composite models~\cite{Gripaios:2014tna,Barbieri:2016las}. The vector leptoquarks could also be the gauge bosons of an enlarged gauge group that is broken not far above the TeV scale~\cite{Das:2016vkr,Bordone:2017bld,DiLuzio:2017vat,Bordone:2018nbg,Blanke:2018sro}.
The scalar singlet leptoquark that contributes to $b \to s \ell \ell$ transitions at the loop level can be identified with the right-handed sbottom in the Minimal Supersymmetric Standard Model with $R$-parity violation~\cite{Das:2017kfo,Earl:2018snx}.

Some leptoquark scenarios are also able to simultaneously address the $b \to s \ell \ell$ anomalies as well as the hints for LFUV in $R({D^{(*)}})$~\cite{Bauer:2015knc,Bhattacharya:2016mcc,Barbieri:2016las,Das:2016vkr,Chen:2017hir,Buttazzo:2017ixm,Bordone:2017bld,DiLuzio:2017vat,Crivellin:2017zlb,Bordone:2018nbg,Blanke:2018sro,Kumar:2018kmr}.
However, many of the models that attempt a simultaneous explanation are strongly constrained by measurements of the $\tau^+\tau^-$ invariant mass spectrum at the LHC~\cite{Faroughy:2016osc}; by existing bounds on $B \to K \nu\nu$ and $B \to K^* \nu\nu$ from BaBar and Belle; existing bounds on LFV tau decays like $\tau \to 3\mu$; from precision measurements of the leptonic couplings of the $Z$ at LEP~\cite{Feruglio:2016gvd,Feruglio:2017rjo,Cornella:2018tfd}, and by lepton universality tests in leptonic tau decays $\tau\to\ell \nu_\tau\bar\nu_\ell$~\cite{Feruglio:2016gvd,Feruglio:2017rjo,Cornella:2018tfd}.



