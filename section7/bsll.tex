In this section we discuss the status of the interpretation of $b\to s$ transitions within the SM and BSM, the prospects from future sensitivities at the LHC, and the complementarity with the corresponding measurements at Belle~II. All current and planned measurements of processes involving $b\to s$ transitions are related to weak decays of $b$-hadrons. At present, most of the data is on $B$ decays, with a small fraction of $B_s$ decays. LHCb has already provided a few measurements on $\Lambda_b$ decays, with data sets that will increase dramatically in future runs, including a significant output on decays of other $b$-hadrons such as $\Omega_b$
or $\Xi_b$~\cite{EoI} (see Sec.~\ref{}). 

Theoretically, the decays of $b$-hadrons are best described within the Weak Effective Theory (WET), where flavor-changing transitions are mediated by ``effective" dimension-six
operators with Wilson coefficients (WCs) that encapsulate all SM weak and heavy NP effects. Thus all observables can be calculated in full generality (model-independently) in terms of these WCs and of QCD hadronic matrix elements. The relevant effective Lagrangian for $b\to s$ transitions at low-energies in the SM is
\begin{equation}
\label{eq:Heff}
  {\cal{H}}_{\rm eff}^{\rm SM}=
  \frac{4 G_F}{\sqrt{2}}\sum_{p=u,c}\lambda_{ps}\left(C_1 \mathcal{O}_1^p+C_2 O_2^p +\sum_{i=3}^{10} C_i  O_i\right),
\end{equation}
with $\lambda_{ps}=V_{pb}^{} V_{p s}^\ast$. The $O_{1,2}^p$,  $O_{3,\ldots,6}$, and $O_{8}$ are the so-called ``current-current'', ``QCD-penguins'' and ``chromo-magnetic'' operators, respectively, and they contribute to the $b\to s\gamma$ and $b\to s\ell\ell$ transitions via an electromagnetic interaction. The $O_{7}$ and $O_{9,10}$ are the electromagnetic penguin and the semileptonic operators, respectively.
The BSM can enter through the WCs $C_1,\ldots,C_{10}$, the chirally-flipped ($b_{L(R)}\to b_{R(L)}$) versions of them, $O_{7,\ldots, 10}'$, or through scalar and tensor semileptonic operators~\cite{Bobeth:2007dw}. Furthermore, in BSM, semileptonic operators can induce lepton-flavor universality violation (LFUV) or charged-lepton flavor violation (CLFV).
For the purpose of this Section, the relevant operators for the interpretation of $b\to s\gamma$ and $b\to s\ell\ell^\prime$ data in terms of short-distances and BSM are
%
\begin{align}
&O_{7\gamma}^{(\prime)}=\frac{e}{16\pi^2}(\bar s\sigma_{\mu\nu}P_{R(L)} b)F^{\mu\nu},\nonumber\\
&O_9^{\ell\ell^\prime(\prime)} =\frac{\alpha_{\rm em}}{4\pi}
(\bar{s} \gamma_{\mu} P_{L(R)} b)(\bar{\ell} \gamma^\mu \ell^\prime)\,,
\,\,
O_{10}^{\ell\ell^\prime(\prime)} =\frac{\alpha_{\rm em}}{4\pi}
(\bar{s} \gamma_{\mu} P_{L(R)} b)( \bar{\ell} \gamma^\mu \gamma_5 \ell^\prime),\label{eq:O10}
\end{align}
where we have added the leptonic flavor labels in the semileptonic operators.

%The semi-leptonic operators with scalar or tensor structure cannot explain the observed deviations, nor can the 4-quark operators if there is lepton-flavor non-universality (however, see~\cite{Jager:2017gal}).

By calculating and measuring a large set of independent observables, one can perform global fits to all the relevant WCs, and comparing with their SM values,
extract information on NP model-independently. After that, the resulting information can be contrasted to NP models to learn about the details of the underlying dynamics. This program has been carried out since the start of the LHC, and has led to the current $b\to s$ anomalies~\cite{1304.6325,1307.5683,1308.1501}.
The future runs at the LHC, including the HL
phase will allow to establish or refine these discoveries. Here we discuss the {\bf status} and {\bf prospects} in this program, from the point of view of 1) the calculation of the observables, 2) the global fits to the data, and 3) the interpretation in terms of models.


\subsubsection{Observables and Hadronic Matrix Elements}

When measuring the WCs of the electromagnetic dipole and semileptonic operators $C^{(\prime)}_{7\gamma}$ and $C^{(\prime)\ell}_{i}$ one looks at
radiative, leptonic and semileptonic inclusive and exclusive decays of $b$-hadrons. 



\subsubsubsection{$B_{q}\to\ell\ell^\prime$}

From the theoretical perspective, the purely leptonic channels, $B_{s}\to \ell^+ \ell'^-$, are the cleanest exclusive decays. Up to QED corrections~\cite{1708.09152}, all QCD effects are contained in a decay constant  which is precisely and reliably computed in the lattice, $f_{B_s}=228.4(3.7)$~MeV. Going beyond this accuracy in $f_{B_s}$ is difficult (see Sec.~11). The current theoretical uncertainty on the muonic $B_s$ branching ratio is at the level of 5\%~\cite{1708.09152} and is dominated by those of the relevant CKM parameters. The latest experimental LHCb measurement has an error of~$\sim 25\%$~\cite{Aaij:2017vad}, while future measurements at the LHC will bring this down to the percent level (see Sec.~\ref{}). 

Beyond the SM, the decay $B_{s}\to \mu^+ \mu^-$ give very strong constraints to the scalar and pseudoscalar operators, and also to $C^\mu_{10}$, which has an impact on the fits and the $b\to s\mu\mu$ anomalies.


\subsubsubsection{$B_q\to M \ell^+\ell^-$}
The current $b\to s$ anomalies are all observed in exclusive $B_q\to M \mu^+\mu^-$ modes, specifically in $B\to K^{(*)}\mu^+\mu^-$
and $B_s\to \phi \mu^+\mu^-$. These are also the modes that dominate the global fits, because of the level of precision that has been achieved in their measurements
(mostly by LHCb, but also Babar, Belle, ATLAS and CMS), and by the huge number of independent observables they lead to, which constrain many independent
combinations of WCs as well as many hadronic parameters. Keeping in mind there is no experimental information related to the polarization of the final-state leptons,
measurable observables arise from the kinematic differential distributions of the final state momenta. These differential distributions are customarily written in the form
of an angular distribution with coefficients that depend on the dilepton invariant mass squared $q^2$. These coefficients, integrated in specific bins of $q^2$,
are the observables that are predicted and measured, and which enter the global fits. In the case of the three-body mode $B\to K\mu^+\mu^-$, there are three
angular coefficients: the differential branching fraction $d{\cal B}/dq^2$, the forward-backward asymmetry $A_{\rm FB}(q^2)$
and the ``flat term" $F_H(q^2)$~\cite{0709.4174}. The kinematic distribution of the four-body decays $B\to V(\to M_1M_2) \ell^+\ell^-$ contains many more
independent angular coefficients, up to 12 in the most general case, called $I_i(q^2)$ or $J_i(q^2)$~\cite{0811.1214,1202.4266}.
Normalizing these by the total differential rate $d\Gamma/dq^2$ and symmetrizing or antisymmetrizing on the two charge-conjugated modes leads to the
observables $S_i$, $A_i$~\cite{0811.1214}.
The angular observables $J_i$ are not always independent~\cite{1005.0571,1202.4266}. In addition, it is convenient to define certain combinations of angular
observables where form factor uncertainties cancel in certain limits, called ``optimized observables". An independent set of observables optimized at low-$q^2$
is given by the $P_i^{(\prime)}$ basis~\cite{1202.4266,1207.2753,1303.5794}. Optimized observables at large-$q^2$ also exist~\cite{1006.5013,1303.5794}.
The whole set of angular observables $P_i^{(\prime)}$ and $S_i,A_i$ in $B^0\to K^{*0}\mu^+\mu^-$ in bins across the whole $q^2$ range have been measured by LHCb~\cite{1304.6325,1308.1707,1512.04442}, and some subset of these observables have also been measured by Babar~\cite{1508.07960},
Belle~\cite{1604.04042}, CMS~\cite{1710.02846} and ATLAS~\cite{ATLAS-CONF-2017-023}. A subset of these observables is also available from LHCb for
$B_s\to \phi \mu^+\mu^-$~\cite{1506.08777}, and also for $B^0\to K^{*0} e^+ e^-$ but only at very low $q^2$~\cite{1501.03038}.


The exclusive semileptonic decays of the form $B\to M \ell^+ \ell^-$ discussed in this section and all the observables that can be defined from them are specified
by transversity amplitudes which depend on two types of hadronic matrix elements: ``local" (form factors) and ``non-local" (see e.g.~\cite{1707.07305}).
Local form factors for $B\to K$, $B\to K^*$ and $B_s\to \phi$ transitions are known at low-$q^2$ from two different versions of light-cone sum rules (LCSRs)~\cite{hep-ph/0611193,1503.05534}, and at large-$q^2$ from Lattice QCD~\cite{1509.06235,1310.3722}. A comparison between both determinations is done by parametrizing
the $q^2$ dependence via the $z$-expansion~\cite{0807.2722}, which is based on the analytic structure of the matrix elements. Future prospects on the theoretical
precision on the calculation of the form factors rely mostly on the improvements from the Lattice QCD side. In the case of vector mesons, however, both determinations
assume the narrow-width limit. A calculation beyond the narrow-width limit is possible within the LCSRs~\cite{1701.01633,1709.00173}, and points towards a correction
of up to $10\%$. There are some prospects on the treatment of hadronic resonances on the lattice, too~\cite{1704.05439}.

Non-local effects are significantly more difficult to approach theoretically~\cite{hep-ph/0106067,1006.4945,1101.5118}, but data-driven methods might be able to eliminate most of the uncertainties.
Some of these methods will be possible and improve significantly with the high statistics collected at LHCb after the upgrade.
They are all based on precise measurements of the $q^2$ spectra, together with a theoretically motivated parametrization of the $q^2$ dependence
of the amplitudes and a theory benchmark that allows to separate short- from long-distance contributions.
The situation is different at low and high $q^2$:\\[2mm]
%
$\blacktriangleright$ At low $q^2$, theory input is based on a light-cone Operator Product Expansion (OPE)
at very low (or negative) $q^2$~\cite{hep-ph/0106067,1006.4945},
an expansion that breaks down at the perturbative $c\bar c$ threshold $q^2\simeq 4m_c^2$.
Parametrizations of the $q^2$ dependence are based on a naive expansion in powers of $q^2$~\cite{1512.07157}, on a dispersion relation~\cite{1006.4945} or on a
$z$-expansion~\cite{1707.07305} very similar to the one for local form factors. The latter two parametrizations are consistent with analyticity and allow to use
extra information, such as data on the $B\to \psi K^*$ with $\psi=\{J/\psi, \psi(2S)\}$. In~\cite{1707.07305} it is shown quantitatively how this information can be used
\emph{a priori} to produce theory predictions for the non-local effect independent of NP, or \emph{a posteriori} to fit all the $B\to \psi K^*$ and $B\to K^*\mu\mu$ spectra up to $q^2=m_{\psi(2S)}^2$ simultaneously to the hadronic parameters and NP. In this last approach, short- and long-distance effects are disentangled by the experimental input from $B\to \psi K^*$, the fixed $q^2$ dependence of the NP contribution, and by the theory constraints at negative $q^2$.
A notable byproduct is the fact that experimental data between the two narrow charmonia are not only used, but very useful.
The prospects for this data-driven approach with the future data from LHCb, including the prospects of doing without theory input altogether,
have been studied in~\cite{1805.06378} with very positive outcome. Thus, high statistics studies of $b\to s\mu\mu$ exclusive transitions at the LHC with fine
$q^2$ binning (or unbinned) will not only probe for NP, but also improve the reliability of non-local effects. While current global fits to different $q^2$ bins show
consistency with the treatment of non-local effects~\cite{1510.04239}, future LHC data will require, and provide, a higher level of control over them.\\[2mm]
%
$\blacktriangleright$ At high $q^2$, the theory input is based on the low-recoil OPE~\cite{hep-ph/0404250,1006.5013,1101.5118}. This method in practice relies
on the fact that resonant effects from ``above-threshold" charmonia will average out to some extent within sufficiently broad $q^2$ bins. Thus only a single bin
in the whole low-$q^2$ region is typically considered in the global fits, but the $q^2$ spectrum can be used to test models for the ``duality violating" contributions
(those that do not average out), and give reliable estimates of these intrinsically non-perturbative effects. Currently these analyses are carried out within the
``Kruger-Sehgal" (naive factorization) approach~\cite{hep-ph/9603237}, which allows to use data on the $R(s)$ ratio in $e^+e^-$ annihilation.
These analyses have been performed in~\cite{1101.5118,1406.0566,1606.00775}. Ref.~\cite{1606.00775} uses all currently available data on $B\to K^*\mu\mu$
at low recoil and finds agreement with the OPE within $\sim 20\%$ in all the bins. Notably, future precision data from the LHC with the
expected fine binning will be essential in refining these data-driven methods and disentangle NP contributions, with the prospects of confirming the
$b\to s\mu\mu$ anomalies in the large $q^2$ region. As in the low-$q^2$ case, combined fits to hadronic parameters and NP are also beneficial~\cite{1606.00775}.\\

Tests of lepton-flavor universality (LFU) are also intrinsically a data-driven approach to hadronic effects. Ratios such as $R_{K^{(*)}}$~\cite{hep-ph/0310219}
have been measured by Babar~\cite{1204.3933}, Belle~\cite{0904.0770}, and LHCb~\cite{1406.6482,1705.05802}, the latter showing a $\sim 2.5\,\sigma$
tension with respect to the precise SM (very close to universal) prediction $R_{K^{(*)}}\simeq 1$.
Much higher precision will come from future data at LHCb~\cite{EoI},
and also from Belle-II which has access to the low- and high-$q^2$ region and can confirm $R_K$ with $5\,\sigma$ with $20\,\text{ab}^{-1}$~\cite{Belle2}. 
But in the presence of LFU violation, predictions are not so precise.
Some ``optimized" observables in this case have also been defined, such as $Q_{4,5}$~\cite{1605.03156}, which have also been measured by Belle~\cite{1612.05014}. 
As these tensions are consistent with the $b\to s\mu\mu$ anomalies if one assumes that NP is present only in the muon modes, one may adopt this assumptions and
use the future precise and fine-binned measurements in $b\to se^+e^-$ exclusive modes to fit to the hadronic parameters directly, as discussed above, providing
model-independent SM predictions for $b\to s\mu^+\mu^-$. These
precise measurements will be performed at the LHC in its HL phase, and constitute again a physics case for NP searches with data-driven determinations of
the SM background. A study of the prospects at LHCb, in the $z$-expansion framework of Ref.~\cite{1707.07305} has been performed in~\cite{1805.06401}
for the $B\to K^*\ell^+\ell^-$ case, with promising results.

Other measurements will be possible at the HL phase with a significant impact on the $b\to s\ell\ell$ anomalies. First, the excellent sensitivity of the
measurements on $B_s$ and $\Lambda_b$ decays will make possible to implement the program outlined above to these modes, too.
In particular, the $\Lambda_b\to \Lambda \mu^+\mu^-$ measurements~\cite{1503.07138} have only been used at low recoil~\cite{1603.02974} and do not seem
to confirm the $C^\mu_{9}$ anomaly. 
For $B_s\to\phi \mu^+\mu^-$, a flavor-tagged time-dependent analysis allows to access independent observables~\cite{1502.05509}, with sensitivity to NP and
different properties. In addition, the study of exclusive $b\to d\mu^+\mu^-$ transitions will reach the current level of precision we have in $b\to s\mu^+\mu^-$.
This on its own will allow to perform analogous global analyses to test $b\to d\mu^+\mu^-$ transitions, but can also be used to study hadronic effects
in $b\to s\mu^+\mu^-$ in the U-spin limit, which constitutes an independent test, albeit with some limitations given by unknown U-spin-breaking effects.



\subsubsubsection{$B_q\to M \nu \bar \nu$}

The second cleanest exclusive $b\to s$ modes are $B\to K^{(*)} \nu \bar \nu$ and relatives~\cite{1409.4557}, which depend only on local form factors (see below).
These modes, however, are very challenging at the LHC. Belle-II, on the other hand, is expected to provide measurements with an uncertainty of about $\sim 10\%$,
(assuming the rates are SM-like)~\cite{Belle2}. These modes do not probe $C^\mu_{9}$ directly either, but a NP contribution to  $C_{9\mu}$ is likely correlated with a NP
contribution to  $C^\nu_{9}$ by $SU(2)_L$ gauge invariance, if the NP is above the EW scale.

\subsubsubsection{Inclusive $B$ decays}

The inclusive observables used in current fits are the branching fractions
${\cal B}(B\to X_s \gamma)$ and ${\cal B}(B\to X_s \ell^+\ell^-)$. These inclusive observables are accessible at $B$-factories and very difficult for the LHC experiments.
Belle-II will improve these measurements, including good measurements of the Forward-Backward asymmetry in $B\to X_s \ell^+\ell^-$, which will also have
an interesting effect in the fits~\cite{Belle2}. In particular, Belle-II measurements of $B\to X_s \mu^+\mu^-$ will be able
to test the LHCb anomalies independently by 2024~\cite{Belle2}. Theoretically, these inclusive rate are dominated by the perturbative
partonic decay of the $b$ quark up to small non-perturbative effects~\cite{1003.5012,1705.10366}, given an appropriate cut on the invariant mass of the hadronic final state. Calculations of both rates have been done to high accuracy~\cite{1503.01789,1503.04849}.

\subsubsection{Model-Independent Fits}

Existing measurements have shown deviations from SM expectations in three
different classes of measurements:
in $B\to K^*\mu^+\mu^-$ angular observables \cite{Aaij:2015oid},
in branching fractions of exclusive $b\to s\mu^+\mu^-$ decays (in particular
$B_s\to\phi\mu^+\mu^-$ \cite{Aaij:2015esa}),
and in $\mu$-$e$ universality (LFU) tests \cite{Aaij:2014ora,Aaij:2017vbb}.
None of the individual measurements is in tension with the SM by more than
$4\sigma$. However, a global significance can be defined in a specific
framework of new physics, such as the model-independent framework provided by the weak effective Hamiltonian.

%The most model-independent framework is the general weak effective Hamiltonian for $b\to s\ell\ell$ transitions. Splitting this effective Hamiltonian in a SM and a new physics part, $\mathcal H_\text{eff} = \mathcal H_\text{eff}^\text{SM} +\mathcal H_\text{eff}^\text{NP}$, the latter contains semi-leptonic operators that could generate a deviation from lepton flavour universality,


Several groups have performed global fits of the Wilson coefficients to existing $b\to s\ell\ell$ data
(see \cite{Altmannshofer:2017fio,Capdevila:2017bsm,Altmannshofer:2017yso,Hurth:2017hxg,Ciuchini:2017mik,Geng:2017svp}
for recent fits).
Three classes of fits can be distinguished: fits to $b\to s\mu\mu$ data only,
fits to $\mu$-$e$ LFU ratios only, and combined fits assuming no new physics
in $b\to see$ transitions. All these fits agree -- up to differences that can be
attributed to different theoretical inputs or to incomplete inclusion
of experimental data -- on two basic conclusions.
First, that there is a discrepancy with $b\to s\mu\mu$ data alone, that could be
explained by a shift in the Wilson coefficient $C_9^\mu$~\cite{1307.5683}, or by an unexpectedly large
hadronic effect that is moreover hard to bring into agreement with non-leptonic
measurements \cite{Bobeth:2017vxj}.
Second, that the new physics explanation of the $b\to s\mu\mu$ anomaly, assuming
it does not affect $b\to see$ transitions, is in perfect agreement with the
deviations from LFU observed in the $\mu/e$ ratios $R_K$ and $R_{K^*}$~\cite{Alonso:2014csa}.
Singling out the Wilson coefficient $C_9^\mu$ in the muonic transition and performing
a global fit to all the data, the log-likelihood ratio between the best-fit
point and the SM hypothesis corresponds to more than six Gaussian standard deviations.

Clearly, future experimental efforts to settle the question whether
these deviations are due to physics beyond the SM are of the utmost
importance. If LFU is indeed violated, LFU ratios like $R_K$ and
$R_{K^*}$ are theoretically extremely clean smoking guns to
establish a deviation from the SM. However, global fits to
all relevant observables will
remain an indispensible tool for several reasons,
\begin{itemize}
 \item if the hints for LFU violation disappear with more statistics, a LFU new physics effect might still hide in the less clean observables,
 \item to identify the {\em nature} of the new physics (and not just its presence), the values of the BSM Wilson coefficients
 have to be determined (and even ratios such as $R_K$ and
$R_{K^*}$ have non-negligible theory uncertainties {\em away}
from the SM limit),
 \item they allow in principle to simultaneously determine poorly known hadronic effects from the data.
\end{itemize}


\begin{figure}
    \centering
    \includegraphics[width=0.46\textwidth]{section7/C9mumu_C10mumu_pp_Run4.pdf}
    \includegraphics[width=0.46\textwidth]{section7/C9mumu_C9ee_pp_Run4.pdf}
    \caption{Current (not filled) and projected global fit results for
    LHCb and Belle-II (inclusive and exclusive decays) in the planes
    of the Wilson coefficients $C_9^\mu$, $C_{10}^\mu$, and $C_9^e$, 
    allowing two of them to vary at a time.
    The central values of the extrapolations correspond to NP scenarios
    with different central values. The contours correspond to $1\sigma$ uncertainties.
    The light gray contours around 
    the Standard Model point (black dot)
    show the hypothetical exclusion power attainable with a combined sensitivity
    of LHCb's 50~fb$^{-1}$ and Belle-II's 50~ab$^{-1}$ datasets.
    Figure taken from~\cite{Albrecht:2017odf}.}
    \label{fig:WC}
\end{figure}

Extrapolations of the sensitivity of global fits
after future LHCb runs and Belle-II,
assuming the current theory technologies,
have been performed in ref.~\cite{Albrecht:2017odf}.
The projections in the planes of the Wilson coefficients $C_9^\mu$, $C_{10}^\mu$
and $C_9^\mu$, $C_{9}^e$, are shown in Figure~\ref{fig:WC}.
These projections can however be affected significantly by
additional qualitative improvements on the theoretical and
the experimental side. For instance,
\begin{itemize}
\item the above projections have not yet taken into account the angular analysis
of $B\to K^*e^+e^-$, which will however play a crucial role in disentangling
NP models;
\item in the presence of LFU violation, 
a simultaneous amplitude analysis of $B\to K^*\mu^+\mu^-$ and 
$B\to K^*e^+e^-$ decays is even more powerful than separate 
analyses~\cite{1805.06401};
\item combined fits to semi-leptonic and non-leptonic decays as well as the
precise dilepton invariant mass spectra will allow to better disentangle
long- and short-distance effects in $B\to K^*\mu^+\mu^-$~\cite{1805.06378};
\item using the full event information would allow an unbinned determination
of the Wilson coefficients~\cite{Hurth:2017sqw}, fully exploiting the experimental
potential. In this case, an important problem to solve is communicating this
full even information to theorists to allow extracting the Wilson coefficients
with different theoretical models for form factors and hadronic contributions.
\end{itemize}


\subsubsection{Models of new physics}

The model independent new physics analysis of the anomalies points to a new physics scale of
% 
\begin{equation}
 \Lambda_\text{NP} = \frac{4\pi}{e} \frac{1}{\sqrt{|V_{tb} V_{ts}^*|}} \frac{1}{\sqrt{|\Delta C_9^\mu|}} \frac{v}{\sqrt{2}} \simeq \frac{35~\text{TeV}}{\sqrt{|\Delta C_9^\mu|}}~,
\end{equation}
%
where $\Delta C_i^\ell$ denotes the NP contribution to $C_i^\ell$.
The actual mass of the new physics degrees of freedom that are responsible for an explanation of the anomalies can be much smaller, if the new physics couplings are smaller than 1 and/or the new physics contributions arise not at the tree level, but at the loop level.

There are many new physics models that can address the observed anomalies. At tree level there are two types of new physics particles that can contribute to $b \to s \ell \ell$ transitions: $Z^\prime$ and leptoquarks (see left and center diagrams in Fig.~\ref{fig:bsll_diagrams}). Also loop level contributions of leptoquarks have been studied extensively in the literature (see right diagram in Fig.~\ref{fig:bsll_diagrams}). Other loop level models have been considered for example in~\cite{Belanger:2015nma,Gripaios:2015gra,Arnan:2016cpy} but will not be discussed here.\\

%%%%%%%%%%%%%%%%%%%%%%%%%%%%%%%%%%%%%%%%%%%%%%%%%%%
\begin{figure}[tb]
\begin{center} 
\includegraphics[width=0.25\textwidth]{section7/diagram_Zprime.pdf} ~~~~~~~~
\includegraphics[width=0.25\textwidth]{section7/diagram_LQ.pdf} ~~~~~~~~
\includegraphics[width=0.29\textwidth]{section7/diagram_loop.pdf}
\caption{Examples of new physics contributions to $b \to s \mu \mu$ transitions. Left: tree level $Z^\prime$ contribution. Center: tree level leptoquark contribution. Right: 1-loop leptoquark contribution.}
\label{fig:bsll_diagrams}
\end{center}
\end{figure}
%%%%%%%%%%%%%%%%%%%%%%%%%%%%%%%%%%%%%%%%%%%%%%%%%%%

\noindent $\blacktriangleright$ \underline{\bf \boldmath $Z^\prime$ models}:
Models with a $Z^\prime$ that has flavour violating couplings to quarks and that couples non-universally to leptons can explain the observed anomalies in $b \to s \ell \ell$. In a ``simplified model'' approach, treating the $Z^\prime$ mass and its coupling to SM fermions as completely free parameters, irreducible constraints arise from $B_s$ mixing, neutrino trident production and LEP bounds on four-lepton contact interactions.

Combining these constraints one finds the following maximal values for the $Z^\prime$ contributions to the relevant Wilson coefficients $\Delta C_{9,10}^{e,\mu}$ in the case of LH lepton couplings~(\ref{eq:C910_LH}) and of vectorial lepton couplings~(\ref{eq:C910_vec})~\cite{Altmannshofer:2014rta}
%
\begin{eqnarray} \label{eq:C910_LH}
 && |\Delta C_9^\mu| = |\Delta C_{10}^\mu| \lesssim 5.4 ~,\quad |\Delta C_9^e| = |\Delta C_{10}^e| \lesssim 0.64 ~, \\  \label{eq:C910_vec}
 && |\Delta C_9^\mu| \lesssim 9.3 ~,\quad |\Delta C_9^e| \lesssim 0.72 ~,\quad \Delta C_{10}^\mu = \Delta C_{10}^e = 0 ~.
\end{eqnarray}
%
This shows that $Z^\prime$ models can comfortably explain the anomalies in $R_K$ and $R_{K^*}$ as well as the other anomalies in $b \to s \mu \mu$ transitions by affecting the muonic modes. Adressing $R_K$ and $R_{K^*}$ through a $Z^\prime$ that affects the electronic modes is only possible in parameter regions close to the current bounds from $B_s$ meson mixing and bounds on four-lepton contact interactions. Constraints from $B_s$ mixing will likely become stronger at the time scale of the HE/HL LHC due to improved lattice predictions of hadronic matrix elements.
The $Z^\prime$ mass can be at most several TeV, otherwise an explanation of the anomalies requires couplings to leptons that are non-perturbatively large. If the $Z^\prime$ has very week couplings to light quarks and electrons, its production cross section at colliders is tiny. Therefore the $Z^\prime$ could in principle be light, for example at the electro-weak scale, or in certain models even much lighter~\cite{Datta:2017pfz,Sala:2017ihs,Altmannshofer:2017bsz}. 

Many explicit $Z^\prime$ models have been put forward as possible explanations of the anomalies.
One popular class of models is based on gauging the difference of muon-number and tau-number, $L_\mu - L_\tau$. Once physics is introduced that generates flavor violating couplings of the $Z'$ to quarks, the observed tensions in $b \to s \ell \ell$ decays can be explained~\cite{Altmannshofer:2014cfa,Crivellin:2015mga,Altmannshofer:2015mqa,Fuyuto:2015gmk,Altmannshofer:2016jzy,Baek:2017sew,Chen:2017usq}. $L_\mu - L_\tau$ models predict
%
\begin{equation}
 \Delta C_9^e = 0 ~,\quad \Delta C_9^\mu = - \Delta C_9^\tau ~,\quad \Delta C_{10}^e = \Delta C_{10}^\mu = \Delta C_{10}^\tau = 0~.
\end{equation}
%
Besides $L_\mu - L_\tau$, various other combinations of gauged flavor symmetries have been used to construct $Z'$ models that can address the $b \to s \ell \ell$ anomalies~\cite{Crivellin:2015lwa,Celis:2015ara,Falkowski:2015zwa,Boucenna:2016wpr,Boucenna:2016qad,Celis:2016ayl,Crivellin:2016ejn,Alonso:2017bff,Ellis:2017nrp,Alonso:2017uky,Bonilla:2017lsq,Babu:2017olk,Bian:2017rpg,Tang:2017gkz,Cline:2017ihf}.
Also scenarios where the $Z'$ couples to both quarks and leptons indirectly through the mixing with heavy vectorlike fermions have been considered~\cite{Sierra:2015fma,King:2017anf,Fuyuto:2017sys}.

In models with partial compositeness the $Z'$ can be identified with a heavy neutral spin-1 resonance of the composite sector, typically denoted as $\rho$. Such a resonance generically features flavor violating couplings to quarks. A large degree of compositeness of the left-handed muons is required to explain the $B$ decay anomalies~\cite{Niehoff:2015bfa,Carmona:2015ena,Megias:2016bde,Carmona:2017fsn,Megias:2017ove,Sannino:2017utc}.
The generic expectation in models with partial compositeness is that the $\rho$ couplings are strongest to SM fermions of the third generation, reflecting the mass hierarchy of the SM fermions that is related to their degree of compositeness.
If one assumes the dominance of couplings to left-handed leptons, these models therefore suggest the pattern 
%
\begin{equation}
 \Delta C_9^e \simeq -\Delta C_{10}^e \ll \Delta C_9^\mu \simeq -\Delta C_{10}^\mu \ll \Delta C_9^\tau \simeq -\Delta C_{10}^\tau ~.
\end{equation}
%
Models in which the $Z^\prime$ is part of a $SU(2)_L$ triplet have been suggested as an simultaneous explanation of the $b \to s \ell \ell$ anomalies and hints for lepton flavor universality violation in semileptonic charged current decays $R_D$ and $R_{D^*}$~\cite{Bhattacharya:2014wla,Greljo:2015mma,Bhattacharya:2016mcc,Boucenna:2016wpr,Boucenna:2016qad,Buttazzo:2017ixm,Kumar:2018kmr}. 

The different $Z'$ models predict different patterns of new physics effects in $b \to s \ell \ell$ and the related $b \to s \nu \nu$ transitions. Future measurements of these transitions at LHCb and Belle II will therefore allow us to narrow down viable $Z^\prime$ models.
For example, the $Z^\prime$ models based on the gauged $L_\mu - L_\tau$ symmetry predict effects in the semileptonic $b \to s \mu^+ \mu^-$ and $b \to s \tau^+ \tau^-$ transitions of opposite sign, while $b \to s e^+ e^-$ transitions remain SM-like.
Also the purely leptonic $B_s \to \mu^+\mu^-$ and $B_s \to \tau^+\tau^-$ decays as well as the neutrino modes $B \to K^{(*)} \nu \bar \nu$ are predicted to be SM-like in these models~\cite{Altmannshofer:2014cfa}.

A markedly different pattern arises in $Z^\prime$ scenarios that are based on dominant couplings to left-handed fermions of the third generation. In those models the $b \to s \tau^+ \tau^-$ and $B \to K^{(*)} \nu \bar \nu$ rates are typically enhanced compared to SM predictions by a factor of few. The $B_s \to \mu^+\mu^-$ rate is predicted to be suppressed by approximately $25\%$ compared to the SM prediction. Finally, in contrast to the $L_\mu - L_\tau$ models, rare lepton flavor violating decays like $B \to K^{(*)} \tau \mu$ are predicted at levels of $O(10^{-8})$~\cite{Glashow:2014iga} which might be in reach at the HE/HL LHC.\\


\noindent $\blacktriangleright$ \underline{\bf Leptoquark models:}
There are seven quantum number assignments for leptoquarks that allow tree-level couplings to down-type quarks and charged leptons of the SM.
These are~\cite{Dorsner:2016wpm} $S_3 = (\bar 3,3,1/3)$, $R_2 = (3,2,7/6)$, $\tilde R_2 = (3,2,1/6)$, $\tilde S_1 = (\bar 3,1,4/3)$, $U_3 = (3,3,2/3)$, $V_2 = (\bar 3,2,5/3)$ and $U_1 = (3,1,-1/3)$.
Among those, the couplings of $R_2$, $\tilde R_2$, $\tilde S_1$, and $V_2$ necessarily involve right-handed currents and therefore cannot explain the anomalies.
Thus one is left with the triplet scalar $S_3$, the singlet vector $U_1$, and the triplet vector $U_3$. They all contribute at tree level to the operator $(\bar s \gamma_\alpha P_L b)(\bar \mu \gamma^\alpha P_L \mu)$ and can explain the $b \to s \ell \ell$ anomalies.

In a simplified model approach the constraints on the leptoquark models are very weak.
While the leptoquarks $S_3$, $U_1$, and $U_3$ contribute to $B_s$ mixing, they only do so at the 1-loop level. 
Correspondingly, the bounds on the leptoquark couplings from $B_s$ mixing allow leptoquark masses as high as several 10's of TeV.
Lower bounds on the leptoquark masses come from direct searches at hadron colliders. As leptoquarks necessarily have color charge, they can be pair produced through the strong interactions. Depending on details of the models, current bounds from direct searches at the LHC are at the level of several 100~GeV to 1~TeV.

There is an additional leptoquark that has been identified as possible explanation of the rare $B$ decay anomalies. With the appropriate couplings, the scalar singlet leptoquark $S_1 = (3,1,-1/3)$ can contribute to $b \to s \ell \ell$ transitions through 1-loop box diagrams~\cite{Bauer:2015knc}. At the tree level, $S_1$ contributes to the neutrino modes $B \to K^{(*)} \nu \bar \nu$, and at the loop level to $B_s$ mixing as well as $Z \to \mu\mu$. If $S_1$ is responsible for the anomalies, new physics effects in these processes are expected close to the current experimental sensitivities. 

Most studies treat leptoquarks in the simplified model approach~\cite{Hiller:2014yaa,Bauer:2015knc,Fajfer:2015ycq,Alonso:2015sja,Becirevic:2016oho,Bhattacharya:2016mcc,Hiller:2017bzc,Chen:2017hir,Crivellin:2017zlb,Kumar:2018kmr}.
Going beyond the simplified model approach one finds that the leptoquark couplings that are required to explain the anomalies go likely beyond the principle of Minimal Flavor Violation~\cite{Aloni:2017ixa} and constitute new sources of flavor violation beyond the SM Yukawa couplings.
Both scalar and vector leptoquarks could arise for example in composite models~\cite{Gripaios:2014tna,Barbieri:2016las}. The vector leptoquarks could also be the gauge bosons of an enlarged gauge group that is broken not far above the TeV scale~\cite{Das:2016vkr,Bordone:2017bld,DiLuzio:2017vat,Bordone:2018nbg,Blanke:2018sro}.
The scalar singlet leptoquark that contributes to $b \to s \ell \ell$ transitions at the loop level can be identified with the right-handed sbottom in the Minimal Supersymmetric Standard Model with R-parity violation~\cite{Das:2017kfo,Earl:2018snx}.

Some leptoquark scenarios are also able to simultaneously address the $b \to s \ell \ell$ anomalies as well as the hints for lepton flavor universality violation in $R_{D^{(*)}}$~\cite{Bauer:2015knc,Bhattacharya:2016mcc,Barbieri:2016las,Das:2016vkr,Chen:2017hir,Buttazzo:2017ixm,Bordone:2017bld,DiLuzio:2017vat,Crivellin:2017zlb,Bordone:2018nbg,Blanke:2018sro,Kumar:2018kmr}.
However, many of the models that attempt a simultaneous explanation are strongly constrained by measurements of the $\tau^+\tau^-$ invariant mass spectrum at the LHC~\cite{Faroughy:2016osc}, by existing bounds on $B \to K \nu\nu$ and $B \to K^* \nu\nu$ from BaBar and Belle, existing bounds on lepton flavor violating tau decays like $\tau \to 3\mu$, from precision measurements of the leptonic couplings of the $Z$ at LEP~\cite{Feruglio:2016gvd,Feruglio:2017rjo,Cornella:2018tfd}, and by lepton universality tests in leptonic tau decays $\tau\to\ell \nu_\tau\bar\nu_\ell$~\cite{Feruglio:2016gvd,Feruglio:2017rjo,Cornella:2018tfd}.



