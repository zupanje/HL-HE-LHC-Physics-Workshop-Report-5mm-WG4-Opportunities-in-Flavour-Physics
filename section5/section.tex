%cms don't remove the folling lines, and edit the defintion of \main if needed
\documentclass[../report.tex]{subfiles}
\providecommand{\main}{..}
\IfEq{\jobname}{\currfilebase}{\AtEndDocument{\biblio}}{}
% until here
%\input{definitions_notes}
\renewcommand{\em}{\it} 
%References   
\newcommand{\fref}[1]{Fig.~\ref{fig:#1}} 
\newcommand{\eref}[1]{Eq.~\eqref{eq:#1}} 
\newcommand{\erefn}[1]{ (\ref{eq:#1})}
\newcommand{\erefs}[2]{Eqs.~(\ref{eq:#1}) - (\ref{eq:#2}) } 
\newcommand{\aref}[1]{Appendix~\ref{app:#1}}
\newcommand{\sref}[1]{Section~\ref{sec:#1}}
\newcommand{\cref}[1]{Chapter~\ref{ch:.#1}}
\newcommand{\tref}[1]{Table~\ref{tab:#1}}
  
 
%Equation enviroment
\newcommand{\nn}{\nonumber \\}  
\newcommand{\nnl}{\nonumber \\}  
\newcommand{\nl}{& \nonumber \\ &}
\newcommand{\bnl}{\right .  \nonumber \\  \left .}
\newcommand{\dbnl}{\right .\right . & \nonumber \\ & \left .\left .}

%Begin-end
\newcommand{\beq}{\begin{equation}} 
\newcommand{\eeq}{\end{equation}} 
\newcommand{\ba}{\begin{array}}  
\newcommand{\ea}{\end{array}} 
\newcommand{\bea}{\begin{eqnarray}}  
\newcommand{\eea}{\end{eqnarray} }  
\newcommand{\be}{\begin{eqnarray}}  
\newcommand{\ee}{\end{eqnarray} }  
\newcommand{\bal}{\begin{align}}
\newcommand{\eal}{\end{align}}   
\newcommand{\bi}{\begin{itemize}}  
\newcommand{\ei}{\end{itemize}}  
\newcommand{\ben}{\begin{enumerate}}  
\newcommand{\een}{\end{enumerate}}  
\newcommand{\bc}{\begin{center}}
\newcommand{\ec}{\end{center}} 
\newcommand{\bt}{\begin{table}}
%\newcommand{\et}{\end{table}}  
\newcommand{\btb}{\begin{tabular}}
\newcommand{\etb}{\end{tabular}}  
\newcommand{\bvec}{\left ( \ba{c}}
\newcommand{\evec}{\ea \right )}


% caligraphic fonts 
\newcommand{\cO}{{\mathcal O}} 
\newcommand{\co}{{\mathcal O}} 
\newcommand{\cL}{{\mathcal L}} 
\newcommand{\cl}{{\mathcal L}} 
\newcommand{\cM}{{\mathcal M}}   


%Physics
\newcommand{\const}{\mathrm{const}}


%\newcommand{\ev}{ \mathrm{eV}}
%\newcommand{\kev}{\mathrm{keV}}
%\newcommand{\mev}{\mathrm{MeV}}
%\newcommand{\gev}{\mathrm{GeV}}
%\newcommand{\tev}{\mathrm{TeV}}

\newcommand{\mpl}{M_{\mathrm Pl}}

\def\mgut{\, M_{\rm GUT}}
\def\tgut{\, t_{\rm GUT}}
\def\mpl{\, M_{\rm Pl}}
\def\mkk{\, M_{\rm KK}}
\newcommand{\msusy}{M_{\rm soft}}

\newcommand{\dslash}[1]{#1 \! \! \! {\bf /}}
\newcommand{\ddslash}[1]{#1 \! \! \! \!  {\bf /}}

\def\ads{AdS$_5$\,}
\def\adse{AdS$_5$}
%\def\ds{${\rm dS}_4\,$}
\def\intdk{\int {d^4 k \over (2 \pi)^4}} 

\def\ra{\rangle}
\def\la{\langle}  

%Math
\def\sgn{{\rm sgn}}
\def\pa{\partial}  
\newcommand{\dlr}{\overleftrightarrow{\partial}}
\newcommand{\Dlr}{\overleftrightarrow{D}}
\newcommand{\re}{{\mathrm{Re}} \,}
\newcommand{\im}{{\mathrm{Im}} \,}
\newcommand{\tr}{\mathrm T \mathrm r}  

\newcommand{\Ra}{\Rightarrow}
\newcommand{\lra}{\leftrightarrow}
\newcommand{\llra}{\longleftrightarrow}

\newcommand\simlt{\stackrel{<}{{}_\sim}}
\newcommand\simgt{\stackrel{>}{{}_\sim}}   
\newcommand{\zt}{$\mathbb Z_2$ } 


\newcommand{\ha}{{\hat a}}
\newcommand{\hab}{{\hat b}}
\newcommand{\hac}{{\hat c}} 

\newcommand{\ti}{\tilde}  
\def\hc{{\rm h.c.}} 
%\def\ZZ{\mathbb{Z}}    
\def\ov{\overline}  
  
%\newcommand{\eps}{\epsilon}

\newcommand{\lz}{\lambda_z}
%\newcommand{\gz}{g_{1,z}}
%\newcommand{\kg}{\kappa_\gamma}
\newcommand{\dgz}{\delta g_{1,z}}
\newcommand{\dkg}{\delta \kappa_\gamma}


\newcommand{\eL}{\epsilon_L}
\newcommand{\eR}{\epsilon_R}
\newcommand{\eS}{\epsilon_S}
\newcommand{\eP}{\epsilon_P}
\newcommand{\eT}{\epsilon_T}
%\newcommand{\ket}[1]{| #1 \rangle}
%\newcommand{\bra}[1]{\langle #1 |}



%%Color shortcuts$$$$$$
\def\cog{\color{OliveGreen}}
\def\cor{\color{Red}}
\def\copu{\color{purple}}
\def\coro{\color{RedOrange}}
\def\coma{\color{Maroon}}
\def\cob{\color{Blue}}
\def\cobr{\color{Brown}}
\def\cobl{\color{Black}}
\def\cost{\color{WildStrawberry}}

\begin{document}

\section{Tau leptons}
{\bf Theory Authors:  Vincenzo Cirigliano, M. Gonzalez-Alonso, Adam Falkowski, Emilie Passemar}

%\subsection{Introduction}
%
The physics of the tau lepton is an exceptionally broad topic, both experimentally and theoretically, and a large variety of processes involving taus have been used in the last decades to learn about fundamental physics~\cite{Pich:2013lsa}. 
In some cases the results obtained in $e^+e^-$ machines~\cite{Schael:2005am} are very hard to improve in the LHC environment, due to large backgrounds, even if the number of taus produced is much larger. This is the case, {\it e.g.}, of the study of the basic tau properties (mass, lifetime, etc.) or the standard tau decay channels. However, there are also many processes involving taus that are relevant for HL/HE-LHC. We focus here on those that are more directly connected to flavour and that are not covered by other sections or Working Groups. This is the reason why we do not discuss for instance $h \to \tau \tau$, $h\to\tau\ell$, or heavy meson decays to taus.

We assume in the rest of this chapter that the New Physics  (NP) scale relevant for the processes under consideration is much heavier than the energy scales probed by the  LHC.
In consequence, we can use the so-called SMEFT framework for our discussion~\cite{Buchmuller:1985jz}. We first discuss lepton-flavor-conserving observables, where the SM contribution has to be calculated with some accuracy, and then lepton-flavor-violating processes where the challenge is purely experimental, since the SM contribution is completely negligible. As a result, the NP scales that are probed are much higher in the latter case.

\subsection{Lepton-flavour-conserving processes}

First of all, let us note that the collider phenomenology is usually very different for the so-called vertex corrections and contact interactions. In the former case, NP contributions mimic the structure of the SM gauge couplings $Z\tau\tau$ and $W\tau\nu$, and its study requires in most cases very precise measurements, which are challenging in a hadron collider. On the other hand, four-fermion contact interactions (generated for example by the tree-level exchange of a heavy mediator) give a contribution to high-energy observables qualitatively different from the SM one. In fact they give a contribution that grows with the energy of the partonic process, and thus one does not require so much precision in the measurements~\cite{Cirigliano:2012ab,deBlas:2013qqa}. For the contact interactions, we focus on flavor-diagonal couplings involving first-generation quarks and third-generation leptons.

\subsubsection{High-energy tails}
%
LHC measurements of differential distributions in the Drell-Yan lepton production can be sensitive to new effective interactions between leptons and quarks~\cite{Cirigliano:2012ab,deBlas:2013qqa}. 
As an example, consider the following dimension-6 interaction added to the SM Lagrangian:
\begin{equation}
\label{eq:olq}
{\cal L} \supset {C_{lq}^{(3)} \over \Lambda^2} (\bar L_3 \gamma_\mu  \sigma^i L_3)  (Q_1 \gamma^\mu \sigma^i Q_1), 
\end{equation}  
where $\sigma^i$ are the Pauli matrices, $L_3 = (\nu^\tau_L,\tau_L)$ is the 3rd generation lepton doublet, and $Q_1 = (u_L,d_L)$ is the 1st generation quark doublet. This interaction term can be interpreted as an effective field theory (EFT) description of a more fundamental theory, for example one with an SU(2) triplet of vector bosons with masses $m_V = \Lambda \gg v$ and coupling $g_{V} \approx  2 |C_{lq}^{(3)}|^{1/2}$ to $L_3$ and $Q_1$.  
Even when $m_V$ is too heavy to be directly produced at the LHC, the effective interaction in Eq.~(\ref{eq:olq}) may produce observable effects, thus providing an important piece of information about the theory completing the SM. 

At the LHC, contact interactions between quarks and 3rd generation leptons can be probed via $p p \to \tau^+ \tau^-$ and $p p \to \tau \nu$~\cite{Faroughy:2016osc,Cirigliano:2018dyk}. 
Here we focus on the latter process, taking as a template the existing Run-2 ATLAS analysis in Ref.~\cite{Aaboud:2018vgh}. 
That analysis relies on the differential distribution in the transverse mass of the $\tau \nu$ pairs defined as 
$m_T =\sqrt{ 2 p_{T} E_T (1-\cos \phi) }$, where $\vec p_{T}$  is the transverse momenta of the (hadronic) tau candidate,  $\vec E_T$ is the missing transverse momentum, and $\phi$ is the azimuthal angle between the two. 
The effects of the interaction in Eq.~(\ref{eq:olq}) are most pronounced at higher $m_T$.
To estimate the current and future sensitivity we pick $m_T > 1.8$~TeV as our signal region (a more elaborate analysis with a wider and binned $m_T$ range would lead to stronger bounds~\cite{Cirigliano:2018dyk}).
Using the {\tt Madgraph}\cite{Alwall:2014hca}/{\tt Pythia~8}~\cite{Sjostrand:2014zea}/{\tt Delphes} \cite{deFavereau:2013fsa} simulation chain and imposing the selection cuts we can estimate the expected number of events in the presence of the interaction in Eq.~(\ref{eq:olq}): 
\begin{equation}
N_{p p \to \tau \tau}^{m_T > 1.8~\tev} \approx  \left ({ {\cal L} \over 36.1~{\rm fb}^{-1} } \right )
 \left [ 0.36  + 4.0 \times 10^3 {C_{lq}^{(3)} v^2 \over \Lambda^2} + 3.6 \times 10^5  \left ( {C_{lq}^{(3)} v^2 \over \Lambda^2}  \right )^2 \right ], 
\end{equation}
 where $ {\cal L}$ is the integrated luminosity, $v \approx 246$~GeV, and we will ignore the theoretical error of this estimation. 
The ATLAS collaboration observed $0$ events in  ${\cal L} = 36.1~{\rm fb}^{-1}$~\cite{Aaboud:2018vgh},  from which  one can derive the 68\%~CL limit on the Wilson coefficient in Eq.~(\ref{eq:olq}):  $- (5.3~\tev)^{-2} \leq  C_{lq}^{(3)} / \Lambda^2 \leq (7.7~\tev)^{-2}$.  
%$- {1 \over (5.3~\tev)^2} \leq  {C_{lq}^{(3)} \over \Lambda^2} \leq {1 \over (7.7~\tev)^2}$.  
For the HL-LHC with ${\cal L} = 3000~{\rm fb}^{-1}$,  assuming the observed number of events will be exactly equal to the SM prediction, the expected limit becomes
\beq
- {1 \over (17.5~\tev)^2} \leq  {C_{lq}^{(3)} \over \Lambda^2} \leq {1 \over (20.5~\tev)^2} \qquad @~68\%~{\rm CL}. 
\eeq 
Compared to the present LHC bound, this represents an $\cO(10)$ stronger bound on $C_{lq}^{(3)}/\Lambda^2$, or $\cO(3)$ times improvement in the reach for the mass scale of new physics. 
Assuming maximally strongly coupled new physics, $g_V \sim 4 \pi$, the HL-LHC can probe particles even as heavy as $m_V \sim 100$~TeV. 

There are 3 more independent dimension-6 operators that can be probed by the $p p \to \tau \nu$ process: 
 $O_{l e d q}  =  (\bar L_3 \tau_R)  (\bar d_R  Q_1)$,  $O_{l e q u} =  (\bar L_3 \tau_R) (Q_1 u_R)$, and  
 $O^{(3)}_{l e q u}  =   (\bar L_3  \sigma_{\mu \nu} \tau_R)  (\bar Q_1 \sigma_{\mu \nu} u_R)$~\cite{Cirigliano:2012ab,Grzadkowski:2010es}. 
We estimate that the HL-LHC will be sensitive to $C / \Lambda^2 \sim (10~\tev)^{-2}$ through a one-bin analysis like the one presented above. 
The slightly smaller sensitivity than for the operator in Eq.~(\ref{eq:olq}) follows from the fact that they do not interfere with the SM amplitudes, and thus their effect enters only at the quadratic level in $C/\Lambda^2$.

We expect that our simple analysis provides a good qualitative estimate of the HL-LHC reach.  
However, a more sophisticated analysis using the full information about the $m_T$ distribution (such as in Ref.~\cite{Cirigliano:2018dyk}), and possibly also about the $\tau$ polarization, should lead to a further $\cO(1)$ increase of sensitivity. 
Ideally, the optimal analysis should be able to distinguish between the different dimension-6 operators (except between $O_{l e d q}$ and  $O_{l e q u}$, which in fact give exactly the same contribution to the $p p \to \tau \nu$ cross section), and provide constraints on the Wilson coefficients in the situation when all the independent operators are simultaneously present.  
Furthermore, the process $p p \to \tau^+ \tau^-$ probes a large set of dimension-6 operators: 
$O_{l q} =  (\bar L_3 \gamma_\mu  L_3)  (Q_1 \gamma^\mu Q_1)$, 
$O_{l u} =  (\bar L_3 \gamma_\mu  L_3)    (\bar u_R \gamma^\mu u_R)$,   
$O_{l d} =  (\bar L_3 \gamma_\mu  L_3)    (\bar d_R \gamma^\mu d_R)$,   
$O_{e q} =  (\bar \tau_R \gamma_\mu  \tau_R)    (\bar Q_1 \gamma^\mu Q_1)$,    
$O_{e u} =  (\bar \tau_R \gamma_\mu  \tau_R)  (\bar u_R \gamma^\mu u_R)$, and
$O_{e d} =  (\bar \tau_R \gamma_\mu  \tau_R)$  $(\bar d_R \gamma^\mu d_R)$, in addition to the operators discussed above. 
We expect comparable sensitivity of  $p p \to \tau^+ \tau^-$ to $C/\Lambda^2$ as in the case of $p p \to \tau \nu$~\cite{Cirigliano:2012ab}. 
It is unlikely that the LHC alone can discriminate between all these operators; to this end combination with low energy precision measurements of $\tau$ decays will be necessary. 
Finally, we mention that $\tau \tau$ and $\tau \nu$ production also probes analogous operators with heavier ($s$, $c$, $b$) quarks instead of the 1st generation ones \cite{Faroughy:2016osc,Altmannshofer:2017poe}. 
The presence of such operators in the Lagrangian can be motivated by the anomalies observed by BaBar and LHCb in the $B \to D^{(*)} \tau \nu$ decays~\cite{Lees:2013uzd,Huschle:2015rga,Sato:2016svk}.  
 

\subsubsection{Beyond tails: lepton flavour universality}
%
It has been recently pointed out that there is an interesting complementarity between the above-presented high-energy searches and low-energy precision studies with hadronic tau decays~\cite{Cirigliano:2018dyk}. The latter are equally sensitive to vertex and contact interactions, and thus they effectively become model-independent probes of LFU violations in the $W\tau\nu$ vertex once the strong LHC bounds described above are taken into account. This is interesting because the only low-energy model-independent measurement of this effect, which was carried out at LEP2, found a $\sim2\sigma$ tension with LFU~\cite{Patrignani:2016xqp,Filipuzzi:2012mg}. Including also hadronic tau decays and LHC bounds on contact interactions, the result is improved by a factor of two, but the (dis)agreement with LFU remains at the $\sim2\sigma$ level: $\delta g_L^{W\tau}-\delta g_L^{We}=0.0134(74)$~\cite{Cirigliano:2018dyk}.

Finally, it is interesting to mention the possibility of accessing LFU violations in the vertex corrections at the (HL/HE) LHC. In the ratio of $W\to\tau\nu$ and $W\to\ell\nu$ many experimental and QCD uncertainties  cancel, which make a per-cent level extraction a bit less complicated. In fact, past studies carried out by the D0 collaboration~\cite{Abbott:1999pk} showed that such precision is indeed possible in a collider environment.


%%%%%%%%%%%%%%%%%%%%%%
\subsection{Lepton Flavour Violation}
\label{LFV}
%%%%%%%%%%%%%%%%%%%%%%%
Lepton Flavour Violating (LFV)   processes  involving charged leptons  are  very interesting because 
their observation would be a clear indication of physics beyond the Standard Model. 
While lepton family number is an accidental symmetry of the Standard Model, 
we know it must somehow be broken to account for neutrino masses and mixing. 
If the only low-energy manifestation of LFV is  in neutrino masses and mixing (this corresponds to 
a very high scale for LFV),  then charged LFV amplitude are suppressed by the GIM mechanism 
and the predicted rates are un-observably  small 
(e.g. $\mathcal B  (\mu \to e\gamma) \sim 10^{-52}$ and $\mathcal B (\tau \to \mu\gamma) \sim 10^{-45}$~\cite{Cheng:1977nv, Lee:1977tib}). 
However, if the breaking of the lepton family symmetry happens at a scale not too much higher than 
the electroweak symmetry breaking scale, as borne out in many new physics scenarios,  
then one can expect charged LFV BRs  quite close to existing limits, 
and therefore within reach of ongoing searches\footnote{In fact in some cases experimental limits are already excluding regions of parameter space 
in specific weak scale new physics models.  While less severe than in the quark sector,  one has a ``flavour problem" in the lepton sector as well.}.
A rich literature exists on this topic,  including studies in  supersymmetric extensions of the 
Standard Model,  little Higgs models, low-scale seesaw models, 
leptoquark models,   $Z'$ models,  left-right symmetric models,  and extended Higgs models. 
For  recent reviews on both theoretical and experimental aspects we refer the reader to Refs.~\cite{Calibbi:2017uvl,Bernstein:2013hba}. 
Current limits on BRs in  $\mu$-$e$ transitions are at the level of  $10^{-13}$  
(e.g.  $\mathcal B(\mu^+ \to e^+ \gamma) < 4.2
\times 10^{-13}$ (90\% CL) \cite{TheMEG:2016wtm}), while  
$\tau$-$\mu$ and $\tau$-$e$ BRs are bound at the $10^{-8}$ level~~\cite{Amhis:2016xyh}. 
%
As discussed later,  improvements in LFV $\tau$ decays
 will offer the opportunity to explore  
(i) correlations with  $\mu$-$e$ transitions, probing underlying sources of flavour breaking; 
(ii) correlations among different LFV $\tau$ decays, probing the nature of the underlying 
mechanism. 






%%%%%%%%%%%%%%%%%%
\subsubsection{Lepton Flavour Violation in $\tau$ decays}
\label{LFVTAU}
%%%%%%%%%%%%%%%%%
Tau decays offer a  rich  landscape to search for CLFV. 
The  $\tau$  lepton is heavy enough to decay into hadrons and up to now 48 LFV modes have been bounded at the level of $10^{-8}$~\cite{Amhis:2016xyh}
as can be seen in figure~\ref{fig:Tau_LFV}.  
%{\color{red} [Add figure]}
%%%%%%%%%%%%%%%%%%%%%%%%%%
\begin{figure}[h!!]
\begin{center}
\includegraphics[width=1.\textwidth]{TauLFV_UL}
\caption{\it Bounds on Tau Lepton Flavour Branching Ratios from CLEO, BaBar, Belle and projections from Belle II~\cite{Kou:2018nap}. 
Data from the existing experiments are compiled by HFLAV~\cite{Amhis:2016xyh}; projections of the Belle II bounds are performed by the Belle II collaboration assuming 50 ab$^{-1}$ luminosity~\cite{Kou:2018nap}.}
\label{fig:Tau_LFV}
\end{center}
\end{figure}
%%%%%%%%%%%%%%%%%%%%%%%
The $B$ factories BaBar and Belle have improved by more than one order of magnitude
~\cite{Aubert:2006cz,Aubert:2007kx,Aubert:2009ag,Aubert:2009ys,Aubert:2009ap,Lees:2010ez,Miyazaki:2005ng,Hayasaka:2007vc,Miyazaki:2007jp,
Miyazaki:2008mw,Miyazaki:2010qb,Hayasaka:2010np,Miyazaki:2011xe} 
the previous CLEO bounds~\cite{Bowcock:1989mq,Bonvicini:1997bw,Chen:2002ug} on a significant 
number of modes. 
Some modes like $\tau \to \ell \omega$ have also been bounded for the first time~\cite{Aubert:2007kx}. 
As can be seen on figure~\ref{fig:Tau_LFV}, 
a similar improvement is expected with Belle-II~\cite{Kou:2018nap}. 

The limits obtained for the $\tau \to 3\mu$ channel are shown explicitly in Table~\ref{tab:tauto3mu} because this channel is particularly important for our discussion. We find in the table the strongest limits from the B-factories, the competitive limit from LHCb~\cite{Aaij:2014azz}, and the recent measurement by ATLAS~\cite{Aad:2016wce}. We also show in the Table the expected limit from the Belle II experiment at the SuperKEKB collider, which will improve current limits by almost two orders of magnitude~\cite{Kou:2018nap}. Finally, the Table summarizes the expected limits from the HL-LHC that are discussed in more detail below.
%
\begin{table}[t]
\centering
\caption{Actual and expected limits on BR$(\tau\to 3\mu)$. See main text for further details. {\color{red} Projections, luminosities, and references (if any) to be completed for the HL-LHC cases}.}
\label{tab:tauto3mu}
\begin{tabular}{ p{4cm}  l  l }
 \hline\hline
 BR$(\tau\to 3\mu)$		&	Ref.						&	Comments	\\ 
 (90\% CL limit)		&								&				\\ \hline
$3.8\times 10^{-7}$		&	ATLAS~\cite{Aad:2016wce}	&	Actual limit (Run 1)	\\
$4.6\times 10^{-8}$		&	LHCb~\cite{Aaij:2014azz}	&	Actual limit (Run 1)	\\
$3.3\times 10^{-8}$		&	BaBar~\cite{Lees:2010ez}	&	Actual limit	\\
$2.1\times 10^{-8}$		&	Belle~\cite{Hayasaka:2010np}&	Actual limit	\\ \hline
$3.7\times 10^{-9}$		&	CMS at HL-LHC~\cite{}		&	Expected limit (3000 fb$^{-1}$)	\\
$NN\times 10^{-9}$		&	ATLAS at HL-LHC~\cite{}		&	Expected limit (NN fb$^{-1}$)	\\
${\cal O}(10^{-9})$		&	LHCb at HL-LHC~\cite{}		&	Expected limit (NN fb$^{-1}$)	\\
$3.3\times 10^{-10}$	&	Belle-II~\cite{Kou:2018nap}	&	Expected limit	(50 at$^{-1}$)\\
\hline\hline
\end{tabular}
\end{table}

The physics reach and model-discriminating power of LFV tau decays is most efficiently analyzed in the SMEFT above the electroweak scale and in a corresponding 
low-energy EFT below the weak scale~\cite{Celis:2014asa}.
Several  classes of  dimension-six operators contribute  to LFV tau decays at the low-scale, with effective couplings denoted by $C_i/\Lambda^2$.  
Loop-induced dipole  operators mediate radiative decays $\tau \to \ell \gamma$ as well as  purely leptonic $\tau \to 3 \ell$  and semi-leptonic 
decays. Four-fermion -- both four-lepton and semi-leptonic -- operators with different Dirac structures 
can be induced both at tree-level or loop-level, and contribute to $\tau \to 3 \ell$ and $\tau \to \ell  \, +  \, {\rm hadrons}$.  
As a typical example,  we note that current limits on $\tau \to \mu \gamma$ probe  scales on the order of $\Lambda / \sqrt{C_{\rm Dipole}} \sim 500$~TeV. 
Besides probing  high scales,  LFV $\tau$ decays offer two main handles to discriminate among underlying models of new physics, i.e. to identify 
which operators are present at low energy and what is their relative strength:
(i) correlations among the different LFV  $\tau$ decay rates~\cite{Celis:2014asa}; 
(ii)  differential distributions in higher multiplicity decays, such as the $\pi \pi$ invariant mass in 
$\tau \to \mu \pi \pi$ ~\cite{Celis:2014asa} 
and the dalitz plot in $\tau \to 3 \mu$~\cite{Dassinger:2007ru,Matsuzaki:2007hh}.

In this arena,  HL/HE LHC will be competitive  with Belle-II in the search for the 
``background-free"   $\tau \to 3 \mu$, 
with projected sensitivity of $3.7 \times 10^{-9}$ at 90\% C.L. with 3000 fb$^{-1}$ at 14 TeV for CMS presented below. 
Efforts should also go into understanding the 
backgrounds and improving the sensitivity in semi-leptonic three-body decays $\tau  \to \mu \pi^+ \pi^-$ 
and $\tau \to \mu K \bar{K}$, given their sensitivity to Higgs-mediated LFV~\cite{Paradisi:2005tk,Celis:2013xja}.

The relevance of searching for $\tau \to 3 \mu$ and semileptonic modes such as  $\tau  \to \mu \pi^+ \pi^-$  at HL-LHC is twofold. First, these could be "discovery channels", as  there exist NP scenarios in which these modes are the ones with the largest branching ratios. 
In a broad brush, one expects  that the $\tau\rightarrow \mu \gamma$ decay has the largest BR in models where the LFV processes are induced by
one-loop diagrams including heavy particles  such as in supersymmetric models. 
In this case  the ratio of Br($\tau\rightarrow 3\mu$)/ Br($\tau\rightarrow \mu\gamma$) is $2.2\times 10^{-2}$. 
However if the LFV processes are induced by
tree-level exchange of $Z'$ bosons or doubly charged Higgs,  the branching
ratio of $\tau\rightarrow 3\mu$ will be the dominant one and bounds on this decay will give stringent constraints on these NP scenarios. 
Similarly, when operators are generated at tree level (as in leptoquark models) semileptonic modes 
such as  $\tau \to \ell \pi \pi$ are expected to have the largest BRs.

Second,  these modes have significant model-diagnosing power.  
In case of discovery, with sufficient statistics the Daliz plot in $\tau \to 3 \mu$ and the $\pi \pi$ invariant mass distribution 
in $\tau \to \mu \pi \pi$  would allow one to pin-point the underlying  mechanism.
These channels  probe a larger set of mechanisms compared to $\tau \to \mu \gamma$.
Of particular interest is the sensitivity of $\tau \to \mu \pi \pi$ to extended Higgs sectors and 
 to non-standard Yukawa couplings of the SM Higgs to light quarks and leptons~\cite{Celis:2013xja,Celis:2014asa}. 
While we are not aware of a dedicated sensitivity study at HL-LHC, we strongly encourage 
exploring semileptonic modes in general and $\tau \to \mu \pi \pi$ in particular, as this channel admits  
at the moment the most robust theoretical interpretation due to advances in the calculation of all the relevant hadronic form factors~\cite{Celis:2013xja}. 

\subsubsection{HL-LHC experimental prospects}
    \def\mupm   {{\ensuremath{\mu^\pm}}\xspace}
    \def\mump   {{\ensuremath{\mu^\mp}}\xspace}
    \def\taupm   {{\ensuremath{\tau^\pm}}\xspace}
    \def\taump   {{\ensuremath{\tau^\mp}}\xspace}
    \def\taummm{\decay{\taupm}{\mupm \mu^+ \mu^-}}
    \def\Dsetamunu{\decay{\Ds}{\eta(\mu^+\mu^-\gamma)\mu^+\nu_{\mu}}}
The LHC proton collisions at 13 TeV produces $\tau$ leptons, with a cross-section five orders of magnitude 
larger than at \belletwo. As a result, during the HL-LHC running period, about $10^ {15}$~$\tau$~leptons will be produced in 4$\pi$. Most will be
produced in the decay of heavy flavour hadrons, specifically $D_s$ meson decays. This high
production cross-section compensates for the higher background levels and lower integrated luminosity, in particular for the $\tau \to 3\mu$ golden mode.

It is worth noting that $W$~and $Z$~bosons are also fairly prolific sources of $\tau$  leptons.
Their production cross sections are considerably smaller than those for $B$~and $D$~mesons, but $\tau$~leptons from $W$~and $Z$~afford much cleaner experimental signatures with far better signal-to-background ratios for CMS and ATLAS; the LHCb forward geometry is less well suited to exploting these decays. For instance, in a $\tau \to 3 \mu$ search relying on $W \to \tau \nu$ decays as a source of $\tau$~leptons, one can benefit from $\tau$~leptons having relatively large transverse momenta and being isolated, from large missing transverse momentum $p_{\mathrm{T}}^{\mathrm{miss}}$ in an event, and from the transverse mass of the $\tau$-$p_{\mathrm{T}}^{\mathrm{miss}}$ system being close to the $W$~mass. 

\subsubsection{ATLAS $\tau \to 3\mu$ projection}

The ATLAS sensitivity to the neutrinoless $\tau$ decay into three muons is projected scenario considering two different production channels: the W-channel, in which the $\tau$ lepton originates from W-boson decays, 
and the heavy flavour (HF) channel, in which the $\tau$ lepton is produced in decays of c- and b-hadrons (dominated by the $D_{s}\rightarrow\tau\nu$ decay). 
Background events arise dominantly from lepton fakes from hadrons ($c\bar{c}/b\bar{b}\to X\mu\mu$), with additional contributions due to pile-up. 
The muons originating from the tau decay are expected to have very low momenta. %This pushes the trigger and reconstruction algorithms to perform in challenging conditions. 
Among the several ATLAS improvements foreseen for the HL-LHC era, two are particularly important for this study: the installation of a new tracking system will improve the vertex and momentum resolution and the trigger system upgrade will include additional capabilities ultimately improving the online selection and allowing to maintain a low muon triggering threshold. The detector upgrade results also in an improved mass resolution for muon triplets, exemplified in Figure~\ref{fig:mass} by comparing the three muon invariant mass distribution obtained in Run-2 and HL-LHC Monte Carlo (MC) simulation. Widths are estimated from a double-Gaussian fit.

\begin{figure}[t]
\begin{center}
\includegraphics[width=0.4\textwidth]{section5/figs/mass_reso_W.pdf}
\includegraphics[width=0.4\textwidth]{section5/figs/mass_reso_HF.pdf}
\caption{Comparison of tau mass resolutions in the W- (left) and HF-channel (right) in run-2 and under HL-LHC detector conditions.}
\end{center}
\label{fig:mass}
\end{figure}
Based on the ATLAS measurement in Run 1 \cite{Aad:2016wce} and considering the improvements described above, the expected exclusion limit on the branching fraction, BR($\tau\rightarrow 3\mu)$, at 90\% CL$_{\mathrm{s}}$ for an absence of a signal is projected with varying assumptions on acceptance, efficiency and background yields.  Systematic uncertainties are extrapolated from the Run-1 measurement scaling down by the increased statistics with preserved constant terms for the reconstruction efficiency. A 15\% systematic uncertainty dominated by the background estimation is derived. Varying the systematic uncertainty by 5\% translates into a 10\% change of the expected upper limit.  A summary of the expected 90\% CL$_{\mathrm{s}}$ exclusion limits for each studied scenario is given in Figure~\ref{fig:limits}. Limits of up to BR($\tau\rightarrow 3\mu) = 1.03 (5.36) \times 10^{-9}$ are derived in the HF-(W-)channel. Table \ref{limits} summarises the upper limit predictions in the different scenarios considered.

\begin{figure}[t]
\begin{center}
\includegraphics[width=0.4\textwidth]{section5/figs/W_overview_015.pdf}
\includegraphics[width=0.4\textwidth]{section5/figs/HF_overview_015.pdf}
\caption{CL$_{\mathrm{s}}$ versus the $\tau\rightarrow 3\mu$ branching fraction, BR($\tau\rightarrow 3\mu)$ for each of the discussed scenarios in the W-channel (left) and HF-channel (right). The horizontal red line denotes the 90\% CL. The limit is obtained from the intersection of the CLs and this line.}
\end{center}
\label{fig:limits}
\end{figure}

\begin{table}[t]
\center
\begin{tabular}{lcc}
\hline\hline
Scenario & W-channel                      & HF-channel  \\
               & 90\% CL UL [$10^{-9}$]  & 90\% CL UL [$10^{-9}$]  \\
\hline
High      & 13 & 6.4 \\
Medium & 6   & 2.3 \\
Low       & 7   & 1    \\
\hline\hline
\end{tabular}
\caption{Summary of the ATLAS expected 90\% CL upper limit on the LFV branching fraction $\tau\to3\mu$ for an assumed luminosity of 3 ab$^-1$ of pp collisions at $\sqrt{s} = 14$ TeV in the HF and W-channels, for the different signal and background yield scenarios considered.}
\label{limits}
\end{table}

\subsubsubsection{CMS $\tau \to 3\mu$ projection}
Muons arising from decays of $\tau$ leptons produced at the LHC tend to have low momenta and large pseudorapidities, which make their reconstruction a challenge. 
The CMS upgraded  muon system, whose coverage is extended from $|\eta|$ = 2.4 to ~2.8,  increases  
the signal fiducial acceptance by a factor of two and, also, enhances the capability to trigger on and reconstruct low momentum muons \cite{CMSTDR:muon}.
The additional events with muons at high $|\eta|$ have worse trimuon mass resolution. Hence, two event categories are introduced: 
Category 1 for events with all three muons reconstructed only with the Phase-1 detectors, and Category 2 for
events with at least one muon with at least one muon 
reconstructed by the new triple GEM detectors, which  will be installed in the first station of the upgraded muon system.

Figure~\ref{fig:cms_Tau3Mu} presents the trimuon mass distribution for signal and background for Category 1 (left) and Category 2 (right) events.

Table~\ref{tab:Tau3Mu} shows the numbers of selected signal and background events in the mass window 1.55-2.0 GeV, 
trimuon mass resolutions, and the expected search sensitivities.

\begin{figure}[t]
\begin{center}
  \includegraphics[width=0.495\textwidth]{section5/figs/m3mu_cscfit.pdf}
  \includegraphics[width=0.495\textwidth]{section5/figs/m3mu_me0fit.pdf}
 \caption{Trimuon invariant mass m$_{3\mu}$  for a $\tau \to 3\mu$
signal (red) and background (blue) after all event selection cuts, in the
Category 1 (left) and Category 2 (right) events. Event categories are defined in the text.
 The signal is shown for  $\mathcal{B}(\tau \to 3\mu) = 2\times10^{−8}$.}
   \label{fig:cms_Tau3Mu}
\end{center}
\end{figure}

\begin{table}[!htb]
\begin{center}
\caption{The expected numbers of signal and background events in mass window 1.55 -2.0 GeV for integrated luminosity of 3000 fb$^{-1}$ (for signal  $\mathcal{B}(\tau \to 3 \mu$) = 2  $\times 10^{-8}$ is assumed).}
\label{tab:Tau3Mu}
\begin{tabular}{l c c}   & Category 1 & Category 2 \\ \hline 
Number of background events & 2.4 $\times 10^6$ & 2.6 $\times 10^6$ \\
Number of signal events & 4580 & 3640 \\
Trimuon mass resolution & 18 MeV& 31 MeV \\
$\mathcal{B}(\tau \to 3 \mu$) limit per event category & 4.3 $\times 10^{-9}$ & 7.0 $\times 10{^{-9}}$ \\ \hline
\end{tabular}
\begin{tabular} { l c} $\mathcal{B}(\tau \to 3 \mu$) 90\% C.L. limit &  3.7 $\times 10{^{-9}}$ \\
$\mathcal{B}(\tau \to 3 \mu$) for 3-$\sigma$ evidence   &  6.7 $\times 10{^{-9}}$ \\
$\mathcal{B}(\tau \to 3 \mu$) for 5-$\sigma$ observation   &  1.1 $\times 10{^{-8}}$ \\ \hline
\end{tabular}
\end{center}
\end{table}

The event selection requires trimuons to form a good-quality trimuon vertex that should be displaced with respect to the primary pp-interaction point and uses a number of other observables to help suppress background.
The absence of tails in the signal trimuon mass distribution shows that, despite the high pileup (PU=200), muons from $\tau \to 3\mu$ decays are picked correctly.
The projected exclusion sensitivity in the absence of a signal is $\mathcal{B}( \tau \to 3 \mu$) < 3.7 $\times 10^{-9}$  at 90\% CL,while the expected 5 $\sigma$-observation sensitivity is $\mathcal{B}(\tau \to 3 \mu$) = 1.1  $\times 10^{-8}$.

CMS studies focusing on $\tau$~leptons originating from W~boson decays are ongoing.

\subsubsubsection{LHCb $\tau \to 3\mu$ projection}
%An important test of the SM is the search for the  lepton-flavour-violating process \taummm. 
%Within the SM with zero neutrino masses this process is strictly forbidden. Depending on the %mechanism of neutrino mass generation, 
%many theories~\cite{Babu:2002et,Brignole:2003iv,Paradisi:2005fk,Hays:2017ekz} beyond the SM %predict this branching ratio to be in the region $(10^{-9}-10^{-8})$. 
%The current experimental limit~\cite{LHCb-PAPER-2014-052,Hayasaka:2010np,Lees:2010ez}, $\cal{B} (\taummm) <$ $1.2 \times 10^{-8}$, {\it [{\color{red} Isn't it $2.1 \times 10^{-8}$?]}} is a combination of the results from LHCb and the \bfactories and reaches the starting point of this range. 
%\belletwo will probe this interesting region of sensitivity when accumulating up to $50~\rm{ab}^{-1}$. 
%{\it [{\color{red} This is common to LHCb, CMS and ATLAS. I'd say it once above instead of repeating it every time.]}}, 
Extrapolations based on the current LHCb $\tau \to 3\mu$ result~\cite{LHCb-PAPER-2014-052} and assuming no detector or trigger improvements show that, similarly to \belletwo, CMS, and ATLAS, \upgradetwo LHCb would also be able to probe down to ${\cal O}(10^{-9})$. This will allow \upgradetwo LHCb to independently confirm any earlier \belletwo discovery, or to
significantly improve the combined limit. In addition, the proposed improvements to the LHCb calorimeter during the \upgradetwo  will be helpful in suppressing 
backgrounds such as \Dsetamunu, enabling LHCb to make best use of its statistical power. 

%%%%%%%%%%%%%%%%%%
\subsubsection{Lepton Flavour Violation in $Z^0$ decays}
\label{LFVZ}
%%%%%%%%%%%%%%%%%
Of considerable  interest are also the LFV decay modes of $Z^0$, 
which are also free from theoretical backgrounds from the SM including neutrino masses, in which one has 
\begin{eqnarray}
&&{\rm BR}(Z \to e^\pm \mu^\mp)  \sim {\rm BR}(Z \to e^\pm \tau^\mp) \sim 10^{-54}~, \\
&&{\rm BR}(Z \to \mu^\pm \tau^\mp) \sim 10^{-60}~.
\end{eqnarray}
The current upper limits come from LEP measurements~\cite{Decamp:1991uy,Adriani:1993sy,Akers:1995gz,Abreu:1996mj}, namely
\begin{eqnarray}
&&{\rm BR}(Z \to e^\pm \mu^\mp) < 1.7\times 10^{-6}~, \\
&&{\rm BR}(Z \to e^\pm \tau^\mp)< 9.8\times 10^{-6}~, \\
&&{\rm BR}(Z \to \mu^\pm \tau^\mp) < 1.2 \times 10^{-5}~,
\end{eqnarray} 
at 95\% CL. Interestingly enough, the LHC can obtained competitive bounds on some of these channels~\cite{Davidson:2012wn}. Indeed recently the ATLAS Collaboration obtained the following upper limits (95\% CL): ${\rm BR}(Z \to e^\pm \tau^\mp)< 5.8  \times 10^{-5}$~\cite{Aaboud:2018cxn}, and ${\rm BR}(Z \to \mu^\pm \tau^\mp) <1.3  \times 10^{-5}$~\cite{Aaboud:2018cxn}. 
The HL/HE LHC will therefore provide the best limit on LFV Z decays. 
The detection of a signal in the channel $Z^0 \to \ell \ell^\prime$, in combination with 
information from charged lepton LFV decays,  
would also allow one to learn about features of the underlying LFV dynamics. 
An explicit example is provided by the Inverse Seesaw (ISS) and ``3+1"  effective models 
which add one or more sterile neutrinos to the particle content of the SM~\cite{Abada:2014nwa} (see also e.g. Ref.~\cite{DeRomeri:2016gum}).

\end{document}
